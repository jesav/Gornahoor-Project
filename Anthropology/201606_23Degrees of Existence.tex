\section{Degrees of Existence}

\begin{quotex}
The next generation of American Indians could radically transform scientific knowledge by grounding themselves in traditional knowledge about the world and demonstrating how everything is connected to everything else. Advocacy of this idea would involve showing how personality and a sense of purpose must become part of the knowledge that science confronts and understands. The present posture of most Western scientists is to deny any sense of purpose and direction to the world around us, believing that to do so would be to introduce mysticism and superstition Yet what could be more superstitious than to believe that the world in which we live and where we have our most intimate personal experiences is not really trustworthy and that another, mathematical world exists that represents a true reality?

Knowledge was derived from individual and communal experiences in daily life, in keen observation of the environment, and in interpretive messages that they received from spirits in ceremonies, visions, and dreams. \flright{\textsc{Vine Deloria, Jr.}}

\end{quotex}
\paragraph{The World of Experience}
The next topic in the ongoing review of the works of Wolfgang Smith involves the various degrees of existence. Since this material is scattered throughout his works, I will instead focus on the fascinating, but compact, book \emph{Tantra Vidya} by \textbf{Oscar Hinze}. Smith recommends this book highly, so it is hardly a distraction. We have previously mentioned four of these degrees, or domains as he calls them.

\begin{itemize}
\item Prime matter 
\item Secondary matter 
\item The physical world 
\item The corporeal world 
\end{itemize}
Prime and secondary matter are described in Tradition and is beyond the scope of this essay. Smith means by the ``physical world" the one that is described by scientists by eliminating all qualitative aspects from it. This is the world invented by Bacon, Galileo and Descartes. It is the quantum world. It is the world of superstition according to Native American tradition. In short, it is a chimerical world.

The corporeal world is the world of lived experience, which is the reason that Smith rated Edmund Husserl and phenomenology so highly. This does not mean the denial of the physical world. For example, I see and touch a table, not atoms and molecules. You often hear such canards along these lines: ``The table is not really solid, but is mostly empty space." That is to confuse the reality with an abstraction: the table is real, and the physical world an abstraction, what Whitehead called the fallacy of misplaced concreteness. That leads to all sorts of anomalies when electrons, for example, are treated like particles in the corporeal domain.

\paragraph{Spatial and Temporal Gestalt}
Smith then points to Goethe for another way to do science based on the intuition of wholeness. He recommends \emph{The Wholeness of Nature} by \textbf{Henri Bortoft}. Goethe opposed the idea of a ``mechanism behind the phenomenon".

\begin{quotex}
The real is in fact none other than that which can in principle be known, and this is ultimately the phenomenon. 

\end{quotex}
The whole, or gestalt, must be grasped at once, in its totality. In understanding this, Hinze makes some important points. Like Smith, he distinguished between the world of experience [Smith's corporeal world] and the physical world. As examples, he refers to some optical illusions like the parallel line illusion\footnote{\url{https://www.google.com/search?q=illusion+parallel+lines\&rlz=1C1SNNT_enUS383US383\&espv=2\&biw=1132\&bih=654\&tbm=isch\&tbo=u\&source=univ\&sa=X\&ved=0ahUKEwiO3_7Olb3NAhUELB4KHcogBmEQsAQIGw\#imgrc=VzvFqSMB5w5XnM\%3A}}. The experienced world is spontaneous while the physical is the result of measurement. Since the modern mind prefers to regard the physical world as the ``real" one, much gets lost. Hinze describes the difference:

\begin{quotex}
If we compare the physical world-concept with the spontaneous world concept, the former looks like one which is fixed and rigid … a mere sterile skeleton vis-à-vis the latter. The direct, spontaneous world-concept is infinitely more many-sided, more ``colourful" and more comprehensive than the physical world-concept; it is, above all, full of a pulsating liveliness and ``warmth". This is not astonishing if we consider that the physical world-concept is derived out of the spontaneous world-concept through a step-by-step elimination of almost all signs which make up our lively world. 

\end{quotex}
Hinze next distinguishes two types of gestalt. The more obvious one is spatial gestalt since we still have the sense of recognizing wholes in space. The temporal gestalt, or the ability to experience a whole spread out in time, has been largely attenuated. Perhaps the most obvious example today is a musical piece: although it is spread out over time, we can recognize it as a single composition. Although Hinze does not go beyond that, a higher consciousness is able to experience larger gestalts, i.e., the inter-connectedness of things as mentioned by Deloria.

By exploring the psychological makeup of archaic man, and even its development in children, Hinze is able to come to some conclusions about archaic astronomy, or astrology. ``Presence time" is the gap in time beyond which a temporal gestalt cannot be grasped, but is experienced as isolated events instead. Hinze draws on examples from anthropology, art, and psychedelic drugs to illustrate this idea. The astronomers of ancient Babylon had a greater presence time than people today. Hence, they were able to experience the night sky as a gestalt. The movement of Mars was experienced as a whole, whereas we can only see isolated movements of a body. Moreover, they experienced the heavens as full of cosmic symbols which related to every aspect of human life.

\paragraph{The Chakras}

Hinze shows that in the ancient esoteric astronomy, the traditional planets correspond to the chakras of Tantra Yoga, in short, the macrocosm corresponds to the microcosm. Moreover, he shows how similar ideas were part of the progress of initiation in Mithraism. But, even more interestingly, Hinze notes that the mystic \textbf{Johann Gichtel}, a student of Jacob Boehme, was aware of the same correspondence as revealed in his book \emph{Theosophia Practica}. Table~\ref{fig:chackrasGichtel} shows these correspondences.

\begin{table}[h]\centering\small
\begin{tabular}{cc}\toprule
Chakra &
Planet (Gichtel) \\\toprule
Sahasrara &
Saturn\\\midrule
Ajna &
Jupiter\\\midrule
Visuddha &
Mars\\\midrule
Anahata &
Sun\\\midrule
Manipura &
Venus\\\midrule
Svadhisthana &
Mercury\\\midrule
Muladhara &
Moon\\\bottomrule
\end{tabular}
\caption{Correspondence between planets and chakras}
\label{fig:chackrasGichtel}
\end{table}
Remarkably, Gichtel claimed to have discovered the subtle centers in the body and their correspondence to the planets through his own contemplations and experience. In the Tantric system, each chakra is represented by a lotus with a unique set of petals. Hinze demonstrates that the number of petals corresponds to the ``gestalt number" of each planet. These numbers are derived from the way the ancient astrologers experienced the sky. For example, the gestalt number of the Moon is 4, which represents its phases. Hence, the Muladhara chakra has four petals. By making these connections in detail, Smith claims that it could not have arisen by chance. In other words, it is ``complex specified information" in Dembski's sense, hence it must be the result of intelligent design.

\paragraph{Various Bodies}
Obviously, the chakras and planets cannot be understood strictly in the physical sense, but rather as lived experience. In particular, Gichtel's order of the planets match the Ptolemaic system; Mercury follows the Moon due to their shorter temporal period of revolution. In the heliocentric system, Venus would follow the Moon. Hence, geocentrism and heliocentrism are not really opposed, but are rather two different perspectives.

To explain these levels, Smith revives the traditional idea of various sheaths or envelopes around the corporeal body. Thus, there is a ``living body" beyond, but intimately related to, the corporeal body. In the Vedanta, there are three bodies that correspond to the three principle degrees of manifestation: the gross body, the subtle body, and the causal body. Furthermore, the subtle body itself is comprised of three sheaths: pranamayakosha, manomaykosha, and the vijnanamayakosha. In the West, these are expressed as the vital force, mind, and intellect. Thomas Aquinas distinguishes between a vegetative, sensitive, and intellective soul. Esoteric science must take all this into account.

\paragraph{The Meaning of Love}
\begin{quotex}
One ought not to occupy oneself with evil, other than in keeping a certain distance and a certain reserve, if one wishes to avoid the risk of paralysing the creative élan and a still greater risk—that of furnishing arms to the powers of evil. One can grasp profoundly, i.e. intuitively, only that which one loves. Love is the vital element of profound knowledge, intuitive knowledge. Now, one cannot love evil. Evil is therefore unknowable in its essence. One can understand it only at a distance, as an observer of its phenomenology. \flright{\textsc{Valentin Tomberg}}

Idolatry is only a projection of individualism; it wears the mask of love but knows nothing of love. For it is not enough to love (everybody loves somebody or something); we have to know whether the beings and things we love are for us doors leading to the world and to God, or mirrors which send us back upon ourselves. \flright{\textsc{Gustave Thibon}}

\end{quotex}
The chakras, when paired up, are the centers of the three worlds, or Triloka, of the Tantric and Buddhist cosmology: kamaloka (the world of desire), rupaloka (the world of form), and the Arupaloka (the world of formlessness). The three pairs of the male and female principle, are the three steps of love: \textbf{Eros}, \textbf{Philia}, and \textbf{Agape}. Thus love is experienced respectively as a drive or instinct, as an emotion, and ultimately as an act of will. Since we are commanded to love, only Agape can be commanded; instincts and emotions cannot be commanded.

Nevertheless, the intellect is powerless to act without the emotion to drive it. Hence, a ``Dr. Spock" type is an anomaly, not beneficent, but evil. Esoterically, a living creature lacking an emotional center is called a ``chimera". It is imbalanced, since the overdeveloped intellect can cause much mischief. Emotion is replaced by the need for sensation. Without positive emotions, there is nothing to keep it in check. Even still, the relationship between the intellect and emotions is imbalanced even in ``normal" people.

At its lower manifestations love is indiscriminate. Although many speak today of ``love" and against ``hate", this lack of differentiation is a sign of the Modern World, the Kali Yuga. Love needs to be directed toward the Good. Love cannot be directed toward Evil. Since the intellect determines what is Good, the feminine principle of emotion must allow itself to be dominated by the male principle. \textbf{Emmanuel Swedenborg} describes the male principle as ``understanding" and the female principle as ``love".

At the highest level, the roles are reversed. The higher intellect, the female principle, must be dominated by Agape, the male principle. In this case, it is the Intellect that needs to be directed unlike in the lower worlds. \textbf{Soren Kierkegaard} in the Works of Love also described three types of love: erotic love (\textbf{Eros}), love of friends (\textbf{Philia}), and \textbf{Agape}. Only Agape is ``more than a feeling" since it is an act of will. Confusing this with the emotions or instinct leads to sentimentalism.

If Agape is to will the Good of someone, then spiritual goods are more important than material goods. This is probably what Vintila Horia\footnote{\url{https://gornahoor.net/?p=8440}} meant when he asserted that Kierkegaard gave him the ``moral key" to understand Dante.



\flrightit{Posted on 2016-06-23 by Cologero }

\begin{center}* * *\end{center}

\begin{footnotesize}\begin{sffamily}



\texttt{Tom Walker on 2016-06-23 at 12:33 said: }

Thanks . Interesting as always . However wouldn't secondary matter ( materia signata quantitate) be the same as the " physical world " , and in that case doesn't the latter expression have the potential to cause confusion ? I also don't see how the parallel line illusion illustrates anything apart from the fallibility of the senses.


\hfill

\texttt{Matt A on 2016-06-23 at 15:47 said: }

So, are there methods for increasing one's ``presence time"? I presume that certain exercises that incorporate into daily life, like returning to oneself when crossing a threshold, are helpful when followed for long periods (like months and years). Unfortunately, I'm the type of person who makes plans while driving, and completely forgets them upon entering the house. Perhaps a greater presence time would be helpful against such deviations.


\hfill

\texttt{Boreas on 2016-06-24 at 11:40 said: }

Matt A, in Evola's `Magic' opus there are excellent articles about those methods presicely what you might need. I personally especially love the essay about the three ways to experience time; the physical time, the psychical time, and the sub specie interioritatis. There are also very good practices in the book.

http://www.cakravartin.com/wordpress/wp-content/uploads/2006/08/Julius-Evola-Introduction-to-Magic.pdf

\texttt{Cologero on 2016-06-24 at 23:41 said: }

@Tom Walker

Let me quote the entire paragraph from Tantra Vidya:

\begin{quotex}
This is the objective result of a spontaneous perception of the diagrammatical illustrations; [i.e., the diagrams I linked to].

\emph{objective}, because \emph{everyone} arrives at this result and it is consequentially valid \emph{inter-individually}. This result of perception does not change, even though we find out on the basis of measurement and know an account of this that ``in reality" the two lines in question are parallel ….

According the experienced reality distinguishes itself clearly from the physical reality, which is ascertained as a result of measurement. Today, we are even accustomed to acknowledge only the latter one as \emph{really} true — and, therefore, one talks of geometrical-optical \emph{illusions}: The criterion for reality is the physical result of measurement.

Physical results of measurement are naturally objective perceptions throughout, as they are verifiable \emph{inter-individually}. Only, they are \emph{not spontaneous} perceptions. In this way, I would like to separate, conceptually, the objective results of spontaneous perceptions from those objective perceptions \emph{carried out through measurements}. 

\end{quotex}
So what other ``illusions", or fallibilities of the senses, do you want to give up?

I recall some scientist tried to ``weigh" the soul of dying patients; there was no difference before and after death.

Neuroscientists cannot ``see" your thoughts with their instruments. All they can measure is brain activity.

Your experience of your own free will cannot be measured, because science relies only on measurable ``laws".

What about your sense of purpose in life? How can that be measured.

Need I go on?


\hfill

\texttt{Cologero on 2016-06-25 at 00:12 said: }

@Tom Walker:

Yes, the vocabulary is lacking. Smith is more careful than I am with his words. He distinguishes, for example, between the corporeal object X and its underlying quantum object SX.

Prime matter has no form. A modern analogy would be the universe as a Schrodinger equation, all possibilities, nothing in particular until a consciousness creates a form.

Second matter is the substantial basis of an object. It is true that Guenon says that the ``matter" of the physicists is the ``materia secunda". Yet I think it is a mistake to identify the quantitative object, i.e., the quantum level object of atoms and molecules, as the substance of a thing. I believe that is Smith's point. Quantity is just the lowest stage, and above it is the thing with its various qualities.


\end{sffamily}\end{footnotesize}
