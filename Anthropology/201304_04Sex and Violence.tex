\section{Sex and Violence}

The typological classification of people has always formed part of Hermetic teachings. This classification was often couched in astrological terms or as the four temperaments, among other schemes. Julius Evola has used the idea of spiritual races, which he claims to have found in ancient symbolisms, to the same end. Now these classification schemes transcend psychological states or tendencies, and, a fortiori, biological or genetic factors. As such, they are metaphysical types of different spiritual natures.

\begin{wrapfigure}{rt}{.3\textwidth}
 \includegraphics[scale=.6]{a20130404SexandViolence-img001.jpg}
\end{wrapfigure}

The purpose of such classifications is not only to understand the world, but also one's own inner nature. By developing awareness and observing the inner workings of consciousness, a man learns to recognize his own type. In this way he can transcend his own limitations or restrictive points of view. In this way, he also learns to recognize the types of other people and thus can intuit their inner states.

Since these types are metaphysical, they are not amenable to scientific or psychological research. That is why so few people become aware of them. Since a man's type, or spiritual race, dominates his entire way of being and manner of thinking, he finds it impossible to conceive of a different way of seeing the world. Hence, these national ``calls to dialogue" on fundamental existential issues are doomed to fail; there just is no common ground on which to base any such discussion. The ways of relating to the world are radically different.

A related concept, when passing beyond the merely individual perspective, is the idea of a national spirit as documented by the idealist philosophers, particularly Giovanni Gentile (which is why we will be discussing him more in the future). This spirit would be manifested in its religion, art, law, science, folklore, etc. However, that presumes a spiritual unity, even if only virtual, which is no longer present in most Western countries. It is striking, when reading books written some 75 years or so ago, to note how prescient they were; here we can point to Guenon, Evola, Berdyaev, Spengler, and a few others. From our perspective now, everything then seemed quite normal, as the decline was barely visible in manifestation although it was quietly moving ahead in the invisible spiritual world. These men were able to discern those negative spiritual forces and anticipate their long term effects. For, it may take several generations for the full consequences of one age to become manifest. The iniquities of the father will affect the next three or four generations.

Evola, for example, points to the infiltration of the ``telluric" spirit into western civilization. This telluric race is earthbound, unable to experience or even understand anything transcendent. The connection to life of the telluric being is marked by immediacy, instinctiveness, and irrationality. He is dominated by sexual impulse that is lacking in virile qualities. With a limited sense of personality or individual responsibility, he goes along with the crowd, which is often tied to his natural peer group, but in recent times seems to be groups of ideological affinity.

The positive law is supposed to reflect the cosmic law in concrete circumstances; when the leaders are guided by transcendent principles, then the positive law will act to restrict and restrain the worst impulses of telluric man. However, once such men themselves make the positive law, these restraints are one by one eliminated in a frenzy of liberation from them and the unleashing of primitive and instinctive forces. Evola points out that the final decomposition of a traditional civilization is marked by the appearance of the telluric spirit within it.

As intriguing as all that is, Gornahoor thinks that these types need to be understood from the ``inside" as it were; here we are always guided by Augustine's maxim that the truth lies in the interiority of man. This we shall do by following our earlier analysis of the various faculties of the soul. In a solar race, the forces of thymos and epithymia (or eros) channeled toward higher values, and are dominated by the rational or intellectual soul. Hence, from thymos he experiences the energy or force to strive for transcendence. Epithymia acts as the attractive force.

However, that is not how the telluric man experiences them. In their crudest form, these faculties are experienced as the urge to violence and sex. That is why popular culture, whether it appears in movies, TV shows, ``music", and so on, is so often dominated by those two themes. That is why appeals to make movies, for example, with more sublime themes (e.g., spiritual elements, etc.) are so naïve. Not only are the consumers of pop culture dominated by the telluric element, but the producers themselves of the culture are themselves telluric and have no other understanding. Anything of an even slightly higher nature is portrayed in a sentimental and stereotypical way. However, Evola points out that there is a melancholic element associated with that, although it is not very deep. This is demonstrated, for example, in movies of the type created by a Woody Allen, which are dominated by a superficial type of spiritual angst; nevertheless, films of such a character are highly regarded by the ``intellectuals" of the telluric type. A more recent example is the HBO series ``Girls" which again is highly touted by the same elements.

This morning I heard a discussion about the ``gun control" issue that is dominant today in the USA. One ``journalist" suggested that Hollywood should be part of any so-called solution; in other words, it should be called to task for promoting so many movies with graphic violence. Of course, this generated the mechanical response about our alleged ``hypocrisy", in allowing graphic violence—assumed to be ``bad" —while prohibiting graphic sex—presumed to be ``good". Readers, don't you see how this exactly proves my fundamental point? Why the automatic, and probably mindless, contrast of sex to violence, instead of a hundred other possible qualities?

Naturally, this led to a further stereotype: viz., in Europe they ban violent depictions but allow the sexual. Now, don't get me wrong, I am heartened when American men today recall their ancient European heritage. However, they always relate to the most recent and decadent elements of Europe today rather than the sane and normal era prior to the French Revolution.

Before moving on to the metaphysical understanding of sex and violence, we can point to some rather obvious empirical results. First of all, violence has always existed, prior to mass media, prior to video games, prior to the printing press, and it continues to exist even among peoples far removed from such influences. We recently pointed to the existence of primitive Brazilian tribes, with no access to the artifacts of western civilization, yet who are quite violent in their daily lives. Furthermore, rates of violent crime have actually decreased in the USA despite such movies and games; the factors for that are beyond the scope of this article.

The next empirical point is that graphic sex, aka pornography, has not led to happiness. As a matter of fact, the opposite has occurred. Rather than leading to better and more fulfilling relationships between men and women, it has instead resulted in widespread impotence among men. (See the article by Lasha Darkmoon in the January 2013 issue of Culture Wars in this regard). A case can be made that graphic sex is being used to control and subvert the population, whether deliberately or incidentally.

I will conclude, first, with the traditional understanding of sex and violence, followed by an explanation. The erotic impulse in traditional art is sublimated toward a higher end. That is reflected in the ideals of chivalry, troubadours, the Fedeli d'Amore, etc. That allows the power of sexual attraction to be transferred to the desire for superior values.

Similarly, violence is not gratuitous, but is ordered to heroic and noble ideals; such stories have always been portrayed in myths, legends, and scriptural stories. Hence, recent trends in not allowing boys to play with toy guns, or play as soldiers, or watch films depicting violence are fundamentally misguided. It makes them passive instead of training them to be active agents who understand that sometimes the higher good entails the proper use of power and coercion.

Even if you have understood nothing thus far, I will provide two simple concepts that summarize it all:

\begin{enumerate}
\item When boys role play as heroes, fighters, conquerors, etc., or even engage in sports, they are not incited to commit actual violence. To the contrary, the urge for violence is mitigated by such play. 
\item However, the very purpose of pornography and graphic sexuality is to excite erotic impulses, yet they do not provide any way to satisfy them. 
\end{enumerate}
Hence, sex and violence act in opposed ways, are directed by different psychic centers, and cannot at all be understood in the same way.



\flrightit{Posted on 2013-04-04 by Cologero }

\begin{center}* * *\end{center}

\begin{footnotesize}\begin{sffamily}



\texttt{Sparrow on 2016-01-20 at 16:25 said: }

If one wants further proof about the tellurism of modern times, simply consider the obsession with sexlessness. Being a ``virgin" or ``single," are both used as insults, and celibacy is almost blasphemous since the telluric man can't conceive of any reason that someone would not indulge in sexual pleasures. The obsessive fear about sexual inadequacy that spawns numerous pills that seek to cure the problem is truly comical.


\end{sffamily}\end{footnotesize}
