\section{Emile Coue}
\label{sec:Coue}
\begin{quotationx}
This is the promised translation of \textbf{Julius Evola}'s essay on \textbf{Emile Coué}. It originally appeared in \emph{Bilychnis}, in the Jan-Feb issue of 1925 under the title \textit{Emile Coué and ``Acting without Acting"}. Here Evola goes well beyond Coué himself and provides metaphysical, philosophical, and psychological explanations for the phenomena described by Coué.

Along with New Thought, which itself has been heavily influenced by Coué, couism, or something like it, has been a part of Hermetic teachings when properly understood. In reading this essay, perhaps some of you will be intrigued by this seeming paradox:

\begin{enumerate}
\item Coué demonstrated almost ``miraculous" results 
\item Similar results have been very difficult to reproduce 
\end{enumerate}

If point 2 were not true, then everyone would be demonstrating miraculous powers today. That I have not witnessed, not even in the New Thought circles that I was formerly part of. Yet point 1 has some truth, and Evola himself was convinced of it. Following the translation, we will offer an explanation of why points 1 and 2 are both true … unless someone else can explain it better. 

\end{quotationx}

The name of the Frenchman Emile Coué\footnote{\url{https://en.wikipedia.org/wiki/\%C3\%89mile_Cou\%C3\%A9}} has made quite a splash in recent times. There was a period in which, especially in France and in England, Couism had become the word of the day: a real interest and very lively discussion gathered little by little around the figure of this modern thaumaturge, his famous psychotherapeutic method of ``conscious autosuggestion", and the wake of almost miraculous healings left by his travels in the countries of central Europe and the New World. In these days, Coué has also come to Italy to hold a series of conferences that, if not of true enthusiasm—perhaps because of the rather skeptical nature of the greater Italian public—certainly has attracted much attention of the type which, among the observations of modern psychology of the supernormal, is always more inclined to believe that the power of man can in reality reach far beyond that which the small, humble, what everyday life shows us as possible in general. To tell the truth, the need that stands at the foundation of Coué's doctrine is rather important; it will therefore not be useless to attempt a reconstruction of it and, at the same time, to investigate up to what point the methods of Coué can be sufficient to it.

\subsection*{Section I}
The starting point, undoubtedly, is this. Hypnotic phenomena are real. Recent studies in this regard—it is enough to cite Liebeault, Bernheim, De Rochas, Richet—have truly been conducted according to the principles of the strictest positivism, and put the matter beyond dispute. Now it is factual that the hypnotized person realizes a number of things of which, in the normal or awake state, he is not absolutely capable: concerning perceptions, the emotional element, or various organic functions, he has a power that touches the miraculous. How is that possible? Here we have the first position of couism, expressed in the principle—moreover already announced by Liebeault and Myers—that \textit{every heterosuggestion (i.e, a suggestion made by another) is realized only through an autosuggestion}. That is, the theory that the power of the hypnotizer works directly on the hypnotized person is excluded. The hypnotizer only transmits a command, he only suggests the idea of the thing to be accomplished: but in order for the suggestion to be effective, it is necessary that the subject assume it, transform it into an autosuggestion and thereby realize it with his own strength. Receptivity is in fact an essential condition for the success of the suggestion. It would follow from that that the operative power in the phenomenology of hypnosis must be pushed back not to another, but to the I itself: that, at least, in its material aspect of a force ``that does", aside from the principle of the impulse that causes it.

It must be placed back on the I, so far, so good. But which I? Clearly not the I of the conscious personality: the hypnotic sleep effectively entirely abolishes such an I. It therefore necessarily pertains to another, deep power of subjectivity beyond that which consciously thinks and wills, a much vaster and stronger power that Coué designated sometimes as the \textit{imagination}, and at other times—following Jung—as the \textit{subconscious}. Now in the phenomenon of heterosuggestion these two I's fall into two distinct beings: in the hypnotized person, the unconscious principle and realizer; in the hypnotizer, the conscious and directing principle. The problem that couism poses is this: to substitute the I of wakefulness for the hypnotizer and to reunite the two principles in the same subject. Hence, the concept of a suggestion to be made by oneself to oneself, according to directives that the I of awakening should formulate and the subconscious carry out: hence, the theme of \textit{conscious autosuggestion}.

Before going further, let us lay down a more precise understanding of what the subconscious is. Here Coué—keeping in mind what the subconscious shows us it is capable of in hypnotic phenomena—refers to an entity that presides over \textit{all} organic functions, from the most humble of the vegetative life to those that govern the mental processes themselves, and, therefore, it corresponds to entelechy as described by Aristotle and Driesch, the ``\textit{logos}" of Sthal, the ``physical personality" of Ribot. But this determination is still exterior: it does not tell us what such a principle is in and for the I. And that is must be something for the I, which it in one way or another must emerge, must be put at the head of the luminous zone of consciousness, which is clearly presupposed by the problem of conscious autosuggestion: since if the two I's were such that they mutually excluded each other, or that when the subconscious emerges the consciousness is submerged and vice versa, it is clear that the action \textit{ab intra} that couism proposes would not be possible.

Duality must therefore be in a certain way present in the very sphere of the life of awakening. To that we refer to Coué's other term, the \textit{imagination}. He notes that in the same conscious life, two entirely distinct powers are in play: one relates to the plane of intellection, the clear consciousness and reflective will; the other to the murky reign of the instinct, emotion, passion, and deep and irrational convictions whose dominant principle is that of faith and imagination. The one is, to use Platonic terminology, the \textit{logos}, the other, \textit{eros}. Coué distinguishes rather cleanly between these two powers: they have individuality and absolutely heterogeneous modes of acting. Now while the logos falls outside that deep principle that rules the whole of organic functions, the eros communicates with it so that, in a certain way, it is identified with it. The corollary follows immediately from that—the key point for couism—that the problem of the control of one's own personality—and that not only with regard to the element of character, but also with regard to the physical being and, therefore in particular, the problem of eliminating psychic disturbances and maladies (psychotherapy)—it is mutual with that of the determination of the eros from part of the \textit{logos}, for the ``imagination" from part of the conscious I. This is the task of conscious autosuggestion: to succeed in making directives originating from the I from the imagination accepted, so that they moreover become directives for the organic subconscious which, blind and inert, cannot fail to obey them as it happens in hypnotic phenomena.


\hfill

\begin{quotationx}
This is part 2 of \textbf{Julius Evola}`s review of the movement initiated by \textbf{Emile Coué}. It would be curious that Coué discovered all these techniques himself, since they are part of Hermetic and magic training. Here are some of the salient points to notice:

\begin{itemize}
\item A monoideism is a state of focusing the mind on a single idea as in hypnosis. Negatively, it is related to the ``idee fixe" described by \textbf{Valentin Tomberg}, or more colorfully, and not completely inaccurately, as ``possession by a gnome". Positively, it is an aspect of deliberate concentration exercises. 
\item Man, as he is, i.e., the man untrained in mental or spiritual development, is not as conscious as he believes himself to be. Rather, he is a ``puppet" under the domination of myriads of suggestions, most of which do not emanate from himself. 
\item The so-called conscious mind, i.e., the discursive intellect is a mass of contradictory ideas which very often impede our lives. 
\item There is a deeper ``I" or self, beyond that seemingly awake I, that is the true and real source of our lives. 
\item That deeper self responds to suggestions arising in the conscious mind or those entering the conscious mind from outside sources. 
\item To ``reach" that deeper self, it is necessary to somehow suppress the incessant chatter of the conscious I. That is the real goal of meditation, concentration, and other so-called magical techniques. Even ``sex magick" has nothing to do with sex, beyond using it as a means to bypass the conscious mind. 
\item Notice, for example, how Coué's ideas track the notion of ``concentration without effort" as described by Valentin Tomberg. 
\item It is clear that these techniques sound simple, but are quite difficult in practice. That would account for the seeming lack of success in most cases. The state of ``conviction", faith, or the ability to ``will one thing" is elusive to those who think too much. 
\item Based on the examples that Evola provides, perhaps the purpose of these tests for \textit{Qualifications for initiation}\footnote{\url{https://www.gornahoor.net/?p=1161}} may be clearer. 
\end{itemize}
\end{quotationx}

Since autosuggestion is possible in principle, Coué sees it confirmed by a thousand facts of daily life. Three fourths of our actions —as he claimed to one of his disciples— are, more or less directly, supported by autosuggestion: only that we habitually suggest to ourselves without either knowing or willing it and, in most cases, as though it were all our doing. Here Coué emphasized the opposition of \textit{logos} to \textit{eros}, he often loves to highlight how our conscious will and reason can do nothing against an autosuggestion, since what makes us act is always the imagination. He says,

\begin{quotex}
When the will is in battle with the imagination, it is the will that, without exception, loses the game. 
\end{quotex}

Not only that, but the efforts made by the will in similar cases succeed simply in reinforcing the power against which it faces. This is the \textit{law of reversed effort}. Examples:

\begin{enumerate}
\item The novice bicyclist sees a stone in the middle of the road, he \textit{imagines} that he is going to run over it, he \textit{struggles} to avoid it; the harder he tries, the more likely he goes right over the rock. 
\item A person notices he cannot fall sleep; he wants to sleep: the more he wants it, the more he does not succeed in falling sleep. 
\item A sufficiently large plank is placed across an abyss; if the same plank were place on the ground, a person would walk across it with the greatest nonchalance. But being placed over the abyss, he \textit{imagines} he could fall; he does not then dare to traverse it: if, in spite of everything, by forcing an effort, he attempts it, in the great number of cases he is going to fall off. And so on. 
\end{enumerate}
These and numerous other examples show that the will can do nothing against a present suggestion. The secret of success is instead to oppose a suggestion not with an effort, but rather with another suggestion of the opposite meaning: it is necessary, i.e., to translate the will into a corresponding ``imagination". Without that, the effort succeeds only in precipitating the occurrence in the direction opposite to the one desired. Therefore Coué says:

\begin{quotex}
We who are so proud of our will, we who believe we freely do what we do, we are in reality only poor puppets of which our imagination holds all the strings. We cease to be these puppets only when we have learned to guide the imagination.
\end{quotex} 

Here we can make a rather important observation about the mode of action of the two principles. In apparent contrast with what at first view would let us think of the classifications of logos and eros, we see that the mode of the first is action properly called, or effort, tension; in contrast, the mode of the second is purely mental, it is an ``imagining", a sudden and inner, effortless, self-representation. This pure ``imagining" realizes, and realizes it to the very depths of the one's being: instead action or will, purely called, is condemned to the periphery, and can do nothing in respect to the deepest stratum of the physical and emotive personality, and when it intervenes, it succeeds only in leading to battles and disagreements with almost always detrimental consequences.

The problem therefore is centered in the question of provoking consciously and intentionally that autosuggestive process which, in the background, plays such an essential part in daily life: or, in other words, of discovering the method by which one can act on the ``imagination" starting from the I of awakening. Coué's principle in this:

\begin{quotex}
every idea that engages the spirit exclusively becomes ``true" for the imagination; the concentration of the mind on an idea transforms this into a belief. 
\end{quotex}

In technical terms: it is about creating in oneself a ``monoideism". Faith and monideism, ultimately, would be equivalent to each other. That said, the technique of couism is the following:

\begin{quotex}
The principles of the will and the imagination exclude each other; in the hypnotized person the power of the subconscious is so great because a free field is left to it, while everything that is related to the will and consciousness is put aside. The ideal would therefore be the ability to enter a hypnotic state: but since the hypnotic state suppresses the conscious I by hypothesis, it is necessary to find a compromise, viz., of the intermediate stages in which the surfacing of the subconscious has the maximum, compatibly with the minimum permanence of the conscious principle. 
\end{quotex}

Such are the so-called ``hypnoid states", among which Coué emphasizes that of \textit{détente} or relaxation. It is a matter of realizing a complete relaxation of the conscious faculties, of excluding every mental and muscular effort or tension, just as happens immediately before falling asleep and in so-called \textit{reveries}. When the mood of \textit{relaxation} ends up sufficiently incited, the conscious principle that up to that point was put aside—withdrawn almost to the extreme corner of oneself—it can intervene and make the idea slip into the mind, or better put, the \textit{image} of what one desires to realize. The mind must fixate on this image to the exclusion of all the rest, moreover without entering into the field of efforts, which would destroy the state of \textit{relaxation} and, along with it, the state of receptivity and of suggestibility of the ``imagination". How then can one realize such concentration? Coué's answer is:

\begin{quotex}
By means of the very rapid repetition of a formulation which includes the idea in question; with the repetition, the mind is automatically and spontaneously brought to the idea, and its speed makes it such that other images are not inserted in the intervals. In particular: any counter-suggestions are not formed which would neutralize it and thereby render it ineffective. Through repetition, monoideism is formed: absorbing the whole spirit, the image strengthens itself slowly in the imagination until it becomes a dim and deep conviction; it is then employed by the power of the subconscious from which it certainly is turned into action. 
\end{quotex}

If, e.g., in a hemorrhage, one succeeds in concentrating on the formula ``the hemorrhage has stopped" —like a mood freed from conscious faculties—, to the point of rendering itself fully active, the related image lives in the mind—in a word: to the point of believing it, then this idea becomes a truth and a command for the subconscious that, as director of every organic function, makes the arterioles and venules contract naturally, just as would happen artificially with a tourniquet: and the hemorrhage is stopped.

This, in a few words, is Coué's theory. In experiments, it seems to have been demonstrated as capable almost of a miracle. Limiting ourselves to the physio-pathological order, to which it particularly is oriented, not only functions but also organic paralyses, advanced stage consumption, generalized eczema, tubercular lesions, tumors, etc. have been cured, sometimes in very few sessions, by means of the autosuggestion developed from the illness itself under the guidance of Coué or his disciples.


\hfill

\begin{quotationx}
In Section II, \textbf{Julius Evola} offers a critique of \textbf{Emile Coué}`s explanation for the successes of his method, and then provides his own explanation.

Making the connection between the ``deep conviction of the subconscious" and ``faith", Evola then relates it to the ``vexed question" of ``grace". In other words, why do some have faith and others do not? Or, in more practical terms, why are some healed by couism while most are not? However, Evola here is interested only in a psychological explanation, not a metaphysical one. So ``grace" means only that its origin seems to come from outside rather than from one's own powers. This is a different question from the metaphysical meaning of grace.

First of all, while faith is the belief in something unseen, it is more than that since it reaches deeper that our everyday I that we consider to be conscious. This I, however, is engaged embroiled in dualisms: subject-object, true-false, right-wrong. The deeper I, on the other hand, is non-dual: it simply is and it wills one thing. There is a certain seemingly ``heroic" attitude that sees doubt as concomitant with faith. But that is to remain perpetually at the level of that ``I of awakening" while never penetrating to depths.

Rather, the proper development of faith is gnosis, i.e., the direct realization that has been mentioned several times in Evola's essay. In this regard, Valentin Tomberg is a more reliable guide as we have shown in \textit{The Elements of Sacred Magic}\footnote{\url{https://www.meditationsonthetarot.com/the-elements-of-sacred-magic}}.

To his credit, Evola does not deny the numerous preternatural and miraculous accounts of the saints, and, growing up in Italy, he would have heard of many. Therefore, he rejects a glib positivism that rejects miracles out of hand. In contrast, he develops a ``metaphysical positivism" that would account for the reality described in those stories. Nevertheless, in the final analysis even metaphysical positivism leaves out something essential.

He rightly rejects the claim that the method of couism is sufficient in itself, apart from the person of the patient. Otherwise, if the method worked indifferently, there would be couism centers in every city instead of hospitals. At some point, it is up to the patient to develop faith or the image of his healing. Every suggestion, to be effective, eventually becomes an auto-suggestion. Not only the belief in couism, but also the belief in any ceremonial, ritualistic, sacramental, or symbolic forms have no power in themselves other than to motivate faith in those who need it.

Some, however, do not require any such aids. This would mean, fundamentally, that a man can do without a Tradition. Such a man, however, would not know what to have faith in, and hence would not find gnosis. Typically, he would actually be a secularist. The symbols cannot be discarded so readily. Yet, any psychological explanation can never be complete as long as something supernatural or transcendent remains unaccounted for (which it must, in psychology).

Nevertheless, Evola's emphasis that a personal realization is somehow always involved and that repetition for its own sake is useless is sound. Yet, there is one miracle that Evola's purely psychological explanation does not capture. That is Peter's raising Tabitha from the dead and the dead, presumably, are immune to a hetero- or auto- suggestion.

\end{quotationx}

\subsection*{Section II}
The great problem that comes into play in couism is that of the construction of faith. In fact, it becomes clear to everyone that that deep conviction of the subconscious that Coué speaks of is nothing other than faith. Fundamentally, it comes back to the vexed theological question of grace. The principle that faith, once reached, is an irresistible power, which immediately realizes what it believes is true, has been known from the most remote times and, at least in a certain measure, has since been scientifically sanctioned by modern psychology. And the significance of faith is not only subjective, it is also objective. The accounts of the miraculous phenomena that envelopes the figures of saints and creators of religion — so lightly discarded by so-called positivism — in reality take inspiration from faith. Of the similar phenomena of yoga and magic, the principle, in the final analysis, is faith or, if one prefers, autosuggestion. Explanation:

\begin{itemize}
\item At the first level, I can have the simple, empty concept of a thing. 
\item Beyond that, I can bring it to life in the imagination. 
\item In the third place I can perceive it exteriorly as a subjective hallucination. 
\item In the fourth place, I can act on other consciousnesses in a way that they also perceive it (collective suggestion). 
\item The \textit{same} power, continued in an ever more intense affirmation that invests the level of physical being, becomes objective and, as such, is a magical act. 
\end{itemize}
As the mage can be defined as the one who knows how, so to speak, \textit{to influence the same nature}, to communicate his faith to it, with which one is preliminarily put in relationship by means of an act of love or sympathy.

However beyond that, the great question is to know \textit{if}, and in this case, \textit{how}, faith can be constructed positively and consciously by the I, and, to tell the truth, beyond any suggestion whatsoever. In the great number of cases, it is to be noted that the I does not possess faith as much as it is instead possessed by it. I.e., in them, faith is realized to the extent it does not flow from an absolute sufficiency, from a pure self-assertion of the individual, but from the suggestion derived from some idea, close to which the I originally — in the moment in which it was triggered — is passive and unconscious. So that if it is true that no suggestion (hetero-suggestion) succeeds if it does not make itself an autosuggestion, it is also true that, almost uniformly, no autosuggestion (with particular regard to those that can extend the power of the I beyond normal) is actuated if it does not have a preliminary suggestion as a basis for motivation. That is, the I does not succeed in being present, to assert itself at the moment of the initiative. So the person of faith at Lourdes, in as much as it heals, to that extent he believes that it is not he, but divinity, who works the healing; to the extent the hypnotized person carries out miracles, to that extent the initiative, the suggestion, comes to him not from himself, by from the hypnotizer. And that, to tell the truth, can extend to a great number of Coué's patients, who heal to the extent that they believe either in Coué, or in the efficacy of the ``method", or, at least, in the existence of the ``subconscious", that he, and not themselves in naked individual affirmation, will know how to heal them.

Actually, for example, one would like to see if the healing of so many persons —\textit{after Coué's invitation}— as soon as they affirm that they are cured, would have likewise occurred if it were any of the readers who made that invitation to them. On the other hand, we must note that the various miracles of the saints, at least in Western mysticism, were experienced in the spirit of the intervention of a higher and transcendent power. The same magic is related to the whole of ceremonial, ritual, symbolic, etc., manifestations which are only the necessary substitute for those who do not know how to create the powers of the imagination by means of a positive central initiative, but attain them only indirectly by virtue of the suggestion emanating from a complex of extrinsic elements. Nothing other than this absence of oneself from the principle of a positive initiative, this incapacity to create faith \textit{kath'auto} [in himself], gives the meaning to what is indicated as ``grace" in Western theology. However, even if there is grace, conscious autosuggestion is a useless name: for the starting point, it has an unconscious autosuggestion (a hetero-suggestion)—a mystery and a passivity; and then the I appears the instrument of faith, not its creator and the possessor.

Such therefore is the true light under which the requirement emerging from couism must be examined. Now the idea that for the construction of faith an automatic repetition of formula in a state of relaxation is sufficient, is rather ingenuous. First of all, we must note — as was done above — that a great number of persons who succeed with such a method are already hetero-suggested, directly or indirectly, by Coué or by those who tell them about him and his doctrines. In the second place is the fact that the \textit{imagination} or faith is \textit{invariably presumed in itself}; i.e., one does not achieve it starting from something different from it, but rather starting from an initiative that resides in it and not in the peripheral faculties of normal consciousness. So that it is said: ``The blind man does not have any possibility of making himself a guide." If the repetition is simply mechanical, if it is not already accompanied by a certain level of interior evidence, it results in nothing, and regarding that, anyone can perform the experiment whenever he wants.

In the East, where these things were studied at their foundation, they likewise recognize the importance of \textit{japa} (repetition) through the ``realization" of the so-called mantra (magical formula); however, it is explicitly said that one can do \textit{japa} even a million times, but unless the mantra is ``understood", ``awakened" (\textit{sphota}), the \textit{japa} remains a mere flapping of the lips. This ``self-awakening" of the mantra is an illuminative moment; of pure inner evidence (so it is said in the texts that ``awakening" the mantra means to realize it in one's essence ``made of light", \textit{Jyotirmaya}), which can be propitiated and intensified from repetition, but not generated, for that demands a true spiritual spontaneity, an effective passing of consciousness from one ``dimension" to another.

It is a question of the same qualitative heterogeneity that intercedes between the concentration of the fire of a lens over a substance and its sudden burning up—naturally considering the phenomenon not from the physical point of view, but from that of the psychological effect in a spectator. Therefore, it is not necessary to have any illusions of the efficacy of the ``method": one can be certain that wherever it succeeds without a true initiative or inner conversion coming into play to animate it — of which very few are capable — it succeeds not through its own power, but rather through some hetero-suggestion that unconsciously is insinuated into the process. And then, from the practical point of view, the result will be equally good: its spiritual value, in reference to the previously mentioned problem, is therefore nullified.


\hfill

\begin{quotationx}
This is the final installment of \textbf{Julius Evola}`s essay on \textbf{Emile Coué}. Section I was an overview of couism, Section II dealt with it from the psychological point of view, and Section III analyzes it from a metaphysical or spiritual point of view.

The spiritual level overcomes all dualities, so force cannot be understood as working against an ``outside" object. Rather, the unmoved mover ``acts without acting" in the Taoist and Aristotelean sense. Therefore, in conceiving or imagining something, it brings this about ``without effort". Thus, Evola claims that couism needs to be raised up from the unconscious. The ``imagination" is a higher faculty (it is one of the ``wits" or internal senses. See, e.g., \textit{The Phenomenology of the Medieval Mind}\footnote{\url{https://www.gornahoor.net/?p=3559}}).

In this state, Evola claims, one will have power not just over the body and its vegetal functions, but even over physical nature. Unfortunately, couism itself is just a tool and not a road to get to that state. For that, ``faith" is necessary. But faith in what? Does the ``I" simply believe whatever it wants, or is there really a transcendent revelation that is necessary? Here Evola seems to fall short. As long as the quicksand of the subconscious is full of ``obscure and murky" ideas, the I cannot lift itself by its own bootstraps from that trap. 

\end{quotationx}

\subsection*{Section III}
In order to specify the meaning of that change of interior level, we can add the following observations. A rather profound truth stands behind the opposition of the will and imagination. We are accustomed to consider force as a near synonym to coercion, to reduce the will to mean \textit{muscular}, but that however presumes an antithesis, a resistance (whether physical or moral), which the I faces and struggles against. All of so-called Western ``activism" is based essentially on the concept of tension and effort. Now such a concept of action is fundamentally inferior. Referring precisely to this idea, Lao Tzu was compelled to indicate the mode of spiritual power as \textit{wei-wu-wei}, or acting without acting. It is important to understand this point well.

It finds its best illustration in the Aristotelian concept of \textit{akineton kinoun}, i.e., of the unmoved mover. The one, who is really the cause, does not move: he remains firm at the point of pure creative initiative. Movement takes place only in the effect, in what proceeds from him and of which he remains the master and calm director. In this sense he, properly speaking, does not act: he conceives—he only produces the movement, he causes action without being dragged into the movement and passion, without engaging himself, but \textit{effecting} it and dominating it. Such is the meaning of \textit{wei-wu-wei}.

The same idea is found in the Shakti Tantra. The moment of the permanent (Shiva) is in the positive or masculine power, and the negative, or feminine is that of the dynamic and active power (Shakti). This is in open contrast with the western views which, having lost at this point any trace of the higher mode of the spirit, consider action relating to the simple demiurgic force, passion, manias, mere actuality or spontaneity as positive and masculine, i.e., having its own principle outside itself (and such is the truth of the ``evolutionistic" doctrines such as those of Bergson and Gentile). This is, actually, to be called passive and feminine. Hence, even the deep sense of the opposition, in Christian Gnosticism, of the ``pneumatic God" to Yaldabaoth, the demiurge God and creator.

The spiritual, by hypothesis, is superior to every antithesis or polarity; the material is instead formed in it and for this reason, its principle is struggle. The same concept of force—just as Guyau correctly noted—arises outside of the relationship to a resistance, so that it, in an absolute way, does not indicate potency, but impotency. Instead, the spiritual never needs to ``struggle", it unfolds in a calm way without fighting. Hence, the profound meaning of the paradoxical aphorism of Lao Tzu, that ``truly victorious army is the one that has never fought."

It is a matter of an inner and subtle action that is irresistible since it is part of the level of a principle, that has nothing against itself, that is entirely and hierarchically superior to every antithesis and polarity of the material or moral plane. This ``creative indifference"—as Friedlander-Mynona calls it—that is pure, subtle conceiving, is referred to again, in the Kabbalistic tradition, and the eleventh Arcanum of Force in the Tarot, in which the symbol of a woman holds tight the jaws of a furious lion \textit{without any effort}, exhibits, precisely the manner of spiritual causality, the mistress of every violent force and every physical determinism. On the other hand, the Lodge, it is in the curious words: ``he walks with a stick; therefore rifles discharge with a wave of his hand."

Now, it is precisely such acting without acting that it is necessary to refer to when Coué speaks of the ``imagination": the world of this imagination, just as we expressly mentioned, is indeed not an acting in the proper sense, but instead a self-representation, a pure, interior apperception — something that becomes, insensibly and subtly, almost from nothing. The will is, to tell the truth, a principle connected to the physical being and the environment, a product of biological contingencies: it therefore necessarily submits to the limits and the conditions of the life of relation and the material plane. In the ``imagination", the reflection of ``acting without acting" is instead expressed, or acting without effort for higher spiritual spontaneity. In it, the I can therefore have a true magical organ, an organ by which it can assert itself over whatever eludes the willpower in the strict sense: over its emotivity and over its physiological being, and not only, but even— this affirmation does not seem too audacious—over physical nature.

The day will come in which one will realize that everything that man realizes by means of positive science over nature is only an illusion of power: in fact, it does not truly dominate in the various mechanisms devised by technology, but makes itself the slave and always more dependent on the various natural laws that it presupposes, recognizes, and exploits. And that is because the scientific attitude is essentially extroverted and separative and because its level is that of the I opposed to the non-I and not that of the principle that understands this duality and is interiorly superior to it. The secret of our impotency is precisely this: that we are preoccupied by things, that our action is almost magnetized by them, that it never truly draws its determination from its own interiority. We put ourselves on the same plane of phenomena or the outer world, although we are subject to its laws. Should one instead place himself for an inner conversion at the level of that which rules phenomena, what external thing could no longer be resisted? Now in ``faith", ``imagination", etc., there is precisely a suggestion of such a level. If we would know that we do not see or act relying only on a pure affirmation or inner certainty, then we would unlock the realm of the only real power. This is active faith: the popular saying that faith is blind is spot on. But such blindness is not a privation, it is instead a perfection; it symbolizes the pure freedom where the ``seeing" here would signify it was tied, it was subject to the law and the principle of the ``other".

Coué's merit is therefore to have brought to light the inferior, negative character of the will properly (or improperly) called, in general, that of violent and antithetical action against the subtle mode of a realization without effort that, to tell the truth, stops being ``sensible". His defect is not having understood ``imagination" as a higher level of the spirit, but of having substantialized it superstitiously in a type of distinct entity opposed to the conscious I, to be used via small expedients. His method acutally does not center on the I in the role of the ``imagination", but seeks instead to excite it from the outside through the technique of relaxation and repetition. Thus, in order to empower the conscious individuality, to elevate its pure energy to the luminous peak of a pure act, couism estranges the I from itself, doubles it in a type of hysterical dissociation. Then the imagination adopts the obscure murky power of the subconscious, studied by psychoanalysis that, in such cases, has this advantage over cousim: it seeks to penetrate and resolve it in clear consciousness, where cousim does not care about that, but only to make it work. And such a scission, admitted in principle, brings a serious consequence. If the subconscious then comes about as a distinct entity, man will also be able to dominate with it this or that element that previously eluded him, he will also be able to make his warts and his asthma disappear, but never will he be able to guarantee an autonomy to what he wants and that he can only be the effect of obscure subconscious processes, which he does not know, nor can he know anything about them.

The same reasons in the name of which he could use or subject the subconscious—given also that he succeeds in it — would be able in their turn to be just the symbols of subconscious impulses: every freedom could be only the peripheral appearance of the instrument of an irrational and impenetrable force.

This essay on couism and its problems that arise from it suffices for now. Everyone sees that it is a matter of passionate questions, which would merit being considered and amplified rather more than what has so far been done.

\hfill

\flrightit{Posted on 2013-12-16 by Aeneas}

\begin{center}* * *\end{center}

\begin{footnotesize}\begin{sffamily}

\texttt{scardanelli on 2013-12-19 at 10:41 said: }

Thank you for this translation Cologero. It has been extremely useful to me in clarifying Tomberg's third letter and the concept of sacred magic.

Here's wishing you and your family a Merry Christmas. I offer my prayers for health and wisdom for you and all of the readers of Gornahoor.

``Thou, chaste Moon, full of joy,

Favour, since thy Apollo now reigns,

The Child who was born this day.

He alone will cast iron out of the world

And populate both Poles with

A most precious lineage of gold"


\end{sffamily}\end{footnotesize}
