\section{The Man of Heart}

\begin{quotex}
You might be saved — i.e., you might acquire spiritual equilibrium and live in the absolute center of being. \flright{\textsc{Pavel Florensky}, \textit{At the Crossroads of Science \& Mysticism}}

\end{quotex}
\paragraph{Prayer of the Heart}
The Prayer of the Heart was practiced by the early Egyptian Desert Fathers as a method of the purification of the Mind and the Heart. The Prayer of the Mind is the inner recitation of a prayer. When the prayer runs on its own, apart from the Mind, then it becomes the Prayer of the Heart. Ultimately, one can pray even during dreams.

The Prayer of the Heart is most often associated with the recitation of the Jesus Prayer. However, the unknown author of The Cloud of Unknowing recommends some simpler prayers like the repetition of ``God" and ``Love". If it is difficult to bring attention from the Mind to the Heart, then attention can focus initially on the hands or another external body part. The Prayer of the Heart bring attention ``in" the heart. In other words, the heart becomes the center of awareness, not the object of awareness.

\paragraph{The Law of the Heart}
In the \emph{Letter on the Hermit} from Meditations on the Tarot, we see that the heart is where contemplation and will are united, where knowledge becomes will and where will becomes knowledge. It is important to keep in mind that the ``heart", in this context, does not at all signify the centre of emotions and passions as it does in the popular imagination. Rather, it is the middle centre, or chakra, of man's psychic and vital constitution. It does signify ``love" however, making it the most human of the centers. ``Knowledge" is what man knows, ``Will" is what a man can do, but ``Heart" is what he is.

The great work of the man of heart is the transmutation of the substance of other chakras into the substance of the heart. \textbf{Seyyed Nasr} wrote that the heart is the only organ that connects the human state to a transcendent state. This notion is confirmed by \textbf{Valentin Tomberg} when he explains that the heart, alone of all the centres, is not attached to the organism. Hence, it can go out of the organism and live.

\paragraph{The Planets and the Chakras}
The heart, as the central chakra, is therefore the ``sun" of the microcosm. \textbf{Oscar Hinze}, in his book \textit{Tantra Vidya}\footnote{\url{https://www.gornahoor.net/library/TantraVidya.pdf}}, shows that in ancient esoteric astronomy, the traditional planets correspond to the chakras of Tantra Yoga; in short, the macrocosm corresponds to the microcosm. Moreover, he shows how similar ideas were part of the progress of initiation in Mithraism. But, even more interestingly, Hinze notes that the mystic \textbf{Johann Gichtel}, a student of \textbf{Jacob Boehme}, was aware of the same correspondence as revealed in his book \textit{Theosophia Practica}\footnote{\url{https://www.gornahoor.net/library/TheosophiaPractica.pdf}}. The following table shows the correspondences. The table also includes the ``I am" saying associated with the chakra, and the transformation that occurs with its awakening, as described in the \emph{Letter on the Hermit}.

\begin{table}[h]\small
\begin{tabularx}{\textwidth}{lllXX}\toprule
\textbf{Chakra} &
\textbf{Petals} &
\textbf{Planet} &
\textbf{Transformation} &
\textbf{I am}\\\toprule
Sahasrara &
8/1000 &
Saturn &
Abstract and transcendent wisdom &
The resurrection and the life\\
&&&Full of warmth like the fire of Pentecost&\\\midrule
Ajna &
2 &
Jupiter &
Intellectual initiative &
The light of the world\\
&&& Compassion-filled insight into the world&\\\midrule
Visuddha &
16 &
Mars &
Creative word &
The good shepherd\\
&&& Magical: illumining, consoling, healing&\\\midrule
Anahata &
12 &
Sun &
Love &
The bread of life\\
&&&Exteriorisation of love&\\\midrule
Manipura &
10 &
Venus &
Science &
The door\\
&&&Conscience&\\\midrule
Svadhisthana &
6 &
Mercury &
Center of health&
The way, the truth, and the life\\
&&& Holiness, i.e., harmony of spirit, soul, body&\\\midrule
Muladhara &
4 &
Moon &
Creative force &
The true vine\\
&&& Source of energy and élan&\\\bottomrule
\end{tabularx}
\caption{Chakra and Planetary Correspondences}
\end{table} 

Remarkably, Gichtel claimed to have discovered the subtle centers in the body and their correspondence to the planets through his own contemplations and experience. In the Tantric system, each chakra is represented by a lotus with a unique set of petals. Hinze demonstrates that the number of petals corresponds to the ``gestalt number" of each planet. These numbers are derived from the way the ancient astrologers experienced the sky. For example, the gestalt number of the Moon is 4, which represents its phases. Hence, the Muladhara chakra has four petals.

In the chart, the column labeled ``Transformation" shows the changes that occur when the chakra is transformed by the heart. The column labeled ``I am" shows Jesus’ ``I am" saying that is associated with each chakra.

\paragraph{Waking Up}
When the chakras are asleep a man becomes dominated by instinctual life, motivated by fear (muladhara), sex (svadhisthana), and hunger (manipura), interspersed with random eruptions from higher chakras. Another way of saying this, following Gichtel's diagram, is that the human being is under the influence of the planets, hence subjected to sponaneous forces beyond his knowledge and control.

So the obvious question is how to ``awaken" the higher chakras. That puts us in a bind, since Hermetism rejects any mechanical process or technique to do so. By analogy, we can look at how you wake up from a night's sleep. Who is doing the awakening? Commonly, it may be the result of an outside stimulation, or enough it comes about after a ``crisis point" in a dream.

So, analogously, we could say that ``waking up" into a higher state of consciousness are reaching certain ``crises", as described in the Letter on the Tower of Destruction. Such a crisis may result from either an internal or external event. Example, perhaps, are the boundary situations described by \textbf{Karl Jaspers}, which often arise from chance, traumatic events. Of course, the Hermetist may choose boundary situations deliberately, by meditating on a particular topic. Perhaps, in this case, a meditation on one of the ``I am" sayings would be helpful. I think it is a bit of a mystery for the ``sleeping" person to try to wake up. All the forces that lead to that, need to be encouraged. Ultimately, it is a matter of grace from above.

\paragraph{The Christianisation of the Chakras}
\begin{quotex}
Therefore, if anyone is in Christ, he is a new creation \flright{\textsc{2 Corinthians 5:17}}

\end{quotex}
Valentin Tomberg mentions the traditional Tantric method of awakening the chakras through their corresponding mantras: Om, Ham, Yam, and so on. That will awaken the chakras as they are. The Hermetist, however, has a different aim: the Christianisation of the centres, i.e., their transformation in conformity with their divine-human prototype. In other words, the aim is to make of oneself a new creation. The corresponding ``I am" saying can be used as a mantra in the process of the Christianisation of the chakras.

The Christianisation of the inner organization is the transformation of the human being into a man of heart. The heart is the third, or neutralizing, force mediating ``knowledge" and ``will". This leads to three transformational moments.

\begin{table}[h]
\begin{tabularx}{\textwidth}{llX}\toprule
\textbf{Intellectual intuition} &
Feeling for truth &
Subordinate spontaneous movements of thought as well as the directing intellectual initiative to the heart of thought\\\midrule
\textbf{Moral intuition} &
Feeling for beauty &
Subordinate both spontaneous imagination and actively directed imagination to the direction of the heart\\\midrule
\textbf{Practical intuition} &
Feeling for good &
Subordinate spontaneous impulses and designs directed from the will to the feeling of practical intution\\\bottomrule
\end{tabularx}
\end{table}
Note that there are two stages of subordination to the heart:

\begin{itemize}
\item Spontaneous arisings 
\item Directed mental activities 
\end{itemize}
We have dealt with spontaneous arisings extensively in the past. We have noticed that, in our normal waking state — which is usually far from fully conscious — thoughts, images, and impulses spontaneous arise, most often in a very negative way. We have used them as ``crises" to lead to a moment of awareness, since they need to be brought under conscious control. In the past, we have used the exterior parts of the body, e.g., hands, feet, etc., as our objects of concentration and attention. Perhaps, now, we can begin to bring attention to the heart rather than a body part.

Next, there can be deliberate and consciously directed thoughts, images, and plans. Those are recognizably human activities since they are self-directed and self-willed. However, for them to be Christianised, then they, too, must be subordinated to the heart.

The goal of the Christianisation of the centres is to transform the human being into a \emph{man or woman of heart}.

\paragraph{Scientific Postscript}
Although secular science is not the last word for us, it should not be surprising to learn that the heart has neurons. The HeartMath Solution, by Doc Childre and Howard Martin develops the idea of the heart as the central intelligence of the body. We do not consider this a ``proof" for the man of heart, but an effect. Nevertheless, some of you, perhaps in the healing professions, may be interested in such topics. The downside is that, like all new age teachings, it sees the ``knowledge of the heart" strictly in instrumental terms, as the means to an end, be it inner calm, physical health, treatment for psychological problems. We, on the other hand, consider becoming a man or woman of the heart is an end in itself.



\flrightit{Posted on 2022-04-23 by Cologero }
