\section{Evola on the Tibetan Book of the Dead}
\label{sec:EvolaTibetanDead}

\begin{quotex}
This review by \textbf{Julius Evola} of the Evan-Wentz translation of \textit{The Tibetan Book of the Dead} originally appeared in \textit{Bilychnis}, volume XXXIII, November, 1929. 
\end{quotex}

This volume contains an excellent, fully annotated, translation of a very interesting Tibetan text that can be considered as the companion of the Egyptian Book of the Dead, and that until now was available only as manuscript. It concerns the science of post-mortem activity, as they say, ``between the two" (bardo), because the reference is actually to the intermediate period that consciousness passes through two forms of existence.

The doctrine, to which the text refers, is that of Mahayana Buddhism, although with the influences of elements of still older tradition, the lamaic, pre-Buddhist tradition of Bon. The interesting aspect is in the idea that in the post-mortem state, consciousness can remain active, so as to be able to determine its final destiny from itself with its own behavior. So the text warns of some experiences that gradually arrive, instructs on their meaning and on the attitude to take to victoriously overcome them.

Starting from the general Hindu teaching that the I, substantially, is the Supreme, except it does not know it (\emph{avidya}), the text explains the apparitions of a dazzling and unbearable light that would strike the soul immediately after the separation from the body, with the experience of the real nature of the soul itself. Then one has a choice: either one has the strength and the intrepidity to identify with the Light and then the ``liberation" and the escape from \emph{samsara} will ensue; or else the illusion of believing oneself to be ``other" is stronger, and then one passes through successive experiences, in which however one has to deal with the same reality that takes different forms in relation to different psychological states. The same inability of identifying oneself, so to speak, is projected and makes the undifferentiated state of the Light appear under the form of personal beings, of gods.

And even here the same dilemma arises: either the consciousness succeeds in identifying itself with them, on the basis of a corresponding practical worship in life; or else it cannot endure the vision of them and fears them; and then this same fear is projected and kaleidoscopically transforms the vision of a hierarchy of lucent and benevolent gods into a hierarchy of furious and flaming gods.

For the third time the dilemma is presented, except that the chances of identification, and therefore of liberation, in the person of the terrible gods, are much smaller, unless in mortal life one practiced the worship of the divinity of that type. But if even this test fails, it means that in the soul the will for a conditioned existence is stronger than that for a higher existence. By its very nature it is therefore brought back into the cycle of rebirth.

Now there are its same deep tendencies that, being unleashed with the inevitable violence of natural phenomena and being projected pliably in visions of Furies and demons who pursue and chase, push the soul toward that form of existence to which it is the most conformed. Strange enough—and not deprived of reference to certain views of modern psychoanalysis—is the detail indicated from the text regarding the moment of the incarnation. The consciousness has the vision of those who will have to be the father and mother, in the moment of their sexual intercourse: then, if a female, she imagines a desire for the father, and if a male, for the mother, and making its own their craving and their sexual pleasure, the soul finds itself drawn into the womb.

Then instantaneously the metaphysical awareness is interrupted and so also the memory: in its place, the soul will gradually develop its own consciousness proper to the new being. Furthermore, formulas are given, so to speak, in extremis, with the goal of causing at the last moment the ``closure of the womb", like other formulas, oriented to a certain domain of deep tendencies, so as to reach to a power of direction in order to be led to forms of existence suitable to prepare a better outcome when, at the new separation from the body, they will make before the same experiences.

All the interest that the text presents appears even from this short review; this book belongs in the library of every student of these matters.



\flrightit{Posted on 2013-03-31 by Cologero }

\begin{center}* * *\end{center}

\begin{footnotesize}\begin{sffamily}

\texttt{David on 2013-03-31 at 22:13 said: }

Evola insight on buddhism is always clear. He should have written more on the subject (i.e. The Doctrine of Awakening). Thank you for the translation.


\hfill

\texttt{X on 2013-03-31 at 23:53 said: }

``The same inability of identifying oneself, so to speak, is projected and makes the undifferentiated state of the Light appear under the form of personal beings, of gods."

Maybe for Evola those gods have Nordic features.


\hfill

\texttt{Saladin on 2013-04-03 at 21:25 said: }

He won't encounter them! If we take the Tibetan Book of the Dead as our guide, after his ``death" Evola would have identified with Atma (the very first stage of the ``Postmortem activity") and therefore would not have encountered ``personal beings, gods." Only men with less than perfect knowledge wait for this stage.


\end{sffamily}\end{footnotesize}
