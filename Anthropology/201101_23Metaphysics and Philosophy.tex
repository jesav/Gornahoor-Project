\section{Metaphysics and Philosophy}

Because we write about topics such as being, religion, knowledge, history and the like — topics philosophers like to deal with — some readers of Gornahoor have the mistaken impression that we are also philosophers. Hence, they presume we are defending a particular philosophico-religious system, or trying to promote one; then they desire to engage in debate, for their own unfathomable purposes. Philosophy in our time involves speculative theories about the nature of the world and mind or else a tedious logical analysis of words. This is far from our task. Readers wishing to discuss topics are obliged to read the entire blog, or at least related material, prior to drawing any conclusions.

As we pointed out in Metaphysical Positivism\footnote{\url{https://www.gornahoor.net/?p=1189}}, we regard metaphysics as an exact science analogous to physics.

\begin{description}
\item[Physics ]

The science of the natural, or external, world 

\item[Metaphysics ]

the science of the states of the inner world 

\end{description}
When someone denies the possibility of such metaphysical knowledge, they are like the bishops who denied the existence of Jupiter's moons discovered by Galileo. In the latter case, the solution is simple: look through the telescope. Analogously, in the former case, the solution lies in meditation or other spiritual exercises that will unveil the inner world. The obstacle is, however, that in metaphysics, ``to know is to be". That means that, in order to know a particular inner state, one's very being much change to be in that state. Few are willing to go through the trouble; perhaps it is not even an option for many.

Such a change in being is not required in order to look through the telescope, a task that is readily accomplished by anyone. Therefore, physics is universal, democratic and egalitarian. Metaphysics is necessarily just the opposite.

We relate this to the trichotomous structure of man\footnote{\url{https://www.gornahoor.net/?p=1029}} in Table~\ref{tab:trichotomous}.

\begin{table}[h]\small\centering
\begin{tabular}{ll}\toprule
Spirit &
Non-formal manifestation\\
Soul &
Subtle manifestation\\
Body &
Gross manifestation\\\bottomrule
\end{tabular}
\caption{The trichotomous structure of man}
\label{tab:trichotomous}
\end{table}
The body is part of gross manifestation, or nature. It is subject to the laws of physics. We may not be accustomed to regarding our soul life as part of manifestation, but it consists of our thought, desires, moods, feelings, daydreams, fantasies, likes, dislikes and so on. It is part of our human state. Physics is the science of gross manifestation, metaphysics of subtle manifestation or higher states. Hence, just as the physicist is detached when he studies the movements of the planets, so also must the metaphysician be when he observes the movement of thoughts, fantasies, and so on in his consciousness.

In the Kali Yuga, man's identity is centered in his soul life\footnote{\url{https://www.gornahoor.net/?p=1374}}, so it is very unnatural to have that sense of detachment in regards to the events in one's own consciousness. The ``analog I" is being asked to detach from the very objects of consciousness he is creating. That is why so many spiritual practices are aimed at destroying that identity of one's being with the analog I.

The Spirit refers to non-formal manifestation. Whereas the soul and body are part of the phenomenal world, that is, part of our experience, the spirit cannot be an object of consciousness. In other words, it is noumenal; we can't know it as an object, but rather as subject, a direct intuitive knowledge of oneself as subject. In most men, this awareness will be vague or even non-existent, that is, it is virtual. In the True Man, it is actualized as one's true will.



\flrightit{Posted on 2011-01-23 by Cologero }

\begin{center}* * *\end{center}

\begin{footnotesize}\begin{sffamily}



\texttt{Ted on 2011-01-23 at 17:52 said: }

Good article. However, the positions of Soul and Spirit should be reversed As Evola shows in his book ``The Hermetic Tradition", the attainment of the Soul is a stage in Alchemy which occurs after the attainment of the Spirit. The Spirit is feminine and represents the Moon and subtle while the Soul is masuline and represents the Sun and higher states of being above that of the subtle realm. This seems to be a common error in much esoteric writing.

Other than that, this article was excellent.


\hfill

\texttt{Cologero on 2011-01-23 at 22:51 said: }

Unfortunately, Ted, there is merely a confusion of terminology. In the beginning of the chapter ``Soul, Spirit, and Body" in Hermetic Tradition, Evola makes this clear: ``It should be noted henceforth that soul and spirit do not possess the same meaning here that they do in our time."

To make it more confusing, in other parts of the book, Evola seems to reverse the meanings of soul and spirit again. That is why it is more important to understand principles, not vocabulary. English is not a very precise language when it comes to metaphysical topics.

It gets even worse. In the chapter ``Sulfur, Mercury, and Salt" in Guenon's ``The Great Triad", Guenon uses the word soul and spirit in our sense. So both Guenon and Evola make the equivalence between mercury and passivity. But Evola writes ``Spirit is Mercury" (ch 13) and Guenon claims that Mercury corresponds to the soul. Nevertheless, they are not disagreeing with each other on this point, despite the seeming verbal contradiction.


\hfill

\texttt{Ismo Meinander on 2011-01-24 at 03:58 said: }

I think the terminology should many times be looked from different angles depending on where one stands currently, something like the planetary archetypes in astrology, in which different archetypes reside in each other and affect each other while one remains the primary dominating one. For example, ``mercury as spirit" would mean the spirit as a dissolving force that draws everything back into the ``one life", back into the liquid state without distinction, and ``mercury as soul" would mean the higher intellect so. buddhi; and ``sulphur as spirit" means the fiery, dry force of will that asserts itself and ``sulphur as soul" would mean the fiery, passionate and timid forces of the individual. Salt seems to be the most stable symbol of the body, but we can also remember the gospel saying about ``the salt of the earth", in which it means people who can bring light of reason and meaning into peoples lives, and salt can be related also to the reasonal, logical faculties via its symbolical structure. And so on.

It can be confusing, that's for certain. The alchemists and hermeticists didn't hide their art behind apparently meaningless mumbo-jumbo without sufficient reason. Just as Cologero said: it is more important to understand principles, not vocabulary.


\hfill

\texttt{Cologero on 2011-01-24 at 20:08 said: }

Thumos is not identical to the soul, it is one tendency. Eros is another. Get out of theory and observe your own consciousness.

You can experience a desire (eros) independent of thumos. The intelligence will decide whether or not to act on it, or how best to satisfy it. The desire may be strong, so some ``will-power" (thumos) may have to be called up. In this case, thumos may be active in relation to eros, yet passive in relation to nous; so it is not absolutely masculine, just relatively.

Observe a moment of anger arising. Perhaps someone slighted you, a friend irritated you, a car cut you off on the highway. In this case, do you summon that anger or does it arise spontaneously in reaction to the event? I'm certain it is the latter. That is what makes it passive, or feminine. If the nous is not strong enough to channel it, you will act on that anger, probably in an ineffective way.

Say you are preparing for a sporting contest or some other competitive activity. In this case, you will ``pump yourself up", that is, use the energy of thumos to improve performance.

The point is that this is not really a matter for opinions or thoughts. One must learn to observe oneself over time in many different situations. Then one comes to understand himself, one's nous is strengthened. One then sees the folly and futility of all the debates he used to participate in.

That is why we are taking Gornahoor ``private". There is only so much that can be shared via theoretical discussions. These need to be replaced with self-observation and spiritual exercises, at least for those able and willing to make the efforts.


\hfill

\texttt{Graham on 2011-01-24 at 21:47 said: }

Ah, thumos is passion. I was mistranslating thumos as will.
\end{sffamily}\end{footnotesize}
