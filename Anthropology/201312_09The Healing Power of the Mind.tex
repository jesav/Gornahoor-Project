\section{The Healing Power of the Mind}

Although \textbf{Emile Coué}, who originated the affirmation, “Every day, in every way, I am getting better and better”, may not be so well known today, his influence is nevertheless very great indeed. Anyone on facebook cannot help but notice the constant flow of “positive affirmations” in his feed, which is undoubtedly part of Coué's legacy. While I agree it is fine to think positively when appropriate, it often rises to the level of superstition. I knew a woman who refused to dance the tango with me because she felt the music was too sad.

It may even become delusional when the hard facts of life confront us. Prince Siddhartha's life was totally altered when he came to terms with illness, ageing, and death; positive thinking alone cannot prevent them. In the teaching of the dominant religion closer to home, the loss of the Primordial State brought death, toil, and pain to the human race. The solution to death has to come from the outside, not from nice thoughts.

Evola, however, was quite interested in Coué and devoted a longish essay about his teachings\footnote{See Section~\ref{sec:Coue} in this book.}. Evola related Couéism to the “acting without effort” of Taoism, using language not unlike \textbf{Valentin Tomberg}'s description of the Juggler or Bateleur\footnote{\url{https://www.meditationsonthetarot.com/personal-effort-and-spiritual-reality}}. But before offering up the translation of that effort, I would like to provide a little background for comparison and contrast to Evola's views. On the one hand, we will use some concepts from German philosophy to make a plausible case and also Fr. \textbf{Alois Wiesinger}'s comments from his book \emph{Occult Phenomena}\footnote{\url{https://www.gornahoor.net/library/occultPhenomena.pdf}}.

\paragraph{The Soul Principles of Healing}
Fr. Wiesinger devotes several pages to Couéism. First of all, he notes the well documented possibility of the healing disease through interior powers. This is not implausible when we recall that the body is the form of the soul, so that changes in the soul life can and will affect the body. As a trivial example, we can note how erotic thoughts effectuate bodily changes. Fr. Wiesinger notes that

\begin{quotex}
there are in the subconscious those purely spiritual powers of the soul which are remains of preternatural gifts. Sometimes these can achieve wonderful results. 

\end{quotex}
There two principles that account for Coué's success:

\begin{itemize}
\item Every thought strives towards its own realization 
\item The law of effort produces an opposite effect. 
\end{itemize}
In the first law, the more a thought is entertained, the more the nervous system will seek to realize it. That is, the sense-stimulus nerve system will create a pathway for the motor nerve system. There are some drawbacks, the most obvious of which is that the vegetative soul is not very responsive to conscious control. (E.g., you don't digest your food by thinking through the process of digestion.) The other more subtle one is that every thought engenders some doubt, which may negate the original positive thought. Hence, Coué turned to the subconscious for the second principle. Fr. Wiesinger explains:

\begin{quotex}
When the will commands an act, then the reason judges whether such an act is possible, reasonable, useful, etc., and so by its doubts and reflections prevents the first law from being effective. 

\end{quotex}
Hence, the positive suggestions need to take place in an altered state, i.e., a “half-sleep”, or “alpha state”, and so on. That is why so much operative magic relies on techniques to quiet the conscious mind in order to reach down into deeper states. These states are more receptive to autosuggestion and don't give rise to doubts. The modern Westerner is too dependent on reflection and the conclusions of science, so he finds it difficult to reach such states. Hence there results a self-fulfilling prophecy. This is unlike Jesus' teaching recorded in Mark 11:23:

\begin{quotex}
Amen I say to you, that whosoever shall say to this mountain, Be thou removed and be cast into the sea, and shall not stagger in his heart, but believe, that whatsoever he saith shall be done; it shall be done unto him. 

\end{quotex}
Although Fr. Wiesinger devotes several pages of examples, including those from Christian Science (a close relative of New Thought), we can skip past them for our purposes. The main points of interest are these:

\begin{itemize}
\item To be effective the discursive mind must be repressed. Evola refers to this as the First Trial\footnote{\url{https://www.gornahoor.net/?p=1902}}. Valentin Tomberg also makes clear that the mind must be silenced. 
\item The various techniques are of value only insofar as they manage to bypass the chattering mind to reach deeper into one's soul life. Hence, there are no absolute techniques, but only those that are most suitable to one's personal equation. 
\item Phenomena like healings, astral travel, etc., are not necessarily the result of miracles. Rather they are usually the result of the activities of the spirit-soul over the body, since some vestigial preternatural powers still remain if they can be reached and activated. 
\end{itemize}
\paragraph{Creativity in German Philosophy}
As should be well known, Evola was heavily influenced by German philosophy, to which he incorporated some elements of Thomism in an interesting way. In particular, \textbf{Johann Fichte}, following Kant, came to the conclusion that what could be known with certainty was the I and the freedom of the moral will. That concerns the noumenal, or transcendent, will. As this I “posits” itself, it must also posit the non-I so that the moral will has a field of action. Hence, this non-I is a privation and a free creation of the I. As the I becomes more aware of what it has done, it develops, perhaps even up to the point of becoming the “Absolute I” or Absolute Self.

For \textbf{Friedrich Schelling}, this creation takes the form of Art. Art is the human equivalent of “\emph{creatio ex nihilo}”, or we Hermetists would say, “As above, so below”. So the question then is to what extent is the creation of the non-I really due to the I. Here, we should rely on the distinction between natural and artificial kinds. Natural kinds are the ideas in the mind of God, the Infinite Possibilities in Guenon's language. Artificial kinds are the product of human design. It should be obvious, then, that a great deal of our personal and social life is actually the creation from artificial kinds, whether individually or as part of a group.

The course of our lives is to a greater or lesser extent the result of our decisions and our thinking about persons, places, things, and events encountered within it. The boundary between freedom and fate is not always clear cut, but it may be wise to err on the side of freedom. Similarly, our social organizations are also products of the human will. A people arises when there is a common mind on these things. Again, there is a natural social organization, the Traditional one, but even within that framework, there is ample room for human creativity and, unfortunately, the anti-creativity resulting from weakness, ignorance, and malice.

Here we see the possibility for the clash of worldviews. In this scheme, to some extent the “best” worldview will prevail, the one that leads to survival and thriving and is best aligned to the natural kinds. Nevertheless, passion and strength of will also have a lot to do with it.



\flrightit{Posted on 2013-12-09 by Cologero }

\begin{center}* * *\end{center}

\begin{footnotesize}\begin{sffamily}


\texttt{scardanelli on 2013-12-10 at 09:33 said:}

“Here, we should rely on the distinction between natural and artificial kinds.”

So, in other words, we must learn to discern the spirits- to recognize the demons and egregores engendered by humanity, and to differentiate between these and genuine spiritual impulses.

It's interesting…since the first mention of New Thought on Gornahoor I have run into it in several places. Paul Foster Case mentions it in one of his books on the Tarot. Another commenter mentioned Lisiewski, who beleives that at the time of its inception, life was simpler and the techiques of New Thought were sufficient to gain the results that it promised. Since then, life has become mush more complex. In the words used here, there are many more artificial kinds of ideas that make it much more difficult for us to gain control. Thus the need to seperate from discursive thought, to acheive silence.


\hfill

\texttt{Michael on 2013-12-10 at 21:40 said: }

What is the mechanism whereby thoughts affect the external environment? If I understand New Thought correctly, they believe that everything is made out of God (sorry for putting this so crudely) and that we are emanations of God. Because we are, in a sense, God, we can affect our external environment.

If the New Thought metaphysics are incorrect, how can we affect our external environment via pure thought? I accept that we exercise a degree of control over our bodies.

Scardanelli, what are your thoughts on Paul Foster Case? He has a lot of good insights into the Tarot but seems to come to a conclusion that is opposite of Tradition.


\hfill

\texttt{scardanelli on 2013-12-10 at 22:34 said: }

My only knowledge of New Thought is derived from what Cologero has shared here, but in a general sense, if the subtle rules the dense, then changes in the soul effect changes in the body. New Thought seeks to use daily affirmations to effect change in the material world. However, since these daily affirmations are only a few of the legion of thoughts, feelings, etc that we experience each day, they are mostly ineffective. Thus, it's not necessarily that their metaphysics are incorrect, but that they misunderstand the nature of the problem and it's practical solution (concentration, silence, the quieting of the mind). This is why Tomberg states that it is first necessary to learn concentration without effort, lest you become a mere charlatan. 

As far as Paul Foster Case is concerned, I don't know if i'd recommend him necessarily, but there were a few interesting bits of information that helped fill in some gaps for me in his Introduction to the Study of the Tarot. I really only read it because I found it available as a free PDF. 

There seem to be two competing interpretations of the Tarot, a Golden Dawn interpretation which he seems to be involved with and the French Hermetic school of Papus, Levi, Tomberg, etc. He places The Fool before the Magician rather than between Judgement and The World. So his interpretation won't exactly line up with Tomberg's. The PDF, as well as some other interesting offerings, can be found at the Ordre Martiniste Operatif au Quebec website.


\hfill

\texttt{francismercuri on 2013-12-15 at 22:37 said: }

“New Thought”, or at least certain premises therein, deserve consideration. New Thought, and all of its repackagings err not so much in their interpretations of mechanisms of human consciousness, but in their evaluation that human consciousness operates in a vacuum–that it is the single force between “us” and all else. It has value–minimally in encouraging individuals to examine and remain attentive to their thoughts–to master thoughts, perceptions, and reactions to them–aka self-mastery, thus even if some of New Thought's premises and conclusions fall short, this is sound enough spiritual advice. But it roughly ends there.

New Thought often describes itself as a “metaphysical” system. It would be better if it regarded itself as a system (or science) of understanding and operating mind and consciousness in relation to matter, as it does not meet the muster of a “metaphysics”. Metaphysics deals with “Total Possibility”–New Thought imposes the limit of consciousness and/or human consciousness in the process of “creating reality” (“we create our own reality” as the famous saying goes), and as such, can not be a true metaphysics (although perhaps more an epistemology, phenomenology, and “psychology”). 

Many more “traditional models” exist to contrast. In broad strokes there is the “Great Triad” formula–three special “Qi” existing–Tian (Heaven), Ren (Human), and Ti (Earth). Each have numerous subdivisions, characteristics, and qualities. The “band” of human qi that New Thought works with isn't even all of the human state, much less all states beyond human.

In more, and most radically “practical” terms, vis a vis producing “magical” results, New Thought, while only considering human qi, often “fails” when ignoring other streaming qis. “Heaven” in this context becomes astrological factors assessed by the cosmic observer before and during times of working. “Earth” becomes not only the concrete environmental conditions one finds oneself within (including their hereditary lot); but also the type of Telluric energies investigated by things like Classical Feng Shui, or otherwise handled under the guise of “local spirits”.

New Thought then, imo, is not a failure, it just requires reassessment, and its limitations need be noted.


\end{sffamily}\end{footnotesize}
