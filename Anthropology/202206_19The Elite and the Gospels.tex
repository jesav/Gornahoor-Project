\section{The Elite and the Gospels}

\begin{quotex}
From Pythagoras to Virgil to Dante, the chain of Tradition was never lost on Italian Soil. \flright{\textsc{Rene Guenon}}

\end{quotex}
\textbf{Professor Robyn Faith Walsh}\footnote{\url{https://www.youtube.com/watch?v=XYqghYtia6k}} wrote an intriguing book asserting that the Gospels were written, not by members of the religious communities, but rather by a Greco-Roman cultural and intellectual class. She makes a good case, but does not understand the actual role of such an elite class. So that topic is well worth exploring.

The ``elite" does not refer to the ruling or oligarchic classes, who are content to maintain the public cult, but rather to a class with sufficient education and resources to compose complex symbolic texts. Hence, she bases her thesis on those two factors.

\paragraph{Resources}
The first point is that knowledge requires physical resources. Access to wiring materials like ink, parchment, papyrus was costly. Writing books by hand must have been a tedious process. If you look at the sizes of ancient texts, you have to marvel at the effort involved. How were thoughts organized? How were corrections made? And so on. Only a class that had the wherewithal to purchase such materials and the leisure time, could possibly have written those texts. Such a class had to be small in number.

\paragraph{Intellectual Training}
Secondly, knowledge requires access to teachings. The intellectual elite had access to resources, and they probably studied at the Academy or various mystery schools. They had the ability to write well, which illiterate fishermen did not. The intellectual elite would have been trained in the liberal arts and therefore understood grammar, rhetoric, and logic.

The professor does not make clear why that class bothered with the effort, so here is where we must depart from her thesis. Today's academic outlook is buried in historical consciousness and thus is limited by the ``superstition of facts". Such a worldview does not take into account that historical events are reflections of spiritual realities. The intellectual elite that she is describing is actually an esoteric elite. Their purpose, therefore, was to take the bare facts of the Gospel narratives and reveal their deeper symbolic meaning.

\paragraph{Historical Meaning}
Rene Guenon, in \emph{The Symbolism of the Cross}, explains the how that would work. First of all, he makes the point that the symbolic meaning has been largely forgotten.

\begin{quotex}
Christianity, at any rate in its outward and generally known aspect, seems to have somewhat lost sight of the symbolic character of the cross and come to regard it as no longer anything but the sign of a historical event. Actually, these two viewpoints are in no wise mutually exclusive; indeed the second is in a sense a consequence of the first; but this way of looking at things is so strange to the great majority of people today that it deserves dwelling on for a moment in order to avoid possible misunderstandings. \flright{\emph{The Symbolism of the Cross}}

\end{quotex}
\paragraph{The Symbolic Meaning}
The elite understand the law of correspondence. Therefore, they superadded the symbolic meaning of the events.

\begin{quotex}
The fact is that people too often tend to think that if a symbolical meaning is admitted, the literal or historical sense must be rejected; such a view can only result from unawareness of the law of correspondence which is the very foundation of all symbolism. By virtue of this law, each thing, proceeding as it does from a metaphysical principle from which it derives all its reality, translates or expresses that principle in its own fashion and in accordance with its own order of existence, so that from one order to another all things are linked together and correspond in such a way as to contribute to the universal and total harmony, which, in the multiplicity of manifestation can be likened to a reflection of the principial unity itself. \flright{\emph{The Symbolism of the Cross}}

\end{quotex}
We can interject here that throughout history, there have often been conflicts between the dogmatists and the esoterists. Dogmatists fear that symbolic interpretations denigrate the historical record. And faux esoterists mistakenly believe that the ``symbolic" meaning is the only true one, so that the historical meaning can be left behind. Rather, meaning must be grasped on multiple levels, from the physical to the soul, and to the spiritual levels. Only the can one understand the great unity of existence, with each level having its proper place in the whole.

\paragraph{Symbolism and the Superconscious}
Since psychoanalysis has been attempting to unravel the mysteries of the subconscious, spiritual symbols have, in our day, been mistakenly identified with eruptions of unconscious events into ordinary consciousness. This technique is particularly associated with Jungian analysis. It is certainly not obvious that esoteric knowledge is hidden away in some collective unconscious, accessible to all.

The best that such an approach can achieve is a psychological interpretation of symbols, which is gruel, not heartly food for the soul. But is you want millions of youtube likes, it seems to be a popular approach.

Jung failed to distinguish between the subconscious and the superconscious. The latter refers to realities that transcend the human state and are certainly hidden from most people. The subconscious, on the other hand, is accessible to everyone, including the insane. Special training in concentration and the purification of the soul is necessary to gain conscious access to the transcendent states.

The elite writers of the Gospels would have deliberately included the transcendent meaning in the text.

\paragraph{Gospel of John}
\begin{quotex}
there are also many other things which Jesus did; which, if they were written every one, the world itself, I think, would not be able to contain the books that should be written. \flright{\textsc{John 21:25}}

\end{quotex}
Although the professor refers to the synoptic Gospels, that is because John and Paul can be considered to be representatives of the elite she is describing. They are comfortable expressing themselves in metaphysical ideas. So we can justifiably use John's Gospel as an example.

Assume that John is correct and that it would take perhaps thousands of books to record the entire chronicle of Jesus's life on earth. Then why did John mention just seven of those thousands of miracles? Clearly, it is because those seven miracles have a special significance.


\hfill

\flrightit{Posted on 2022-06-19 by Cologero }

\begin{center}* * *\end{center}

\begin{footnotesize}\begin{sffamily}



\texttt{Jupiter on 2022-06-20 at 14:12 said: }

Condescending


\hfill

\texttt{Cologero on 2022-06-20 at 16:50 said: }

Au contraire, Grasshopper. We are grateful for all the Apostles, Prophets, Saints, Fathers, Sages, Mystics, etc., who have organized, preserved, and made the material available to us.


\end{sffamily}\end{footnotesize}
