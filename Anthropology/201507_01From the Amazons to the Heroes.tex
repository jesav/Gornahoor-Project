\section{From the Amazons to the Heroes}

In \emph{Revolt Against the Modern World}, \textbf{Julius Evola} expanded on the ages of these three races. The Amazonian period was a reaction against the male revolt. It represents the lunar principle that tried to restore the Demetrian race as it was being defeated by the Solar principle. We see the same mythological motifs return in our own day with the appearance of strong female warriors in virtually all action movies of recent years. All traces of vestigial patriarchy are being attacked to facilitate the return of the dominance of the Demetrian civilization.

Parallel with this movement, there is also the Aphroditic principle. The chthonic nature tries to reduce the masculine principle to the phallic. Men thus become enslaved to sensualism. Manifestations of this would include the addiction to pornography as well as the ``hookup" culture. In the phallic, or genital, stage, sexuality is centered on the genitals and finds its fulfilment in intercourse. Yet, there is a further degeneration as predicted by \textbf{Herbert Marcuse} in \textit{Eros and Civilization}: with the decline and overthrowing of patriarchy, people would become polymophous perverse\footnote{\url{https://en.wikipedia.org/wiki/Polymorphous_perversity}}, that is, they would find sexual gratification beyond the socially normative behaviors permitted by the patriarchy.

Thus, both through force and seduction, the hold of patriarchy is weakened.

The Heroic age represents the attempt to recover the primordial solar state by force. The Heroes retain the inner impulse toward transcendence and overcome the Amazonian and Aphroditic forces. He sublimates the phallic, erotic drive raising it to the ``spiritual plane of virility".

\paragraph{Esoteric Astrology}
Evola makes a brief allusion to the esoteric science of astrology. In its degenerate form today, astrology has become merely materialistic and quantitative. Instead traditional astrology was spiritual and qualitative, as he explains:

\begin{quotex}
The planets naturally are not the physical planets, but are designations used to define spiritual, super-individual forces, of which the physical planets are at most symbolic, sensible manifestations. 

\end{quotex}
In other words, esoteric astrology represents our ``essence", who we are, what we are born with, something that in some way we ourselves have chosen or accepted. Pop astrology, on the other hand, is something that ``happens" to us, from the outside, beyond our control, through merely material forces.

If you sense some truth in this, even if beyond the possibility of clear articulation, then you have the possibility to understand these ideas. We have been using the term ``preconception" to describe this inner essence of who you are.

\paragraph{Manifestations of the Races of the Spirit}
In the \emph{Sintesi}, Evola brings up the idea of the relationship between the third level of race and the second and first. Obviously, the first place to start is with the civilizations that have been dominated by the various spiritual races. Here, Evola refers back to the collection of \textbf{J J Bachofen}'s writings that he translated under the title of \emph{La Razza Solare}. I haven't been able to track down any copies of that work. However, I recently discovered that it has been republished, sort of, as \emph{Le Madri e la Virilità Olimpica: Storia segreta dell'antico mondo mediterraneo}\footnote{\url{http://www.amazon.com/Madri-Virilit\%C3\%A0-Olimpica-dellantico-mediterraneo-ebook/dp/B00HA7ITSQ}}; the other work apparently was never actually published.

That is why, in the German edition, Evola adds a paragraph making his ideas explicit since, obviously, German readers would have had no access to \emph{La Razza Solare}. Here is the German paragraph, followed by an English translation.

\begin{quotex}
Was die Entsprechung anbetrifft, die sich normalerweise zwischen Rassen des Geistes, der Seele und des Körpers verwirklichen soll, können also vorläufig diese Anspielungen genügen: die sonnenhaften und heroischen Rassen sind artverbunden dem Stil der Rasse des Leistungsmenschen und – als physische Rasse – dem nordisch-arischen, arisch-römischen und arisch-abendländischen Menschen. Die lunare Rasse fände den gemäßesten Ausdruck in den seelischen und somatischen Merkmalen der ostischen Rassen und den Überbleibseln jener uralten mittelmeerischen Rasse, die allgemein „pelasgisch`` genannt werden kann. Die aphroditische und dionysische Rasse könnte gut zu einigen Zweigen der westischen Rasse, insbesondere – wie gesagt – in ihren keltischen Formen passen. Die dionysische aber auch zur wüstenländischen und ostisch-baltischen Rasse und, ihren gespalteneren Aspekten nach, zur vorderasiatischen. Ein titanisches Element könnte sich wohl in der Seele und im Körper des Menschen fälischer Rasse ausdrücken, schließlich würde das tellurische Element physische Rassenkomponenten erfordern, die aus nichtarischen oder vorarischen Stämmen hervorgehen, wie es beispielsweise bei den afrikanisch-mittelmeerischen und teilweise bei den im semitischen Typ vorhandenen Rassen der Fall ist. Es liegt also ein neues und weit gespanntes Forschungsfeld vor uns, für welches es hauptsächlich gilt, das ihm zustehende Interesse in den neuen Generationen zu wecken. Dann wird das schon Errungene entsprechend entwickelt werden bis zu einem wirklich umfassenden Rassenbewußtsein. 

\end{quotex}

\hfill

H/T to marcelkol\footnote{\url{https://mmarcelkolaric.wordpress.com/}} for the following English translation:

\begin{quotex}
In regard to the correlation that should manifest itself between the races of the body, the soul and the spirit the following remarks should suffice for now: the solar and heroic races are intimately connected with the nature of the man of achievement and – as physical race – with the Nordic-Aryan, Roman-Aryan and Occidental-Aryan human. The lunar race expresses itself in its most complete form in the soul and the body of the Eastern European races and the remnants of the old Mediterranean race which could generally be described as `Pelasgian'. The Dionysian and Aphroditic races could correspond to some branches of the Western race, especially – as already mentioned – in its Celtic forms. The Dionysian race could also correspond to the desert race and the East Baltic race, and – due to its conflicted aspects – to the Near Eastern race. A Titanic element could express itself in the body and the soul of the Phalian race, since the Telluric race would require a physical racial component stemming from non-Aryan or pre-Aryan lines, as is the case with the African-Mediterranean and some of the races of the Semitic type for example. We're looking at a new and very broad field of research and priority should be given to awakening the interest of the young generations so our present knowledge can be developed further towards a comprehensive racial consciousness. 

\end{quotex}

\hfill

\paragraph{The Amazonian Race}
When faced with the usurpations of the titanic type, a Demetrian race no longer has available the higher authority from above which is characteristic of solar man in order to reaffirm itself, and makes the violent and materializing forms of its adversary its own; it then marks out a new type, the Amazonian man. In the myth, the Amazon actually appears as the woman (lunar spirituality) who can no longer affirm herself against the abuses of man or, simply, up against man (Titanic spirituality), except by resorting to a mode of being that is just as masculine, diverging thereby from its original Demetrian nature. Briefly, it is a question of the usurpation of force on the part of degenerate lunar elements. More generally, the Amazonian man would be the type who remains lunar in essence, even if he asserts himself through the use of force, although the force is material, not spiritualized (as we will instead see to be the case for the heroic race). Thus, for example, as far as this rapprochement will seem paradoxical to some, there is an Amazonian phenomenon when a priestly caste challenges the secular power to impose dominion that can no longer be secured solely on the basis of its spiritual authority.

The myth shows us the contrast between the Amazons and both the Dionysian and heroic types. In the first case the defeated Amazons are brought back under the Demetrian law, that is, their normal mode of feminine-lunar being. In the second, their destruction gives rise to a new solar and masculine period. When one sees what the heroic type of race means in this context, everything will confirm the interpretation given. There could therefore be a certain relation between the Amazonian and the Titanic or Promethean man, since even the latter is characterized by the usurpation of a force which is not suitable to his own nature. However, in the case of the Amazonian man, it is a material force, while in the case of Titanism, it is instead a transcendent force, of which only the solar type can take possession without an abuse of power. This short note will suffice, since it is not difficult to deduce the different distinctive characteristics for the type of Amazonian race through transpositions in the various domains.

\paragraph{The Aphroditic Race}
Another race of the spirit is the properly Aphroditic race; tellurism, i.e., the adherence to terrestriality, assumes the forms of an extreme refinement of material existence in it, and not infrequently then promotes an opulent development of everything that is pomp and luxury in the outer life, therefore also in the world of the Arts and aesthetic sentiment. But, in the inner life, a passivity and a lunar inconsistency subsist, compensated by a particular prominence given to eroticism thus also to everything which is related to woman, which then exercises control because of that and silently secures dominance for itself.

Bachofen followed the development of such a way of being in its relations with the twilight phases of the Dionysian and Aphroditic cult of antiquity. He even proposed a correspondence to the races of the body, where he noted the particular diffusion that these forms of ancient worship had among the Celtic races. It is indeed not arbitrary to recognize a strong component of the Aphroditic race both in the branch the race theorists call the West European race, as well in what Clauss defined as the type or race of the \emph{Darbietungsmensch}. The Aphroditic race also preserves the Dionysian theme to a certain degree, where the quest for pleasure and sensation is unified in the joyful feeling of a destruction, of a passing away of sensation, i.e., of the same law of mortal natures, of life that rises and fatally dies in the eternal cycle of the generations.

The Aphroditic race on the one hand, and the telluric on the other, represent the extreme limits of the forms contained in the Nordic-Aryan cycle, the points beyond which they descend, through regression and through the lower elements introduced by crossbreeding, into the domain of the races of nature.

\paragraph{The Heroic Race}
Finally, we can consider the race of the heroes. The term hero is taken not in its common meaning, but with reference to the mythical traditions recounted by Hesiod, according to which, in the cycles of an already deviated and materialized humanity, Zeus, that is, the Olympic principle, generated a race virtually equipped with the possibility of regaining the primordial state through action, that golden or solar state of the first generation of the hyperborean cycle. Beyond the myth, it is a question of a type in which the Olympic or solar quality is no longer its nature, but a task, to be realized on the basis of distinct heredity or, better said, of a more pronounced atavistic component of the primordial race. Nevertheless, it alone is capable of making actual what had become latent and to regain what had been lost also by means of an internal transformation, of an overcoming, often depicted in terms of a second birth or initiation.

In \emph{La Razza Solare} with the extracts from the works of Bachofen as well as a more precise description of these types, allusions to the most likely correspondence to the various races of the body are found and, in part, also to those of the research of the second degree that Clauss carried out. If we limit ourselves to pointing out the characteristics related to the highest plane, i.e., to the relationships of man with the spiritual world, in that work some applications and conclusions are found and one can find the values, institutions, symbols, customs, and forms of law that are predominantly reflected in one or the other race of the spirit.

Using such points of reference provides the possibility of surpassing two-dimensional history, of discovering the influences that have clashed, intertwined, or been superimposed behind the scenes of ancient civilizations and also the meaning of the prevalence, decline, or change of religious and ethico-social conceptions. One of our other works, Revolt Against the Modern World was an essay of such a metaphysics of the ancient civilizations, while in the extracts of Bachofen's works, many elements are pointed out that are apt to favour further researches in that direction. Even many aspects of the modern world and contemporary civilization are presented in an unexpected and revealing light when this information is utilized.

We must not neglect to notice that some designations used by this classification of the races of the spirit—solar, telluric, lunar, etc. — like others that could have been adopted instead, while they were dictated by analogical reasons and references to typical ancient cults, they also raise the possibility of investigating the most profound meaning of traditions, like the one, e.g., according to which the decisive characteristics of men and even their terrestrial destinies, to a certain degree, would be determined by the choice of a given planet made by the spiritual nucleus of the personality before birth. For example, there is the conviction, professed even in the Roman world, that the regal man destined to regal dignity, because dominus natus [royal by birth], was the one who had made the influences of the sun his own. In this symbolic teaching that is found in much more precise and detailed forms in Aryan-Iranian and Indo-Aryan traditions, everything that we previously said in regard to the mystery of birth is veiled: the planets naturally are not the physical planets, but are designations used to define spiritual, super-individual forces (not without relationship to the notion of ``demons" which everyone chooses), of which the physical planets are at most symbolic, sensible manifestations. The essence of such a doctrine is related therefore to that ``nature" or transcendental election that was able, alone, to resolve the strongest objection that could be raised against the racial idea and the results of racial theory of the second degree, in their turn, can illuminate to the extent possible for human comprehension. Spontaneously beginning to hear terms like that of ``solar man", ``lunar man", etc., as apt and expressive, is already cogent for such an understanding.



\flrightit{Posted on 2015-07-01 by Aeneas }

\begin{center}* * *\end{center}

\begin{footnotesize}\begin{sffamily}



\texttt{marcelkol on 2015-07-02 at 15:12 said: }

The nomenclature of German 20th physical anthropology isn't very elaborate and doesn't translate to English very well, so the racial types described in this paragraph have to be understood as very general categories. Evola also draws on the racial classification developed by Ludwig F. Clauss which is concerned with the `style'. or nature, of the various physical races. This roughly corresponds to the second level of racial classsification – the race of the soul – in Evola's system. Clauss coined the term `Leistungsmensch’, ‘man of achievement'. for a type who feels an irresistable pull towards the infinite, which he thought would correspond to the Nordic race of the body.

My translation (feel free to correct):

\begin{quotex}
As regards to the correlation that should manifest itself between the races of the body, the soul and the spirit the following remarks should suffice for now: the solar and heroic races are intimately connected with the nature of the man of achievement and – as physical race – with the Nordic-Aryan, Roman-Aryan and Occidental-Aryan human. The lunar race expresses itself in its most complete form in the soul and the body of the Eastern European races and the remnants of the old Mediterranean race which could generally be described as `pelasgian'. The Dionysian and Aphrodisian races could correspond to some branches of the Western race, especially – as already mentioned – in its Celtic forms. The Dionysian race could also correspond to the desert race and the Eastern-Baltic race, and – due to its conflicted aspects – to the West Asian race. A Titanic element could express itself in the body and the soul of the Falian race, since the Telluric race would require a physical racial component stemming from non-Aryan or pre-Aryan lines, as is the case with the African-Mediterranean and some of the races of the Semitic type for example. We're looking at a new and very broad field of research and priority should be given to awakening the interest of the young generations so our present knowledge can be developed further towards a comprehensive racial consciousness.

\end{quotex}
Of interest perhaps: the Falian type was often idealized in National-Socialist propaganda as the embodification of the original German race, which was purported to exist in its highest purity in the peasants and the rural population. The fact that Evola identifies in this race the Telluric-Titanic element in Europe shows the contrast between his thought and political reality at the time.


\hfill

\texttt{Max on 2015-07-14 at 16:06 said: }

The spiritual races of Evola are different ways of relating to divine creation and its source – modes of being – that reveals something fundamental of the person. It affects ones ``preconceptions", of which some may be acquired but the true are innate and natural. The Spirit descends vertically, and the spiritual race describe how strong this influence is compared to the horizontal influence, and in which manner and degree it is known or felt by an individual. In the realm of spirit, beings are differentiated and unique, making the concept of spiritual race somewhat lacking as ``race" may give the impression of something contingent. It rather seems to be used, in the ``third degree", to represent the manner in which existence itself is conceived, which is dependent on how we actually came into existence. It could be quite difficult for people with qualitatively different experiences in this area to understand another, transcendent, viewpoint. In short, what people believe regarding their own ``genesis" expresses something of their inner nature. In a time like ours, the circumstances constitutes a ``trial" and test of strength for anyone wishing to escape consensus reality. In another time, these sorts of challenges did not occur ``naturally" in the same way, so there are advantages of living at this moment. The concrete state of the world has increasingly come to provide an outer resistance that could provide opportunities of spiritual growth for those with a heroic approach to life. Strangely enough, some who think of themselves as defenders of traditional values seem to be of the opinion that the only option is to passively await better times.

Evola mentioned that language is an area of investigation for the third degree of race, and I would like to draw attention to a perhaps insignificant detail that has occupied me. In Germanic languages ``art" means ``mode" ``type" ``manner" ``species" or something of that nature, basically a unique way of being. In English on the other hand, art is a skill or the result of a creative work, and i think that we would come closer to the underlying idea by combining these aspects. The word is used in the German text presented here as ``artverbunden" and translated as ``intimately connected". It is a matter of conforming to form, being freely ``bound" to ones essence. We should also know that due to the free will of man, this amounts to a creative activity – we have to will ourselves. Man realizes himself, in a sense, by ``creating himself" in union with the will of God. It is conveyed in the passage that a distinguishing mark of the solar and heroic races is as ``man of achievement" – that is, a type that seeks to achieve himself to become accomplished. Compare the idea of the ``solar race" with Aguéli's claim of the Artist being someone who experiences a ``personal sun" as an inner light to give expression to. This work amounts to actualizing ones potential, by which this light creatively comes to be known through a unique mode or art of being conforming to ones essence. Thus the heroic and solar man regards living itself as an art form to perfect during our brief time here.


\hfill

\texttt{Max on 2015-07-17 at 09:14 said: }

To illustrate my previous comment, here is a quotation that is included in the last letter of the Meditations on the Tarot, treating the subject of art. We can see from the context that the author is aware of the multilayered meaning of the word even as it is not spelled out directly.

``Schopenhauer once said that things appear to the child so bathed in splendour and of such a paradisical nature because they experience naively in each particular thing the idea of type (species). This splendour of inner reality is entirely lost for man who has attained to the maturity of rational thought, when he comes out of the `childlike state' of animated and living perception and is given over to laws of pure abstraction. Thus, each time we are in the state of experiencing the idea in form, we are — like the child — within Nature. Goethe was such a child. (p. 114)" Edgar Dacqué, Leben als Symbol


\hfill

\texttt{Sparrow on 2018-03-17 at 01:05 said: }

Despite how fascinating this is, I can't help but wonder how it is supposed to apply to the ``real world." the abstract wording and lack of historical background makes it hard to see how it's any different from a fantasy novel about elves, giants, and other mythical beasts.

Furthermore, ``The lunar race expresses itself in its most complete form in the soul and the body of the Eastern European races." Russian culture and history has been dominated by an absolute patriarchal father figure since time immemorial, and exalted masculinity, quite the opposite of the traits that are supposed to be the Lunar man.

Wouldn't it be better for young traditional-minded folk to read the heroism of Christ and the saints rather than what has no more basis in the Tradition of the West than a Dan Brown novel?


\hfill

\texttt{Aeneas on 2018-03-17 at 16:39 said: }

Just because, Sparrow, it is so fascinating. You seem too stuck on the historical and the empirical, while the ``real world" is happening on the astral and mythological planes.


\end{sffamily}\end{footnotesize}
