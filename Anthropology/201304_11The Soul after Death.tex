\section{The Soul after Death}

It is little known, even to many Orthodox, that there is a theology of postmortem states that rivals in detail the \textit{Tibetan Book of the Dead}\footnote{See Section~\ref{sec:EvolaTibetanDead} in this book.}. Moreover, the Orthodox teaching is practical and based on experiences, not idle speculation. \textbf{Fr. Seraphim Rose} authored a book by the title The Soul after Death that gathered these teachings, while relating them to contemporary descriptions of near-death experiences, out-of-body experiences, not to mention the tales of Emmanuel Swedenborg and even the \textit{Tibetan Book of the Dead}.

Although Fr. Rose is said to have been influenced by Rene Guenon at one point in his life, precious little of that shows through in this book, even when discussing the spirits, a topic to which Guenon dedicated a book-length treatment. There is possibly a faint echo when Fr. Rose decided not to use the \emph{Egyptian Book of the Dead} because there is not a living tradition to explain its symbolism. That is unfortunate, since some basic metaphysics would have made some difficult points clearer. So before reviewing the book, it may be useful lay the groundwork.

\paragraph{The Psychic Realm}
The experiences Fr. Rose describes belong to the psychic realm, that is, soul experiences, between the material world and the spirit, properly speaking. This is the realm of images, visions, feelings, and so on. In other words, it is that part of our experience that cannot be explained by the five senses. Guenon and Fr Rose take it for granted that such a realm exists and pace the materialists, it is independent and not reducible to physics.

On the other hand, it is not the ``spirit", either, although moderns, following Descartes, tend to classify everything that is not material as spiritual. The spirit is transcendent, not only to the material world, but also to the psychic world. But is this psychic realm totally subjective or is there an objective reality to it, i.e., something that is not-I, yet participates in ``my" psychic or soul experiences.

\paragraph{The Dream World}
Since we all have had experiences of dreams, it will be beneficial to analyze them first. Guenon explains that a dream is a creation of the I, which creates all the characters, places, and events; or put another way, it could be the projection onto the undifferentiated state of deep sleep. If post-mortem experiences are ``like" these, we can ask the question of exactly where they occur.

This is the same as asking where a thought occurs. Thoughts have a temporal component, since we think one thought after another, which can all be related sequentially. However, there is no ``where"; i.e., one thought is not to the left of another, or across from it, etc. Yet, images in a dream do have a spatial aspect; in the dream, we ``go places", there is different scenery for different places, and so on. Furthermore, since we ``see" things, that brings up the question of a ``body"; do the figures in a dream have an ``astral body" that we then ``see"?

Space and time are strange in that they are not metaphysical principles, yet they are not objects of sensory experience. That is why we can speak of God being beyond time and space. However, they are required in order to even have sensual experiences. Specifically, for example, I can see a tree or a desk among all the objects that are not those things; but I don't see space one place and not-space somewhere else. Space and time, then, are conditions of manifestation but not the principle of manifestation.

Now Guenon tells us that space and time may be quite different experiences in different states of being. In a dream, space can be traversed instantly just by thinking of or visualizing something else. Curiously, space is different in a dream, but time seems to be related to our earth time.

\paragraph{Knowing and Seeing}
Another issue to explore is to be clear about how we ``know" something. According to metaphysical teachings, there are three ways: subject-object perceptions or sensations, discursive thought, and non-dual intuition. When I ``see" something, how do I know what it is? It the hoary example, the sage and the ignorant man both have the experience of something in the grass. The latter is startled, since he sees a snake, but the sage is calm because he knows it is a rope. The ignorant man is projecting onto his experience, whereas the sage is intuiting the real nature of the thing. It is naïve realism to assume that we know what something is, just by looking at it. No, the external thing and our sensation are united in the intuition of the idea. That is why it is said that to know and to be are identical. When I see a squirrel, for example, I have the idea, or form, of it in my mind. To that extent I am a squirrel, although just in form, not in matter.

So that leads to a conundrum. When I experience a being in the psychic realm, how do I recognize it for what it is? With intelligent beings, this gets tricky. For example, if I see two men, one holding a gun pointed at the other, exactly what am I seeing? Is it a mugger accosting a tourist? An FBI agent capturing a criminal? A scene from a movie being filmed? Here the senses are clearly lacking and I need to understand the interiority of the other beings in order to grasp the full nature of the situation.

This will come up in Fr. Rose's discussions. In an out-of-the-body experience, how do I really know what I am experiencing? That is why Tradition is so important. The Tibetans prepare the dying one to understand what he will be experiencing in the days after death. Likewise, Fr. Rose is offering the same type of preparation.

\paragraph{Psychic Experience}
Around the beginning of the 20th century, the attempt to prove life after death in a scientific way was being taken seriously. This attracted men of exceptional intelligence and erudition, including William James and Carl Jung. Of course, those they depended on were of lesser accomplishment, as Rene Guenon amply documented in his book \textit{The Spiritist Fallacy}. Fr. Rose does not mention Guenon's book, although he dealt with the topic.

This brings up the notion of ``scientific proof" which is different from just seeing. For example, from just looking, it appears that the earth is still and the sun is in motion. Yet that is not a scientific conclusion. Now, there may be cases where looking suffices. The existence of a black swan can be shown simply enough, but finding such a bird. We know what a swan is and what black is; the opinion of the swan is irrelevant.

The situation is not the same in regards to a dead soul.  Usually, there is nothing more than a medium and her claim to be communicating with the spirit of a dead human, or perhaps a discarnate higher being. I have had the occasion more than once to witness a medium communicate with such a spirit. She was a pretty Swiss woman who would enter the trance state, close her eyes, alter her voice, and then ramble on for an hour or more in florid prose. I recommend it for its soporific properties.

Sometimes the spirits affect material things through rappings or breaking things. Guenon denies that discarnate spirits can affect physical objects, so I'll accept that for now. Sometimes they may make an appearance, but that is hardly a proof of anything. What does it mean for a spirit to ``appear"? A non-material being cannot be experienced through the senses. Thus, it can never be as conclusive as finding the black swan.

So ultimately the only proof there is of the identity of such a being is his own claim. Fr. Rose then shows how to understand what is really going on based on the Traditional teachings of the Eastern Empire.

\paragraph{Out of Body Experiences}
Fr. Rose begins by discussing certain phenomena that have been gaining currency. These are near death experiences and deliberately induced out-of-body experiences. Although these reports are not new, they seem to be, probably because of the improvements in medical technology that have made it possible to resuscitate patients who were nearly dead. Fr. Rose summarizes such near death experiences. The person seems to have left his body while retaining consciousness of events around his body. Fr. Rose provides many such examples, whose veracity he does not deny, when he concludes:

\begin{quotex}
None of this should sound very strange to an Orthodox Christian; the experience here described is what Christians know as the separation of the soul from the body at the moment of death. 

\end{quotex}
To make his point, he quotes similar experiences from Orthodox sources. He then makes use of the cross-cultural study, using data from the USA and India. What is common is the apparition of dead relatives and friend, although in India there are also apparitions of Hindu gods around the time of death. Fr. Rose again makes the point that such studies are startling to the modern mind, but not at all to the Orthodox Christian. Interestingly, he quotes \textbf{Pope Gregory the Great}, who wrote extensively on this topic. Although Pope Gregory was quite influential in the Medieval west, this aspect of his teaching has not been effectively transmitted. He explains:

\begin{quotex}
It frequently happens that a soul on the part of death recognizes those with whom it is to share the same eternal dwelling for equal blame or reward. … It often happens that the saints of heaven appear to the righteous at the hour of death in order to reassure them. And, with the vision of the heavenly company before their minds, they die without experiencing any fear or agony. 

\end{quotex}
Fr. Rose draws attention to what this really means. At the time of death, ordinary sinners recognize people, but the saints of heaven appear to the righteous. That is because deceased sinners have no contact with the living, save in exceptional circumstances, but the saints actively intercede for the living.

We also have the authority of St. Augustine who asserts that the souls of the dad are in a place where they do not see the things which go on and transpire in this mortal life. I know this is contrary to the sentimental feelings of people today that their deceased relatives are somehow watching over them. It is not so simple, as there may be apparent manifestations of the dead, through the workings of the devil, as well as true manifestations of the saints through the workings of the angels.

There are two other experiences to draw attention to. One is the contact with the ``Being of Light", which rapidly increases in brightness. Since it is formless, it is described in different ways since the person will project his own interpretation onto it. Note that the \textit{Tibetan Book of the Dead} describes the same experience.

The other one is the feeling of peacefulness that is experienced. This is accounted as the normal sensation of the soul separating from the body. Since he body is so heavy and tends to dominate the life of the soul, it should not be surprising to have that feeling of release. Since in most of these cases the person has been sick or physically hurt, it makes sense that the separation from such pains and agonies would bring such relief. However, such natural phenomena do not necessarily reflect a supernatural cause.

\paragraph{Angels}
In discussing angles, Fr. Rose makes an important point about the damaging influence of Descartes, who regarded everything that is not ``matter" to be ``spirit". This would be the angels on the same level as God, since they are all spirit. This is a warning that everything spiritual is not the same, and may not even be good. Thus the demonic, the angelic, even human imagination are not material, yet are quite different. Hence, we understand the angelic orders as consisting of purely spiritual beings, i.e., they are beings but without a material form. This agrees with Guenon and Thomas Aquinas.

Yet, scripture and tradition say that angels can be ``seen", typically as ``arrayed in white". To speak of their form is a metaphor and ``they have need to be clothed in a subtle body whenever God permits them to act on bodies," and that ``when God opens the spiritual eyes of a man, he is capable of seeing spirits in their own form." Nevertheless, Fr. Rose goes to great lengths to make it clear that angels ``look like" men. In support, he quotes St. Augustine, who said that the soul after death experiences himself as a body. Well, this makes some sense since the soul is indeed the ``form" of the body.

In my opinion, one can go too far in that direction of assuming our only experience of angels (or demons) is in the outward form of a man. It encourages a craving for the unusual and the spectacular. Actually, our experience of such beings is through our intellect, since they are themselves pure intellect. That is, they will come as thoughts, or a constellation of ideas, an inspiration, or a temptation. It behooves us to pay close attention to our thoughts and how they affect us. We exist truly in a vast noosphere, although in more depth and complexity than its merely humanized form described by Teilhard de Chardin. Some, then, may even experience these thoughts as real beings.

That leads Fr. Rose to the teaching of the two angels, one of whom is the guardian angel, who meet the soul at the hour of his death. Of course, Fr. Rose backs this all up with extensive examples from tradition and the experiences of sinners and saints alike. Since it was established that the soul has a human form, these angels grasp the soul's subtle body to lead him on his journey into the afterlife.

Hence, Fr. Rose dismisses the common experience of a formless being of light in near death experiences, suspecting that it is most likely satanic. He relates some stories of sinners on their deathbeds who experience great horrors, in a manner reminiscent of the \textit{Tibetan Book of the Dead}. Fr. Rose makes an interesting point about the experience of Americans who seldom report such experiences. He attributes it to the unjustified optimism of Americans or the Protestant presumption that they are ``saved". In these cases, the temptation is more like a seduction than terror. That is why the spirits must be ``tested".

\paragraph{Spirits of the Air}
Fr. Rose describes man in the primordial state: his body was immortal, without infirmities, unbothered by heaviness or sinful land fleshly feelings. His senses are most subtle and totally free. Man was capable of the sensuous perception of spirits.

Man's soul and body were changed after the fall. The soul separated from the body, which is the definition of death. The body was subject to infirmities and subject to hostile influences. The body became like the bodies of the beasts. Where, previously, man was in communion with the angels, he is now more comfortable with the fallen spirits.

Fr. Rose points out that demons usually assume the appearance of bright angels in order to deceive men, typically by mixing truth with error. Of course, we would say they appear as lovely thoughts, thoughts that claim to be compassionate or progressive. They make the hearer ``feel good about himself", or superior. Fr. Rose warns about engaging such spirits in conversation when they appear in human form. However, as this is quite unlikely, my opinion is that he should have spent more time on revealing how those spirits infiltrate the mind. He does quote \textbf{Bishop Ignatius} in this regard:

\begin{quotex}
The fallen spirits act on men, bringing them sinful thoughts and feelings; but very few men attain to the sensuous perception of spirits. 

\end{quotex}
We find ourselves in the realm of fallen spirits, surrounded by them, enslaved b them. Having no possibility to break in on us, they make themselves known to us from outside, causing various sinful thoughts and fantasies.

It is interesting that Fr. Rose places the demons in the ``air". In the Hermetic diagram of the spiritual hierarchy\footnote{\url{http://www.gornahoor.net/?p=1941}}, the air is the lowest level of subtle beings. The density of these fallen spirits in the air acts to block access to all the superior levels.

\paragraph{Tollhouses}
Finally, Fr. Rose brings us to the teaching of the toll houses. Again, there is a warning to avoid picturing the tollhouses in crude sensuous imagery, but to understand them in a spiritual sense. This does not mean that the tollhouses are not real, but rather that the experience of time and space in that realm is different from that of the physical world. The debate about whether the souls ``sees" these images or finds it necessary to ``express" them in sensual terms is moot. The important thing is that the experience itself is real.

Although I am not bringing in all the details in this review, Fr. Rose backs up his account through the teachings of the major theologians. Christians, he points out, need to keep the fact of death always in mind; in this, they are like the Buddhists and even others. \textbf{St. Isaiah the Recluse} writes:

\begin{quotex}
Christians should daily have death before our eyes and take care how to accomplish the departure from the body and how to pass by the powers of darkness who are to meet us in the air. 

\end{quotex}
There are other eerie similarities to the \textit{Tibetan Book of the Dead}, which gives credence to the idea that this teaching is based on actual experience and not mere dogma. For example, \textbf{St. Cyril} writes:

\begin{quotex}
What fear and trembling await you, O soul, in the day of death! You will see frightful, wild, cruel, unmerciful and shameful demons, like dark Ethiopians, standing before you. The very sight of them is worse than any torment. The soul, seeing them, becomes agitated, is disturbed, troubled, seeks to hid, hastens to the angels of God. 

\end{quotex}
I cannot go into the complete teaching here, but can only briefly summarize it. The soul is guided by the tow angels past a series of up to twenty tollhouses. Each one is dedicated to a different test, and the demons test the soul in regard to the particular tollhouse. That is, they will point out how the soul was guilty of a particular sin at any point in this life. The guiding angels then would point out his repentance or good acts that outweighed those sins. If the soul passes the test, it moves on to the next tollhouse; otherwise, the guiding angels abandon him to the demons.

Once again, it needs to be emphasized that these tests are to be understood in a spiritual, not material, sense. I would also be remiss if I didn't point out that this is something you can try at home without waiting for death and the tollhouse tests. As a matter of fact, if you have the fact of your death always in mind, you would make it a high priority.

\textit{The fact of death is certain, the time of death is unknown.}



\flrightit{Posted on 2013-04-11 by Cologero }

\begin{center}* * *\end{center}

\begin{footnotesize}\begin{sffamily}

\texttt{Cologero on 2020-09-27 at 16:14 said: }

From Multiple States:

\begin{quotex}
The difficulty of this transposition or this passage from the manifested to the non-manifested, and the apparent obscurity that results) is the same as is encountered in trying to express, in the measure that they are expressible, the relations between time (or more generally duration in all its modes, that is to say, the whole condition of successive existence) and eternity.

\end{quotex}
So, yes, it is difficult to relate corporeal time to successive existence in other states, although I gave it a good shot.

I quoted Guenon re dreams: ``Guenon explains that a dream is a creation of the I, which creates all the characters, places, and events", in other words, they are ``extensions of us". As for the subtle body, see \textit{Subtle Body}\footnote{\url{https://en.wikipedia.org/wiki/Subtle_body}} on Wikipedia, for example. In Hinduism subtle body is a generic term referring to different bodies or sheaths. In English, etheric and astral bodies, etc. are more or less equivalent to the Sanskrit terms.

\hfill

\texttt{Mercurius on 2013-04-16 at 01:11 said: }

It is curious to observe in these things the ongoing regression of scientism into materialism. At the time Rose wrote the book, for a number in science, psychology, and parapsychology, ``NDE's" and ``OBE's were minimally, approached as indicative of either ``another" reality, or aspect of reality–even if, from the metaphysical point of view, their interpretations were incorrect. So, the picture was sort of like this–spiritists viewed the phenomenon as ``heaven", or some desirable realm. Metaphysics, not denying the claims or experiences, always pointed out that the realm experienced is the ``intermediary" world, the ``psychic" world, that ``darkly-splendid world wherein continually lieth a faithless depth, Hades wrapped in clouds, delighting in unintelligible images", as the Chaldean Oracles describe it–and warn us to ``not stoop" into it. Spiritists and the scientists who studied or followed their leads affirmed this world, and could not imagine anything beyond it; while metaphysics affirmed it, yet pointed beyond it still. Now, interestingly, not only does science deny the transcendent, but denies the validity of the intermediary world itself, as it had not in the past. This article is quite recent, although several similar have appeared in the past year or so:

\url{http://edition.cnn.com/2013/04/09/health/belgium-near-death-experiences/index.html}


\hfill

\texttt{Mihai on 2013-04-16 at 03:44 said: }

Mercurius, could it be a sign that the scientistic establishment intuits its bankruptcy and is attempting all it can to keep itself in power ?

It is obvious that the dream-world of the scientistic rationalist lies in tatters- the new-age and neo-spiritualist movements are getting a stronger grip on many levels of society, a fact which, for the materialist, is as equal a disaster as it is for the traditionalist, although the two have completely different motives.


\hfill

\texttt{scardanelli on 2013-04-16 at 09:47 said: }

I'm currently reading Orthodoxy and the Religion of the Future, but will get to The Soul After Death next. For any others as ignorant of Eastern Orthodoxy as I am, there are useful podcasts offered by Ancient Faith Radio that detail the history, theology, and practical aspects of Orthodoxy. I am too new to this area of Christianity to make any claim to their ultimate reliability as a source, but they are useful as a broad overview. Much better ``car listening" than the news.


\hfill


\end{sffamily}\end{footnotesize}
