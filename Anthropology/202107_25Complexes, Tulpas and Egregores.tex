\section{Complexes, Tulpas and Egregores}

\begin{quotex}
When the Prophet said, ``People are asleep and when they die, they wake up," he meant that everything that man sees in the life of this world is in the rank of the dreams of someone who is asleep, so it must be interpreted. Phenomenal being is imagination (khayal), but it is God in reality. \flright{\textsc{Ibn Arabi}, \emph{Bezels of Wisdom}}

\end{quotex}
Scientists often ponder the ``mystery of sleep", although ``waking up" is the real mystery. Trees are asleep, indicating that sleep is the natural state of life. What we take to be the ``real world" is to the awakened person as the dream is to the dreamer. Waking up, then, is understanding that the phenomenal world is an image and that God is ultimate reality.

Just as the dreamer creates all the characters and aevents in his dream, so too does the awake person. Julius Evola repeats the traditional teaching in his intellectual autobiography:

\begin{quotex}
I have always subscribed to the traditional doctrine that we have wished all relevant events in our life before birth. 

\end{quotex}
A deeper way to understand that is that God's Will and the personal will are fully aligned at the moment of conception.

You might suppose that everyone would wish for a happy life, but that does not take into account the perverse aspects of consciousness. Just as the dreamer may create a nightmare, the perverse will can do something similar. Esoteric teaching identifies three possible outcomes, in which the creative imagination creates an evil and then projects it onto the world as though it were an external being. These are:

\begin{itemize}
\item \textbf{Complexes}. These are subjective creations of the individual consciousness. 
\item \textbf{Tulpas}. These are projections usually of a group mind. 
\item \textbf{Egregores}: These are the creation of false doubles of collective entities. 
\end{itemize}
\textbf{Valentin Tomberg}, in \emph{Meditations on the Tarot}, treats these phenomena extensively, so I will be satisfied to quote him directly.

\paragraph{Complexes}
\begin{quotex}
Autonomous philosophical systems separated from the living body of tradition are parasitic structures, which seize the thought, feeling and finally the will of human beings. In fact, they play a role comparable to the psycho-pathological complexes of neurosis or other psychic maladies of obsession.

\end{quotex}
Such systems are also called ideologies. Ideologies ignore Tradition, so they seem novel and attractive. Unfortunately, once a being is under the spell of an ideology, it is difficult to shake it off. The high level of psychological and emotional problems seen today are the result.

As a result, they take on an apparently objective life of their own.

\begin{quotex}
Both Eliphas Levi and the Tibetan masters are in agreement not only with respect to the subjective and psychological origin of demons but also with respect to their objective existence. Engendered subjectively, they become forces independent of the subjective consciousness which engendered them. They are, in other words, magical creations, for magic is the objectification of that which takes its origin in subjective consciousness. 

\end{quotex}
Demons, then, are engendered subjectively but experienced objectively. In that case, they appear in consciousness almost like a different being.

\begin{quotex}
Demons that have not arrived at the stage of objectification, i.e., at that of an existence separate from the psychic life of their parents, have a semi-autonomous existence which is designated in modern psychology by the term ``psychological complex". … A psychopathological ``complex" is therefore a demon, when it has not come from outside but is engendered by the patient himself.

\end{quotex}
\paragraph{Tulpa}
In Tibetan esoteric work, a Tulpa is created consciously and deliberately. The purpose is to learn to recognize and then overcome one's fears.

\begin{quotex}
With respect to Tibet, we find there the singular phenomenon of the conscious practice of the creation and destruction of demons. It appears that in Tibet, it is practised as one of the methods of occult training of the will and imagination. The training consists of three parts:

\end{quotex}
\begin{itemize}
\item \begin{quotex}
the creation of tulpas (magical creatures) through concentrated and directed imagination, 
\end{quotex}
\item \begin{quotex}
then their evocation and, 
\end{quotex}
\item \begin{quotex}
lastly, the freeing of consciousness from their hold on it by an act of knowledge which destroys them 
\end{quotex}
\end{itemize}
\begin{quotex}
— through which it is realised that they are only a creation of the imagination, and therefore illusory. The aim of this training is therefore to arrive at disbelief in demons after having created them through the force of imagination and having confronted their terrifying apparitions with intrepidity. 

\end{quotex}
\textbf{Alexandra David-Neel} in \emph{Magic and Mystery in Tibet} explains that early scepticism offers no spiritual benefit. These Tulpas appear real to those not part of the experiment. Moreover, Tulpas exist through the creations of others.

\begin{quotex}
The teachers do not approve of simple incredulity, they deem it contrary to truth. The disciple must understand that gods and demons do really exist for those who believe in their existence, and that they are possessed with the power of benefitting or harming those who worship or fear them. However, very few reach incredulity in the early part of their training. Most novices actually see frightful apparitions. Must we not also consider that we are not the only ones capable of creating such formations? And if such entities (tulpas, magical creatures) exist in the world, are we not liable to come into touch with them, either by the will of their maker or from some other cause? Could one of these causes not be that, through our mind or through our material deeds, we bring about the conditions in which these entities are capable of manifesting some kind of activity? … One must know how to protect oneself against the tigers to which one has given birth, as well as against those that have been begotten by others. 

\end{quotex}
\paragraph{Egregore}
An egregore is like a tulpa except that it is the creation of the perverse collective human will and imagination. ``There are superhuman spiritual entities which are not artificially engendered, which manifest themselves and reveal themselves." Egregores, then, are artificially created doubles of such genuine spiritual entities. They are created from below, not by God. Tomberg lists some examples of these doubles:

\begin{quotex}
Although God, Christ, the Holy Virgin, the spiritual hierarchies, the saints, the Church (or the Mystical Body of Christ) are real entities, there still exists also a phantom or egregore of the Church, which is its ``double", just as every man, every nation, every religion, etc., have their ``doubles". But just as he who sees in Russia, for example, only the bear, in France only the cock and in Germany only the wolf, is being unfair towards the country of the Heart, the country of Intelligence and the country of Initiative — so is one being unjust towards the Catholic Church when one sees, instead of the Mystical Body of Christ, only its historical phantom, the fox. In order to see rightly one has to look rightly. And to look rightly means to endeavour to see through the mists of the phantoms of things. 

\end{quotex}
Our age is fixated on egregores, at least in the West, so that only the worst aspects of a collective entity are noticed. Only by dissolving these fixations can one make spiritual progress, not just individually but collectively. Finally, Tomberg identifies the Anti-Christ as the egregore of the human race.

\begin{quotex}
The antichrist is the phantom of the whole of mankind, engendered through the whole historical evolution of humanity. He is the ``superman" who haunts the consciousness of all those who seek to elevate themselves through their own effort, without grace.

\end{quotex}
Tomberg singles out Nietzsche and Karl Marx as those who have fallen under the spell of the Anti-Christ.



\flrightit{Posted on 2021-07-25 by Cologero }

\begin{center}* * *\end{center}

\begin{footnotesize}\begin{sffamily}



\texttt{James on 2021-07-26 at 00:24 said: }

The clamouring for ``what is real" for many people , therefore , and what exists in the popular imagination is not a qualitatively different form of consciousness , but a quantitatively increased volume of lower consciousness .

I recently watched an episode of a comedy where a character is meant to `ascend' to a higher state of being and this ascension merely grants him the maddening and lovecraftian horror of being able to `see everything' and that the universe is balanced on the back of an animal (again , this was supposed to be comedy) . In other words , one popular conception of ascension to a higher state of consciousness is to see more of this lower world as it pertains to its physicality . It is not a union with some(one)thing higher but a union with the physical universe as it is and the maddening result of returning to such a mineral-plasmic state .

Instead of working towards waking from the dream , we fall into the foolishness of demanding more sensual dreams . It is no wonder , therefore , that the author of Meditations also speaks about how if respiration is not achieved vertically , death can feel like a ``sting" . We can feel that same analogous motion whenever we wake with a start from a ``truly wonderful" dream to waking reality . The soul-body can even suffer a form of spiritual sleep apnea .

Nonetheless , waking up in the drudgery of purgatory must be preferable to the most pleasant sleep since despite the pain of waking life , we at least have `access' to ourselves and our true(er) senses . We can build more and more authentic experiences and achieve a higher level of true agency . Just as the dreamer believes that there is no waking life , we continue on in this world believing that higher levels of consciousness either does not exist or , as most people now consider to be vogue , a form of highly pleasurable sense-experience . They're not wrong in that it's pleasurable , but they simply play around with gold chalices thinking it's ``shiny" rather than realizing it is a font of life . Even those who pursue the esoteric can be guilty of the same fetishism if not guarded carefully .

It is interesting , too , that one consequence of this analogy of memory-consciousness-life/forgetting-sleep-death is that one popular theory of dreams is that it is our own unconscious attempting to work through thoughts/emotions we neglected in the conscious life .

In some analogous way , perhaps life , then , is the phenomenon of our lower selves not having yet been reunited with our dreamer self . ``The dreamer must awaken" as Frank Herbert put it . The process of individuation by Jung is the ``below" to the ``above" of reintegration of the soul with its archetype in Purgatory — in Purgatory where Dante poetically described as still having ``day and night" and that all souls must rest from the climb up the mountain at night . Our lives , then , in some ultra-realistic way are the dreams of our souls in Purgatory — what has come `before' and `after' being more set in the vertical rather than temporal plane .

If we never awaken from the dream or refuse to wake and therefore continue this disconnect between our selves and our archetype — would this be our own choice to descend to Hell ?

The fact that the cross demonstrated that Death was necessary for salvation (otherwise the Cup would have passed from the Master) , we also recognize that sleep is necessary for the human physical body . Agnosis is also necessary in some ways in order to advance forward to true knowledge . There is an interesting avenue to explore vis a vis whether or not this life we lead here on Earth is necessary . Obviously the Incarnation is one example of how ``sleep is necessary" and that even ``God rested" . Can we say that ``God dreamed" ? I will take these callow thoughts to further meditation .


\end{sffamily}\end{footnotesize}
