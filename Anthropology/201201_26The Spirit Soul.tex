\section{The Spirit Soul}

\begin{quotex}
Everything belonging to the subtle state is very closely connected with the nature of life itself, which is inseparable from heat; and it may be recalled that on this point, as on many others, the conceptions of Aristotle are in complete agreement with those of the East. \flright{\textsc{Rene Guenon}, Man and his Becoming}

\end{quotex}
\textbf{Rene Guenon} often used \textbf{Aristotle} as the exemplar in the West of Indian metaphysics. Whether this is the result of a direct influence or from plumbing the same depths is open to debate. More likely, both the East and pagan Antiquity were drawing on a common source, viz., the primordial tradition of the original Indo-European peoples.

We recently outlined the Western tradition on the corporeal soul\footnote{\url{https://www.gornahoor.net/?p=3662}}. Although the soul is the form of the body, it also transcends that function. Guenon mentions that the corporeal state belongs to the human individual. Moreover, there is the subtle state of the individual. Guenon makes clear that the subtle state is not totally coincident with the corporeal state:

\begin{quotex}
The corporeal state [and the subtle state] are strictly and essentially states of the living man. This does not necessarily involve admitting that the subtle state comes to an end at the precise moment of bodily death and simply as a result thereof. 

\end{quotex}
The Western Tradition has an analogous teaching. In his book, \textit{Occult Phenomena}\footnote{\url{https://www.gornahoor.net/library/occultPhenomena.pdf}}, \textbf{Alois Wiesinger} describes this subtle state in some detail to explain the experience of preternatural and mystical phenomena. He gives the name spirit-soul to the subtle state to distinguish it from the corporeal soul; this corresponds to the Hermetic distinction between spirit, or \emph{pneuma}, and soul, or \emph{psyche}.

As interesting as Fr. Wiesinger's analysis of occult phenomena is, for our purposes we now restrict ourselves to his description of spirit. Here are some salient points, without further explanation.

\begin{itemize}
\item The existence of spirit is not a philosophical hypothesis. The knowledge of it comes from experience beyond the material. 
\item He points out that this doctrine was taught by the greatest philosophers of mankind, for thousands of years, hence predating Christianity. He points out that this doctrine is either unknown or ignored; the reader is free to offer his own explanation for this curious neglect. 
\item The activity of spirit proceeds by means of thought and will. The two basic faculties of the spirit are, then, the acts of understanding and the will. 
\item Unlike the corporeal soul, whose knowing is mediated through the senses, spirit knows directly. 
\item To recognize an object the spirit must have the thing within itself, that is, the form without its matter. Said differently, the spirit is conscious of the Idea, which it knows directly rather than through a process of abstracting. 
\item Pure spirits communicate through thought transference. This has consequences for the Occult War. Men believe their thoughts are their own, even though they cannot predict what they will think of one minute from now, and seldom consider the source of thoughts. But thoughts come from the hierarchies of higher beings and angels and demons. That is why ideas can take hold of large populations at the same time. It also explains why certain thoughts tend to lump together in different individuals. 
\end{itemize}
These qualities of spirit exist in man insofar as he is a spiritual being, but usually only virtually since his mind is most often split and concentrated on the body. Fr. Wiesinger criticizes the simplistic explanation that all spiritualist and occult phenomena are evil or attributable to the devil. To the contrary, he explains that these powers of the spirit-soul were part of the primordial state, but were then mostly lost after the Fall. While we cannot describe the powers of the spirit-soul and related phenomena at this time, we will conclude with a description of mystical states and what they reveal about the spirit-soul.

Fr. Wiesinger is a conventional priest and his area of concern is the right hand path, or the way of ecstasy\footnote{\url{https://www.meditationsonthetarot.com/solar-and-lunar-initiation}}. He first describes Mystical Sleep during which God brings leads the mystic to the Primordial State. It may be helpful to read this in conjunction with Chapter XI of \emph{Man and his Becoming} where Guenon discusses the states of sleep. He mentions the state of ecstatic trance, which is a state between deep sleep and death, in which the living soul, \emph{jivatma}, withdraws into the bosom of the Universal Spirit (\emph{Atma}).

Fr. Wiesinger points out that at the end of a culture, when the mystical sense has been lost, there arise numerous attempts to recover it. Most are misguided, but sometimes they are recovered by ``honest striving after a form of self-preparation that was far-seeing, scientific, and wholly in accord with nature." He specifically mentions \textbf{Buddha}, at the decline of Indian culture, \textbf{Plotinus} at the end of Greek culture, and \textbf{Rudolf Steiner} at the end of Western culture.

Analogous to the three layers of the corporeal soul, there are also three layers of the spirit-soul, corresponding to the Trinity. Once again, theology is in conformance with Hermetic, esoteric, and to an extent, Anthroposophical, teaching. We summarize Fr. Wiesinger in Table~\ref{tab:SubtleStatesWiesinger}.

\begin{table}[h]\centering\scriptsize
\begin{tabular}{ccc}\toprule
\textbf{Procession}  &
\textbf{Stage of Prayer} &
\\\midrule
Holy Spirit &
The prayer of quiet &
Imagination still retains its freedom.\\\midrule
Son &
The prayer of union &
Ecstasy\\\midrule
Father &
The prayer of spiritual marriage &
\\\bottomrule
\end{tabular}
\caption{Layers of the subtle states of the spirit-soul}
\label{tab:SubtleStatesWiesinger}
\end{table}


\flrightit{Posted on 2012-01-26 by Cologero }

\begin{center}* * *\end{center}

\begin{footnotesize}\begin{sffamily}



\texttt{Caleb Cooper on 2012-01-26 at 00:35 said: }

``Pure spirits communicate through thought transference."

One of the more frustrating things about going over to the other side is when I can tell an entity is communicating with me telepathically, but I have no idea what it's saying because I don't speak telepahthese! 

It gets really amusing when I realize my soul (spirit?) is communicate back, but I have absolutely no idea what I'm saying, it usually just sounds like a bunch of bird chirping. Need to take the wax out of my `ears of the heart' I guess.


\hfill

\texttt{Charlotte Cowell on 2012-01-26 at 05:23 said: }

I have also had this experience of communicating and knowing a two way conversation is going on but not `hearing' the words. It's more like there is a `question'. in the form of a `thought' and then as the answer is given it becomes an action before you even realise you've asked – this must be partly to do with the dissolution of time but also because one acts on pure will in that state, it's all instantaneous. Also impossible to lie. So for example once I was in a very deep guided meditation that took place in primarily an astral context. As I was being led through a particular building it occurred to me at a certain moment that there was no discernible decoration that I could use as a point of reference to recall later and try to look further into. As I was thinking this my guide `told' me to look down at the floor, whereupon I saw we were following a row of tiles that all had identical swastika patterns on them. As soon as this was noticed we moved into the next tableau. All of it happened without words, almost without the thought being formed, so while I DID know what the conversation was about, there were no words spoken, I just wondered what I needed to wonder and was given the answer I sought. At other times I've been in situations where passwords were needed but I didn't realise this until a word just popped into my head. It's very seldom indeed I become aware of actual sound in the higher astral or etheric dimension (which is a rare occurrence for me anyway), but if I do it is always a sublime experience but difficult to describe the quality of the sound. So it is not like an outside sound coming in, it is an inner sound that seems to come from the very centre of your head that in turn seems to contain the whole universe, it's not like a vague inner voice, it is super-clear. Once I was taken to the Happy Hunting grounds and I heard their drums, beating very rhythimically, right in the middle of my mind just as I was coming back and becoming conscious. This was shortly followed by a sight of their campfire, which was incredibly bright, phosphorescent, and led me to say it was like `magnesium burning'. Once I heard thousands of harps very loudly as I was waking up, which was truly amazing and I believe this to be the sound of the sun rising. Funnily enough a few days ago I was half in and out of sleep in the early hours of the morning (something woke me up), when I very distinctly heard the sound of running water in the same place, the inner kingdom. Immediately I knew it was a stream but it startled me as to hear things is very unusual for me. Just as I was nodding off again I was woken up once more, this time by the sounds of a choir singing Gloria, which was really wonderful but mystifying also. It didn't seem to last very long. I have also had experiences of talking to birds but not hearing any sounds. It seems to me that commuication is done telepathically, the will responds to questioning, but that sound is a very elevated sensation as far as spiritualilty goes, I would love to hear that music all the time, one can't describe the beauty and clarity of the tones….


\hfill

\texttt{Charlotte Cowell on 2012-01-26 at 05:58 said: }

Regarding the (very interesting) main article, these points here from Fr Wiesinger's book ring very true (in fact most of it makes a lot of sense): 

`The activity of spirit proceeds by means of thought and will’ – this is very much my experience, the will is so much stronger and more active as it is not hampered by phsycial restraints for one thing, so can function as the pure agent of action. `The two basic faculties of the spirit are, then, the acts of understanding and the will'. This seems to be a fair summary.

`spirit knows directly'. yes, it is this `knowing' I was trying to describe, as it is so clear and direct, there are no doubts whatsoever, things just are what they are, there isn't a sense of judgement, although there are times one can be made aware of dangers, things not to do and so on. 

Regarding these strange `conversations'. Cologero (who I've pestered about the subject several times!) will remember my endless descriptions of a visit to Atlantis, which I embarked upon following a suggestion and nothing more, I had very little curiosity with regards to it and treated it as a `necessary exercise' if that makes sense. In fact I could hardly be bothered at the time it seemed a chore. Nevertheless, I did as suggested and asked for angelic protection as I had been told to visit Atlantis. I did not expect anything whatsoever to happen after that but a moment later a VERY bizarre looking fish-headed being (I only saw his top half or top third) appeared very clearly in a semi-astral, etheric dimension in front of me. I was startled at the very strange appearance and also felt sheepish as he seemed really irritated. It all happened incredibly quickly and I had no time to think, but somehow he asked me why I wanted to go and before I knew what had happened he seemed to extract from me the details that led up to me asking angels about a visit to Atlantis – so I totally agree with the `transference' description, it just happens and you can't do anything about it, these spiritual beings just know – you couldn't lie and pretend to have good intentions if really you wanted to spy on a secret book or something, as your will is `read' by them. I don't think anything `bad' would necessarily happen, you just wouldn't get further.

So in my case the actual situation was relayed – I had a study partner who himself, it seemed, was curious to know about Atlantis, and expecting I might manage to get there he had therefore planted the idea in my head. I did exactly as he suggested and asked for protection first, and it would also be true to say that I fully expected at that time to always have angelic protection in any situation, so there wouldn't have been any doubt or fear in me (now there might, but that's another story!).

However where I ended up next is perhaps a warning note to us all, as I found myself at the very bottom of a very deep, murky green-grey ocean, hiding behind huge clumps of seaweed, staring absolutely boggle-eyed at the profile view of a giagantic being that had a man's torso and very, very long, corkscrew-curled tail. He was floating at a distaince of around 30 ft away, with both arms stretched out in front of him with elbows at right angles. His form was dark but clear, as if a solid shadow, and it is impossible for me to describe the terror this struck in me – those who know how our emotions are far more easy to govern in the astral state will hopefully understand just how frightening this therefore was. For a start I have an intense phobia about deep water having nearly drowned several times and been on some scary boat journeys, so that was bad enough, but secondly this creature was so beyond anything I've ever even thought of, let alone comprehended, and so huge, so generally weird in every way I just couldn't imagine what might happen next. My main fear – beyond the fact I was underwater with `time running out’ – was, if I can see him, what if HE can see me?! 

Things had just reached a crisis point when from out of nowhere this word just popped into my head, like Eureka – `CAPRICORN!'. I was elated, triumphant even, and in that moment I came back into my body feeling immensely relieved. I was there for a second more before going back into another scene (long story this so I won't relate it all here). 

Would I have embarked upon this journey if I'd known what was `down there’ – no, of course not! I'm rather ashamed to say that I'm not a particularly courageous person, although I am impulsive and can be rash at times. All the seemingly brave or exciting things I've done on earth have been kind of by accident or out of foolishness, and this Atlantis trip has given me cause to wonder, `should I have gone'. ever since. I guess the moral here is be careful what you ask for as you might get it! (the rest of the visit had a very different tone, by the way, and I subsequently identified the Fish-head being with Pisces).


\hfill


\end{sffamily}\end{footnotesize}
