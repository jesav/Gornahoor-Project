\section{Matter and Form: The Dionysian Race}

\begin{quotex}
The new victories of the maternal principle over the revelation of purely spiritual paternity show how hard it has been for men, at all times and amid the most varied religious constellations, to overcome the inertia of material nature and to achieve the highest calling, the sublimation of earthly existence to the purity of the divine father principle. \flright{\textsc{J J Bachofen}, \emph{Mother Right}}

\end{quotex}
Having traced the stages of history from the Telluric to the Demetrian to the Apollonian or solar periods, the latter peaking in the Roman Imperium, \textbf{J J Bachofen} then identifies the Dionysian period. \textbf{Julius Evola} calls this the Dionysian Race.

The Dionysian race represents the paternity of the Apollonian falling back into materiality. The Apollonian ideal was unable to withstand the assaults of baser doctrines. The Dionysian stage prepared the way for a new victory of the feminine principle and the mother cult. The victory of Dionysus over Apollo was assured in the meeting of the Greek and Oriental worlds under Alexander. Bachofen points out the spiritual principle of the Delphic Apollo was powerless to overcome the lower material views of the sexual relation. He writes:

\begin{quotex}
We see paternity falling back from Apollonian purity to Dionysian materiality, so preparing the way for a new victory of the feminine principle, for a new flowering of the mother cults. Although the intimate union which the two luminous powers concluded in Delphi seemed calculated to purify Dionysus' phallic exuberance through Apollo's immutable repose and clarity, and to lift it above itself, the consequence was the exact opposite: the greater sensuous appeal of the fecundating god outweighed his companion's more spiritual beauty, and increasingly usurped the power that should have been Apollo's. Instead of the Apollonian age, it was a Dionysian age that dawned. 

\end{quotex}
This shows that the spiritual aspect alone is insufficient. Rome, instead, was able to maintain the Apollonian through its structure. Bachofen explains:

\begin{quotex}
Mankind owes the enduring victory of paternity to the Roman political idea, which gave it a strict juridical form and consequently enabled it to develop in all spheres of existence; it made this principle the foundation of all life and safeguarded it against the decadence of religion, the corruption of manners, and a popular return to matriarchal views. Roman law maintained it traditional principle against all the assaults and threats of the Orient, against the spreading mother cult of Isis and Cybele, and even against the Dionysian mystery. 

\end{quotex}
Evola's final point seems to be directed against a Hans Gunther or his followers. Gunther, more than 80 years ago, was concerned about the ``degradation" of the Nordic peoples, which resulted from factors like the decline in marriage and birth rates, particularly among the higher ``more Nordic" classes. Obviously, that decline has continued in the ensuing decades. Evola attributed that decline to ``processes of involution" on the spiritual plane among the Nordic people.


\paragraph{The Dionysian Race}
In his treatment of the various degrees of virility and solarity as types in the order of the ancient Mediterranean mystery traditions, Bachofen appropriately distinguishes the Apollonian and the Dionysian stages. Here again cosmic analogies will serve as the basis. There are in fact two characteristics of solarity. One is that of light as such, i.e., as an unchanging and heavenly luminous nature: such is the Apollonian or Olympic symbol, for example, of the Delphic cult considered as an inspiration of pure Hyperborean spirituality, which reached to the Mediterranean. As we saw, that is the stage which indicates the race of the solar man. The other characteristic of solarity is that of a light that rises and sets, that experiences death and resurrection and a new death and a new dawn, in other words, a law of growth and transformation.

This is Dionysian solarity in contrast to the Apollonian principle. It is a virility that aspires to the light through a passion that cannot free itself from the sensual and telluric element and even from the ecstatic-orgiastic element typical of the lowest forms of the Demetrian cycle. The association, in myth and symbol, of feminine and lunar figures with Dionysius is rather significant in this respect. Dionysius does not accomplish the transition, the transformation of his nature. It is a still earth-bound virility in spite of its luminous and ecstatic nature. The fact that the Dionysian and Bacchic mysteries were associated with the Demetrian mystery, instead of with the purely Apollonian mystery, clearly indicates to us the end point of the Dionysian experience: it is a ``dying and becoming" in the sign not of the infinite which is above form and the finite, but of that infinite that is fulfilled and delights itself in the destruction of form and the finite, harking back therefore to the forms of Telluric-Demetrian promiscuity.

The Dionysian man, however, also has traits in common with the Titanic man. He is the one who aspires to reconquer the lost level that is capable of surpassing, in part, the human condition by means of a radical incitement of all the forces connected to the senses, but who nevertheless cannot surpass the ecstasies, where the virile quality vacillates and cannot conserve itself, where the sensible is mixed with the supersensible and, fundamentally, where liberation ensues only at the price of the decline of the affirmative principle of the personality therefore, with a solution in a way of being quite different from the solar and Olympian.

With the right transpositions of the level, it would not be arbitrary to establish a correlation between the Dionysian man and the romantic man. Together they belong to the same race of the spirit, defined as opposition to the Olympian or solar spirit. And such a connection can spare us from providing further characterological details because the reader will see in it whatever is necessary to identify them. From the race theoretic point of view, one must not be surprised at the thought that the Dionysian man, acting as the romantic, is represented to a large extent in the Nordic races, the Germanic as well as the Anglo-Saxon. So this confirms the previously indicated point to clearly distinguish the primordial Nordic-Aryan race from the Nordic races of more recent times. Even in the latter the part that had the feminine, Demetrian, and gynocratic element in their appearance at the entrance of historic times is significant enough (even today, the German languages is unique among the Indo-European, or Aryan, languages in which the sun—die Sonne—is of the feminine gender while the moon—der Mond—is of the masculine). This leads us to think that, in this respect, it is a completely different question of imitators who ``line up" and do not even pause to consider the simple race of the body: on the spiritual planes, certain processes of involution seem to have happened among the most recent Nordic peoples in no less degree than among the West Atlantic or Nordic-Atlantic Aryans, in the Mediterranean traces of which many divergent forms of the pure solar tradition are likewise found.



\flrightit{Posted on 2015-06-19 by Aeneas }

\begin{center}* * *\end{center}

\begin{footnotesize}\begin{sffamily}



\texttt{Logres on 2015-06-22 at 20:29 said: }

That last paragraph clarifies a whole two decades of my struggle to understand the Classical world: thank you. If I am remembering rightly, Evola dates the decline of the classical world from the 6-8th centuries BC, in Revolt Against the Modern World. Around the genesis of Rome, who was then the counterweight.


\end{sffamily}\end{footnotesize}
