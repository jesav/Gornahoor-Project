\section{Entheogens}

The use of mind-altering drugs has been associated with various mystical, magical, and shamanic rites. This is very appealing to the modern mind which is impressed by technological and materialistic explanations. Entheogens, or drug use for allegedly spiritual purposes, began to be widespread in the 1960s following the discovery of the psychedelic drug LSD and the ensuing publicity. The early adopters can be easily found, but we will focus on \textbf{Timothy Leary}, an erstwhile professor of psychology at Harvard University. He initially ran experiments with the drug as a form of therapy, but eventually began experimenting on himself and the group of acolytes who assembled around him.

Leary believed that psychedelics could open up the mind to greater spiritual experiences and encourage their use for explorers of the mind. The stars were aligned: the books of \textbf{Carlos Castaneda} came out shortly thereafter. Filled with wild tales of a Mexican shaman whose knowledge of plant-based drugs led to amazing powers, the books became the spiritual nourishment for many. The Beats became users as well as high profile entertainers. Even \textbf{Alan Watts}, after allegedly years of Eastern practices, eventually resorted to LSD experiments to learn about the ``mystical experience". For anyone who has spent hours at a Zen center walking in circles while chanting Buddhist texts, or sitting still in Zazen, the idea of an instant pill could sound truly appealing.

At any given moment we are bombarded with external sensations, a ``blooming buzzing confusion" (\textbf{William James}), from which we choose a world. Simultaneously, although few pay attention to it, we are the receivers of thoughts of all types from various sources, or levels of reality. Normally, they dissipate or recede into the memory. Sometimes, the thoughts are powerful enough to come into our attention, then related thoughts latch onto each other, often producing a strong sensation. This may be as simple as reviewing plans for the day.

But the most powerful arise from the internal forces of eros and thymos, which, at the most primeval, are experienced as sex and violence. Hence, a sexual fantasy will totally engage our minds for an extended period, even to the point of affecting the body as if an actual sexual encounter were taking place. Fantasies arising from thymos usually take the form of domination, anger, revenge, and the like. Hence, we envision ourselves as wildly successful in some realm. Or we may recall a past slight, which even agitates the body. We will envision the cutting phrase we should have used against someone, but didn't. It goes on and on.

Most people, unfortunately, cannot shut these thought threads down. They may cause continuing anxiety and self-doubt. The neutralizing force, the nous, whose task is to dominate and channel the forces of eros and thymos, is too weak, or, in truth, is not even known. The nous must transcend these forces, fantasies, and thoughts, regarding them as arbitrary, contingent, and external to one's true self. Instead, people regard these emotions, fantasies, and thoughts as their ``own", even though they are totally unaware of their true source.

Psychedelics work by slowing down the awareness of these impinging thoughts. Thus, a particular thought—that may be pushed aside by a stronger thought in the ordinary state of consciousness—instead can take hold in consciousness. Then, related thoughts can follow along in sequence producing an extended vision. This is called a ``rush", and is the pleasurable sensation associated with psychedelics. For artist and mystic types, these are regards as deep insights or creative inspirations. However, I have also seen those on a ``bad trip", during which the user experiences inconsolable anxiety, requiring an antidote such as thorazine.

\begin{quotex}
It should be clear that the theory behind the use of enthogens for spiritual ``enlightenment" is deeply flawed. It assumes that such enlightenment involves a particular experience, or set of experiences, that are somehow to be distinguished from all other experiences. This idea comes from the confusion of the psychic and the spiritual. \flright{\textsc{Rene Guenon}, \emph{The Reign of Quantity}}

\end{quotex}
An intellectual conversion, the only thing that enlightenment can mean, involves the nous. The nous transcends the psychic, it transcends every experience whatsoever, whether a brilliant insight or a bad trip. Instead of moving from rush to rush, the mind becomes clear, thoughts separate from each other, are rarer, and seem heavier, or else are wispier, evaporating as soon as they appear. Such a man is free; by definition, then, it cannot depend on some biochemical agent.



\flrightit{Posted on 2012-06-20 by Cologero }

\begin{center}* * *\end{center}

\begin{footnotesize}\begin{sffamily}



\texttt{Dominion on 2012-06-21 at 01:42 said: }

``Instead of moving from rush to rush, the mind becomes clear, thoughts separate from each other, are rarer, and seem heavier, or else are wispier, evaporating as soon as they appear."

This sounds like an experience or change that the nous is undergoing, as far as I can tell. Now, that the brain, which creates the ego, is material is a self-obvious thing. However, since the brain and the ego it creates are physical, they are part of a metaphysical reality. Overcoming the ego, which in transient, and identifying with the metaphysical reality which is eternal, is how the true Self is discovered. Evola, in Revolt, uses the image of Krsna and Arjuna riding in the chariot in the Bhagavad-Gita; Arjuna represents the ego, involved in the world, doing battle, its vision obscured by ignorance and transience. Krsna, of course, is the true Self, who holds the reign of the chariot, who is aware and indeed a manifestation of the Supreme Reality, Brahman (through Vishnu). 

Though I have not taken DMT or other entheogenic drugs, I understand that they (especially DMT, which was used by many shamanic tribes) have the biochemical effect of breaking down the ego that the brain has created. They are thus material substances having a material effect on a material brain which creates the ego. When this ego is broken down, reports I have read claim that the resulting effect is one where you appear to see reality in a new way and comprehend its nature in a way which the ego prevented one from doing so before. Now although this effect passes, I see no reason why it could not be a valuable aid to the mind in understanding intellectually the transcendent metaphysical reality of which the physical is but a part.


\hfill

\texttt{logres on 2012-06-21 at 10:40 said: }

A mind altering drug would only be useful to someone who didn't need it or else everyone would become a shaman.


\hfill

\texttt{Cologero on 2012-06-21 at 12:12 said: }

That's the democratic way, Logres! That is why a few weeks ago I pointed out the essential egalitarian nature of drug and alcohol use, although it was met with incomprehensible opposition.


\hfill


\hfill

\texttt{Nonamynous on 2012-06-22 at 06:18 said: }

This article is a superb explication of drug use vis-à-vis the elements of intelligence.

It must, however, be faulted for not explicitly differentiating, in its criticism, between modern Western drug `experimentation'. and the use of psychoactive substances within the strict context of traditional ritual. The term `entheogen'. despite being an awkward neologism, is more applicable to the latter case; in the former, there is at best only an aspiration toward `entheogeny'.

Nothing can really be said against Dominion's statement here, provided we accept the caveat that it only applies to the actual or potential `l'uomo differenziato' with some preparatory theoretical knowledge: ``I see no reason why it could not be a valuable aid to the mind in understanding intellectually the transcendent metaphysical reality of which the physical is but a part." Yet this understanding will never be imposed by the effects of a drug, where transient (and basically illusory) insights are lost as regular consciousness is reinstated. Drugs are dissolutive agents; coagulation is the positive phase of the work.

Drug use is not entirely a `democratic way'. because the effects ultimately differ from one person to the next. But we understand what you mean.


\hfill

\texttt{Jupiter on 2012-06-23 at 08:17 said: }

Unfortunately the simple fact is that the philosophical and spiritual oppositions the Traditionalist school, and thinkers influenced by them, have to the usage of mind-altering substances are entirely incongruous with the basic historical reality of the matter.


\hfill

\texttt{Cologero on 2012-06-23 at 10:12 said: }

Best wishes on your spiritual quest, Jupiter. There are some exotic bath salts available in Miami, and I've heard rumours of spiritual enlightenment becoming a mass phenomenon there. There may even be cults like the Aghori\footnote{\url{http://www.gornahoor.net/?p=3161}}. Remind me to send you a link!


\hfill

\texttt{Avery on 2012-06-23 at 22:27 said: }

Not just Miami — they're causing life-changing mystical experiences in California\footnote{\url{http://www.dailymail.co.uk/news/article-2163641/Robert-William-White-bashes-elderly-womans-head-shovel-high-bath-salts.html}}, and transforming the way we see art criticism in Denver\footnote{\url{http://www.nydailynews.com/news/woman-claims-bath-salts-fueled-destruction-30-million-painting-article-1.1100993}}. Yes, bath salts are the preferred method of experiencing modernity.


\hfill

\texttt{Jupiter on 2012-06-24 at 01:17 said: }

Nice strawmen, fellows.

You explain the frequency of cannabis and other psychoactive herbs in archaeological digs associated with Indo-European, Indo-Iranian, and Indo-Aryan ethno-linguistic groups, if such things are so entirely outside the realm of orthodoxy, not to mention the frequent references in Vedic and Tantric texts.


\hfill

\texttt{Cologero on 2012-06-24 at 09:40 said: }

Not a straw man, Jupiter, you made a claim about the usage of ``mind altering substances", and now you add in archaeological digs. What we pointed out is that the usage of mind altering substances is with us today and we don't need any ``digs" to prove it. What these examples do show is that drug use in itself is insufficient to lead to any sort of understanding of ``orthodoxy".

So, what we can't see, and you have provided no justification yourself, is the alleged connection with ``orthodoxy". As for orthodoxy, we can point out the ultimate goal as Guenon describes it:

\begin{quotex}
the summit of the initiatic hierarchy [is] the Supreme Identity, the absolutely permanent and unconditioned state beyond the limitations of all contingent and transitory existence

\end{quotex}
Clearly, this is not a description of an intoxicated state, which is necessarily conditioned by the ``psychoactive herb". So perhaps you mean they can be an ``aid" at some stage of the initiatic path. Perhaps you have this statement from Guenon in mind:

\begin{quotex}
The diversity of methods corresponds to the very diversity of individual natures for which they were made; there is a multiplicity of ways all leading to a unique goal.

\end{quotex}
The temptation, then, may be that psychoactive drugs may be appropriate for some ``individual natures". However, ``orthodoxy" requires an active intellect, whereas the drug experience is entirely passive. Hence, it can have no place in any ``method". It can never be any more than a human, all-too-human, means, something exterior rather than interior. Guenon makes this all too clear:

\begin{quotex}
degrees [of initiation] … are not achieved by the outer and human means … but exclusively as a result of a completely interior work.

\end{quotex}
Although I have no interest in your personal life, Jupiter, if you came here for some sort of intellectual justification for your personal peccadilloes, you will be disappointed. The alleged finds of archaeology prove nothing about Tradition; only an understanding of principles can do so. Regretfully, I will have to put you under moderation unless you show you can grasp that distinction.


\hfill

\texttt{Paulo on 2014-09-02 at 22:15 said: }

Entheogens have made a truly miracle in my life,they help me to find my will.It is funny,but it seams that there is really some entitie in them.The awarenes that you have during the experience is somenthig that goes on and on.Every gain that you have stays with you,so there is no"coming back"from the trip.The drawbacks are in most cases caused by the person itself,that cannot deal with some facets of himself.The sad thing is that the majority of people does not have the inner drive to transcendence,so the use of those substances will not help them too much!


\hfill

\texttt{F on 2019-06-20 at 02:56 said: }

I think the right drugs can definitely act as a shortcut, but they are extremely dangerous at that.

Like anything from the LHP it's a double edged sword.

When you take a shortcut and skip the work that morally prepares you it just ends up turning on you and end up with more than you gambled for.

And you don't want to gamble with this.


\hfill

\texttt{Michael on 2020-06-20 at 18:32 said: }

``Instead of moving from rush to rush, the mind becomes clear, thoughts separate from each other, are rarer, and seem heavier, or else are wispier, evaporating as soon as they appear."

On these types of thoughts that seem heavier, is it like a voice that is so quick and immediate yet seems like not of one's own individual mind, generally answering a question or directing a command that does not conflict with metaphysics?

For the wispier, are these like remembering in a dream? That sensation of words which you are barely holding onto, tightening a grip will let them loose but allowing them to come and go allows for a type of current?

Thanks!


\end{sffamily}\end{footnotesize}
