\section{Psychic Distortion}

\begin{quotex}
Living in the consciousness soul, man experiences isolation, loneliness, materialism, loss of faith in the spiritual world, above all, uncertainty. The soul has to make up its mind and to act in a positive way on its own unsupported initiative. And it finds great difficulty in doing so. For it is too much in the dark to be able to see any clear reason why it should, and it no longer feels the old (instinctive) promptings of the spirit within. \flright{\textsc{Owen Barfield}, \emph{Romanticism Comes of Age}}

We are now the sons of God; and it hath not yet appeared what we shall be. We know that, when he shall appear, we shall be like to him: because we shall see him as he is. \flright{\textsc{1 John 3:2}}

\end{quotex}
\paragraph{Types of People}
Plato identified three forces in the soul: eros, thumos, and nous. These correspond to the willing, feeling, and thinking functions of the soul, each of which has its own relatively independent centre. The once-born will be centred in one of the three centres:

\begin{enumerate}
\item Centre of gravity is sensations and movement 
\item Centre of gravity is the feeling function 
\item Centre of gravity is the thinking function 
\end{enumerate}
\paragraph{The Psyche}
Psyche, for our purposes, refers to the soul life and all its functions: thinking, feeling, willing. The Real I, or Self, transcends the psyche. A result of the Fall is that one's self-identification fell from the Self to the I of the Personality, or Ego. The Ego is the conscious part of the psyche, but it accounts for just a small part of the psyche, most of which is unconscious.

Self-observation reveals that the Ego is not unified, since there are multiple, potentially hundreds of I's, each claiming to be the Ego. Some of them may work together, creating a psychological ``complex" which takes on a life of its own.

The Ego falsely believes it is one, because of

\begin{itemize}
\item the unity of the body and 
\item a common name. 
\end{itemize}
In extreme cases, like multiple personality disorder, the different complexes even take on their own names. The task is to integrate the various parts of the personality and move the centre of gravity up from the Ego to the Self. Integration means to make whole, which means to make holy. Thus, it is a spiritual task, not merely a psychological process. That is probably why \textbf{Carl Jung} could claim that most of his patients were non-religious.

\paragraph{Resistance of the Ego}
The Ego resists this process, which it regards as death. In a way it is, since the Real I then takes its rightful place, replacing the Ego. You often hear people say something like, ``I need to contact my higher Self." Of course, the Self is never an object, but always a subject. A claim like that shows the Ego's resistance to the Self. In the The Parable of the Coach, the Ego is like the coachman who refuses to take the orders of the Master in the cabin.

\paragraph{Image and Likeness}
We are born in the image of God, although the likeness has been lost or severely distorted. The image is beyond words and thoughts, so that our awareness of it is a mystical experience.

The likeness is the image as reflected in the soul or psyche. There are three stages:

\begin{itemize}
\item \textbf{Gnosis}: Gives form to the mystical experience 
\item \textbf{Magic}: Leads to action 
\item \textbf{Philosophy}: Makes the ideas communicable 
\end{itemize}
The psyche is perturbed by negative emotions, obsessive thoughts, and insistent desires. Moreover, much of it is subconscious and not readily available to the conscious mind. That is why the likeness of God gets distorted in the psyche. Hence, it is necessary to purify the psyche so that it reflects God's likeness perfectly. We will briefly examine the distortions in the feeling and willing functions here; the intellectual function will be discussed later.

\paragraph{Existence of Demons}
As was mentioned in the previous post on Demonic Possession\footnote{\url{https://gornahoor.net/?p=10454}}, the feeling and willing functions are under the influence of demons. We are not referring to any Hollywood-style depictions. Rather, the demons are known by their effects, which can be described. This is actually an important teaching. That is because it shows that our psychic functions are not intrinsically evil, but are under the heavy influence of outside forces. This was recognized in the baptismal rite prior to 1968, which used to include an exorcism. In particular:

\begin{quotex}
In the traditional Roman baptismal liturgy, we find a sequence of exorcisms that directly represent baptism's role as releasing us from the devil's possession. 

\end{quotex}
See \textit{The Significance of the Exorcisms before Baptism}\footnote{\url{https://rorate-caeli.blogspot.com/2018/11/thomas-pink-on-significance-of.html}} for more details. Unfortunately, the exorcism was subsequently removed, so anyone baptized after that year may experience more problems.

So this is hardly a novel teaching and is verified by experienced exorcists.

There is a purpose to demonic possession: to provide the friction required to become more virtuous or conscious. Although they are part of God's plan, that does not make them somehow ``good", as some false teachings assert.

I will use the terms Luciferic and Ahrimanic to refer to the demons affecting the emotional centre (astral body) and willing function (etheric body) respectively. Keep in mind \textbf{Valentin Tomberg}'s reservations about the cavalier use of those terms in certain anthro circles to account for all sorts of unrelated phenomena. Nevertheless, they are useful.

\paragraph{Disordered Emotions}
In sound functioning, the Intellect will determine what is good, just, noble, moral, rational, prudent, or loving. The Emotions and the Will then follows the intellect. This is not the case with psychic distortions. A person finds himself subject to the capriciousness of passions, particularly of negative emotions.

Work must be done on three levels:

\begin{itemize}
\item willpower, 
\item emotional life 
\item thought control 
\end{itemize}
These must be ordered, educated, and shaped. By self-observation, the contradictions in one's psychic life become visible; in particular, one must learn to resist the outward expression of negative emotions.

\paragraph{Disordered Will}
The will is a force that manifests differently in the different levels:

\begin{itemize}
\item \textbf{Mineral}: electromagnetic forces 
\item \textbf{Vegetable}: Tropism \footnote{\url{https://www.gornahoor.net/?p=8349}}
\item \textbf{Animal}: Desire 
\item \textbf{Human}: Will 
\end{itemize}
In a disordered will, a being is motivated by desires, which then distort the thinking function to justify the desire and then to determine a way to act on it. On a more primal level, one simply follows impulses such as like/dislike, attraction/aversion, and so on, quite apart from any intelligence.

The human will is still a gnomic will, that is, it depends on deliberation. This leaves the will open to doubt, viz., it becomes double-minded, hearing the second voice. This often leads to confusion and distress rather than to liberation. This is the fate of the man who is awakening to his Real I. He then finds himself alone spiritually, without traditional support. Refer to the Barfield quote on the consciousness soul in the epigraph.

The True Will, on the other hand, means that a person is acting according to his real, unfallen nature. This is the purified will, because it will only one thing. In this case, what one believed through custom, habit, conditioning, and so on, no longer suffices. Such a person chooses his beliefs because he wills them.

\paragraph{Imagery}
The demon acts on the Will through imagery. However, it cannot manufacture images, but has to rely on pre-existing images. That is why it is so important to monitor the images the one pays heed to. Our contemporary world is full of manufactured images, designed to manipulate and control people. That is why the use of TV, movies, and related media should be reduced if not eliminated.

Although some may think this topic is off-colour, I am convinced that in this day and age, specific examples are necessary. Therefore, we can provide a study of how the use of imagery works in distorting the Will.

Men and women are typically dominated by one main feature. A very common feature is Lust, since it is so tightly tied to pleasure. Take the case of Jim, for example, who is civilly divorced, although still married, morally. Having succumbed to temptation, he goes to the confessional. He starts by confessing erotic thoughts. The priest asks him if he willingly entertained them. That is because the spontaneous arising of such thoughts is not ipso facto sinful. This is completely unlike secular psychology which presumes that such thoughts are indicative of one's deepest nature. Not at all: our deeper nature is to be in the image and likeness of God, i.e., centred in our Real I or Self.

Therefore, we attribute the initial image to the Ahrimanic demon, not to some repressed desire within. The priest then asks Jim if he looks at pornography. That is because the demon cannot create the imagery, but only rearrange what is already there in one's mind. Hence, the priest is looking for the real source in external causes. Jim denies watching pornography, but explains that he has had many girlfriends, so that the memory of sexual activity with them provides ample imagery. Thus, the demon can use Jim's own life experience against him.

This is actually good news, since our faults can be forgiven. By understanding how these outer forces act on our inner psychic life, we can learn to be liberated from such forces.

\paragraph{Postmortem Purification}
This should make it clear why postmortem purification is usually necessary. Death does not by itself remove the psychic distortions, making the divine likeness impossible without further work. Then, in the Beatific Vision, one knows God as he is, in his essence, not just in his energies.



\flrightit{Posted on 2018-11-25 by Cologero }

\begin{center}* * *\end{center}

\begin{footnotesize}\begin{sffamily}



\texttt{jonh oliver on 2018-11-26 at 15:07 said: }

I have a doubt regarding the nature of the true self, according to the conception given by Tomberg, and more specifically to the distinction that he makes of it from the vedic conception. The eastern conception affirms that atman=brahman, and thus, that by knowing our true self, we know god or the absolute. Tomberg goes against this, by stating that there are other transcendental immortal selves, that are higher than our own, that of the celestial hierarchies and the holy trinity, and that knowing them constitutes a higher level of spiritual experience.

My question is about the status of all of these different immortal selves, and whether are they really distinct, or are they just like the different qualities of god, that only have an analogical difference between them, but in the end being fundamentally the same. Everything at the highest level is indistinct in god, on what tomberg would call the world of emanation, but from a logical standpoint, i can't grasp how can there be more than one transcendental immortal and fully free self in existence, so as to allow an distinction between them. Because Tomberg affirms that in knowing our true self, we are only knowing the spiritual microcosm and not the macrocosm, which implies that these beings are separated from the human individuality. If i'm not misrepresenting what he's saying in some way, could you explain how this distinction is possible.

I know that this goes off on a bit of a tangent from the post, but i still think it's somewhat related.


\hfill

\texttt{Jack on 2018-11-26 at 21:53 said: }

Thank you for this excellent post. The last line seems to reference, and contradict, the Eastern Orthodox doctrine of the distinction between the essence and energies of God; that we can know and participate in the latter, but not the former. Could you refer me to a good Catholic critique of this doctrine?


\hfill

\texttt{Xavier Galindo on 2018-11-27 at 09:34 said: }

Jack: There is a disagreement between the EO's and the RC on whether God's ``essence can be known" –more specifically between essence-energies and Augustinian-Thomistic divine simplicity. Good sources to read about this are `Ground of Union' by Williams (very irenic), Aristotle East \& West by Bradshaw (informative but polemic, blames Augustine a lot), and (Byzantine Catholic) Fr. Klappes works (available on Academia.edu). There's a part of the Summa (in the treatise on the Holy Trinity) where St. Thomas explicitly (but respectfully) says St. John Damascene was wrong about energeia, and the EO naturally don't concede this.

Truthfully there could be a whole new Council of Florence devoted to this issue alone.

John Oliver: I wonder how your question relates to the metaphysical principle that identifies knowledge with being?


\hfill

\texttt{Cologero on 2018-12-08 at 08:28 said: }

In the future, Jonh, should you post another comment, please quote Tomberg (or whomever) directly; in other words, it is not helpful to tell me what you think that Tomberg said.

First of all, the vedas make reference to a multitude of Devas or immortal selves, so your first point is unclear.

Perhaps from a logical point of view, separate selves may not make sense. For example, F H Bradley in Appearance and Reality logically reaches the Absolute, which alone is ``real". Other selves, therefore, cannot ``really" exist.

However, the Hermetic teaching is that God made a sacrifice, or withdrawal, thereby creating a void into which created things, including other selves, could reside.

Tomberg writes about two substances and one essence … ultimately only God is ``essence" or Being, and substances are part of the world of becoming. Nevertheless, when we know our True Self most deeply, we discover that we also know God. This is not a matter of logic, but rather of gnosis or direct experience.


\hfill

\texttt{Cologero on 2018-12-14 at 22:47 said: }

@Jack, there is no intent to be polemical.

The ``energies" is rendered as ``acts" in the west, since \emph{energeia} in Greek is translated as \emph{actus} in Latin. So we do know God, in ordinary circumstances, through his acts (e.g., actual grace). However, there can be no real distinction between God's essence and existence (or acts).

A critique is not a valid starting point, but rather direct knowledge or gnosis is the only guide. For starters, I would recommend \emph{Christian Gnosis} by \textbf{Wolfgang Smith}, in which he writes:

\begin{quotex}
Gnosis in the ultimate sense, if attainable in this life at all, will be the lot, here below, of exceedingly few.

The possibility of gnosis in this life (understood as knowing ``\emph{the essence of God}``) has indeed been recognized by the Church. St. Thomas Aquinas, for example, has this to say:

``As God works miracles in corporeal things, so also he does supernatural wonders above the natural order, raising the minds of some living in the flesh beyond the use of sens, even up to the vision of his own essence; as Augustine says of Moses, the teacher of the Jews, and of St. Paul, the teacher of the gentiles."

And again, referring to eh same exalted state of gnosis: ``It is not incredible that this sublime revelation is vouchsafed certain saints without their departing this life so completely as to leave nothing but a corpse for burial." 

\end{quotex}
We remind you of the epigraph:

We are now the sons of God; and it hath not yet appeared what we shall be. We know that, when he shall appear, we shall be like to him: because we shall see him as he is. \flright{\textsc{1 John 3:2}}


\hfill

\texttt{Vimana on 2019-06-05 at 12:17 said: }

``In sound functioning, the Intellect will determine what is good, just, noble, moral, rational, prudent, or loving. The Emotions and the Will then follows the intellect. This is not the case with psychic distortions. A person finds himself subject to the capriciousness of passions, particularly of negative emotions."

It seems, then, that Schopenhauer was right that the vast majority of people have a blind and impulsive Will, but he was mistaken in his promotion of complete rejection of the Will rather than aligning it with the Intellect?


\end{sffamily}\end{footnotesize}
