\section{The Creative Process}

Training in esoteric psychology not only enables us to understand our own inner natures, but also provides a key to understanding much about how the world works. In the books on \emph{Gnosis}, \textbf{Boris Mouravieff} expresses traditional psychology in terms suitable for our contemporaries. So instead of speaking of the soul elements – viz., the vegetative, animal, and intellectual souls – he refers to various centers.

Animal life, in this scheme, has sexual, motor, and emotional (at least in higher animals) centers. Man has an intellectual center, actually two of them, a lower center for rational thinking and a higher center for intuitive thinking. In the Primordial State, represented by Eden, Adam had the lower centers, but his consciousness was dominated by the higher intellectual center, i.e., direct intuition of the spiritual realm. Nevertheless, as part of his creative activity in the world, the lower intellectual center, or critical spirit, was necessary to formulate aims and to determine the means to achieve them. This formation of the critical mind is, according to Mouravieff, the \emph{Tree of the Knowledge of Good and Evil}. As the critical spirit becomes dominant, the intuitive center or divine spark recedes and is generally forgotten.

Now animal and plant life lacks the intellectual centers, hence they do not have any inner conflicts that arise from the over-development of the critical spirit. This is the attraction many people feel toward animals. Moreover, there are systems of thought, most famously that of Rousseau, that seek that animal state in some primeval ``state of nature". Of course, that is not possible, nor is it really a return to the Primordial State. That path is not a rejection of the critical spirit from below as a return to animal nature, but rather the transcendence of that spirit from above. The esoteric path seeks the regeneration of man through conscious efforts to become aware of the divine spark.

To illustrate esoteric psychology, I will use Mouravieff's description of how the creative process functions. There are three stages from the idea, the thought, and the will.

\begin{quotex}
The joint action of the creative energy of the sexual centre and the faculties of intuition and discernment of the lower intellectual centre caused imagination to arise in man. After this, his development, the fruit of his conscious efforts, takes on an epicyclic form:

\end{quotex}
\begin{enumerate}
\item \begin{quotex}
Man explores the unknown: this operation, fruit of this creative imagination, is characteristic of every project; 
\end{quotex}
\item \begin{quotex}
Then, by the return path his ideas become concrete, he accumulates the necessary data to establish a plan of action and assembles the elements needed to put it into execution; 
\end{quotex}
\item \begin{quotex}
Lastly, thus enriched, he acts on this plan. 
\end{quotex}
\end{enumerate}
\begin{quotex}
This is the scheme of every human enterprise which puts into play all the factors in this activity.

\end{quotex}
The first phase involves the action of sexual energy, or eros. It causes phantasms, or images unrelated to the external world, to spontaneously arise in consciousness. When it is sublimated by means of its interaction with the higher centers, there will be creative activity. Otherwise, left on its own, the eros will simply produce sexual phantasies of the type that \textbf{John of Ruysbroeck} warned us against. This is not the place to discuss the practice of ``concentration" in which the formation of thoughts and images are consciously directed rather than spontaneously arisen.

If the higher powers are weak, then the power of the eros will be dissipated in sexual imagery. Moreover, \emph{pace} the Rousseauists, the state of nature is not necessarily pure and healthy. For example, \textbf{Sigmund Freud} last century made a career out of documenting the unhealthiness of the unchecked libido. Although Freud distorted traditional psychology and was completely unaware of the higher centers, he recognized the decadence of modern man. Keep in mind that the ``id" represents the lower centers, and the lower intellectual center he called the ``reality principle". Unfortunately, his system was unable to point the way to a cure.

Step (2) is not always so straight-forward. Of course, ideally the critical spirit should be able to judge rationally the ideas of the mental imagery. The spirit would determine if the plan is just and reasonable. It would then formulate the means to realize the plan. But at this point it is helpful to recall some of \textbf{Julius Evola}'s ideas. Of course, he agrees about new representations of the world spontaneously arising in consciousness.

Beyond that, Evola recognizes three stages:

\begin{enumerate}
\item \textbf{Spontaneity}. The self is not fully developed. Evola describes it as ``immersed in an immediate, indistinct coalescence with nature and the world, we can say that it is not so much he who thinks, speaks, and asserts himself, but rather that various forces and impulses think, speak and assert themselves in him." 
\item \textbf{Autonomy}. The self begins to separate from the world. Evola's description is much like Mouravieff's description of the lower intellect: ``Consequently, what used to be familiar to the individual is made alien and impenetrable, what intuitive certainty used to reveal to him as indisputable fact is made dubious and problematic."

In the attempt to discover some unity in the random fluctuations of phenomena, the self looks for solutions in adhering to some ideology, typically in the form of science or religious doctrines. 
\item \textbf{Mastery}. By transcending phenomena and ideologies, the self gains the power of mastery and becomes a ``deep centre of will and power". 
\end{enumerate}
\paragraph{Primal Traditions}
As an example of the first stage, we can rely on \textbf{Julian Jaynes}'s study on the \emph{Origin of Consciousness in the Breakdown of the Bicameral Mind}. In those early civilizations, there was a priesthood that still had an active and direct connection to transcendent reality. There was also a caste of aristocrats who knew how to wield political power, while maintaining the societal order determined by the spiritual authorities. The workers were still at stage one, living spontaneously. They were sub-rational, with a poorly developed lower intellectual center, and responded spontaneously to the images arising in consciousness. They were not plagued with the doubts that arise from the uncertainty of critical thinking. Hence, they primary mode of control and communication required the use of commands, poetry, songs, or other art. That is obvious from the perusal of ancient spiritual texts. Now Jaynes claims that those people were content in those roles, and that is probably true. Such civilizations were able to persist for extended periods.

\paragraph{Ideologies}
Step (2) of the creative process can be distorted or even thwarted by the adherence to ideologies that are at best partially true, if not completely false. That is Evola's second stage. When attempting to solve a difficult societal problem, ideologues will be blinded by their own worldviews. Since a worldview forms the foundation for their being in the world, there is a strong psychological need to adhere to it; the alternative is to fall back into the uncertainty of critical thinking and fluctuating phenomena.

Hence, the human race is prevented from achieving its creative potential. People get stuck at step (2) by debating worldviews without any possible resolutions. They become convinced that their own worldview is correct, so that it would be necessary to convince everyone else to adopt it. That is simply not possible because men have different capabilities and are of various spiritual races which prevent a common way of relating to the world.

\paragraph{The Degeneration of Castes}
Since the question of pseudo-traditions came up in the comments recently, this is a good moment to look at the question from the eternal perspective of Tradition. There is still this idea in certain circles that claim to be ``traditional" that the goal is to formulate the perfect ideology. Hence, they usually blame ``Christianity" for replacing the pure spontaneity of animal life with a critical attitude, especially its moral judgments. Since the identity of the self still resides in the lower centers or chakras, they come to experience the Christian teachings as ``alien" and therefore resent its intrusion onto the spontaneity of the self-expression of the lower instincts or desires.

As we see, however, even Evola, who was no Christian, recognized that stage (2) is a necessary element in spiritual maturity. However, since the spirit is free, there is no compulsion or mechanical process to traverse the three stages. The Christian teaching therefore is the ``way" to promote the movement into stage (3) or theosis. At that stage, there is a second spontaneity as one lives in the intuitive awareness of the presence of God. Now, someone can try to make the case that Christendom has lost its power for that. Their burden, in that case, is to come up with a living and viable alternative, not to reject the teaching in toto.

Back to the point. According to \textbf{Rene Guenon} and Evola, the decline in the world process results from the degeneration of castes, not from the adoption of specific ideologies. In that decline, the spiritual authority and political power proper to the higher castes are transferred to lower castes who are incapable of maintaining the traditional structures. It would be impossible to convince them to adopt a different ideology that would deprive them of power; that is why logical arguments are beside the point in political discourse. It is a democratic illusion to believe that a single ideology can be grasped and adopted by the entire population. If a spiritual elite arises with demonstrable powers, then circumstances may change, otherwise people are satisfied with what they have.

\flrightit{Posted on 2016-01-21 by Cologero }
