\section{The German New Thought Movement}

\begin{quotationx}
This essay was originally published by \textbf{Julius Evola} in \emph{Bilychnis} in June, 1928. It is a review of the \emph{Neugeist} movement in Germany, which was derived from the \textbf{New Thought} movement in USA that had been developing previously. Now ``Neugeist" is the German translation of ``New Thought", but, following Evola, I have sometimes translated it as ``New Spirit" … Evola sometimes uses the German term, and others the Italian \emph{Spirito Nuovo}, a convention which I have followed. The essay will be presented in two parts. This first part is a general introduction. The second part will describe the actual exercises recommended by the New Thought Alliance.

These early articles from Evola that I have been posting demonstrate his spiritual quest, and are quite removed from the later preoccupations with politics, race, etc. It is clear, if readers will look back over the many reviews translated here, that Evola was willing to accept a form of Christianity that was ``practical", mystically inclined, and aligned more with Eastern teachings. Apparently, he was never able to find it, since he subsequently turned away from that quest. He even rejected what he, not incorrectly, called the ``vedantized Christianity" of Guido De Giorgio.

The other failed quest was Evola's search for ``special" powers, whether in the Tibetan Buddhists, yogis, or even the ``New Spirits". It is clear, given the number of references to them, that he was obsessed with the idea. Once again, all the magical exercises probably did not create those powers in the expected way. His correspondence with \textbf{Rene Guenon}, which was have made available, show this.

As for the German New Spirits, they seem to have disappeared without a trace; I am only familiar with Driesch, but only in a different context. I am sure that they were more intellectually sophisticated than their American counterparts. Since New Thought owed a certain intellectual debt to German philosophy, the Neugeist movement probably found a point of contact in it.

Of course, in a sense, they were on the right track. Thoughts do control our lives, that is why we deem it so important to understand one's own thought process. However, they seemed to have been inventing stuff on the fly without ties to a formal tradition. They retained the language of Christianity, but not its essence. As Guenon makes clear, the esoteric cannot be divorced from the exoteric, something that the New Spirits believed they could do. That is why the movement had to fail. On the other hand, a \textbf{Valentin Tomberg}, for example, managed to keep the esoteric and the exoteric together. That is the path to the future, the worthwhile way to ``change your life"; this path is immune to the petty criticisms of the dominant — and only — Western tradition that are heard in some circles. 

\end{quotationx}

In contemporary central Europe one of the spiritual currents that is gaining more ground, is that of the New Thought Alliance (\textit{Neugeistbund}). In essence, it is an offshoot of that reaction against religious dogmatism and scientific materialism, that that had already given rise to other more or less spiritual and mystical movements, like New Thought [in English in original], neo-Rosicrucianism, Christian Science, Anglo-Indian Theosophy and even spiritism. In regards to Neugeist there is however a sense of seriousness and enthusiasm not very common in most of these sects.

Not that it represents actually something original and united: on the contrary, there are clearly found in it ideas from Hindu yoga, medieval mysticism, and even classic German thinkers like Herder, Lessing, Goethe, Schiller, while on the practical side, the influence of the studies on autosuggestion brought in vogue by Coué and Baudouin is very clear. In spite of all this syncretism, in Neugeist there is a certain impulse that in a good measure unifies such elements close to an experienced need to bring man to recapture the sense of interiority and spiritual force.

The movement has its center in Pfullingen (Württemberg), where the editor J. Baum publishes the review Die Wiesse Fahne and a collection of small volumes among which in fact there are enough good ones (e.g., F. Eberspaecher, \textit{Der Giest sie Führer}, K. O. Schmidt, \textit{Selbst und Lebensbemesterung durch Gedankenfraft} and \textit{Wie knonzentrire ich mich?}, etc.). From Pfullingen, the movement extends especially toward Switzerland, Hungary, and Austria, where it has another organ: \textit{Das Neue Licht}, edited by F. V. Schöffel. There are among the adherents some rather noted names, such as Prof Verweyen and the vitalist H. Driesch.

Neugeist intends to be essentially a practical spiritualism, an active mysticism, with disdain for everything that is simple theory or belief. ``Neugeist", says Eberspacher, ``as vision of the world proceeds from this principle: that no knowledge has value if it is not applied practically; that the best ideas serve no purpose, if they are not transformed into action. Men must instill practically every day in life the teachings of a vision of the world."

The first opposition is naturally to materialism; in the second, to the ancient concept of faith and religion, which it wants precisely to replace with the New Spirit (\textit{Neugeist}). According to this ancient concept, says the author cited above, ``faith is nothing more than believing in undemonstrated propositions. Instead, for the New Spirit, faith is a force, by means of which the highest energies can be awakened." Dogma and confession are enemies and lethal for the spirit. Neugeist wants to be an \textit{``undogmatisches Tatchristentum"}, i.e., an active Christianity without dogmas.

At the center of its praxis, there is the concept of the sovereignty and power of thoughts. ``First of all, the most powerful of all forces lies in the force of though. The entire cosmos, all forms, and all phenomena, starting from the atom up to the solar system, are nothing if not materializations of the thought of God. Thoughts construct character, form bodies and faces, regulate health, determine our external relations, our destiny, happiness, or misfortune, the joy or sorrow of men. That which we form, shape, produce, is our creation from the inside—thoughts are the seeds, and the fruit is our fate, our good or our evil." (\textit{Schöffel in Das Neue Licht}).

That said, it is clear that attention is then brought to the techniques to make oneself master of thought in order to act, by means of it, in the direction of spirituality, elevation, and strength, on which it depends. Hence, the noted connection of Neugeist with the teaching of Yoga, with the modern practices of suggestion and autosuggestion, with the mystical disciplines of silence and meditation, purified however of their devotional or moral coloration and undertaken only from the point of view of their practical efficacy.

The inner journey given by Schmidt (\textit{Wie konz, ich mich?}) is divided into four major phases.

\begin{enumerate}
\item \textbf{Concentration}. This includes liberation, internal calm, ``Silence" and the concentration of thinking, feeling, and willing into a single point. 
\item \textbf{Meditation}, understood as the highest level of concentration in which man begins to direct thought onto a determined goal (concentrated meditation) and to make himself capable of keeping, this direction (pure meditation) fixed above everything, 
\item \textbf{Contemplation}, or inner mystical absorption, where the ear is turned to the ``inner voice" to which then one must try to identify oneself, proceeding then to the next level 
\item \textbf{Realization}, or self-completion, the level that precedes ``making oneself Christ" (Durchchristung), that is the highest stage of all the practices of the ``new Spirit". It is the same as ``cosmic consciousness" and ``being one with God" of all mysticisms. 
\end{enumerate}
We don't think it is without interest to reproduce some samples of the exercises used in the New Thought Alliance. The following are found in \textit{Die Giest sei Fuhrer}.

\begin{quotex}
Upon awakening in the morning, get dressed quickly. Turn to the East, extending the arms toward the Sun, draw seven complete breaths, breathing with the formula: ``Primordial creative force, fill me up." Feel the force as it floods you. Then cross your arms on your chest and say, or think, intensely: ``I am one with infinite life, its force fills me, today I will have full dominion over myself, I have its force. I am, I will, I can." Then say this formula in the same position: ``I send thoughts of love, peace, and harmony to the entire world, may all beings be happy and blessed." 

\end{quotex}
Another formula that is concerned more with the breath is found in Schmidt. Here the ancient Hindu doctrine of prana is understood, i.e., the mystical energy of life carried by the air.

\begin{quotex}
The air contains powerful forces that I introduce while breathing in. I draw these forces consciously. They flood my entire body and make me healthy, vital, free! They make me energetic and healthy! The forces restore me and animate me. I feel vital, energetic, healthy.

While exhaling, mentally and consciously chase away all the waste that has accumulated in your body, and at the same time, all the evil dispositions and impressions, negative and destructive thoughts. Think and feel: ``I breathe, I am free … free! I feel free — I am conscious, I breathe. Cosmic forces flood me and unite me to the Divine!" 

\end{quotex}
The repetitions, naturally, are made with the aim of autosuggestion. And now a formula for calm and inner silence. It follows exercises by which the complete relaxation of the body and every tension, almost to the point of no longer feeling it, are first stimulated. Again, from Schmidt:

\begin{quotex}
The external world has disappeared. I am alone, deeply in myself … I am silent, I am calm. I think and I feel that I am completely calm. Tranquility and calm are in me. Everything is calm rhythmic proportion, cosmic harmony, infinite peace. 

I am calm

Calm as though in a distant abandoned tomb.

Calm as though in the base to a clear transparent mountain lake.

Calm as though in a city underneath the heat of the summer sun, calm, tranquil, desolate, without noise, in anticipation of the freshness of the evening. 

\end{quotex}
Finally we translate a collective formula used by the Stuttgart group which was published in \textit{Die weisse Fahne}.

\begin{quotationx}
Om!

Rise up in me, you Conqueror of Death! Light of Christ! Infinite splendour of my soul!

You are! You dare! You burn up! In my soul. Light of Christ in me, I adore you!

In me was your Golgotha! In me is your Resurrection! Your eternal life lives in me! In me your paradise blooms! In me your liberation rejoices. In me your victory exults! In me your ancient flame shines. 

The Victor rises again!

You bring me back to the lost Father. You gave back the Father in me, who is also the Father of your original generation.

Your Father is also in me! Om! I am!

You, Lord, have risen again in me! New Spirit!

Ardent Spirit, resplendent with the original Light! Holy Spirit!

You who secures everything. You who hears everything. You breathe the world. You, the generator of the new Spirit. You the adsorbing Principal, to be bursting, the wise end, you, again the only Principle. You, the Everything and the Nothing, you, the One in Everything. 

You, the eternally active spiritual presence! Rising again!

Heaven jumps in me. The most sacred veil tears asunder in me. Your Light erupts from it. Your dawn Light of Eternity. 

In front of me, in me, every night!

Ancient Light, jubilant victory of the resurrection in me! Om! Victory! Victory! Victory! Om! Christ, you are the new spirit of Resurrection in me! Om! Peace! Peace! Peace is with us all! Om! 

\end{quotationx}
As you see, it is all rather well constructed in suggestive and intensely charged emotions in order to lead to a state of mystical exhilaration, not without certain character of virility.

What is interesting is to notice the scientific and, we would say, secular elements in the background, with which all the various means are syncretistically gathered in order to produce experiences, which are of value essentially from their practical efficacy. In other words, technique, in place of faith, and inner facts to acquire from pure individual experience, without dogmatic and profane ``interpretations", with an eye simply on their pragmatic ``truth" for those who demand a sense of exaltation of liberation of vital energy. Reaching to the foundation in this direction, one can expect that one day religion, as well as theology itself, will become an experimental science, certainly an upheaval, not lacking interest, that leads us back to a proper view of mystical and traditional esoterism.

These and similar movements are therefore observed with curiosity, as signs of the times. To the critic, for now it is more opportune to replace the watchful gaze that deepens the new themes that could begin to speak out in them.



\flrightit{Posted on 2013-12-04 by Aeneas }

\begin{center}* * *\end{center}

\begin{footnotesize}\begin{sffamily}



\texttt{scardanelli on 2013-12-04 at 21:45 said: }

``The first opposition is naturally to materialism; in the second, to the ancient concept of faith and religion, which it wants precisely to replace with the New Spirit (Neugeist)"

It seems that all such spiritual movements that are predicated on an opposition are doomed to failure. If the nature of this opposition is dualism then this dualism would make true transcendence impossible. This might also be applied to those traditionalists whose only rule for membership is being ``against" the modern world. 

Likewise for any attempt to strip Christianity of it's dogmas or confession. It seems to me that the truly esoteric should grow like a tree from the ground of the exoteric. The way of most new age movements is an attempt to usurp or get around the law rather than to develop and gain freedom from the law.


\hfill

\texttt{Anna Kaiser-Swadling on 2013-12-06 at 12:15 said: }

I've used something similar to the first ``auto suggestion" for many years. I find it empowering. Evola also has some techniques for falling asleep that I've applied for years after the ``I am alone, deeply in myself", i.e. visualising and moving toward a golden mountain peak, lying on your side and visualising you're looking from the back of your head, and another, lying on your back letting your eyes fall back then sense you're flipping back with them into the air above.


\hfill

\texttt{anon on 2013-12-07 at 04:37 said: }

This stuff showed up on my radar when I stumbled across Lisiewski, who hated New Age, but liked New Thought (NT). I woke up early today, and have been trying to find (online) the succinct Lisiewksi quote about NT that first impacted me, but to no avail. I guess these long ones will do. 

``You are the first of my readers to acknowledge that they have looked into that form of Mysticism referred to as `New Thought.' I am very happy you have indeed. You will find that—as I have written in the Magical Thought of the Week column weeks ago— Mysticism completes Magic: it is not the other way around, no matter how much this upsets people who are so entirely devoted to Magic. The entire matter of Spiritual `Unfoldment' as I term it—not Spiritual `growth' as the New Age insists—eventually leads to this end.

``Further, the amazing thing is that what the individual so desires; what he or she so sincerely wants or needs in their life, is all obtainable through this form of Mysticism, and ever so easily. Yet the paradox is that one cannot force one's self into accepting, working, and mastering New Thought simply because the results they require are so readily obtainable from it. Quite the contrary. One must `unfold' into it. And when that glorious days dawns, then all is made perfectly clear and works almost effortlessly for the individual so blessed with this `discovery.'

``People who are attracted to Magic must persist in it until the day comes when they `step over the line' that leads from Magic to Mysticism. It cannot be hurried. It cannot be rushed. But it will come. This is what Percy Bullock, one of the original members of the Golden Dawn Society meant, when he said at the turn of the 20th century, `In the end, we all become Mystics.' 

"Congratulations on your being attracted to it so early on. Perhaps—as sometimes does happen—it is the right—and only—Path for you to follow, even from the beginning."

– Joseph C. Lisiewski

—————————–

``My Reply: again, these individuals neither understand the difference between the `imaging' I teach, and the `visualization' which their New Age calls for. Nor do they understand the difference between one of their so-called `Invocations,' a `Prayer,' and an `Affirmation,' all of which I have thoroughly spelled out both in my books and on this website. And as the next reply will further show, New Thought has nothing whatsoever to do with `visualization' (in fact, those methods of New Thought that truly do work to produce full results have very little to do with Imaging either. That is, they employ Imaging only in so far as one forming the most `general thought template' of that which is desired.)"

– Joseph C. Lisiewski

—————————–

``Always use a nine foot diameter Circle of Art for every ritual and ceremonial act. `Nine' is the number of Yesod, of course, and corresponds to the subconscious (unconscious) mind, wherein all Magic truly occurs- we only `see' the results of that `occurrence' in the outer world (Malkuth.) That is, the subconscious mind should be thought of as the Causal Agent of the External Effect we wish to produce. There is a great deal of correlation between this concept and the central doctrine of New Thought. That is, with the extremely pragmatic philosophy and practice of the latter. But this is another matter for another time."

– Joseph C. Lisiewski

—————————–

``It may help if I give you an example. As I stated in Ceremonial Magic, I was born into a Roman Catholic family and raised in that Faith. And while I cannot abide the Catholic church in its present form, I yet consider myself a Christian, and more exactly, a Catholic! How can this be? Quite simply, I found the philosophical teachings of Pierre Teilhard de Chardin, a Jesuit Priest, theologian, philosopher, and paleontologist-and whose complex writing unify certain aspects of Science and Religion-to have a profound impact upon me. Together with the New Thought Philosophy, the two enabled me to form my own eclectic system of thought; one which is in perfect agreement with my own subconscious state of subjective synthesis, and which allows my magical efforts to succeed splendidly."

– Joseph C. Lisiewski

—————————–

``Apparently, they never heard of the work of Ralph Waldo Trine and his book, In Tune With the Infinite, which is regarded as the first independently published book on New Thought, and which is given credit for being one of the major works on the subject that sparked the New Thought Movement itself. And it was published in—1897! That is, his final manuscript, originally written in 1889, was published years before their famous Golden Dawn was being torn asunder by internal strife, and moving toward its ultimate 1900 collapse. Further, one has only to read Trine in the most cursory way to see if any `Golden Dawn' material—even theoretical—is in it. The philosophy of the Golden Dawn—and the Mysticism of Trine—are as different in theory and practice as are night and day.

``In the same vein, the people demanding I acknowledge the GD as having created the Mysticism of New Thought have no knowledge whatsoever of the existence and work of others in the field of mystical enquiry which has come to be termed `New Thought.' For instance: they apparently are completely oblivious to the late 19th century writings and teachings of Emilie Cady and Charles Fillmore, cofounder, who began the Unity Movement of circa 1890, the very time at which the first Golden Dawn Temple was getting underway in England. Formally called the Unity School of Christianity, the groundbreaking works of Cady and Fillmore—along with Trine's and a number of others less well known—constituted the initial creative impulse that launched the New Thought Movement.

``Further, unlike the bickering and backstabbing of the famous London GD Temple `elite' of the 1890's, enthusiastic collaboration between the schools of thought arising within the original New Thought Movement was the order of the day. Example: Emilie Cady—a staunch independent—published several of her most notable works in Unity Magazine, the official organ of the Unity School of Christianity in 1894-95 (these were then published as three paperbacks for the general public in 1896-1897.)"

– Joseph C. Lisiewski

—————————–

``But the Golden Dawn did has its effect, or so I speculate it did. That is, I cannot prove it empirically, but my research suggests more than a casual connection exists between the New Thought Movement and the original Golden Dawn as operated in England in the 1890's. And that is, that after the Golden Breakup in 1900, a plethora of new books arose on the subject of New Thought, the great majority of which suddenly began teaching `visualization' as part of their process. Not only that, but the visualization techniques they insisted upon are in every way exactly the same as can be found in any New Age book today—be it on Magick or on visualization techniques designed to bring about one's desires. For instance, the numerous works of Wattles (1902) Wilcox (1902) Towne (1904) Fritz (1906) Larson (1908) Sears (1919) Schubel (1922) and Van Resselaer Day (1928) are only a few of the many that initiated this unfortunate trend, while the latter works of Fersen (1929) Landone (1937) and Collier (1943) continued it. And of course, these later efforts were seized upon by the New Age and --‘expanded' into the non-functional miasma of so-called `New Thought' that generally exists today.

``In short, as the GD history states, most members of the original London Temple were `unknowns' who drifted away after the demise of the Temple. It may be that such people as Wattles—amidst dozens of others—were indeed members of the Order, or were acquainted with those who were. In either case, disillusioned due to the breakup of the GD, it is my opinion that they seized upon the New Thought Movement and added their `visualization' techniques to it, thus destroying the genuine techniques behind this form of Mysticism; techniques that indeed do allow New Thought—or `Higher Mysticism' as I term it— to work—fully, and without any Slingshot Effect. So much then for the people who lay claim that the GD `created' New Thought. As far as I am concerned, all that the (former) members of the original GD did to the New Thought Movement was contaminate it…"

– Joseph C. Lisiewski


\hfill

\texttt{anon on 2013-12-07 at 11:08 said: }

Warning: long and perhaps self-indulgent post follows.

———————————

Lisiewski's marriage of grimoires and New Thought reminds me, to some degree anyway, of those people who try to combine NLP with Crowley and/or with chaos magic. The whole Disinfo.com/RAW/Gen P-Orridge scene. Except that stuff is more modernist, I guess, whereas Lisiewski is relatively more traditional.

I read Jason Augustus Newcomb's ``21st Century Mage: Bring the Divine Down to Earth", years ago, and absolutely loved it. So much so that I joined his online forum for a while. And gushed words of appreciation at him. I couldn't get into his other book that was available at the time, ``New Hermitics". I had found a pdf copy of it, but couldn't force my way through it. It was NLP meets Leary meets Crowley meets RAW. Oh yeah, and hermetix too.

I remember when I first read an Anthony Robbins book, on the recommendation of someone I respected at the time, and thought to myself: ``This is magic." I never would have thought that. I saw Robbins as crass infomercial huckster. A bullshit artist. And I guess in many ways he is. But his neuro-associative conditioning (NAC) techniques are basically a form of crude and uncontrolled magic. Or something akin to it. As I played around with that stuff, those many years ago, I remembered the Sorcerer's Apprentice segment of the Disney cartoon Fantasia. Where Mickey Mouse casts a spell over some brooms, and then loses control over them. It was like that. NAC would allow you to effect change on the world around you. But this was done by first affecting changes inside you. And some of these changes were damaging, blinding you. Perhaps good for taking someone from tamas to rajas, but blinding them to many important aspects of sattva. Anyway…

Newcomb's Mage book is basically about the K\&C of the HGA, but in a sort of modernized and New Age-ified form. I neither know nor care if the practical suggestions that he offers in that book work. I already have my own practice. But the theoretical aspect of the book I found extremely and profoundly validating.

When you're navigating these meagerly-charted waters you need a compass. And the HGA provides that. Actually, I don't think of it as the HGA. As best as I can tell what people call the HGA and the K\&C of the HGA are really fragments of something much larger. Much more beautiful. The Abramelin operation, I believe, allows one to access only some of what is there. Like seeing an iceberg, and knowing that more is there under water.

Newcomb's book goes deeper than the HGA, but I'm not even sure if that is conscious on his part, or if his ``HGA" slipped some things in there, between the lines, in the connections between what he says, never fully spelled out. It's worth reading. 

Before I read Newcomb, I read one of the best books that I have ever read in my life. A book whose concepts still deeply nourish me, nine years later. When I found it, it was like the culmination and synthesis of a bunch of stuff that I was studying. I could not believe that such a thing existed. It was as though it had been created just for me. Of course, none of us are all that special, and any thought that has occurred to us, has also occurred to someone else, somewhere else. 

The book was like a synthesis and distillation of many aspects that I found most helpful and relevant in Jung, Tony Robbins, Crowley, Kabbalah, yoga, bhakti, chaos magic, psychology, New Age, religion and God knows what else. Plus I was physically chronically quite sick. And seriously worried about how the hell I the hell I was going to be able to support myself. I was working only part time. I was tired all the time. Low-grade fever almost all of the time. Heart-rate unnaturally high. All the time. Recovering from a very serious life-saving surgery. I did not seem to be getting better. I was afraid. 

The crass and cheesy title of the book is ``How to Get Lots of Money for Anything Fast". I have shared the pdf with significant people in my life, but none have had the experience with it that I have. Maybe it really is suited to a particular temperament? Or to people looking for a particular thing? Some are turned off by the title. Understandably. But the book really isn't about money. Or that is only the surface. It is about the K\&C of the HGA, although the author doesn't use those words. In practice, that is what it is.

The first time I did the exercises, it was profound. The strange chest pains I was having for some years, both before and after the surgery, were gone the next day. I kept expecting them to return. A week went by. Then weeks. Then months. Now years. The pains return only in periods of extreme stress. And that in very mild forms. Was it C.S. Lewis who said that pain was God's megaphone? I listen more carefully now, so I guess maybe He feels less of a need to shout as loud. 

The book is about navigating life, with the guidance of the HGA. Actually, something deeper, lusher and fuller than the HGA. Some will use that guidance to make money, but that is really like using a smartphone to hammer a nail.

But it can be used for that also. I remember an interview with Lichtman where he talks about something tho the effect that his idea was to provide a process whereby people could take care of their material needs simply and quickly, leaving time and energy for their spiritual pursuits.

My understanding is that there is a reality. Different people describe that reality, to varying degrees of accuracy. With varying degrees of clumsiness. Insofar as their gunas will allow them to perceive and describe it. Depending on what their goal is, reality can be sliced up in different ways, sub-divided into different categories. Some more misleading than others.

Lichtman calls his system Cybernetic Transposition (CT), being inspired in part, by Maxwell Maltz's Psycho-Cybernetics. But there is also a big dose of Lichtman's own spiritual beliefs intertwined in all of it. This troubled me somewhat, as his system goes deep into the unconscious, and I was worried about implanting some of Lichtman's own blind spots and perception-distortions into my own heart. 

So for a while I was really interested in what his path was. Finally, I was able to piece together who his spiritual teacher was. The dude's name escapes me at the moment. I could do a CT exercise to get my memory to cough up his name, but the important thing is that I found him both underwhelming, and mostly non-threatening. Maybe Lichtman has eclipsed his teacher?

Is Lichtman's CT Traditional? No, I wouldn't say that. And some of his blind spots I find loathsome. But I think CT can (largely) be used whoever one chooses. There is always a spiritual price to be paid with this kind of stuff, but I think CT fairly neutral, like logic. But then again, don't take my word for it. For all you know, I might simply be a self-deluded cultist.

Some of Lichtman's CT reminds me of a highly-fine-tuned version of chaos magic's sigilization process. Except done with words, paragraphs, vivid imaginary experiences. Instead of the commonly-used pictogram method. His ideas about getting the different parts of the brain working together, in concert, working towards the same end, cannot help but remind me of Kabballah's description of different parts of the soul. And also the Gita's vyavasay-atmika-buddhi. I will resist the temptation here to discuss the parallels between the source of buddhi and the functions of the HGA. This post is already long enough. 

CT has a bunch of controls built-in, to avoid having it blow up in your face, and for getting back on track. Since life is, in many ways, a moving target. You know now, things that you did not know then. And so on. 

This is not meant as an infomercial, or endorsement of Lichtman. I'm not getting paid for this. It's a genuine outpouring. Lisiewski has his New Thought, the Disinfonauts have their NLP Bandlerism. I prefer CT. 

Dig it.


\hfill

\texttt{IA on 2013-12-07 at 11:26 said: }

I've used daoist stretching, breathing and visualing techniques for over 18 years following recurring back pain from an injury. You face the sun every morning. It has worked very well for the back pain and may have some emotional effect as well.


\hfill

\texttt{pareidoliastic on 2014-08-11 at 11:48 said: }

anon on 2013-12-07 at 04:37, 

I have read all of Lisiewski's works and he helped open my eyes to Alchemy and Hermetics. After having a profound encounter with NLP in helping me literally evaporate an addiction I was trapped in. I am now very hungry for more and how NLP/NewThought could help me grow and evolve and get past all the other blocks from brainwashing and false beliefs I am clouded by.

I found this very page on the Internet because I to have a vague recollection of him talking about NLP and it's relationship to `magic/mysticism'

I have a great deal of trust in Lisiewski so his opinions and advice I give much more weight to than most.

From the quotes you gave he seems to indicate New Thought has been corrupted? Naturally because of the power these techniques have to modify the subconscious and influence ones subjective synthesis I am leery of just blindly following my natural instinct to read everything I can find on it. Other than his (apparent?) endorsement of Ralph Waldo Trine and his book, In Tune With the Infinite do you know of any of his recommendations on works for studying and working to find true liberation? I am also considering Hyatt's work in this area as I know he and Lisiewski were good friends.

Thank you, and anyone else who could perhaps aid me.


\hfill

\texttt{Cologero on 2014-08-12 at 07:51 said: }

pareidoliastic,

You are referring to some of the comments on Lisiewski, which were motivated by the translate essay from the German New Thought movement.

``True liberation" is not found just from studying, but from ``working" as you point out. It is not likely that much can be accomplished on one's own.

The danger is that before attempting to ``modify the subconscious", you should have a very clear idea of what you are doing. In other words, don't tinker with the mechanisms until you understand the inner workings of the soul quite well.

Originally, NT was based on certain philosophical systems, mostly forgotten because of the low quality of the NT movement. We have been discussing the intellectual foundations of magical idealism, which would serve as a better foundation. After all, what does it mean that representations create reality? We plan to revisit this theme in the next week or two.


\hfill

\texttt{pareidoliastic on 2014-08-12 at 15:19 said: }

Thank you for the reply Cologero. I guess the danger you mention is my trepidation about just `jumping in'. I understand the power of it now firsthand by using aspects of these techniques crafted by another to help root out destructive beliefs I had held subconsciously which made changing something the typical world thinks is nearly impossible. But it was actually quite easy. It has been said this person's book uses NLP so that rang a bell in my memory from when I studied Lisiewski's work and writings.

So naturally I am excited and very interested in how far that rabbit hole goes (in a positive sense) in how much greater I can truly be with the right constellation in my subconscious.

I have not really studied this site yet and what it contains in this regard. Can you kindly point me in the direction of the right place or post to dive into the stream on exploring this `magical foundation' theme you speak of?

And I look forward to your further posts on this theme you mentioned.

Many many thanks!


\end{sffamily}\end{footnotesize}
