\section{Week 3: Hallowed be thy name}

\textsc{Meditations}: The nine Beatitudes: Mt 5:1-12, Lk 6:20-23

This verse will require two weeks. The beatitudes awaken the Christ impulse in the various centers of the human being.

The nine beatitudes are the nine activities of the Comforter, the Paraclete, and are oriented to the perfect human being
of the future. That is, they relate to the spirit-imbued human being.

\begin{quotationx}
The Sermon on the Mount is not concerned merely with human beings preserving their true natures in the face of natural
evolution, nor even with them simply obeying the divinely revealed law, but that, in accordance with their archetype
— “the image and likeness of God” — they become as God. “Be ye perfect
therefore even as thy Father in heaven is perfect.” This central statement from the Sermon on the Mount is a call to
ascend from the kingdoms of nature and the human being to the kingdom of God. 
\begin{flushright}\textit{Covenant of the Heart}\end{flushright}

\end{quotationx}
Jesus based Baptized in complete emptiness in the Jordan.

Then cometh Jesus from Galilee to the Jordan, unto John, to be baptized by him. But John stayed him, saying: I ought to
be baptized by thee, and comest thou to me? And Jesus answering, said to him: Suffer it to be so now. For so it
becometh us to fulfill all justice. Then he suffered him. And Jesus being baptized, forthwith came out of the water:
and lo, the heavens were opened to him: and he saw the Spirit of God descending as a dove, and coming upon him. And
behold a voice from heaven, saying: This is my beloved Son, in whom I am well pleased. (Matthew 3:13-17)

And it came to pass, in those days, Jesus came from Nazareth of Galilee, and was baptized by John in the Jordan. And
forthwith coming up out of the water, he saw the heavens opened, and the Spirit as a dove descending, and remaining on
him. And there came a voice from heaven: Thou art my beloved Son; in thee I am well pleased. (Mark 1:9-11)

Now it came to pass, when all the people were baptized, that Jesus also being baptized and praying, heaven was opened;
And the Holy Ghost descended in a bodily shape, as a dove upon him; and a voice came from heaven: Thou art my beloved
Son; in thee I am well pleased. (Luke 3:21-22)

The next day, John saw Jesus coming to him, and he saith: Behold the Lamb of God, behold him who taketh away the sin of
the world. This is he, of whom I said: After me there cometh a man, who is preferred before me: because he was before
me. And I knew him not, but that he may be made manifest in Israel, therefore am I come baptizing with water. And John
gave testimony, saying: I saw the Spirit coming down, as a dove from heaven, and he remained upon him. And I knew him
not; but he who sent me to baptize with water, said to me: He upon whom thou shalt see the Spirit descending, and
remaining upon him, he it is that baptizeth with the Holy Ghost. And I saw, and I gave testimony, that this is the Son
of God. (John 1:29-34)

\subsection*{Day 1: First Beatitude: emptying the head}
\emph{Blessed are the poor in spirit, for theirs is the kingdom of heaven}. (Mt 5:3, Lk 6:20)

Alternatively, Blessed are the beggars for spirit, for theirs is the kingdom of heaven.

In his Baptism in the Jordan, Jesus had become empty and the greatest beggar for spirit.

\begin{quotationx}
The first beatitude means to say that those who are rich in spirit, who are filled with the “spiritual kingdom of man”,
have no room for the “kingdom of heaven”. Revelation presupposes emptiness — space put at its
disposal — in order to manifest itself. This is why it is necessary to renounce personal opinion
in order to receive the revelation of the truth, personal action in order to become an agent for sacred magic, the way
(or method) of personal development in order to be guided by the Master of ways, and one's
personally chosen mission in order to be charged with a mission from above.
\begin{flushright}\textit{Meditations on the Tarot}\end{flushright} 

\end{quotationx}
As a fact the wandering in the desert was an historical occurrence, as a symbol it is an expression of the timeless law
of the necessity of purification and the “emptying of consciousness” as precondition for the revelation of God in His
truth.

\begin{quotationx}
the first Beatitude of the Sermon on the Mount proclaims: “Blessed are the poor in spirit for theirs is the kingdom of
heaven,” i.e., blessed are those who regard as poor any knowledge of power without God—that is,
any knowledge or power not of God himself—for they shall participate in the divine archetypal
creative work of God. 
\begin{flushright}\textit{Covenant of the Heart}\end{flushright}

\end{quotationx}
\subsection*{Day 2: Second Beatitude: courage of the heart}
\emph{Blessed are they that mourn, for they shall be comforted}. (Mt 5:4, Lk 6:21)

The blessing of Mary is contained in this; the inspiration that comes from suffering (“And a sword shall pierce thy
heart”). The courage to endure suffering.

\begin{quotationx}
They that mourn or bear sorrow neither strive after a pain-free existence nor turn away from pain, but bear it with
acceptance. For the fullness of existence, life's true richness, does not consist solely in health
and happiness but in an ever-expanding range of joy and sorrow; and the broader the range, the richer life becomes.
\begin{flushright}\textit{Covenant of the Heart}\end{flushright}

\end{quotationx}
\subsection*{Day 3: Third Beatitude: control of the will, mindfulness}
\emph{Blessed are the meek for they shall inherit the earth}. (Mt 5:5, Lk 6:21)

Behold the course of one's life with inner calm.

\begin{quotationx}
Natural evolution rests on the principle that dominion over the kingdom of nature—over the earth
—belongs to those with the greatest will to power. It is predestined for the tough and the
toughest … Not the tough but the meek shall rule earth's natural kingdoms. The power that St.
Francis of Assisi, for instance, wielded over birds and fish and wild wolves was not of a kind that any natural
scientist, fisherman, forester, or hunter has ever possessed. The same is true of the obedience rendered toward St.
Anthony by the hyenas in the Egyptian desert, as well as of many other instances of deference from the side of
so-called dumb nature toward truly meek humans … That Christianity prevailed in ancient times despite its persecution
is an instance on a global scale—one that cannot simply be explained away—
where meekness took possession of the “earthly kingdom” (orbis terrarum) of that time. 
\begin{flushright}\textit{Covenant of the Heart}\end{flushright}

\end{quotationx}
\textsc{Meditation}:

And behold an angel of the Lord stood by them, and the brightness of God shone round about them; and they feared with a
great fear. And the angel said to them: Fear not; for, behold, I bring you good tidings of great joy, that shall be to
all the people: For, this day, is born to you a Saviour, who is Christ the Lord, in the city of David. And this shall
be a sign unto you. You shall find the infant wrapped in swaddling clothes, and laid in a manger. And suddenly there
was with the angel a multitude of the heavenly army, praising God, and saying\textbf{: Glory to God in the highest; and
on earth peace to men of good will}. And it came to pass, after the angels departed from them into heaven, the
shepherds said one to another: Let us go over to Bethlehem, and let us see this word that is come to pass, which the
Lord hath shewed to us.  (Luke 2:9-15)

Glory (head) – Peace (heart) – Good Will (will)

\textbf{Vladimir Solovyov}, in his \emph{Lectures on Divine Humanity,} wrote:

\begin{quotationx}
One can be considered free from passions only when one has them but has power over them, when one possesses, but is not
possessed by them.

\end{quotationx}
In Matthew, the word translated as “meek” is \emph{praos}, which was also used to mean the gentling of a horse. The
horse is the most noble of animals, but it must be broken to be of use to the knight.

So we see that the “meek” are those whose souls are under the control and guidance of the Intellect, or Head, since the
appetitive and incensive aspects of the soul (the Will) have been broken and put in service to the highest aspect in
man. They still retain their driving forces: the appetitive to achieve a goal, and the incensive to provide the
emotional energy and motivation to persist. Far from being mousy or timid, the meek are strong willed and inner
directed. This is why they shall inherit the earth.

\subsection*{Day 4: Fourth Beatitude: Christ impulse in the Sentient Soul}
\emph{Blessed are they that hunger and thirst after justice: for they shall have their fill}. (Mt 5:6, Lk 6:21)

Emptied head à Courageous Heart à Mindfulness in the Will à righteousness

\begin{quotationx}
The law of righteousness of the “kingdom of God” as expressed in the Sermon on the Mount … is operative in the highest
and most essential region of all. It heals the wounds of the heart suffered at the hand of injustice and transforms the
pain of unjustly inflicted suffering into everlasting bliss. At the same time, without inflicting punishment, it leaves
offenders to the tribunal of their own conscience—that is, to their karma. Thus, does the
righteousness of the kingdom of God transcend both righteousness of retribution and that of atonement (karma) in that
it is a bountiful and merciful righteousness. It bestows gifts of eternal value in whose light the shadows cast through
suffering injustice disappear. 
\begin{flushright}\textit{Covenant of the Heart}\end{flushright}

\end{quotationx}
\subsection*{Day 5: Fifth Beatitude: Christ impulse in the Intellectual Soul}
\emph{Blessed are the merciful: for they shall obtain mercy}. (Mt 5:7)

Mercy is justice combined with love. It is judgment and at the same time creation of the means whereby the guilty can
atone for their guilt, carrying not only the past and the present, but also the future.

\begin{quotationx}
All those who desire—that is, actually make the effort—to practice a morality
transcending retribution and atonement will have their place in the kingdom of God and his righteousness … It is thus
not the will for the good favor and benefaction of merciful righteousness that makes one a partaker thereof, but the
will for the kingdom of God and his righteousness in itself—will that is put into actual practice.
\begin{flushright}\textit{Covenant of the Heart}\end{flushright}

\end{quotationx}
Through the sentient soul man is related to the animal. In animals, also, we observe the presence of sensations,
impulses, instincts and passions. But the animal obeys these immediately. They do not, in its case, become interwoven
with independent thoughts, transcending the immediate experiences. This is also the case to a certain extent with
undeveloped human beings. \textbf{The mere sentient soul is therefore different from the evolved higher member of the
soul which brings thinking into its service. This soul that is served by thought will be termed the intellectual soul}.
… The intellectual soul permeates the sentient soul. Whoever has the organ for “seeing” the soul sees, therefore, the
intellectual soul as a separate entity, in relation to the mere sentient soul.

Note: The Intellectual Soul is also called the Rational Soul. (e.g., Thomas Aquinas, Aristotle)

\subsection*{Day 6: Sixth Beatitude: Christ impulse in Consciousness}
\emph{Blessed are the pure of heart: for they shall see God}. (Mt 5:8)

This is the quality of deepened and extended mercifulness. With radiant heart, behold nature with a gaze that asks what
nature needs. The healing gaze into the world is the pure heart; then one beholds God who is otherwise missing in the
natural world.

Purity of heart is to will one thing. This is the return to the Primordial State before the Fall, i.e., before Adam and
Eve listened to two competing, incompatible voices.

\begin{quotationx}
Duality therefore signifies the establishment of two centers of contemplation, two separate and rival principles
—one real and the other apparent —and this is the origin of evil, which is
only illegitimate twofoldness. 
\begin{flushright}\textit{Meditations on the Tarot. Letter II: The High Priestess}\end{flushright}

\end{quotationx}
\subsection*{Day 7: Seventh Beatitude: Christ impulse in the Self}
\emph{Blessed are the peacemakers: for they shall be called children of God}. (Mt 5:9)

The quality of the third beatitude, gentleness, is directed outward in the seventh beatitude, engendering peace.

\textsc{Temptation}: to see Christianity as a revelation given once and for all.

\begin{quotationx}
[The peacemakers are those] who refuse to take sides in the face of partial truths and prejudices, being dedicated to
the cause of the whole truth which unites the world and bears peace to it.
\begin{flushright}\textit{Meditations on the Tarot. Letter
IX: The Hermit}\end{flushright}

The path of transformed evolution, in the case of individual human beings and of humanity as a whole, begins with a
purification of the impulses, instincts, habits, and customs attached to natural evolution. Then it leads on to
illumination—intuition of the truth, beauty, and goodness of divine evolution, the kingdom of God
and his righteousness. Finally, it culminates in the union of will, feeling, and thought with the will, feeling, and
thought that underlie divine evolution or the work of salvation. 
\begin{flushright}\textit{Covenant of the Heart}\end{flushright}

\end{quotationx}

\flright{\small\textit{Posted on 2023-02-27 by Cologero}}