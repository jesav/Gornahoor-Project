\section{Week 1: First reading}

\subsection*{Day 1: Our Father who art in Heaven}
\textsc{Reading}: The Story of Paradise

The temptation in paradise was threefold, just as was the temptation of Jesus Christ in the wilderness. The following
are the essential elements of the triple temptation in paradise, as it is described in the account of the Fall in the
book of Genesis (from \emph{Meditations on the Tarot}):

\begin{enumerate}
\item Eve \emph{listened} to the voice of the serpent; 
\item She “\emph{saw} that the tree was good for food, and that it was a delight to the eyes” (Genesis 3:6); 
\item She “\emph{took} of its fruit and ate; and she also gave some to her husband, and he ate” (Genesis 3:6). 
\end{enumerate}

\begin{multicols}{2}\small
Now the serpent was more subtle than any of the beasts of the earth which the Lord God had made. And he said to the
woman: Why hath God commanded you, that you should not eat of every tree of paradise?

And the woman answered him, saying: Of the fruit of the trees that are in paradise we do eat:

But of the fruit of the tree which is in the midst of paradise, God hath commanded us that we should not eat; and that
we should not touch it, lest perhaps we die.

And the serpent said to the woman: No, you shall not die the death.

For God doth know that in what day soever you shall eat thereof, your eyes shall be opened: and you shall be as Gods,
knowing good and evil. And the woman saw that the tree was good to eat, and fair to the eyes, and delightful to behold:
and she took of the fruit thereof, and did eat, and gave to her husband who did eat. And the eyes of them both were
opened: and when they perceived themselves to be naked, they sewed together fig leaves, and made themselves aprons. And
when they heard the voice of the Lord God walking in paradise at the afternoon air, Adam and his wife hid themselves
from the face of the Lord God, amidst the trees of paradise.

And the Lord God called Adam, and said to him: Where art thou?

And he said: I heard thy voice in paradise; and I was afraid, because I was naked, and I hid myself.

And he said to him: And who hath told thee that thou wast naked, but that thou hast eaten of the tree whereof I
commanded thee that thou shouldst not eat?

And Adam said: The woman, whom thou gavest me to be my companion, gave me of the tree, and I did eat.

And the Lord God said to the woman: Why hast thou done this? And she answered: The serpent deceived me, and I did eat.

And the Lord God said to the serpent: Because thou hast done this thing, thou art cursed among all cattle, and beasts of
the earth: upon thy breast shalt thou go, and earth shalt thou eat all the days of thy life. I will put enmities
between thee and the woman, and thy seed and her seed: she shall crush thy head, and thou shalt lie in wait for her
heel.

To the woman also he said: I will multiply thy sorrows, and thy conceptions: in sorrow shalt thou bring forth children,
and thou shalt be under thy husband's power, and he shall have dominion over thee.

And to Adam he said: Because thou hast hearkened to the voice of thy wife, and hast eaten of the tree, whereof I
commanded thee that thou shouldst not eat, cursed is the earth in thy work; with labour and toil shalt thou eat thereof
all the days of thy life. Thorns and thistles shall it bring forth to thee; and thou shalt eat the herbs of the earth.
In the sweat of thy face shalt thou eat bread till thou return to the earth, out of which thou wast taken: for dust
thou art, and into dust thou shalt return.

And Adam called the name of his wife Eve: because she was the mother of all the living. \flright{\itshape Genesis 3:1-20}
\end{multicols}

\subsection*{Day 2: Hallowed be thy name}
\textsc{Reading}: The Nine Beatitudes

\begin{quotationx}
It is said that, “Nature has a horror of emptiness” (horror vacui). The spiritual counter-truth here is that, “the
Spirit has a horror of fullness”. It is necessary to create a natural emptiness —and this is what
renunciation achieves — in order for the spiritual to manifest itself. The beatitudes of the
Sermon on the Mount (Matthew v, 3-12) state this fundamental truth. 

\begin{flushright}\textit{Meditations on the Tarot}\end{flushright}

\end{quotationx}
\begin{enumerate}
\item Blessed are the poor in spirit: for theirs is the kingdom of heaven. (Mt 5:3, Lk 6:20) 
\item Blessed are they that mourn: for they shall be comforted. (Mt 5:4, Lk 6:20) 
\item Blessed are the meek: for they shall inherit the earth. (Mt 5:5, Lk 6:21) 
\item Blessed are they which do hunger and thirst after righteousness: for they shall be filled. (Mt 5:6, Lk 6:21) 
\item Blessed are the merciful: for they shall obtain mercy. (Mt 5:7.) 
\item Blessed are the pure in heart: for they shall see God. (Mt 5:8) 
\item Blessed are the peacemakers: for they shall be called the children of God. (Mt 5:9) 
\item Blessed are they which are persecuted for righteousness' sake: for theirs is the kingdom of
heaven. (Mt 5:20, Lk 6:20 
\item Blessed are ye, when men shall revile you, and persecute you, and shall say all manner of evil against you
falsely, for my sake. Rejoice, and be exceeding glad: for great is your reward in heaven. (Mt 5:11-12, Lk 6:22-23) 
\end{enumerate}

\subsection*{Day 3: Thy Kingdom Come}
\textsc{Reading:} The seven stages of the Cross in John's Gospel.

\begin{quotationx}
Christian meditation pursues the aim of deepening the two divine revelations: holy scripture and the creation, but it
does so above all with a view to awakening a more complete consciousness and appreciation of Jesus
Christ's work of redemption. For this reason, it culminates in the contemplation of the seven
stages of the Passion: the washing of the feet, the scourging, the crowning with the crown of thorns, the way of the
cross, the crucifixion, the laying in the tomb, and the resurrection. 

\begin{flushright}\textit{Meditations on the Tarot}\end{flushright}

\end{quotationx}
\textsc{Meditation}: The seven stages of the Passion

\begin{enumerate}
\item Washing of the Feet (Jn 13:1-20) 
\item Scourging (Jn 18:22, Jn 19:1-3) 
\item Crowning with Thorns (Jn 19:1-2) 
\item Bearing of the Cross (Jn 19:16-17) 
\item Crucifixion (Jn 19:18-19) 
\item Laying in the Tomb (Jn 19:40-42) 
\item Resurrection (Jn 20:1-18) 
\end{enumerate}

\subsection*{Day 4: Thy will be done on earth as it is in Heaven}
\textsc{Meditation}: The last things.

\begin{itemize}
\item Mt 24:1-51 
\item Mt 25:1-46 
\item Mk 13: 1-37 
\end{itemize}
\begin{quotationx}
Thus the first is also the last, and the “first day of creation” is the Last Day, the day of universal resurrection.
Therefore the history of Christianity—moving in the direction of the Last Things, toward the
future—is at the same time the history of the reawakening of the past, i.e., the resurrection of
the total past, insofar as truth and love have dwelt therein. So gradually there will revive in Christendom the
forgotten, deeply sleeping, and perished treasures of wisdom and sacrificial deeds of the
past—right back to the primeval revelation and the paradisiacal state of humanity. Thus all truth
and all love of all times will have their home in the Church of Christ, which will then be the all-embracing (catholic)
unity of all things and all beings who are striving for timeless values—in the sense of realizing
the ideal of one Shepherd and one flock. 

\begin{flushright}\textit{Covenant of the Heart}\end{flushright}

\end{quotationx}
\subsection*{Day 5: Give us this day our daily bread}
\textsc{Meditation}: Institution of the Last Supper

\begin{quotationx}
Christ's last words at the institution of the holy sacrament at the Last Supper: “Do this in memory
of me” point towards the sacraments, too, as being a re-enlivening in the present of what happened in the past. In the
holy sacrament at the altar, memory becomes an act of the divine magic of transubstantiation, an act relating to the
real (not just remembered) presence of the body and blood of the Redeemer. What once took place, takes place now in the
present. In the sacrament, memory does not become a journey into the past, but instead a making-present of the past, an
evocation that summons something up out of the realm of forgetting, sleep, and death. Memory becomes the bearer of the
power which sounded forth in the call of the Master — “Lazarus, come forth!”
— a call that proved effective. Memory becomes divine magic, a miracle of great love and faith. In
this sense the words: “Do this in memory of me” actually mean: “Do this, so that I may be present”. For, one may add,
the Son of Man is Lord over time too. 

\begin{flushright}\textit{Covenant of the Heart}\end{flushright}

\end{quotationx}

\subsubsection*{Imagination}
\begin{multicols}{2}\small
Now as they were eating, Jesus took bread, and blessed, and broke it, and gave it to the disciples and said,
“Take, eat; this is my body.”  And he took a cup, and when he had given thanks he gave it to them, saying, “Drink of it,
all of you; for this is my blood of the covenant, which is poured out for many for the forgiveness of sins.  I tell you
I shall not drink again of this fruit of the vine until that day when I drink it new with you in my
Father's kingdom.” \flright{\itshape Mt 26:26-29}

And as they were eating, he took bread, and blessed, and broke it, and gave it to them, and said, “Take; this is
my body.” And he took a cup, and when he had given thanks he gave it to them, and they all drank of it. And he said to
them, “This is my blood of the covenant, which is poured out for many. Truly, I say to you, I shall not drink again of
the fruit of the vine until that day when I drink it new in the kingdom of God.” \flright{\itshape Mk 14:22-25} 

And when the hour came, he sat at table, and the apostles with him. And he took bread, and when he had given
thanks he broke it and gave it to them, saying, “This is my body which is given for you. Do this in remembrance of me.”
And likewise the cup after supper, saying, “This cup which is poured out for you is the new covenant in my blood. But
behold the hand of him who betrays me is with me on the table. For the Son of man goes as it has been determined; but
woe to that man by whom he is betrayed!” And they began to question one another, which of them it was that would do
this. \flright{\itshape Lk 22:14, 22:19-23} 
\end{multicols}

\subsubsection*{Inspiration}
\begin{multicols}{2}\small
I am not speaking of you all; I know whom I have chosen; it is that the scripture may be fulfilled,
`He who ate my bread has lifted his heel against me.' I tell you this now,
before it takes place, that when it does take place, you may believe that I am he. Truly, truly, I say to you, he who
receives any one whom I send receives me; and he who receives me receives him who sent me.” When Jesus had thus spoken,
he was troubled in spirit, and testified, “Truly, truly, I say to you, one of you will betray me.” The disciples looked
at one another, uncertain of whom he spoke. One of his disciples, whom Jesus loved, was lying close to the breast of
Jesus; so Simon Peter beckoned to him and said, “Tell us who it is of whom he speaks.” So lying thus, close to the
breast of Jesus, he said to him, “Lord, who is it?” Jesus answered, “It is he to whom I shall give this morsel when I
have dipped it.” So when he had dipped the morsel, he gave it to Judas, the son of Simon Iscariot. Then after the
morsel, Satan entered into him. Jesus said to him, “What you are going to do, do quickly.” Now no one at the table knew
why he said this to him. Some thought that, because Judas had the money box, Jesus was telling him, “Buy what we need
for the feast”; or, that he should give something to the poor. So, after receiving the morsel, he immediately went out;
and it was night. \flright{\itshape Jn 13:18-30}
\end{multicols}

\subsubsection*{The 3rd Temptation of Jesus: Turning stones into bread}
\begin{quotationx}
Then Jesus was led up by the Spirit into the wilderness to be tempted by the devil. And he fasted forty days and forty
nights, and afterward he was hungry. And the tempter came and said to him, “If you are the Son of God, command these
stones to become loaves of bread.” But he answered, “It is written, ‘Man shall not live by bread
alone, but by every word that proceeds from the mouth of God.'” \flright{\itshape Mt 4:1-4}
\end{quotationx}

\subsection*{Day 6: Forgive us our trespasses as we forgive those who trespass against us}
\textsc{Meditation}: The seven miracles in John's Gospel

\begin{quotationx}
The Gospel is proclaimed to us in events, signs, parables, and teachings. Thereby the events are simultaneously signs,
parables, and teachings. The signs, however, are also simultaneously events, parables, and teachings. The parables are
also events, signs, and teachings; and the teachings are at the same time events, signs, and parables. Everything in
the Gospel is event, sign, parable, and teaching, i.e., everything is fact, miracle, symbol, and revelation of the
truth. The miracles of the Gospels are thus also facts — as well as symbols and revelations of
truth.
\begin{flushright}\textit{Covenant of the Heart}\end{flushright}

\end{quotationx}
\begin{enumerate}
\item Wedding at Cana (John 2:1-11) 
\item Healing of nobleman's son (John 4:46-54) 
\item Healing of sick man at pool of Bethesda (John 5:1-10) 
\item Feeding of the five thousand (John 6:1-15) 
\item Jesus walks on the water (John 6:16-21) 
\item Healing of the man born blind (John 9:1-7) 
\item Raising of Lazarus (John 11:1-44) 
\end{enumerate}
\subsubsection*{The 2nd Temptation of Jesus: Casting down from the pinnacle of the temple}
Then the devil took him to the holy city, and set him on the pinnacle of the temple, and said to him, “If you are the
Son of God, throw yourself down; for it is written, `He will give his angels charge of
you,' and `On their hands they will bear you up, lest you strike your foot
against a stone.'” Jesus said to him, “Again it is written, `You shall not
tempt the Lord your God.'” (Mt 4:5-7)

\subsection*{Day 7: And lead us not into temptation}
\begin{quotationx}
Moses did not cross the Jordan, but crossed the threshold of death. His “promised land” lay on the other side of the
threshold of death. The people of Israel prepared themselves for the future encounter with the expected Messiah in the
promised land; Moses was granted this meeting in the disembodied state. It took place in the scene of the
Transfiguration on Mt. Tabor in the accompaniment of Elijah. Peter, John, and, James were present as witnesses to this
encounter. 

\begin{flushright}\textit{Covenant of the Heart}\end{flushright}

\end{quotationx}
Meditation:

\begin{itemize}
\item The transfiguration on Mount Tabor (Mt 17:1-9, Mk 9:2-28, Lk 9:28-36) 
\item Healing of the sick child (Lk 9:37-43) 
\end{itemize}
\subsubsection*{The 1st Temptation of Jesus: All the kingdoms of the world}
Again, the devil took him to a very high mountain, and showed him all the kingdoms of the world and the glory of them;
and he said to him, “All these I will give you, if you will fall down and worship me.”  Then Jesus said to him,
“Begone, Satan! for it is written, `You shall worship the Lord your God and him only shall you
serve.'” Then the devil left him, and behold, angels came and ministered to him. (Mt 4:8-11)

\subsection*{Day 8: But deliver us from Evil}
\begin{quotationx}
The only Son of the eternal Father nailed to the cross for our sake — this is what is divinely
impressed upon all open souls, including the robber crucified to the right. This impression is unforgettable and
inexpressible. It is the immediate breath of God which has inspired and still inspires thousands of martyrs, confessors
of the faith, virgins and recluses. 

\begin{flushright}\textit{Meditations on the Tarot. Letter IV: The Emperor}\end{flushright}

\end{quotationx}
\textsc{Meditation}: The seven Words from the Cross

\begin{enumerate}
\item Father, into thy hands I commend my spirit. (Lk 23:46) 
\item My God, my God, why hast thou forsaken me? (Mt 27:45-46, Mk 15:34) 
\item I thirst! (Jn 19:28) 
\item Verily I say unto thee, today shalt thou be with me in paradise. (Lk 23:43) 
\item Father, forgive them; for they know not what they do. (Lk 23:34) 
\item Woman, behold thy son! (Jn 19:26-27) 
\item It is finished. (Jn 19:30) 
\end{enumerate}
\textsc{Readings}:

\subsubsection*{Revelation 12:}
And there appeared a great wonder in heaven; a woman clothed with the sun, and the moon under her feet, and upon her
head a crown of twelve stars: And she being with child cried, travailing in birth, and pained to be delivered. And
there appeared another wonder in heaven; and behold a great red dragon, having seven heads and ten horns, and seven
crowns upon his heads. And his tail drew the third part of the stars of heaven, and did cast them to the earth: and the
dragon stood before the woman which was ready to be delivered, for to devour her child as soon as it was born. And she
brought forth a man child, who was to rule all nations with a rod of iron: and her child was caught up unto God, and to
his throne.

\subsubsection*{Revelation 13:}
And I stood upon the sand of the sea, and saw a beast rise up out of the sea, having seven heads and ten horns, and upon
his horns ten crowns, and upon his heads the name of blasphemy. And the beast which I saw was like unto a leopard, and
his feet were as the feet of a bear, and his mouth as the mouth of a lion: and the dragon gave him his power, and his
seat, and great authority. And I saw one of his heads as it were wounded to death; and his deadly wound was healed: and
all the world wondered after the beast.

\flright{\small\textit{Posted on 2023-02-13 by Cologero}}