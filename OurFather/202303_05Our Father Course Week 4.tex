\section{Week 4: Hallowed be thy name (II)}

\subsection*{Day 1: Eighth Beatitude: Christ impulse in the Manas}
Blessed are they which are persecuted for righteousness'sake: for theirs is the kingdom of heaven.
(Mt 5:10, Lk 6:20)

Turns suffering — the quality of the second Beatitude — into a creative
activity.

The evil in nature is to be called forth out of the realm of dream into the waking world.

\textsc{Temptation}: To proletarize Christianity, making it exoteric and severed from the Mysteries.

\emph{Manas} = ego consciousness or personal consciousness (\emph{Cf. Meditations on the Tarot. Letter XXII: The Fool}).

\begin{quotationx}
The higher emotional center is to be found at the level of the heart, and the higher intellectual center at the level of
the head. Their functions are different. In the Tradition they are sometimes called the eyes of the Soul. Thus, St
Isaac the Syrian said: “While the two eyes of the body see things in an identical way, the eyes of the Soul see
differently: one contemplates the truth in images and symbols, the other face to face.” 
\begin{flushright}\textit{Gnosis}\end{flushright}

Righteousness—that is, freedom and equality—remains an illusion when sought
within the realm of natural evolution with its extension in human history. It is simply not to be found
there —and never can be— because the “struggle for survival,” translated from
natural evolution to the arena of human history, has nothing to do with righteousness. Righteousness must be looked for
elsewhere, in another dimension … The fact of their persecution for righteousness' sake is
manifestly something unrighteous that befalls them in earthly life, but the share in the kingdom of heaven they attain
thereby reduces such unrighteousness to naught.
\begin{flushright}\textit{Covenant of the Heart}\end{flushright}

\end{quotationx}

\subsection*{Day 2: Ninth Beatitude: Christ impulse in the Buddhi}
Blessed are ye, when men shall revile you, and persecute you, and shall say all manner of evil against you falsely, for
my sake. Rejoice, and be exceeding glad: for great is your reward in heaven: for so persecuted they the prophets which
were before you. (Mt 5:10-12, Lk 6:22-23)

The quality of the first Beatitude to become pure in spirit is creatively transformed: to be a representative of Christ
on earth. One's own personality counts for nothing.

\textsc{Meditation}: This beatitude exposes the evil of sub-nature, which was hidden in deep sleep. In the Buddhi, or
higher intellectual center, the meditation is beyond images.

\textsc{Danger}: to turn Christianity into a principle of domination.

\emph{Buddhi} = consciousness of higher Self or cosmic consciousness (\emph{Cf. Meditations on the Tarot. Letter XXII: The
Fool}).

The compensatory balance lies in the vertical earth-heaven axis and not along the horizontal axis of terrestrial events.
It is the “reward in heaven” that illumines from above the unrighteousness endured in the current of evolution, driving
it away like a shadow before the light. This compensatory balance, in the sense of vertical or divine righteousness,
consists in the “reward in heaven,” that is, in the enrichment of humanity's being and not in the
punishment of the perpetrator of unrighteousness in accordance with the principle of horizontal justice: “An eye for an
eye, a tooth for a tooth”

\begin{quotationx}
However, just as the law of karma morally surpasses both determinism and retributive justice, so it in turn is surpassed
by the law of righteousness of the “kingdom of God” as expressed in the Sermon on the Mount. For the latter is
operative in the highest and most essential region of all. It heals the wounds of the heart suffered at the hand of
injustice and transforms the pain of unjustly inflicted suffering into everlasting bliss. At the same time, without
inflicting punishment, it leaves offenders to the tribunal of their own conscience—that is, to
their karma. \begin{flushright}\textit{Covenant of the Heart}\end{flushright}

\end{quotationx}
\subsection*{Day 3: Authentic experience of the spiritual world}
\begin{quotationx}
Human nature in ancient times was such that it was possible, without difficulty, to enable a man to partake in the
happenings of the spiritual world. Today it is very arduous, relatively speaking, to undergo the true esoteric training
leading to the attainment of clairvoyance. 
\begin{flushright} \textsc{Rudolf Steiner}. \emph{Meditation on the Beatitudes in The
Gospel of St. Matthew. Lecture IX}
\footnote{\url{https://wn.rsarchive.org/Lectures/GA123/English/RSP1965/19100909p02.html}}
\end{flushright}

Rudolf Steiner has certainly said things of a nature to awaken the greatest creative elan! His series of
\textbf{lectures on the four Gospels}, his lectures at Helsingfors and Dusseldorf on the celestial hierarchies
— without mentioning his book on the inner work leading to initiation (\emph{Knowledge of the
Higher Worlds.}) — would alone suffice to inflame a deep and mature creative enthusiasm in every
soul who aspires to authentic experience of the spiritual world. 
\begin{flushright}\textit{Meditations on the Tarot}\end{flushright}

\end{quotationx}
\subsection*{Day 4: The proclamation of the kingdom}
The Sermon on the Mount was not simply about new doctrines. Nor is it merely symbolic. Rather, it is the spiritual
activity of the Word. Specifically, it is not a matter of presenting new knowledge; it is supposed to change the level
of being of those who hear it.

The most important effect of the Sermon was a spiritual, suprasensory stimulation of the forces of the inner I,
transcending the physical, etheric and astral bodies (vegetative and sensitive souls).

\begin{quotationx}
This transformation, entailing the conversion of natural evolution to the good, leads above all to the replacement of
its guiding principle—the struggle for survival—by that of peace as the basis
of the new evolution. For this reason, it is said in the Sermon on the Mount that the peacemakers are called “sons of
God,” i.e., not only do they behold God but also take up his work, just as sons take up and continue the work of their
fathers. \begin{flushright}\textit{Covenant of the Heart}\end{flushright}

\end{quotationx}
The Sermon on the Mount, considered a historical event of universal significance, marked a turning point of evolution
after which the principle of peace is gradually coming to replace the principle of war. The path of transformed
evolution, in the case of individual human beings and of humanity as a whole, begins with a purification of the
impulses, instincts, habits, and customs attached to natural evolution. Then it leads on to
illumination—intuition of the truth, beauty, and goodness of divine evolution, the kingdom of God
and his righteousness. Finally, it culminates in the union of will, feeling, and thought with the will, feeling, and
thought that underlie divine evolution or the work of salvation. St. Bonaventure characterized this path of
purification (\emph{purgatio}), illumination (\emph{iliuminatio}), and perfection (\emph{perfectio}) in the simplest
and clearest possible way (\emph{De triplici via}, Prologus, I).

\subsection*{Day 5: The Beatitudes as the Seed of the Future of Humanity}
\textsc{Meditation}: Follow the stages of Christ's passion

First, our experiences with the three temptations in the desert.

Then, passing through the stages of the passion:

\begin{enumerate}
\item washing of the feet 
\item scourging 
\item crowning with thorns 
\item carrying of the cross 
\item crucifixion 
\item entombment 
\item resurrection 
\end{enumerate}
To be blessed does not mean an escape from suffering and pain, but the experience of a new kind of suffering and pain.
\includegraphics[width=3.577cm,height=5.122cm]{a20230305OurFatherCourseWeek4-img001.jpg} 

There will always be the fortunate and the unfortunate, but the blessed form a new class of human beings.

The fundamental conditions to be among the blessed: emptiness, inner poverty, and spiritual beggary.

Tomberg prefers the term “beggary” because the beggar knows he is spiritually poor.

Concentration exercise is on the Rose Cross in three stages:

\begin{enumerate}
\item Combine a series of images that evoke feelings into a single image and retained in consciousness. 
\item Consciously wipe away that picture, so that the only thing remaining as an object of meditation is the
soul's own activity that built and held the picture. 
\item Blot out this image-free content from awareness, leaving a perfectly empty consciousness. 
\end{enumerate}

\subsection*{Day 6: Knights of the Grail}
The physical body is representative of the stage where a complete reconciliation between heaven and Earth could be
brought about. There are three stages for the ego to unite:

\begin{itemize}
\item The ego is consciously united with the spirit 
\item The union includes the astral body 
\item The union goes beyond the etheric body to the physical body 
\end{itemize}
High moral and spiritual truths can be learned through the experiences of the physical body. The physical body was seen
as the communion body by those who understood the Grail tradition; it was the highest possibility of human communion
with the spirit.

\begin{quotationx}
The search for the Grail testifies that there has always existed a striving for a conscious participation in the logic
of the Logos, a quest for a Christian initiation. \begin{flushright}\textit{Covenant of the Heart}\end{flushright}

St. Bernard advanced not only active contemplation for the monks but also contemplative activity for the knights
— just as Krishna did more than fifteen centuries before him. The one and the other did so because
they knew that man is at one and the same time a contemplative and an active being, that “faith without works is
death”. 
\begin{flushright}\textit{Meditations on the Tarot. Letter XIV: Temperance}\end{flushright}

\end{quotationx}
\subsection*{Day 7: Guilt, Need, Care, Death}
\begin{quotationx}
The life of an initiate must forever be veiled in mystery and is a secret between man's heart and
his God. 

\begin{flushright}\textit{Count Germain}\end{flushright}

\end{quotationx}
The path to poverty of spirit has four elements, taken from \emph{Faust}:

\begin{enumerate}
\item \textsc{Guilt}. The experience of guilt. 
\item \textsc{Need}. The growing consciousness of what had been sacrificed and lost 
\item \textsc{Care}: The suggestions of a false future led to an awareness of care. 
\item \textsc{Death}. Death as the gate into the world of real life, not as a gate to the realm of false existence 
\end{enumerate}
Initiation needs to be understood in the light of guilt, need, care, and death. These can only be experienced in the
physical body. Some spiritual paths, e.g., Hinduism, had the goal of transcending the body in order to avoid the
experiences of guilt, need, care, and death. Moreover, the darkness of the physical body comes with conflict with evil.

Without the experience of guilt, there is no experience of the need for redemption, i.e., the Christ impulse.

Meditate on these stages. This may lead to the realization of the intuition stage of the knowledge of the spiritual
world, the goal of the First Beatitude.

\flright{\small\textit{Posted on 2023-03-05 by Cologero}}