\section{Week 2: Our Father who art in Heaven}

\subsection*{Day 1: First curse of the Father}
\textsc{Reading}: Gen 3: 1-24

Meditate on the connection between disobedience (verse 6) and the necessity of toil (verse 17).

So when the woman saw that the tree was good for food, and that it was a delight to the eyes, and that the tree was to
be desired to make one wise, she took of its fruit and ate; and she also gave some to her husband, and he ate. (Genesis
3:6)

And to Adam he said: Because you have listened to the voice of your wife, and have eaten of the tree of which I
commanded you, `You shall not eat of it,' cursed is the ground because of
you; in toil you shall eat of it all the days of your life. (Genesis 3:17)

\begin{quotationx}
The fruit of the Tree of Knowledge of Good and Evil has had a triple effect:

\begin{itemize}
\item Toil 
\item Suffering 
\item Death 
\end{itemize}

Toil or work took the place of mystical union with God, which union (without effort) is the teaching of the first
Arcanum of the Tarot, the Magician. The mystical spontaneity of the first Arcanum is that relationship between man and
God which was before the Fall. 
\begin{flushright}\textit{Meditations on the Tarot}\end{flushright}

\end{quotationx}
\textsc{First psychurgical operation}: Concentration without effort.

\subsection*{Day 2: Second curse of the Father}
\textsc{Reading}: Gen 3: 1-24

Meditate on the connection between the feeling of shame (verse 7) and the necessity of suffering (verse 16).

Then the eyes of both were opened, and they knew that they were naked; and they sewed fig leaves together and made
themselves aprons. (Genesis 3:7)

To the woman he said, “I will greatly multiply your pain in childbearing; in pain you shall bring forth children, yet
your desire shall be for your husband, and he shall rule over you.” (Genesis 3:16)

\begin{quotationx}
Suffering replaced the directly reflected revelation or gnosis, whose direct revelation is the teaching of the second
Arcanum of the Tarot, the High Priestess. The gnosis of the second Arcanum is that consciousness which was before the
Fall. 
\begin{flushright}\textit{Meditations on the Tarot. Letter III: The Empress}\end{flushright} 

\end{quotationx}
\textsc{Second psychurgical operation}: Vigilant inner silence

\subsection*{Day 3: Third curse of the Father}
\textsc{Reading}: Gen 3:1-24

Meditate on the connection between the feeling of fear (verse 10) and the necessity of death (verses 19-24).

And he said, “I heard the sound of thee in the garden, and I was afraid, because I was naked; and I hid myself.”
(Genesis 3:10)

you are dust, and to dust you shall return. Then the Lord God said, “Behold, the man has become like one of us, knowing
good and evil; and now, lest he put forth his hand and take also of the tree of life, and eat, and live
forever”— therefore the Lord God sent him forth from the garden of Eden, to till the ground from
which he was taken. He drove out the man; and at the east of the garden of Eden he placed the cherubim, and a flaming
sword which turned every way, to guard the way to the tree of life. (Genesis 3: 19-24)

\begin{quotationx}
Death entered into the domain of life or creative, sacred magic, which is the teaching of the third Arcanum of the
Tarot, the Empress. For sacred magic is that life which was before the Fall. \begin{flushright}\textit{Meditations on the Tarot. Letter III: The Empress}\end{flushright} 

\end{quotationx}
\textsc{Third psychurgical operation}: Inspired activity of imagination and thought.

\subsection*{Day 4: The spiritualization of toil}
\textsc{Reading}: Gen 3: 1-24

The spiritualization of toil in creative spiritual work and meditation. Through the Holy Spirit.

\begin{quotationx}
the Fall changed the destiny of humanity —so that mystical union became replaced by struggle or
toil, gnosis by suffering, and sacred magic by death. This is why the formula announcing the “good news” that the
effects of the Fall can be overcome and that the way of human evolution can return to that of mystical union instead of
struggle, that immediately reflected revelation or gnosis can replace the teaching of the truth through suffering, and
that sacred magic or transforming life can take the place of destructive death.
\begin{flushright}\textit{Meditations on the Tarot. Letter III: The Empress}\end{flushright} 
\end{quotationx}

\subsection*{Day 5: The transfiguration of suffering}
\textsc{Reading}: Gen 3: 1-24

The transfiguration in the soul of suffering by passing through purgatory. Through the Son.

\begin{quotationx}
Take the terms “limbo”, “purgatory” and “paradise” in their meaning as understood by analogy and you have a clear and
precise formula for the working of the magic of the sacred pentagram of five wounds; it effects a change from the
natural state (“limbo”) and from the state of human suffering (“purgatory”) to that of the blessedness of the divine
state (“paradise”). The operation of the magic of the sacred pentagram of five wounds therefore consists in
transforming the natural state into the human state and this latter into the divine state. This is the work of
spiritual alchemy of the transformation from Nature (“limbo”), and from the Human (“purgatory”), into the Divine
(“paradise”), according to the traditional threefold division — Nature, Man and God. 
\begin{flushright}\textit{Meditations on the Tarot. Letter V: The Pope}\end{flushright} 

\end{quotationx}
Concerning the experience relating to “purgatory”, it comprises all purging of suffering —physical,
psychic and spiritual. It is corporeal, moral and intellectual suffering which is our intermediate state between the
experience of the natural innocence of “limbo” and the moments of heavenly joy when the rays of “paradise” reach us.

\subsection*{Day 6: The transformation of death}
\textsc{Reading}: Gen 3: 1-24

The transformation of death into the ideal of initiation. Through the Father.

Initiation is the Second Birth that Jesus revealed to Nicodemus.

\begin{quotationx}
The rebirth from Water and Spirit which the Master indicates to Nicodemus is the re-establishment of the state of
consciousness prior to the Fall, where the Spirit was divine Breath and where this Breath was reflected by virginal
Nature. 
\begin{flushright}\textit{Meditations on the Tarot. Letter II: The High Priestess}\end{flushright} 

Neither did death then play the role of liberating consciousness, through the destruction of the forms which enclose it,
that it has played since the Fall. Instead of the destruction of forms, their continual transformation took place. This
was operated by the perpetual action of life effecting the metamorphosis of forms, in conformity with changes in the
consciousness using them. This perpetually liberating constructive action of life was—and still
is—the function of sacred or divine magic. And it is this transforming function, opposed to the
destructive function of death, that Moses' Genesis designates by the symbol of the Tree of Life.
\begin{flushright}\textit{Meditations on the Tarot. Letter III: The Empress}\end{flushright} 

\end{quotationx}

\subsection*{Day 7: Meditation on the Trinity}
Thine is the Kingdom, and the Power, and the Glory.

Then the righteous will shine like the sun in the kingdom of their Father. (Matthew 13:43)

\textsc{Glory}: The Rainbow is the imagination of the Holy Spirit.

\textsc{Power}: The radiant solar cross in the blue sky is the imagination of the Son.

\textsc{Kingdom}: Stars shining in the dark night sky, strewn with stars, the soul turning towards the Father,
accompanied by the words: Our Father, who art in Heaven.

\begin{quotationx}
This is the rainbow of seven colours of the manifestation of “glory” or mastership and also the octave of the seven
tones of revelation of the “name” or mission of the vanquisher of the three temptations. And this rainbow shone around
the empty and somber place in the wilderness where the temptations took place. 
\begin{flushright}\textit{Meditations on the Tarot. Letter VII: The Chariot}\end{flushright} 

\end{quotationx}

\flright{\small\textit{Posted on 2023-02-19 by Cologero}}