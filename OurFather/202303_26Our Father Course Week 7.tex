\section{Week 7: Thy Will be done (II)}

\textbf{Task: Work on the seven messages to the seven churches.}

And he laid his right hand upon me, saying: Fear not.

I am the First and the Last, and alive, and was dead, and behold I am living for ever and ever, and have the keys of
death and of hell. Write therefore the things which thou hast seen, and which are, and which must be done hereafter.
The mystery of the seven stars, which thou saw in my right hand, and the seven golden candlesticks.

The seven stars are the angels of the seven churches. And the seven candlesticks are the seven churches. (Rev 1:17-20)

The seven churches can also be imagined as representing different historical epochs. Each of the churches has a positive
and a negative element. These still live in every human being, regardless of the epoch. Our task is to develop the
positive elements and overcome the negative elements. These cultural periods are:

\begin{itemize}
\item \textbf{Ephesus}: The old Indian (Vedic) culture 
\item \textbf{Smyrna}: The old Persian (Zoroastrian) culture 
\item \textbf{Pergamos}: The Chaldean-Egyptian (Hermetic) culture 
\item \textbf{Thyatira}: The Greco-Roman (pagan) culture 
\item \textbf{Sardis}: The Anglo-Germanic (current) culture 
\item \textbf{Philadelphia:} The Slavic-Russian (coming) culture 
\item \textbf{Laodicea:} The American, i.e., Western Hemisphere, (future) culture 
\end{itemize}
There are three churches of the past, two of the present, and two of the future. But do not take the history and
geography too literally, for they represent streams that are always active in us.

Further reading: Valentin Tomberg's essays on the letters to the seven churches.

\begin{quotationx}
The arcanum of inspiration is of vital practical importance not only for Hermeticism but also for the spiritual history
of mankind in general. For just as in the individual human biography there are decisive moments of inspiration, so
there are in mankind's biography — which is history —
decisive points where far-reaching inspirations enter into the spiritual life of humanity. The great religions are such
inspirations. \begin{flushright} \emph{Meditations on the Tarot. Letter XIV: Temperance}\end{flushright}

\end{quotationx}

\subsection*{Day 1: Message to Ephesus}
\textbf{Reading: Matthew 24, Rev 2:1-7}

\emph{The first message to the church of Ephesus concerns the ancient Indian culture.}

Please refer to this letter for the complete hymn.

\begin{quotationx}
One finds a profound and breathtaking feeling of these cosmic depths in the cosmogonic hymn of the Rigveda. It awakens
in the meditator at least a feeling of the profundity of the fundamental cosmic incentive towards, or feeling for
zodiacality. \begin{flushright} \emph{Meditations on the Tarot. Letter XII The Hanged Man}\end{flushright}

\end{quotationx}

Unto the angel of the church of Ephesus write: These things saith he, who holdeth the seven stars in his right hand, who
walketh in the midst of the seven golden candlesticks (Rev 2:1)

The mystery of the seven stars, which thou sawt in my right hand, and the seven golden candlesticks. The seven stars are
the angels of the seven churches. And the seven candlesticks are the seven churches. (Rev 1:20)

I know thy works, and thy labour (toil), and thy patience, and how thou canst not bear them that are evil, and thou hast
tried them, who say they are apostles, and are not, and hast found them liars: And thou hast patience, and hast endured
for my name, and hast not fainted. But I have somewhat against thee, because thou hast left thy first charity. Be
mindful therefore from whence thou art fallen: and do penance, and do the first works. Or else I come to thee, and will
move thy candlestick out of its place, except thou do penance. (Rev 2:2-5)

But this thou hast, that thou hatest the deeds of the Nicolaites, which I also hate. (Rev 2:6)

\textbf{Negative Side}

The flight from earthly reality in the first (Old Indian) culture, in not wanting to fully incarnate.

\textbf{Positive Side}

Hatred of the Nicolaitans and the false apostles; i.e., either one-sided materialism or false conceptions of Christ.

He, that hath an ear, let him hear what the Spirit saith to the churches: To him, that overcometh, I will give to eat of
the tree of life, which is in the paradise of my God. (Rev 2:7)

\subsection*{Day 2: Message to Smyrna}
\textbf{Reading: Matthew 24, Rev 2:8-11}

\emph{The second message to the church of Smyrna concerns the Persian culture.}

\begin{quotationx}
It is in the Iranian and Judaeo-Christian spiritual currents — i.e., in Zoroastrianism, Judaism and
Christianity — that the idea and ideal of resurrection has taken root. The advent of the idea and
ideal of resurrection was “as lightning coming from the east and shining as far as the west” (Matthew xxiv, 27). The
inspired prophet of the East, namely the great Zarathustra in Iran, and the inspired prophets of the West
—Isaiah, Ezekiel and Daniel in Israel — announced it almost simultaneously.
\begin{flushright} \emph{Meditations on the Tarot. Letter XX: The Judgment}\end{flushright}

\end{quotationx}
And to the angel of the church of Smyrna write: These things saith the First and the Last, who was dead, and is alive: I
know thy tribulation and thy poverty, but thou art rich: and thou art blasphemed by them that say they are Jews and are
not, but are the synagogue of Satan. Fear none of those things which thou shalt suffer. Behold, the devil will cast
some of you into prison that you may be tried: and you shall have tribulation ten days. Be thou faithful until death:
and I will give thee the crown of life. (Rev 2:8-10)

\textbf{Negative Side}

The false “we” consciousness: prisoner in the synagogue of Satan or prisoner in one's own
subjective I-consciousness. “Jews” refers to refers to the souls that have decided to serve the Christ impulse. Hence
the synagogue of Satan (Ahriman) is a caricature of the Christian community. Or else the devil (Lucifer) will isolate
some in their subjective prison.

\textbf{Positive Side}

Faithfulness to the spirit and to the human task on earth. Those united by the Christ impulse freely form a true
community.

He, that hath an ear, let him hear what the Spirit saith to the churches: He that shall overcome, shall not be hurt by
the second death. (Rev 2:11)

\begin{quotationx}
The “good news” of Zarathustra was that the world and Man represented an admixture of two distinct world orders
— that of the principle of light and that of the principle of darkness, or the divine-archetypal
world order and that of natural evolution — and that the latter would in the end be vanquished by
Soshyans “who through will overcomes death”, to be followed by the resurrection of the dead. \begin{flushright} \emph{Covenant of the
Heart}\end{flushright}

\end{quotationx}

\subsection*{Day 3: Message to Pergamos}
\textbf{Reading: Matthew 24, Rev 2:12-17}

\emph{The third message to the church of Pergamos concerns the Egypto-Chaldean culture.}

\begin{quotationx}
The spiritual impulse behind the third (Egypto-Chaldean) cultural epoch, which has persisted in human souls since that
time, is to strive for the experience of immortal individuality and for harmony among immortal individual beings. \begin{flushright}
\emph{Christ and Sophia}\end{flushright}

\end{quotationx}
And to the angel of the church of Pergamus write: These things, saith he, that hath the sharp two edged sword: I know
where thou dwellest, where the seat of Satan is: and thou holdest fast my name, and hast not denied my faith. Even in
those days when Antipas was my faithful witness, who was slain among you, where Satan dwelleth.

But I have against thee a few things: because thou hast there them that hold the doctrine of Balaam, who taught Balac to
cast a stumbling block before the children of Israel, to eat, and to commit fornication: So hast thou also them that
hold the doctrine of the Nicolaites. In like manner do penance: if not, I will come to thee quickly, and will fight
against them with the sword of my mouth. (Rev 2:12-16)

\begin{quotationx}
This is what St. Justin said about certain Greek philosophers. Although the close inner connection between Alexandrian
theosophy and the Christian doctrine is one of the firmly established theses of Western scholarship, for one reason or
another, this perfectly correct thesis does not enjoy common acknowledgment in our theological literature. Therefore, I
consider it necessary to devote to this question a special appendix at the end of these lectures, where I will touch
upon the significance of the native Egyptian theosophy (the revelations of Thoth or Hermes Trismegistus) in its
relation to the doctrines of [the Logos and the Trinity]. \begin{flushright} \textsc{Vladimir Solovyov}, \emph{Lectures on Divine
Humanity}\end{flushright}

\end{quotationx}
\textbf{Negative Side}

Guidance is sought in the subconscious, out of the blood (Balaam) on the false magical path (Balac). The Nicolaites
strain place the ego inside the body, as a substitute for the Real Self.

\textbf{Positive Side}

\begin{quotationx}
Because in the depths of the unconscious — which knocks at the door and wants to become conscious
— there is present the “sanctuary of the everlasting zones”, where the “Sacred Book of Thoth”
remains deposited, from whence symbolic and Hermetic works are born, or reincarnate. The Tarot is such a work. \begin{flushright}
\emph{Meditations on the Tarot. Letter X: The Wheel of Fortune}\end{flushright}

\end{quotationx}
To retain the forces of the Manas, in order to preserve one's name, instead of sinking in the blood
bonds. The “name” refers to the immortal “I”; the spiritual impulse in this cultural epoch is to strive for the
experience of immortal individuality. The I is inscribed on a white stone as a force of community building.

He, that hath an ear, let him hear what the Spirit saith to the churches: To him that overcometh, I will give the hidden
manna, and will give him a white counter, and in the counter, a new name written, which no man knoweth, but he that
receiveth it. (Rev 2:17)

\begin{quotationx}
Then he (Saoshyant) shall restore the world, which will thenceforth never grow old and never die, never decay and never
perish, ever live and ever increase, and be master over its wish, when the dead will rise, when life and immortality
will come, and the world will be restored at God's wish. \begin{flushright} \textsc{Zoroaster}\end{flushright}

\end{quotationx}

\subsection*{Day 4: Message to Thyatira}
\textbf{Reading: Matthew 24, Rev 2:18-29}

\emph{The fourth message to the church of Thyatira concerns the Greco-Roman culture.}

\begin{quotationx}
The “pagan” initiates and philosophers knew of the unique God — the creator and supreme Good of the
world. The difference between the religion of the so-called “pagan” initiates and philosophers and that of Moses is
simply the fact that the latter made monotheism a popular religion, whilst the former reserved it for the elite, for
the spiritual aristocracy …  With respect to the cult of the “gods” and the iconolatry that this cult entailed, the
“pagan” initiates and philosophers saw in it the practice of theurgy, i.e., that of intercourse with entities of the
celestial hierarchies either by raising themselves to them, or by rendering possible their descent and presence on
earth … It goes without saying that the “paganism” of the initiates and sages, when not degenerated, had nothing to do
with the cult of collectively engendered demons. … Its “gods” were, truth to tell, human personages
— heroes and heroines, divinised or poerised, who were prototypes of the development of the human
personality, i.e., planetary and zodiacal types. Thus Jupiter, Juno, Mars, Venus, Mercury, Diana, Apollo, etc., were
not at all demons, but leading prototypes of the development of the human personality who, in their turn, corresponded
to cosmic — planetary and zodiacal — principles. … “naturalistic paganism”
was “cosmolatry”, i.e. it did not go beyond the limits of Nature like natural science today. It was, therefore,
“neutral” from the point of view both of the true spiritual world and of the demons. \begin{flushright} \emph{Meditations on the
Tarot. Letter XV: The Devil}\end{flushright}

\end{quotationx}

And to the angel of the church of Thyatira write: These things saith the Son of God, who hath his eyes like to a flame
of fire, and his feet like to fine brass. I know thy works, and thy faith, and thy charity, and thy ministry, and thy
patience, and thy last works which are more than the former.

But I have against thee a few things: because thou sufferest the woman Jezebel, who calleth herself a prophetess, to
teach, and to seduce my servants, to commit fornication, and to eat of things sacrificed to idols. And I gave her a
time that she might do penance, and she will not repent of her fornication. Behold, I will cast her into a bed: and
they that commit adultery with her shall be in very great tribulation, except they do penance from their deeds. And I
will kill her children with death, and all the churches shall know that I am he that searcheth the reins and hearts,
and I will give to every one of you according to your works.

But to you I say, and to the rest who are at Thyatira: Whosoever have not this doctrine, and who have not known the
depths of Satan, as they say, I will not put upon you any other burden. Yet that, which you have, hold fast till I
come. And he that shall overcome, and keep my works unto the end, I will give him power over the nations. And he shall
rule them with a rod of iron, and as the vessel of a potter they shall be broken, as I also have received of my Father:
and I will give him the morning star. (Rev 2:18-28)

\textbf{Negative Side}

\begin{quotationx}
The fourth form of paganism is that of the worship of collectively engendered demons. This form of paganism, which is
due to the degeneration of the other three forms is the only form of paganism where demons were engendered, worshipped
and obeyed, and which led to the whole of paganism being renamed unjustly and calumniously as the “demoniacal
religion”. \begin{flushright} \emph{Meditations on the Tarot. Letter XV: The Devil}\end{flushright}

\end{quotationx}

There is a confused life in Thyatira. Fornication is a serious transgression, is not compromise, it is a false surrender
to everything. It is the failure to keep to what one has said. The not wanting to say yes or no. Decadent false
prophets, false devotion to everything without wishing to make a choice. They get involved with Sybilline oracles
through Jezebel.

\textbf{Positive Side}

The force of the I, which has backbone in the spirit. The upright posture in the new spirit, the spirit of Christ. They
show the way to overcome the curses of toil (through service), suffering (through patience), and death (through faith).
And love (charity) is the fourth quality.

He that hath an ear, let him hear what the Spirit saith to the churches. (Rev 2:29)

\subsection*{Day 5: Message to Sardis}
\textbf{Reading: Luke 21, Rev 3:1-6}

\emph{The fifth message to the church of Sardis concerns the Anglo-Germanic culture.}

\begin{quotationx}
The historian of the future, if he has discerned the difference between the way, the truth, and the life on the one hand
and the stream of natural evolution on the other, will not compose a history of civilization —
that is, the story of technological progress and socio-political struggles — but will trace the
path of mankind through the stages of purification and illumination to its ultimate attainment of perfection. His
narrative will detail mankind's temptations and their vanquishment, the standards set by
particular individuals and groups, and the progressive lighting-up of new insights and the awakening of spiritual
faculties among human beings. \begin{flushright} \emph{Covenant of the Heart}
\end{flushright}
\end{quotationx}

And to the angel of the church of Sardis, write: These things saith he, that hath the seven spirits of God, and the
seven stars:

I know thy works, that thou hast the name of being alive: and thou art dead. Be watchful and strengthen the things that
remain, which are ready to die. For I find not thy works full before my God. Have in mind therefore in what manner thou
hast received and heard: and observe, and do penance. If then thou shalt not watch, I will come to thee as a thief, and
thou shalt not know at what hour I will come to thee.

But thou hast a few names in Sardis, which have not defiled their garments: and they shall walk with me in white,
because they are worthy. He that shall overcome, shall thus be clothed in white garments, and I will not blot out his
name out of the book of life, and I will confess his name before my Father, and before his angels. (Rev 3:1-5)

\textbf{Negative Side}

The church at Sardis was vice-ridden and restless. Remaining asleep to all that is spiritual, and affirmation of what is
fated to die. Without God everything is fated to die. Remaining asleep to this mission, just as our entire culture is
without God. Science, art, and religion, without God, are fated to die.

\textbf{Positive Side}

Through the forces of death, to experience an incentive to seek for life and for God. If one has found God and one
breathes in God, then one has life. Be wakeful and conscious of your task.

He that hath an ear, let him hear what the Spirit saith to the churches. (Rev 3:6)

\subsection*{Day 6: Message to Philadelphia}
\textbf{Reading: Luke 21, Rev 3:7-13}

\emph{The sixth message to the church of Philadelphia concerns the Slavic epoch.}

\begin{quotationx}
In this period, culture will be that of the Christ impulse flowing through all humankind — no
longer just a doctrine, but most of all a social force. This culture will have settlements in “all nations,” a bond of
friendly unity among humankind that binds nations and lands together all round the Earth. It will be the fruit of
adjusting the relationship between right and left in the spirit of the experience that results from the trial by
scourging. \begin{flushright} \emph{Christ and Sophia}\end{flushright}

\end{quotationx}
And to the angel of the church of Philadelphia, write: These things saith the Holy One and the true one, he that hath
the key of David; he that openeth, and no man shutteth; shutteth, and no man openeth:

I know thy works. Behold, I have given before thee a door opened, which no man can shut: because thou hast a little
strength, and hast kept my word, and hast not denied my name. Behold, I will bring of the synagogue of Satan, who say
they are Jews, and are not, but do lie. Behold, I will make them to come and adore before thy feet. And they shall know
that I have loved thee.

Because thou hast kept the word of my patience, I will also keep thee from the hour of the temptation, which shall come
upon the whole world to try them that dwell upon the earth. Behold, I come quickly: hold fast that which thou hast,
that no man take thy crown. He that shall overcome, I will make him a pillar in the temple of my God; and he shall go
out no more; and I will write upon him the name of my God, and the name of the city of my God, the new Jerusalem, which
cometh down out of heaven from my God, and my new name. (Rev 3:7-12)

\textbf{Negative Side}

This message points to a weakness of the will and fear of the world. One loses one's hold on things
and becomes passive. One's own will comes to a halt and falls asleep.

\textbf{Positive Side}

To stand like a pillar in the temple \&mdash; a stage of consciousness in the human being where he stands firmly in God
and in the name of Christ. The crown must be worn actively and with dignity, although the I is already resting in the
Christ. The crown is attained through effort.

He that hath an ear, let him hear what the Spirit saith to the churches. (Rev 3:13)

\subsection*{Day 7: Message to Laodicea}
\textbf{Reading: Luke 21, Rev 3:14-22}

\emph{The seventh message to the church of Philadelphia concerns the American epoch.}

\begin{quotationx}
The seventh cultural epoch (called Laodicea in the Apocalypse), on the other hand, will have as its main destiny the
fight between denial of the future, or hopelessness, and the Christian affirmation of the future's
resurrection, or hope. \begin{flushright} \emph{Christ and Sophia}\end{flushright}

\end{quotationx}
And to the angel of the church of Laodicea, write: These things saith the Amen, the faithful and true witness, who is
the beginning of the creation of God:

I know thy works, that thou art neither cold, nor hot. I would thou wert cold, or hot. But because thou art lukewarm,
and neither cold, not hot, I will begin to vomit thee out of my mouth. Because thou sayest: I am rich, and made
wealthy, and have need of nothing: and knowest not, that thou art wretched, and miserable, and poor, and blind, and
naked. I counsel thee to buy of me gold fire tried, that thou mayest be made rich; and mayest be clothed in white
garments, and that the shame of thy nakedness may not appear; and anoint thy eyes with eye salve, that thou mayest see.

Such as I love, I rebuke and chastise. Be zealous therefore, and do penance. Behold, I stand at the gate, and knock. If
any man shall hear my voice, and open to me the door, I will come in to him, and will sup with him, and he with me. To
him that shall overcome, I will give to sit with me in my throne: as I also have overcome, and am set down with my
Father in his throne. (Rev 3:14-22)

\textbf{Negative Side}

There is restlessness and arrogance in Laodicea. Feeling rich in the treasures gained in the past and resting upon
these, with no further striving. The opposite of begging for spirit. Neither cold nor warm, but lukewarm.

\textbf{Positive Side}

To strive continually onwards and in humility to give away even the highest one has attained, so that it can become an
organ for what is still higher. The greatest wealth, without knowledge of it. Renunciation of what one has attained and
to do so eagerly.

He that hath an ear, let him hear what the Spirit saith to the churches. (Rev 3:22)

\flright{\small\textit{Posted on 2023-03-26 by Cologero}}