\section{Guide to Meditative Reading}

\emph{Genesis}, the Gospels, and the \emph{Apocalypse} dominate the readings for the Our Father Course. Since those
books (and there are others) are understood by Valentin Tomberg to be composed of spiritual exercises, he gives us some
clues about how to read such books. Keep these points in mind as you do the meditative readings for day. In particular,
in the \emph{Meditations on the Tarot}, Tomberg explains how to read the Apocalypse as well as the Gospels. These
techniques will apply to the other readings as well.

\textsc{How to read the Apocalypse:}
You may come to a different understanding from Tomberg's commentary on the Course. But this is how
he suggests reading the Apocalpyse:

\begin{quotationx}
the “key” to the \emph{Apocalypse of St. John} is nowhere to be found… for it is not at all a matter of interpreting it
with a view to extracting a philosophical, metaphysical or historical system. The key to the Apocalypse is to practise
it, i.e., to make use of it as a book of spiritual exercises which awaken from sleep ever-deeper layers of
consciousness.

The \emph{seven letters to the churches}, the \emph{seven seals of the sealed book}, the \emph{seven trumpets} and the
\emph{seven vials} signify, all together, a course of spiritual exercises composed of twenty-eight exercises. For as
the \emph{Apocalypse} is a revelation put into writing, it is necessary, in order to understand it, to establish in
oneself a state of consciousness which is suited to receive revelations.

\begin{enumerate}
\item
It is the state of concentration without effort (taught by the first Arcanum), 
\item
followed by a vigilant inner silence (taught by the second Arcanum), 
\item
which becomes an inspired activity of imagination and thought, where the conscious self acts together with
superconsciousness (teaching of the third Arcanum). 
\item
Lastly, the conscious self halts its creative activity and contemplates — in letting pass in review — everything which preceded, with a view to summarising it (practical teaching of the fourth
Arcanum). 
\end{enumerate}
The mastery of these four psychurgical operations, symbolised by “The Magician”, The High Priestess”, “The Empress” and
“The Emperor”, is the key to the Apocalypse. One will search in vain for another. 

\end{quotationx}

\textsc{How to read the Gospels:}
This requires the use of the imagination (step 3 above) to properly read the Gospels. This is so radically different
from the scholarly techniques of textual analysis, etc., which mostly desiccate the texts.

\begin{quotationx}
The Gospels, likewise, are spiritual exercises, i.e., one has not only to read and re-read them, but also to plunge
entirely into their element, to breathe their air, to Participate as an eye-witness, as it were, in the events
described there — and all this not in a scrutinising way, but as an “admirer”, with ever-growing
admiration. 

\end{quotationx}
\textsc{The Higher Self:}
I recently listened to a super-correct theologian who objected to Hermetic teaching, calling the idea of raising
one's state of consciousness “satanic”. It is best to leave such types at peace, but, for us, we
follow St Bonaventura, St Augustine, John Climacus, Dante, St John of the Cross, and many others who documented the
ascent to God. I have documented their teachings on my blog; I do not create new doctrines. Tomberg uses traditional
Hermetic terms to represent these higher states. Thomas Aquinas claimed we cannot know God in his essence in this life,
yet we can go as far as we are capable of. This is how Tomberg describes it in the \emph{Meditations on the Tarot}:

\begin{quotationx}
The transcendental Self is not God. It is in his image and after his likeness, according to the law of analogy or
kinship, but it is not identical with God. There are still several degrees on the ladder of analogy which separate it
from the summit of the ladder from God. These degrees which are higher than it are its “stars”, or the ideals to which
it aims. The Apocalypse specifies the number of them: there are twelve degrees higher than that of the consciousness of
the human transcendental Self. It is necessary, therefore, in order to attain to the ONE God, to elevate oneself
successively to degrees of consciousness of the nine spiritual hierarchies and the Holy Trinity. 

\end{quotationx}

\flright{\small\textit{Posted on 2022-10-21 by Cologero}}
