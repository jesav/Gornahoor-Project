\section{Week 9: Forgive Us our Trespasses}
 

As we forgive those who trespass against us.

\begin{quotationx}
Each miracle of the Gospel of St. John is simultaneously a \textbf{teaching}, a \textbf{parable}, a \textbf{sign}, and
an \textbf{event}. \begin{flushright} \emph{Covenant of the Heart}\end{flushright}

\end{quotationx}
\textbf{Meditation}: The seven miracles in John's gospel

These represent 7 stages of healing: the 7 stages of guilt and their forgiveness.

\begin{quotationx}
\textbf{Creation and healing have to take place in reverse sequence}: there, where the creation was completed, lies the
starting point for the healing (i.e., restoring) effect. Accordingly, the prologue of the Gospel of St. John also gives
the stages of Moses' account of creation (\textbf{\emph{light, life,
man}}) in the reverse sequence: “In him (the Word) was \textbf{\emph{life, and the life was the
\textbf{light} of \textbf{men}}}” (John 1:1-4). The work of salvation takes place in reverse to that of the creation in
so far as the last stage of the creation is the first of the work of salvation. \begin{flushright} \emph{Covenant of the Heart}\end{flushright}

\end{quotationx}

\subsection*{Day 1: The Wedding at Cana}
On the third day there was a wedding in Cana in Galilee, and the mother of Jesus was there. Jesus and his disciples were
also invited to the wedding. When the wine ran short, the mother of Jesus said to him, “They have no wine.” And Jesus
said to her, “Woman, how does your concern affect me? My hour has not yet come.” His mother said to the servers, “Do
whatever he tells you.” Now there were six stone water jars there for Jewish ceremonial washings, each holding twenty
to thirty gallons. Jesus told them, “Fill the jars with water.” So they filled them to the brim. Then he told them,
“Draw some out now and take it to the headwaiter.” So they took it. And when the headwaiter tasted the water that had
become wine, without knowing where it came from (although the servers who had drawn the water knew), the headwaiter
called the bridegroom and said to him, “Everyone serves good wine first, and then when people have drunk freely, an
inferior one; but you have kept the good wine until now.” Jesus did this as the beginning of his signs in Cana in
Galilee and so \emph{revealed his glory, and his disciples began to believe in him}. \begin{flushright} John 2:1-11\end{flushright}

\begin{itemize}
\item \textbf{Guilt}: doubt 
\item \textbf{Sickness}: blood which had become lukewarm after the Fall 
\item \textbf{Forgiveness}: It is made fresh and vigorous. At the wedding, a new mixing of blood was taking place 
\item \textbf{Name of the Master}: I am the True Vine. 
\end{itemize}
\begin{quotationx}
The wine at the wedding of Cana was not \emph{created from} nothing, but rather it was \emph{water} which was changed
into wine. Let us also note the fact that the Virgin-Mother was not only present at the wedding but also that she took
part in an explicit manner in the miracle of changing the water into wine —since it was thanks to
her initiative that the miracle took place. \begin{flushright} \emph{Meditations on the Tarot. Letter X: Force}\end{flushright}

\end{quotationx}
The healing miracle work of the Word made flesh takes place in the reverse sequence to the creative miracle working of
the divine Word depicted in Genesis. The divine magic of the seventh day becomes the first healing miracle of the Word
made flesh, that of the wedding at Cana.

The free union, in love between God and the world, which was celebrated, blessed, and consecrated on the seventh day of
creation, became broken off through the Fall. The world was unfaithful toward God. And since this divine cosmic union
is mirrored in the marriage relationship, for which it is the ideal and archetype as well as being the meaning of
marriage, the original sickness of the world consists in the breach of the free love union that existed between God and
the world. Analogously, this is mirrored in human life, in the distortion and degeneration of the nature and experience
of marriage. The marriage relationship — as it has become — begins with
enthusiasm, with the “wine” of the honeymoon period, and ends with the “water” of routine habit.

\begin{quotationx}
When there was no more wine, Jesus transformed water into wine, and the second wine was better than the first. Thereby
the miracle of the wedding at Cana was the “sign,” the symbol, and the event of the healing of marriage (healing in the
service of restoring the marriage relationship to correspond to its divine cosmic archetype, which is the seventh day
of creation). \begin{flushright} \emph{Covenant of the Heart}\end{flushright}

\end{quotationx}
\subsection*{Day 2: Healing of the Nobleman's Son}
He came again therefore into Cana of Galilee, where he made the water wine. And there was a certain ruler, whose son was
sick at Capharnaum. He having heard that Jesus was come from Judea into Galilee, went to him, and prayed him to come
down, and heal his son; for he was at the point of death.

Jesus therefore said to him: Unless you see signs and wonders, you believe not.

The ruler saith to him: Lord, come down before that my son die.

Jesus saith to him: Go thy way; thy son liveth. The man believed the word which Jesus said to him, and went his way.

And as he was going down, his servants met him; and they brought word, saying, that his son lived. He asked therefore of
them the hour wherein he grew better. And they said to him: Yesterday, at the seventh hour, the fever left him. The
father therefore knew, that it was at the same hour that Jesus said to him, Thy son liveth; and himself believed, and
his whole house. This is again the second miracle that Jesus did, when he was come out of Judea into Galilee.  \begin{flushright}
John 4:46-54\end{flushright}

\begin{itemize}
\item \textbf{Guilt}: desire 
\item \textbf{Sickness}: the sin of the change of heredity from the vertical to the horizontal 
\item \textbf{Forgiveness}: The father bore the guilt for the son's illness. The father believed
and the son was healed. 
\item \textbf{Name of the Master}: I am the Way, the Truth, the Life. 
\end{itemize}
\textbf{The sixth day of Creation}

The sixth day of creation in Genesis is an account of the origin of the animal kingdom and the human being,
corresponding to their archetype:

\begin{itemize}
\item The archetype of the human being is God: God created humans in God's own image, in the image
of God, God created them. 
\item The animals were brought forth from the earth formed by God, each according to its kind. 
\end{itemize}
\textbf{Heredity}

Heredity means the transmission of similarity from ancestors to their descendants. The archetypal nature of heredity is
vertical: the archetype is above and the form mirroring it is shaped below, on earth. Thus, the invisible divinely
created archetypes are the “ancestors” of the visible species of animals. And the invisible archetype of man, the
divine being itself, is the “ancestor” of the human being.

The sickness that arose as a tragic consequence of the Fall was a change of direction in the mirroring process of
heredity; it changed from being vertical to become horizontal. This meant that similarity no longer descended from the
invisible supratemporal archetype above, but from the visible ancestors in the temporal succession of generations here
below.

\textbf{The Sin of Heredity}

The \emph{sickness} that arose as a tragic consequence of the Fall was a change of direction in the mirroring process of
heredity; it changed from being vertical to become horizontal. This meant that similarity no longer descended from the
invisible supratemporal archetype above, but from the visible ancestors in the temporal succession of generations here
below.

Instead of becoming the direct “image and likeness” of their archetype, human beings and other beings of nature became
formed in the “image and likeness” of their earthly ancestors, thus only indirectly mirroring their archetype. Thereby
heredity, as we know it, became a horizontal stream in the sequence of time, transmitting not only the original
mirroring of the archetype, but also everything that entered into the stream of generations with the Fall and that has
occurred in this stream since the Fall. It has become a stream that also transmits the “sins” of sickness and death,
and through it the “sins of the fathers” have become a reality.

\textbf{The Healing}

The second miracle of the Gospel of St. John, the healing of the nobleman's son, where the healing
of the son took place through the faith of the father, was fulfilled, like the first, at Cana in Galilee. It comprises
the transformation of the relationship father—son (i.e., heredity) from being a stream
transmitting sickness to a stream transmitting healing. The second miracle of the John Gospel is the event, sign, and
teaching, which has to do with the divine archetypal heredity of the sixth day of creation. It has to do with the
distortion of this “vertical” heredity through the Fall, in that the original relationship to the “image and likeness
of God,” as it was on the sixth day of creation, was restored by the father bringing his son into a direct relationship
to the divine archetype—through his faith in Jesus Christ, the new Adam.

\begin{quotationx}
He brought his son into connection with the new Adam, who, in place of himself, entered into the hereditary relationship
“father-son.” Thus the healing of the nobleman's son took place. \begin{flushright} \emph{Covenant of the Heart}\end{flushright}

\end{quotationx}
\subsection*{Day 3: Healing of the Paralyzed Man}
After these things was a festival day of the Jews, and Jesus went up to Jerusalem. Now there is at Jerusalem a pond,
called Probatica, which in Hebrew is named Bethsaida, having five porches. In these lay a great multitude of sick, of
blind, of lame, of withered; waiting for the moving of the water. And an angel of the Lord descended at certain times
into the pond; and the water was moved. And he that went down first into the pond after the motion of the water, was
made whole, of whatsoever infirmity he lay under. And there was a certain man there, that had been eight and thirty
years under his infirmity. Him when Jesus had seen lying, and knew that he had been now a long time, he saith to him:
Wilt thou be made whole?

The infirm man answered him: Sir, I have no man, when the water is troubled, to put me into the pond. For whilst I am
coming, another goeth down before me. Jesus saith to him: Arise, take up thy bed, and walk. And immediately the man was
made whole: and he took up his bed, and walked. And it was the sabbath that day. \begin{flushright} John 5:1-9\end{flushright}

\begin{itemize}
\item \textbf{Guilt}: cutting oneself from the cosmic current 
\item \textbf{Sickness}: paralysis 
\item \textbf{Forgiveness}: The restoration of movement 
\item \textbf{Name of the Master}: I am the door 
\end{itemize}
\textbf{The fifth day of Creation}

The fifth day of creation in Genesis is the account of the waters bringing forth ensouled movement:

\begin{itemize}
\item in the \textbf{horizontal} direction: swarms of living creatures 
\item in the vertical direction: birds that fly above the earth across the firmament of the heavens 
\end{itemize}
\textbf{The Healing}

The result of the third miracle in the Gospel of St. John is the healing of the man who was paralyzed for thirty-eight
years, i.e., the restoration of ensouled faculty of movement to the paralyzed man, who lay there waiting for healing
through the water brought into movement by an angel.

The words of Jesus --\textit{Arise, take up thy bed and walk}-- create ensouled movement in:

\begin{itemize}
\item The vertical direction: \emph{arise} 
\item The horizontal direction: take up they bed and \emph{walk} 
\end{itemize}
The faculty of movement is essentially cosmic, not only according to its effect, where every movement, even the
slightest, exerts an effect ultimately upon the whole world, but also according to the causes stimulating it. For the
human being stands within a stream of cosmic energies—his thoughts in the streams of the thought
world, his feelings in the streams of the world's psychic forces, and his impulses of will are
immersed in the streams of world will energy and are “plugged in” to them.

Someone who cuts himself off from the streams of cosmic energies becomes paralyzed.

\begin{quotationx}
The third miracle of the John Gospel accomplished the “re-plugging in” of this human being into the ensouled movement of
the world, who, through sin, had become cut off from it and thereby paralyzed. The third miracle is the archetype of
the healing effect of the sacrament of the forgiveness of sins, i.e., penance. \begin{flushright} \emph{Covenant of the Heart}\end{flushright}

\end{quotationx}
\subsection*{Day 4: Feeding of the Five Thousand}
After these things Jesus went over the sea of Galilee, which is that of Tiberias. And a great multitude followed him,
because they saw the miracles which he did on them that were diseased. Jesus therefore went up into a mountain, and
there he sat with his disciples. Now the pasch, the festival day of the Jews, was near at hand.

When Jesus therefore had lifted up his eyes, and seen that a very great multitude cometh to him, he said to Philip:
Whence shall we buy bread, that these may eat? And this he said to try him; for he himself knew what he would do.

Philip answered him: Two hundred pennyworth of bread is not sufficient for them, that every one may take a little.

One of his disciples, Andrew, the brother of Simon Peter, saith to him: There is a boy here that hath five barley
loaves, and two fishes; but what are these among so many?

Then Jesus said: Make the men sit down. Now there was much grass in the place. The men therefore sat down, in number
about five thousand. And Jesus took the loaves: and when he had given thanks, he distributed to them that were set
down. In like manner also of the fishes, as much as they would. And when they were filled, he said to his disciples:
Gather up the fragments that remain, lest they be lost.

They gathered up therefore, and filled twelve baskets with the fragments of the five barley loaves, which remained over
and above to them that had eaten. Now those men, when they had seen what a miracle Jesus had done, said: This is of a
truth the prophet, that is to come into the world. \begin{flushright} John 6:1-14\end{flushright}

\begin{itemize}
\item \textbf{Guilt}: egoism 
\item \textbf{Sickness}: cutting oneself from the community 
\item \textbf{Forgiveness}: The reinstatement of the original community of beings 
\item \textbf{Name of the Master}: I am the bread of life 
\end{itemize}
\textbf{Fourth Day of Creation}

The fourth day of creation “stands before” the “play” of the many kinds of spontaneous movement of the fifth day and
“leads” it. For the fourth day of creation is that of the coming into being of those principles of the world orchestra
that direct “time and tempo” — the creation of the “sun, moon, and stars”: “And God made two great
lights: the greater light to rule the day and the lesser light to rule the night; he made stars also; and God set them
in the firmament of the heavens to separate the day from the night and let them be for signs and for seasons and for
days and years”. What are these other than organs of direction, i.e., conductors of time and tempo for the world
orchestra, in accordance with the music score of the stars?

The fourth day of creation is the genesis of that all-embracing world rhythm, in which all beings partake and that
unifies them into a world-embracing community. Consider human consciousness. It does not become chaotic through the
strife of wishes, desires, whims, moods, notions, and countless impulses from without and
within—from fantasy and from memory. Rather, it arranges itself around a central point, the self,
which represents the center of gravity of the soul life, i.e., the permanency of the identity of the personality.

Moreover, aided by the light of reason it works in such a way as to bring order even into the “night” of the
subconscious, leading the whole soul life (conscious and unconscious, or “day” and “night”) in the direction of the
ideals. (Ideals can be likened to “stars,” enabling orientation and pointing the way.)

This “sun” leads and supports the whole—the “day world” and also the “night
world”—in harmony with the world of stars. The sun (or rather the inner nature of the sun) in the
great world corresponds to the creative, leading, and ordering role of the self in the “small world” of the human soul.
The moon in its inner nature corresponds to the rational capacity for reflection, which casts an evaluating light on
the irrational urges of the soul life. The inner nature of the stars in the great world correspond to the ideals that
give direction to human soul life.

The fourth day of creation is the account of the origin of the universal community, of the unity embracing all beings of
the world. It is therefore the divine cosmic archetype of the sacrament upon the altar, that of holy communion.

\textbf{The Fourth Miracle}

The feeding of the five thousand in the wilderness —is the corresponding healing work of the Word
made flesh. This work consisted in the reinstatement of the original community of beings from the fourth day of
creation, and the corresponding ordering given on this day of creation by the sun, moon, and stars. For as the Sun
— raying out light, warmth, and life — “nourishes” all beings and unites them
in a “community of nourishment,” so Jesus Christ functioned at the feeding of the five thousand as the “nourishment
giving center” for the five thousand.

\subsection*{Day 5: Jesus Walks on Water}
The fifth miracle of St. John's Gospel is the “sign” of individual generative power as it came into
being on the third day of creation.

The third day of creation is the “day” of procreation, of the mystery of seed and of growth.

And when evening was come, his disciples went down to the sea. And when they had gone up into a ship, they went over the
sea to Capernaum; and it was now dark, and Jesus was not come unto them.  And the sea arose, by reason of a great wind
that blew. When they had rowed therefore about five and twenty or thirty furlongs, they see Jesus walking upon the sea,
and drawing nigh to the ship, and they were afraid.

But he said to them: \textbf{It is I; be not afraid}. They were willing therefore to take him into the ship; and
presently the ship was at the land to which they were going. \begin{flushright} John 6:16-21\end{flushright}

\begin{itemize}
\item \textbf{Guilt}: lack of faith 
\item \textbf{Sickness}: separation from the instinctual life, the power of the seed principle 
\item \textbf{Forgiveness}: the restoration of pure faith, unsupported by anything but inner certainty 
\item \textbf{Name of the Master}: I am the good shepherd 
\end{itemize}
\textbf{Third Day of Creation}

And God said: Let the earth bring forth grass, plants yielding seed, and fruit trees each bearing fruit after its kind,
whose seed is in itself upon the earth; and it was so. And the earth brought forth grass, and plants yielding seed each
according to its kind, and trees bearing fruit, whose seed was in itself after its kind. \begin{flushright} Genesis 1:11\end{flushright}

\textbf{The third day of creation is the generation of the seed principle, of the principle of potential formative force
becoming actualised and bringing to visible realisation its own inner, invisible shape}. The third day of creation is
the coming into being of the seed principle in the world—that is, not only of the plant world
manifest to us as plants and trees, but also such “trees” growing in paradise as “the tree of knowledge of good and
evil” and the “tree of life.” Also included here is all that grew out of the “seed of Abraham,” and that in the New
Testament that, as the Kingdom of God, was sown as “seed” and shall in the future become a mighty tree. The words of
Jesus, too, that fell on hard or on good ground and bear fruit accordingly—all these belong in the
realm of the seed principle that came into being on the third day of creation.

The language of the Bible is neither merely realistic, nor merely symbolic. It is real symbolic, i.e., instead of
employing abstract concepts it makes use of real facts, each corresponding to a principle embodying a basic and
essential truth. Therefore the trees of which “each bore fruit after its kind, whose seed was in itself” were not just
ash trees and oak trees, but also the tree of knowledge and the tree of life. And the fruit of the tree of knowledge
bears the seed of death; the tree of life, on the other hand, bears the seed of an uninterrupted metamorphosis of
growth.

\begin{quotationx}
The biblical words: seed, trees, the sea, and the dry land, as also heaven and earth, mean much more than the concrete
things designated by these words. The “seas” that in Genesis' account of the third day of creation
are described as “the gathering together of the waters under the heavens unto one place” signify the state of
concentration (“gathering”) of force substance, whereby it retains its mobility in the sense of being able to be moved.
On the other hand, the “dry land” means a state of still greater concentration (“gathering unto one place”), where the
force substance is so condensed that a coagulation into solidity is reached. \begin{flushright} \emph{Covenant of the Heart}\end{flushright}

\end{quotationx}
\textbf{The Fifth Miracle}

\textbf{The miracle of the walking on the water—as an event and as a “sign”—is
the revelation of the independence of the Son of Man with regard to the sea and the land}. Jesus Christ walking on the
sea needs no support, for he supports himself.

The Word revealed by the walking on the water is the creative Word of the third day of creation. It is the Word that
lies at the basis of the seed principle. Just as the seed determines the future ways and stages of growth, so the Good
Shepherd decides the ways of development of the true being of humanity and leads it along these ways.

\begin{quotationx}
The third day of creation is the divine cosmic background of the sacrament of confirmation. \begin{flushright} \emph{Covenant of the
Heart}\end{flushright}

\end{quotationx}
Peter answered him: Lord, if it is you. bid me come to you on the water. He said: Come! So Peter got out of the boat and
walked on the water and came to Jesus. But when he saw the wind, he was afraid, and beginning to sink he cried our:
Lord, save me! Jesus immediately reached out his hand and caught him, saying to him: O man of little faith, why did you
doubt? \begin{flushright} Matthew 14:28-31\end{flushright}

Fear is due to the menace of being engulfed by elemental forces of gravitation of a lower order, i.e. of being carried
away by the play of blind forces from the agitated “sea” of the “electrical field” of death. “I am; be not afraid” is
therefore the message of the centre, or Master, of celestial gravitation.

There is another field of gravitation than that of death, and he who unites himself with it can walk on water, i.e.,
transcend the agitated element of “this world”, the electrical gravitational field of the serpent.

\begin{quotationx}
He got out of the boat, which means — in view of all the laws of reason and memory
— that he was taken out of the domain of ordinary consciousness, i.e., that of reason, memory, and
sense perception, and he walked on the water, drawn by Jesus. He experienced, therefore, the same elevation of the soul
which draws the body up after it as spoken of by St. Teresa. \begin{flushright} \emph{Meditations on the Tarot. Letter XII: The
Hanged Man}\end{flushright}

\end{quotationx}
\subsection*{Day 6: Healing of the Man born Blind}
The basis of man's destiny is the decision to renounce the sight that is blind to the revelations
of the Divine, and to wait for the miracle of the reinstatement of the true, original faculty of beholding that sees
all things visible as “works of God revealed.”

The second day of creation is the creation of the firmament, dividing the waters that were under the firmament from the
waters that were above the firmament.

And Jesus passing by, saw a man, who was blind from his birth. And his disciples asked him: Rabbi, who hath sinned, this
man, or his parents, that he should be born blind?

Jesus answered: Neither hath this man sinned, nor his parents; but that the works of God should be made manifest in him.
I must work the works of him that sent me, whilst it is day: the night cometh, when no man can work. As long as I am in
the world, I am the light of the world.

When he had said these things, he spat on the ground, and made clay of the spittle, and spread the clay on his eyes, and
said to him: Go, wash in the pool of Siloam, which means “Sent”. He went therefore, and washed, and he came seeing.
\begin{flushright} John 9:1-7\end{flushright}

\begin{itemize}
\item \textbf{Guilt}: shame 
\item \textbf{Sickness}: separation from the light 
\item \textbf{Forgiveness}: the true, original vision is restored 
\item \textbf{Name of the Master}: I am the light of the world 
\end{itemize}
\textbf{Second Day of Creation}

The second day of creation is the creation of the firmament, dividing the waters that were under the firmament from the
waters that were above the firmament. “And God called the firmament heaven.” \textbf{It is the “day” of the coming into
being of “vision,” of knowledge, of true gnosis}. For the “firmament” or “heaven” that divides the waters above from
those below is the “light” of the first day of Genesis — which now not only divides the waters
above from those below, but is also the connecting link between them.

The waters below mirror the heavens, and likewise the waters above mirror them. This double reflection is the principle
of vision, of knowledge itself, for knowledge is the transformation of the seen into insight, of what is perceived into
truth. It is the realization of the connection between things and ideas, between the real and the ideal.

Now, the reflection of the heavens in the waters below is the realm of things (“the real”), while ideas (“the ideal”)
are represented by the reflection of the heavens in the waters above. And knowledge, or understanding, is the process
of relating the real to the ideal corresponding to it. The active, creatively functioning light (of “heaven”) is
reflected both above as ideas and also below as realities.

\begin{quotationx}
The \textbf{light} that forms the firmament of heaven and that is reflected both in the waters above the firmament and
also in the waters below the firmament is the Logos, the Word, by which all things were made. \begin{flushright} \emph{Covenant of
the Heart}\end{flushright}

\end{quotationx}
\textbf{Sixth Miracle}

The Word bore witness to himself through the lips of Jesus Christ at the sixth miracle of St.
John's Gospel — the healing of the man born blind — “As
long as I am in the world, I am the \textbf{light} of the world”

The Logos is not only the intelligence of the world, that is, the connection of the ideal with the real, but also the
perception of the ideal and the real. All seeing, hearing, touching, etc., presupposes an intermediate connecting link,
an organ, between the percept (the object of perception) and the perceiver (the subject of perception).

The miracle of the healing of the man born blind is a pure act of God. It does not presuppose a willingness to repent,
nor any act of faith on the part of the man born blind. For neither had the man born blind sinned in his life before
birth in the spiritual world or in a previous life on earth, nor had the sin of his parents caused his blindness. The
cause lay in the future. He was born blind “that the works of God might be made manifest in him.”

Through the Fall vision became horizontal, i.e., the “lower eye” became the sole organ of seeing, then “Adam and Eve saw
that they were naked.” This means that vision became devoid of ideas, a perception of the “bare facts,” i.e., of facts
alone, without the corresponding ideas in the waters above the firmament. It became basically cynical.

It was this vision, limited to the bare facts, which the man born blind had renounced at the time of his pre-birth
choice of destiny, in order that the true, original vision—the vision of the real combined with
the ideal, as prefigured on the second day of creation—might be reinstated in him.

\begin{quotationx}
The \textbf{sixth miracle} is the archetype of the sacrament of ordination, the sacrament of the regeneration of
original and true vision. \begin{flushright} \emph{Covenant of the Heart}\end{flushright}

\end{quotationx}
\subsection*{Day 7: The Raising of Lazarus}
\textbf{Reading}: John 11:1-44

The first day of creation is the genesis of the world in its seed condition, i.e., with all its latent possibilities.

The Lazarus miracle is that of the calling forth of light out of darkness.

\begin{itemize}
\item \textsc{Guilt}: losing consciousness 
\item \textsc{Sickness}: excess of spirituality 
\item \textsc{Forgiveness}: all the previous miracles 
\item \textsc{Name of the Master}: I am the resurrection and the life 
\end{itemize}
\textbf{First Day of Creation}

The “days” that follow the first are those of unfolding and actualizing in detail what had been created as a foundation
on the first day. The account of the following days represents a kind of explanatory commentary on the first day of
creation. The miracle of the first day embraces and contains in itself the whole miracle of creation in essence.

The account of the first day of creation is the beginning of the beginnings, i.e., the theme and principle of creation.
It is not a question of what is first in time, but of the deepest and highest foundations of existence
— of the primal foundation (in the sense of the archetypal ground) underlying the world.

The text of Genesis concerning the first day of creation must be read in the deeper language of spiritual morality. We
must comprehend it not as an allegory, but as a concrete moral spiritual happening.

God allowed two tendencies, two impulses, to proceed in opposite directions: in the direction “above” and in the
direction “below.”  There is the striving to be a “likeness” of God (Heaven) and the striving to give oneself up to
God, to mirror Him, to be an “image” of God (Earth).

The world was sunk in sleep; it breathed, but was unconscious. The darkness over the deep is absence of reflection,
i.e., unconsciousness. The creative primal words: “Let there be light,” are the awakening words for the sleeping world.
It signifies the same as “wake up,” the act of waking, or awaking consciousness. The sleeping world wrapped in darkness
was awakened to consciousness by the words “let there be light”; the “waters” began to reflect and the “earth” was no
longer formless and void. The spirits of the hierarchies were awakened out of the sleep of their “rest in God,” and so
there was \textbf{light}, the \textbf{consciousness} of the world.

\textbf{Seventh Miracle}

The Lazarus miracle signifies the awakening of consciousness for all that is relative — relative
for the soul immersed in “rest in the Absolute,” in the sleep of death. This miracle at the same time encompasses the
six preceding miracles, for Lazarus was not only a soul and spirit who was called back; he was also a corpse
— that is, blind, deaf, and without movement. Thus each miracle was repeated in his case:

\begin{itemize}
\item the miracle of the healing of the man born blind 
\item the miracle of the reinstatement of an independent bearing (the walking on the water) 
\item the miracle of nourishment, i.e., the regeneration of the used up and devastated organic substances of the body
(the miracle of the feeding of the five thousand) 
\item the miracle of the reinstatement of movement (the healing of the paralysed man) 
\item the miracle of the healing of sickness (which in this case had led to death) through the regeneration of “vertical
heredity” issuing from the new Adam (the healing of the nobleman's son) 
\item the miracle of the changing of water into wine, by which the liquids in the body were transformed into circulating
warm blood. 
\end{itemize}
\begin{quotationx}
Just as the first day of creation in essence contains and encompasses the entire account of the creation, so does the
seventh miracle of St. John's Gospel contain and encompass the other six miracles. It is the
archetype of the sacrament of extreme unction, the sacrament for the dying, which is directed toward future
resurrection — thereby in essence containing and encompassing the other six sacraments. \begin{flushright}
\emph{Covenant of the Heart}\end{flushright}

\end{quotationx}

\flright{\small\itshape Posted on 2023-04-10 by Cologero}

\section{Afterword}

Dear reader,

here comes to an end this \textit{Our Father Course}. Shortly after the last post, God called his servant Cologero from this earth, and thus the work remains unfinished.