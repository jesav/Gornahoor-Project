\documentclass[a4paper,12pt,twoside]{book}
\usepackage[utf8]{inputenc}
\usepackage[english]{babel}
%\usepackage{fontspec}
%\setmainfont[
%  Ligatures=TeX,
%  Extension=.otf,
%  BoldFont=cmunbx,
%  ItalicFont=cmunti,
%  BoldItalicFont=cmunbi,
%  SlantedFont=cmunsl
%]{cmunrm}

%\usepackage{polyglossia}
%\setmainlanguage{spanish}

\usepackage[c5paper]{geometry}
%\geometry{inner=2.5cm,outer=2.5cm,bmargin=3.2cm}
%\usepackage[DIV=14,BCOR=2mm,headinclude=true,footinclude=false]{typearea}

%\usepackage{ulem} %Hace que \emph sea subrayar

%\usepackage[p,osf]{scholax}
%% T1 and textcomp are loaded by package. Change that here, if you want
%% load sans and typewriter packages here, if needed

\usepackage{ebgaramond}
%\usepackage[type1]{libertine} % Linux Libertine for zweispaltige Texte
%\usepackage{textcomp}% Required to get special symbols
\usepackage[scaled=.8]{FiraMono}% Typewriter font
\usepackage{PTSansNarrow} 
%\gilliuscondensed
%\usepackage[sfdefault]{FiraSans}
%\usepackage{bm}% Extra bold faces
%\usepackage[lf]{carlito}

\usepackage{lettrine} %Capital letters at the beginning of a chapter
\usepackage[activate={true,nocompatibility},final,tracking=true,kerning=true,spacing=true,factor=1100,stretch=10,shrink=10]{microtype}
\SetTracking{encoding={*}, shape=sc}{-20} % versalitas menos separadas
% activate={true,nocompatibility} - activate protrusion and expansion
% final - enable microtype; use "draft" to disable
% tracking=true, kerning=true, spacing=true - activate these techniques
% factor=1100 - add 10% to the protrusion amount (default is 1000)
% stretch=10, shrink=10 - reduce stretchability/shrinkability (default is 20/20)

%\usepackage{array,multirow,booktabs,colortbl,chngcntr} % El último es para counterwithout;
\usepackage[strict]{changepage}
%\usepackage{caption}
%\captionsetup{format=plain,labelsep=newline,labelfont={small,sc},
%textfont={small,it},singlelinecheck=false}
\usepackage[Bjornstrup]{fncychap} % Para cabeceras de capítulos sofisticados:     Sonny,    Lenny,    Glenn,    Conny,    Rejne,    and Bjarne.

\usepackage{graphicx,wrapfig,booktabs,multicol,tabularx} % wallpaper: poner imágenes de fondo; wrapfigure: figuras a un lado del texto
%\graphicspath{{figures/}}
\usepackage{fancyhdr}
\usepackage{emptypage,pdfpages,fancybox} % Para que las páginas en blanco no tengan encabezado;
\usepackage{enumitem} %paralist: para compactenum, enumerate sin espacios
%\setlist[itemize]{nosep} %Espacio entre items en itemize
\usepackage[hyperref]{xcolor}
\usepackage[hidelinks]{hyperref}
\usepackage{url}

%\usepackage{minipage}

%\usepackage{quotchap} %Encabezados de capítulos
\usepackage{syntonly,verbatim}
%\syntaxonly

\usepackage{setspace,xspace} % xspace: Da \xspace para no tener que poner {} después de los comandos; pdflscape: páginas en horizontal;

\newenvironment{quotex}{\begin{quote}\small}{\end{quote}}
\newenvironment{quotationx}{\begin{quotation}\small}{\end{quotation}}


\begin{document}

% Por alguna razón, los marginados se creaban al revés. Así los corrijo. Feo, pero eficaz:
\let\tmp\oddsidemargin
\let\oddsidemargin\evensidemargin
\let\evensidemargin\tmp
\reversemarginpar

%\setcounter{secnumdepth}{4} % Para que llegue a numerar hasta las subsubsecciones;
%\renewcommand{\heavyrulewidth}{0.14em} % Grosor de las líneas extremas de las tablas;
%\renewcommand\thempfootnote{\alpha{mpfootnote}} % Símbolo de notas dentro de minipage
%\let\oldcaptionof\captionof
%\renewcommand{\captionof}[2]{\oldcaptionof{#1}{\newline \textit{#2} }}
%\renewcommand{\tablename}{Tabla}
%\counterwithout{figure}{chapter}
%\counterwithout{table}{chapter} % Así la numeración es 1, 2, 3... y no 1.1, 1.2... y no reinicia la num. en cada capítulo;

%\providecommand{\ggl}{\guillemotleft}
%\providecommand{\ggr}{\guillemotright\xspace}
\providecommand{\flright}[1]{\begin{flushright}#1\end{flushright}}
\providecommand{\flrightit}[1]{\begin{flushright}\itshape #1\end{flushright}}
%\renewcommand\UrlFont\sffamily
\urlstyle{tt}

\pagestyle{fancy}
\renewcommand{\sectionmark}[1]{\markright{#1}}
\renewcommand{\chaptermark}[1]{\markboth{#1}{}}

%Portada:
%\includepdf{00Portada}

\frontmatter
%
%\onehalfspacing
%\pagenumbering{Roman} %gobble es como empty

%\include{preindice}

%\fancyhf{}
%\fancyhead[LE]{\small \textbf{\thepage}$\quad$ Índice general}
%\fancyhead[RO]{\small Índice general $\quad$\textbf{\thepage}}
%\clearpage
\tableofcontents

%\doublespacing
\mainmatter

\fancyhf{}
\fancyhead[LE]{\small \thepage$\quad${\scshape\chaptername{} \thechapter}: \nouppercase{\itshape\leftmark}}
\fancyhead[RO]{\small \textsc{Section }\thesection{}: \nouppercase{\itshape\rightmark} $\quad$\upshape\thepage}
%\cfoot{\bfseries\thepage}

\pagenumbering{arabic}

%\onehalfspacing

\chapter{Our Father Course}
\section{Guide to Meditative Reading}

\emph{Genesis}, the Gospels, and the \emph{Apocalypse} dominate the readings for the Our Father Course. Since those
books (and there are others) are understood by Valentin Tomberg to be composed of spiritual exercises, he gives us some
clues about how to read such books. Keep these points in mind as you do the meditative readings for day. In particular,
in the \emph{Meditations on the Tarot}, Tomberg explains how to read the Apocalypse as well as the Gospels. These
techniques will apply to the other readings as well.

\textsc{How to read the Apocalypse:}
You may come to a different understanding from Tomberg's commentary on the Course. But this is how
he suggests reading the Apocalpyse:

\begin{quotationx}
the “key” to the \emph{Apocalypse of St. John} is nowhere to be found… for it is not at all a matter of interpreting it
with a view to extracting a philosophical, metaphysical or historical system. The key to the Apocalypse is to practise
it, i.e., to make use of it as a book of spiritual exercises which awaken from sleep ever-deeper layers of
consciousness.

The \emph{seven letters to the churches}, the \emph{seven seals of the sealed book}, the \emph{seven trumpets} and the
\emph{seven vials} signify, all together, a course of spiritual exercises composed of twenty-eight exercises. For as
the \emph{Apocalypse} is a revelation put into writing, it is necessary, in order to understand it, to establish in
oneself a state of consciousness which is suited to receive revelations.

\begin{enumerate}
\item
It is the state of concentration without effort (taught by the first Arcanum), 
\item
followed by a vigilant inner silence (taught by the second Arcanum), 
\item
which becomes an inspired activity of imagination and thought, where the conscious self acts together with
superconsciousness (teaching of the third Arcanum). 
\item
Lastly, the conscious self halts its creative activity and contemplates — in letting pass in review — everything which preceded, with a view to summarising it (practical teaching of the fourth
Arcanum). 
\end{enumerate}
The mastery of these four psychurgical operations, symbolised by “The Magician”, The High Priestess”, “The Empress” and
“The Emperor”, is the key to the Apocalypse. One will search in vain for another. 

\end{quotationx}

\textsc{How to read the Gospels:}
This requires the use of the imagination (step 3 above) to properly read the Gospels. This is so radically different
from the scholarly techniques of textual analysis, etc., which mostly desiccate the texts.

\begin{quotationx}
The Gospels, likewise, are spiritual exercises, i.e., one has not only to read and re-read them, but also to plunge
entirely into their element, to breathe their air, to Participate as an eye-witness, as it were, in the events
described there — and all this not in a scrutinising way, but as an “admirer”, with ever-growing
admiration. 

\end{quotationx}
\textsc{The Higher Self:}
I recently listened to a super-correct theologian who objected to Hermetic teaching, calling the idea of raising
one's state of consciousness “satanic”. It is best to leave such types at peace, but, for us, we
follow St Bonaventura, St Augustine, John Climacus, Dante, St John of the Cross, and many others who documented the
ascent to God. I have documented their teachings on my blog; I do not create new doctrines. Tomberg uses traditional
Hermetic terms to represent these higher states. Thomas Aquinas claimed we cannot know God in his essence in this life,
yet we can go as far as we are capable of. This is how Tomberg describes it in the \emph{Meditations on the Tarot}:

\begin{quotationx}
The transcendental Self is not God. It is in his image and after his likeness, according to the law of analogy or
kinship, but it is not identical with God. There are still several degrees on the ladder of analogy which separate it
from the summit of the ladder from God. These degrees which are higher than it are its “stars”, or the ideals to which
it aims. The Apocalypse specifies the number of them: there are twelve degrees higher than that of the consciousness of
the human transcendental Self. It is necessary, therefore, in order to attain to the ONE God, to elevate oneself
successively to degrees of consciousness of the nine spiritual hierarchies and the Holy Trinity. 

\end{quotationx}

\flright{\small\textit{Posted on 2022-10-21 by Cologero}}

\section{Daily Spiritual Exercises}

As preparation for the Our Father course, please look at these daily exercises. There is an exercise for each day of the week. Spend 5 minutes in the morning with the exercise, and then throughout the day when it occurs to you.

\paragraph{Monday: Right Word}
\textbf{Talking}. Only what has sense and meaning should come from the lips of one striving for higher development. All talking for the sake of talking — to kill time — is in this sense harmful.

The usual kind of conversation, a disjointed medley of remarks, should be avoided. This does not mean shutting oneself off from intercourse with one's fellows; it is precisely then that talk should gradually be led to significance. Adopt a thoughtful attitude to every speech and answer, taking all aspects into account. Never talk without cause and be gladly silent. One tries not to talk too much or too little. First listen quietly; then reflect on what has been said.

\paragraph{Tuesday: Right Deed}
\textbf{External actions}. These should not be disturbing for our fellowmen. Where an occasion calls for action out of one's inner being, deliberate carefully how one can best meet the occasion — for the good of the whole, the lasting happiness of man, the eternal.

Where you do things of your own accord, out of your own initiative: consider most thoroughly beforehand the effect of your actions.

\paragraph{Wednesday: Right Standpoint}
\textbf{The ordering of life}. Live in accordance with Nature and Spirit. Do not be swamped by the external trivialities of life. Avoid all that brings unrest and haste into life. Hurry over nothing, but also do not be indolent. Look on life as a means for working towards higher development and to behave accordingly.

\paragraph{Thursday: Right Habit}
\textbf{Human Endeavour}. Take care to do nothing that lies beyond your powers. But also leave nothing undone which lies within them.

Look beyond the everyday, the momentary, and set yourself aims and ideals connected with the highest duties of a human being. For instance, in the sense of the prescribed exercises, try to develop yourself so that afterwards you may be able all the more to help and advise your fellowmen, though perhaps not in the immediate future.

This can be summed up as: Let all the foregoing exercises become a habit.

\paragraph{Friday: Right Memory}
Remember what has been learnt from experiences. Endeavour to learn as much as possible from life.

Nothing goes by us without giving us a chance to gain experiences that are useful for life. If you have done something wrongly or imperfectly, that becomes a motive for doing it rightly or more perfectly, later on.

If you see others doing something, observe them with the like end in view (yet not coldly or heartlessly). And do nothing without looking back to past experiences which can be of assistance in your decisions and achievements.

You can learn from everyone, even from children if you are attentive.

\paragraph{Saturday: Right Opinion}
Pay attention to your ideas.

Think only significant thoughts. Learn little by little to separate in your thoughts the essential from the nonessential, the eternal from the transitory, truth from mere opinion.

While listening to the talk of others, try to become quite still inwardly, foregoing all assent, and still more, all unfavourable judgments (criticism, rejection), even in your thoughts and feelings.

\paragraph{Sunday: Right Judgment}
On even the most insignificant matter. judge only after fully reasoned deliberation. All unthinking behaviour, all meaningless actions, should be kept far away from the soul. You should always have well-weighed reasons for everything. And you should definitely abstain from doing anything for which there is no significant reason.

Once you are convinced of the rightness of a decision, hold fast to it, with inner steadfastness.

Right judgments are formed independently of sympathies and antipathies.

\paragraph{Every Day: Right Examination}
Turn your gaze inwards from time to time, even if only for five minutes daily at the same time. In so doing you should sink down into yourself, carefully take counsel with yourself, test and form your principles of life, run through in thought your knowledge — or lack of it — weigh up your duties, think over the contents and true purpose of life, feel genuinely pained by your own errors and imperfections.

In a word: labour to discover the essential, the enduring, and earnestly aim at goals in accord with it: for instance, virtues to be acquired. Do not fall into the mistake of thinking that you have done something well, but strive ever further towards the highest standards.

\begin{itemize}
\item Turn your gaze inwards from time to time, even if only for five minutes daily. 
\item Sink down into yourself. 
\item Carefully take counsel with yourself. 
\item Test and form your principles of life. 
\item Run through in thought your knowledge — or lack of it 
\item Weigh up your duties. 
\item Think over the contents and true purpose of life. 
\item Feel genuinely pained by your own errors and imperfections. 
\item Labour to discover the essential, the enduring, and earnestly aim at goals in accord with it. 
\end{itemize}


\flrightit{Posted on 2022-10-22 by Cologero }

\section{Week 1: First reading}

\subsection*{Day 1: Our Father who art in Heaven}
\textsc{Reading}: The Story of Paradise

The temptation in paradise was threefold, just as was the temptation of Jesus Christ in the wilderness. The following
are the essential elements of the triple temptation in paradise, as it is described in the account of the Fall in the
book of Genesis (from \emph{Meditations on the Tarot}):

\begin{enumerate}
\item Eve \emph{listened} to the voice of the serpent; 
\item She “\emph{saw} that the tree was good for food, and that it was a delight to the eyes” (Genesis 3:6); 
\item She “\emph{took} of its fruit and ate; and she also gave some to her husband, and he ate” (Genesis 3:6). 
\end{enumerate}

\begin{multicols}{2}\small
Now the serpent was more subtle than any of the beasts of the earth which the Lord God had made. And he said to the
woman: Why hath God commanded you, that you should not eat of every tree of paradise?

And the woman answered him, saying: Of the fruit of the trees that are in paradise we do eat:

But of the fruit of the tree which is in the midst of paradise, God hath commanded us that we should not eat; and that
we should not touch it, lest perhaps we die.

And the serpent said to the woman: No, you shall not die the death.

For God doth know that in what day soever you shall eat thereof, your eyes shall be opened: and you shall be as Gods,
knowing good and evil. And the woman saw that the tree was good to eat, and fair to the eyes, and delightful to behold:
and she took of the fruit thereof, and did eat, and gave to her husband who did eat. And the eyes of them both were
opened: and when they perceived themselves to be naked, they sewed together fig leaves, and made themselves aprons. And
when they heard the voice of the Lord God walking in paradise at the afternoon air, Adam and his wife hid themselves
from the face of the Lord God, amidst the trees of paradise.

And the Lord God called Adam, and said to him: Where art thou?

And he said: I heard thy voice in paradise; and I was afraid, because I was naked, and I hid myself.

And he said to him: And who hath told thee that thou wast naked, but that thou hast eaten of the tree whereof I
commanded thee that thou shouldst not eat?

And Adam said: The woman, whom thou gavest me to be my companion, gave me of the tree, and I did eat.

And the Lord God said to the woman: Why hast thou done this? And she answered: The serpent deceived me, and I did eat.

And the Lord God said to the serpent: Because thou hast done this thing, thou art cursed among all cattle, and beasts of
the earth: upon thy breast shalt thou go, and earth shalt thou eat all the days of thy life. I will put enmities
between thee and the woman, and thy seed and her seed: she shall crush thy head, and thou shalt lie in wait for her
heel.

To the woman also he said: I will multiply thy sorrows, and thy conceptions: in sorrow shalt thou bring forth children,
and thou shalt be under thy husband's power, and he shall have dominion over thee.

And to Adam he said: Because thou hast hearkened to the voice of thy wife, and hast eaten of the tree, whereof I
commanded thee that thou shouldst not eat, cursed is the earth in thy work; with labour and toil shalt thou eat thereof
all the days of thy life. Thorns and thistles shall it bring forth to thee; and thou shalt eat the herbs of the earth.
In the sweat of thy face shalt thou eat bread till thou return to the earth, out of which thou wast taken: for dust
thou art, and into dust thou shalt return.

And Adam called the name of his wife Eve: because she was the mother of all the living. \flright{\itshape Genesis 3:1-20}
\end{multicols}

\subsection*{Day 2: Hallowed be thy name}
\textsc{Reading}: The Nine Beatitudes

\begin{quotationx}
It is said that, “Nature has a horror of emptiness” (horror vacui). The spiritual counter-truth here is that, “the
Spirit has a horror of fullness”. It is necessary to create a natural emptiness —and this is what
renunciation achieves — in order for the spiritual to manifest itself. The beatitudes of the
Sermon on the Mount (Matthew v, 3-12) state this fundamental truth. 

\begin{flushright}\textit{Meditations on the Tarot}\end{flushright}

\end{quotationx}
\begin{enumerate}
\item Blessed are the poor in spirit: for theirs is the kingdom of heaven. (Mt 5:3, Lk 6:20) 
\item Blessed are they that mourn: for they shall be comforted. (Mt 5:4, Lk 6:20) 
\item Blessed are the meek: for they shall inherit the earth. (Mt 5:5, Lk 6:21) 
\item Blessed are they which do hunger and thirst after righteousness: for they shall be filled. (Mt 5:6, Lk 6:21) 
\item Blessed are the merciful: for they shall obtain mercy. (Mt 5:7.) 
\item Blessed are the pure in heart: for they shall see God. (Mt 5:8) 
\item Blessed are the peacemakers: for they shall be called the children of God. (Mt 5:9) 
\item Blessed are they which are persecuted for righteousness' sake: for theirs is the kingdom of
heaven. (Mt 5:20, Lk 6:20 
\item Blessed are ye, when men shall revile you, and persecute you, and shall say all manner of evil against you
falsely, for my sake. Rejoice, and be exceeding glad: for great is your reward in heaven. (Mt 5:11-12, Lk 6:22-23) 
\end{enumerate}

\subsection*{Day 3: Thy Kingdom Come}
\textsc{Reading:} The seven stages of the Cross in John's Gospel.

\begin{quotationx}
Christian meditation pursues the aim of deepening the two divine revelations: holy scripture and the creation, but it
does so above all with a view to awakening a more complete consciousness and appreciation of Jesus
Christ's work of redemption. For this reason, it culminates in the contemplation of the seven
stages of the Passion: the washing of the feet, the scourging, the crowning with the crown of thorns, the way of the
cross, the crucifixion, the laying in the tomb, and the resurrection. 

\begin{flushright}\textit{Meditations on the Tarot}\end{flushright}

\end{quotationx}
\textsc{Meditation}: The seven stages of the Passion

\begin{enumerate}
\item Washing of the Feet (Jn 13:1-20) 
\item Scourging (Jn 18:22, Jn 19:1-3) 
\item Crowning with Thorns (Jn 19:1-2) 
\item Bearing of the Cross (Jn 19:16-17) 
\item Crucifixion (Jn 19:18-19) 
\item Laying in the Tomb (Jn 19:40-42) 
\item Resurrection (Jn 20:1-18) 
\end{enumerate}

\subsection*{Day 4: Thy will be done on earth as it is in Heaven}
\textsc{Meditation}: The last things.

\begin{itemize}
\item Mt 24:1-51 
\item Mt 25:1-46 
\item Mk 13: 1-37 
\end{itemize}
\begin{quotationx}
Thus the first is also the last, and the “first day of creation” is the Last Day, the day of universal resurrection.
Therefore the history of Christianity—moving in the direction of the Last Things, toward the
future—is at the same time the history of the reawakening of the past, i.e., the resurrection of
the total past, insofar as truth and love have dwelt therein. So gradually there will revive in Christendom the
forgotten, deeply sleeping, and perished treasures of wisdom and sacrificial deeds of the
past—right back to the primeval revelation and the paradisiacal state of humanity. Thus all truth
and all love of all times will have their home in the Church of Christ, which will then be the all-embracing (catholic)
unity of all things and all beings who are striving for timeless values—in the sense of realizing
the ideal of one Shepherd and one flock. 

\begin{flushright}\textit{Covenant of the Heart}\end{flushright}

\end{quotationx}
\subsection*{Day 5: Give us this day our daily bread}
\textsc{Meditation}: Institution of the Last Supper

\begin{quotationx}
Christ's last words at the institution of the holy sacrament at the Last Supper: “Do this in memory
of me” point towards the sacraments, too, as being a re-enlivening in the present of what happened in the past. In the
holy sacrament at the altar, memory becomes an act of the divine magic of transubstantiation, an act relating to the
real (not just remembered) presence of the body and blood of the Redeemer. What once took place, takes place now in the
present. In the sacrament, memory does not become a journey into the past, but instead a making-present of the past, an
evocation that summons something up out of the realm of forgetting, sleep, and death. Memory becomes the bearer of the
power which sounded forth in the call of the Master — “Lazarus, come forth!”
— a call that proved effective. Memory becomes divine magic, a miracle of great love and faith. In
this sense the words: “Do this in memory of me” actually mean: “Do this, so that I may be present”. For, one may add,
the Son of Man is Lord over time too. 

\begin{flushright}\textit{Covenant of the Heart}\end{flushright}

\end{quotationx}

\subsubsection*{Imagination}
\begin{multicols}{2}\small
Now as they were eating, Jesus took bread, and blessed, and broke it, and gave it to the disciples and said,
“Take, eat; this is my body.”  And he took a cup, and when he had given thanks he gave it to them, saying, “Drink of it,
all of you; for this is my blood of the covenant, which is poured out for many for the forgiveness of sins.  I tell you
I shall not drink again of this fruit of the vine until that day when I drink it new with you in my
Father's kingdom.” \flright{\itshape Mt 26:26-29}

And as they were eating, he took bread, and blessed, and broke it, and gave it to them, and said, “Take; this is
my body.” And he took a cup, and when he had given thanks he gave it to them, and they all drank of it. And he said to
them, “This is my blood of the covenant, which is poured out for many. Truly, I say to you, I shall not drink again of
the fruit of the vine until that day when I drink it new in the kingdom of God.” \flright{\itshape Mk 14:22-25} 

And when the hour came, he sat at table, and the apostles with him. And he took bread, and when he had given
thanks he broke it and gave it to them, saying, “This is my body which is given for you. Do this in remembrance of me.”
And likewise the cup after supper, saying, “This cup which is poured out for you is the new covenant in my blood. But
behold the hand of him who betrays me is with me on the table. For the Son of man goes as it has been determined; but
woe to that man by whom he is betrayed!” And they began to question one another, which of them it was that would do
this. \flright{\itshape Lk 22:14, 22:19-23} 
\end{multicols}

\subsubsection*{Inspiration}
\begin{multicols}{2}\small
I am not speaking of you all; I know whom I have chosen; it is that the scripture may be fulfilled,
`He who ate my bread has lifted his heel against me.' I tell you this now,
before it takes place, that when it does take place, you may believe that I am he. Truly, truly, I say to you, he who
receives any one whom I send receives me; and he who receives me receives him who sent me.” When Jesus had thus spoken,
he was troubled in spirit, and testified, “Truly, truly, I say to you, one of you will betray me.” The disciples looked
at one another, uncertain of whom he spoke. One of his disciples, whom Jesus loved, was lying close to the breast of
Jesus; so Simon Peter beckoned to him and said, “Tell us who it is of whom he speaks.” So lying thus, close to the
breast of Jesus, he said to him, “Lord, who is it?” Jesus answered, “It is he to whom I shall give this morsel when I
have dipped it.” So when he had dipped the morsel, he gave it to Judas, the son of Simon Iscariot. Then after the
morsel, Satan entered into him. Jesus said to him, “What you are going to do, do quickly.” Now no one at the table knew
why he said this to him. Some thought that, because Judas had the money box, Jesus was telling him, “Buy what we need
for the feast”; or, that he should give something to the poor. So, after receiving the morsel, he immediately went out;
and it was night. \flright{\itshape Jn 13:18-30}
\end{multicols}

\subsubsection*{The 3rd Temptation of Jesus: Turning stones into bread}
\begin{quotationx}
Then Jesus was led up by the Spirit into the wilderness to be tempted by the devil. And he fasted forty days and forty
nights, and afterward he was hungry. And the tempter came and said to him, “If you are the Son of God, command these
stones to become loaves of bread.” But he answered, “It is written, ‘Man shall not live by bread
alone, but by every word that proceeds from the mouth of God.'” \flright{\itshape Mt 4:1-4}
\end{quotationx}

\subsection*{Day 6: Forgive us our trespasses as we forgive those who trespass against us}
\textsc{Meditation}: The seven miracles in John's Gospel

\begin{quotationx}
The Gospel is proclaimed to us in events, signs, parables, and teachings. Thereby the events are simultaneously signs,
parables, and teachings. The signs, however, are also simultaneously events, parables, and teachings. The parables are
also events, signs, and teachings; and the teachings are at the same time events, signs, and parables. Everything in
the Gospel is event, sign, parable, and teaching, i.e., everything is fact, miracle, symbol, and revelation of the
truth. The miracles of the Gospels are thus also facts — as well as symbols and revelations of
truth.
\begin{flushright}\textit{Covenant of the Heart}\end{flushright}

\end{quotationx}
\begin{enumerate}
\item Wedding at Cana (John 2:1-11) 
\item Healing of nobleman's son (John 4:46-54) 
\item Healing of sick man at pool of Bethesda (John 5:1-10) 
\item Feeding of the five thousand (John 6:1-15) 
\item Jesus walks on the water (John 6:16-21) 
\item Healing of the man born blind (John 9:1-7) 
\item Raising of Lazarus (John 11:1-44) 
\end{enumerate}
\subsubsection*{The 2nd Temptation of Jesus: Casting down from the pinnacle of the temple}
Then the devil took him to the holy city, and set him on the pinnacle of the temple, and said to him, “If you are the
Son of God, throw yourself down; for it is written, `He will give his angels charge of
you,' and `On their hands they will bear you up, lest you strike your foot
against a stone.'” Jesus said to him, “Again it is written, `You shall not
tempt the Lord your God.'” (Mt 4:5-7)

\subsection*{Day 7: And lead us not into temptation}
\begin{quotationx}
Moses did not cross the Jordan, but crossed the threshold of death. His “promised land” lay on the other side of the
threshold of death. The people of Israel prepared themselves for the future encounter with the expected Messiah in the
promised land; Moses was granted this meeting in the disembodied state. It took place in the scene of the
Transfiguration on Mt. Tabor in the accompaniment of Elijah. Peter, John, and, James were present as witnesses to this
encounter. 

\begin{flushright}\textit{Covenant of the Heart}\end{flushright}

\end{quotationx}
Meditation:

\begin{itemize}
\item The transfiguration on Mount Tabor (Mt 17:1-9, Mk 9:2-28, Lk 9:28-36) 
\item Healing of the sick child (Lk 9:37-43) 
\end{itemize}
\subsubsection*{The 1st Temptation of Jesus: All the kingdoms of the world}
Again, the devil took him to a very high mountain, and showed him all the kingdoms of the world and the glory of them;
and he said to him, “All these I will give you, if you will fall down and worship me.”  Then Jesus said to him,
“Begone, Satan! for it is written, `You shall worship the Lord your God and him only shall you
serve.'” Then the devil left him, and behold, angels came and ministered to him. (Mt 4:8-11)

\subsection*{Day 8: But deliver us from Evil}
\begin{quotationx}
The only Son of the eternal Father nailed to the cross for our sake — this is what is divinely
impressed upon all open souls, including the robber crucified to the right. This impression is unforgettable and
inexpressible. It is the immediate breath of God which has inspired and still inspires thousands of martyrs, confessors
of the faith, virgins and recluses. 

\begin{flushright}\textit{Meditations on the Tarot. Letter IV: The Emperor}\end{flushright}

\end{quotationx}
\textsc{Meditation}: The seven Words from the Cross

\begin{enumerate}
\item Father, into thy hands I commend my spirit. (Lk 23:46) 
\item My God, my God, why hast thou forsaken me? (Mt 27:45-46, Mk 15:34) 
\item I thirst! (Jn 19:28) 
\item Verily I say unto thee, today shalt thou be with me in paradise. (Lk 23:43) 
\item Father, forgive them; for they know not what they do. (Lk 23:34) 
\item Woman, behold thy son! (Jn 19:26-27) 
\item It is finished. (Jn 19:30) 
\end{enumerate}
\textsc{Readings}:

\subsubsection*{Revelation 12:}
And there appeared a great wonder in heaven; a woman clothed with the sun, and the moon under her feet, and upon her
head a crown of twelve stars: And she being with child cried, travailing in birth, and pained to be delivered. And
there appeared another wonder in heaven; and behold a great red dragon, having seven heads and ten horns, and seven
crowns upon his heads. And his tail drew the third part of the stars of heaven, and did cast them to the earth: and the
dragon stood before the woman which was ready to be delivered, for to devour her child as soon as it was born. And she
brought forth a man child, who was to rule all nations with a rod of iron: and her child was caught up unto God, and to
his throne.

\subsubsection*{Revelation 13:}
And I stood upon the sand of the sea, and saw a beast rise up out of the sea, having seven heads and ten horns, and upon
his horns ten crowns, and upon his heads the name of blasphemy. And the beast which I saw was like unto a leopard, and
his feet were as the feet of a bear, and his mouth as the mouth of a lion: and the dragon gave him his power, and his
seat, and great authority. And I saw one of his heads as it were wounded to death; and his deadly wound was healed: and
all the world wondered after the beast.

\flright{\small\textit{Posted on 2023-02-13 by Cologero}}
\section{Week 2: Our Father who art in Heaven}

\subsection*{Day 1: First curse of the Father}
\textsc{Reading}: Gen 3: 1-24

Meditate on the connection between disobedience (verse 6) and the necessity of toil (verse 17).

So when the woman saw that the tree was good for food, and that it was a delight to the eyes, and that the tree was to
be desired to make one wise, she took of its fruit and ate; and she also gave some to her husband, and he ate. (Genesis
3:6)

And to Adam he said: Because you have listened to the voice of your wife, and have eaten of the tree of which I
commanded you, `You shall not eat of it,' cursed is the ground because of
you; in toil you shall eat of it all the days of your life. (Genesis 3:17)

\begin{quotationx}
The fruit of the Tree of Knowledge of Good and Evil has had a triple effect:

\begin{itemize}
\item Toil 
\item Suffering 
\item Death 
\end{itemize}

Toil or work took the place of mystical union with God, which union (without effort) is the teaching of the first
Arcanum of the Tarot, the Magician. The mystical spontaneity of the first Arcanum is that relationship between man and
God which was before the Fall. 
\begin{flushright}\textit{Meditations on the Tarot}\end{flushright}

\end{quotationx}
\textsc{First psychurgical operation}: Concentration without effort.

\subsection*{Day 2: Second curse of the Father}
\textsc{Reading}: Gen 3: 1-24

Meditate on the connection between the feeling of shame (verse 7) and the necessity of suffering (verse 16).

Then the eyes of both were opened, and they knew that they were naked; and they sewed fig leaves together and made
themselves aprons. (Genesis 3:7)

To the woman he said, “I will greatly multiply your pain in childbearing; in pain you shall bring forth children, yet
your desire shall be for your husband, and he shall rule over you.” (Genesis 3:16)

\begin{quotationx}
Suffering replaced the directly reflected revelation or gnosis, whose direct revelation is the teaching of the second
Arcanum of the Tarot, the High Priestess. The gnosis of the second Arcanum is that consciousness which was before the
Fall. 
\begin{flushright}\textit{Meditations on the Tarot. Letter III: The Empress}\end{flushright} 

\end{quotationx}
\textsc{Second psychurgical operation}: Vigilant inner silence

\subsection*{Day 3: Third curse of the Father}
\textsc{Reading}: Gen 3:1-24

Meditate on the connection between the feeling of fear (verse 10) and the necessity of death (verses 19-24).

And he said, “I heard the sound of thee in the garden, and I was afraid, because I was naked; and I hid myself.”
(Genesis 3:10)

you are dust, and to dust you shall return. Then the Lord God said, “Behold, the man has become like one of us, knowing
good and evil; and now, lest he put forth his hand and take also of the tree of life, and eat, and live
forever”— therefore the Lord God sent him forth from the garden of Eden, to till the ground from
which he was taken. He drove out the man; and at the east of the garden of Eden he placed the cherubim, and a flaming
sword which turned every way, to guard the way to the tree of life. (Genesis 3: 19-24)

\begin{quotationx}
Death entered into the domain of life or creative, sacred magic, which is the teaching of the third Arcanum of the
Tarot, the Empress. For sacred magic is that life which was before the Fall. \begin{flushright}\textit{Meditations on the Tarot. Letter III: The Empress}\end{flushright} 

\end{quotationx}
\textsc{Third psychurgical operation}: Inspired activity of imagination and thought.

\subsection*{Day 4: The spiritualization of toil}
\textsc{Reading}: Gen 3: 1-24

The spiritualization of toil in creative spiritual work and meditation. Through the Holy Spirit.

\begin{quotationx}
the Fall changed the destiny of humanity —so that mystical union became replaced by struggle or
toil, gnosis by suffering, and sacred magic by death. This is why the formula announcing the “good news” that the
effects of the Fall can be overcome and that the way of human evolution can return to that of mystical union instead of
struggle, that immediately reflected revelation or gnosis can replace the teaching of the truth through suffering, and
that sacred magic or transforming life can take the place of destructive death.
\begin{flushright}\textit{Meditations on the Tarot. Letter III: The Empress}\end{flushright} 
\end{quotationx}

\subsection*{Day 5: The transfiguration of suffering}
\textsc{Reading}: Gen 3: 1-24

The transfiguration in the soul of suffering by passing through purgatory. Through the Son.

\begin{quotationx}
Take the terms “limbo”, “purgatory” and “paradise” in their meaning as understood by analogy and you have a clear and
precise formula for the working of the magic of the sacred pentagram of five wounds; it effects a change from the
natural state (“limbo”) and from the state of human suffering (“purgatory”) to that of the blessedness of the divine
state (“paradise”). The operation of the magic of the sacred pentagram of five wounds therefore consists in
transforming the natural state into the human state and this latter into the divine state. This is the work of
spiritual alchemy of the transformation from Nature (“limbo”), and from the Human (“purgatory”), into the Divine
(“paradise”), according to the traditional threefold division — Nature, Man and God. 
\begin{flushright}\textit{Meditations on the Tarot. Letter V: The Pope}\end{flushright} 

\end{quotationx}
Concerning the experience relating to “purgatory”, it comprises all purging of suffering —physical,
psychic and spiritual. It is corporeal, moral and intellectual suffering which is our intermediate state between the
experience of the natural innocence of “limbo” and the moments of heavenly joy when the rays of “paradise” reach us.

\subsection*{Day 6: The transformation of death}
\textsc{Reading}: Gen 3: 1-24

The transformation of death into the ideal of initiation. Through the Father.

Initiation is the Second Birth that Jesus revealed to Nicodemus.

\begin{quotationx}
The rebirth from Water and Spirit which the Master indicates to Nicodemus is the re-establishment of the state of
consciousness prior to the Fall, where the Spirit was divine Breath and where this Breath was reflected by virginal
Nature. 
\begin{flushright}\textit{Meditations on the Tarot. Letter II: The High Priestess}\end{flushright} 

Neither did death then play the role of liberating consciousness, through the destruction of the forms which enclose it,
that it has played since the Fall. Instead of the destruction of forms, their continual transformation took place. This
was operated by the perpetual action of life effecting the metamorphosis of forms, in conformity with changes in the
consciousness using them. This perpetually liberating constructive action of life was—and still
is—the function of sacred or divine magic. And it is this transforming function, opposed to the
destructive function of death, that Moses' Genesis designates by the symbol of the Tree of Life.
\begin{flushright}\textit{Meditations on the Tarot. Letter III: The Empress}\end{flushright} 

\end{quotationx}

\subsection*{Day 7: Meditation on the Trinity}
Thine is the Kingdom, and the Power, and the Glory.

Then the righteous will shine like the sun in the kingdom of their Father. (Matthew 13:43)

\textsc{Glory}: The Rainbow is the imagination of the Holy Spirit.

\textsc{Power}: The radiant solar cross in the blue sky is the imagination of the Son.

\textsc{Kingdom}: Stars shining in the dark night sky, strewn with stars, the soul turning towards the Father,
accompanied by the words: Our Father, who art in Heaven.

\begin{quotationx}
This is the rainbow of seven colours of the manifestation of “glory” or mastership and also the octave of the seven
tones of revelation of the “name” or mission of the vanquisher of the three temptations. And this rainbow shone around
the empty and somber place in the wilderness where the temptations took place. 
\begin{flushright}\textit{Meditations on the Tarot. Letter VII: The Chariot}\end{flushright} 

\end{quotationx}

\flright{\small\textit{Posted on 2023-02-19 by Cologero}}
\section{Week 3: Hallowed be thy name}

\textsc{Meditations}: The nine Beatitudes: Mt 5:1-12, Lk 6:20-23

This verse will require two weeks. The beatitudes awaken the Christ impulse in the various centers of the human being.

The nine beatitudes are the nine activities of the Comforter, the Paraclete, and are oriented to the perfect human being
of the future. That is, they relate to the spirit-imbued human being.

\begin{quotationx}
The Sermon on the Mount is not concerned merely with human beings preserving their true natures in the face of natural
evolution, nor even with them simply obeying the divinely revealed law, but that, in accordance with their archetype
— “the image and likeness of God” — they become as God. “Be ye perfect
therefore even as thy Father in heaven is perfect.” This central statement from the Sermon on the Mount is a call to
ascend from the kingdoms of nature and the human being to the kingdom of God. 
\begin{flushright}\textit{Covenant of the Heart}\end{flushright}

\end{quotationx}
Jesus based Baptized in complete emptiness in the Jordan.

Then cometh Jesus from Galilee to the Jordan, unto John, to be baptized by him. But John stayed him, saying: I ought to
be baptized by thee, and comest thou to me? And Jesus answering, said to him: Suffer it to be so now. For so it
becometh us to fulfill all justice. Then he suffered him. And Jesus being baptized, forthwith came out of the water:
and lo, the heavens were opened to him: and he saw the Spirit of God descending as a dove, and coming upon him. And
behold a voice from heaven, saying: This is my beloved Son, in whom I am well pleased. (Matthew 3:13-17)

And it came to pass, in those days, Jesus came from Nazareth of Galilee, and was baptized by John in the Jordan. And
forthwith coming up out of the water, he saw the heavens opened, and the Spirit as a dove descending, and remaining on
him. And there came a voice from heaven: Thou art my beloved Son; in thee I am well pleased. (Mark 1:9-11)

Now it came to pass, when all the people were baptized, that Jesus also being baptized and praying, heaven was opened;
And the Holy Ghost descended in a bodily shape, as a dove upon him; and a voice came from heaven: Thou art my beloved
Son; in thee I am well pleased. (Luke 3:21-22)

The next day, John saw Jesus coming to him, and he saith: Behold the Lamb of God, behold him who taketh away the sin of
the world. This is he, of whom I said: After me there cometh a man, who is preferred before me: because he was before
me. And I knew him not, but that he may be made manifest in Israel, therefore am I come baptizing with water. And John
gave testimony, saying: I saw the Spirit coming down, as a dove from heaven, and he remained upon him. And I knew him
not; but he who sent me to baptize with water, said to me: He upon whom thou shalt see the Spirit descending, and
remaining upon him, he it is that baptizeth with the Holy Ghost. And I saw, and I gave testimony, that this is the Son
of God. (John 1:29-34)

\subsection*{Day 1: First Beatitude: emptying the head}
\emph{Blessed are the poor in spirit, for theirs is the kingdom of heaven}. (Mt 5:3, Lk 6:20)

Alternatively, Blessed are the beggars for spirit, for theirs is the kingdom of heaven.

In his Baptism in the Jordan, Jesus had become empty and the greatest beggar for spirit.

\begin{quotationx}
The first beatitude means to say that those who are rich in spirit, who are filled with the “spiritual kingdom of man”,
have no room for the “kingdom of heaven”. Revelation presupposes emptiness — space put at its
disposal — in order to manifest itself. This is why it is necessary to renounce personal opinion
in order to receive the revelation of the truth, personal action in order to become an agent for sacred magic, the way
(or method) of personal development in order to be guided by the Master of ways, and one's
personally chosen mission in order to be charged with a mission from above.
\begin{flushright}\textit{Meditations on the Tarot}\end{flushright} 

\end{quotationx}
As a fact the wandering in the desert was an historical occurrence, as a symbol it is an expression of the timeless law
of the necessity of purification and the “emptying of consciousness” as precondition for the revelation of God in His
truth.

\begin{quotationx}
the first Beatitude of the Sermon on the Mount proclaims: “Blessed are the poor in spirit for theirs is the kingdom of
heaven,” i.e., blessed are those who regard as poor any knowledge of power without God—that is,
any knowledge or power not of God himself—for they shall participate in the divine archetypal
creative work of God. 
\begin{flushright}\textit{Covenant of the Heart}\end{flushright}

\end{quotationx}
\subsection*{Day 2: Second Beatitude: courage of the heart}
\emph{Blessed are they that mourn, for they shall be comforted}. (Mt 5:4, Lk 6:21)

The blessing of Mary is contained in this; the inspiration that comes from suffering (“And a sword shall pierce thy
heart”). The courage to endure suffering.

\begin{quotationx}
They that mourn or bear sorrow neither strive after a pain-free existence nor turn away from pain, but bear it with
acceptance. For the fullness of existence, life's true richness, does not consist solely in health
and happiness but in an ever-expanding range of joy and sorrow; and the broader the range, the richer life becomes.
\begin{flushright}\textit{Covenant of the Heart}\end{flushright}

\end{quotationx}
\subsection*{Day 3: Third Beatitude: control of the will, mindfulness}
\emph{Blessed are the meek for they shall inherit the earth}. (Mt 5:5, Lk 6:21)

Behold the course of one's life with inner calm.

\begin{quotationx}
Natural evolution rests on the principle that dominion over the kingdom of nature—over the earth
—belongs to those with the greatest will to power. It is predestined for the tough and the
toughest … Not the tough but the meek shall rule earth's natural kingdoms. The power that St.
Francis of Assisi, for instance, wielded over birds and fish and wild wolves was not of a kind that any natural
scientist, fisherman, forester, or hunter has ever possessed. The same is true of the obedience rendered toward St.
Anthony by the hyenas in the Egyptian desert, as well as of many other instances of deference from the side of
so-called dumb nature toward truly meek humans … That Christianity prevailed in ancient times despite its persecution
is an instance on a global scale—one that cannot simply be explained away—
where meekness took possession of the “earthly kingdom” (orbis terrarum) of that time. 
\begin{flushright}\textit{Covenant of the Heart}\end{flushright}

\end{quotationx}
\textsc{Meditation}:

And behold an angel of the Lord stood by them, and the brightness of God shone round about them; and they feared with a
great fear. And the angel said to them: Fear not; for, behold, I bring you good tidings of great joy, that shall be to
all the people: For, this day, is born to you a Saviour, who is Christ the Lord, in the city of David. And this shall
be a sign unto you. You shall find the infant wrapped in swaddling clothes, and laid in a manger. And suddenly there
was with the angel a multitude of the heavenly army, praising God, and saying\textbf{: Glory to God in the highest; and
on earth peace to men of good will}. And it came to pass, after the angels departed from them into heaven, the
shepherds said one to another: Let us go over to Bethlehem, and let us see this word that is come to pass, which the
Lord hath shewed to us.  (Luke 2:9-15)

Glory (head) – Peace (heart) – Good Will (will)

\textbf{Vladimir Solovyov}, in his \emph{Lectures on Divine Humanity,} wrote:

\begin{quotationx}
One can be considered free from passions only when one has them but has power over them, when one possesses, but is not
possessed by them.

\end{quotationx}
In Matthew, the word translated as “meek” is \emph{praos}, which was also used to mean the gentling of a horse. The
horse is the most noble of animals, but it must be broken to be of use to the knight.

So we see that the “meek” are those whose souls are under the control and guidance of the Intellect, or Head, since the
appetitive and incensive aspects of the soul (the Will) have been broken and put in service to the highest aspect in
man. They still retain their driving forces: the appetitive to achieve a goal, and the incensive to provide the
emotional energy and motivation to persist. Far from being mousy or timid, the meek are strong willed and inner
directed. This is why they shall inherit the earth.

\subsection*{Day 4: Fourth Beatitude: Christ impulse in the Sentient Soul}
\emph{Blessed are they that hunger and thirst after justice: for they shall have their fill}. (Mt 5:6, Lk 6:21)

Emptied head à Courageous Heart à Mindfulness in the Will à righteousness

\begin{quotationx}
The law of righteousness of the “kingdom of God” as expressed in the Sermon on the Mount … is operative in the highest
and most essential region of all. It heals the wounds of the heart suffered at the hand of injustice and transforms the
pain of unjustly inflicted suffering into everlasting bliss. At the same time, without inflicting punishment, it leaves
offenders to the tribunal of their own conscience—that is, to their karma. Thus, does the
righteousness of the kingdom of God transcend both righteousness of retribution and that of atonement (karma) in that
it is a bountiful and merciful righteousness. It bestows gifts of eternal value in whose light the shadows cast through
suffering injustice disappear. 
\begin{flushright}\textit{Covenant of the Heart}\end{flushright}

\end{quotationx}
\subsection*{Day 5: Fifth Beatitude: Christ impulse in the Intellectual Soul}
\emph{Blessed are the merciful: for they shall obtain mercy}. (Mt 5:7)

Mercy is justice combined with love. It is judgment and at the same time creation of the means whereby the guilty can
atone for their guilt, carrying not only the past and the present, but also the future.

\begin{quotationx}
All those who desire—that is, actually make the effort—to practice a morality
transcending retribution and atonement will have their place in the kingdom of God and his righteousness … It is thus
not the will for the good favor and benefaction of merciful righteousness that makes one a partaker thereof, but the
will for the kingdom of God and his righteousness in itself—will that is put into actual practice.
\begin{flushright}\textit{Covenant of the Heart}\end{flushright}

\end{quotationx}
Through the sentient soul man is related to the animal. In animals, also, we observe the presence of sensations,
impulses, instincts and passions. But the animal obeys these immediately. They do not, in its case, become interwoven
with independent thoughts, transcending the immediate experiences. This is also the case to a certain extent with
undeveloped human beings. \textbf{The mere sentient soul is therefore different from the evolved higher member of the
soul which brings thinking into its service. This soul that is served by thought will be termed the intellectual soul}.
… The intellectual soul permeates the sentient soul. Whoever has the organ for “seeing” the soul sees, therefore, the
intellectual soul as a separate entity, in relation to the mere sentient soul.

Note: The Intellectual Soul is also called the Rational Soul. (e.g., Thomas Aquinas, Aristotle)

\subsection*{Day 6: Sixth Beatitude: Christ impulse in Consciousness}
\emph{Blessed are the pure of heart: for they shall see God}. (Mt 5:8)

This is the quality of deepened and extended mercifulness. With radiant heart, behold nature with a gaze that asks what
nature needs. The healing gaze into the world is the pure heart; then one beholds God who is otherwise missing in the
natural world.

Purity of heart is to will one thing. This is the return to the Primordial State before the Fall, i.e., before Adam and
Eve listened to two competing, incompatible voices.

\begin{quotationx}
Duality therefore signifies the establishment of two centers of contemplation, two separate and rival principles
—one real and the other apparent —and this is the origin of evil, which is
only illegitimate twofoldness. 
\begin{flushright}\textit{Meditations on the Tarot. Letter II: The High Priestess}\end{flushright}

\end{quotationx}
\subsection*{Day 7: Seventh Beatitude: Christ impulse in the Self}
\emph{Blessed are the peacemakers: for they shall be called children of God}. (Mt 5:9)

The quality of the third beatitude, gentleness, is directed outward in the seventh beatitude, engendering peace.

\textsc{Temptation}: to see Christianity as a revelation given once and for all.

\begin{quotationx}
[The peacemakers are those] who refuse to take sides in the face of partial truths and prejudices, being dedicated to
the cause of the whole truth which unites the world and bears peace to it.
\begin{flushright}\textit{Meditations on the Tarot. Letter
IX: The Hermit}\end{flushright}

The path of transformed evolution, in the case of individual human beings and of humanity as a whole, begins with a
purification of the impulses, instincts, habits, and customs attached to natural evolution. Then it leads on to
illumination—intuition of the truth, beauty, and goodness of divine evolution, the kingdom of God
and his righteousness. Finally, it culminates in the union of will, feeling, and thought with the will, feeling, and
thought that underlie divine evolution or the work of salvation. 
\begin{flushright}\textit{Covenant of the Heart}\end{flushright}

\end{quotationx}

\flright{\small\textit{Posted on 2023-02-27 by Cologero}}
\section{Week 4: Hallowed be thy name (II)}

\subsection*{Day 1: Eighth Beatitude: Christ impulse in the Manas}
Blessed are they which are persecuted for righteousness'sake: for theirs is the kingdom of heaven.
(Mt 5:10, Lk 6:20)

Turns suffering — the quality of the second Beatitude — into a creative
activity.

The evil in nature is to be called forth out of the realm of dream into the waking world.

\textsc{Temptation}: To proletarize Christianity, making it exoteric and severed from the Mysteries.

\emph{Manas} = ego consciousness or personal consciousness (\emph{Cf. Meditations on the Tarot. Letter XXII: The Fool}).

\begin{quotationx}
The higher emotional center is to be found at the level of the heart, and the higher intellectual center at the level of
the head. Their functions are different. In the Tradition they are sometimes called the eyes of the Soul. Thus, St
Isaac the Syrian said: “While the two eyes of the body see things in an identical way, the eyes of the Soul see
differently: one contemplates the truth in images and symbols, the other face to face.” 
\begin{flushright}\textit{Gnosis}\end{flushright}

Righteousness—that is, freedom and equality—remains an illusion when sought
within the realm of natural evolution with its extension in human history. It is simply not to be found
there —and never can be— because the “struggle for survival,” translated from
natural evolution to the arena of human history, has nothing to do with righteousness. Righteousness must be looked for
elsewhere, in another dimension … The fact of their persecution for righteousness' sake is
manifestly something unrighteous that befalls them in earthly life, but the share in the kingdom of heaven they attain
thereby reduces such unrighteousness to naught.
\begin{flushright}\textit{Covenant of the Heart}\end{flushright}

\end{quotationx}

\subsection*{Day 2: Ninth Beatitude: Christ impulse in the Buddhi}
Blessed are ye, when men shall revile you, and persecute you, and shall say all manner of evil against you falsely, for
my sake. Rejoice, and be exceeding glad: for great is your reward in heaven: for so persecuted they the prophets which
were before you. (Mt 5:10-12, Lk 6:22-23)

The quality of the first Beatitude to become pure in spirit is creatively transformed: to be a representative of Christ
on earth. One's own personality counts for nothing.

\textsc{Meditation}: This beatitude exposes the evil of sub-nature, which was hidden in deep sleep. In the Buddhi, or
higher intellectual center, the meditation is beyond images.

\textsc{Danger}: to turn Christianity into a principle of domination.

\emph{Buddhi} = consciousness of higher Self or cosmic consciousness (\emph{Cf. Meditations on the Tarot. Letter XXII: The
Fool}).

The compensatory balance lies in the vertical earth-heaven axis and not along the horizontal axis of terrestrial events.
It is the “reward in heaven” that illumines from above the unrighteousness endured in the current of evolution, driving
it away like a shadow before the light. This compensatory balance, in the sense of vertical or divine righteousness,
consists in the “reward in heaven,” that is, in the enrichment of humanity's being and not in the
punishment of the perpetrator of unrighteousness in accordance with the principle of horizontal justice: “An eye for an
eye, a tooth for a tooth”

\begin{quotationx}
However, just as the law of karma morally surpasses both determinism and retributive justice, so it in turn is surpassed
by the law of righteousness of the “kingdom of God” as expressed in the Sermon on the Mount. For the latter is
operative in the highest and most essential region of all. It heals the wounds of the heart suffered at the hand of
injustice and transforms the pain of unjustly inflicted suffering into everlasting bliss. At the same time, without
inflicting punishment, it leaves offenders to the tribunal of their own conscience—that is, to
their karma. \begin{flushright}\textit{Covenant of the Heart}\end{flushright}

\end{quotationx}
\subsection*{Day 3: Authentic experience of the spiritual world}
\begin{quotationx}
Human nature in ancient times was such that it was possible, without difficulty, to enable a man to partake in the
happenings of the spiritual world. Today it is very arduous, relatively speaking, to undergo the true esoteric training
leading to the attainment of clairvoyance. 
\begin{flushright} \textsc{Rudolf Steiner}. \emph{Meditation on the Beatitudes in The
Gospel of St. Matthew. Lecture IX}
\footnote{\url{https://wn.rsarchive.org/Lectures/GA123/English/RSP1965/19100909p02.html}}
\end{flushright}

Rudolf Steiner has certainly said things of a nature to awaken the greatest creative elan! His series of
\textbf{lectures on the four Gospels}, his lectures at Helsingfors and Dusseldorf on the celestial hierarchies
— without mentioning his book on the inner work leading to initiation (\emph{Knowledge of the
Higher Worlds.}) — would alone suffice to inflame a deep and mature creative enthusiasm in every
soul who aspires to authentic experience of the spiritual world. 
\begin{flushright}\textit{Meditations on the Tarot}\end{flushright}

\end{quotationx}
\subsection*{Day 4: The proclamation of the kingdom}
The Sermon on the Mount was not simply about new doctrines. Nor is it merely symbolic. Rather, it is the spiritual
activity of the Word. Specifically, it is not a matter of presenting new knowledge; it is supposed to change the level
of being of those who hear it.

The most important effect of the Sermon was a spiritual, suprasensory stimulation of the forces of the inner I,
transcending the physical, etheric and astral bodies (vegetative and sensitive souls).

\begin{quotationx}
This transformation, entailing the conversion of natural evolution to the good, leads above all to the replacement of
its guiding principle—the struggle for survival—by that of peace as the basis
of the new evolution. For this reason, it is said in the Sermon on the Mount that the peacemakers are called “sons of
God,” i.e., not only do they behold God but also take up his work, just as sons take up and continue the work of their
fathers. \begin{flushright}\textit{Covenant of the Heart}\end{flushright}

\end{quotationx}
The Sermon on the Mount, considered a historical event of universal significance, marked a turning point of evolution
after which the principle of peace is gradually coming to replace the principle of war. The path of transformed
evolution, in the case of individual human beings and of humanity as a whole, begins with a purification of the
impulses, instincts, habits, and customs attached to natural evolution. Then it leads on to
illumination—intuition of the truth, beauty, and goodness of divine evolution, the kingdom of God
and his righteousness. Finally, it culminates in the union of will, feeling, and thought with the will, feeling, and
thought that underlie divine evolution or the work of salvation. St. Bonaventure characterized this path of
purification (\emph{purgatio}), illumination (\emph{iliuminatio}), and perfection (\emph{perfectio}) in the simplest
and clearest possible way (\emph{De triplici via}, Prologus, I).

\subsection*{Day 5: The Beatitudes as the Seed of the Future of Humanity}
\textsc{Meditation}: Follow the stages of Christ's passion

First, our experiences with the three temptations in the desert.

Then, passing through the stages of the passion:

\begin{enumerate}
\item washing of the feet 
\item scourging 
\item crowning with thorns 
\item carrying of the cross 
\item crucifixion 
\item entombment 
\item resurrection 
\end{enumerate}
To be blessed does not mean an escape from suffering and pain, but the experience of a new kind of suffering and pain.
\includegraphics[width=3.577cm,height=5.122cm]{a20230305OurFatherCourseWeek4-img001.jpg} 

There will always be the fortunate and the unfortunate, but the blessed form a new class of human beings.

The fundamental conditions to be among the blessed: emptiness, inner poverty, and spiritual beggary.

Tomberg prefers the term “beggary” because the beggar knows he is spiritually poor.

Concentration exercise is on the Rose Cross in three stages:

\begin{enumerate}
\item Combine a series of images that evoke feelings into a single image and retained in consciousness. 
\item Consciously wipe away that picture, so that the only thing remaining as an object of meditation is the
soul's own activity that built and held the picture. 
\item Blot out this image-free content from awareness, leaving a perfectly empty consciousness. 
\end{enumerate}

\subsection*{Day 6: Knights of the Grail}
The physical body is representative of the stage where a complete reconciliation between heaven and Earth could be
brought about. There are three stages for the ego to unite:

\begin{itemize}
\item The ego is consciously united with the spirit 
\item The union includes the astral body 
\item The union goes beyond the etheric body to the physical body 
\end{itemize}
High moral and spiritual truths can be learned through the experiences of the physical body. The physical body was seen
as the communion body by those who understood the Grail tradition; it was the highest possibility of human communion
with the spirit.

\begin{quotationx}
The search for the Grail testifies that there has always existed a striving for a conscious participation in the logic
of the Logos, a quest for a Christian initiation. \begin{flushright}\textit{Covenant of the Heart}\end{flushright}

St. Bernard advanced not only active contemplation for the monks but also contemplative activity for the knights
— just as Krishna did more than fifteen centuries before him. The one and the other did so because
they knew that man is at one and the same time a contemplative and an active being, that “faith without works is
death”. 
\begin{flushright}\textit{Meditations on the Tarot. Letter XIV: Temperance}\end{flushright}

\end{quotationx}
\subsection*{Day 7: Guilt, Need, Care, Death}
\begin{quotationx}
The life of an initiate must forever be veiled in mystery and is a secret between man's heart and
his God. 

\begin{flushright}\textit{Count Germain}\end{flushright}

\end{quotationx}
The path to poverty of spirit has four elements, taken from \emph{Faust}:

\begin{enumerate}
\item \textsc{Guilt}. The experience of guilt. 
\item \textsc{Need}. The growing consciousness of what had been sacrificed and lost 
\item \textsc{Care}: The suggestions of a false future led to an awareness of care. 
\item \textsc{Death}. Death as the gate into the world of real life, not as a gate to the realm of false existence 
\end{enumerate}
Initiation needs to be understood in the light of guilt, need, care, and death. These can only be experienced in the
physical body. Some spiritual paths, e.g., Hinduism, had the goal of transcending the body in order to avoid the
experiences of guilt, need, care, and death. Moreover, the darkness of the physical body comes with conflict with evil.

Without the experience of guilt, there is no experience of the need for redemption, i.e., the Christ impulse.

Meditate on these stages. This may lead to the realization of the intuition stage of the knowledge of the spiritual
world, the goal of the First Beatitude.

\flright{\small\textit{Posted on 2023-03-05 by Cologero}}
\section{Week 5: Thy Kingdom Come}

The awakened will of the human being calls upon Christ for penance. The stages of the Passion are compensation for human
guilt.

The Transgressions of man as prefiguration of Christ's sufferings in the passion:

\begin{tabular}{ll}\toprule
\textbf{Transgressions of Man} &
\textbf{Stages of the passion}\\\midrule
The \textbf{Fall} of man into sin &
Washing of the feet\\
The curse against \textbf{Eve} &
Scourging\\
The curse against \textbf{Adam} &
Crowning with thorns\\
The fratricide between \textbf{Cain} and \textbf{Abel} &
Carrying of the cross\\
\textbf{Cain} becomes homeless &
Crucifixion\\
Redemption of sins through the \textbf{Flood} &
Entombment\\
The experience of the end of the \textbf{Flood} &
Resurrection\\\bottomrule
\end{tabular}

\begin{quotationx}
Christian meditation pursues the aim of deepening the two divine revelations: holy scripture and the creation, but it
does so above all with a view to awakening a more complete consciousness and appreciation of Jesus
Christ's work of redemption. For this reason it culminates in the contemplation of the seven
stages of the Passion: the washing of the feet, the scourging, the crowning with the crown of thorns, the way of the
cross, the crucifixion, the laying in the tomb, and the resurrection. \begin{flushright} \emph{Meditations on the Tarot. Letter XXI: The Fool}\end{flushright}

\end{quotationx}

\subsection*{Day 1: First stage of the passion: Washing of the feet}
The washing of the Feet is the \emph{humiliation} through which the human being must pass because he submitted to the
temptation “to be as gods”.

\begin{quotationx}
True progress, true evolution, is the advance of beings from life under one law to life under another law, i.e., the
structural change of beings. It is thus that the law “an eye for an eye, and a tooth for a tooth” is in the process of
being gradually replaced by the law of forgiveness. It is thus again that the law “the weak serve the strong, the
people serve the king, the disciple serves the master” will one day give way to the law shown by the Master through the
act of the \textbf{Washing of the Feet}. According to this higher law, it is the strong who serve the weak, the king
who serves the people, the master who serves the disciple — just as it is in heaven, where Angels
serve human beings. Archangels serve Angels and men. Principalities serve Archangels, Angels and human beings, and so
on. And God? He serves all beings without exception. 
\begin{flushright} \textit{Meditations on the Tarot. Letter IX: The Hermit}\end{flushright}

\end{quotationx}

\textbf{Washing of the Feet (John 13:1-20)}

Before the festival day of the pasch, Jesus knowing that his hour was come, that he should pass out of this world to the
Father: having loved his own who were in the world, he loved them unto the end. And when supper was done, (the devil
having now put into the heart of Judas Iscariot, the son of Simon, to betray him). Knowing that the Father had given
him all things into his hands, and that he came from God, and goeth to God; He riseth from supper, and layeth aside his
garments, and having taken a towel, girded himself. After that, he putteth water into a basin, and began to wash the
feet of the disciples, and to wipe them with the towel wherewith he was girded. He cometh therefore to Simon Peter. And
Peter saith to him: Lord, dost thou wash my feet?

Jesus answered, and said to him: What I do thou knowest not now; but thou shalt know hereafter. Peter saith to him: Thou
shalt never wash my feet. Jesus answered him: If I wash thee not, thou shalt have no part with me. Simon Peter saith to
him: Lord, not only my feet, but also my hands and my head. Jesus saith to him: He that is washed, needeth not but to
wash his feet, but is clean wholly. And you are clean, but not all. For he knew who he was that would betray him;
therefore he said: You are not all clean. Then after he had washed their feet, and taken his garments, being set down
again, he said to them: Know you what I have done to you? You call me Master, and Lord; and you say well, for so I am.
\textbf{If then I being your Lord and Master, have washed your feet; you also ought to wash one
another's feet. For I have given you an example, that as I have done to you, so you do also.}

Amen, amen I say to you: The servant is not greater than his lord; neither is the apostle greater than he that sent him.
If you know these things, you shall be blessed if you do them. I speak not of you all: I know whom I have chosen. But
that the scripture may be fulfilled: He that eateth bread with me, shall lift up his heel against me. At present I tell
you, before it come to pass: that when it shall come to pass, you may believe that I am he. Amen, amen I say to you, he
that receiveth whomsoever I send, receiveth me; and he that receiveth me, receiveth him that sent me.

\textbf{Temptation in Paradise (Gen 3:1-12)}

Now the serpent was more subtle than any of the beasts of the earth which the Lord God had made. And he said to the
woman: Why hath God commanded you, that you should not eat of every tree of paradise? And the woman answered him,
saying: Of the fruit of the trees that are in paradise we do eat: But of the fruit of the tree which is in the midst of
paradise, God hath commanded us that we should not eat; and that we should not touch it, lest perhaps we die.
\textbf{And the serpent said to the woman: No, you shall not die the death. For God doth know that in what day soever
you shall eat thereof, your eyes shall be opened: and you shall be as Gods, knowing good and evil.}

And the woman saw that the tree was good to eat, and fair to the eyes, and delightful to behold: and she took of the
fruit thereof, and did eat, and gave to her husband who did eat. And the eyes of them both were opened: and when they
perceived themselves to be naked, they sewed together fig leaves, and made themselves aprons. And when they heard the
voice of the Lord God walking in paradise at the afternoon air, Adam and his wife hid themselves from the face of the
Lord God, amidst the trees of paradise. And the Lord God called Adam, and said to him: Where art thou? And he said: I
heard thy voice in paradise; and I was afraid, because I was naked, and I hid myself. And he said to him: And who hath
told thee that thou wast naked, but that thou hast eaten of the tree whereof I commanded thee that thou shouldst not
eat? And Adam said: The woman, whom thou gavest me to be my companion, gave me of the tree, and I did eat.

\subsection*{Day 2: Second stage of the passion: The scourging}
The scourging is to endure, to bear pain without resisting. The bearing of pain was handed to Eve as a curse.

\textbf{Scourging (John 19:1); Striking on the check (John 18:22, John 19:3)}

Then therefore, Pilate took Jesus, and scourged him.

When he had said these things, one of the servants standing by, gave Jesus a blow, saying: Answerest thou the high
priest so?

They came to him, and said: Hail, king of the Jews; and they gave him blows.

\textbf{The curse against Eve (Gen 3:16)}

To the woman also he said: I will multiply thy sorrows, and thy conceptions: in sorrow shalt thou bring forth children,
and thou shalt be under thy husband's power, and he shall have dominion over thee.

\subsection*{Day 3: Third stage of the passion: The crowning with thorns}
Toilsome labor is Adam's curse. Earthly man must deal with thistles and thorns.

\begin{quotationx}
Every crown is essentially a crown of thorns. Not only is it heavy, but also it calls for a painful restraint with
regard to the thought and free or arbitrary imagination of the personality. It certainly emits rays outwards, but these
same rays become thorns for the personality within. They play the role of nails piercing and crucifying each thought or
image of the personal imagination. \begin{flushright} \textit{Meditations on the Tarot. Letter IV: The Emperor} \end{flushright}

\end{quotationx}

\textbf{Crowning with Thorns (John 19:2)}

The soldiers platting a crown of thorns, put it upon his head; and they put on him a purple garment.

\begin{quotationx}
The “thorns” of the crown centre function as the “nails” of objectivity, which give conscience to thought. It is thanks
to them that thought has not become wholly emancipated and as arbitrary, for example, as the imagination is. Thought as
such is, in spite of all, the organ of truth, not of illusion. \begin{flushright} \textit{Meditations on the Tarot. Letter V: The Pope}\end{flushright}

\end{quotationx}

\textbf{The curse on Adam (Gen 3:17-19)}

And to Adam he said: Because thou hast hearkened to the voice of thy wife, and hast eaten of the tree, whereof I
commanded thee that thou shouldst not eat, cursed is the earth in thy work; with labour and toil shalt thou eat thereof
all the days of thy life. Thorns and thistles shall it bring forth to thee; and thou shalt eat the herbs of the earth.
In the sweat of thy face shalt thou eat bread till thou return to the earth, out of which thou wast taken: for dust
thou art, and into dust thou shalt return.

\subsection*{Day 4: Fourth stage of the passion: Carrying the cross}
Humanity must pass through the carrying of the cross after Cain's murder of Abel. Together they
form a cross:

\begin{itemize}
\item \textbf{Abel}. The vertical stream, through the ascending and accepted sacrifice. 
\item \textbf{Cain}. The horizontal stream. He created the line extending in breadth. 
\end{itemize}

\begin{quotationx}
The cause of all wars and revolutions — in a word, of all violence — is always
the same: the negation of hierarchy. This cause is found already, germinally, at such a lofty level as that of the
communal act of worship of the same God by two brothers — this is the staggering revelation of the
story of Cain and Abel. And as murders, wars and revolutions continue, the story of Cain and Abel remains ever valid
and relevant. Being always valid and relevant throughout the passage of centuries, this is a myth and, moreover, a myth
of the first order. \begin{flushright} \textit{Meditations on the Tarot. Letter I: The Magician}\end{flushright}

\end{quotationx}

\textbf{Carrying of the cross (John 19:17, Matthew 27:32, Mark 15:21, Luke 23:26)}

And bearing his own cross, he went forth to that place which is called Calvary, but in Hebrew Golgotha. And going out,
they found a man of Cyrene, named Simon: him they forced to take up his cross. And they forced one Simon a Cyrenian who
passed by, coming out of the country, the father of Alexander and of Rufus, to take up his cross. And as they led him
away, they laid hold of one Simon of Cyrene, coming from the country; and they laid the cross on him to carry after
Jesus.

\textbf{Cain and Abel (Genesis 4:1-10)}

And Adam knew Eve his wife: who conceived and brought forth Cain, saying: I have gotten a man through God. And again she
brought forth his brother Abel. And Abel was a shepherd, and Cain a husbandman. And it came to pass after many days,
that Cain offered, of the fruits of the earth, gifts to the Lord. Abel also offered of the firstlings of his flock, and
of their fat: and the Lord had respect to Abel, and to his offerings. But to Cain and his offerings he had no respect:
and Cain was exceedingly angry, and his countenance fell. And the Lord said to him: Why art thou angry? and why is thy
countenance fallen? If thou do well, shalt thou not receive? but if ill, shall not sin forthwith be present at the
door? but the lust thereof shall be under thee, and thou shalt have dominion over it. And Cain said to Abel his
brother: Let us go forth abroad. And when they were in the field, Cain rose up against his brother Abel, and slew him.
And the Lord said to Cain: Where is thy brother Abel? And he answered, I know not: am I my
brother's keeper? And he said to him: What hast thou done? the voice of thy
brother's blood crieth to me from the earth.

\subsection*{Day 5: Fifth stage of the passion: The crucifixion}
The Crucifixion is (but not only) the state of belonging nowhere, neither in heaven nor on earth. Cain is the eternal
wanderer. The two thieves are symbols of joining together of the two streams (vertical and horizontal).

\begin{quotationx}
The unconscious, instead of shocking intelligence, allies itself with it, penetrates it, and becomes luminous within it.
But this takes place only after the more-or-less long and painful experience of the crucifixion of consciousness on the
cross formed by the pair of opposites: subjectivity—objectivity (vertical), and the pair of
opposites: intelligence—unconscious wisdom (horizontal). \begin{flushright} \textit{Meditations on the Tarot. Letter XVIII: The Moon}\end{flushright} 

\end{quotationx}

\textbf{The Crucifixion (John 19:17-22, Matthew 27:31-38, Mark 15:20-32, Luke 23:26-43)}

\begin{quotationx}
In the times when the papacy was combined with worldly power the situation of inner crucifixion, and of “inner swooning”
which it produces, was often too heavy a burden inwardly for individual popes, as human beings. Thus there arose
attempts  impossible though this might be at “flight from the cross”: into politics, court life, the enjoyment of art
and culture, and the intoxications of sex. \begin{flushright} \textit{Covenant of the Heart}\end{flushright}

\end{quotationx}

And bearing his own cross, he went forth to that place which is called Calvary, but in Hebrew Golgotha. Where they
crucified him, and with him two others, one on each side, and Jesus in the midst. And Pilate wrote a title also, and he
put it upon the cross. And the writing was: JESUS OF NAZARETH, THE KING OF THE JEWS. This title therefore many of the
Jews did read: because the place where Jesus was crucified was nigh to the city: and it was written in Hebrew, in
Greek, and in Latin. Then the chief priests of the Jews said to Pilate: Write not, The King of the Jews; but that he
said, I am the King of the Jews. Pilate answered: What I have written, I have written.

And after they had mocked him, they took off the cloak from him, and put on him his own garments, and led him away to
crucify him. And going out, they found a man of Cyrene, named Simon: him they forced to take up his cross. And they
came to the place that is called Golgotha, which is the place of Calvary. And they gave him wine to drink mingled with
gall. And when he had tasted, he would not drink. And after they had crucified him, they divided his garments, casting
lots; that it might be fulfilled which was spoken by the prophet, saying: They divided my garments among them; and upon
my vesture they cast lots. And they sat and watched him. And they put over his head his cause written: THIS IS JESUS
THE KING OF THE JEWS. Then were crucified with him two thieves: one on the right hand, and one on the left.

And after they had mocked him, they took off the purple from him, and put his own garments on him, and they led him out
to crucify him. And they forced one Simon a Cyrenian who passed by, coming out of the country, the father of Alexander
and of Rufus, to take up his cross. And they bring him into the place called Golgotha, which being interpreted is, The
place of Calvary. And they gave him to drink wine mingled with myrrh; but he took it not. And crucifying him, they
divided his garments, casting lots upon them, what every man should take. And it was the third hour, and they crucified
him. And the inscription of his cause was written over: THE KING OF THE JEWS. And with him they crucify two thieves;
the one on his right hand, and the other on his left. And the scripture was fulfilled, which saith: And with the wicked
he was reputed. And they that passed by blasphemed him, wagging their heads, and saying: Vah, thou that destroys the
temple of God, and in three days builds it up again; Save thyself, coming down from the cross. In like manner also the
chief priests mocking, said with the scribes one to another: He saved others; himself he cannot save. Let Christ the
king of Israel come down now from the cross, that we may see and believe. And they that were crucified with him reviled
him.

And as they led him away, they laid hold of one Simon of Cyrene, coming from the country; and they laid the cross on him
to carry after Jesus. And there followed him a great multitude of people, and of women, who bewailed and lamented him.
But Jesus turning to them, said: Daughters of Jerusalem, weep not over me; but weep for yourselves, and for your
children. For behold, the days shall come, wherein they will say: Blessed are the barren, and the wombs that have not
borne, and the paps that have not given suck.

Then shall they begin to say to the mountains: Fall upon us; and to the hills: Cover us. For if in the green wood they
do these things, what shall be done in the dry? And there were also two other malefactors led with him to be put to
death. And when they were come to the place which is called Calvary, they crucified him there; and the robbers, one on
the right hand, and the other on the left. And Jesus said: Father, forgive them, for they know not what they do. But
they, dividing his garments, cast lots. And the people stood beholding, and the rulers with them derided him, saying:
He saved others; let him save himself, if he be Christ, the elect of God. And the soldiers also mocked him, coming to
him, and offering him vinegar, and saying: If thou be the king of the Jews, save thyself. And there was also a
superscription written over him in letters of Greek, and Latin, and Hebrew: THIS IS THE KING OF THE JEWS. And one of
those robbers who were hanged, blasphemed him, saying: If thou be Christ, save thyself and us. But the other answering,
rebuked him, saying: Neither dost thou fear God, seeing thou art condemned under the same condemnation? And we indeed
justly, for we receive the due reward of our deeds; but this man hath done no evil. And he said to Jesus: Lord,
remember me when thou shalt come into thy kingdom. And Jesus said to him: Amen I say to thee, this day thou shalt be
with me in paradise.

\textbf{The curse on Cain (Genesis 4:8-16)}

And Cain said to Abel his brother: Let us go forth abroad. And when they were in the field, Cain rose up against his
brother Abel, and slew him. And the Lord said to Cain: Where is thy brother Abel? And he answered, I know not: am I my
brother's keeper? And he said to him: What hast thou done? the voice of thy
brother's blood cries to me from the earth. Now, therefore, cursed shalt thou be upon the earth,
which hath opened her mouth and received the blood of thy brother at thy hand. When thou shalt till it, it shall not
yield to thee its fruit: a fugitive and a vagabond shalt thou be upon the earth. And Cain said to the Lord: My iniquity
is greater than that I may deserve pardon. Behold thou dost cast me out this day from the face of the earth, and I
shall be hidden from thy face, and I shall be a vagabond and a fugitive on the earth: every one, therefore, that finds
me, shall kill me. And the Lord said to him: No, it shall not be so: but whosoever shall kill Cain, shall be punished
sevenfold. And the Lord set a mark upon Cain, that whosoever found him should not kill him. And Cain went out from the
face of the Lord, and dwelt as a fugitive on the earth, at the east side of Eden.

\subsection*{Day 6: Sixth stage of the passion: Entombment}
\textbf{Noah} protected all living things until there was again the possibility for new life. Then they re-emerged.
Meditate on the image of the Ark.

\begin{quotationx}
Just as the Emerald Table saved the essence of ancient wisdom, and just as the Tarot saved the essence of mediaeval
wisdom, across the deluges which occurred in the time that separates us from them, may the essence of modern wisdom be
saved in a spiritual “\textbf{Noah's ark}” from the deluge which is going to come, and may it
thereby be transmitted to the future, just as the essence of ancient wisdom and that of mediaeval wisdom has been
transmitted to us by means of the Emerald Table and the Major Arcana of the Tarot. The tradition of Hermeticism
blossomed in the past and must live in the future. \begin{flushright} \textit{Meditations on the Tarot. Letter XVII: The Star}\end{flushright}

\end{quotationx}

\textbf{Laying in the tomb (John 19:38-42, Matthew 27:57-61, Mark 15:42-47, Luke 23:50-56)}

And after these things, Joseph of Arimathea (because he was a disciple of Jesus, but secretly for fear of the Jews)
besought Pilate that he might take away the body of Jesus. And Pilate gave leave. He came therefore, and took the body
of Jesus. And Nicodemus also came, (he who at the first came to Jesus by night,) bringing a mixture of myrrh and aloes,
about an hundred pound weight. They took therefore the body of Jesus, and bound it in linen cloths, with the spices, as
the manner of the Jews is to bury. Now there was in the place where he was crucified, a garden; and in the garden a new
sepulchre, wherein no man yet had been laid. There, therefore, because of the parasceve of the Jews, they laid Jesus,
because the sepulchre was nigh at hand.

And when it was evening, there came a certain rich man of Arimathea, named Joseph, who also himself was a disciple of
Jesus. He went to Pilate, and asked the body of Jesus. Then Pilate commanded that the body should be delivered. And
Joseph taking the body, wrapped it up in a clean linen cloth. And laid it in his own new monument, which he had hewed
out in a rock. And he rolled a great stone to the door of the monument, and went his way. And there was there Mary
Magdalen, and the other Mary sitting over against the sepulchre.

And when evening was now come, (because it was the Parasceve, that is, the day before the sabbath,) Joseph of Arimathea,
a noble counsellor, who was also himself looking for the kingdom of God, came and went in boldly to Pilate, and begged
the body of Jesus. But Pilate wondered that he should be already dead. And sending for the centurion, he asked him if
he were already dead. And when he had understood it by the centurion, he gave the body to Joseph. And Joseph buying
fine linen, and taking him down, wrapped him up in the fine linen, and laid him in a sepulchre which was hewed out of a
rock. And he rolled a stone to the door of the sepulchre. And Mary Magdalen, and Mary the mother of Joseph, beheld
where he was laid.

And behold there was a man named Joseph, who was a counsellor, a good and just man, (The same had not consented to their
counsel and doings;) of Arimathea, a city of Judea; who also himself looked for the kingdom of God. This man went to
Pilate, and begged the body of Jesus. And taking him down, he wrapped him in fine linen, and laid him in a sepulchre
that was hewed in stone, wherein never yet any man had been laid. And it was the day of the Parasceve, and the sabbath
drew on. And the women that were come with him from Galilee, following after, saw the sepulchre, and how his body was
laid. And returning, they prepared spices and ointments; and on the sabbath day they rested, according to the
commandment.

\textbf{The Flood (Genesis 6:5-22, 7:1-23)}

And God seeing that the wickedness of men was great on the earth, and that all the thought of their heart was bent upon
evil at all times, It repented him that he had made man on the earth. And being touched inwardly with sorrow of heart,
He said: I will destroy man, whom I have created, from the face of the earth, from man even to beasts, from the
creeping thing even to the fowls of the air, for it repenteth me that I have made them. But Noah found grace before the
Lord. These are the generations of Noah: Noah was a just and perfect man in his generations, he walked with God. And he
begot three sons, Sem, Cham, and Japheth.

And the earth was corrupted before God, and was filled with iniquity. And when God had seen that the earth was corrupted
(for all flesh had corrupted its way upon the earth,) He said to Noah: The end of all flesh is come before me, the
earth is filled with iniquity through them, and I will destroy them with the earth. Make thee an ark of timber planks:
thou shalt make little rooms in the ark, and thou shalt pitch it within and without. And thus shalt thou make it: The
length of the ark shall be three hundred cubits: the breadth of it fifty cubits, and the height of it thirty cubits.
Thou shalt make a window in the ark, and in a cubit shalt thou finish the top of it: and the door of the ark thou shalt
set in the side: with lower, middle chambers, and third stories shalt thou make it. Behold I will bring the waters of a
great flood upon the earth, to destroy all flesh, wherein is the breath of life, under heaven. All things that are in
the earth shall be consumed. And I will establish my covenant with thee, and thou shalt enter into the ark, thou and
thy sons, and thy wife, and the wives of thy sons with thee. And of every living creature of all flesh, thou shalt
bring two of a sort into the ark, that they may live with thee: of the male sex, and the female. Of fowls according to
their kind, and of beasts in their kind, and of every thing that creepeth on the earth according to its kind; two of
every sort shall go in with thee, that they may live. Thou shalt take unto thee of all food that may be eaten, and thou
shalt lay it up with thee: and it shall be food for thee and them. And Noah did all things which God commanded him.

And the Lord said to him: Go in thou and all thy house into the ark: for thee I have seen just before me in this
generation. Of all clean beasts take seven and seven, the male and the female. But of the beasts that are unclean two
and two, the male and the female. Of the fowls also of the air seven and seven, the male and the female: that seed may
be saved upon the face of the whole earth. For yet a while, and after seven days, I will rain upon the earth forty days
and forty nights; and I will destroy every substance that I have made, from the face of the earth. And Noah did all
things which the Lord had commanded him. And he was six hundred years old, when the waters of the flood overflowed the
earth. And Noah went in and his sons, his wife and the wives of his sons with him into the ark, because of the waters
of the flood. And of beasts clean and unclean, and of fowls, and of every thing that moveth upon the earth, Two and two
went in to Noe into the ark, male and female, as the Lord had commanded Noah.

And after the seven days were passed, the waters of the flood overflowed the earth. In the six hundredth year of the
life of Noah, in the second month, in the seventeenth day of the month, all the fountains of the great deep were broken
up, and the flood gates of heaven were opened: And the rain fell upon the earth forty days and forty nights. In the
selfsame day Noah, and Shem, and Ham, and Japheth his sons: his wife, and the three wives of his sons with them, went
into the ark: They and every beast according to its kind, and all the cattle in their kind, and every thing that moveth
upon the earth according to its kind, and every fowl according to its kind, all birds, and all that fly, Went in to
Noah into the ark, two and two of all flesh, wherein was the breath of life. And they that went in, went in male and
female of all flesh, as God had commanded him: and the Lord shut him in on the outside.

And the flood was forty days upon the earth, and the waters increased, and lifted up the ark on high from the earth. For
they overflowed exceedingly: and filled all on the face of the earth: and the ark was carried upon the waters. And the
waters prevailed beyond measure upon the earth: and all the high mountains under the whole heaven were covered. The
water was fifteen cubits higher than the mountains which it covered. And all flesh was destroyed that moved upon the
earth, both of fowl, and of cattle, and of beasts, and of all creeping things that creep upon the earth: and all men.
And all things wherein there is the breath of life on the earth, died. And he destroyed all the substance that was upon
the earth, from man even to beast, and the creeping things and fowls of the air: and they were destroyed from the
earth: and Noe only remained, and they that were with him in the ark.

\subsection*{Day 7: Seventh stage of the passion: The resurrection}
The seventh stage is the Resurrection which came when everything re-emerged from Noah. The rainbow is the symbol of
reconciliation with God.

\begin{quotationx}
The concept, idea, and ideal of resurrection is different from the concept, idea, and ideal of eternal life, as a
resting in God, where the soul returns to the “Father's house” and remains there for all eternity.
Resurrection is not merely the end of the old, but actually means the beginning of a new world-creation,  with “a new
heaven and a new earth”. It is not the soul's rest in eternity, not the eternal rest of Nirvana,
but an active cooperation with God in a renewed world, healed of the consequences of the Fall into sin. The “new
Jerusalem” is a new world-order related to the old one in the same way as the resurrected man is related to mortal man.
The resurrection may be conceived of as the reappearance and active participation of the total human being
— as spirit, soul, and body — in the domain of the “world-in-­progress”. As
an idea, the resurrection is the realisation that God is divinely generous, that he does not take away again what he
has once given and granted, but that his gifts — existence, consciousness, freedom, and creative
activity — are valid for all eternity. \begin{flushright} \textit{Covenant of the Heart}\end{flushright}

\end{quotationx}

\textbf{The Resurrection (John 20:1-10, Matthew 28:1-10, Mark 16:1-11, Luke 24:1-12)}

And on the first day of the week, Mary Magdalen cometh early, when it was yet dark, unto the sepulchre; and she saw the
stone taken away from the sepulchre. She ran, therefore, and cometh to Simon Peter, and to the other disciple whom
Jesus loved, and saith to them: They have taken away the Lord out of the sepulchre, and we know not where they have
laid him. Peter therefore went out, and that other disciple, and they came to the sepulchre. And they both ran
together, and that other disciple did outrun Peter, and came first to the sepulchre. And when he stooped down, he saw
the linen cloths lying; but yet he went not in. Then cometh Simon Peter, following him, and went into the sepulchre,
and saw the linen cloths lying, And the napkin that had been about his head, not lying with the linen cloths, but
apart, wrapped up into one place. Then that other disciple also went in, who came first to the sepulchre: and he saw,
and believed. For as yet they knew not the scripture, that he must rise again from the dead. The disciples therefore
departed again to their home.

And in the end of the sabbath, when it began to dawn towards the first day of the week, came Mary Magdalen and the other
Mary, to see the sepulchre. And behold there was a great earthquake. For an angel of the Lord descended from heaven,
and coming, rolled back the stone, and sat upon it. And his countenance was as lightning, and his raiment as snow. And
for fear of him, the guards were struck with terror, and became as dead men. And the angel answering, said to the
women: Fear not you; for I know that you seek Jesus who was crucified. He is not here, for he is risen, as he said.
Come, and see the place where the Lord was laid. And going quickly, tell ye his disciples that he is risen: and behold
he will go before you into Galilee; there you shall see him. Lo, I have foretold it to you. And they went out quickly
from the sepulchre with fear and great joy, running to tell his disciples. And behold Jesus met them, saying: All hail.
But they came up and took hold of his feet, and adored him. Then Jesus said to them: Fear not. Go, tell my brethren
that they go into Galilee, there they shall see me.

And when the sabbath was past, Mary Magdalen, and Mary the mother of James, and Salome, bought sweet spices, that
coming, they might anoint Jesus. And very early in the morning, the first day of the week, they come to the sepulchre,
the sun being now risen. And they said one to another: Who shall roll us back the stone from the door of the sepulchre?
And looking, they saw the stone rolled back. For it was very great. And entering into the sepulchre, they saw a young
man sitting on the right side, clothed with a white robe: and they were astonished. Who saith to them: Be not
affrighted; you seek Jesus of Nazareth, who was crucified: he is risen, he is not here, behold the place where they
laid him. But go, tell his disciples and Peter that he goeth before you into Galilee; there you shall see him, as he
told you.  But they going out, fled from the sepulchre. For a trembling and fear had seized them: and they said nothing
to any man; for they were afraid. But he rising early the first day of the week, appeared first to Mary Magdalen, out
of whom he had cast seven devils. She went and told them that had been with him, who were mourning and weeping. And
they hearing that he was alive, and had been seen by her, did not believe.

And on the first day of the week, very early in the morning, they came to the sepulchre, bringing the spices which they
had prepared. And they found the stone rolled back from the sepulchre. And going in, they found not the body of the
Lord Jesus. And it came to pass, as they were astonished in their mind at this, behold, two men stood by them, in
shining apparel. And as they were afraid, and bowed down their countenance towards the ground, they said unto them: Why
seek you the living with the dead? He is not here, but is risen. Remember how he spoke unto you, when he was in
Galilee, Saying: The Son of man must be delivered into the hands of sinful men, and be crucified, and the third day
rise again. And they remembered his words. And going back from the sepulchre, they told all these things to the eleven,
and to all the rest. And it was Mary Magdalen, and Joanna, and Mary of James, and the other women that were with them,
who told these things to the apostles. And these words seemed to them as idle tales; and they did not believe them. But
Peter rising up, ran to the sepulchre, and stooping down, he saw the linen cloths laid by themselves; and went away
wondering in himself at that which was come to pass.

\textbf{The End of the Flood (Genesis 8:24-33)}

Therefore in the six hundredth and first year, the first month, the first day of the month, the waters were lessened
upon the earth, and Noe opening the covering of the ark, looked, and saw that the face of the earth was dried. In the
second month, the seven and twentieth day of the month, the earth was dried. And God spoke to Noah, saying:

Go out of the ark, thou and thy wife, thy sons, and the wives of thy sons with thee. All living things that are with
thee of all flesh, as well in fowls as in beasts, and all creeping things that creep upon the earth, bring out with
thee, and go ye upon the earth: increase and multiply upon it.

So Noah went out, he and his sons: his wife, and the wives of his sons with him. And all living things, and cattle, and
creeping things that creep upon the earth, according to their kinds, went out of the ark. And Noah built an altar unto
the Lord: and taking of all cattle and fowls that were clean, offered holocausts upon the altar. And the Lord smelled a
sweet saviour, and said:

I will no more curse the earth for the sake of man: for the imagination and thought of man's heart
are prone to evil from his youth: therefore I will no more destroy every living soul as I have done. All the days of
the earth, seedtime and harvest, cold and heat, summer and winter, night and day, shall not cease.

\textbf{The Rainbow, symbol of the reconciliation with God (Genesis 9:12-17)}

And God said:

This is the sign of the covenant which I give between me and you, and to every living soul that is with you, for
perpetual generations. I will set my bow in the clouds, and it shall be the sign of a covenant between me, and between
the earth. And when I shall cover the sky with clouds, my bow shall appear in the clouds: And I will remember my
covenant with you, and with every living soul that beareth flesh: and there shall no more be waters of a flood to
destroy all flesh.

And the bow shall be in the clouds, and I shall see it, and shall remember the everlasting covenant, that was made
between God and every living soul of all flesh which is upon the earth.

And God said to Noah: This shall be the sign of the covenant which I have established between me and all flesh upon the
earth.

\flright{\small\textit{Posted on 2023-03-12 by Cologero}}
\section{Week 6: Thy Will be done}

Meditation on God's blessings and judgment.

\begin{tabular}{ll}\toprule
\textbf{Blessing} &
\textbf{Judgment}\\\midrule
The seeds in the Garden &
The seeds of the Earth\\
Man shall not be alone &
Cain must wander alone\\
The four rivers of the Garden &
The Flood\\
Adam names the animals &
The Tower of Babel: God confuses human languages\\\bottomrule
\end{tabular}
 

\subsection*{Day 1: First blessing of God}
The blessing of eating the seeds in the Garden, i.e., spiritual nourishment, changes into the curse of feeding on plant
matter

\begin{quotationx}
The third day of creation is the “day” of procreation, of the mystery of seed and of growth. The third day of creation
is the generation of the seed principle, of the principle of potential formative force becoming actualized and bringing
to visible realization its own inner, invisible shape. The third day of creation is the coming into being of the seed
principle in the world, that is, not only of the plant-world manifested to us as plants and trees, but also such
“trees” growing in paradise as “the tree of knowledge of good and evil” and the “tree of life”. \begin{flushright} \emph{Covenant of
the Heart}\end{flushright}

\end{quotationx}

\textsc{Mt 7:16-18:} You will know them by their fruits. Are grapes gathered from thorns, or figs from thistles? So, every sound tree bears
good fruit, but the bad tree bears evil fruit. A sound tree cannot bear evil fruit, nor can a bad tree Dear good fruit.

\textbf{First Blessing of God: Genesis 1:26-31}

And he said: Let us make man to our image and likeness: and let him have dominion over the fishes of the sea, and the
fowls of the air, and the beasts, and the whole earth, and every creeping creature that moveth upon the earth. And God
created man to his own image: to the image of God he created him: male and female he created them. And God blessed
them, saying: Increase and multiply, and fill the earth, and subdue it, and rule over the fishes of the sea, and the
fowls of the air, and all living creatures that move upon the earth. And God said: Behold I have given you every herb
bearing seed upon the earth, and all trees that have in themselves seed of their own kind, to be your meat: And to all
beasts of the earth, and to every fowl of the air, and to all that move upon the earth, and wherein there is life, that
they may have to feed upon. And it was so done. And God saw all the things that he had made, and they were very good.
And the evening and morning were the sixth day.

\textbf{First Judgment of God: Genesis 3:18}

Thorns and thistles shall it bring forth to thee; and thou shalt eat the herbs of the earth.

\subsection*{Day 2: Second blessing of God}
\begin{quotationx}
Revolt, possession, and substitution of the fabricated for the revealed — with these three sins
there correspond three “falls” and effects entailed by them. Cain, who killed his brother Abel, became an exile
— he became a wanderer. \begin{flushright} \emph{Meditations on the Tarot. Letter XVI: The Tower of Destruction}\end{flushright}

\end{quotationx}

\textbf{Second Blessing of God: Genesis 1:27, 2:18-25}

God gave the human being Adam another human being, Eve, so that man would not be alone.

\begin{quotationx}
He who finds silence in the solitude of concentration without effort, is never alone. He never bears alone the weights
that he has to carry; the forces of heaven, the forces from on high, are there taking part from now on. \begin{flushright}
\emph{Meditations on the Tarot. Letter I: The Magician}\end{flushright}

\end{quotationx}

\textbf{Second Judgment of God: Genesis 4:1-16}

Cain murders another human being, so Cain must wander alone again.

\subsection*{Day 3: Third blessing of God}
\begin{quotationx}
In the world there are therefore two different kinds of arriving at a conviction: one can be illumined by the serene
clarity of contemplation, or one can be swept away by an electrifying flood of passionate arguments aiming at a desired
end. The faith of the illuminated is full of tolerance, patience and calm steadfastness — like
crystal; the faith of those who are swept away is, in contrast, fanatical, agitated and aggressive
— in order to live it needs conquests without end. because it is conquest alone which keeps it
alive. The faith of those who are swept away is greedy for success, this being its reason for existence, its criterion
and its motivating force. \begin{flushright} \emph{Meditations on the Tarot. Letter XI: Force}\end{flushright}

\end{quotationx}

\textbf{Third Blessing of God: Genesis 2:4-17}

God gave to the human being the Garden of Eden and its four rivers, or ether streams, as a dwelling place.

\begin{quotationx}
Such are the “trees” of the garden that we cultivate and maintain, i.e., the mysteries of union —
mystical, gnostic, magical and Hermetic — of that which is below with that which is above. For
mysticism, gnosis, magic and Hermetic science are the four branches of the “river” which flows out of our garden of
Eden “to water”—the “river” which “divided and became four rivers”. \begin{flushright} \emph{Meditations on the Tarot. Letter XVI: The Tower of Destruction}\end{flushright}

\end{quotationx}

\textbf{Third Judgment of God: Genesis 6:11-22, 7:20-24}

In anger, God sends the Flood in order to destroy the corrupted human being.

\subsection*{Day 4: Fourth blessing of God}
\textbf{Fourth Blessing of God: Genesis 2:19-20}

God gives human beings the word: nature was just as human beings named it.

\begin{quotationx}
Let us consider the following passage in the Bible, where it says: “And out of the ground the Lord God (Elohim) formed
every beast of the field and every bird of the air; and He brought them to Adam to see what he would name them; and
whatever Adam called every living creature, that was its name”. This does not mean that the human being invented
designations for the beings of the animal kingdom, and even less that he classified them according to genera and
species — along the lines of the system of Linnaeus — but rather that he
received and fulfilled the divine mandate to determine the vocations or missions relative to the human being of the
living creatures hierarchically subordinate to him. For, in the language of the Bible, “name” means “mission” or
“activity of being”, and “naming” is the magical act of determining the vocation or mission of a being. \begin{flushright}
\emph{Covenant of the Heart}\end{flushright}

\end{quotationx}
\textbf{Fourth Judgment of God: Genesis 11:1-9}

God confuses human language (Tower of Babel).

\begin{quotationx}
At the root of the building of the tower of Babel is the collective will of “lower selves” to achieve the replacing of
the “true Self of the celestial hierarchies and God with a superstructure of universal significance fabricated through
this will. The building of the tower of Babel had as its effect the “thunderbolt” of the “descent of the Lord”, who
“confused their language”— that of the builders —and “scattered them abroad
over the face of all the earth” so that they would no longer understand one another's language.
\begin{flushright} \emph{Meditations on the Tarot. Letter XVI: The Tower of Destruction}\end{flushright}

\end{quotationx}

\flright{\small\textit{Posted on 2023-03-19 by Cologero}}
\section{Week 7: Thy Will be done (II)}

\textbf{Task: Work on the seven messages to the seven churches.}

And he laid his right hand upon me, saying: Fear not.

I am the First and the Last, and alive, and was dead, and behold I am living for ever and ever, and have the keys of
death and of hell. Write therefore the things which thou hast seen, and which are, and which must be done hereafter.
The mystery of the seven stars, which thou saw in my right hand, and the seven golden candlesticks.

The seven stars are the angels of the seven churches. And the seven candlesticks are the seven churches. (Rev 1:17-20)

The seven churches can also be imagined as representing different historical epochs. Each of the churches has a positive
and a negative element. These still live in every human being, regardless of the epoch. Our task is to develop the
positive elements and overcome the negative elements. These cultural periods are:

\begin{itemize}
\item \textbf{Ephesus}: The old Indian (Vedic) culture 
\item \textbf{Smyrna}: The old Persian (Zoroastrian) culture 
\item \textbf{Pergamos}: The Chaldean-Egyptian (Hermetic) culture 
\item \textbf{Thyatira}: The Greco-Roman (pagan) culture 
\item \textbf{Sardis}: The Anglo-Germanic (current) culture 
\item \textbf{Philadelphia:} The Slavic-Russian (coming) culture 
\item \textbf{Laodicea:} The American, i.e., Western Hemisphere, (future) culture 
\end{itemize}
There are three churches of the past, two of the present, and two of the future. But do not take the history and
geography too literally, for they represent streams that are always active in us.

Further reading: Valentin Tomberg's essays on the letters to the seven churches.

\begin{quotationx}
The arcanum of inspiration is of vital practical importance not only for Hermeticism but also for the spiritual history
of mankind in general. For just as in the individual human biography there are decisive moments of inspiration, so
there are in mankind's biography — which is history —
decisive points where far-reaching inspirations enter into the spiritual life of humanity. The great religions are such
inspirations. \begin{flushright} \emph{Meditations on the Tarot. Letter XIV: Temperance}\end{flushright}

\end{quotationx}

\subsection*{Day 1: Message to Ephesus}
\textbf{Reading: Matthew 24, Rev 2:1-7}

\emph{The first message to the church of Ephesus concerns the ancient Indian culture.}

Please refer to this letter for the complete hymn.

\begin{quotationx}
One finds a profound and breathtaking feeling of these cosmic depths in the cosmogonic hymn of the Rigveda. It awakens
in the meditator at least a feeling of the profundity of the fundamental cosmic incentive towards, or feeling for
zodiacality. \begin{flushright} \emph{Meditations on the Tarot. Letter XII The Hanged Man}\end{flushright}

\end{quotationx}

Unto the angel of the church of Ephesus write: These things saith he, who holdeth the seven stars in his right hand, who
walketh in the midst of the seven golden candlesticks (Rev 2:1)

The mystery of the seven stars, which thou sawt in my right hand, and the seven golden candlesticks. The seven stars are
the angels of the seven churches. And the seven candlesticks are the seven churches. (Rev 1:20)

I know thy works, and thy labour (toil), and thy patience, and how thou canst not bear them that are evil, and thou hast
tried them, who say they are apostles, and are not, and hast found them liars: And thou hast patience, and hast endured
for my name, and hast not fainted. But I have somewhat against thee, because thou hast left thy first charity. Be
mindful therefore from whence thou art fallen: and do penance, and do the first works. Or else I come to thee, and will
move thy candlestick out of its place, except thou do penance. (Rev 2:2-5)

But this thou hast, that thou hatest the deeds of the Nicolaites, which I also hate. (Rev 2:6)

\textbf{Negative Side}

The flight from earthly reality in the first (Old Indian) culture, in not wanting to fully incarnate.

\textbf{Positive Side}

Hatred of the Nicolaitans and the false apostles; i.e., either one-sided materialism or false conceptions of Christ.

He, that hath an ear, let him hear what the Spirit saith to the churches: To him, that overcometh, I will give to eat of
the tree of life, which is in the paradise of my God. (Rev 2:7)

\subsection*{Day 2: Message to Smyrna}
\textbf{Reading: Matthew 24, Rev 2:8-11}

\emph{The second message to the church of Smyrna concerns the Persian culture.}

\begin{quotationx}
It is in the Iranian and Judaeo-Christian spiritual currents — i.e., in Zoroastrianism, Judaism and
Christianity — that the idea and ideal of resurrection has taken root. The advent of the idea and
ideal of resurrection was “as lightning coming from the east and shining as far as the west” (Matthew xxiv, 27). The
inspired prophet of the East, namely the great Zarathustra in Iran, and the inspired prophets of the West
—Isaiah, Ezekiel and Daniel in Israel — announced it almost simultaneously.
\begin{flushright} \emph{Meditations on the Tarot. Letter XX: The Judgment}\end{flushright}

\end{quotationx}
And to the angel of the church of Smyrna write: These things saith the First and the Last, who was dead, and is alive: I
know thy tribulation and thy poverty, but thou art rich: and thou art blasphemed by them that say they are Jews and are
not, but are the synagogue of Satan. Fear none of those things which thou shalt suffer. Behold, the devil will cast
some of you into prison that you may be tried: and you shall have tribulation ten days. Be thou faithful until death:
and I will give thee the crown of life. (Rev 2:8-10)

\textbf{Negative Side}

The false “we” consciousness: prisoner in the synagogue of Satan or prisoner in one's own
subjective I-consciousness. “Jews” refers to refers to the souls that have decided to serve the Christ impulse. Hence
the synagogue of Satan (Ahriman) is a caricature of the Christian community. Or else the devil (Lucifer) will isolate
some in their subjective prison.

\textbf{Positive Side}

Faithfulness to the spirit and to the human task on earth. Those united by the Christ impulse freely form a true
community.

He, that hath an ear, let him hear what the Spirit saith to the churches: He that shall overcome, shall not be hurt by
the second death. (Rev 2:11)

\begin{quotationx}
The “good news” of Zarathustra was that the world and Man represented an admixture of two distinct world orders
— that of the principle of light and that of the principle of darkness, or the divine-archetypal
world order and that of natural evolution — and that the latter would in the end be vanquished by
Soshyans “who through will overcomes death”, to be followed by the resurrection of the dead. \begin{flushright} \emph{Covenant of the
Heart}\end{flushright}

\end{quotationx}

\subsection*{Day 3: Message to Pergamos}
\textbf{Reading: Matthew 24, Rev 2:12-17}

\emph{The third message to the church of Pergamos concerns the Egypto-Chaldean culture.}

\begin{quotationx}
The spiritual impulse behind the third (Egypto-Chaldean) cultural epoch, which has persisted in human souls since that
time, is to strive for the experience of immortal individuality and for harmony among immortal individual beings. \begin{flushright}
\emph{Christ and Sophia}\end{flushright}

\end{quotationx}
And to the angel of the church of Pergamus write: These things, saith he, that hath the sharp two edged sword: I know
where thou dwellest, where the seat of Satan is: and thou holdest fast my name, and hast not denied my faith. Even in
those days when Antipas was my faithful witness, who was slain among you, where Satan dwelleth.

But I have against thee a few things: because thou hast there them that hold the doctrine of Balaam, who taught Balac to
cast a stumbling block before the children of Israel, to eat, and to commit fornication: So hast thou also them that
hold the doctrine of the Nicolaites. In like manner do penance: if not, I will come to thee quickly, and will fight
against them with the sword of my mouth. (Rev 2:12-16)

\begin{quotationx}
This is what St. Justin said about certain Greek philosophers. Although the close inner connection between Alexandrian
theosophy and the Christian doctrine is one of the firmly established theses of Western scholarship, for one reason or
another, this perfectly correct thesis does not enjoy common acknowledgment in our theological literature. Therefore, I
consider it necessary to devote to this question a special appendix at the end of these lectures, where I will touch
upon the significance of the native Egyptian theosophy (the revelations of Thoth or Hermes Trismegistus) in its
relation to the doctrines of [the Logos and the Trinity]. \begin{flushright} \textsc{Vladimir Solovyov}, \emph{Lectures on Divine
Humanity}\end{flushright}

\end{quotationx}
\textbf{Negative Side}

Guidance is sought in the subconscious, out of the blood (Balaam) on the false magical path (Balac). The Nicolaites
strain place the ego inside the body, as a substitute for the Real Self.

\textbf{Positive Side}

\begin{quotationx}
Because in the depths of the unconscious — which knocks at the door and wants to become conscious
— there is present the “sanctuary of the everlasting zones”, where the “Sacred Book of Thoth”
remains deposited, from whence symbolic and Hermetic works are born, or reincarnate. The Tarot is such a work. \begin{flushright}
\emph{Meditations on the Tarot. Letter X: The Wheel of Fortune}\end{flushright}

\end{quotationx}
To retain the forces of the Manas, in order to preserve one's name, instead of sinking in the blood
bonds. The “name” refers to the immortal “I”; the spiritual impulse in this cultural epoch is to strive for the
experience of immortal individuality. The I is inscribed on a white stone as a force of community building.

He, that hath an ear, let him hear what the Spirit saith to the churches: To him that overcometh, I will give the hidden
manna, and will give him a white counter, and in the counter, a new name written, which no man knoweth, but he that
receiveth it. (Rev 2:17)

\begin{quotationx}
Then he (Saoshyant) shall restore the world, which will thenceforth never grow old and never die, never decay and never
perish, ever live and ever increase, and be master over its wish, when the dead will rise, when life and immortality
will come, and the world will be restored at God's wish. \begin{flushright} \textsc{Zoroaster}\end{flushright}

\end{quotationx}

\subsection*{Day 4: Message to Thyatira}
\textbf{Reading: Matthew 24, Rev 2:18-29}

\emph{The fourth message to the church of Thyatira concerns the Greco-Roman culture.}

\begin{quotationx}
The “pagan” initiates and philosophers knew of the unique God — the creator and supreme Good of the
world. The difference between the religion of the so-called “pagan” initiates and philosophers and that of Moses is
simply the fact that the latter made monotheism a popular religion, whilst the former reserved it for the elite, for
the spiritual aristocracy …  With respect to the cult of the “gods” and the iconolatry that this cult entailed, the
“pagan” initiates and philosophers saw in it the practice of theurgy, i.e., that of intercourse with entities of the
celestial hierarchies either by raising themselves to them, or by rendering possible their descent and presence on
earth … It goes without saying that the “paganism” of the initiates and sages, when not degenerated, had nothing to do
with the cult of collectively engendered demons. … Its “gods” were, truth to tell, human personages
— heroes and heroines, divinised or poerised, who were prototypes of the development of the human
personality, i.e., planetary and zodiacal types. Thus Jupiter, Juno, Mars, Venus, Mercury, Diana, Apollo, etc., were
not at all demons, but leading prototypes of the development of the human personality who, in their turn, corresponded
to cosmic — planetary and zodiacal — principles. … “naturalistic paganism”
was “cosmolatry”, i.e. it did not go beyond the limits of Nature like natural science today. It was, therefore,
“neutral” from the point of view both of the true spiritual world and of the demons. \begin{flushright} \emph{Meditations on the
Tarot. Letter XV: The Devil}\end{flushright}

\end{quotationx}

And to the angel of the church of Thyatira write: These things saith the Son of God, who hath his eyes like to a flame
of fire, and his feet like to fine brass. I know thy works, and thy faith, and thy charity, and thy ministry, and thy
patience, and thy last works which are more than the former.

But I have against thee a few things: because thou sufferest the woman Jezebel, who calleth herself a prophetess, to
teach, and to seduce my servants, to commit fornication, and to eat of things sacrificed to idols. And I gave her a
time that she might do penance, and she will not repent of her fornication. Behold, I will cast her into a bed: and
they that commit adultery with her shall be in very great tribulation, except they do penance from their deeds. And I
will kill her children with death, and all the churches shall know that I am he that searcheth the reins and hearts,
and I will give to every one of you according to your works.

But to you I say, and to the rest who are at Thyatira: Whosoever have not this doctrine, and who have not known the
depths of Satan, as they say, I will not put upon you any other burden. Yet that, which you have, hold fast till I
come. And he that shall overcome, and keep my works unto the end, I will give him power over the nations. And he shall
rule them with a rod of iron, and as the vessel of a potter they shall be broken, as I also have received of my Father:
and I will give him the morning star. (Rev 2:18-28)

\textbf{Negative Side}

\begin{quotationx}
The fourth form of paganism is that of the worship of collectively engendered demons. This form of paganism, which is
due to the degeneration of the other three forms is the only form of paganism where demons were engendered, worshipped
and obeyed, and which led to the whole of paganism being renamed unjustly and calumniously as the “demoniacal
religion”. \begin{flushright} \emph{Meditations on the Tarot. Letter XV: The Devil}\end{flushright}

\end{quotationx}

There is a confused life in Thyatira. Fornication is a serious transgression, is not compromise, it is a false surrender
to everything. It is the failure to keep to what one has said. The not wanting to say yes or no. Decadent false
prophets, false devotion to everything without wishing to make a choice. They get involved with Sybilline oracles
through Jezebel.

\textbf{Positive Side}

The force of the I, which has backbone in the spirit. The upright posture in the new spirit, the spirit of Christ. They
show the way to overcome the curses of toil (through service), suffering (through patience), and death (through faith).
And love (charity) is the fourth quality.

He that hath an ear, let him hear what the Spirit saith to the churches. (Rev 2:29)

\subsection*{Day 5: Message to Sardis}
\textbf{Reading: Luke 21, Rev 3:1-6}

\emph{The fifth message to the church of Sardis concerns the Anglo-Germanic culture.}

\begin{quotationx}
The historian of the future, if he has discerned the difference between the way, the truth, and the life on the one hand
and the stream of natural evolution on the other, will not compose a history of civilization —
that is, the story of technological progress and socio-political struggles — but will trace the
path of mankind through the stages of purification and illumination to its ultimate attainment of perfection. His
narrative will detail mankind's temptations and their vanquishment, the standards set by
particular individuals and groups, and the progressive lighting-up of new insights and the awakening of spiritual
faculties among human beings. \begin{flushright} \emph{Covenant of the Heart}
\end{flushright}
\end{quotationx}

And to the angel of the church of Sardis, write: These things saith he, that hath the seven spirits of God, and the
seven stars:

I know thy works, that thou hast the name of being alive: and thou art dead. Be watchful and strengthen the things that
remain, which are ready to die. For I find not thy works full before my God. Have in mind therefore in what manner thou
hast received and heard: and observe, and do penance. If then thou shalt not watch, I will come to thee as a thief, and
thou shalt not know at what hour I will come to thee.

But thou hast a few names in Sardis, which have not defiled their garments: and they shall walk with me in white,
because they are worthy. He that shall overcome, shall thus be clothed in white garments, and I will not blot out his
name out of the book of life, and I will confess his name before my Father, and before his angels. (Rev 3:1-5)

\textbf{Negative Side}

The church at Sardis was vice-ridden and restless. Remaining asleep to all that is spiritual, and affirmation of what is
fated to die. Without God everything is fated to die. Remaining asleep to this mission, just as our entire culture is
without God. Science, art, and religion, without God, are fated to die.

\textbf{Positive Side}

Through the forces of death, to experience an incentive to seek for life and for God. If one has found God and one
breathes in God, then one has life. Be wakeful and conscious of your task.

He that hath an ear, let him hear what the Spirit saith to the churches. (Rev 3:6)

\subsection*{Day 6: Message to Philadelphia}
\textbf{Reading: Luke 21, Rev 3:7-13}

\emph{The sixth message to the church of Philadelphia concerns the Slavic epoch.}

\begin{quotationx}
In this period, culture will be that of the Christ impulse flowing through all humankind — no
longer just a doctrine, but most of all a social force. This culture will have settlements in “all nations,” a bond of
friendly unity among humankind that binds nations and lands together all round the Earth. It will be the fruit of
adjusting the relationship between right and left in the spirit of the experience that results from the trial by
scourging. \begin{flushright} \emph{Christ and Sophia}\end{flushright}

\end{quotationx}
And to the angel of the church of Philadelphia, write: These things saith the Holy One and the true one, he that hath
the key of David; he that openeth, and no man shutteth; shutteth, and no man openeth:

I know thy works. Behold, I have given before thee a door opened, which no man can shut: because thou hast a little
strength, and hast kept my word, and hast not denied my name. Behold, I will bring of the synagogue of Satan, who say
they are Jews, and are not, but do lie. Behold, I will make them to come and adore before thy feet. And they shall know
that I have loved thee.

Because thou hast kept the word of my patience, I will also keep thee from the hour of the temptation, which shall come
upon the whole world to try them that dwell upon the earth. Behold, I come quickly: hold fast that which thou hast,
that no man take thy crown. He that shall overcome, I will make him a pillar in the temple of my God; and he shall go
out no more; and I will write upon him the name of my God, and the name of the city of my God, the new Jerusalem, which
cometh down out of heaven from my God, and my new name. (Rev 3:7-12)

\textbf{Negative Side}

This message points to a weakness of the will and fear of the world. One loses one's hold on things
and becomes passive. One's own will comes to a halt and falls asleep.

\textbf{Positive Side}

To stand like a pillar in the temple \&mdash; a stage of consciousness in the human being where he stands firmly in God
and in the name of Christ. The crown must be worn actively and with dignity, although the I is already resting in the
Christ. The crown is attained through effort.

He that hath an ear, let him hear what the Spirit saith to the churches. (Rev 3:13)

\subsection*{Day 7: Message to Laodicea}
\textbf{Reading: Luke 21, Rev 3:14-22}

\emph{The seventh message to the church of Philadelphia concerns the American epoch.}

\begin{quotationx}
The seventh cultural epoch (called Laodicea in the Apocalypse), on the other hand, will have as its main destiny the
fight between denial of the future, or hopelessness, and the Christian affirmation of the future's
resurrection, or hope. \begin{flushright} \emph{Christ and Sophia}\end{flushright}

\end{quotationx}
And to the angel of the church of Laodicea, write: These things saith the Amen, the faithful and true witness, who is
the beginning of the creation of God:

I know thy works, that thou art neither cold, nor hot. I would thou wert cold, or hot. But because thou art lukewarm,
and neither cold, not hot, I will begin to vomit thee out of my mouth. Because thou sayest: I am rich, and made
wealthy, and have need of nothing: and knowest not, that thou art wretched, and miserable, and poor, and blind, and
naked. I counsel thee to buy of me gold fire tried, that thou mayest be made rich; and mayest be clothed in white
garments, and that the shame of thy nakedness may not appear; and anoint thy eyes with eye salve, that thou mayest see.

Such as I love, I rebuke and chastise. Be zealous therefore, and do penance. Behold, I stand at the gate, and knock. If
any man shall hear my voice, and open to me the door, I will come in to him, and will sup with him, and he with me. To
him that shall overcome, I will give to sit with me in my throne: as I also have overcome, and am set down with my
Father in his throne. (Rev 3:14-22)

\textbf{Negative Side}

There is restlessness and arrogance in Laodicea. Feeling rich in the treasures gained in the past and resting upon
these, with no further striving. The opposite of begging for spirit. Neither cold nor warm, but lukewarm.

\textbf{Positive Side}

To strive continually onwards and in humility to give away even the highest one has attained, so that it can become an
organ for what is still higher. The greatest wealth, without knowledge of it. Renunciation of what one has attained and
to do so eagerly.

He that hath an ear, let him hear what the Spirit saith to the churches. (Rev 3:22)

\flright{\small\textit{Posted on 2023-03-26 by Cologero}}
\section{Week 8: Give us this day our daily bread} 

\textsc{Task for the week}: Meditate on the good communion, be wary of the bad communion.

\textsc{Reading for the week}: John 13-17

\subsection*{Day 1: Spiritual-physical nourishment}
\textbf{The Good Communion}

And God blessed them, saying: Increase and multiply, and fill the earth, and subdue it, and rule over the fishes of the
sea, and the fowls of the air, and all living creatures that move upon the earth.  And God said: Behold I have given
you every herb bearing seed upon the earth, and all trees that have in themselves seed of their own kind, to be your
meat: And to all beasts of the earth, and to every fowl of the air, and to all that move upon the earth, and wherein
there is life, that they may have to feed upon. And it was so done. 

Genesis 1:28-30

Through spiritual-physical nourishment (the seed forces), Adam connected himself with all the good forces. He mastered
the active forces.

Seed forces arise from the root chakra on the Tree of Life.

\begin{quotationx}
Chastity is the state of the human being where the centre named in occidental esotericism as the “twelve-petalled lotus”
is awakened and becomes the sun of the microcosmic “planetary system”. The three lotus-centres situated below it (the
ten-petalled, the six-petalled, and the four-petalled) begin then to function in conformity with the life of the heart
(the twelve-petalled lotus), i.e., “according to solar law”. \begin{flushright} \emph{Meditations on the Tarot. Letter V: The Pope}\end{flushright}

\end{quotationx}
The four-petalled lotus is that of creative force.

\textbf{The Evil Communion}

The evil seed-force was forbidden. That was the origin of the good and evil communions.  It destroyed the petals of the
root chakra, or life forces. This is the forbidden fruit.

\begin{quotationx}
Concerning the centre from which the current necessary for “Simonian” levitation is made to emanate: it is that of the
“four-petalled lotus” (\emph{muladhara chakra}), where the “serpent power” (\emph{kundalini}) —
latent electrical force — is found. Now, this “serpent power” can be awoken and directed either
above (yoga) or below and outside (arbitrary magic). \begin{flushright} \emph{Meditations on the Tarot. Letter XII: The Hanged Man}\end{flushright}

\end{quotationx}

\subsection*{Day 2: Abraham's Initiation}
\textbf{The Good Communion}

And Sarai said to Abram: Thou dost unjustly with me: I gave my handmaid into thy bosom, and she perceiving herself to be
with child, despiseth me. The Lord judge between me and thee. (Genesis 16:5)

Abram was confronted with the choice between good and evil communion.

Abram = sublime father; Abraham = father of a multitude.

But Melchizedek the king of Salem, bringing forth bread and wine, for he was the priest of the most high God, Blessed
him, and said: Blessed be Abram by the most high God, who created heaven and earth. 

Genesis 14:18-19

Melchizedek = king of righteousness; Salem = king of peace. Abraham was initiated with bread and wine.

For this Melchizedek was king of Salem, priest of the most high God, who met Abraham returning from the slaughter of the
kings, and blessed him: To whom also Abraham divided the tithes of all: who first indeed by interpretation, is king of
justice: and then also king of Salem, that is, king of peace: Without father, without mother, without genealogy, having
neither beginning of days nor end of life, but likened unto the Son of God, continueth a priest forever. Now consider
how great this man is, to whom also Abraham the patriarch gave tithes out of the principal things. 

Hebrews 7:1-4

\begin{quotationx}
Primordial water penetrated by divine breath is the essence of blood; breath reflected by the water is light; the
rhythmic alternation from absorption of the breath by water to its reflection by it is respiration. Light is the day,
blood is the night, and respiration is plenitude (Salem). Melchizedek, king of Salem, priest of the Most High God is
therefore appointed to plenitude, to respiration, whilst the anointed king, guardian of the throne of David, or the
Emperor, is appointed to the day. \begin{flushright} \emph{Meditations on the Tarot. Letter V: The Pope}\end{flushright}

\end{quotationx}

\textbf{The Evil Communion}

And the king of Sodom said to Abram: Give me the persons, and the rest take to thyself. And he answered him: I lift up
my hand to the Lord God the most high, the possessor of heaven and earth, that from the very woof thread unto the shoe
latchet, I will not take of any things that are thine, lest thou say I have enriched Abram: Except such things as the
young men have eaten, and the shares of the men that came with me, Aner, Escol, and Mambre: these shall take their
shares. (Genesis 14:21-24)

Abraham refused the evil offer to become king of Sodom and Gomorrah.

The king of Sodom offers Abram bread (the provision for the two cities) and stone (the conquered cities of the enemy kings).

\begin{quotationx}
If ten righteous men had been found in Sodom and Gomorrah, God would have spared these cities. \begin{flushright} \emph{Meditations on
the Tarot. Letter X: The Wheel of Fortune}\end{flushright}

\end{quotationx}

\subsection*{Day 3: Moses feeds his people}
\textsc{The Good Communion}:
It is a good communion when Moses feeds his people with Manna.

\begin{multicols}{2}\small
And the Lord said to Moses: Behold I will rain bread from heaven for you: let the people go forth, and gather what is
sufficient for every day: that I may prove them whether they will walk in my law, or not.  But the sixth day let them
provide for to bring in: and let it be double to that they were wont to gather every day. \flright{\itshape Exodus 16:4-5}

So it came to pass in the evening, that quails coming up, covered the camp: and in the morning, a dew lay round about
the camp. And when it had covered the face of the earth, it appeared in the wilderness small, and as it were beaten
with a pestle, like unto the hoar frost on the ground. And when the children of Israel saw it, they said one to
another: Manhu! which signifieth: What is this! for they knew not what it was. And Moses said to them: This is the
bread, which the Lord hath given you to eat. This is the word, that the Lord hath commanded: Let everyone gather of it
as much as is enough to eat: a gomor for every man, according to the number of your souls that dwell in a tent, so
shall you take of it. And the children of Israel did so: and they gathered, one more, another less. And they measured
by the measure of a gomor: neither had he more that had gathered more: nor did he find less that had provided less: but
everyone had gathered, according to what they were able to eat. \flright{\itshape Exodus 16:13-18}

They said therefore to him: What sign therefore dost thou shew, that we may see, and may believe thee? What dost thou
work? Our fathers did eat manna in the desert, as it is written: He gave them bread from heaven to eat. Then Jesus said
to them: Amen, amen I say to you; Moses gave you not bread from heaven, but my Father giveth you the true bread from
heaven. For the bread of God is that which cometh down from heaven, and giveth life to the world. They said therefore
unto him: Lord, give us always this bread. And Jesus said to them: \textbf{I am the bread of life}: he that cometh to
me shall not hunger: and he that believeth in me shall never thirst. \flright{\itshape John 6:30-35}
\end{multicols}

\begin{quotationx}
St. John of the Cross understands the way of purification through the inner desert as a path for mature souls who are
strong enough to bear the tediousness, stillness, and loneliness of the “night of the spirit.” Nevertheless, these
souls also have need of being strengthened, encouraged, and cheered up. It is one of the experiences that every human
being who treads the “desert path” knows, that at night, in the condition of sleep, something happens that gives him
again and again new strength to endure and not to despair. The night, which seems to be just as desert-like, unmoving,
and dark as the “night of the spirit” experienced during the waking consciousness of day, in time transforms itself
into a giver of strength of soul and courage of spirit for the human being. It is as if something received as a kind of
aftereffect from the heavenly choirs of the spiritual hierarchies was actively giving new life and strength to the soul
(and often to the body, too) of the awakened wanderer. The result is renewed courage for life and the temporary
disappearing of any lack of hope. This strengthening is not due to a dream or any kind of instruction during dreaming,
but purely and simply to the condition resulting during the night. Despondency, or even despair, simply disappears by
itself, and one is reinvigorated and strengthened to continue the “path through the desert.” It is not arbitrary to
compare this aftereffect of the night with the miraculous nourishment of \textbf{manna}, to see here an analogy with
this feeding of the chosen people in the desert during their wandering through the wilderness. \begin{flushright} \emph{Covenant of
the Heart}\end{flushright}

\end{quotationx}

\textsc{The Evil Communion}:
Evil communion is the worship of the golden calf and yearning for the fleshpots of Egypt.

The golden calf is imitated in the worship of those who attend to external life without awakening to life, i.e., without
making an effort to acquire exact knowledge of the laws of life.

Now, there is the Word, and there are egregores before whom humanity bows down; there is revelation of divine truth, and
the manifestation of the will of human beings; there is the cult of God, and that of idols made by man. Is it not a
diagnosis and prognosis of the whole history of the human race that at the same time that Moses received the revelation
of the Word at the summit of the mountain, the people at the foot of the mountain made and worshipped a golden calf?

\begin{quotationx}
The Word and idols, revealed truth and “ideological superstructures” of the human will, operate simultaneously in the
history of the human race. Has there been a single century when the servants of the Word have not had to confront the
worshippers of idols, egregores? \begin{flushright} \emph{Meditations on the Tarot. Letter XV: The Devil}\end{flushright}

\end{quotationx}

Those at the foot of the mountain collected contributions of gold jewelry and made from them the so-called golden calf,
the idol of a golden bull. With respect to this archetypal phenomenon of “falling away,” it is not a question merely of
the victory of inclination—preferring the sense perceptible and material to what is supersensible
and purely moral—but of something deeper and more significant. It is actually a matter of an
insurrection of the collective will of the people asserting itself against the aristocratic, hierarchical order that
Moses represented.

The turning away from the God who was revealed and proclaimed, in favor of the self-chosen and created god is, as it is
depicted in the Bible, the original phenomenon of all stages and forms of falling away from the truths of revelation in
favor of the collective will of the people, which usually comes to expression as the demand for being in tune with the
“spirit of the times.”

The “golden calf” was made and took the place of the dogma of the God of Being. The “golden calf” came about not through
doubt in the revelation proclaimed through Moses, but to make it plausible.

\begin{quotationx}
When Yahweh wanted to destroy the people of Israel because of their turning away from Him and on account of their
worship of the golden calf at the foot of Mt. Sinai, Moses asked that he, instead, be blotted out from the book of life
rather than that the people of Israel be destroyed. Thus he attained pardon for the people of Israel. In his
sacrificial willingness a deeper understanding of the divine expressed itself. \begin{flushright} \emph{Covenant of the Heart}\end{flushright}

\end{quotationx}

\begin{multicols}{2}\small
And the people seeing that Moses delayed to come down from the mount, gathering together against Aaron, said: Arise,
make us gods, that may go before us: for as to this Moses, the man that brought us out of the land of Egypt, we know
not what has befallen him. And Aaron said to them: Take the golden earrings from the ears of your wives, and your sons
and daughters, and bring them to me. And the people did what he had commanded, bringing the earrings to Aaron. And when
he had received them, he fashioned them by founders' work, and made of them a molten calf. And
they said: These are thy gods, O Israel, that have brought thee out of the land of Egypt. And when Aaron saw this, he
built an altar before it, and made proclamation by a crier's voice, saying: Tomorrow is the
solemnity of the Lord. And rising in the morning, they offered holocausts, and peace victims, and the people sat down
to eat, and drink, and they rose up to play. And the Lord spoke to Moses, saying: Go, get thee down: thy people, which
thou hast brought out of the land of Egypt, hath sinned. They have quickly strayed from the way which thou didst shew
them: and they have made to themselves a molten calf, and have adored it, and sacrificing victims to it, have said:
These are thy gods, O Israel, that have brought thee out of the land of Egypt. And again the Lord said to Moses: See
that this people is stiffnecked: Let me alone, that my wrath may be kindled against them, and that I may destroy them,
and I will make of thee a great nation. \flright{\itshape Exodus 32:1-10}
\end{multicols}

\subsubsection*{The fleshpots of Egypt}

The exodus out of Egypt and the wandering in the desert preceded the revelation on Mt. Sinai. For Egypt was the epitome
of all kinds of worship of the elements of existence in space (sun, moon, and stars) and in time (fertility, the power
of procreation, life and death, natural evolution), i.e., those very things that work as necessities of nature and
represent what is coercive about worldly existence.

\begin{quotationx}
Egypt was the “house of bondage” because of the worship of the compelling necessities of existence, the “gods” of
material being. The exodus of the Israelites out of Egypt was for this reason an unprecedented revolutionary event: a
multitude of human beings wanted to go forth into the desert in order to worship there the God who is not to be found
in material existence. \begin{flushright} \emph{Covenant of the Heart}\end{flushright}


And they chode with Moses, and said: Give us water, that we may drink. And Moses answered them: Why chide you with me?
Wherefore do you tempt the Lord? So the people were thirsty there for want of water, and murmured against Moses,
saying: Why didst thou make us go forth out of Egypt, to kill us and our children, and our beasts with thirst? 

\flright{\itshape Exodus 17:2-3}
\end{quotationx}

\subsection*{Day 4: The Feeding of the 5000}
The feeding of the 5000 with five barley loaves and two fishes.

\begin{multicols}{2}\small
After these things Jesus went over the sea of Galilee, which is that of Tiberias. And a great multitude followed him,
because they saw the miracles which he did on them that were diseased. Jesus therefore went up into a mountain, and
there he sat with his disciples. Now the pasch, the festival day of the Jews, was near at hand. When Jesus therefore
had lifted up his eyes, and seen that a very great multitude cometh to him, he said to Philip: Whence shall we buy
bread, that these may eat? And this he said to try him; for he himself knew what he would do. Philip answered him: Two
hundred pennyworth of bread is not sufficient for them, that every one may take a little. One of his disciples, Andrew,
the brother of Simon Peter, saith to him:

There is a boy here that hath five barley loaves, and two fishes; but what are these among so many?

Then Jesus said: Make the men sit down.

Now there was much grass in the place. The men therefore sat down, in number about five thousand. And Jesus took the
loaves: and when he had given thanks, he distributed to them that were set down. In like manner also of the fishes, as
much as they would. And when they were filled, he said to his disciples: Gather up the fragments that remain, lest they
be lost. They gathered up therefore, and filled twelve baskets with the fragments of the five barley loaves, which
remained over and above to them that had eaten. Now those men, when they had seen what a miracle Jesus had done, said:
This is of a truth the prophet, that is to come into the world. Jesus therefore, when he knew that they would come to
take him by force, and make him king, fled again into the mountain himself alone. \flright{\itshape John 6:1-15}
\end{multicols}

\textsc{The Good Communion}:
Drives that spring from love bring about healings. Love is present only where the human being is able to act out of
intuition consciously acquired.

\begin{quotationx}
The seven miracles in John's Gospel did not have the serpent as their agent, nor was the brain the
instrument of their accomplishment, nor was cerebral intellectuality the source of their initiative. The agent here is
the dove, i.e., the Spirit which is above the brain, above the head, and which \emph{descends} upon the head and
remains there — the Spirit which \emph{transcends} cerebral intellectuality. This Spirit is the
source of initiative and, simultaneously, is the agent and instrument of divine or sacred magic. \begin{flushright} \emph{Meditations
on the Tarot. Letter X: The Wheel of Fortune}\end{flushright}

\end{quotationx}

As the Sun — raying out light, warmth, and life — “nourishes” all beings and
unites them in a “community of nourishment,” so Jesus Christ functioned at the feeding of the five thousand as the
“nourishment giving center” for the five thousand. He carried out during the short time of the “feeding” what the sun
effects in the course of the year, when it brings about a “multiplication of bread” through the sprouting, growth, and
ripening of corn.

The disciples received and passed on the blest (eucharistic) bread, just as the moon receives and passes on the light of
the sun in a dimmed, toned-down form.

This mediating effect of the moon, which transforms the boundless, streaming strength of the sun such that it becomes
more individually acceptable, can also be understood in relation to the experience of the sacrament of holy communion
received at the altar.

\textsc{The Evil Communion}:
Feedings that take place without Christ fulfil the purposes of Ahriman. They are feeding out of the heart of Satan and
strengthening the I via currents of the fallen chakras. The feeding always enters into a man when a destruction of the
seed is undertaken out of passion. Ahriman enters their hearts, haunting the life-sphere and destroying the life-forces
in the life-centre of the heart.

\subsection*{Day 5: The Last Supper}
\textsc{The Good Communion}: The Last Supper.

Scriptural readings: Matthew 26:20-30, Mark 14:17-26, Luke 22:7-23

\begin{quotationx}
Teresa Neumann lived in our time at Konnarsreuth (Bavaria) solely from Holy Communion for decades; St. Catherine of
Sienna lived nine years from Holy Communion alone; St. Lidvina of Schiedam (near Rotterdam, Holland) likewise lived for
many years exclusively from Holy Communion —to cite only the cases that are well-verified.

This is the significance of the words: “Man shall not live by bread alone but by every word that proceeds from the mouth
of God.” Here is its principal implication: as the law of evolution, the law of the serpent, comprises the struggle for
existence and as “bread” or food is the principal factor in the struggle for existence, the fact of the entry of grace
into human history since Jesus Christ signifies at the same time the possibility of gradually abolishing the struggle
for existence. \begin{flushright} \emph{Meditations on the Tarot. Letter VI: The Lover}\end{flushright}

\end{quotationx}
Christ's last words at the institution of the holy sacrament at the Last Supper: “Do this in memory
of me” point toward the sacraments, too, as being a reenlivening in the present of what happened in the past. In the
holy sacrament at the altar, memory becomes an act of the divine magic of transubstantiation, an act relating to the
real (not just remembered) presence of the body and blood of the Redeemer. What once took place, takes place now in the
present. In the sacrament, memory does not become a journey into the past, but instead brings the past into present, an
evocation that summons something out of the realm of forgetting, sleep, and death.

\begin{quotationx}
The words: “Do this in memory of me” actually mean: “Do this, so that I may be present.” \begin{flushright} \emph{Covenant of the
Heart}\end{flushright}

\end{quotationx}

\textsc{The Evil Communion:} The temptation of Jesus in the wilderness.

Scriptural Reading: Matthew 4:1-11

The temptation to Materialism:

\begin{quotationx}
And when he had fasted forty days and forty nights, afterwards he was hungry. And the tempter coming said to him: If
thou be the Son of God, command that these stones be made bread. Who answered and said: It is written, Not in bread
alone doth man live, but in every word that proceedeth from the mouth of God. \flright{\itshape Matthew 4:2-4}
\end{quotationx}

\subsubsection*{Bible Passages}
\begin{multicols}{2}\small
But when it was evening, he sat down with his twelve disciples. And whilst they were eating, he said: Amen I say to you,
that one of you is about to betray me. And they being very much troubled, began everyone to say: Is it I, Lord? But he
answering, said: He that dippeth his hand with me in the dish, he shall betray me. The Son of man indeed goeth, as it
is written of him: but woe to that man by whom the Son of man shall be betrayed: it were better for him, if that man
had not been born. And Judas that betrayed him, answering, said: Is it I, Rabbi? He saith to him: Thou hast said it.
And whilst they were at supper, Jesus took bread, and blessed, and broke: and gave to his disciples, and said: Take ye,
and eat. This is my body. And taking the chalice, he gave thanks, and gave to them, saying: Drink ye all of this.  For
this is my blood of the new testament, which shall be shed for many unto remission of sins. And I say to you, I will
not drink from henceforth of this fruit of the vine, until that day when I shall drink it with you new in the kingdom
of my Father. And a hymn being said, they went out unto mount Olivet. \flright{\itshape Matthew 26:20-30}

And when evening was come, he cometh with the twelve. And when they were at table and eating, Jesus saith: Amen I say to
you, one of you that eateth with me shall betray me. But they began to be sorrowful, and to say to him one by one: Is
it I?  Who saith to them: One of the twelve, who dippeth with me his hand in the dish. And the Son of man indeed goeth,
as it is written of him: but woe to that man by whom the Son of man shall be betrayed. It were better for him, if that
man had not been born. And whilst they were eating, Jesus took bread; and blessing, broke, and gave to them, and said:
Take ye. This is my body. And having taken the chalice, giving thanks, he gave it to them. And they all drank of it.
And he said to them: This is my blood of the new testament, which shall be shed for many. Amen I say to you, that I
will drink no more of the fruit of the vine, until that day when I shall drink it new in the kingdom of God. And when
they had said an hymn, they went forth to the mount of Olives. \flright{\itshape Mark 14:17-26}

And the day of the unleavened bread came, on which it was necessary that the pasch should be killed. And he sent Peter
and John, saying: Go, and prepare for us the pasch, that we may eat. But they said: Where wilt thou that we prepare?
And he said to them: Behold, as you go into the city, there shall meet you a man carrying a pitcher of water: follow
him into the house where he entereth in. And you shall say to the goodman of the house: The master saith to thee, Where
is the guest chamber, where I may eat the pasch with my disciples? And he will shew you a large dining room, furnished;
and there prepare. And they going, found as he had said to them, and made ready the pasch. And when the hour was come,
he sat down, and the twelve apostles with him. And he said to them: With desire I have desired to eat this pasch with
you, before I suffer. For I say to you, that from this time I will not eat it, till it be fulfilled in the kingdom of
God. And having taken the chalice, he gave thanks, and said: Take, and divide it among you: For I say to you, that I
will not drink of the fruit of the vine, till the kingdom of God come. And taking bread, he gave thanks, and brake; and
gave to them, saying: This is my body, which is given for you. Do this for a commemoration of me. In like manner the
chalice also, after he had supped, saying: This is the chalice, the new testament in my blood, which shall be shed for
you. But yet behold, the hand of him that betrayeth me is with me on the table. And the Son of man indeed goeth,
according to that which is determined: but yet, woe to that man by whom he shall be betrayed. And they began to inquire
among themselves, which of them it was that should do this thing. \flright{\itshape Luke 22:7-23}

Then Jesus was led by the spirit into the desert, to be tempted by the devil. And when he had fasted forty days and
forty nights, afterwards he was hungry. And the tempter coming said to him: If thou be the Son of God, command that
these stones be made bread. Who answered and said: It is written, Not in bread alone doth man live, but in every word
that proceedeth from the mouth of God. Then the devil took him up into the holy city, and set him upon the pinnacle of
the temple, And said to him: If thou be the Son of God, cast thyself down, for it is written: That he hath given his
angels charge over thee, and in their hands shall they bear thee up, lest perhaps thou dash thy foot against a stone.
Jesus said to him: It is written again: Thou shalt not tempt the Lord thy God. Again the devil took him up into a very
high mountain, and shewed him all the kingdoms of the world, and the glory of them, And said to him: All these will I
give thee, if falling down thou wilt adore me. Then Jesus saith to him: Begone, Satan: for it is written, The Lord thy
God shalt thou adore, and him only shalt thou serve. Then the devil left him; and behold angels came and ministered to
him. \flright{\itshape Matthew 4:1-11}
\end{multicols}

\subsection*{Day 6: Instituting the Eucharist}
Meditation on Luke 22:7-38.

\textbf{The Good Communion: The Eucharist}

\begin{quotationx}
The Eucharist is the whole of Christianity; and through it Christianity has become living magic. Since Jesus there are
still sorcerers, (but) there are no more mages. \begin{flushright} \textsc{Josephin Peladan}\end{flushright}

\end{quotationx}

On the day of the unleavened bread, Jesus and the disciples had a meal in a large, furnished dining room.

The Jesus established the Eucharist with the bread and wine.

\textbf{The Evil Communion: betrayal, strife, and denial}

Although participated in the first Eucharist, there was an evil side.

\begin{itemize}
\item \textsc{Betrayal}: Jesus predicted Judas' betrayal 
\item \textsc{Strife}: The disciples bickered among themselves for positions of power and authority 
\item \textsc{Denial}: Jesus predicted Peter's denial 
\end{itemize}

\subsubsection*{Bible passages: The Last Supper}
\begin{multicols}{2}\small
And the day of the unleavened bread came, on which it was necessary that the pasch should be killed. And he sent Peter
and John, saying: Go, and prepare for us the pasch, that we may eat. But they said: Where wilt thou that we prepare?
And he said to them: Behold, as you go into the city, there shall meet you a man carrying a pitcher of water: follow
him into the house where he entereth in. And you shall say to the goodman of the house: The master saith to thee, Where
is the guest chamber, where I may eat the pasch with my disciples? And he will shew you a large dining room, furnished;
and there prepare. And they going, found as he had said to them, and made ready the pasch. And when the hour was come,
he sat down, and the twelve apostles with him. And he said to them: With desire I have desired to eat this pasch with
you, before I suffer. For I say to you, that from this time I will not eat it, till it be fulfilled in the kingdom of
God.

And having taken the chalice, he gave thanks, and said: Take, and divide it among you: For I say to you, that I will not
drink of the fruit of the vine, till the kingdom of God come. And taking bread, he gave thanks, and brake; and gave to
them, saying: This is my body, which is given for you. Do this for a commemoration of me. In like manner the chalice
also, after he had supped, saying: This is the chalice, the new testament in my blood, which shall be shed for you.

But yet behold, the hand of him that betrayeth me is with me on the table. And the Son of man indeed goeth, according to
that which is determined: but yet, woe to that man by whom he shall be betrayed.  And they began to inquire among
themselves, which of them it was that should do this thing. And there was also a strife amongst them, which of them
should seem to be the greater. And he said to them: The kings of the Gentiles lord it over them; and they that have
power over them, are called beneficent. But you not so: but he that is the greater among you, let him become as the
younger; and he that is the leader, as he that serveth. For which is greater, he that sitteth at table, or he that
serveth? Is it not he that sitteth at table? But I am in the midst of you, as he that serveth: And you are they who
have continued with me in my temptations:

And I dispose to you, as my Father hath disposed to me, a kingdom; that you may eat and drink at my table, in my
kingdom: and may sit upon thrones, judging the twelve tribes of Israel. And the Lord said: Simon, Simon, behold Satan
hath desired to have you, that he may sift you as wheat: But I have prayed for thee, that thy faith fail not: and thou,
being once converted, confirm thy brethren. Who said to him: Lord, I am ready to go with thee, both into prison, and to
death. And he said: I say to thee, Peter, the cock shall not crow this day, till thou thrice deniest that thou knowest
me. And he said to them: When I sent you without purse, and scrip, and shoes, did you want anything? But they said:
Nothing. Then said he unto them: But now he that hath a purse, let him take it, and likewise a scrip; and he that hath
not, let him sell his coat, and buy a sword. For I say to you, that this that is written must yet be fulfilled in me:
And with the wicked was he reckoned. For the things concerning me have an end. But they said: Lord, behold here are two
swords. And he said to them, It is enough. \flright{\itshape Luke 22:7-38}
\end{multicols}

\subsubsection*{The Temptations in the Desert}
\begin{multicols}{2}\small
And Jesus being full of the Holy Ghost, returned from the Jordan, and was led by the Spirit into the desert, for the
space of forty days; and was tempted by the devil. And he ate nothing in those days; and when they were ended, he was
hungry.

And the devil said to him: If thou be the Son of God, say to this stone that it be made bread. And Jesus answered him:
It is written, that Man liveth not by bread alone, but by every word of God.

And the devil led him into a high mountain, and shewed him all the kingdoms of the world in a moment of time; and he
said to him: To thee will I give all this power, and the glory of them; for to me they are delivered, and to whom I
will, I give them. If thou therefore wilt adore before me, all shall be thine. And Jesus answering said to him: It is
written: Thou shalt adore the Lord thy God, and him only shalt thou serve.

And he brought him to Jerusalem, and set him on a pinnacle of the temple, and he said to him: If thou be the Son of God,
cast thyself from hence. For it is written, that He hath given his angels charge over thee, that they keep thee. And
that in their hands they shall bear thee up, lest perhaps thou dash thy foot against a stone. And Jesus answering, said
to him: It is said: Thou shalt not tempt the Lord thy God.

And all the temptation being ended, the devil departed from him for a time.

\flright{\itshape Luke 4:1-13}
\end{multicols}

\subsection*{Day 7: The Road to Emmaus}
The state of the risen Jesus Christ is the goal and the hope of the path of destiny of humanity. It is the most perfect
ideal of which one could ever think or dream. For it unites the most far reaching hopes of the noblest ideals of this
world with the highest and deepest ideals of the world beyond.

\textbf{The Good Communion}

And it happened that, while he was with them at table, he took bread, said the blessing, broke it, and gave it to them.
With that their eyes were opened and they recognized him, but he vanished from their sight. (Luke 24:30-31)

\begin{quotationx}
It was always through a sign that the Risen One let himself be known. For the form and appearance of the Risen One
differed from Jesus of Nazareth through the fact of being timeless and ageless. His countenance was simply the
expression of his being—of his spirit and his soul—and could only be
recognised by those who through the sheath of his bodily self had known something of his true being, his soul, and his
spirit. And the signs by which the Risen One let himself be known were such as to give a pointer and an indication
toward their personal earlier experience and knowledge of the soul-spiritual being of Jesus. They were reminded of
their earlier experience and knowledge of the being of Jesus Christ. \begin{flushright} \emph{Covenant of the Heart}\end{flushright}

\end{quotationx}
The Risen One did not appear in the likeness of Jesus, as those nearest to him had known him immediately before the
Crucifixion and earlier; that is, not as the resuscitated or reawakened Jesus who had been crucified, nor the Jesus who
had been baptised in the Jordan—but Jesus in a new form. For this reason, those people who had
known him recognised him only by some intimate sign.

Thus, Mary Magdalena, who at first took him for a gardener, recognised him only by the manner in which he pronounced her
name “Mary.” Thomas recognised him when the Risen One showed him the marks of his wounds. \emph{The two disciples from
Emmaus knew him in the breaking of bread}.

\begin{quotationx}
The heart perceives diverse presences as impressions and nuances of spiritual warmth. It is thus that the hearts of the
two disciples going to Emmaus recognised the One who went on the way with them before their eyes and their
understanding did, and who said to one another after their eyes opened and they recognised him: “Did not our hearts
burn within us while he talked to us on the road, while he opened to us the scriptures?” The heart burning in diverse
ways — this is the kind of “vision” and spiritual knowledge which is proper to the heart. \begin{flushright}
\emph{Meditations on the Tarot. Letter XII: The Hanged Man}\end{flushright}

\end{quotationx}

\textbf{The Evil Communion}

\begin{quotationx}
Tradition lives not thanks to organisations, but rather in spite of them. One should content oneself purely and simply
with friendship in order to preserve the life of a tradition; it is not necessary to entrust it to the care of the
embalmers and mummifiers par excellence that organisations are, except for the one founded by Jesus Christ. \begin{flushright}
\emph{Meditations on the Tarot. Letter XX: Judgment}\end{flushright}

\end{quotationx}
The mummification of the Pharaohs in ancient Egypt.

\begin{quotationx}
There will come a time when it will be seen that in vain have the Egyptians honoured the deity with heartfelt piety and
assiduous service; and all our holy worship will be found bootless and ineffectual. For the gods will return from earth
to heaven; Egypt will be forsaken, and the land which was once the home of religion will be left desolate, bereft of
the presence of its deities. This land and region will be filled with foreigners. . . In that day will our most holy
land, this land of shrines and temples, be filled with funerals and corpses. \begin{flushright} \textsc{Hermes Trismegistus:} \emph{Asclepius}\end{flushright} 

\end{quotationx}
\textbf{Bible Passage}

\begin{multicols}{2}\small
And behold, two of them went, the same day, to a town which was sixty furlongs from Jerusalem, named Emmaus. And they
talked together of all these things which had happened. And it came to pass, that while they talked and reasoned with
themselves, Jesus himself also drawing near, went with them. But their eyes were held, that they should not know him.
And he said to them: What are these discourses that you hold one with another as you walk, and are sad?

And the one of them, whose name was Cleophas, answering, said to him: Art thou only a stranger to Jerusalem, and hast
not known the things that have been done there in these days? To whom he said: What things? And they said: Concerning
Jesus of Nazareth, who was a prophet, mighty in work and word before God and all the people; and how our chief priests
and princes delivered him to be condemned to death, and crucified him. But we hoped, that it was he that should have
redeemed Israel: and now besides all this, today is the third day since these things were done.

Yea and certain women also of our company affrighted us, who before it was light, were at the sepulchre. And not finding
his body, came, saying, that they had also seen a vision of angels, who say that he is alive. And some of our people
went to the sepulchre, and found it so as the women had said, but him they found not.

Then he said to them: O foolish, and slow of heart to believe in all things which the prophets have spoken. Ought not
Christ to have suffered these things, and so to enter into his glory? And beginning at Moses and all the prophets, he
expounded to them in all the scriptures, the things that were concerning him. And they drew nigh to the town, whither
they were going: and he made as though he would go farther.

But they constrained him; saying: Stay with us, because it is towards evening, and the day is now far spent. And he went
in with them. And it came to pass, whilst he was at table with them, he took bread, and blessed, and brake, and gave to
them. And their eyes were opened, and they knew him: and he vanished out of their sight. And they said one to the
other: Was not our heart burning within us, whilst he spoke in this way, and opened to us the scriptures?

And rising up, the same hour, they went back to Jerusalem: and they found the eleven gathered together, and those that
were staying with them, saying: The Lord is risen indeed, and hath appeared to Simon. And they told what things were
done in the way; and how they knew him in the breaking of the bread. \flright{\itshape Luke 24:13-35}
\end{multicols}

\flright{\small\textit{Posted on 2023-04-02 by Cologero}}
\section{Week 9: Forgive Us our Trespasses}
 

As we forgive those who trespass against us.

\begin{quotationx}
Each miracle of the Gospel of St. John is simultaneously a \textbf{teaching}, a \textbf{parable}, a \textbf{sign}, and
an \textbf{event}. \begin{flushright} \emph{Covenant of the Heart}\end{flushright}

\end{quotationx}
\textbf{Meditation}: The seven miracles in John's gospel

These represent 7 stages of healing: the 7 stages of guilt and their forgiveness.

\begin{quotationx}
\textbf{Creation and healing have to take place in reverse sequence}: there, where the creation was completed, lies the
starting point for the healing (i.e., restoring) effect. Accordingly, the prologue of the Gospel of St. John also gives
the stages of Moses' account of creation (\textbf{\emph{light, life,
man}}) in the reverse sequence: “In him (the Word) was \textbf{\emph{life, and the life was the
\textbf{light} of \textbf{men}}}” (John 1:1-4). The work of salvation takes place in reverse to that of the creation in
so far as the last stage of the creation is the first of the work of salvation. \begin{flushright} \emph{Covenant of the Heart}\end{flushright}

\end{quotationx}

\subsection*{Day 1: The Wedding at Cana}
On the third day there was a wedding in Cana in Galilee, and the mother of Jesus was there. Jesus and his disciples were
also invited to the wedding. When the wine ran short, the mother of Jesus said to him, “They have no wine.” And Jesus
said to her, “Woman, how does your concern affect me? My hour has not yet come.” His mother said to the servers, “Do
whatever he tells you.” Now there were six stone water jars there for Jewish ceremonial washings, each holding twenty
to thirty gallons. Jesus told them, “Fill the jars with water.” So they filled them to the brim. Then he told them,
“Draw some out now and take it to the headwaiter.” So they took it. And when the headwaiter tasted the water that had
become wine, without knowing where it came from (although the servers who had drawn the water knew), the headwaiter
called the bridegroom and said to him, “Everyone serves good wine first, and then when people have drunk freely, an
inferior one; but you have kept the good wine until now.” Jesus did this as the beginning of his signs in Cana in
Galilee and so \emph{revealed his glory, and his disciples began to believe in him}. \begin{flushright} John 2:1-11\end{flushright}

\begin{itemize}
\item \textbf{Guilt}: doubt 
\item \textbf{Sickness}: blood which had become lukewarm after the Fall 
\item \textbf{Forgiveness}: It is made fresh and vigorous. At the wedding, a new mixing of blood was taking place 
\item \textbf{Name of the Master}: I am the True Vine. 
\end{itemize}
\begin{quotationx}
The wine at the wedding of Cana was not \emph{created from} nothing, but rather it was \emph{water} which was changed
into wine. Let us also note the fact that the Virgin-Mother was not only present at the wedding but also that she took
part in an explicit manner in the miracle of changing the water into wine —since it was thanks to
her initiative that the miracle took place. \begin{flushright} \emph{Meditations on the Tarot. Letter X: Force}\end{flushright}

\end{quotationx}
The healing miracle work of the Word made flesh takes place in the reverse sequence to the creative miracle working of
the divine Word depicted in Genesis. The divine magic of the seventh day becomes the first healing miracle of the Word
made flesh, that of the wedding at Cana.

The free union, in love between God and the world, which was celebrated, blessed, and consecrated on the seventh day of
creation, became broken off through the Fall. The world was unfaithful toward God. And since this divine cosmic union
is mirrored in the marriage relationship, for which it is the ideal and archetype as well as being the meaning of
marriage, the original sickness of the world consists in the breach of the free love union that existed between God and
the world. Analogously, this is mirrored in human life, in the distortion and degeneration of the nature and experience
of marriage. The marriage relationship — as it has become — begins with
enthusiasm, with the “wine” of the honeymoon period, and ends with the “water” of routine habit.

\begin{quotationx}
When there was no more wine, Jesus transformed water into wine, and the second wine was better than the first. Thereby
the miracle of the wedding at Cana was the “sign,” the symbol, and the event of the healing of marriage (healing in the
service of restoring the marriage relationship to correspond to its divine cosmic archetype, which is the seventh day
of creation). \begin{flushright} \emph{Covenant of the Heart}\end{flushright}

\end{quotationx}
\subsection*{Day 2: Healing of the Nobleman's Son}
He came again therefore into Cana of Galilee, where he made the water wine. And there was a certain ruler, whose son was
sick at Capharnaum. He having heard that Jesus was come from Judea into Galilee, went to him, and prayed him to come
down, and heal his son; for he was at the point of death.

Jesus therefore said to him: Unless you see signs and wonders, you believe not.

The ruler saith to him: Lord, come down before that my son die.

Jesus saith to him: Go thy way; thy son liveth. The man believed the word which Jesus said to him, and went his way.

And as he was going down, his servants met him; and they brought word, saying, that his son lived. He asked therefore of
them the hour wherein he grew better. And they said to him: Yesterday, at the seventh hour, the fever left him. The
father therefore knew, that it was at the same hour that Jesus said to him, Thy son liveth; and himself believed, and
his whole house. This is again the second miracle that Jesus did, when he was come out of Judea into Galilee.  \begin{flushright}
John 4:46-54\end{flushright}

\begin{itemize}
\item \textbf{Guilt}: desire 
\item \textbf{Sickness}: the sin of the change of heredity from the vertical to the horizontal 
\item \textbf{Forgiveness}: The father bore the guilt for the son's illness. The father believed
and the son was healed. 
\item \textbf{Name of the Master}: I am the Way, the Truth, the Life. 
\end{itemize}
\textbf{The sixth day of Creation}

The sixth day of creation in Genesis is an account of the origin of the animal kingdom and the human being,
corresponding to their archetype:

\begin{itemize}
\item The archetype of the human being is God: God created humans in God's own image, in the image
of God, God created them. 
\item The animals were brought forth from the earth formed by God, each according to its kind. 
\end{itemize}
\textbf{Heredity}

Heredity means the transmission of similarity from ancestors to their descendants. The archetypal nature of heredity is
vertical: the archetype is above and the form mirroring it is shaped below, on earth. Thus, the invisible divinely
created archetypes are the “ancestors” of the visible species of animals. And the invisible archetype of man, the
divine being itself, is the “ancestor” of the human being.

The sickness that arose as a tragic consequence of the Fall was a change of direction in the mirroring process of
heredity; it changed from being vertical to become horizontal. This meant that similarity no longer descended from the
invisible supratemporal archetype above, but from the visible ancestors in the temporal succession of generations here
below.

\textbf{The Sin of Heredity}

The \emph{sickness} that arose as a tragic consequence of the Fall was a change of direction in the mirroring process of
heredity; it changed from being vertical to become horizontal. This meant that similarity no longer descended from the
invisible supratemporal archetype above, but from the visible ancestors in the temporal succession of generations here
below.

Instead of becoming the direct “image and likeness” of their archetype, human beings and other beings of nature became
formed in the “image and likeness” of their earthly ancestors, thus only indirectly mirroring their archetype. Thereby
heredity, as we know it, became a horizontal stream in the sequence of time, transmitting not only the original
mirroring of the archetype, but also everything that entered into the stream of generations with the Fall and that has
occurred in this stream since the Fall. It has become a stream that also transmits the “sins” of sickness and death,
and through it the “sins of the fathers” have become a reality.

\textbf{The Healing}

The second miracle of the Gospel of St. John, the healing of the nobleman's son, where the healing
of the son took place through the faith of the father, was fulfilled, like the first, at Cana in Galilee. It comprises
the transformation of the relationship father—son (i.e., heredity) from being a stream
transmitting sickness to a stream transmitting healing. The second miracle of the John Gospel is the event, sign, and
teaching, which has to do with the divine archetypal heredity of the sixth day of creation. It has to do with the
distortion of this “vertical” heredity through the Fall, in that the original relationship to the “image and likeness
of God,” as it was on the sixth day of creation, was restored by the father bringing his son into a direct relationship
to the divine archetype—through his faith in Jesus Christ, the new Adam.

\begin{quotationx}
He brought his son into connection with the new Adam, who, in place of himself, entered into the hereditary relationship
“father-son.” Thus the healing of the nobleman's son took place. \begin{flushright} \emph{Covenant of the Heart}\end{flushright}

\end{quotationx}
\subsection*{Day 3: Healing of the Paralyzed Man}
After these things was a festival day of the Jews, and Jesus went up to Jerusalem. Now there is at Jerusalem a pond,
called Probatica, which in Hebrew is named Bethsaida, having five porches. In these lay a great multitude of sick, of
blind, of lame, of withered; waiting for the moving of the water. And an angel of the Lord descended at certain times
into the pond; and the water was moved. And he that went down first into the pond after the motion of the water, was
made whole, of whatsoever infirmity he lay under. And there was a certain man there, that had been eight and thirty
years under his infirmity. Him when Jesus had seen lying, and knew that he had been now a long time, he saith to him:
Wilt thou be made whole?

The infirm man answered him: Sir, I have no man, when the water is troubled, to put me into the pond. For whilst I am
coming, another goeth down before me. Jesus saith to him: Arise, take up thy bed, and walk. And immediately the man was
made whole: and he took up his bed, and walked. And it was the sabbath that day. \begin{flushright} John 5:1-9\end{flushright}

\begin{itemize}
\item \textbf{Guilt}: cutting oneself from the cosmic current 
\item \textbf{Sickness}: paralysis 
\item \textbf{Forgiveness}: The restoration of movement 
\item \textbf{Name of the Master}: I am the door 
\end{itemize}
\textbf{The fifth day of Creation}

The fifth day of creation in Genesis is the account of the waters bringing forth ensouled movement:

\begin{itemize}
\item in the \textbf{horizontal} direction: swarms of living creatures 
\item in the vertical direction: birds that fly above the earth across the firmament of the heavens 
\end{itemize}
\textbf{The Healing}

The result of the third miracle in the Gospel of St. John is the healing of the man who was paralyzed for thirty-eight
years, i.e., the restoration of ensouled faculty of movement to the paralyzed man, who lay there waiting for healing
through the water brought into movement by an angel.

The words of Jesus --\textit{Arise, take up thy bed and walk}-- create ensouled movement in:

\begin{itemize}
\item The vertical direction: \emph{arise} 
\item The horizontal direction: take up they bed and \emph{walk} 
\end{itemize}
The faculty of movement is essentially cosmic, not only according to its effect, where every movement, even the
slightest, exerts an effect ultimately upon the whole world, but also according to the causes stimulating it. For the
human being stands within a stream of cosmic energies—his thoughts in the streams of the thought
world, his feelings in the streams of the world's psychic forces, and his impulses of will are
immersed in the streams of world will energy and are “plugged in” to them.

Someone who cuts himself off from the streams of cosmic energies becomes paralyzed.

\begin{quotationx}
The third miracle of the John Gospel accomplished the “re-plugging in” of this human being into the ensouled movement of
the world, who, through sin, had become cut off from it and thereby paralyzed. The third miracle is the archetype of
the healing effect of the sacrament of the forgiveness of sins, i.e., penance. \begin{flushright} \emph{Covenant of the Heart}\end{flushright}

\end{quotationx}
\subsection*{Day 4: Feeding of the Five Thousand}
After these things Jesus went over the sea of Galilee, which is that of Tiberias. And a great multitude followed him,
because they saw the miracles which he did on them that were diseased. Jesus therefore went up into a mountain, and
there he sat with his disciples. Now the pasch, the festival day of the Jews, was near at hand.

When Jesus therefore had lifted up his eyes, and seen that a very great multitude cometh to him, he said to Philip:
Whence shall we buy bread, that these may eat? And this he said to try him; for he himself knew what he would do.

Philip answered him: Two hundred pennyworth of bread is not sufficient for them, that every one may take a little.

One of his disciples, Andrew, the brother of Simon Peter, saith to him: There is a boy here that hath five barley
loaves, and two fishes; but what are these among so many?

Then Jesus said: Make the men sit down. Now there was much grass in the place. The men therefore sat down, in number
about five thousand. And Jesus took the loaves: and when he had given thanks, he distributed to them that were set
down. In like manner also of the fishes, as much as they would. And when they were filled, he said to his disciples:
Gather up the fragments that remain, lest they be lost.

They gathered up therefore, and filled twelve baskets with the fragments of the five barley loaves, which remained over
and above to them that had eaten. Now those men, when they had seen what a miracle Jesus had done, said: This is of a
truth the prophet, that is to come into the world. \begin{flushright} John 6:1-14\end{flushright}

\begin{itemize}
\item \textbf{Guilt}: egoism 
\item \textbf{Sickness}: cutting oneself from the community 
\item \textbf{Forgiveness}: The reinstatement of the original community of beings 
\item \textbf{Name of the Master}: I am the bread of life 
\end{itemize}
\textbf{Fourth Day of Creation}

The fourth day of creation “stands before” the “play” of the many kinds of spontaneous movement of the fifth day and
“leads” it. For the fourth day of creation is that of the coming into being of those principles of the world orchestra
that direct “time and tempo” — the creation of the “sun, moon, and stars”: “And God made two great
lights: the greater light to rule the day and the lesser light to rule the night; he made stars also; and God set them
in the firmament of the heavens to separate the day from the night and let them be for signs and for seasons and for
days and years”. What are these other than organs of direction, i.e., conductors of time and tempo for the world
orchestra, in accordance with the music score of the stars?

The fourth day of creation is the genesis of that all-embracing world rhythm, in which all beings partake and that
unifies them into a world-embracing community. Consider human consciousness. It does not become chaotic through the
strife of wishes, desires, whims, moods, notions, and countless impulses from without and
within—from fantasy and from memory. Rather, it arranges itself around a central point, the self,
which represents the center of gravity of the soul life, i.e., the permanency of the identity of the personality.

Moreover, aided by the light of reason it works in such a way as to bring order even into the “night” of the
subconscious, leading the whole soul life (conscious and unconscious, or “day” and “night”) in the direction of the
ideals. (Ideals can be likened to “stars,” enabling orientation and pointing the way.)

This “sun” leads and supports the whole—the “day world” and also the “night
world”—in harmony with the world of stars. The sun (or rather the inner nature of the sun) in the
great world corresponds to the creative, leading, and ordering role of the self in the “small world” of the human soul.
The moon in its inner nature corresponds to the rational capacity for reflection, which casts an evaluating light on
the irrational urges of the soul life. The inner nature of the stars in the great world correspond to the ideals that
give direction to human soul life.

The fourth day of creation is the account of the origin of the universal community, of the unity embracing all beings of
the world. It is therefore the divine cosmic archetype of the sacrament upon the altar, that of holy communion.

\textbf{The Fourth Miracle}

The feeding of the five thousand in the wilderness —is the corresponding healing work of the Word
made flesh. This work consisted in the reinstatement of the original community of beings from the fourth day of
creation, and the corresponding ordering given on this day of creation by the sun, moon, and stars. For as the Sun
— raying out light, warmth, and life — “nourishes” all beings and unites them
in a “community of nourishment,” so Jesus Christ functioned at the feeding of the five thousand as the “nourishment
giving center” for the five thousand.

\subsection*{Day 5: Jesus Walks on Water}
The fifth miracle of St. John's Gospel is the “sign” of individual generative power as it came into
being on the third day of creation.

The third day of creation is the “day” of procreation, of the mystery of seed and of growth.

And when evening was come, his disciples went down to the sea. And when they had gone up into a ship, they went over the
sea to Capernaum; and it was now dark, and Jesus was not come unto them.  And the sea arose, by reason of a great wind
that blew. When they had rowed therefore about five and twenty or thirty furlongs, they see Jesus walking upon the sea,
and drawing nigh to the ship, and they were afraid.

But he said to them: \textbf{It is I; be not afraid}. They were willing therefore to take him into the ship; and
presently the ship was at the land to which they were going. \begin{flushright} John 6:16-21\end{flushright}

\begin{itemize}
\item \textbf{Guilt}: lack of faith 
\item \textbf{Sickness}: separation from the instinctual life, the power of the seed principle 
\item \textbf{Forgiveness}: the restoration of pure faith, unsupported by anything but inner certainty 
\item \textbf{Name of the Master}: I am the good shepherd 
\end{itemize}
\textbf{Third Day of Creation}

And God said: Let the earth bring forth grass, plants yielding seed, and fruit trees each bearing fruit after its kind,
whose seed is in itself upon the earth; and it was so. And the earth brought forth grass, and plants yielding seed each
according to its kind, and trees bearing fruit, whose seed was in itself after its kind. \begin{flushright} Genesis 1:11\end{flushright}

\textbf{The third day of creation is the generation of the seed principle, of the principle of potential formative force
becoming actualised and bringing to visible realisation its own inner, invisible shape}. The third day of creation is
the coming into being of the seed principle in the world—that is, not only of the plant world
manifest to us as plants and trees, but also such “trees” growing in paradise as “the tree of knowledge of good and
evil” and the “tree of life.” Also included here is all that grew out of the “seed of Abraham,” and that in the New
Testament that, as the Kingdom of God, was sown as “seed” and shall in the future become a mighty tree. The words of
Jesus, too, that fell on hard or on good ground and bear fruit accordingly—all these belong in the
realm of the seed principle that came into being on the third day of creation.

The language of the Bible is neither merely realistic, nor merely symbolic. It is real symbolic, i.e., instead of
employing abstract concepts it makes use of real facts, each corresponding to a principle embodying a basic and
essential truth. Therefore the trees of which “each bore fruit after its kind, whose seed was in itself” were not just
ash trees and oak trees, but also the tree of knowledge and the tree of life. And the fruit of the tree of knowledge
bears the seed of death; the tree of life, on the other hand, bears the seed of an uninterrupted metamorphosis of
growth.

\begin{quotationx}
The biblical words: seed, trees, the sea, and the dry land, as also heaven and earth, mean much more than the concrete
things designated by these words. The “seas” that in Genesis' account of the third day of creation
are described as “the gathering together of the waters under the heavens unto one place” signify the state of
concentration (“gathering”) of force substance, whereby it retains its mobility in the sense of being able to be moved.
On the other hand, the “dry land” means a state of still greater concentration (“gathering unto one place”), where the
force substance is so condensed that a coagulation into solidity is reached. \begin{flushright} \emph{Covenant of the Heart}\end{flushright}

\end{quotationx}
\textbf{The Fifth Miracle}

\textbf{The miracle of the walking on the water—as an event and as a “sign”—is
the revelation of the independence of the Son of Man with regard to the sea and the land}. Jesus Christ walking on the
sea needs no support, for he supports himself.

The Word revealed by the walking on the water is the creative Word of the third day of creation. It is the Word that
lies at the basis of the seed principle. Just as the seed determines the future ways and stages of growth, so the Good
Shepherd decides the ways of development of the true being of humanity and leads it along these ways.

\begin{quotationx}
The third day of creation is the divine cosmic background of the sacrament of confirmation. \begin{flushright} \emph{Covenant of the
Heart}\end{flushright}

\end{quotationx}
Peter answered him: Lord, if it is you. bid me come to you on the water. He said: Come! So Peter got out of the boat and
walked on the water and came to Jesus. But when he saw the wind, he was afraid, and beginning to sink he cried our:
Lord, save me! Jesus immediately reached out his hand and caught him, saying to him: O man of little faith, why did you
doubt? \begin{flushright} Matthew 14:28-31\end{flushright}

Fear is due to the menace of being engulfed by elemental forces of gravitation of a lower order, i.e. of being carried
away by the play of blind forces from the agitated “sea” of the “electrical field” of death. “I am; be not afraid” is
therefore the message of the centre, or Master, of celestial gravitation.

There is another field of gravitation than that of death, and he who unites himself with it can walk on water, i.e.,
transcend the agitated element of “this world”, the electrical gravitational field of the serpent.

\begin{quotationx}
He got out of the boat, which means — in view of all the laws of reason and memory
— that he was taken out of the domain of ordinary consciousness, i.e., that of reason, memory, and
sense perception, and he walked on the water, drawn by Jesus. He experienced, therefore, the same elevation of the soul
which draws the body up after it as spoken of by St. Teresa. \begin{flushright} \emph{Meditations on the Tarot. Letter XII: The
Hanged Man}\end{flushright}

\end{quotationx}
\subsection*{Day 6: Healing of the Man born Blind}
The basis of man's destiny is the decision to renounce the sight that is blind to the revelations
of the Divine, and to wait for the miracle of the reinstatement of the true, original faculty of beholding that sees
all things visible as “works of God revealed.”

The second day of creation is the creation of the firmament, dividing the waters that were under the firmament from the
waters that were above the firmament.

And Jesus passing by, saw a man, who was blind from his birth. And his disciples asked him: Rabbi, who hath sinned, this
man, or his parents, that he should be born blind?

Jesus answered: Neither hath this man sinned, nor his parents; but that the works of God should be made manifest in him.
I must work the works of him that sent me, whilst it is day: the night cometh, when no man can work. As long as I am in
the world, I am the light of the world.

When he had said these things, he spat on the ground, and made clay of the spittle, and spread the clay on his eyes, and
said to him: Go, wash in the pool of Siloam, which means “Sent”. He went therefore, and washed, and he came seeing.
\begin{flushright} John 9:1-7\end{flushright}

\begin{itemize}
\item \textbf{Guilt}: shame 
\item \textbf{Sickness}: separation from the light 
\item \textbf{Forgiveness}: the true, original vision is restored 
\item \textbf{Name of the Master}: I am the light of the world 
\end{itemize}
\textbf{Second Day of Creation}

The second day of creation is the creation of the firmament, dividing the waters that were under the firmament from the
waters that were above the firmament. “And God called the firmament heaven.” \textbf{It is the “day” of the coming into
being of “vision,” of knowledge, of true gnosis}. For the “firmament” or “heaven” that divides the waters above from
those below is the “light” of the first day of Genesis — which now not only divides the waters
above from those below, but is also the connecting link between them.

The waters below mirror the heavens, and likewise the waters above mirror them. This double reflection is the principle
of vision, of knowledge itself, for knowledge is the transformation of the seen into insight, of what is perceived into
truth. It is the realization of the connection between things and ideas, between the real and the ideal.

Now, the reflection of the heavens in the waters below is the realm of things (“the real”), while ideas (“the ideal”)
are represented by the reflection of the heavens in the waters above. And knowledge, or understanding, is the process
of relating the real to the ideal corresponding to it. The active, creatively functioning light (of “heaven”) is
reflected both above as ideas and also below as realities.

\begin{quotationx}
The \textbf{light} that forms the firmament of heaven and that is reflected both in the waters above the firmament and
also in the waters below the firmament is the Logos, the Word, by which all things were made. \begin{flushright} \emph{Covenant of
the Heart}\end{flushright}

\end{quotationx}
\textbf{Sixth Miracle}

The Word bore witness to himself through the lips of Jesus Christ at the sixth miracle of St.
John's Gospel — the healing of the man born blind — “As
long as I am in the world, I am the \textbf{light} of the world”

The Logos is not only the intelligence of the world, that is, the connection of the ideal with the real, but also the
perception of the ideal and the real. All seeing, hearing, touching, etc., presupposes an intermediate connecting link,
an organ, between the percept (the object of perception) and the perceiver (the subject of perception).

The miracle of the healing of the man born blind is a pure act of God. It does not presuppose a willingness to repent,
nor any act of faith on the part of the man born blind. For neither had the man born blind sinned in his life before
birth in the spiritual world or in a previous life on earth, nor had the sin of his parents caused his blindness. The
cause lay in the future. He was born blind “that the works of God might be made manifest in him.”

Through the Fall vision became horizontal, i.e., the “lower eye” became the sole organ of seeing, then “Adam and Eve saw
that they were naked.” This means that vision became devoid of ideas, a perception of the “bare facts,” i.e., of facts
alone, without the corresponding ideas in the waters above the firmament. It became basically cynical.

It was this vision, limited to the bare facts, which the man born blind had renounced at the time of his pre-birth
choice of destiny, in order that the true, original vision—the vision of the real combined with
the ideal, as prefigured on the second day of creation—might be reinstated in him.

\begin{quotationx}
The \textbf{sixth miracle} is the archetype of the sacrament of ordination, the sacrament of the regeneration of
original and true vision. \begin{flushright} \emph{Covenant of the Heart}\end{flushright}

\end{quotationx}
\subsection*{Day 7: The Raising of Lazarus}
\textbf{Reading}: John 11:1-44

The first day of creation is the genesis of the world in its seed condition, i.e., with all its latent possibilities.

The Lazarus miracle is that of the calling forth of light out of darkness.

\begin{itemize}
\item \textsc{Guilt}: losing consciousness 
\item \textsc{Sickness}: excess of spirituality 
\item \textsc{Forgiveness}: all the previous miracles 
\item \textsc{Name of the Master}: I am the resurrection and the life 
\end{itemize}
\textbf{First Day of Creation}

The “days” that follow the first are those of unfolding and actualizing in detail what had been created as a foundation
on the first day. The account of the following days represents a kind of explanatory commentary on the first day of
creation. The miracle of the first day embraces and contains in itself the whole miracle of creation in essence.

The account of the first day of creation is the beginning of the beginnings, i.e., the theme and principle of creation.
It is not a question of what is first in time, but of the deepest and highest foundations of existence
— of the primal foundation (in the sense of the archetypal ground) underlying the world.

The text of Genesis concerning the first day of creation must be read in the deeper language of spiritual morality. We
must comprehend it not as an allegory, but as a concrete moral spiritual happening.

God allowed two tendencies, two impulses, to proceed in opposite directions: in the direction “above” and in the
direction “below.”  There is the striving to be a “likeness” of God (Heaven) and the striving to give oneself up to
God, to mirror Him, to be an “image” of God (Earth).

The world was sunk in sleep; it breathed, but was unconscious. The darkness over the deep is absence of reflection,
i.e., unconsciousness. The creative primal words: “Let there be light,” are the awakening words for the sleeping world.
It signifies the same as “wake up,” the act of waking, or awaking consciousness. The sleeping world wrapped in darkness
was awakened to consciousness by the words “let there be light”; the “waters” began to reflect and the “earth” was no
longer formless and void. The spirits of the hierarchies were awakened out of the sleep of their “rest in God,” and so
there was \textbf{light}, the \textbf{consciousness} of the world.

\textbf{Seventh Miracle}

The Lazarus miracle signifies the awakening of consciousness for all that is relative — relative
for the soul immersed in “rest in the Absolute,” in the sleep of death. This miracle at the same time encompasses the
six preceding miracles, for Lazarus was not only a soul and spirit who was called back; he was also a corpse
— that is, blind, deaf, and without movement. Thus each miracle was repeated in his case:

\begin{itemize}
\item the miracle of the healing of the man born blind 
\item the miracle of the reinstatement of an independent bearing (the walking on the water) 
\item the miracle of nourishment, i.e., the regeneration of the used up and devastated organic substances of the body
(the miracle of the feeding of the five thousand) 
\item the miracle of the reinstatement of movement (the healing of the paralysed man) 
\item the miracle of the healing of sickness (which in this case had led to death) through the regeneration of “vertical
heredity” issuing from the new Adam (the healing of the nobleman's son) 
\item the miracle of the changing of water into wine, by which the liquids in the body were transformed into circulating
warm blood. 
\end{itemize}
\begin{quotationx}
Just as the first day of creation in essence contains and encompasses the entire account of the creation, so does the
seventh miracle of St. John's Gospel contain and encompass the other six miracles. It is the
archetype of the sacrament of extreme unction, the sacrament for the dying, which is directed toward future
resurrection — thereby in essence containing and encompassing the other six sacraments. \begin{flushright}
\emph{Covenant of the Heart}\end{flushright}

\end{quotationx}

\flright{\small\itshape Posted on 2023-04-10 by Cologero}

\section{Afterword}

Dear reader,

here comes to an end this \textit{Our Father Course}. Shortly after the last post, God called his servant Cologero from this earth, and thus the work remains unfinished.

%\fancyhf{}
%\fancyhead[LE]{\small \textbf{\thepage}$\quad${\scshape Appendix{} \thechapter}: \nouppercase{\itshape\leftmark}}
%\fancyhead[RO]{\small \textsc{Appendix \thechapter}: \nouppercase{\itshape\leftmark} $\quad$\textbf{\upshape\thepage}}
%\appendix

\end{document}
