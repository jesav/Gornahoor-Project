\section{Daily Spiritual Exercises}

As preparation for the Our Father course, please look at these daily exercises. There is an exercise for each day of the
week. Spend 5 minutes in the morning with the exercise, and then throughout the day when it occurs to you.

\textsc{Monday: Right Word. Talking}: Only what has sense and meaning should come from the lips of one striving for higher development. All
talking for the sake of talking — to kill time — is in this sense harmful.

The usual kind of conversation, a disjointed medley of remarks, should be avoided. This does not mean shutting oneself
off from intercourse with one's fellows; it is precisely then that talk should gradually be led to
significance. Adopt a thoughtful attitude to every speech and answer, taking all aspects into account. Never talk
without cause and be gladly silent. One tries not to talk too much or too little. First listen quietly; then reflect on
what has been said.

\textsc{Tuesday: Right Deed. External actions}: These should not be disturbing for our fellowmen. Where an occasion calls for action out of
one's inner being, deliberate carefully how one can best meet the occasion —
for the good of the whole, the lasting happiness of man, the eternal.

Where you do things of your own accord, out of your own initiative: consider most thoroughly beforehand the effect of
your actions.

\textsc{Wednesday: Right Standpoint. The ordering of life}: Live in accordance with Nature and Spirit. Do not be swamped by the external trivialities
of life. Avoid all that brings unrest and haste into life. Hurry over nothing, but also do not be indolent. Look on
life as a means for working towards higher development and to behave accordingly.

\textsc{Thursday: Right Habit. Human Endeavour}: Take care to do nothing that lies beyond your powers. But also leave nothing undone which lies
within them.

Look beyond the everyday, the momentary, and set yourself aims and ideals connected with the highest duties of a human
being. For instance, in the sense of the prescribed exercises, try to develop yourself so that afterwards you may be
able all the more to help and advise your fellowmen, though perhaps not in the immediate future.

This can be summed up as: Let all the foregoing exercises become a habit.

\textsc{Friday. Right Memory}:
Remember what has been learnt from experiences. Endeavour to learn as much as possible from life.

Nothing goes by us without giving us a chance to gain experiences that are useful for life. If you have done something
wrongly or imperfectly, that becomes a motive for doing it rightly or more perfectly, later on.

If you see others doing something, observe them with the like end in view (yet not coldly or heartlessly). And do
nothing without looking back to past experiences which can be of assistance in your decisions and achievements.

You can learn from everyone, even from children if you are attentive.

\textsc{Saturday. Right Opinion}:
Pay attention to your ideas.

Think only significant thoughts. Learn little by little to separate in your thoughts the essential from the
nonessential, the eternal from the transitory, truth from mere opinion.

While listening to the talk of others, try to become quite still inwardly, foregoing all assent, and still more, all
unfavourable judgments (criticism, rejection), even in your thoughts and feelings.

\textsc{Sunday. Right Judgment}:
On even the most insignificant matter, judge only after fully reasoned deliberation. All unthinking behaviour, all
meaningless actions, should be kept far away from the soul. You should always have well-weighed reasons for everything.
And you should definitely abstain from doing anything for which there is no significant reason.

Once you are convinced of the rightness of a decision, hold fast to it, with inner steadfastness.

Right judgments are formed independently of sympathies and antipathies.

\textsc{Every Day. Right Examination}:
Turn your gaze inwards from time to time, even if only for five minutes daily at the same time. In so doing you should
sink down into yourself, carefully take counsel with yourself, test and form your principles of life, run through in
thought your knowledge — or lack of it — weigh up your duties, think over the
contents and true purpose of life, feel genuinely pained by your own errors and imperfections.

In a word: labour to discover the essential, the enduring, and earnestly aim at goals in accord with it: for instance,
virtues to be acquired. Do not fall into the mistake of thinking that you have done something well, but strive ever
further towards the highest standards.

\begin{itemize}
\item Turn your gaze inwards from time to time, even if only for five minutes daily. 
\item Sink down into yourself. 
\item Carefully take counsel with yourself. 
\item Test and form your principles of life. 
\item Run through in thought your knowledge — or lack of it 
\item Weigh up your duties. 
\item Think over the contents and true purpose of life. 
\item Feel genuinely pained by your own errors and imperfections. 
\item Labour to discover the essential, the enduring, and earnestly aim at goals in accord with it. 
\end{itemize}

\flright{\small\textit{Posted on 2022-10-22 by Cologero}}
