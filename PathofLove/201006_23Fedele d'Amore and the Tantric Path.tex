\section{Fedele d'Amore and the Tantric Path}

The \emph{Fedele d'Amore} was an initiatic society of Italian poets, and Dante was the most prominent among them. For these poets, the image of the beloved revealed the Divine Sophia, thereby awakening higher stages of consciousness. For Dante, it was Beatrice who served as his guide. 

A similar tradition existed among the Islamic poets. Ibn Arabi had his Nizam, the beautiful and intelligent daughter of a patron. For Hafiz, it was the vision of the beautiful daughter of a nobleman that led to his vocation as poet and mystic. 

Some suspect a direct connection between the Sufis and Dante. However, Dante learned his craft from the Sicilian poets in the court of the Viking Holy Roman Emperor, Frederick II, in Palermo. Since Sicily had previously been ruled by the Arabs, it is not unreasonable to postulate an indirect connection through those Sicilian poets. However, a typological similarity does not necessarily involve an historical connection. 

In an attempt to clarify and justify such a spiritual way, I will list the major ways. 

\paragraph{The Way of the Fakir}
The way of the Fakir involves the use of physical austerities to develop concentration and will. This is continued in certain practices of Christian monks who use painful tourniquets or among Orthodox Jews who wear abrasive underwear. Although certain powers and a strong will can follow from such practices, as the Buddha discovered, they are insufficient to lead to full awakening. 

\paragraph{The Way of the Monk}
The way of the Monk uses devotional practices and rituals to transcend ordinary life. In this way, the emotions can be purified and lower desires transcended. The study of theology can lead to sound doctrine and a steady mind. However, this type of knowing is still at the level of the rational mind or faith, and does not rise to the level of gnosis or Wisdom. 

\paragraph{The Way of the Yogi}
The way of the Yogi is what is most commonly recognized as the ultimate spiritual path. The Yogi transcends all the personal or individual states of being, and may become a \emph{jivan-mukti} — that is, someone enlightened while still in the body. More commonly, such states are not permanent, and usually involve a series of epiphanies that leave an indelible mark on consciousness. The examples of Plotinus and Solovyov come to mind. 

This way is what Guenon describes, and is characteristic of the Brahmin caste. 

\paragraph{The Way of the Knight}
This is the way of someone active in the world, and is characteristic of the Kshatriya , the way favored by Evola. It involves both the development of the intellect and the will. Since all material effects have their origin in spiritual causes, this path requires not only a deep understanding of the world of spirit, but also the power and courage to bring such ideas into manifestation. Furthermore, it is only through the power of the human will that the will of the angels (or gods) can manifest on the plane of human consciousness. 

Although this path involves the whole being, and may be sufficient in itself, since it deals with the action of the world-soul or Sophia, there is often something lacking. That is why the knights of yore took up the practice of courtly love. 

\paragraph{The Way of Tantrika}
This is the most mysterious way, little known until recent times, and subject to many misconceptions. It is usually confused with pure sensuality, whereas its real aim is the development of power through the harnessing of sexual or attractive energies, particularly the power to bring cosmic ideation into material manifestation. 

Often women are attracted to the sensual nature, particularly if they have been introduced to certain tantric practices prematurely. (It is not so important for men.) 

The other manifestation of this way is Sex Magick. However, in practice this devolves to little more than a sexualized way of the Fakir. For example, Aleister Crowley's descriptions of sex magick\footnote{\url{https://www.gornahoor.net/?cat=51}} are hardly appealing and turn out to be little more than tedious sexual gymnastics. Certain powers appropriate to the Fakir may be developed in this way, but it is ultimately self-limiting. 

The true way of Tantrika involves a Knight and a Lady, who participate on the path together. Indications of this are give by Miguel Serrano\footnote{Section \ref{sec:TheTest} in this book.} and Boris Mouravieff in \textbf{Gnosis}. Further elaboration of this topic must await future essays.



\flrightit{Posted on 2010-06-23 by Cologero }

\begin{center}* * *\end{center}

\begin{footnotesize}\begin{sffamily}



\texttt{Will on 2010-06-24 at 12:09 said: }

I agree that Dante used his love for Beatrice as a path to the Divine, but I'm not convinced that the Fedele d'Amore were an initiatic organization. It seems difficult to make that case in regards to Guido Cavalcanti, for example. Do you know of a good source of information about this?

Your point about possible Sufi influence on Dante is well-made. A Spanish author in the 1920s made the case for Dante having effectively `stolen' the work of Ibn Arabi, but I tend to take a more Traditionalist approach that if their insights coincide, it points to a common realization rather than simple borrowing.

It seems to me that the way Dante used his love for Beatrice as a method or technique – a love that was never physically consummated – has a parallel in Plato's descriptions of the erotic ascent, which also involve using the energies of sexual desire and appreciation of beauty to open a path to the Divine. The Phaedrus and Symposium both contain striking descriptions of this.

And by the way, the translation of The Individual and the Becoming of the World is great. Thanks for making it available.


\hfill

\texttt{Will on 2010-06-24 at 18:35 said: }

Do we know that the Fedele d'Amore were an initiatic organization? While it seems clear that Dante used his love for Beatrice as a path to Divine Love, I'm not sure if we can say that Guido Cavalcanti and the other poets were engaged in similar practices. Can you refer me to any good sources on this?

I am struck by the similarity of Dante's method to the one described by Plato in the Phaedrus and Symposium. He also advocates using the energies of sexual desire and appreciation of beauty to open a path to the Divine. I wonder if perhaps the Sufi practices you mention were not influenced by the Platonic tradition. After all, the Muslim world was the keeper of ancient Greek philosophy at this time.

Regardless, I am in agreement that just because Dante and Ibn Arabi and Plato share similar insights, it does not mean there was any direct influence. Rather, it speaks to a common realization, or a kind of `objective metaphysics.'


\hfill

\texttt{Cologero on 2010-06-27 at 11:20 said: }

Are you familiar with “Insights into Christian Esoterism”, where Rene Guenon includes several chapters about the Fedeli d'Amore under the heading of “Some Christian Initiatic Organizations”? There is simply too much information to be summarized in a comment.

In “The Esoterism of Dante”, Guenon writes:

The history of the Hermtic tradition is intimately linked to that of the Orders of Chivalry, and was preserved at the time in question by initiatic organizations such as the \emph{Fede Santa} and the \textbf{Fedeli d'Amore} ,..

You can also look at the reference to Dante in Evola's “The Hermetic Tradition”.

Your insights are good, and you should see the connections between “love”, “eros”, the “woman”, “Wisdom”. Solovyov has similar insights and tried to bring them into the outer church. Even Auguste Comte, in his strange way, stumbled on a rudimentary understanding of this symbolism.


\hfill

\texttt{Will on 2010-06-27 at 20:46 said: }

Thank you. I have read Guenon's book on Dante, but not the other that you mention. I am also largely ignorant of Comte and Solovyov, so I will add them to my list of thinkers to look into.

When I was doing research on Dante a couple years ago, the most helpful sources that I found were The Metaphysics of Dante's Comedy by Christian Moevs, Titus Burckhardt's essay “Because Dante Is Right,” and Henry Corbin's Creative Imagination in the Sufism of Ibn Arabi, in which he has some discussion of Beatrice as a “theophanic” figure.

This is the article that led me to Corbin's work:

\url{http://henrycorbinproject.blogspot.com/2009/02/corbin-dante-i-fedeli-damore.html}

The author of this piece makes similar claims about the Fedele d'Amore, and says that Guido Cavalcanti was their leader. However, when I read Cavalcanti's poetry (albeit in English translation) it seemed to lack the transcendent dimension that so dominates Dante's work.

This, of course, does not mean that he wasn't an initiate or that the Fedeli d'Amore was not an initiatic organization. Perhaps Cavalcanti failed where Dante had succeeded in growing his earthly love up to the heavens. Or perhaps he did not fail, and his realization simply took a different expression which is not as externally religious as Dante's.

I don't pretend to fully understand this method, but from what I can see, I think you are right to compare it to Tantra. Some Buddhist lineages teach a method wherein one views the whole phenomenal world as one's lover. Still other tantric methods involve viewing one's lover as an enlightened being, and thereby practicing `pure view.' The idea, as I understand it, is that when one falls in love, it is due to the enlightened nature shining through, and so one can use that as part of the path, trying to extend the appreciation and pure vision one naturally feels towards the beloved towards all of existence.


\hfill

\texttt{Cologero on 2010-06-27 at 21:17 said: }

There is an excellent book that contains commentary and all of Solovyov's works on Sophia, published as “Divine Sophia” by Judith Kornblatt.

I am not recommending Comte without reservation, as he was somewhat of a strange character. I have a quirky interest in him because he was also a mathematician and for his counter-revolutionary views. He was not a Traditionalist. Nevertheless, I was quite pleased to read a favorable essay about Comte by Solovyov; so there are at least two of us who noticed the same thing about Comte.

For an overview of Comte see here: Auguste Comte\footnote{\url{http://www.gornahoor.net/library/AugusteComte.pdf}}.


\hfill

\texttt{Will on 2010-06-27 at 22:35 said: }

Vladimir Solovyov sounds like a fascinating person. I will check out the book you recommended.

Here is a nice quote from Henry Corbin on this matter:

“In … 1201 when he reached Mecca, the first goal of his pilgrimage, Ibn `Arabi was thirty-six years of age…. He received the hospitality of a noble Iranian family from Ispahan, the head of the house being a shaikh occupying a high post in Mecca. This shaikh had a daughter who combined extraordinary physical beauty with great spiritual wisdom. She was for Ibn `Arabi what Beatrice was to be for Dante; she was and remained for him the earthly manifestation, the theophanic figure, of Sophia aeterna…. For theophanism there is no dilemma, because it is equally far removed from allegorism and literalism; it presupposes the existence of the concrete person, but invests that person with a function which transfigures him, because he is perceived in the light of another world.”

And from Seyyed Hossein Nasr:

“The beauty of woman is, for spiritual man, an unveiling of the beauty of the paradise that he carries at the center of his being … She is the theophany of esotericism and, in certain modes of spirituality, Divine Wisdom … reveals itself to the gnostic as a beautiful woman.”


\hfill


\end{sffamily}\end{footnotesize}
