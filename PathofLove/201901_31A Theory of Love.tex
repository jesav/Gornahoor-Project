\section{A Theory of Love}

\begin{quotex}
The Temptation of Love is to overwhelm the lovers, to hypnotise them as it were, and reduce them willing or unwilling, to a state of passivity which is actually, in the end, the state they desire. \flright{\textsc{Charles Williams}}

\end{quotex}

In our current age of extremes, in which MGTOW and dogmatic feminism each seems reasonable positions to their proponents, we can forget how previous generations understood the ideal of Romance between men and women.

\paragraph{Charles Williams and Romantic Theology}
The Inkling \textbf{Charles Williams} wrote a thesis that attempted to provide an outline for Romantic Theology, which he regarded as a new branch of the subject. Although he could not find a publisher in his lifetime, it was published about a dozen years ago. The theological part of the book is rather conventional, so is less interesting to me. However, Williams' real skill is in literary criticism. He mentions \textbf{Plato}'s \emph{Symposium} and \emph{Phaedrus} as well as the \emph{Zohar} as valuable to his topic. However, since they are not explicitly Christian, he does not make use of them. Instead, he refers to three “masters” in particular:

\begin{itemize}
\item \textbf{Dante} 
\item \emph{La Morte d'Arthur} by \textbf{Malory} 
\item \textbf{John Donne} 
\end{itemize}
Williams describes Donne as a writer of “love-poetry expressing itself in a religious vocabulary.” Besides Donne, Williams looks at several other poets; however, it would be too distracting to cover them in this piece.

\begin{wrapfigure}{rt}{.3\textwidth}
	\centering
	\includegraphics[scale=.3]{a20190131ATheoryofLove-img001.jpg} 
\end{wrapfigure}

Since Williams wrote an influential study of Dante, \emph{The Figure of Beatrice}, it is no surprise to see him as the first Master. Dante, he says, “is the spring of all modern love literature.” Unfortunately, Williams doesn't follow the trail back to the Sicilian poets who influenced Dante, and beyond them, even, to the Troubadours. On the other hand, Dante did it better than his predecessors. He had the ability for deep introspection as well as the ability to express it in words. This is how Dante described his falling in love with Beatrice:

\begin{quotex}
I say that when she appeared from any place, there was no enemy remaining to me, but a flame of caritas possessed me, which made me pardon anyone who had offended me; and if anyone had then asked me concerning anything, my answer would have been only Love, with a face clothed in humility. \flright{\emph{La Vita Nuova}}

\end{quotex}
The story of King Arthur's knights, with the quest for the grail, needs to be unravelled. Williams uses three of the knights to illustrate the different degrees of love.

\begin{itemize}
\item \textbf{Bors}: love in marriage. 
\item \textbf{Percival}: love between two persons who are in contemplation of, but without desire for, each other, their desire being only towards God 
\item \textbf{Galahad}: love whose contemplation and desire is alike towards nothing but God 
\end{itemize}
\paragraph{Three Degrees of Marriage}
\begin{quotex}
There is no accepted agreement upon what the state which our grandfathers used to call `falling in love' involves. It is neither sex appetite pure and simple; not, on the other hand, is it necessarily related to marriage. It is something like a state of adoration. \flright{\textsc{Charles Williams}}

\end{quotex}
I suspect that his grandfathers believed that love led to marriage, but to understand Williams' larger point, we can look at the evolution of the idea of marriage.

\begin{itemize}
\item \textbf{Age of the Father}: Divorce and even polygamy were allowed due to the weakness of the people. 
\item \textbf{Age of the Son}: Marriage is monogamous without the possibility of divorce. Only the death of one of the spouses can end the marriage. The two become one flesh. 
\item \textbf{Age of the Spirit}: Alchemical marriage or the union of souls. The two become one spirit. 
\end{itemize}
\textbf{Baldassare Castiglione} seems to see things more deeply then Williams, when he writes:

\begin{quotex}
Love is defined as desire awakened by beauty, and by progressive illumination passes from sensible beauty to spiritual, and from spiritual beauty to divine: from lust to love, and from love to religion. The duty of the lover is service and honour; the reward of the right lover is intellectual communion with his lady. 

\end{quotex}
The character Jane in one of \textbf{Dumas}' novels says:

\begin{quotex}
Let us forget earth, let us realize heaven; let us share our thoughts, our joys, our griefs, our aspirations, our tears, so that in this unfleshly communion of minds and souls there may be in our eyes pride, in our heart-throbs purity, in our speech chastity, in our consciences calm. 

\end{quotex}
Has your girlfriend ever spoken to you like that? Perhaps the pressures arising from marriage obscure that state of adoration. Finally, here is a thought from Castiglione, that is the very opposite of the Playboy/Charlie Sheen philosophy:

\begin{quotex}
Who does not know that women cleanse our hearts of all evil and low thoughts, of cares, of troubles, and of those heavy dejections that follow in the trains of these? And if we consider well, we shall recognize also, that in respect to the knowledge of high things, so far from turning away men's mind, women rather awaken them. 

\end{quotex}
\paragraph{Francis and Clare}
The relationship between Francis and Clare is an unusual case. It is threatening to clerics and friars who fail to understand it. For example, some see a scandal in the fact the Francis and Clare would sometimes travel together. Hence, they try to downplay the erotic element, even if some moderns can see nothing but the erotic element.

We are embodied beings, so a complete love includes body, soul, and spirit; there is attraction on all three levels of being. In ordinary cases, however, the erotic element takes the form of sexual imagery. Eros then turns inward and demands its satisfaction, thereby preventing its ascent to the spiritual. When the mind is purged of all such sensory images, then eros is turned upwards. Beatrice led Dante to the knowledge of the higher things, yet Dante still experienced his love for her on multiple levels. Williams describes Dante's reaction in terms of the three centres recognised in medieval physiology: the heart, the brain, and the liver.

\begin{itemize}
\item The \textbf{heart}, where the spirit of life dwelled, exclaimed to him: “Behold a god stronger than I, who is to come and rule over me.” 
\item The \textbf{brain} declared: “Now your beatitude has appeared to you”. 
\item The \textbf{liver}, where natural emotions such as sex inhabited, said: “O misery! How I shall be disturbed henceforward!”
\end{itemize}
Yet, the lower emotions can be overcome. Williams describes a different mode.

\begin{quotex}
Virginal love is that which, arising normally between a man and a woman, finds its method in the rejection rather than the acceptance of the ordinary physical approach. It is not particular ascetic: that is, it does not deliberately set aside the graces of the body: it may rather be defined as that kind of love which is so occupied with contemplation that it has no room for desire. 

\end{quotex}
This ideal was \emph{de rigueur} in certain Russian spiritual circles early last century, including Jacques Maritain and his Russian bride Raissa. Perhaps, as saints, Francis and Clare are its perfect embodiment.

\paragraph{Polar Beings}
\begin{quotex}
Nevertheless, neither is the woman without the man, nor man without the woman in the Lord. \flright{\textsc{1 Corinthians 11:11}}

\end{quotex}
Drawing on the Orthodox tradition, \textbf{Boris Mouravieff} describes what the calls “polar beings”. There are three paths to follow, with mastery of the sexual centre a necessary part of the training:

\begin{itemize}
\item \textbf{The monk}: avoids the influences of the world 
\item \textbf{The man in the world}: although the past is difficult, life itself offers more opportunities for spiritual growth. By control of the sexual centre, love can be experienced in the emotional and intellectual parts of the soul. This leads to the realisation of the real I. 
\item \textbf{Polar beings}: a man and a woman working together can produce even more rapid results. However, this requires that they be perfectly compatible for that purpose. They will realise the Real I together, as on. Everyone will meet his or her polar opposite in this life, but will most likely not recognize the other as such. 
\end{itemize}
I will quote Mouravieff extensively without comment, since there is little to add.

\begin{quotationx}
The romance, by which Christian society expressed the principle of reciprocal choice, reached its climax in the Middle Ages. … this romance of tomorrow is called on to cement the indissoluble union between two strictly polar beings, a union which will assure their integration in the bosom of the Absolute.

The vision of such a romance has haunted the highest minds for thousands of years. We find it in platonic love, the basis of the singular romance in the myths of the Androgyne man; of Orpheus and Euridice; of Pygmalion and Galatea… This is the aspiration of the human heart, which cries in secrecy because of its great loneliness. This romance forms the essential aim of esoteric work. Here is that love which will unite man to that being who is unique for him, the Sister-wife, the glory of man, as he will be the glory of God. Having entered into the light of Tabor, no longer two, but one drinking at the fount of true Love, the transfigurer: the conqueror of Death.

This work, done by man and woman working together, can develop with extraordinary power and give rapid results… on condition that from the esoteric point of view the two beings entirely condition that they are a perfect couple, that is, that their combination reservations concerning the peculiarities of their human type — reflects the relation between the absolute `I' and the `You' before the Creation of the Universe.

In centres of culture in the cycle of the Holy Spirit, the love-romance—a feature of the previous Cycle—will give place to the unique love-romance of polar beings, those who will be called upon to shape the society of tomorrow.

In this true Romance, the attitude of the Lady contributes much if not all to the victory of the Knight. Her refined and artistic intuition will understand the meaning of love: that is, to love with all the fibres of one's being, up to integral identification, in a glorious dash towards the same goal.

This return to the perfect unity of polar beings is not given freely. It is the exclusive privilege of those who have crossed, or are ready to cross, the second Threshold of the Way. It is through realization of the totally indivisible unity of their real `I'. by two polar Individualities arrived at the second Birth, that the original sin can and must be redeemed. This is the solution for private and for public life. This is also the peace of the Lord.

A man must begin by a conscious search for his polar being. Polar beings will necessarily meet in life, in certain cases more than once. Only the confused ties contracted in this life by each of them, as a result of their free movements, combined with the karmic consequences of one or more previous experiences, can divert the man or woman from the only being with whom they could form a Microcosmos.

Deeply buried in lies, they do not generally know how to appreciate the gift they are given. Often, they do not even recognize each other. If this is the case, then an agonizing question is put: is there one or more means to detect our polar being, and if so, what are these means? To meet that person, to do so without recognition, to let our polar being pass by, is the worst mistake we could possibly make: because we would remain in our factitious life, without light. Distortions of this kind make it more difficult to recognize the polar being,

It is necessary to permit every being coming into this world to carry within himself the image of the polar being; this image is expressed, in each case, by means of the organ of the opposite sex which exists in every being in a state of non-development.

For man to recognize his polar being, he must be fully attentive on all planes accessible to his consciousness. In fact, as a result of the distortion of the film, the meeting always occurs in circumstances and in a manner least expected, generally at a moment and in a form which resembles nothing he could have ever imagined. The rule enforced is precise: to recognise his polar being, man must know himself. This is obviously logical: to recognize his alter ego, man must first recognize his own ego.

From the first meeting, in the presence of the polar being, both the `I' of the Personality and the `I' of the body vibrate in a manner which resembles nothing felt before. The reason for this is that these `I's find themselves then in the presence of their first love which continues through the centuries. Without clearly being conscious of it, the polar beings know each other; and this knowledge, as ancient as they are themselves, is expressed by the voice of their subconsciousness. This creates an atmosphere of absolute confidence and sincerity from the moment they meet. There is a touchstone here: polar beings do not lie to each other. Soon afterwards, vague reminiscences of past experiences will start to come to the surface in their waking consciousness.

Only an infinitely small minority of human beings feel the anguish caused by their inward isolation and ardently aspire to find the Lady of their dreams. Before one can aspire (to something) one must at least think (about it). This thought must literally devour the Knight's heart, forcing him to accomplish the most perilous feats with the aim of finding the object of his aspirations.

The great mystery lies in the fact that the real I of polar beings is one and indivisible. One for the two of them.

Now we will represent the case of polar beings who are conscious of their polarity and who aspire to royal union: that of the Knight and the Lady of his Thoughts. As a prelude to their complete union this aspiration, after it penetrates into their waking consciousness, will gradually impregnate the I's of their Personalities, thus creating an amorous or courtly love which is quite different from that experienced by the normal run of human beings. This sets their hearts aflame and inspires them with the courage to look for the means which will enable them to overcome all the karmic obstacles that appear on their road.

by the autonomy of his life, every Personality produces a particular Karma. Among other consequences, one result is that two polar beings can be born, not at the same time, as should normally happen, but with a difference in time which in certain cases can be considerable. All this `muddle' explains why polar beings so rarely recognize each other spontaneously at the moment of their encounter.

If a man, burning with a valiant heart, becomes a Knight in order to follow the Fifth Way, he must list for that alone. He must cultivate a double desire:

1. To merit the joy of recognizing in himself the image of his polar being 

2. To merit the joy of recognizing her when they meet 

The general rule, which should be rigorously applied is: to attain the desired goal it is necessary to think of it without ceasing. This is the active concentration that is needed.

Courtly Love is the raison d'etre for the couple of polar beings: for the Knight and the Lady of his Dreams; without it, their polarity remains spiritually sterile and they fall back into the common condition. Its practice, however, demands sacrifices and `exploits'. These are tests. For those who surmount them, the salutary effect of Gnosis is doubled; when it is enriched by experience, theoretical knowledge becomes living knowledge.

In the Middle Ages, the Knight and his Lady, who considered themselves spiritually ONE — in our terminology, polar beings— did not venture into marriage. On the contrary, they parted, accepting the risk of never meeting again and knowing that if they did not triumph over a hard test, their love would degenerate, losing its meaning and its marvellous power. They knew that, by separating from each other for an exploit, they stood the chance, while a premature marriage would be reduced to nothing.

There is no exception to this rule: it applies to all, beginning with the couple of young and just polar beings; and it is even more obligatory between two polar beings who meet at a mature age, when life has already burdened each of them with a karmic load. In such cases, the first sacrifice demanded is the renunciation of a physical relationship, and the first exploit consists in the methodical liquidation of the respective karmic burdens, keeping in mind that the big or small `Gordian knots' which make up these burdens must be untied, not cut.

If the presumed polar beings ardently and effectively undertake some esoteric work in the same direction, useful to the Cause, the moment will come when they will be purified. Having become courtly, their Love will assume all its objective force, and in the purity rediscovered in this way, they will finally be convinced of the reality of a polarity that they had felt intuitively. There is no possibility of error at this stage. At this moment, the Second Birth will unite them forever in the midst of the vivifying Love; and death, finally conquered by this, will lose all semblance of catastrophe for them. 

\end{quotationx}


\flrightit{Posted on 2019-01-31 by Cologero }

\begin{center}* * *\end{center}

\begin{footnotesize}\begin{sffamily}



\texttt{Han Fei on 2019-02-05 at 00:46 said: }

Might I ask the obvious question. Why should lovers avoid physical union? What is wrong with it?

Of course deep down inside some of us may suspect the general direction in which the answer may lie. Among the Romans, there was a seemingly profane custom of keeping a wife, a mistress and a slave. Each woman satisfied one of man's particular needs, for procreation and co-management of the house, another for loving companionship and the last for satisfaction of carnal lusts. When one is in the company of his or her beloved, doesn't the impulse to procreate often fall into the background, at least when our minds are lucid? 

In his lesser known works, J.R.R. Tolkien seems to have touched upon the theme of the polar couple extensively, from his description of Celeborn and Galadriel to the tragic and perilous romance between Beren and Luthien. And of course one can draw a parallel from his own marriage as an impeccable example of such a union.


\hfill

\texttt{33 on 2019-02-06 at 07:52 said: }

There isn't anything `wrong' with it, Han Fei, but what Cologero discusses here is not ordinary marriage. (Of course from the Christian perspective, sexual relationships outside real marriage commitment should be abstained from; why this is so can be explained thoroughly in esoteric.) But in this case we're dealing with a special path. The voluntary renunciation of sexual consummation with a `polar being' here is not due to the `sinfulness' of a legitimate human sexual relationship, but rather one of methodological initiatory expediency: the voluntary renunciation of consummation on that natural bodily plane may serve to intensify realization of the corresponding union on a higher, inner plane. These powerful alchemical energies are directed inwardly instead of being released outwardly. That being said, there are certain tantric traditions, even in Buddhist Vajrayana, that incorporate the energies of union on the physical plane as well in a related form of alchemical romance, attempting to synthesise the higher and the lower union. But for my own part, I tend to believe that the latter path is actually more difficult, and that deliberate renunciation of outer consummation of such polar attraction is more powerful in effectuating the inner realization. This will not be the same in all cases, of course, but the right-handed approach has something very important to teach the contemporary world, with its obsession that `love' must necessarily lead to sexual union, which is its desired end.

By the way, I don't know if Cologero has read what Philip Sherrard has to say on the problem of the sexual relationship in Christianity:

\url{http://www.studiesincomparativereligion.com/public/articles/The\_Sexual\_Relationship\_in\_Christian\_Thought-by\_Philip\_Sherrard.aspx}

Apparently there's also a book titled `Eros and Christianity ' which I have not read.


\hfill

\texttt{Simon on 2020-01-07 at 07:48 said: }

Great article, it inspires me to think about my encounters with women much more deeply. There are two points, both regarding karmic influences, which I could use some further elucidation on, due to my lack of doctrinal knowledge.

1. „Only the confused ties contracted in this life by each of them, as a result of their free movements, combined with the karmic consequences of one or more previous experiences…“

Regarding the „experiences“ mentioned here in the last sentence, do they refer to romantic experiences or to experiences on other planes of being, potentially „before“ this earthly life in which they then play out?

2. „…every Personality produces a particular Karma. Among other consequences, one result is that two polar beings can be born, not at the same time, as should normally happen, but with a difference in time which in certain cases can be considerable.“

If anybody could shed further light on this for me, I would be most thankful. As far as I know (and have read on this site as well), the popular modern conception of reincarnation (i.e. multiple earthly and human incarnations, each carrying karmic load from the preceding ones) is most certainly not in line with Tradition properly so called. If this is so, how am I to properly understand Mouravieff here, that is where do these karmic influences that determine the differential in age come from? From freely chosen actions or thoughts in pre-incarnate existence on another plane?

Thanks in advance and please excuse my naïveté regarding this issue.


\end{sffamily}\end{footnotesize}
