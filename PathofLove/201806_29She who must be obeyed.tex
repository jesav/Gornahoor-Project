\section{She who must be obeyed}

\begin{quotex}
From my earliest youth to the present time I have been aflame with the most exalted and noble love. My love was extremely difficult to bear: certainly not because of the cruelty of the lady I loved but rather because of the overwhelming passion kindled in my mind by my unrestrained desire which often caused me to suffer more pain than was necessary. \flright{\textsc{Giovanni Boccaccio}, \emph{The Decameron}}

Only loving like a pure madman can you continue along the road. But how many times do you believe you love someone and in reality you love no one, not even yourself? \flright{\emph{Nos: Book of the Resurrection}}

\end{quotex}
\paragraph{The Anima}
At the end of the \emph{Letter I: The Magician}, \textbf{Valentin Tom\-berg} reveals that the Magician represents the man who has attained harmony and equilibrium between the spontaneity of the unconscious and the deliberate action of the Ego (conscious sense of I). This state of consciousness is the Self or Real~I.

This process of individuation proceeds in stages described by Jung. The ultimate synthesis – speaking from the male point of view – is the integration of the Anima archetype, as described in Two Essays on Analytical Psychology. The archetypes are inborn in the psychic makeup, so the Anima is the image of the woman in the man. This is represented in three ways to be described later. But prior to conscious awareness, the Anima is a dark force, erupting into awareness as unexplained moods and other ways.

In meditation, one seeks to bring the I consciousness into the lower parts of being. In particular, there needs to be more awareness of the workings of the psyche or subconscious. There, one will find other beings, i.e., different “I's” or different personalities, each claiming to be the true~I.

Jung's psychological approach is useful for this process, provided the limitations of psychology are understood. This post is concerned with just one part of the process, achieving harmony with the Anima. The first things to note are any unexplained mood changes. Beyond that, there needs to be a deliberate encounter with the Anima. For example, initiating a conversation with Her and writing it down. In our workgroups, this has been effective; sometimes the effort results in a poem. Of course, careful attention to dreams is necessary; big dreams – i.e., those that have more than a personal content – should be logged.

Note that we have not recommended more books to read. We learn most effectively by trial and error. Ultimately, the Hermetic path is quite personal and one law for all would be oppression. One's book comes at the end of the exploration, not before. Tomberg describes the process:

\begin{quotex}
The pure act in itself cannot be grasped; it is only its reflection which tenders it perceptible, comparable and understandable or, in other words, it is by virtue of the reflection that we become conscious of it. The reflection of the pure act produces an inner representation, which becomes retained by the memory; memory becomes the source of communication by means of the spoken word; and the communicated word becomes fixed by means of writing, by producing the book. 

\end{quotex}
Keep that in mind. The Anima is a mystery, and our verbal descriptions are of its inner representation. She cannot be exhausted by our verbosity.

\paragraph{She}
Jung refers to the character in \textbf{Rider Haggard}'s novel \emph{She} — the one who is called “She-who-must-be-obeyed” — as descriptive of the Anima. She is a mana personality, that is, “a being full of some occult and bewitching quality, endowed with magical knowledge and power.” This, Jung claims, is a projection of an unconscious self-knowledge, described this way:

\begin{quotex}
I recognize that there is some psychic factor active in me which eludes my conscious will in the most incredible manner. It can put extraordinary ideas into my head, induce in me unwanted and unwelcome moods and emotions, lead me to astonishing actions for which I can accept no responsibility, upset my relations with other people in a very irritating way, etc. I feel powerless against this fact and, what is worse, I am in love with it, so that all I can do is marvel. 

\end{quotex}
Obviously, a man who has ever fallen so totally in love will recognize parts of himself in that description. But if he then manages to integrate the Anima, he will gain her mana. This will raise him archetypally to the “mighty man”, e.g., a hero, chief, magician, saint, ruler of men and spirits, a friend of God. It is one of those types that particularly interests us, viz:

\begin{quotex}
Actually, it is the figure of the \emph{magician} who attracts the mana to himself, i.e., the autonomous valency of the Anima. 

\end{quotex}
\paragraph{Dante and other Initiatory Texts}
This teaching is certainly not novel. Rene Guenon mentions some European initiatory texts. Foremost among them are \textbf{Dante}'s \emph{Divine Comedy} and the medieval poem, \emph{The Romance of the Rose}. Both of them are focused, in differing ways, on the integration of the Anima.

For Dante, Beatrice was a projection of his Anima, an unreal woman who led him to higher states. Now Dante lived a man's life: he raised a family, he was a warrior in the battles of the time, he was a poet, philosopher, mystic, and initiate in the \emph{Fedeli d'Amore}. Boccaccio claimed Dante's countenance displayed a Saturnine melancholy, not surprising in such a deep thinker.

Hence, his writings were not at all dry abstractions, but were taken from his own life. In his oeuvre as a whole, he describes people, personal events, dreams, impressions, etc. The trend of some Traditional writers to leave out the personal element as something inferior is not necessarily superior.

The Romance of the Rose is an allegorical poem about uniting with the Rose, despite a series of obstacles. Various personages representing, for example, Reason, Genius, Friendship, the Old Woman, offer different opinions on the task. To regard it simply as an entertaining tale of romance would be a mistake.

Guenon also surprisingly mentions \textbf{Boccaccio}'s Decameron and \textbf{Rabelais}'s Gargantua and Pantagruel as other initiatory texts. They may be too ribald for some, but that may be a technique to hide the real meaning from the merely curious.

The main point, then, of all these works is to show how the integration with the Anima is essential on the Hermetic path.

\paragraph{The Blue Rose}
Folk tales often contain hidden wisdom unlike much of modern story telling that is cynical with ideologically inspired lessons. For example, \emph{the Legend of the Blue Rose}\footnote{\url{http://www.marilynkinsella.org/Fabulous\%20Folktales/The\%20Blue_rose.htm}} is one such tale from China. Note how the lesson of the story is opposed to the commonly accepted serpentine wisdom of our time. For example, the gardener's son, in some MGTOW circles, would be mocked as being in the “friend zone”, while the princess is really attracted to some “bad boy” who mistreats her. However, in the folk tale, the princess is psychologically healthier than Western woman today. She recognizes her real Animus and they live a happy life together.

\paragraph{Germanic Women}
\textbf{Tacitus} in \emph{Germania}, sections 18 and 19\footnote{\url{https://www.gutenberg.org/files/7524/7524-h/7524-h.htm}}, describes the Germanic women of his time. Jung uses them as exemplars of this quality:

\begin{quotex}
Woman, with her very dissimilar psychology, is and always has been a source of information about things for which a man has no eyes. She can be his inspiration; her intuitive capacity, often superior to man's, can give him timely warning, and her feeling, always directed towards the personal, can show him ways which his own less personally accented feeling would never have discovered. 

\end{quotex}
Tacitus notes that the matrimonial bond is quite strict among the Germans, with monogamy being the norm. The woman is not the weak link in that relationship; rather she shares fully in their common life:

\begin{quotex}
That the woman may not think herself excused from exertions of fortitude, or exempt from the casualties of war, she is admonished by the very ceremonial of her marriage, that she comes to her husband as a partner in toils and dangers; to suffer and to dare equally with him, in peace and in war: Thus she is to live; thus to die. 

\end{quotex}
The society is based on chastity, with no “seductive spectacles”, adultery is extremely rare and immediately punished. Young couples are expected to be virgins, so marriage is lifelong. They don't try to limit children through birth control, nor through infanticide as in other cultures.

That opposition to divorce, promiscuity, adultery, and birth control was very influential on the Roman church. Even the Greeks allow divorce and have been inconsistent on contraception. Of course, the neo-pagans believe in none of it.

\paragraph{A Modern Life}
The modern world tries to foist on us certain falsehoods presumably based on “science”. One of them is that a man is genetically programmed for promiscuity. The traditional life of the Germanics reveals the lie. A real man protects his wife and children, desiring to keep them safe and secure, psychologically and spiritually, not just physically. The effeminate man seeks pleasure over protection; he is not our model.

\paragraph{Three Appearances of the Anima}
\textbf{Friedrich Schlegel} in his roman a clef \emph{Lucinde}\footnote{See Section \ref{sec:Lucinde} in this book.} describes three appearances of the Anima: as Wife, as Mistress, and as âme-soeur. He recognized those three qualities in his wife Dorothy. That is, their relationship clicked on all three levels: hylic, psychical, and spiritual.

\begin{quotex}
The imago of woman (the soul image) becomes a receptacle for these appearances which is why a man, in his love choice, is strongly tempted to win the woman who best corresponds to his own unconscious femininity—a woman, in short who can unhesitatingly receive the projects of his soul. …it may turn out that the man has married his own worst weakness. 

\end{quotex}
\paragraph{Wife and Mother}
A man's first female relationship is normally with his Mother. This imprints itself in his psyche:

\begin{quotex}
His nearest relations, who exercise immediate influences over him, create in him an image which is only partly a replica of themselves, while its other part is compounded of elements derived from himself. The imago is built up of parental influences plus the specific reactions of the child; it is therefore an image that reflects the object with very considerable qualifications. 

\end{quotex}
As he matures, that shifts, as Jung describes:

\begin{quotex}
In place of the parents, woman now takes up her position as the most immediate environmental influence in the life of the adult man. She becomes his companion. 

\end{quotex}
\paragraph{Sexual Attraction}
The strongest impulse of the Anima is sexual attraction, which erupts into consciousness in various forms. This is exacerbated by the distortions of the psyche resulting from the fall.

The first is our imaginative faculty in the astral body (animal soul) which is given over to sexual fantasies of all types. These occlude the image of the Feminine as Sophia, while debasing her. She just has an instrumental value. Although a healthy relationship will involve visual, auditory, and tactile elements, these are used not for their attractive qualities to another human being, but rather as stimuli for personal pleasuring. So in Dante's case, his fantasy of Beatrice leads him to higher realms, whereas the fantasy of a so-called poet today (as in pop music) would be more suitable to for the Penthouse Letters.

Another distortion occurs even lower, in the etheric body (vegetative soul). A healthy eros will be oriented towards the Beloved. Most often, however, the eros is indiscriminate, and is attracted to women other than the Beloved. Moreover, the eros may even be directed to other objects, as in fetishes, for example. Since the eros is ideally intended for one specific person, artificial stimulation of the eros can be an obstacle to spiritual progress.

\paragraph{Polar Beings and Courtly Love}
Since the psychological is the reflection of something higher, the Anima needs to be understood in her spiritual reality. The story of Adam and Eve reveals this, since Eve is split off from Adam. She is then his spiritual double, the outer representation of his Anima. Although Jung's archetype is to be understood in a universal sense, it is also intensely personal.

With the Fall, this connection has been lost. This is meant in Tomberg's sense of the archetype that manifests itself endlessly both in history and in each individual. The task of the integration of the Anima becomes intertwined with finding one's twin soul or âme-soeur. \textbf{Boris Mouravieff} refers to that couple as Polar Beings. Drawing on Eastern Christian traditions from Mount Athos, he asserts that at the general Resurrection, the human elite will be formed of polar couples. The meeting of the Polar Couple is a new initiation.

There is much esoteric effort required for that to happen. The first stage is the purification of one's own being. If you find that you are consistently entering into toxic relationships, that is likely due to defects in your own psyche, particularly with the Anima. We are not speaking here of so-called good relationships which are mostly a mutual agreement to “settle”, which eliminates the element of risk, yet also prevents the full attainment of something higher.

This was understood much better in the Middle Ages with the ideal of Courtly Love involving the Knight (or Warrior) and the Lady of his Dreams. We have the Troubadours and the \emph{Fedeli d'Amore} as examples. This was Love at the spiritual and objective level, beyond both the carnal and the psychological level.

The Knight and the Lady considered themselves to be spiritually united. Yet they renounced any idea of marriage as such. Typically, the social roles of wife, husband, mother, father were with someone else, not the Polar Being. Moreover, a physical relationship must be avoided as a sacrifice. This is emphatically not some “high-minded” Platonic love devoid of passion. Rather, the physical attraction is quite intense, so its renunciation can be quite painful.

The Knight and the Lady will meet each other once in their lives, but will seldom recognize each other, because they are not ready or have not advanced enough. Sometimes they are given a second chance and that must be obeyed.



\flrightit{Posted on 2018-06-29 by Cologero }

\begin{center}* * *\end{center}

\begin{footnotesize}\begin{sffamily}



\texttt{Therion Exú Rei on 2018-07-23 at 15:40 said: }

the anima is an archetype, archetypes are modes of perception and modes of function. Jung talks about 4 phases of the anima as an mode of perception, how men ig going to perceive the female sex.

they are representaded by eva, helena of troy , maria and sophia.

In the first phase men is under the mother figure, the men in this phase is either a homossexual or a weak heterossexual. That kind of men that is submissive to the female figure. In the second phase the men can see woman as a truly sexual object, but he is under a more of an animal sexuality. In the 3 phase, very rare for modern men, it begins a higher perception, woman is see as a spiritual being, capable of rising men to a spiritual ground. In the fourth phase, is the mysterium conjunctuns!


\hfill

\texttt{Mikkel on 2018-07-28 at 10:52 said: }

If we're talking about the Anima as Jung described I think the article puts out well that “Jung's psychological approach is useful for this process, provided the limitations of psychology are understood.” 

In his talking of phases as you've (Jung?) described, it seems limited and counter to the concept of unity of a principle vs. multiplicity which quickly spirals into talks of “progression”. Perhaps these are not stages that follow from one to another or build on each other, but frames or perceptions of types of men who made them. As psychology is really based on some type of cure or understanding to help change an individual, perhaps we can see that stages or perceptions are different ways of being rather than focused on an individuals perspective? As representational, what do all the figures mentioned represent or say about the spirit that illustrated the feminine with their description as true, perhaps these descriptions of the phases here shows more about the type of spirit and how closely related they are to principle rather than a quality of the feminine being illustrated as truth, albeit the stages described have truth to them, but essentially why do certain people see them that way?


\hfill

\texttt{Michael on 2020-11-04 at 11:12 said: }

Circling back to this…”Typically, the social roles of wife, husband, mother, father were with someone else, not the Polar Being. Moreover, a physical relationship must be avoided as a sacrifice. ”

How does this work with “Schlegel and his wife, Dorothy, turned that common wisdom on its head by incorporating all three aspects into a single relationship.”

Seems like an example of turning common wisdom on its head, but perhaps closer to a synthesis we hear the words of M. (paraphrased) : “All must be sacrificed, but we have not said all must be broken.”


\end{sffamily}\end{footnotesize}
