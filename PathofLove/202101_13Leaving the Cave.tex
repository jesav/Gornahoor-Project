\section{Leaving the Cave}


In his science fiction short story “The Crime and the Glory of Commander Suzdal”, former Pentagon official and expert on psychological warfare Cordwainer Smith describes a planet which is toxic to femininity. Colonists from earth survive by making their women male. As a result of this modification the colonists start developing a massive hate for normal women and families back on earth and aim for their destruction.

\paragraph{The Need for Intimacy}
“Feminism” has never been about anything feminine. Neither is the invention of the term “toxic masculinity” aimed at men. Both expressions are meant to conceal femininity. They create a narrative which makes it impossible to achieve any kind of consistency between what you really see and feel and what you are supposed to see and feel according to that narrative. As a consequence, men went into hiding and women became male, which in turn rendered femininity an unknown mystery to both men and women.

\begin{wrapfigure}{rt}{.3\textwidth}
\centering
 \includegraphics[scale=.75]{a20210113LeavingtheCave-img001.jpg}
\end{wrapfigure}

However, most women on planet Earth are still women and many of them would like to develop more intimacy in their relationships with men. We cannot do this however if deep within their souls men remain detached not only from their wives or girlfriends but even from themselves. Femininity is incapable of making a decision. It cannot become an end in itself — it is searching for a purpose. Intimacy then arises from the feeling of being needed. A man's indifference towards himself makes it impossible for a woman to feel needed as a lover, let alone as a resemblance of his Anima. She will never be able to make up for a man's missing self-respect, because she is thriving on that man's Self. She will cherish, inspire and even submit to it, but she cannot create it.

\paragraph{Integrity vs. Potential}
\begin{quotex}
It is foolish to judge woman with the values of the absolute man even in cases where, by doing violence to her own self, she makes a show of following those values and even sincerely believes that she is following them. \flright{\textsc{Julius Evola}, \emph{The Metaphysics of Sex}}

\end{quotex}
The question is, why should a quality be viewed as a flaw? This development has led to so many misunderstandings, anger, and sadness for men and women alike that it is almost impossible to take a step back and have a look at where we came from. When Weininger bemoans a woman's “lack of being” or Evola explains that for a woman a lie never “breaks her own existential law”, it might seem well and rationally explained from a male perspective, but still contains a judgement or rather a measurement. A woman never draws a border like that. In fact, she cannot.

Potential is to woman what integrity is to man. Potential is boundless and needs to be actualised. There have been many such concepts mentioned on this site, e.g., Purusha and Prakriti, Hyle and Morphe etc. In order to understand a woman's sudden change of opinion, mood, preferences and her simultaneous vacuous conformity to so-called public opinion it is helpful to leave judgement aside for a moment and ask why should somebody want to be like that? Without principles? Without imagination? Is it a conscious choice to be like that? Is it a choice at all? Why should something exist that morphs into something else the moment it feels needed? Without bothering that it is contradicting itself or behaving seemingly disloyal to earlier promises?

There are many disappointed mothers out there who take great pleasure in sharing little anecdotes about how the character of their daughters has undergone a sudden change since their wedding day. Not only did these daughters pick up the vocabulary of their spouses, but all of a sudden also preferred different food, changed their denomination, or even their religion altogether — the list of “changes“ those mothers are keen to spot in their estranged married daughters is endless. Likewise, you will hear complaints from ex-boyfriends and ex-husbands who are perplexed when they realise that their former lovers or wives are finally dressing the way they had always begged them to dress or engage in exactly the kind of sexual practices they had always dreamed of — but they do it now with their new man and would have never done so for their ex-boyfriends or ex-husbands.

So again, why would anyone want to be like that? What is the reason for such a behaviour or is it really simply weakness? Are men not better off without women? No risks, no strings attached?

\paragraph{Authority and Submission}
Of course a woman understands the difference between truth and a lie. She even expects to be told the truth and detects a lie faster than any man would be able to, simply because he does not expect to be lied to. So why does she not feel affected by her own lies whereas a man gets the feeling of utter destruction when faced with a lie? If the mode of existence for women is potentiality this means there is no real lie, no right nor wrong unless a certain potential has been realised.

This realisation is something women cannot do by themselves. I agree, this sounds strange after more than a century during which we have been told that women can do just whatever they want. Maybe we can — if we know what we want. But what we want changes ceaselessly unless there is a real call for submission. However, the following act of submission needs to be received by someone. So why do men tend to run away from that gift? Live up to it! That's exactly what a woman's potential is all about: It is a gift to a man's quest for integrity.

As Nietzsche describes the quest for Sophia in \emph{Thus spoke Zarathustra}:

\begin{quotex}
Brave, untroubled, mocking, violent — that is how wisdom wants us to be: she is a woman and never loves anyone but a warrior. 

\end{quotex}
Male integrity needs to be proven in the same way female potentiality needs to remain opaque, uncategorised and uncharacterised. If a woman is constantly forced into the act of deciding and determining, sexual attraction becomes impossible. A man that does not even bother to take care of himself can't expect a woman to decide what his life is all about.

Women are not capable of creating a man. He has to make the effort to know himself before he can reveal that Self to a woman in order to \emph{know her}. What women can do is to create themselves according to the information they receive from a man. It is an expression of gratitude and not a sign of weakness due to a “lack of being” on the part of the woman as Weininger put it. It is the essence of femininity. What follows actually comes close to what has been termed “Authority” on this blog. It is a natural acknowledgement of the risks a man went through to encounter his own Self.

If this quality of femininity is deemed unnecessary, weak or even a flaw, women become superfluous. Where there is no need for femininity, women deteriorate into a state of always remaining the lesser part, insufficient strange creatures who “lack being” and will be measured according to the values of men, which is probably the reason why, perceiving themselves as lesser men, women have a hard time living up to their own expectations.

Nowadays harm is caused not so much anymore by propaganda which is so obviously nonsensical, but by refusing to see what is right there in front of you. Men and women haven't changed. Words and terms can't create reality. However, perception and acknowledgement of what actually exists influence the way we deal with that reality. That's a challenge men face as well as women. Why not try and work this out together? Leaving the cave\footnote{\url{https://www.gornahoor.net/?p=13522}} and opening up that space for women again would be a first step.

In the science fiction story Commander Suzdal managed to save femininity on earth from destruction by the male-females from the space colony. Subsequently he was stripped of his rank and his name. He was denied life as well as death for not asking for permission to act the way he did. Smith writes: “But the crime was that he had succeeded.” — The same goes for his glory.



\flrightit{Posted on 2021-01-03 by Sibylle }

\begin{center}* * *\end{center}

\begin{footnotesize}\begin{sffamily}



\texttt{Solphomeron on 2021-01-08 at 22:42 said: }

“Body of a woman, white hills, white thighs,

you look like a world, lying in surrender.

My rough peasant's body digs in you

and makes the son leap from the depth of the earth.

I was lone like a tunnel. The birds fled from me,

and night swamped me with its crushing invasion.

To survive myself I forged you like a weapon,

like an arrow in my bow, a stone in my sling. …”


\hfill

\texttt{Sibylle on 2021-01-09 at 17:15 said: }

“I am asham'd that women are so simple

To offer war where they should kneel for peace;

Or seek for rule, supremacy, and sway,

When they are bound to serve, love, and obey.

Why are our bodies soft and weak and smooth,

Unapt to toil and trouble in the world,

But that our soft conditions and our hearts

Should well agree with our external parts?”


\end{sffamily}\end{footnotesize}
