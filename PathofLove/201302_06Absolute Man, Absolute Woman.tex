\section{Absolute Man, Absolute Woman}

In the chapter “Man and Woman” in Revolt against the Modern World, \textbf{Julius Evola} establishes the proper relationship between the two principles. The feminine force is centrifugal with its tendency to chaos, but when aligned the masculine stability, a synthesis results. Thus, the masculine principle must become more fully itself, while the feminine becomes aligned with the masculine. Evola mentions the symbolism of the “bride” in this regard, just as \textbf{Ananda Coomaraswamy} points out that the king is the bride in relation to the purohita.

Evola points out that birth is not by chance. Hence, the being who is a man or a woman must reflect a spiritual difference. This differs from the modern view which sees one's birth sex as something arbitrary, and now, given new medical technology, as even a matter of choice. However, traditionally, a man must fulfill himself as a man and a woman as a woman.

For a man, according to Evola, the highest ways belong to the Ascetic and the Warrior, corresponding to the Brahmin and Kshatriya castes respectively. Both these paths “affirm themselves in a life that is beyond life”, the former through jnana yoga, or total detachment, and the latter through bhakti yoga, or pure action. The woman, on the other hand, fulfills herself as Lover or as Mother. Note, here, that in Evola's scheme the man can fulfill himself totally on his own, but a woman only in relation to a man. Through devotion to her lover or her son, a woman gives herself totally to another being, thereby fulfilling herself. Evola concludes this discussion:

\begin{quotex}
To realize oneself in an ever more decisive way along these two distinct and unmistakable directions, limiting in the woman everything that is man and in man everything that is woman, approaching the absolute man and the absolute woman – such is the traditional law for the sexes, according to the various planes of life. 

\end{quotex}
Perhaps Evola did not exhaust all the roles since men can also be Fathers and some women are ascetics. Logically, it is necessary for men to be Fathers otherwise women could not be fulfilled as mothers. Even among the Kshatriyas, there is the archetype of the Judge who decides with perfect justice. The vaishyas are neglected, but there are many who through dedication to knowledge or service to the community can find fulfillment.

To return to the spectrum of the absolute man and woman, we are left wondering. For example, where do the Ascetic and Warrior sit on this spectrum? Guenon answers this question when he points out that the Brahmin is oriented to superhuman states while the Kshatriya (warrior) tends to the realization of all the possibilities of the human state. This makes it clear that the Ascetic is more transcendent than the Warrior.

\paragraph{The Absolute Man}
What then is the archetype of the absolute man? Evola alludes to it at the beginning of the chapter when he mentions the purusha and prakriti, the yang and the yin. But \textbf{Alexander Jacob} in \emph{A Reconstruction of the Solar Cosmology of the Indo-Europeans} confirms it. He writes:

\begin{quotex}
The formation of the deity as Purusha/Vishnu, the Ideal Man, is the result of the promptings of the divine heart or spirit. This ideal Man is however actually androgynous. 

\end{quotex}
Thus Purusha is the Ideal or Absolute Man, which is clear once the masculine principle is fully understood. Moreover, a man approaches to this Ideal not by eliminating the feminine principle but rather by uniting with it in the \emph{Mysterium Coniunctionis} or Spiritual Marriage. To transcend all lower states is the non-dual state, as the Purusha is also Atman.

A clue to this state is given by the warrior who actualizes all the possibilities of the human state. Or stated differently, he manifests in matter all his latent possibilities; this requires the cooperation of the feminine principle and especially the domination of it. A fortiori, the jnani actualizes all the possibilities of all the states so there is no longer duality between essence and existence. This defines precisely, in the West, God, for whom essence and existence are one.

\paragraph{The Absolute Woman}
If Purusha is the absolute man, then Prakriti is the absolute woman. Prakriti alone is chaotic, undifferentiated, formless, and blind. As such, Prakriti can never be manifested as it is a metaphysical impossibility. Prakriti must be informed by the masculine principle in order to be anything in particular. But Evola had just pointed out that the woman is fulfilled by her relationship to a man, so the Absolute Woman is not Prakriti separate from all influences of the male principle, but rather the woman who is in the perfect relationship to the higher male principle. In other words, she participates in the same Spiritual Marriage mentioned above.

\paragraph{Excursus on German Idealism}
To make these principles real, they demonstrate how German philosophy of the 18th and 19th centuries fed the process of the feminization of the West. When Kant attempted to show that the pure reason was incapable of any knowledge, the result was the abandonment of the male principle. The pure reason, or intellect, is the tool of contemplation. Only the practical reason, or action, could lead to knowledge of God, freedom, and immortality. This is equivalent to raising the feminine principle to the peak.

Those who followed Kant realized the consequences of his philosophy and concluded that the Will was the fundamental principle of the world. Schopenhauer did this quite thoroughly and consistently. In his system, the Will is fundamental, but if the will is not guided by the pure reason, it must therefore be arbitrary, blind, illogical, purposeless, and directionless. Nevertheless, there had to be something more, the Platonic ideas, to give form to the world. However, apparently the will was not directed by the ideas. Because of this Schopenhauer's philosophy provides many interesting insights. Had he raised the ideas above the will, he would have come closer to a real metaphysics. At least he understood the Will to be transcendent to the world.

Nietzsche borrowed Schopenhauer's concept of Will and gave it a direction, the Will to Power. He also dropped the ideas and regarded the world of appearances, all superficial, i.e., not as appearances of anything transcendental. The Will to Power, as a particular manifestation of Shakti, is feminine, and the denial of transcendence eliminates the masculine element totally from his system.



\flrightit{Posted on 2013-02-06 by Cologero }

\begin{center}* * *\end{center}

\begin{footnotesize}\begin{sffamily}



\texttt{Jason-Adam on 2013-02-07 at 14:55 said: }

Wow. This article destroys the foundations of the New Right and pseudo-traditionalist movement based on German idealism entirely.

Philosophically, we need to get back to knowledge of God through as Aquinas taught us.


\hfill

\texttt{Mihai on 2013-02-08 at 03:24 said: }

This should also make clear to the enthusiasts who put Nietzsche's philosophy on the same level with traditional ideas, that there is nothing traditional about his worldview, quite the contrary. He is much closer to a Marx than to a Guenon and even Evola.


\hfill

\texttt{Jason-Adam on 2013-02-08 at 15:01 said: }

The famous Protocols of the Learned Elders of Zion, which Evola considered worthy of reading, classify Nietzsche along with Marx and Darwin as agents of subversion.

Also, the Abbe Barruel reported that Kant was a member of the Illuminati. When taken together, the facts show that German Idealism was devised as a means of destroying traditional Catholic Europe.


\hfill

\texttt{Cologero on 2013-02-09 at 06:45 said: }

If this has any merit: the real history of white Europeans\footnote{\url{http://vimeo.com/48152436}}, it would explain why Western history has been dominated by the kshatriya mentality and why true intellectuality has been difficult to establish. German idealism is a philosophy compatible with the kshatriya spirit.


\hfill

\texttt{Janus on 2013-02-10 at 23:56 said: }

I don't think the new right claims to be based on German idealism, though Evola certainly was influenced by it in his early years.

I wonder about the possibilities for initiation or illumination for women, in this case. I have always wondered whether the masculine and feminine are really limited in the hardline way Evola prescribes. For example, he goes so far in this chapter of Revolt to suggest that while the woman must love the man, the man does not return this love (I'd cite directly but I don't have my copy on hand…perhaps someone knows the passage?), or else risks losing his pure masculinity. Among Traditionalists, I think Charles Upton at least would find problems with this. Tradition in Europe seemed to encourage love given by both parties (and Evola notes this, but seems to think it a deviation of sorts). 

Moreover, I find it interesting that it seems like this suggests that the masculine and feminine only “really” exist in the man, whereas the woman in her pure femininity must submit herself to this. Thus, it seems like while the man is ascending toward his “true Self”, the woman has no Self outside of that of the man. Perhaps this could be more clearly explained and corrected? It seems to me more likely that the masculine and feminine manifest themselves in different ways in different people, surely. We may look at St. Joan of Arc, who clearly manifested a more masculine principle akin to the kshatriya. than any feminine role. I have also heard a similar explanation of those cases where relations happen between those of the same sex; there usually seems to be a “masculine” vs “feminine” role, so perhaps these principles may be harmonized even when it does not appear to be the case on the profane or physical level (this is, of course, the exception, not the rule). Even the sacrament of marriage can be viewed in this light as the masculine and feminine being completed in complementary beings, the man and wife, thus manifesting the Divine unity.


\hfill

\texttt{Paulo on 2014-11-16 at 11:44 said: }

Janus

“I wonder about the possibilities for initiation or illumination for women, in this case. I have always wondered whether the masculine and feminine are really limited in the hardline way Evola prescribes. For example, he goes so far in this chapter of Revolt to suggest that while the woman must love the man, the man does not return this love (I'd cite directly but I don't have my copy on hand…perhaps someone knows the passage?), or else risks losing his pure masculinity.” 

You dont get the point her,the love of men toward women is not the love of dependence,the love of using women as a lifeline,of the repruduction of the same dependece a baby have toward his mother.He gives consciousness love.

“Moreover, I find it interesting that it seems like this suggests that the masculine and feminine only “really” exist in the man, whereas the woman in her pure femininity must submit herself to this. Thus, it seems like while the man is ascending toward his “true Self”, the woman has no Self outside of that of the man. Perhaps this could be more clearly explained and corrected?”

Being women the passive element,she must receive the masculine princible in her self.When men is fully,that is,had aquire the true masculiny,then he can”fecundate”women!

” It seems to me more likely that the masculine and feminine manifest themselves in different ways in different people, surely. We may look at St. Joan of Arc, who clearly manifested a more masculine principle akin to the kshatriya. than any feminine role”

Even in her case,she had visions of swords coming from the sky,had seen visions,a clearly feminine mode of function.


\hfill

\texttt{Hank on 2017-02-28 at 18:25 said: }

Ibn Ashir said « Purity is thine through water which naught else hath changed »

On this, Shaykh al-Alawi said, « Purity is reached through Absolute Water, the Water of the Unseen, that is, the Limpidity which is variegated in Its manifestation, One with Itself in Its seeming multiplicity, Self-manifested, Hidden through the intensity of its manifestation, Absolute in Its relativity––this is the Water which is free from any taint and which availeth for purification; and of It one fo the Gnostics said: `With the Water of the Unseen make thine ablution

If thou hast the secret and if not, with earth or stone.' 

This is the Water of the Unseen which avileth for purification, and all other water in relation unto it is as dry sand, not to be used except when this Water hath been lost.»

tr. by Martin Lings


\end{sffamily}\end{footnotesize}
