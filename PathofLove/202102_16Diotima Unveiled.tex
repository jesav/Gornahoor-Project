\section{Diotima Unveiled}

As I have been seeking inspiration for a personal project this week, my muse suggested this addendum to the discussion of Diotima. She can not only see past, present, and future, but she can lift souls out of the depths of ignorance and darkness.

The idea of Absolute Beauty is timeless. Hence, one does not ascend the ladder in successive, well demarcated steps. Each step can be associated with a state of being and Guenon explains its meaning.

\begin{quotex}
All states of the being, considered under their primary aspect, abide in perfect simultaneousness in the eternal now. 

\end{quotex}
Hence, one abides on all the steps of the ladder simultaneously. The step you are “on” is just the one to which you direct conscious attention. The most common state is to be in a dream-like state of semi wakefulness, and will therefore, unconsciously, be drawn to the most material step of the ladder, which is physical love.

\paragraph{Sweet Love}
Unlike blasphemy, murder, theft, perjury and so on, which are intrinsically evil, physical love is in itself not evil in itself. However, it may be situationally evil, or perhaps better said, it may be expressed in appropriate or inappropriate ways. That is why such lovers are in the first, or the least painful, level of hell.

Such love is quite delightful which is why it also serves as a trap. Even when it grows sour, as it often does, the being tries to “fix” it or else find it again with someone else. A being on this step is reluctant to move on to higher stages.

\paragraph{Soul Love}
The first stage is restricted to the two lower sheaths of the subtle body: desire and emotional connections. The creative power of sexual energy is restricted to procreation. Obviously, children are necessary for the nation and the family to perpetuate and thrive. There is human love for nation, family, and children, although even animals care for their offspring. To do less, is to live even below the animals. For the ancient Romans, the \emph{proletariat} were those who had nothing to offer society other than their children. So, it is not yet the fulfilment of all human possibilities.

\begin{wrapfigure}{rt}{.35\textwidth}
\includegraphics[scale=.4]{a20210216DiotimaUnveiled-img001.jpg} 
\end{wrapfigure}

This requires the training of the emotions to feel a wider range of emotions and to experience them more deeply. And what is distinctly human is intellectual development: the ability to distinguish right from wrong, the real from the unreal. It is the seat of creativity, innovation, and so on.

The more one actualizes his intellectual soul, the more he will be able to recognize those qualities in another. His intellectual life becomes so fascinating and pleasurable, the memories of sweet love begin to fade. Instead, he longs for a higher connection. His desire is to share ideas on music, literature, poetry, history, metaphysics, and the like. Procreation reappears as intellectual creation at this stage.

Sweet love follows a pattern of exhaustion and replenishment; energy is dissipated in its act. Soul love is not like that. It is energizing, energy doubles, so nothing is lost.

When the opportunity is presented, it is taken.

\paragraph{Possession}
It must again be emphasized that this has nothing to do with evolution, or psychological growth. All the stages exist simultaneously; the limitations of human language make it sound like some sort of progress.

A fortiori, soul love is not a substitute for failures at a lower level. If there is no desire at the physical level, there can be no desire, not in its full expression, at the intellectual level. And many men today lack such desire.

Since the present age is a deviation from previous eras, what follows my not follow current trends exactly.

In the past, love and marriage was an option reserved for the lower classes. High status people often had to marry for other reasons, particularly to strengthen political or commercial alliances. For them, the purpose of sex was to create children.

Man tends to want to enjoy before possessing while woman needs to possess before enjoying. This was necessary when women needed a stable relationship in which to raise children and to be protected from strangers intent on enjoying. So the high status woman was secure in her possession but love had to be found elsewhere.

Hence, the notion of courtly love arose. The wandering knight became suitable for that purpose. He is able to enjoy, but he is not in a position to possess because of his wanderings. The early knights were purely physical brutes tied to a particular area, but the solo knight errant developed a code of chivalry and high culture. Thus, the knight could fulfil the soul needs of the high status woman without thereby threatening her marriage or children.

\paragraph{The Veil}
There is a misconception about the veil, as though its sole purpose is to control woman; this is used by Westerners to belittle cultures that still practice it. In more ancient times, pagan and even Christian cultures used the veil. However, it was restricted to high status women. It is a sign that she is more than just a beauty that can be put on display willy nilly for just anyone to see.

The custom persisted until a few generations ago. On more formal occasions, my mother wore a light veil that covered her entire face. Even today a bride wears such a veil at her wedding.

In our time, we can use imaginary veils. Keep that thought as a reminder to try to pierce the veil to see the soul life of the other. If you need some other motivation, that is a necessary Hermetic exercise.



\flrightit{Posted on 2021-02-16 by Cologero }

\begin{center}* * *\end{center}

\begin{footnotesize}\begin{sffamily}



\texttt{Tannheuser on 2021-02-17 at 17:51 said: }

The difference between “sweet love” and “soul love” can also be characterized a difference between passivity and activity. Sweet love (“I am in love”) begins as a passion, something that happens to you, whereas soul love (“I love”) is really an action, or something you do – it requires focus, effort, and engages the higher faculties of the soul. As you say, however, the active and passive exist simultaneously and are interdependent.


\hfill

\texttt{Tannheuser on 2021-02-18 at 13:18 said: }

As for the lack of desire and eroticism on a physical level, Yukio Mishima described this condition as resulting from a lack of contact with the Absolute, or God. It takes only a little reflection to realize the truth of this – it is the only thing separating human eroticism from that of animals mating on a Discovery Channel documentary.

From an interview shortly before his death:

“The beauty-eroticism-death diagram, to which I referred a little while ago, is a concept that demands that the second element, eroticism, cannot exist except in the realm of the absolute. As for Europe, eroticism is only found in the world of Catholicism. This religion has severe commandments whose violation constitutes sin. And the sinner, whether he likes it or not, must appear before God. Well, eroticism is the method of establishing contact with divinity through sin… In the relativism of today's world, however, eroticism is no more than a kind of free sex. It's not opposed to anything. It is sex without any relation to the absolute. In my opinion, nothing could be further from true eroticism.

…In my opinion, one should only speak of eroticism when the human being risks his life and seeks pleasure until death, which is as if he arrived at the absolute from the reverse. If the gods did not exist, they would have to be reborn. And without God there is no eroticism. And because of this way of thinking of mine, I have done the impossible to make the absolute reborn. That is when eroticism arises. What does all this have to do with everyday sex? Well, nothing. Let's say it is a kind of `paneroticism'.

That's it. This search is the main objective of my literature.”


\hfill

\texttt{Paulo Adolpho Zeymer Netto on 2021-02-19 at 03:09 said: }

“The âme soeur (literally “soul sister”) is not just another life partner, and certainly not a concubine. The idea of “sister” should indicate that it is a chaste, not sexual, relationship. Moreover, it indicates a close bond of affinity, not in the biological or genetic sense, but certainly in terms of spiritual races.”

Cologero,please clarify this to me,i have found her,but i could never fell sexual desire for her,so how exactly can we finally be together???What exactly,can bind us together for real and for ever,in a physical sense?


\hfill

\texttt{Cologero on 2021-02-21 at 09:39 said: }

You are an astute reader, Paulo, as I had forgotten that passage. If you look back, one of our very first posts, that has set the trajectory all these years, was about that complete relationship that you desire.

But pay attention to the idea expressed that all the states exist simultaneously, so it is not either/or. It depends on where your consciousness is focused. At one time, you desire her sexually. That is how you experience her for yourself.

Yet, at another time, you will see your kindred soul in her, and relate that non-sexual way. Then you will see here as she is in-herself and for-herself, as a thinking, feeling, willing being complementary to you.

I wish you all the best.


\end{sffamily}\end{footnotesize}
