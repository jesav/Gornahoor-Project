\section{Journey through the Planets}

\begin{quotex}
All things are in pairs, each the opposite of the other, but nothing the Lord made is incomplete. Everything completes the goodness of something else. Could anyone ever see enough of this splendor? \flright{\textsc{Sirach 42:24-25}}

\end{quotex}
\paragraph{Quantum Entanglements}
Two quantum physical particles are entangled when their states are not independent of each other. That is, if the quantum state of one particle changes, then the corresponding state of the other also changes. This happens instantaneously, even at a distance.

By the Law of Correspondences, two beings can also be entangled. \textbf{Henry Corbin}, in \emph{The Man of Light in Iranian Sufism}, develops this idea from traditional sources. He describes the entanglement this way:

\begin{quotex}
To speak of the polar dimension as the transcendent dimension of the earthly individuality is to point out that it includes a counterpart, a heavenly “partner”, and that its total structure is that of a bi-unity, a unus-ambo. This unus-ambo can be taken as an alternation of the first and second person, as forming a dialogic unity thanks to the identity of their essence and yet without confusion of persons.

\end{quotex}
That implies that every being in the human state has a counterpart. They share the same essence, yet they are different persons. Their “alteration”, although it takes place as conversation, is more like a mirror. The orientations differ, yet they are intertwined: a change in once results in an immediate and corresponding change in the other; hence they share an essential nature. Yet neither one alone is the “real” being. Rather, the two together form a unity, i.e., the true Self.

\paragraph{Adam and Eve}
In \emph{Swedenborg and Esoteric Islam}, Corbin brings this concept closer to home based on revelations to Emmanuel Swedenborg. Analogous to Paul's description of the pneumatic, psychical, and hylic persons, Swedenborg identifies them based on one's own understanding of selfhood or ipseity:

\begin{itemize}
\item The \textbf{Material person} believes that his own self is everything, so his existence would cease without it. This situation is infernal. 
\item The \textbf{Spiritual person} understands, but only in a theoretical way, that the power of thinking and acting arise from his Principle. 
\item The \textbf{Celestial person} has the actual experience of his Self. This Self is like that of an angel. 
\end{itemize}
Adam, in the state of deep sleep, believes that he lives, thinks, speaks, and acts from himself alone. Spinoza, for example, recognized the illusion of that state, although he lacked an adequate solution. Adam rouses from sleep as he sees the illusory aspect of his belief. Fully awakened, he sees that he is no longer alone. Corbin emphasizes the importance of this insight:

\begin{quotex}
it is one of the most profound arcana that there is to meditate upon.

\end{quotex}
Before this transformation, Adam's proprium, or inmost self, appears to the angels as something bony and inanimate. The ribs surround the heart.

Afterwards, Adam's proprium is vivified. Eve is Adam's proprium or Self. Corbin elaborates on this symbolism:

\begin{quotex}
the masculine or man symbolizes the intelligence and everything that relates to it, consequently all tings of wisdom and faith, while the feminine or woman signifies or symbolizes the will, consequently all things relating to love, and consequently relating to life, since life proceeds only from love.

\end{quotex}
Hence, this will is the nucleus or inmost heart of the human being. Thus, the heart is not the nous as some believe, but rather resides in the emotional center. While Eve is the proprium of the internal man, the projection of Eve as other than Adam results in a process of exteriorization. Adam is no longer simply material, but he is not yet celestial. Although he realizes that he is incomplete, he cannot recognize his Self interiorly. Hence, Eve is a projection onto a different person.

Eve, on the other hand, longs for the wholeness that Adam fails to see.

\paragraph{The Double}
The idea of the Double has been part of Tradition, even if the understanding is not always complete. \textbf{Charles Stang}, in \emph{Our Divine Double}, documents the history of this idea. He distinguishes the double into horizontal and vertical versions. The horizontal double appears in pagan stories as well as more contemporary writers. This double is uncanny and even menacing. Rudolf Steiner saw that double in terms of lower beings named Lucifer and Ahriman.

The vertical, or divine, double, on the other hand, was known to Plato. The divine double makes possible the ascent to the Intelligible reality of the divine forms. In the Phaedo, Socrates suggests that the Lover and the Beloved should serve as mirrors to each other. Each sees the best version of himself in that mirror. That does not mean that “Love is Blind,” but rather just the opposite. Love sees what is True and Beautiful in the other, beyond merely material forms.

\paragraph{The Vertical Ascent}
Dante saw the full significance of the Divine Double. On the Earth, our essences are hidden from each other because of the interference of our physical bodies. The outside can conceal the inside. Specifically, our facial expressions and gestures can hide our true thoughts and feelings as much as they might also reveal those thoughts. Hence, Beatrice did not recognize Dante in this life as her double or polar being.

The vertical assent through the planetary spheres represents the attainment of higher states of being. Here, the polar couple help each other; they see in the other what could not be seen. If the gross body veils the truth, then each planet lifts another veil until all is revealed.

Jacob Boehme reveals that Adam was originally separated from Sophia, his feminine Self. In compensation, Eve, the external feminine, was given to Adam. Corbin explains this:

\begin{quotex}
Adam-Eve is the sophianic couple, because Eve is the proprium of Adam, his proprium vivified by the Divine Presence; the Adam-Eve couple has the perception of all the Good of love and the Truth of fa ith and consequently possesses all wisdom and all intelligence in ineffable joy.

\end{quotex}
The polar couple experiences such joy as they journey through the planets. There is no hurry, they have eternity on their side, leading ultimately to the alchemical marriage.



\flrightit{Posted on 2021-05-16 by Cologero }

\begin{center}* * *\end{center}

\begin{footnotesize}\begin{sffamily}



\texttt{Hugo Smith on 2021-05-17 at 22:35 said: }

In the alchemical marriage, shouldn't the two halves be temperamental opposites, since they are incomplete on their own?


\hfill

\texttt{Cologero on 2021-05-17 at 23:31 said: }

temperamental opposites? You are restricting yourself to the psychological plane, not the metaphysical plane. Mirror images are opposite in a way. Next time, try to quote the text where your concern is properly addressed.


\hfill

\texttt{Hugo Smith on 2021-05-19 at 03:12 said: }

Sorry, I was actually searching for the right word there and settled on “temperamental”. The part of the text that gave me pause was “…Socrates suggests that the Lover and the Beloved should serve as mirrors to each other”.

Like you said, this doesn't necessarily imply that the lover and beloved aren't opposites. I was thinking of the Hermetic idea that the lovers must reconcile themselves in each other, like extremes meeting in a sort of spiritual paradox (passive/active, red king/white queen, sun/moon etc).


\hfill

\texttt{Cologero on 2021-05-19 at 07:45 said: }

Do you understand at all how to have a conversation? You don' get a mulligan, i.e., a second chance to make the same point. Consider this from \textbf{Rumi}:

\begin{quotex}\scriptsize\itshape
Listen to the story told by the reed, of being separated.\\
“Since I was cut from the reedbed,\\
I have made this crying sound.\\
Anyone apart from someone he loves\\
understands what I say.” 

\end{quotex}
Do you feel that? Have you experienced that separation from your Beloved? Have you longed to unite? Then you might understand the point.

Of do you wake up every morning, thinking: “I'd really like to reconcile the spiritual paradox today.” We are talking about persons, not abstractions.

\emph{Start taking your life seriously.}


\hfill

\texttt{Hugo Smith on 2021-05-19 at 12:43 said: }

Did I write something that made you mad?

In conversation, people often rephrase things when they think someone doesn't understand.

I'll try to take that last sentence to heart.


\hfill

\texttt{Cologero on 2021-05-23 at 18:35 said: }

No, I don't get “mad”, I am just a good actor. Trust me, I do understand.


\end{sffamily}\end{footnotesize}
