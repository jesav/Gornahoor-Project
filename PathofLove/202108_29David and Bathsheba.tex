\section{David and Bathsheba}

\begin{quotex}
I have found in David, the son of Jesse, a man after my heart, who will do all my will. 

\vspace{-1.5em}

\flright{\textsc{Acts 13:22}}


My flesh and my heart may fail, but God is the strength of my heart and my portion forever. 

\vspace{-1.5em}

\flright{\textsc{Psalm 73:26}}

And David had success in all his undertakings, for the Lord was with him. 

\vspace{-1.5em}

\flright{\textsc{1 Samuel 18:14}}

\end{quotex}
\paragraph{David}
King David is one of the Nine Worthies\footnote{\url{https://www.gornahoor.net/?p=331}} who best represent the Chivalric ideal. As a proto-knight, he engaged in battle, he loved, he had friends and enemies, he sinned, he repented, he cried out to God. He was the King of the Heavenly City on earth and a Prophet of God

That made him a real man in the traditional sense, a man who has actualized all his possibilities. Valentin Tomberg explains:

\begin{quotex}
King David was more human than all men of his time. This is why he was anointed by divine order by the prophet Samuel, and for this reason the Eternal gave him the solemn promise that his throne would be established for ever.

\end{quotex}
In this respect, he is a type of Adam. Adam, as the first man, was also a ruler, having dominion over the earth, plants, and animals. Adam brought wisdom to his kingdom and taught survival skills to the human race.

So David serves as an example, not necessarily for monks or other religious types, but certainly for men active in the world. That is why Ibn Arabi's meditation on the \emph{Wisdom of Existence in the Word of David} emphasizes the Will:

\begin{quotex}
The power of the Will is immense. Nothing occurs in existence nor disappears from it without the Will. 

\end{quotex}
Nothing can thwart God's will. However, man chooses the means, which may either be in opposition or in conformity to the Divine Will.

\begin{quotex}
It is only commanding the means, not the command, which brings things into being. None opposes God at all in what He does in respect to the command of the Will. Opposition only occurs in respect of the command of means. So understand that!

\end{quotex}
If there is anything transgressive in David's behaviour, it was not because he was opposed to God's will; rather, the means may have been inappropriate. However, “All's fair in love and war,” as the saying goes.

\paragraph{The Imaginal Realm}
\begin{quotex}
Changing the appearances of things, walking on water, climbing Mount Qaf, all fall in the category of events that Suhrawardi mentions as taking place in the “intermediate Orient”. In other words, they are psychic events whose scene and action are set in neither the sensible nor the intelligible world, but the intermediate world of the Imaginable, the world of symbol and of typefications, the place of all visionary recitals. it is the world in which spirits are corporealized and bodies spiritualized. \flright{\textsc{Henry Corbin}, \emph{Avicenna and the Visionary Recital}}

\end{quotex}
We are familiar with the metaphysical distinction between the world of the Intellect and the world of the Senses. What is often lost, however, is awareness of the Imaginal realm in between the two. Corbin has done much to recover the ancient interest in the imaginal realm. Each realm is associated with a specific sense:

\begin{itemize}
\item \textbf{The Intellectual Realm} is associated with the sense of hearing, which is the most subtle sense. That is why the highest revelations come from hearing. 
\item \textbf{The Imaginal Realm} is associated with the sense of sight. It is the place of visionary experiences. 
\item \textbf{The Sensual Realm} is associated with the sense of touch, which is the deciding factor in determining the real. Thus, a rainbow is not real, because it can't be touched, even though it can be seen. 
\end{itemize}
Usually the imaginal realm is unconscious, i.e., that part of the world that is not represented in consciousness. As such, there is little or no awareness of its existence. In esoteric training the student is taught how to become aware of the contents of the subconscious. However, if the visions arise spontaneously or unexpectedly, they can be frightening. Corbin develops this idea:

\begin{quotex}
This darkness is ignorance or, more precisely, unconsciousness of ignorance — that is to say the natural man is in a state of ignorance and cannot even be conscious of that state. To free himself from it, he must pass to the darkness; this is a terrifying and painful experience, for it ruins and destroys all the patencies and norms on which the natural man lived and depended — a true “descent into hell”, the hell of the unconscious. That [Suhrawardi] discerned the psychic event, the redoubtable initiation into knowledge in the true sense, into the saving gnosis, is greatly to his credit.

\end{quotex}
When reading a sacred text, we often come upon passages that seem contradictory, impossible, or even scandalous. That is a clue that the text should not be understood at the sensual level. Rather, the images are part of the Imaginal Realm. So the answer is not more historical research or textual analysis. Instead, the reader must look deep within; often the meaning will be found there. Obviously, there is no point in relating random historical events if no deeper meaning can be attached.

The story of David and Bathsheba is like that. On the material level it sounds scandalous. But on the Imaginal level, there may be some deep insights. This story, then, is not just about two historical personages, but about events in your psyche which you can observe.

After all, David and even Bathsheba have been held in high regard, the former as a model of chivalry, as a king, as a prophet, and even a type of Jesus, not just by men but even by God. And Bathsheba has traditionally been considered a type to the Virgin Mary, Sophia, the Queen of Heaven.

So consider this a small meditation on their love story, which can be recreated today in your own consciousness, as a psychic event.

\paragraph{Bathsheba}
David was walking on the roof when he spotted Bathsheba bathing. The roof indicates that David was observing from a higher state of being, the level of the Spirit. After seven wives, and a harem, the sight of yet another naked woman cannot account for David's strong reaction to Bathsheba. Just as Adam recognized in Eve:

\begin{quotex}
She is bone of my bone and flesh of my flesh.

\end{quotex}
David saw in Bathsheba the “soul of his soul”.

Unlike his other wives and concubines, Bathsheba was unique, the woman of destiny with whom he has been entangled for eternity. What he saw was the beauty of his soul. As \textbf{Saint Theresa d'Avila} expressed it:

\begin{quotex}
we shall see that the soul of the just man is a paradise, in which God takes His delight. Nothing can be compared to the great beauty and capabilities of a soul; however keen our intellects may be, they are as unable to comprehend them as to comprehend God.

\end{quotex}
Ordinary men can only see physical beauty. A real man finds true beauty in the soul of a woman. When a man is truly attracted to a woman, he will go to extraordinary lengths to get her attention. That is because the quest for unity is also the quest for God.

She, on the other hand, had to do nothing to attract David's attention. That is why a man's love can be so mystifying to a woman. Sibylle told me\footnote{\url{https://www.gornahoor.net/?p=13757}}:

\begin{quotex}
To a woman the love of a man is incomprehensible. She sees it, she hears it, she might enjoy it, be overwhelmed by it and eventually fall in love, but she does not comprehend it.

\end{quotex}
Does not that incomprehension echo Saint Theresa? If you try to understand love intellectually, you will misunderstand. If you try to judge by human standards, you will misjudge.

David and Bathsheba desired a spiritual marriage. Earthly marriage ends at death, and sometimes mercifully so. Uriah is a symbol of that, because, on earth, we all too soon get wrapped up in relationships that are inappropriate. Hence, Uriah represents Bathsheba's lingering attachment to carnal things, and her bathing is a symbol of her purification.

David must also be purified. To possess Bathsheba, he needs to come to terms with his lust and his jealously of Uriah. Since David's means displeased God, he was convicted of sin by the prophet Nathan. The spiritual child born of that relationship is spiritually dead. David confesses:

\begin{quotex}
I have gone astray like a lost sheep; seek thy servant, for I do not forget thy commandments. \flright{\textsc{Psalm 119:176}}

\end{quotex}
Then David is purified when Nathan forgives his sins. Only then, was a special son, Solomon, born.

\paragraph{Solomon}

\begin{quotex}
She came from the ends of the earth to hear the wisdom of Solomon, and behold, something greater than Solomon is here. \flright{\textsc{Matthew 12:42}}
\end{quotex}

The hieros gamos produces a child of wisdom, the true self. Solomon personifies Wisdom. As such, he is the type of Jesus. Solomon's kingdom prefigures the Kingdom of Christ, the eternal kingdom.

As Solomon's mother, Bathsheba is the type of Mary, the Queen of Heaven. At that time, the mother of the King served as Queen. (Otherwise, which of Solomon's wives would be queen?) So the true offspring of David and Bathsheba is Wisdom, just as Mary's offspring is Jesus. David and Solomon had faults, otherwise revelation would have ended with them.


\hfill



\paragraph{Note on Art} In the Middle Ages, Eve and Bathsheba were the only nudes that could be painted, for obvious reasons. However, I have refrained from including such art in the body of the text to avoid misunderstandings.



\flrightit{Posted on 2021-08-29 by Cologero }

\begin{center}* * *\end{center}

\begin{footnotesize}\begin{sffamily}



\texttt{Tannheuser on 2022-04-05 at 14:29 said: }

It's worth comparing the Bathsheba episode to how David came beforehand to possess his wife Abigail. She too was married to an unworthy husband, Nabal, but David refrained from killing him, and the Lord himself struck Nabal down (1 Samuel 25:38), after which David married Abigail (the mother of the prophet Daniel). In a similar way to how David threw aside his earthly armor and weapons in his fight with Goliath in order to rely purely on the help of the Lord, here we see the same attitude with regard to his courtship of a married woman.


\end{sffamily}\end{footnotesize}
