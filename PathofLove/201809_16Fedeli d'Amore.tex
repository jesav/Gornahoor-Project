\section{Fedeli d'Amore}

\begin{quotex}
To every heart which the sweet pain doth move,\\
And unto which these words may now be brought\\
For true interpretation and kind thought,\\
Be greeting in our Lord's name, which is Love. \flright{\textsc{Dante}, \textit{from his initiation poem to the Fedeli d'Amore}}

The various `ladies' celebrated by the poets attached to the mysterious organization of the Fedeli d'Amore from Dante, Guido Cavalcante, and their contemporaries to Boccaccio and Petrarch, are not women who actually lived on this earth but are all, under different names, one and the same symbolic `Lady'. who represents transcendent Intelligence or Divine Wisdom. \flright{\textsc{Rene Guenon}, \emph{Insights into Christian a Esoterism}}

\end{quotex}
It was necessary to write a poem in order to be initiated into the Fedeli d'Amore. In 1283, Dante sent the group a poem about a dream he had, asking for an interpretation. In his dream Amor (Love) appeared with Beatrice. He received several responses and was allowed into the group.

\paragraph{Alchemy}
The Divine Comedy can be regarded as an alchemical work, since it describes, the transformation of human matter to the gold. There is this correspondence with the alchemical process:

\begin{itemize}
\item \textbf{Nigredo}: inferno, the fire that burns 
\item \textbf{Albedo}: purgatory, the fire that purifies 
\item \textbf{Rubedo}: heaven, the creative fire of love 
\end{itemize}
There has been speculation about Dante's ties to the Knights Templar. We are not really interested in the conspiracy theories or any external evidence. Rather, for us, what matters is the internal evidence. Esoteric works contain “Easter Eggs”, i.e., unexpected clues that are the key to the deeper meaning.

The obvious connection link is \textbf{Saint Bernard of Clairvaux}, who was both the spiritual director of the Templars and Dante's final guide. Dante chose St Bernard as his final guide to indicate his affinity to the Templars. Before exploring the transformation described by the Fedeli d'Amore, there are some items that may seem speculative initially.

\paragraph{Satanic Trinity}
Satanic trinity is the ape of the Divine Trinity. It is depicted as the three headed Satan. These refer to Lucifer, Mephistopheles or Ahriman, and one unnamed for now.

\paragraph{The Female Trinity}
The other trinity consists of Mary, Lucia, and Beatrice.

Mary is the second Eva. This is also indicated in the Ave, which is the reverse spelling of Eva.

St Lucia is an interesting insertion, since “Lucia” is the anagram of acuil (=acquila), the eagle. \textbf{Valentin Tomberg} deals with the symbolism of the eagle in the Empress card. For example,

\begin{quotex}
The eagle shows the aim of magical power; it is its emblem and its motto, which reads: “Liberation in order to ascend”. Together they represent a magnificent flight; they aim, as a whole and each taken individually, at the ideal of the sublimation of human nature.

“The science of LOVE” is the sceptre of the Empress, which represents the means by which the aim of magic is attained. 

\end{quotex}
Beatrice is the celestial reflection of Dante's first love, and stands for Wisdom.

\paragraph{The Troubadours and the Sicilian School}
Although the ideal of love in spiritual literature is ancient (Song of Songs, for example), the more immediate precursors of the Fedeli d'Amore were the Troubadours of Southern France. They sang of Courtly love, which was based on nobility and chivalry.

Its roots go back to the French Troubadours, whose theme was always courtly love between a Lady and a Knight. Courtly love was [and is?] experienced as erotic attraction mixed with spiritual attainment. However, the relationship was not sexually consummated. Rather, the knight sought the admiration of his Lady through his deeds and accomplishments, often undergoing ordeals at her request.

The same theme was adapted by the Sicilian School\footnote{\url{https://en.wikipedia.org/wiki/Sicilian_School}} under the patronage of Holy Roman Emperor \textbf{Frederick II}. They invented the sonnet form, and an initiate had to write a sonnet in order to be accepted into the school. This spread to the Fedeli d'Amore where Dante perfected the sonnet form.

\paragraph{Polar Beings}
There are three levels of attraction between a man and a woman: carnal, psychic, and courtly or spiritual love. Carnal love, obviously, desire the physical act and nothing more. With psychic love, there is an emotional connection. Traditionally, there has always been a period of engagement before the couple has carnal relations, despite their strong physical attraction.

The deeper the connection, the more creativity is expressed in the relationship. \textbf{Boris Mouravieff} describes this phenomenon:

\begin{quotex}
On all planes, the objective sign of Love's participation is the creative spirit which animates the subjects for whom it has become an aim. Conversely, if we think we are in Love but objectively do not notice an increase in creativity on any plane, either in ourselves or in our partner, we can be sure that the relationship is based on anything but Love. 

\end{quotex}
Tomberg likewise asserts that true creativity cannot exist without love. Emotional bonds may fade, but there is a higher love between polar beings. In this case the commitment is much deeper:

\begin{quotex}
The Knight and the Lady of his dreams, whether they are so in truth or honestly claiming to be so, must endeavour to act in all circumstances of their inner and outer life as if they were already united in their consciousness of the real I, which is indivisible although bipolar, and ONE for their two Personalities and their two bodies. 

\end{quotex}
A Polar couple will meet each other once in their lifetimes, but will seldom recognize each other. The Knight must learn to see the image of his Lady in his consciousness, and vice versa. Even when they do meet, there may still be obstacles. That is due to the deformations in their inner being and different karmic burdens.

\paragraph{Transformation}
Carnal love is connected to the etheric body (vegetative soul) and psychic love to the astral body (animal soul). In the fallen state, these two bodies are under the influence of demonic spirits indifferent to, or even opposed to, human attainment. The etheric body will create images in the astral body, or the astral body may be so overcome with emotion that it does not seek anything higher.

This is the clue to the alchemical transformation. The etheric and astral bodies need to be purified. Eva represents the astral body in the Fall, so the opposite of the Fall is Mary. The goal, then, is the transmutation of the chaotic soul into the pure mirror of the divine word.

For the individual, this will be the development of the Real I. For the polar couple, they will develop the I together.



\flrightit{Posted on 2018-09-16 by Cologero }
