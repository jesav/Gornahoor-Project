\section{Astray in the Human Jungle}

\begin{quotex}
Turn every negativity to your profit. \flright{\textsc{St John Climacus}}

Courtly love can be effective only if it is based on a Gnosis which is lived, for only a lived Gnosis, on that has been acquired through experience and has gone down into the heart—one joined with Hope that is founded on Faith—will ensure that the knight will have the discernment which will prevent him from going astray in the jungle of purely human reasoning and feelings. \flright{\textsc{Boris Mouravieff}}

\end{quotex}
As we pointed out last time, a consistent theme has been the development of the sense of Self or the Real I. Along with that, we have used the idea of the possibility of using feminine energy as the means for that development. This started, actually, we our first real post on Lucinde by \textbf{Friedrich Schlegel}\footnote{See Section~\ref{sec:Lucinde} in this book.}. The difference between masculine and feminine spirituality should be clear by now. The former, as described by \textbf{Julius Evola} in particular, involves an act of possession of what appears to be a lack, but is really a privation.

The feminine, on the other hand, is more concerned with filling that lack, always in a passive way. Thus, the movement based around a “Law of Attraction” has gained ground in recent years. Its teachers claim that the “universe” will satisfy the lack, presumably in some automatic and predictable way. A recent encounter may serve to illustrate these things more clearly, while providing concrete examples for Evola's and Michelstaedter's abstractions\footnote{\url{https://www.gornahoor.net/?p=7338}}.

\paragraph{Rhetoric and the Feminine Consciousness}
Not so long ago, I heard from a woman — let's call her Erica — who had discovered the Meditations on the Tarot web site. Since she was very interested in that topic, she contacted me to discuss it. She gobbled up all the articles and was full of praise. I gave her the source and she bought the book immediately. After a week of exchanging emails, we agreed to meet for dinner at a posh Palm Beach restaurant.

It would be easy to believe that she had won some sort of genetic lottery. A former internationally known cover girl, she still retained her beauty, and was cultured, elegant, with a sharp intelligence. A true cosmopolitan, she had lived in several different countries. Yet, there was still something deeply sad about her. There was a malaise, bordering on spiritual acedia. She managed to keep it so deeply hidden, I was unable to intuit exactly what it was, given the short time we spoke in person. Nevertheless, she was quite spiritual and claimed to be fully committed to her Catholic faith.

She was a bit intimidated by the Tarot commentary, so she asked me where she could find the “kindergarten” to begin her studies. I assured her that the education she received at her convent school was more than enough kindergarten. Tomberg gives us a deeper understanding of just what she would have learned in her religion classes.

She seemed quite pleased to hear that.

\paragraph{The Feminine Path}
When she began describing her own comparable spiritual undertakings, we could see our differences arise. It should surprise no readers that she had taken an active interested in areas such as the development of psychic powers, angels, crystals, and so on, that is, the whole gamut of the new age. Apparently, there is a college in London that claims to teach such things. Just as in our interview with Aphrodite\footnote{\url{https://www.gornahoor.net/?p=7197}}, she, too, was interested in exotic stones from around the world and their alleged effects on the various chakras.

I mentioned that she was following a very feminine path, whereas Medtarot and Gornahoor are quite masculine. That she agreed with, even pointing out that the men at her psychic studies college were less masculine. The fundamental difference is this: feminine forms are interested in seeking experiences, visions, good feelings, and so on. The masculine paths, on the other hand, are interested in transcending the human condition altogether.

\paragraph{Chakras and the Hero}
Let's look at the chakra studies, for example. Different stones are supposed to correspond to different chakras, which in their turn, correspond to the various endocrine glands. Of course, it is “all good”, and there is no concern about artificially stimulating the lower forces or inadvertently triggering a higher chakra before the person is ready and prepared to deal with those energies.

For example, if the lower chakras represent subconscious or animalistic forces, as \textbf{Carl Jung} says, we are not “honoring the earth” by placing our own selves under their power. Obviously, if everything is “good”, if everything is to be “honored”, then there is no place in this world for the Hero. The Hero knows what is good and evil, what is honorable and dishonorable. Without such distinctions, there is no reason for the quest. That is why the modern mind, in particular, has gone astray in the jungle of purely human reasoning and feelings. And since Christ is the archetype of the Hero, this message is profoundly anti-Christian. Here is Jung, speaking to Miguel Serrano:

\begin{quotex}
Do you know what the Self is for Western man? It is Christ, for Christ is the archetype of the hero, representing man's highest aspiration. All this is very mysterious and at times frightening. 

\end{quotex}
So to the extent that the West loses its faith in Christ, it also denies the Hero. It then denies that man has any high aspirations, since the basest things are now celebrated and honored. All this to avoid the mysterious and the scary. The remaining Knights in the world accept the mysterious and move forward despite the terror.

\paragraph{Chakra Energies}
I'm not denying that such energies exists, just that the stone is not its source. In that system, the I, or Self, is passive. Hence, the stone, or endocrine gland, must be vibrated from the outside in order to produce some effect on the Self. However, the stone represents a privation, a lack of understanding our own psychic centers, so we project our own energies onto the stone. If there is energy blockage in the psychic centers, the better course is to deal with them internally, with the Self as the active force, not to rely on external factors to do that work of self-awareness and self-development for us. This whole line of thought is an example of the lack of self-possession.

In his meeting with Miguel Serrano, Carl Jung offers a deeper commentary on the chakras, which he considers as purely psychic centers of consciousness, not physical entities associated with endocrine glands. He pointed out the interesting point that different cultures may be centered on different chakras, in this sense. As an example, he referred to a discussion with a Pueblo Indian chief, who criticized white people for thinking with their heads. For him, only crazy people think with their heads, and normal people think from their hearts.

A very interesting point that Jung made is that the ancient Greeks also thought with their hearts. This means that it is literally impossible for a contemporary scholar to understand ancient Western culture, to the extent that he is dominated by his head. One must learn, first, to think with one's heart.

Yet, there is something even more appropriate, certainly due to synchronicity. Jung goes on so say this about contemporary Westerners:

\begin{quotex}
They think only with their tongues. They think only with words, with words which today have replaced the Logos. 

\end{quotex}
That may take some time to fully sink in, and is worth an entire article. But basically, he is describing those under the influence of “rhetoric” in Michelstaedter's sense. Rhetoric has now replaced the Logos, through whom all things are created, with an artificial ideology. You see, we are back to the illusions of the causal body\footnote{\url{https://www.gornahoor.net/?p=7224}}, although from a different angle.

\paragraph{Rhetoric and Psychoanalysis}
It is curious that Erica knew nothing of this, although she has been undergoing Jungian analysis for over five years. That can't be cheap, especially since her therapist made house calls. Moreover, she has had another analyst for decades, but can't use him all the time due to his exorbitant rates. She made sure I understood that her analysis was expensive.

However, that is mere rhetoric to me. I was more interested in the process. I asked here the obvious question, “Did she understand the goal of Jungian therapy?” Her immediate confusion showed me that she had never even thought about that at all. Of course, the goal of Jungian analysis is the creation of the Self, beyond the Ego, through the process of individuation.

\paragraph{Magic Weddings}
So I offered to help her through the process of individuation, since great gains can be made in our era through Courtly Love, i.e., a common spiritual journey shared by the Knight and his Lady. For misplaced Knights, born in a place and time where they no longer have a role, this would be their best path. I'm afraid I don't have the requisite degrees or certificates, and I certainly do not overcharge. As I said, there is no acknowledgment of Knights in our day. Now, in therapy, the male role is taken by the psychologist; I don't see how a female therapist, as in her case, can be the catalyst for that process.

Serrano builds on Jung's idea with his own description of the magic wedding. In some ways, he comes close to Tomberg's idea\footnote{\url{https://www.meditationsonthetarot.com/the-word-is-made-flesh}} that the fruit of the union of the male and female principles in consciousness gives birth to the Logos, the Christ within, which Jung claims is the archetype of hero in the West. That is the true self.

While both Jung and Serrano describe this wedding in terms of an alchemist and the mystical sister, a physical female is not absolutely essential from the male perspective.

Man is self-sufficient, since in him the anima is the passive element. To give birth to the True Self, the anima must be pacified so that it can receive the imprint of a higher source. That is why Jung says that a physical woman is not absolutely necessary for the individuation process. Rather, she is a privation, so the energy of the anima exists in the man and can be claimed in an act of self-possession.

The situation is different in a woman. For her the anima is the active element. So to make it the passive element would force her to no longer be a woman, psychically. That is why she needs a man to complete the individuation process, and that role was usually taken up by the analyst.



\flrightit{Posted on 2014-06-09 by Cologero }
