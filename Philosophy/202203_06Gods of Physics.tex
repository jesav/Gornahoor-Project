\section{Gods of Physics}

There is a reason why physicists make lousy philosophers. It's because they believe that physics itself is philosophy. What they take to be novel theories turn out to be equivalent to philosophical propositions, both good and bad. Since I listen to physics videos on youtube during my spare time, here are some topics that came up. I would have liked to interview some of them with my own questions.

\paragraph{Hilbert's Hotel}
One fellow who claims to represent the Council of Trent is obsessed with the Kalam Argument. I have some sympathy for the argument because an infinite past is incoherent. Certain types think that Descartes's number line is well-ordered. No, actually, it is not so.

So he posted three videos on the topic, parts of which I attended. The most recent one showed confusion by using amount, number, and cardinality interchangeably. They debated Hilbert's Hotel incessantly as if it represents reality. It is a joke, just like Schrödinger's Cat. The Hotel depends on ordinal numbers, not cardinal numbers. That is why the new guest needs to be moved into the first room rather than at the end. The math of ordinal numbers show why: $1 + \omega = \omega$ but $\omega + 1 \neq \omega$.

Suppose the set of rooms are numbered from 1 onward and each guest is given an even account number. Hence, there is a bijection between the set of rooms and set of guests (each room number matches an account number). So where does the new guest come from? Is there an even number that no one has ever discovered?

\paragraph{Actual Infinity}
This is another confused notion, although it used to be debated. What he really means is that there are an infinite number of things in the world. Our understanding of that matter has changed. ``Infinite" should be restricted to mean ``unbounded", and ``transfinite" for set theory. Then you can ask if the number of things in the world is countably infinite, or even more. But that is a question for the physicist to answer, since it is a question about the material world. There is no theory nor evidence to support it, so the answer is probably no.

\paragraph{Presentism}
This is the belief that only ``now" is real, but not the past nor future. It's an odd way to pose the question, since time is ideal, not real. The metaphysical teaching is that all manifest things are compossible.

\paragraph{Gödel's theorem}
Although the proof brilliant, the impact is exaggerated. No mathematician has been put out of work by this theorem. It shows that computing machines cannot generate proofs algorithmically. But if we know that something is true, then it does not matter at all.

\paragraph{Evolutionary Biology}
A fellow who calls himself the Jolly Heretic has a degree in theology (i.e., religious studies) but now calls himself an evolutionary biologist. That shows why the field is suspect.

He demonstrates statistically that religious belief is healthy both for individuals and society. He attributes this to healthy genes. Unbelievers will fail to breed themselves in sufficient numbers. Nevertheless, he personally is an atheist. So this is an imaginary interview:

Q: Why are you an atheist?

A: Because I don't see the evidence for God. (He expressed this view.)

\begin{enumerate}
\item That can't be true, since you don't believe in free will. 
\item Then what is the reason? 
\item It is because you don't have the religion gene, as you yourself argue. That makes you a dysgenic mutant. Since you accept group selection, that makes you a danger to the social structure. It is more than just a personal opinion. Ergo, a healthy society would work to limit your influence. 
\end{enumerate}
\paragraph{Moral Burden}
One fellow desperately seeking Theories of Everything revealed that a great moral burden was lifted from him when he accepted atheism. He did not reveal his moral flaw. But his position is not unusual. Most atheists are not really persuaded by arguments; rather they despise a world in which God exists.

\paragraph{Necessary Being}
The physicist needs to answer the question about the necessity of the existence of the universe. If the universe is contingent, then what caused it? If the universe is necessary, then that is what we mean by God.

\paragraph{Prime Matter}
Prime matter is ``chaos", that is, it is totally undifferentiated. This is \emph{materia prima}. Physicists then study what Guenon calls \emph{materia seconda}. The lowest order accessible to physicists is the quantum field, which is just a probability distribution. It is nothing in particular.

How and why did particular things arise? That is the measurement problem, which is not understood.

\paragraph{Predetermined}
Some physicists believe that everything that happens was predetermined. As such, there is no physical proof, nor can there ever be. That is because the part cannot understand the whole; moreover, much of the math is not computable in a finite amount of time. This suggests that the universe is not executing an algorithm.

Suppose, however, that determinism is true. That implies that all possibilities of manifestation pre-existed in the Big Bang. A more sensible version of the same idea is that everything exists in God's mind. Hence, they introduced nothing new.

\paragraph{Conformal Geometry}
Conformal geometry is being used to show that the end of one universe becomes the Big Bang of the next. Interesting. Traditional teaching is that the end of one cycle is the beginning of the next. Perhaps the physicists have discovered the series of manvantaras.



\flrightit{Posted on 2022-03-06 by Cologero }
