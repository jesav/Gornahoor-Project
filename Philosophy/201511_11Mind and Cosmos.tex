\section{Mind and Cosmos}

\label{sec:MindandCosmos}
Being a review of \emph{Mind and Cosmos: Why the Materialist Neo-Darwinian Conception of Nature Is Almost Certainly
False}, by \textbf{Thomas Nagel}.

\begin{wrapfigure}{rt}{.4\textwidth}
 \includegraphics[scale=.5]{a20151111MindandCosmos-img001.jpg}
\end{wrapfigure}

A couple of weeks ago I watched the notorious atheist and Darwinist \textbf{Richard Dawkins} interviewed by a
non-descript host on a cable news network. Since the context was a discussion of Ben Carson's belief in
creationism, the host listened with rapt attention to, but little understanding of, Dawkins' presentation.
Of course, for the half-educated intelligentsia represented by the host, a blind belief in the “theory of evolution” is
a status marker even though they neither understand it in depth nor are aware of its ultimate consequences for human
life and thought.

Without defining the term, Dawkins asserted that “evolution” is a “fact”. We agree that the two basic components of
evolution are facts. These are:

\begin{itemize}
\item \textbf{Descent with variation}: the offspring are similar, but not identical to the parents. 
\item \textbf{Natural selection}: some organisms will reproduce themselves better than others in their natural
environments. 
\end{itemize}
There are subsidiary facts, such as:

\begin{itemize}
\item DNA sequences of similar organisms have many commonalities 
\item The age of the earth seems to be quite old 
\item The fossil record shows organisms arising and being replaced by other organisms 
\end{itemize}

\paragraph{Nagel's Thesis}
There is no point in disputing settled scientific facts. Instead, Nagel himself points out some additional facts:

\begin{itemize}
\item \textsc{Consciousness}: its subjective character has no physical explanation 
\item \textsc{Cognition}: thought and reasoning are correct or incorrect independent of the thinker's
beliefs 
\item \textsc{Values}: values are real, not merely subjective 
\end{itemize}

Properly understood, Nagel shows that these facts cannot be explained by nature understood as simply physical and
material. Nagel is an atheist, just like Dawkins, so there is no question of special pleading for a partisan religious
view. The two components produce different results.

\begin{enumerate}
\item Descent with random variation should work like a random walk\footnote{\url{https://en.wikipedia.org/wiki/Random_walk}}. Specifically, “evolution” is not evolving in a
particular direction, rather, it is probably going nowhere. 
\item Yet that is not what is observed. Instead, nature or the environment seems to channel evolution in specific
directions. 
\end{enumerate}

\paragraph{Antireductionism}
As part of organic life on Earth, man is subject to a multitude of laws. First of all, as a corporeal being, he is
subject to the laws of physics: gravity, conservation of energy and momentum, and so on. Then, he is subject to the
laws of chemistry, since a large number of chemical reactions constantly occur in the body.

However, physical and chemical laws are surely insufficient to understand any form of life, never mind human life. For
example, it would not be possible to understand the movement of people in a city just based on force and momentum. It
is not even possible in principle.

So, why would the “theory of evolution”, as a biological law, be able to explain the totality of the human being? That
is what is objectionable in neo-Darwinism. The facts as such are not in dispute. What is far from obvious is that
genetic variation and natural selection together explain everything about human life. How can DNA cause conscious and
sentient beings?

\paragraph{Chance and Intelligibility}
Nagel begins his discussion with the notion of the intelligibility of the world. That is equivalent to the Principle of
Sufficient Reason, the notion that everything about the world can be understood at some level. Absolute Idealism\footnote{\url{https://www.gornahoor.net/?p=8366}} (e.g.,
Plato, Schelling, etc.) considers rational intelligibility to be at the root of the natural order. So Nagel considers
himself an absolute idealist (but never writes of the Absolute in this book).

Since mind is part of that order, it, too, must be intelligible. Nagel denies that physical, chemical, and biological
laws —i.e., efficient causes alone— suffice to explain mind. Therefore, he is compelled to bring
in the idea of teleology, or final causes, to explain the emergence of mind. That acts as a “pull” to the “push” of
efficient causes. Although he does not express it this way, efficient causes are quantitative while final causes are
qualitative. Since the whole scientific enterprise began with Francis Bacon's rejection of final causes and
Galileo's rejection of qualitative explanations, Nagel in effect rolls back thought to a pre-modern era.

Nevertheless, it is not a simple reaction against the modern world, since it also incorporates whatever truths modern
science has given us.

Unfortunately, while science has promised to make the world intelligible, it has done so by leaving out important
features. First of all, the opposite of intelligibility is chance or randomness. In fact, a random sequence is such
because the next element of the sequence cannot be inferred from any of the preceding elements. Perfect randomness\footnote{\url{https://www.gornahoor.net/?p=4970}},
therefore, is the denial of the Principle of Sufficient Reason.

\paragraph{Cosmos}
Every outdoorsman knows that a random walk in the woods leads nowhere; most likely, you would end up close to where you
started. That is why you need to mark your path so you don't traverse the same places twice. Hence, if a
city boy was lost in the woods, but emerged two days later, you might call that a miracle. Or else, you might suspect
he had some skills he hadn't owned up to.

That is the situation as Nagel sees it. The emergence of conscious, intelligent, and rational beings by chance alone
does not seem at all plausible. Now, the first factor in evolution, viz., variation or genetic drift, is certainly
random. If it follows a random walk, it should go nowhere. Fossil records should show species evolving backwards to
more primitive forms, for example. In other words, there is no “direction” to evolution, or, in other words, no
teleology.

On the other hand, the second factor, natural selection, is not random. Dawkins himself did an experiment with
Scrabble-like tiles. By randomly placing the tiles, followed by a selection mechanism, he would end up with an English
sentence\footnote{\url{https://www.gornahoor.net/?p=8253}}. In his example, Dawkins was the intelligent selection factor.

So if life as we know it is the result of random variations and natural selection, Nagel explores the selection factor.
Specifically, what would nature have to be like to produce human beings?

\paragraph{Consciousness}
Nagel endeavors to explain three facts: the emergence of \textbf{consciousness}, \textbf{cognition}, and \textbf{value}
in biological species. As a committed naturalist, he rejects theological explanations that account for those facts from
a force outside nature. That is fine since the general understanding of God in exoteric religious adherents is usually
defective, creating as much confusion as insight. Likewise, he rejects reductive naturalism that, in effect, denies the
three facts rather than explains them.

The distinctive feature of consciousness is its subjective, or we would say qualitative aspect. There is no explanation
of conscious experience in terms of physical laws. While brain states may empirically be shown to create certain
experiences, that opposite is also true. Consciousness can likewise affect brain states.

This all seems difficult for some to accept. A diehard reductionist will rely on behavioristic explanations. For
example, if an organism responds to a flash of light, that behavior is an indication of consciousness. In that view,
then, there is nothing to explain. Similarly, human beings will “report” having certain sensations and experiences. The
reporting is all that matters.

Yet that misses the essential point, viz., the subjective aspect of consciousness, which it attempts to make objective.
Are the automatic doors at the supermarket conscious in any sense? According to the behaviorist criterion, perhaps they
are. So why do we believe an octopus is conscious but not a door?

Nagel concludes, then, that \emph{mind is an essential part of nature}, not a byproduct of material processes. This is a
form of panpsychism.

\paragraph{Cognition}
Nagel then turns to “cognition” as he calls it, which appears in the human being. Metaphysically, the human being is
characterized by “intelligence”, which is different from seemingly intelligent activity in animals. Specifically, Nagel
defines cognition as “the functions that have enabled us to transcend the perspective of the immediate lifeworld given
to us by our sense and instincts, and to explore the larger objective reality of nature and value.”

Thought and reasoning are correct or incorrect in virtue of something independent of the thinker's beliefs.
Logic, mathematics, and metaphysics are timeless, hence immaterial. This is reminiscent of a more sophisticated version
of C. S. Lewis' Argument from Reason\footnote{\url{https://en.wikipedia.org/wiki/Argument_from_reason}}. Cognition certainly cannot be explained solely in terms of behavior.
And it should sound odd that a life form would arise that would seek to understand its own origins.

Now a reductionist may try to refute this in a couple of different ways. One is the emergence of serendipitous uses for
features that evolved because of reproductive fitness. For example, a hand came to be used by a Michelangelo to create
beautiful art. Certainly, that in itself has no reproductive value. But that inadvertently confirms an earlier point:
biology alone cannot explain everything about the human being.

Another is the obvious and glaring lack of logic and rationality in the human race. Evolutionary psychologists have
noted many of the logical fallacies and irrational beliefs of humans. Nevertheless, they have biological fitness. True
rationality, then, is just a special case of the origin of thinking.

It is rather odd that false ways of thinking lead to reproductive success. The rare thinking occasions involving
objective truths probably have little reproductive success. For example, try discussing this review on your next date;
I can guarantee you will spend the night alone. Moreover, the most scientifically advances societies usually have
negative birth rates\footnote{\url{https://en.wikipedia.org/wiki/Population_decline}}.

\paragraph{Values}
Nagel then points out the existence of objective standards of value: good and bad, right and wrong. This he calls “value
realism”. Again, he claims that objective values make no sense in a materialistic universe. Things are good or bad not
because genetically determined behaviors lead to the preference of one thing over another.

Human action involves more than physiology and desires, it requires judgment. Clearly, then, this requires “free will”,
or the ability to make a moral judgment.

Nagel shows the richness of absolute idealism in retrieving a deeper, more human, view of the cosmos, beyond the
materialist reductionism that dominates educated thinking today. Nagel accomplishes this while fully incorporating
scientific knowledge.

Mind, consciousness, intelligibility, rationality, judgment, free will, are all restored in a more comprehensive
understanding of the cosmos. Nagel does this sparingly, a type of philosophic minimalism, with no brick that is not
essential to the edifice he has created.

\flright{\itshape Posted on 2015-11-11 by Cologero}

\begin{center}* * *\end{center}

\begin{footnotesize}\begin{sffamily}

\texttt{Mercurius on 2015-11-13 at 23:08 said: }

A clear, intriguing, and captivating review Cologero, stimulating much interest in reading the five or so major titles
you've been lately discussing, between here, in “The God of Metaphysicians”, and “Spiritual Regeneration”
posts.

Though Nagel is, as noted above, an “atheist” (maybe better really, a non-theist), there is something serious to be
noted in a “modern” science which begins to consider, and recognize, that consciousness is a sort of fundamental
universal constant. Independent and objective, even in the most “materialist” constructs–which then really, changes
everything.

Brings to mind, a small passage in Kingsley, where he exclaims:

“The Iatromantis was someone who was a master of the state of awareness. Waking is a form of consciousness, dreaming is
another. And yet this is what we can live for a thousand years but never discover, what we can theorize or speculate
about and never come close to–consciousness itself.

Its what holds everything together and doesn't change”.

That is Kingsley, but in two beautiful passages, Emperor Aurelius, and Shopenhauer seize upon the same state–creating
“reboot” points, within the Western cannon.

Related, is an interesting, and recent drive, within physics, to in fact address the “c” issue, keeping it within well
enclosed constraints of “materialism”–considering it as indeed, a “panpsychic” element of nature–but entirely, at the
same time, while avoiding ideological conflicts, seeking to stuff in into gross biology–curious nonetheless, to even
BEGIN regarding consciousness as a “state” of “matter”, on par with accepted “states”. 

Perhaps not quite “traditional”–but, maybe with some modifications, at least for an exoteric sake, these ides of
consciousness can find assimilation with Samkhya, Greek “atomism”, and Stoic “Logoism”?:

\url{https://medium.com/the-physics-arxiv-blog/why-physicists-are-saying-consciousness-is-a-state-of-matter-like-a-solid-a-liquid-or-a-gas-5e7ed624986d}


\hfill

\texttt{Olavsson on 2015-11-14 at 12:55 said: }

While this work of Mr. Nagel seems highly valuable, something worth reading in order to improve one's
understanding of what a contemporary, “up-to-date” so to speak, refutation of various fundamental modernist assumptions
about humanity, consciousness, evolution and the world might look like, there is something here which makes his
`perspective' deviate from fully qualifying as Absolute Idealism, don't you think?
In his worldview, the notion of “nature” still seems to be of supreme centrality, although he accepts consciousness as
an inherent, not merely accidental and contingent, “part of the picture.” Is there a presence of actual transcendence
within this worldview, which would make “nature” itself merely one of several aspects of particular “states” of the
total Being rather than the supreme reality per se? While Nagel's understanding certainly is a vast
improvement to the reigning scientific paradigms, as far as integrating Idealism is concerned, it still seems to
preserve a concept of an independent “nature” that didn't really exist anywhere in the Traditional world.
But I might have misinterpreted. Certainly worth a read in any case.

I have not been able to follow Gornahoor as much as I'd like to during the last months as my access to
internet is limited and there's been other things demanding my attention, but I will now try to comment
more, and hopefully participate in one of your Gnosis cycles.

I must repeat what I've said before, that this website offers something truly unique to those on a quest for
truth in the west today.


\hfill

\texttt{Cologero on 2015-11-18 at 18:37 said: }

Olavsson, Nagel referred to the absolute in a passing comment and, apparently, it was not necessary for his argument. He
restricted himself to a form of naturalism. That is probably the most effective approach in our time, since overtly
religious or complex metaphysical schemes are beyond the pale for the educated classes.

Since he mentioned Schelling, that is where we should probably look for a fuller picture. By extrapolating
Nagel's thesis, the physical world does not form the ground of consciousness. Conversely, consciousness
does not “create” the material world. This avoids dualistic solutions. If the mind is part of the world, then
man's mode of being in the world is as a conscious body. So, neither matter nor mind is fundamental; hence,
for Schelling and presumably Nagel, the Absolute is their common ground.

This is also more Tao like: the Abolute as Tao, and mind/matter as male/female or yin/yang.


\hfill

\texttt{Olavsson on 2015-11-22 at 12:57 said: }

Cologero: Yes, I think his approach somewhat focused on the level of naturalism may be useful for countering the gross
materialism dominating the secular pseudo-elite of modern civilization. It is imperative that views more closely
resembling idealist philosophy, in contact with science and `updated' to the current situation,
are made relevant again. On a personal level, however, I am mostly concerned with inner realization and what
traditional doctrines mean for my own path to higher spirituality, so the question of which approach is most effective
and influential on the collective level of academia, the scientific community, culture etc, is of secondary importance;
and it is in that other capacity I find Gornahoor's contribution most appreciable.

Regarding the question of the relationship between consciousness and the so-called material world: Wouldn't
it be correct, from the point of view of traditional metaphysics, to assert that Mind, in a supra-individual sense, is
indeed prior to “the world”? Of course, it is not the conditioned consciousness of individual beings existing in the
world that has `created' that world. Isn't it rather the case of individual mind,
such as that of the fallen humans we know, having its ultimate origin in the ideal primordial Mind (in some traditions
the `Buddha Mind’), from whose emanation the relative existences typical of the limited minds
arise, which are in turn conditioned by and co-dependent on the `world' which is a part of
their state of being, a `world' that is not self-existent? That the mind
`creates' the `world’, of course, can only truly be said if we speak of
a `higher' Mind that is not conditioned to be co-dependent on that world, or rather state of
being. It doesn't count for the grosser levels of individual mortal consciousness still subject to cyclical
existence. To say this is not dualism; it is the same principle as when you have stated that “the subtle rules the
dense.” If we accept the traditional possibility of spiritual liberation from the conditions of this
`world', as a state of being, then the conclusion that Mind, if integrated with the ultimate
truth of its origin, is superior to material existence, not just two sides of the same coin, is inescapable. Just to
avoid misunderstanding: I'm not trying to correct you, only throwing some spontaneous reflections out
there. I don't think we disagree here in terms of ultimate principles when looking beyond different
wordings in one specific context. Since the individual minds that we know all experience the same
`world', they do not all `create' it in the sense of some arbitrary
illusive projection from their own subjective starting-point; it obviously has a cause beyond their subjective
experience, now corresponding to a very conditioned form, extremely reduced even compared to the Primordial State
(which is still too conditioned to corresponds to the state of one fully awakened and liberated). So in order to find
the `Mind' that is indeed superior to and more Real than material existence, such lower,
conditioned minds would have to be transcended. The material world does not form the ground of this ordinary
consciousness as commonly believed today, as you point out, because its true `ground' from
which it is emanated or `reflected' is transcendent, but nevertheless, because of this
`reflection' in the lower `waters', these minds now find themselves
conditioned by this `world’ – which is why the materialists believe that the very principle of
consciousness originated from these conditions, in which case no higher freedom would be possible. Likewise, the
material `world' itself is conditioned by consciousness, albeit in a higher sense, as it would
not exist outside of conscious experience, since we are, in this metaphysical perspective, dealing with a state of
being, not just objective “stuff” existing by itself “out there.” I believe this is the best view (though here
expressed simply and not with the subtlest profundity) on the co-depending relationship between consciousness and
matter. On the one hand, you have the facts of our experienced consciousness “here and now”, which is manifested in
certain conditions and may be affected by them, and on the other hand, the higher possibility of consciousness in an
ideal sense. If a state very close to the Absolute may actually be attained through the realization of a being, which
is affirmed by the highest initiatic doctrines (for example Buddhahood, diamantine and indestructible, even in this
life having realized its centre beyond life and death), then mind must somehow be more fundamental than matter, since
matter, as in a stone, cannot serve as the starting-point for a transcendence of its own condition, while mind can. The
point is: the mortal mind needs these material conditions to operate as long as this state of being lasts, of which
matter is one of the relative, dependent conditions, but consciousness may use these conditions as a springboard to
reach beyond them, and the mind-stream will outlast the death of the physical constituents, though in a different form.
This, of course, does not hold as an argument in debates with materialists, since certain metaphysical premises must
already be accepted as true. That is usually the approach I choose to take, since I'm not so much
interested in finding common ground on which to debate with materialists as I am in realizing for myself what walking
the path of the sages of old has to offer a being today.


\hfill

\texttt{Cologero on 2016-08-30 at 19:38 said: }

Update on science.\newline
Note the claim that we made in this post:

\begin{quotex}
Instead, nature or the environment, \emph{seems to channel evolution in specific directions}. 

\end{quotex}
Compare that claim to this most recent scientific findings, of which I was just made aware:

\begin{quotex}
Rather than genes simply “offering up” a random smorgasbord of traits in each new generation, which then either prove
suited or unsuited to the environment, it seems that \emph{the environment plays a role in creating those traits in
future generations}. 

\end{quotex}
The full article is available at \textit{Why everything you've been told about evolution is wrong}\footnote{\url{https://www.theguardian.com/science/2010/mar/19/evolution-darwin-natural-selection-genes-wrong}}.

This shows that metaphysical principles can provide information even before it can be verified empirically.

\end{sffamily}\end{footnotesize}
