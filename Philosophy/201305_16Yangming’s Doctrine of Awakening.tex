\section{Yangming's Doctrine of Awakening}

It was pointed out in a discussion here that \textbf{Valentin Tomberg} wrote the following on “exteriorization”, summarizing the respective attitudes of Buddha and Christ to the vision of a damaged world:

\begin{quotex}
The Buddha saw the true nature of the world and that it was sick. Considering it incurable he instituted an effective path to escape it. Christ saw the same truth, but believed that the world could be cured and instituted the path to heal the fallen world from the inside.

\end{quotex}
In simpler times, men referred to the Buddha's reluctance to confront the world as “Oriental quietude”. These days, we are privileged to know that many in the Orient did not share that attitude, not at that time nor at other times. One example that has been on my mind recently is the Neo-Confucian philosopher-\emph{cum}-general \textbf{Wang Yangming} (1472-1528). The Chinese characters in Wang Yangming's name literally spell out, “King of Elucidating the Solar [Yang].” This name was given to him by his students.

Yangming founded an entire school of Confucian thought, which caused reverberations in China and Japan. One Japanese student, Kumazawa Banzan, said of Yangming, “He initiated a new movement in Confucianism, which lays stress on self-introspection. … We owe to him much for this, because the man who gives attention to his own inner self finds there the true man, and the truths of the classics revealed.” The ordinary Confucian draws his morals from his interpretation of society's needs; Yangming, in contrast, rediscovered the knowledge of the inner spirit and transcendence found in the earliest Confucian texts.

Yangming's school is one of practicality and action. In contrast to the tonsured, tenured, and timid Confucian scholars of his era, but in accordance with the Master himself, Yangming accepted government commissions and calmly directed armies into battle. He said, “No one who really has knowledge fails to practice it. Knowledge without practice should be interpreted as lack of knowledge.” He in fact denounced Buddhism as cowardly:

\begin{quotex}
The Buddhists are afraid of the responsibilities of father and son, and hence avoid such responsibility. They are afraid of the perplexities of prince and minister, and avoid becoming prince or minister. They are afraid of the responsibility of husband and wife, and hence avoid becoming husband and wife. … If we scholars have a father or a son, we recompense them with affection; if we have a prince or a minister, we use righteousness in dealing with them; if we have a husband or a wife, we pay attention to differences. Have we then been influenced by the relationships of father, son, prince, minister, husband and wife?

\end{quotex}
Banzan went even farther, pointing out that reincarnation was an unprovable concept, so therefore all Buddhism's warnings about the impact of bad karma and future rebirth were mere empty threats. Banzan and Yangming share a belief that good behavior does not come from threats of hell, but from a much simpler source: the Conscience or Intuition within the human mind. To define the nature of this Intuition, Yangming approved of this remarkable statement by one of his students:

\begin{quotex}
There is nothing with which the intuitive faculty does not make one acquainted. It is not necessary to deliberate in the least, nor is it necessary to assist its development in any way, for it is very trustworthy and perfectly clear. When it is stimulated it responds, and when it is influenced it perceives clearly. There is nothing that it dos not make clear, nothing that it does not realize, nothing that it does not apprehend. All the sages have traversed this road, all the virtuous men have followed this track. There is nothing else that is like a spirit, for it is the spirit; nothing else emulates Heaven, for it is Heaven; nothing else is more in accordance with the Supreme Ruler, for it is the Supreme Ruler. 

\end{quotex}
To this Yangming only adds: “If you know this, you know that there is no further task before you than that of extending intuitive knowledge to the utmost.” We see here that Conscience or Intuition is that part of mind with an inherent understanding of Principle. “If reflection has reverence to the action of Intuition, there are no thoughts that are not in accordance with natural law.” Cologero points out to me that the Eastern Orthodox similarly saw the “gnomic will”, a type of free will where private desires are entertained, as a potential hindrance on the “natural will”, the free will displayed by Christ who acted perfectly in accordance with God.

Like Buddha, Yangming says that our selfish desires only get in the way of serving higher interests, in this case an interest which is to be found within the mind, but which has nothing to do with the narrowly conceived self. The only knowledge we are lacking in is the Way to acting with perfect intuition, natural law, Heaven, and the Supreme Ruler. It is this Way that must be taught, as Confucius tried to teach, but the \emph{destination} is already known.

The Way according to Yangming is recognizing the origin of phenomena in mind, and tempering the mind to enhance one's naturally skillful judgment. As in Buddhism, the passions and desires of the mind are viewed as hindrances, “clouds” blocking the “sun”. A perfectly clear mind will be able to make correct decisions without the confusion of private interests. This is the sincere mind, “having no depraved thoughts” (\emph{Analects} 2.2). Unlike Buddhism, though, this does not mean that practitioners should refrain from judgment about worldly things. Practice should accompany contemplation. Our intuition tells us to love the Truth and the Right; it also tells us, with equal validity, to fight for what we love.

Some must be wondering, did \textbf{Julius Evola} know about this guy? He did, but not directly. A Japanese Zen writer, calling Yangming by his Japanese name Ō-Yōmei, referenced him, and Evola misquoted the writer in \emph{Doctrine of Awakening}.

\begin{quotex}
When commanding an army in battle, even in his headquarters, Ō-yō-mei would discuss Zen doctrines. He was informed, on one occasion, that his advanced troops had been defeated; he calmly continued his discourse. Shortly after, he was told that, in the later developments of the battle he had become the victor. The commander remained as calm as before, and did not, even then, change his discourse This is how one gradually apprehends the existence of a principle that cannot be altered by doubt or fear any more than the light of the sun can be destroyed by fog or clouds. [\emph{Doctrine of Awakening}, 228-9]

\end{quotex}
I find it interesting that Julius Evola read through a book on Zen and pulled from it one of the few lines that definitely had nothing to do with Zen. Many Japanese of that time combined Zen and Yangming thought, but Yangming himself considered his path of action as a world different from Buddhist contemplation, and he would consider Evola to have confused the two. He wrote, “To sink one's self into abstraction and keep perfectly motionless, and to prearrange one's thinking, truly imply using wisdom according to one's selfish purposes. This is akin to throwing away your intuitive knowledge.” The “natural will” is a will which acts on the world, not merely an understanding of the world's true nature.

In an essay translated on this website\footnote{\url{http://www.gornahoor.net/?tag=okawa-shumei}}, Okawa Shumei argued that Japan of the Edo period implemented a better, more living Confucianism than China's stagnant Legalism. If this is at all true, it is because they were able to draw on Yangming's Confucianism of the spirit. His teaching is far different from the “Confucianism as social system” we learn of in modern schools. Banzan wrote: “If I give attention to my inner self, I can find truth; no matter how clever and exact a man's teaching may be, if he does not study his heart, his teaching is empty.”

\paragraph{Notes/Sources}
For a footnote to this post about Yukio Mishima's use of Wang Yangming, see my blog\footnote{\url{http://avery.morrow.name/blog/2013/05/wang-yangming-on-gornahoor/}}.

This post was brought to you by the Internet Archive: \textit{The philosophy of Wang Yang-ming} (1916)\footnote{\url{http://archive.org/details/thephilosophyofw00henkuoft}} and \textit{Light from the East; studies in Japanese Confucianism}\footnote{\url{http://archive.org/details/cu31924022939205}} (1914), both able translations.

\flrightit{Posted on 2013-05-16 by Avery Morrow}

\begin{center}* * *\end{center}

\begin{footnotesize}\begin{sffamily}

\texttt{Jason-Adam on 2013-05-16 at 13:41 said: }

Wow – I need to read these books – I never heard of Wang Yangmin before but am really liking this. I gotta research his influence in Korea soon.

\hfill

\texttt{August on 2013-05-16 at 19:46 said: }

To `act without acting’ is simply contemplation with an additional layer of activity on top. The activity is effaced in the presence of contemplative knowledge, even if it be inspired by it – this is why it is called `action without acting’.

Whether such a contemplative does a little or a lot is indifferent. After contemplation, there is only contemplation.


\hfill

\texttt{August on 2013-05-16 at 19:50 said: }

Fundamentally though, `act without acting’ is a reference to contemplation as pure act.


\hfill

\texttt{Caleb Cooper on 2013-05-17 at 18:50 said: }

In Buddhisms' defense, after Christ it developed the Mahayana `Greater Vehicle' school in which the highest ideal was not Nirvana, but the Bodhisattva; once a soul had become enlightened and could achieve Nirvana, it was supposed to renounce Nirvana out of compassion for all other beings, and devote itself to continually reincarnating until all sentient beings are liberated from suffering. 

According to Tomberg this development occurred because Christ made possible the redemption of the fallen world, and so the Buddhist tradition sensed the change in the possibilities of the world, and was inspired to institute a path to help realize this possibility.


\hfill

\texttt{Avery Morrow on 2013-05-17 at 21:33 said: }

This gentleman who lived in China in the 15th century and studied Mahayana then seems to disagree with Tomberg about compassionate, worldly qualities in Mahayana. But I understand Chinese Buddhism was not doing too well at that time, and Mahayana definitely succeeded in other nations based on different local adaptions.


\hfill

\texttt{Jason-Adam on 2013-05-18 at 14:56 said: }

I think the Buddhism vs Oyomei dispute is another manifestation of contemplation vs action we saw with Guenon and Evola. 

My expertise is Korea, so I cannot speak of Chinese history without fear of error, but let me mention some parts of Korean history that may be useful : in 1392 there was a change of dynasty and religion – from the Koryo kingdom of the Buddhist Wang family to the Confucian Chosun kingdom of the Lee family.In the last years of Koryo the established Buddhist hierarchy was very corrupt (research the name Shin Don) and so many people in the elite turned against Buddhism and saw it as a supersition that led people to despie the world and their country – a very Evolian critique.


\hfill

\texttt{Lionel Chan on 2018-08-08 at 00:47 said: }

Re: “the intuitive faculty”, all very well, even when clearly differentiated from bodily instincts (hence Guenons coining or at least usage of “intellectual intuition”.

But only for the elite that are able to process this properly in isolation. Such an idea is utter disaster for a world where the just hierarchies are crumbling or gone, where it will propel and empower fools to make things worse.

China did end up embracing Maoism afterall… some monotheistic veiling (esp. Islamic, which embraces the political and military, as does Yangming) for the Mercy of those not equipped by God to be Gornahoor frequenters, might be of great use for my people's.


\hfill

\end{sffamily}\end{footnotesize}
