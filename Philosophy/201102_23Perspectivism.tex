\section{Perspectivism}

\begin{quotex}
It seems to me important that one should get rid of the All, the unity, some force, something unconditioned; otherwise one will never cease regarding it as the highest court of appeal and baptizing it `God'. One must shatter the all; unlearn respect for the all.

\end{quotex}
Some definitions:

\begin{description}
\item[Naïve Realism ]

The view that reality is ``out there, right now" and that knowing is like ``seeing" (see Bernard Lonergan, \emph{Insight}). This is probably the view of the average unreflecting person. However, it is difficult to defend philosophically. 

\item[Correspondence Theory of Truth ]

The idea that truth consists of statements that ``correspond" to reality ``out there". Naïve realists usually believe this, though some philosophers accept it, too (e.g., the logical positivists). In this case knowledge is equivalent to a collection of true statements. The philosopher Alvin Plantinga describes the omniscience of God to be: ``He believes all true statements and doesn't believe any false statements." This assumes that all Truth can be expressed verbally. 

\item[Perspectivism ]

Nietzsche's position regarding truth, which asserts that there is no such thing as an absolute truth, but merely different perspectives that one can adopt. We could think of truth as a sculpture, where there is no single ``right" perspective to look at it. To properly appreciate the sculpture, we must walk around it, looking at it from as many different perspectives as possible. Similarly, Nietzsche insists that we should not get caught up in dogmatism, but rather look at the truth from as many perspectives as possible. 

\end{description}
Of course, Tradition denies the validity of Naïve Realism, the Correspondence Theory and Perspectivism, for a few reasons.

\begin{enumerate}
\item Against naïve realism – the objection is that appearances are deceptive (Maya) and reality is best understood as the source of the appearances. This source is beyond all appearances. 
\item Against the correspondence theory – this theory assumes that all truth can be known/expressed by the rational mind. However, Tradition has always insisted that there is a knowing (gnosis, jnana) higher and more certain than the rational mind. This is clear from the Eastern Traditions. However, it used to be true even in the West. The Medieval Scholastics distinguished between the intellect and the rational mind (although they are mere synonyms today). Or they insisted that faith (which again meant something different from what it means today) is a higher truth than the rational mind. The intellect is associated with the heart — on certain French cathedrals, there are statues of headless saints holding their heads in front of their hearts (the legend of St Denis) — this symbolizes the subservience of the rational mind (head) to the intellect (heart). 
\item Against perspectivism – that's precisely the point: ultimate truth cannot be known from the phenomenal world, no matter how extensive the experience (i.e., number of perspectives). One is left only with appearances, or mere opinion. Truth can only be know by transcending the phenomenal world. This is brought out most clearly by Nagarjuna in Madhamyka Buddhism (see Murti: \emph{The Central Philosophy of Buddhism}). 
\end{enumerate}
The Traditional teaching is that the phenomenal world of appearances is always changing, and so cannot be the ultimately Real. From observing the arising in consciousness of the phenomenal world, one discerns the one constant is consciousness itself. In that sense it is unchanging, since it is present in every act of consciousness, yet is not itself a part of the phenomenal world. It is also without quality – i.e., it has no color, weight, taste, etc, but is rather the source of all qualities. In that sense it is the most real – and this is the common understanding of the real as the (hidden) underlying source of what only appears to be going on.


\hfill

Originally, class notes for a seminar delivered in 2004.



\flrightit{Posted on 2011-02-23 by Cologero }
