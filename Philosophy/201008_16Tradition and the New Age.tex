\section{Tradition and the New Age}

There are two competing spiritual attitudes, and often they are confused because they seem to deal with the same subject
matters: metaphysics, spirituality, and so on. Yet there is a fundamental dichotomy, so divisive, in fact, that mutual
conversation is barely possible. The New Ager is comprehensible from the Traditional viewpoint, but the New Agers has
no understanding of the Traditional worldview, and can only regard it with contempt.

\paragraph{Tradition}
Tradition regards man as he is as the result of a fall from a perfect, primordial state. From a Golden Age to the Kali
Yuga, there has been a continual degeneration in both man and society. This degeneration affects man's
intellect, will, and moral sense. From an ordered, hierarchical, and differentiated world, there is a decline into
chaos and egalitarianism. The way out is to achieve a reintegration of those chaotic elements. 

\textbf{Vertical Orientation}: The Traditional worldview is \emph{transcendence}. Man's task is to transcend
the human state and reach deeper states of being. This process is \emph{theosis}, or the God-man (NOT the deification
of man).

\paragraph{New Age}
The New Age observes the same changes as Tradition, but its judgment is it very opposite. Rather than seeing the current
world as the result of a Fall, it sees it as the outcome of a process of evolution, from a primitive state to a more
enlightened state. Evolution is given a moral sense: what has evolved is “good” and what it replaces is “evil”. All the
symptoms of degeneration that Tradition decries, the New Age, instead, embraces.

\textbf{Horizontal Orientation}: The New Age worldview is \emph{immanence}. God is expressed through the human
situation, so there is nothing to transcend. This is the deification of humanity.

\flright{\itshape Posted on 2010-08-16 by Cologero}

\begin{center}* * *\end{center}

\begin{footnotesize}\begin{sffamily}

\texttt{VisionsOfGlory14 on 2010-08-18 at 05:45 said: }

Why does the New-Ager bother with the study of old religions if he believes man has ‘evolved’
past them? Is there any deeper reason than to claim them for his own and abuse them for his own ends?


\hfill

\texttt{Cologero on 2010-08-18 at 23:51 said: }

The consistent new-ager does indeed believe he has “evolved”. Yet, there is still a great deal of prestige associated
with traditional religious forms, so he adopts and re-interprets them beyond recognition.

Another factor is that, most of those interested in such topics are in fact a mixture of both tradition and new age.
This is because Tradition has been mostly lost in the West, so it is not clear how to be consistently traditional. I am
reminded of a French bishop who recently claimed that the motto of the French Revolution “Liberty, Equality,
Fraternity” is equally Christian. Of course, at that time both sides knew perfectly well it expressed an anti-Christian
sentiment. Even Evola conceded that pre-Revolution Europe retained many Traditional elements.

Even odder are the anti-Christian neo-pagans who accuse Christianity of inaugurating the era of egalitartianism and
universal brotherhood, when the case was just the opposite.


\hfill

\texttt{Ernest on 2010-08-19 at 13:22 said: }

Have any of the Gornahoor people encountered the ‘Integral’ movement centred around Ken Wilber?
Some people have called it the ‘New New Age’. 

If you have encountered it what do you think about it?

I had always been ‘spiritual’ to some extent, but my spirituality has rapidly intensified since
my mid-teens (not too long ago). I also became consciously anti-modern at this time and I did see ‘history
as a fall’. Not having any real mentors local to me, I first discovered the Tao Te Ching, which was my main
spiritual reference, but then later the Wilberian Integral movement. I read a few of Wilber's
very-similar-to-each-other books and, while I was never a true-believer I did accept many of the ideas even to the
point of converting to a more linear ‘progressive’ sense of history. This change was aided by
the discovery of A.N. Whitehead while studying philosophy at University.

Around a couple of years ago I finally rejected Wilberian Integral for several reasons. Among others , I could not share
their enthusiasm for capitalism and, like other New-Age groups there is far too much nauseating giddy feel-good
spirituality in that ‘movement’.

A short while after this rejection I came across Alain de Benoist (who would possibly be one of those anti-Christian
neo-pagans that Cologero mentioned) and his book ‘On Being a Pagan’. I had also studied
paganism in my teens (without ever claiming to be one) and had become somewhat pre-Christian myself at that time. I
felt some affinity for the ideas in the book but could not agree with his advocacy of pantheism. 

That book has several quotations of Evola and from there I eventually read Revolt. I immediately felt a far greater
affinity for Evola's views than I had with de Benoist (though I still admire him), or Wilber or the other
religious groups I have explored and experimented with along the way. I have since read most of his books that have
been translated into English and have begun to read Guenon, beginning with Crisis of the Modern World.

I have been reading with great interest the posts on Gornahoor regarding Tradition and Chrisitianity.

I noticed that there is a quotation of Whitehead's on ‘The Hyperborean Page’; what
are your opinions on Whitehead?

\hfill

\texttt{Will on 2010-08-19 at 21:56 said: }

As for myself, I have no opinion of Ken Wilber, having never read him. Ditto for Whitehead. I relate to what you say
about the Tao Te Ching, as that was also the first book that opened up the world of Tradition for me. It was like a
breath of fresh air after too much of the wrong western philosophers.

Thanks for your interest. Keep searching and studying!

\hfill

\texttt{Cologero on 2010-08-21 at 18:01 said: }

The early Wilber is useful, if only for the encyclopedic knowledge he displays; he has done his homework. However, his
later attempts at a theory of “everything” is unconvincing. There is an intricate schemata (the static) that
doesn't quite fit into the evolutionary scheme (dynamic). His system does not offer any explanation about
why it should be so.

His colour scheme of moral development is really self-serving, and is unrelated to spiritual development. Most people
will see themselves as highly advanced in their moral outlook, without having achieved anything spiritual, to speak of.
Merely holding an opinion about morality is not in the least an indication of high spirituality.

\end{sffamily}\end{footnotesize}
