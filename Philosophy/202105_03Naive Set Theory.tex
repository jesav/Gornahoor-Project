\section{Naive Set Theory}

California has revamped its math education program by eliminating accelerated learning programs. The assumption is that anyone can learn math. A good way to test this is Naïve Set Theory. This requires no math knowledge: just the ability to count and to reason logically. A ``set" is defined by the everyday understanding of the term and does not rely on abstruse concepts in mathematical logic. Hence, it can be used, in principle, for self study.

Because of its informal method, this class is typically offered by the philosophy, rather than the mathematics, department at universities. When I took it, about two dozen students showed up on the fist day, possibly, I presume, because it was labeled Introduction to Logic.

Students kept dropping out until there were just three of use left. I finished with a perfect score on all the tests, because such knowledge is certain and never forgotten. The other two students were women, which demonstrates that women are capable to think logically, even if they sometimes choose not to. But those who could not think logically – or chose not to – included pre-law, political science, and journalist majors among those I recognized. That explains much.

Mathematics is intermediate between metaphysics and physics, yet is neither. There have been several mathematicians who have been great metaphysicians, and physicists who have been great mathematicians, but no one, as far as I can recall, has been all three.

The first thing to learn is that logic does not operate in a vacuum. It requires self-evident truths called ``axioms" to get started. To deny an axiom makes the project impossible or results in a contradiction. This is good practice in learning to think in terms of principles rather than in terms of individual things. Understanding the set means understanding all the individuals belonging to the set.

A tricky point is to show that a set actually exists, so that sets are not merely thought experiments. Once that has been accomplished, useful manipulations of sets can begin. I don't want to summarize them here.

\paragraph{Infinite Sets}
The most fascination result has to do with the notion of infinity, since it can be more deeply understood with the tools learned. Everyone knows the distinction between ordinal and cardinal numbers, even if you hadn't thought of it before. Ordinal numbers are first, second, third, etc., which show order; cardinal numbers are 1, 2, 3, etc., which are used for counting. It is easy to see that the natural numbers have infinite cardinality; this is called ``countably infinite", since they can in principle be counted However, what is not so obvious is that there are even higher degrees of infinity, called appropriately, uncountably infinite. There is then a dizzying array of infinities but not ``highest" infinity. That infinity is ``outside the system" altogether.

A surprising result is that every set can be ordered. Clearly, countably infinite sets can be ordered; there is always a lowest natural number, be it 1, 23, or whatever. That is not so for the real numbers. Ask yourself what is the first real number after 1.0 and you can convince yourself. Nevertheless, it can be proved that there is an unknown order to the real numbers, but its exact nature is unknown. Albert Einstein famously said, ``God does not play dice," when confronted by the random patterns of quantum experiments. He meant that there must be a deeper theory that explains those seemingly random patterns. That assumption is that the deeper order must be computable, but that is not necessarily true. Rather, God – the True Infinity above all the infinities – knows the hidden order. We counter Einstein with this thought:

\begin{quotex}
God works in mysterious ways.

\end{quotex}
\paragraph{Other Logics}
Although formal logic of the type described here is necessary, it is not sufficient to explain the world. Set theory, is nominalist rather than realist. That is, there can be a set of unrelated elements like cabbages and kings. A realist set would only contain individual things that share the same divine idea.

Moreover, our human and spiritual lives depend on an even higher form of logic. Organic logic and Moral logic will be discussed in a future post.

\paragraph{Reference}
\emph{Naïve Set Theory} by \textbf{Paul Halmos}

\flrightit{Posted on 2021-05-23 by Cologero }

\begin{center}* * *\end{center}

\begin{footnotesize}\begin{sffamily}



\texttt{Hugo Smith on 2021-05-27 at 19:21 said: }

Could a realist universal set defy Russel's Paradox and contain itself?


\hfill

\texttt{Cologero on 2021-05-27 at 19:58 said: }

No, because a set is always in thought and is never in being. So there is no set that you can see, feel, hear, or touch.


\hfill

\texttt{Hugo Smith on 2021-05-28 at 15:58 said: }

Then are you saying there is no such thing as a realist set? Aren't some categories essential?


\hfill

\texttt{Sensitive Ears on 2021-06-06 at 17:40 said: }

``There is always a numerous host of the stupid and the weak, and in a republican constitution it is easy for them to suppress and exclude the men of ability, so that they may not be outflanked by them. They are fifty to one; and here all have equal rights at the start."

``The ancient Masters

who understood the way of the Tao,

did not educate people, but made them forget."

The greatest generations lived without advanced mathematics. The question is, when people are liberated from accelerated math, what will fill that empty space in their minds? Resentment? Boredom? Video games? Critical Race Theory? Piety?


\hfill

\texttt{Cologero on 2021-06-07 at 22:46 said: }

Hnh? Of course, the greatest generations knew advanced math. The Egyptians knew trigonometry; that is how the recovered land boundaries after Nile floods receded. Euclid knew geometry, not to mention Pythagoras. Eratosthenes knew prime numbers. Ancient India was rich in mathematics, inventing decimal numbers among other accomplishments. Arabs knew about algebra. Mathematics form two of the seven liberal arts. I could go on.

What people need ``liberation" from maths? It was always for the few, and maths are so important that Plato required knowledge before being admitted to his School.

Of course, numbers, as with Pythagoras had mystical connotations. What I want to recover is that mystical dimension, to rescue maths from mechanical manipulation of symbols in order to reach the the hidden metaphysical meaning behind maths. Maths is the dividing line between metaphysics and manifestation, so its importance cannot be overstated.

Perhaps if you filled your mind with mathematics, you would write more coherent comments — not to mention factually correct comments.


\end{sffamily}\end{footnotesize}
