\section{NASA, the Moon, Genesis and the Tao}

\begin{quotex}
\textbf{In the beginning God created heaven and earth}. \flright{\textsc{Genesis 1:1}}

The Tao that can be stated, is not the eternal Tao;

The name that can be named, is not the eternal name.

\textbf{The nameless is the origin of heaven and earth}. \flright{\textsc{Tao Te Ching}}

\end{quotex}
In the esoteric traditions, Heaven and Earth have been interpreted as Essence (or Form) and Matter. \textbf{Meister Eckhart} acknowledges this understanding:

\begin{quotex}
Matter and Form are not two kinds of existent entities, but two principles of created beings. That is the meaning of the words: In the beginning God created heaven and earth — viz., form and matter, two principles of things. 

\end{quotex}
In the interview linked below, Wolfgang Smith seems to be endorsing geocentrism, not as one possible frame of reference, but in an absolute sense: the earth does not move while the universe revolves around it every 24 hours. Yet he hedges by acknowledging the truth of heliocentrism in the sense that the Sun is the symbol of God. So geocentrism would be the exoteric teaching, i.e., as it appears to the senses, while heliocentrism is the truer, esoteric teaching.

So when the interviewer quotes Genesis 1:1 to ``prove" that the physical earth was the first object created, Dr. Smith rejoins with the Eckhart quote.

However, Dr. Smith apparently has committed to the unmoving earth theory. He asserts that the Michelson–Morley experiment shows that the Earth stood still, which is not necessarily so. That requires that the \textbf{Theory of Relativity} be abandoned in toto, presumably because classical mechanics and Maxwell's equations are adequate for physics. It would need to be demonstrated that physics can account for a world that is created fully formed, unmoved and unchanging. Only God is unmoved and only the eternal is unchanging.

We than make the following proposal to NASA when it resumes voyages to the moon:

\begin{quotex}
\textbf{Perform the Michelson–Morley experiment on the Moon}. 

\end{quotex}
That would settle the question. If the result is the same, then the moon cannot also be still. If the result shows a change in the velocity of light, then the earth is still.

\paragraph{Postscript}
In traditional cosmology, the position of Earth is the furthest from Heaven and the closest to Hell. If that is what geocentrism means, then it must be true. 

\url{https://www.youtube.com/watch?v=71i22w5G9KE}



\flrightit{Posted on 2020-06-27 by Losang Shenphen }

\begin{center}* * *\end{center}

\begin{footnotesize}\begin{sffamily}



\texttt{Dennis on 2020-06-29 at 13:17 said: }

As it happens, just last week I saw both ``The Principle" and the new film based on Smith's life and work, ``The End of Quantum Reality," and also read Smith's companion book, ``Physics and Vertical Causation."

Frankly, I'm not too sure what to think of all of it, as I don't have adequate physics background to asses the details. It seems convincing from a certain philosophical standpoint, especially his effort to separate physics as such from the philosophy or theology of physics (though I'm not quite clear on his distinction between corporeal and physical, which I had always seen as basically synonymous). He does indeed, in the book anyway (the latter film is rather schematic, and without the book or knowledge of Smith's previous work, would not be very convincing or clear) abandon Einsteinian Relativity in toto, as well as Cartesian ``bifurcation" between res extensae and res cogitantes. 

Several of the people interviewed in The Principle later denounced the film and claimed they were deceived or their interviews distorted in some way, though they seemed to me to be given plenty of space in the film to make their views clear. I think they are engaging in some post-facto CYA to avoid the risk of being tarnished in the academy and the eyes of their peers for having appeared at all). Smith also seems to rely heavily of William Dembski's controversial work. I can't tell if the criticisms some have of Dembski's work is purely mathematical, or whether the scientific establishment, committed as it is to materialism and secularism is just circling the wagons and doing everything to discredit any ideas that open the door for belief in God. 

The stuff about the CMB lining up with the ecliptic was quite interesting, and you could see a couple of the interviewees struggling in The Principle to avoid drawing the conclusions from it that they felt the science otherwise would compel them to (because it contradicts their prior philosophical commitments regarding the place of earth in the cosmos).

I tend to see geocentrism as a metaphysical and spiritual truth about Man's place in the cosmos, one that doesn't depend on the earth being literally stationary or placed at the physical center (How could one ever determine the center anyway unless we could define the actual boundaries of the universe?). An in any case, isn't our view of the cosmos necessarily geocentric in that we can only view the universe from where we are and have no vantage point outside the universe from which to look into it (from a God's Eye view as it were)?


\hfill

\texttt{Cologero on 2020-06-29 at 22:45 said: }

@Dennis, there are too many layers to this. I believe the goal is to prove that the Bible requires a stationary earth. We can't accept that, since there are multiple layers of interpretation. Dante in the Paradiso accepts the same model of the universe: Paradise is depicted as a series of concentric spheres surrounding the Earth, consisting of the Moon, Mercury, Venus, the Sun, Mars, Jupiter, Saturn, the Fixed Stars, the Primum Mobile and finally, the Empyrean. (from Wikipedia)

But Dante understood that as the soul's spiritual ascent to God, not as a physical journey. Otherwise, you could reach Heaven just by building a nice rocket ship.

It is one thing to assume that the earth does not move, but to assume that the earth does not even rotate on its axis leads to some impossibilities. As I understand it, the claim is that geocentrism is consistent with classical physics (mechanics and Maxwell's equations), as long as the Theories of Relativity are rejected. But Neptune — never mind the stars — would have to move faster than the speed of light to make it around the earth in 24 hours. There is no physics to explain that motion; certainly not Newton's gravity equation.

We will accept whichever model accords with observation and physics; we don't ``root" for any particular physical model to be true.

However, the description of the soul's assent to God through the spheres is much more important. That is non-negotiable.


\hfill

\texttt{Boreas on 2020-07-03 at 16:07 said: }

The last time I read Smith's views he advocated ``a relativistic geo-centrism" and hadn't thrown the theory of relativity straight out from the window. That was something I could accept since it accords with the traditional view and is spiritually and physically tenable. It seems that his views have hardened on the matter, and have taken an absolute stance. That is something I reject.


\hfill

\texttt{Cologero on 2020-07-05 at 20:42 said: }

I don't know what to make of it. The fellows he works with are quite sincere, very creative, and have put a lot of effort into their videos and web site. But you can't have relativity and geocentrism together, so I understand the rejection. Unfortunately, since geocentrism a non-inertial reference frame, Newton's laws of force and gravity don't even apply. So you end up with a model that has no physical explanation for the revolution of the universe around earth. I haven't seen the videos, so perhaps they contain the explanation.


\hfill

\texttt{Boreas on 2020-07-07 at 05:07 said: }

The way he explained the ``relativistic geocentrism" in his earlier works was that from the perspective of the theory of relativity you can logically assume that either the earth revolves around the sun or the opposite, and these are the esoteric and exoteric applications.


\hfill

\texttt{Tom Hart on 2020-07-17 at 16:29 said: }

Related topic: NASA just announced a new star sign, the serpent-bearer (represented by a Hermetic caduceus). This sign, the 13th, is rocking some kundalini energy, but my intuition is that it's an attempt to corrupt astrology, since, in a very corrupted form, astrology one of the few Traditional practices that has survived into modernity. I recall reading that the 13th seat at a table, per the legend of King Arthur, should be kept empty—i.e. by making something that should be occult visible, NASA has corrupted the symbolism. However, the visibility of the snake is also, probably, a sign that the age is turning—the old order is dying in snake-like chaos, as current political events demonstrate. In attempting to destroy the sacred order, NASA becomes the unwitting handmaiden of purification.


\end{sffamily}\end{footnotesize}
