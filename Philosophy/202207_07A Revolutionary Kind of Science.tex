\section{A Revolutionary Kind of Science}
\label{sec:202207_07A Revolutionary Kind of Science}

\textbf{Adam Kisby}\footnote{\url{https://lifeboat.com/ex/bios.adam.kisby}} is a philosopher and a member of the Omega Society\footnote{\url{http://www.theomegasociety.com/}}, which only accepts one in a million on a test of
general intelligence. One of his primary interests is anomalistics, which is the study of unusual and unexplained
phenomena. What follows is a brief review of his book \textit{A Revolutionary Kind of Science}\footnote{\url{https://www.blurb.com/b/4168343-a-revolutionary-kind-of-science}} in which Kisby critiques the
conventional understanding of the scientific method. He thinks that this method leaves out unusual, uncommon, or unique
phenomena to the detriment of science. He then offers alternative scientific principles in which anomalies can be
studied in a reasonable way.

\paragraph{Anomalistic Method}
“Normal Science” is the common use of science, such as the claims that “science shows …”. It remains within a generally
accepted paradigm, so it is often little more than establishing more accurate measurements.

Extraordinary science, on the other hand, seeks to explain the anomalies that contradict a prevailing paradigm, the sort
exemplified by Copernicus, Galileo, Newton, Darwin, and Einstein, for example.

However, that sort of science has become moribund, mostly because anomalistics — the study of anomalous data
— has failed to affect prevailing paradigms. Charles Fort, in a series of books, documented many cases of
unusual events; thus, he is considered to be the father of anomalistics. Since then, several other researchers have
contributed many more examples. Marcello Truzzi defined four regulatory principles for a science of anomalistics:

\begin{itemize}
\item \textbf{Principle of Testability}: data should be verifiable (Bacon) or falsifiable (Popper) 
\item \textbf{Principle of Parsimony}: explanation should be as simple as possible (William of Ockham) 
\item \textbf{Principle of Burden of Proof}: data should be doubted in the absence of proof (Descartes) 
\item \textbf{Principle of Proportionality of Evidence}: evidence should be commensurate with the degree of
extraordinariness (Hume) 
\end{itemize}
Kisby goes on to analyze each of those principles in depth from philosophic, scientific, and anomalistics perspectives.
Here, we can only highlight some of the main points.

\paragraph{Testability}
Testability requires that a theory should be verifiable or falsifiable. These criteria may seem obvious, but Kisby shows
that there is more ambiguity in them than is obvious at first sight.

Verifiability excludes phenomena that are unverifiable, unmeasurable, or undetectable. Examples are the anecdotal
stories of reincarnation or the efficacy of very dilute homeopathic medicines.

Falsifiability excludes unfalsifiable phenomena such as unknown hominids and extraterrestrial intelligences.

\paragraph{Parsimony}
The human mind prefers a comprehensible map of theory to an incomprehensible territory of data. The simpler theory is
better all other factors being equal. It does not mean that all explanations are simple. In practice it is not always
possible to eliminate complex theories. Examples are Chaos Theory and Mandelbrot sets; they explain everyday phenomena,
yet are far from simple.

Ultimately, an explanation must “save the appearances”, whether simple or complex. Parsimony excludes phenomena that are
irreducibly complex, chaotic, or nonlinear.

\paragraph{Burden of Proof}
\begin{quotex}
I ought to reject as absolutely false all opinions in regard to which I could suppose the least ground for doubt, in
order to ascertain whether after that, there remained aught in my belief that was wholly indubitable. \flright{\textsc{Rene
Descartes}}

\end{quotex}
In practice, this means that the claimant has the burden of proof. Kisby points out several reasons to doubt skepticism.
For example, it took decades to convince all scientists of the extraterrestrial origins of meteorites.

The proper attitude should be \emph{zetetic}, which entails a suspension of judgment about facts. In popular language,
it means that it is OK not to be sure. Ultimately, prudent judgment is required which dogmatic skepticism dispenses
with.

This principle excludes the possibility of cold fusion or the discovery of hidden causes behind historical processes.

\paragraph{Proportionality of Evidence}
\begin{quotex}
A wise man proportions his belief to the evidence. \flright{\textsc{David Hume}}

\end{quotex}
This principle states that extraordinary claims —i.e., those that disagree with ordinary experience
— require data of greater quantity or higher quality. In practice, however, this is not always useful. For
example, rare events such as encounters with ghosts, UFOs, and yetis, may nevertheless correspond to real phenomena.
The number of times an event occurs has no bearing on its truth.

This principle excludes phenomena that are uncommon, unusual, or unique, such as the existence of a fifth fundamental
force of nature and psychokinesis.

\paragraph{A New Scientific Method}
Kisby concludes with a new scientific method. The four principles, while ostensibly maintaining scientific rigor, also
operate as epistemic filters. These filters are inconsistent with reason, disagree with the findings of scientific
research, and are incompatible with selected anomalies. They don't take into account:

\begin{itemize}
\item The counter-intuitive distinction between simplicity and complexity 
\item The profound effects of expectations on perception 
\item The supernatural beliefs of the general population 
\item The confusion about double blind experiments 
\end{itemize}
Otherwise, there is the risk of omitting important data:

\begin{quotex}
There are bona fide phenomena that are unverifiable, unmeasurable, undetectable, unfalsifiable, irreducibly complex,
chaotic, nonlinear, unexpected, unwanted, unexplainable, uncommon, unusual, and unique. 

\end{quotex}
Kisby claims that such phenomena have implications for health and for a better self-understanding of humanity, among
several other benefits. He then enumerates four principles that modify Truzzi's. Very briefly, these are:

\begin{itemize}
\item \textbf{The Principle of Mappability}. The data needs to be formally expressible and logically relatable to other
data. It allows for data from different states of consciousness. 
\item \textbf{The Principle of Plenitude}. Since phenomena may be complex, then so may be the explanations. 
\item \textbf{The Principle of Suspended Judgment}. Since the level of belief or unbelief has effects on the perception
of data, then the level of certainty of data must be on a scale. 
\item \textbf{The Principle of Truly Proportional Evidence}. Phenomena that occur less frequently will produce fewer
data, and irregular events will not be predictable. 
\end{itemize}
In summary, we can use these principles for a necessary extension of science. Some things happen once, like the
beginning of the universe, the origin of life on earth, and rational consciousness. Who can doubt it, even if there are
no fully convincing explanations for any of them.

Other phenomena may occur over the course of generations, too long to be observed in the lifetime of a scientific.
Examples are the birth, death, and decay of cultures.

\flright{\itshape  Posted on 2022-07-07 by Cologero}

\begin{center}* * *\end{center}

\begin{footnotesize}\begin{sffamily}
\texttt{Dimitri on 2022-07-10 at 14:32 said: }

Cologero, are you aware of Library Genesis (libgen.is) or ZLibrary (b-ok.cc) by any chance? If not you could certainly
get a lot of use out of them.

\hfill

\texttt{Cologero on 2022-07-10 at 15:13 said: }

It's more likely that those authors could get a lot of use from reading Gornahoor. Why don't you
point that out to them?

\hfill

\texttt{Dimitri on 2022-07-10 at 18:05 said: }

These are file sharing platforms where you can upload and download books in digital form like pdf and epub for free;
they have books by all sorts of authors, even rare works or things written by the men cited here on Gornahoor from
Guenon to Tomberg to Solovyov and so on. I thought you might find it useful if you hadn't heard of it
before.

\hfill

\end{sffamily}\end{footnotesize}