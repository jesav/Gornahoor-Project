\section{Esoteric Stoicism}

\begin{quotex}
Do not go outside, go back into yourself: the truth dwells in the inner man. \flright{\textsc{St. Augustine}}

\end{quotex}
Three factors go into creative misreading.

\begin{enumerate}
\item Place it in a larger context 
\item Draw out its logical conclusions 
\item Extract a deeper meaning 
\end{enumerate}
In this post, we will see how the key concepts of Stoicism were reworked. Stoicism was an exoterism because it held that
the representations in the mind were of objects in the material world. However, when the entire psychic contents become
the data of analysis, these same concepts lead to an esoterism. We can see that the modern mind is more inclined to the
Stoic understanding of the inner life, which reveals a reversion to paganism. The following phenomenological analysis
of Stoicism and the Church Fathers will bring to light the workings of the mind.

Stoicism in the ancient world provided an appealing worldview based on living a rational and ethical life. Its weakness
lay in its materialism and empiricism. The Stoics held that everything is corporeal so that immaterial or spiritual
reality did not exist. The soul is constituted of finer matter while the body of God is the cosmos. Nevertheless, the
cosmos is animated by Logos which had the attributes of thought, consciousness, and providence. In this pantheistic
system, everything is a part of God. Hence, God determines all that happens so that there is no distinction between
Providence and Destiny. All is fated.

The concept of the Logos, therefore, was known before the Gospel of John, including its identification with God. With
their deeper understanding of the Logos, the Church Fathers reworked certain key Stoic concepts into a larger
framework. They did this by interiorizing the Stoic’s materialism.  In particular, the following seven
concepts will illustrate how that process worked:

\begin{enumerate}
\item The governing principle of the human soul 
\item Preconceptions 
\item Representations or Phantasies 
\item Assent 
\item Relation 
\item Ataraxy or Tranquility 
\item Apatheia or passionlessness 
\end{enumerate}
\paragraph{The Human Soul}
For the Stoics, the idea of lower centers in the soul made no sense. If God is Logos or Reason, and the soul is part of
God, there cannot be any irrational part of the soul. Rather they recognized a governing principle. Concepts reside in
it and it is the faculty which exercises judgment and applies concepts to particular situations.

This is clearly the intellectual center of the soul, without making the distinction between intuitive and discursive
reason. When this principle guides a person’s life, he is happy and free from passions. Whereas for the
Stoic this highest state is natural, for the Fathers it is supernatural. The governing principle is the activity of
inner attention, the power of discrimination between good and evil, and even sacred contemplation. Obviously, that is
the activity of the higher intellectual center or \emph{nous}.

Since the Fathers do not deny the irrational appetitive and emotional parts of the soul, they need to be transcended.

\paragraph{Preconceptions}
The Stoic concept of preconception is that they are innate principles common to all men. As such, they are not
contradictory, but become contradictory when applied to concrete situations in different ways. The purpose of education
is to learn how to apply these preconceptions to specific instances in conformity to nature. Epictetus provides some
examples: knowing good from evil, beautiful from ugly, knowing what one ought to do and not to do. Other things, such
as mathematics, are not innate and need to be acquired.

The Stoics and Fathers agreed that there is innate moral knowledge. However, the Christian understanding of conscience
goes beyond that. Not only is it a moral guide, it is also the impartial moral judge. Without that judge, the Stoic
follows his preconceptions.

However, for the Fathers, preconceptions also include negative elements: prepossession, prejudice, a predisposition to
sin. These need to be opposed and eradicated. These days we hear that so and so has a “good heart”. This is more in
tune with the Stoic ideal. However, preconceptions can prevent the right decision. Prejudice or the desire to please
will cloud our judgment. Negative preconceptions need to be expunged before they become passions.

\paragraph{Representations}
A phantasy in Stoicism is an impression made in the mind by an external object through the senses. As in the past, we
will use the term “representation” instead of “phantasy” due to the unfortunate connotations of the latter term. The
problem for the Stoics was to distinguish illusory representations from true ones, and they provided some criteria to
do that. In that, they agreed with Socrates that the Stoic should not accept a representation as true without
subjecting it to critical examination. Epictetus pointed out the need for inner attention to discriminate between
acceptable and unacceptable representations. In this regard, the teaching is sound.

The Fathers then brought this idea into a wider context. Our inner attention and discrimination should not be limited
just to the impressions of the external world, but should include all the contents of the soul: thoughts, concepts,
memories, dispositions, and so on.

The Fathers realized that much of our inner life is demonic. The Stoic must deny that because in pantheism, everything
is God, so there cannot be anything demonic. However, once you get past the Hollywood style depiction of demons, then
this realization will make sense. Of course, the modern mind regards all inner soul contents as “natural”, not unlike
the Stoics, so they are compelled by that logic to accept demonic influences as fully natural, with just the echo of
moral preconceptions to restrain them.

Note how this coincides with our earlier discussions. First, representations are accepted uncritically as true or good:
this is how it is for the bulk of people today. The philosopher, in the next stage, learns to question or judge the
representations. In the third stage, the representations are seen for what they are.

\paragraph{Assent}
Once representations are recognized as true or false, good or evil, then one must either assent to them or reject them.
Otherwise, says Epictetus, “if we fail to do this, the impression will take possession of us and go off with us
wherever it will.” There are many examples in the news illustrating how an initial impression, if left on its own, will
lead to unnatural or even insane acts. The modern mind has been losing that faculty of discrimination.

The Fathers used this Stoic insight in their understanding of temptation, as shown in this analysis:

\begin{itemize}
\item First a \emph{suggestion} comes into our mind via a representation or thought 
\item Then there is a mingling of the suggestion with our own thoughts 
\item This is followed either by opposition or assent to the suggestion 
\item If there is assent, then sin is the result 
\item Assent then leads to enslavement, so that the same temptation is repeated with the same result 
\item The refusal to assent results in spiritual combat which will lead to victory or defeat 
\end{itemize}

\paragraph{Relation}
An important philosophical category of being for the Stoics is relationship: to your body, to God, to those who live
with you, and so on. The Fathers also adopted this notion in regard to our relation to God, our fellow men, material
things, and secular values.

There is a binding character of psychical relations, which is on a scale. The human soul establishes relations between
itself and various things such as money, possessions, glory, and people. Relations involve two things:

\begin{enumerate}
\item An awareness of the existence of things or persons. 
\item An emotional or conative attitude towards them. 
\end{enumerate}
This attitude binds the soul to them in stages:

\begin{enumerate}
\item First, there is an interest in a thing, a “feeling towards” it. 
\item When the interest gets stronger, an emotional response, called a “passion”, arises. 
\item Finally, the relation may become one of bondage or enslavement. E.g.: 

\begin{itemize}[nosep]
\item \textsc{Avarice}: bondage to money 
\item \textsc{Greed}: bondage to material possessions 
\item \textsc{Ambition}: bondage to human glory 
\item \textsc{Lust}: bondage to sex 
\end{itemize}
\end{enumerate}
To be free, a person needs to overcome such bondages. Obviously, it is easier to do this at stage (1) rather than (3),
but it takes inner discipline to recognize something that subtle.

\paragraph{Ataraxy}
\textbf{Ataraxy} is the state of the soul which is peaceful, undisturbed by external events, thoughts, phantasies,
desires, or emotions. For the Stoic, following the governing principle and living according to reason will lead the
wise man to this state.

The Fathers recognized a similar state without, however, considering it the highest state. As such, ataraxy is a natural
state. This state of inner quiet or tranquility is a preparatory stage for union with God, or \emph{theosis}
— or the Supreme Identity as some Traditional writers have phrased it.

The Stoics lived in cities, full of noise, turmoil, and distractions; they tried to reach the state of tranquility in
that environment. The Fathers, however, held that outer tranquility was necessary for inner tranquility. Hence, they
retired to the desert or the mountains or any quiet, solitary place.

In our time, however, we have to hark back to the Stoics and achieve Ataraxy in the midst of our everyday life. It is
not outer things that ultimately perturb us, but rather our own cares, passions, temptations, etc.

\paragraph{Apatheia}
Alongside ataraxy, \textbf{apatheia}, or passionlessness, constitute the highest Stoic ideal. The modern world has lost
the understanding of that state, since “apathy” has taken on a negative connotation, not indicative of the highest
state. A fortiori, the modern mind is instead \emph{impressed} by passion or emotion. The intensity of feeling is the
measure of truth. I don’t need to provide examples, since you can find them everywhere.

The Stoics unfortunately overdid it, regarding all emotions as sinful, irrational, or unnatural. They identified
pleasure, grief, fear, and desire as the four chief emotions. This is an unsatisfactory position since it will lead to
contradictions.

The Fathers, instead, equated “passions” just to negative emotions. Hence, there are some distinctions to be made about
what is properly a negative emotions:

\begin{enumerate}
\item \textbf{Those emotions which are bad in themselves}: e.g., conceit, gluttony, lust, vanity, pride, greed, malice 
\item \textbf{Those emotions which are bad only when contrary to nature}: e.g., anger, hatred, sorrow, fear. When these
emotions are in conformity to our nature, they are not “negative emotions”. So, for example, misplaced anger is
negative, but anger directed against an injustice is not. 
\item \textbf{Pleasures and desires}: Once again, these depend on the object. For example, a normal desire for food,
rest, or sex is not negative. Negative pleasures and desires are of two types: 

\begin{enumerate}
\item \textbf{Excessive}: For example, when the desire for food becomes gluttony 
\item \textbf{Disordered}: For example, when the desire for food becomes coprophagia 
\end{enumerate}
\end{enumerate}
Sorrow for sins, fear of God, fear of hell, are not passions, or negative emotions, in this sense. Rather, the impel us
to reject temptations and regain the health of the soul.

Love becomes the highest goal, even higher than apatheia. Love, in this sense, is more than worldly friendship or family
affection; it is a spiritual love. The overcoming of negative emotions, the state of apatheia, is preparation for this
higher stage.\footnote{This post is adapted from the chapter on Stoicism in \textit{The Hellenic-Christian Philosophical Tradition} by Constantine
Cavarnos. The reader is encouraged to check out that work for references to works by the Stoics, Church Fathers, and in
the \textit{Philokalia}.}

\flright{\itshape Posted on 2015-06-04 by Cologero}

\begin{center}* * *\end{center}

\begin{footnotesize}\begin{sffamily}

\texttt{Tom Blanchard on 2015-06-05 at 12:46 said: }

I was actually just reading yesterday a similar reflection on Stoicism by the Rev. John Toshimichi Imai, a Japanese
Anglican priest who wrote a treatise called “Bushido: In The Past and In the Present” (1906). After emphasising the
Bushido is not a religion or a philosophy (his position being that it is more a product of the “spirit of the Yamato
race”), and has adapted itself over the centuries to the various religions and philosophies that have predominated in
Japan, he compares and contrasts the ethics of Bushido with the ethics of various philosophical and religious systems.
On Stoicism, he writes:

“Again, the sternness of the Stoic, and the self control of the Samurai over his emotions have much seeming likeness,
but in Bushido at least there was no condemning of the emotional spirit. Duty was indeed the highest object of
self-sacrifice, and reason ranked higher than the feelings, but what is called ‘bushi no
nasake’, that is to say, `the humane feelings of a bushi’ were warm and tender.
Thus it is that we do not find a Seneca in our Bushido to condemn tears and sympathy. The hardship of self-denial on
the part of a bushi was to have a heart and to conquer it when duty so required.”

There is a fourth section of the book in which Imai reflects on the present features of Bushido and its bearing toward
Christianity, which I have not yet obtained and read (the most available copy of this book is scattered across four
different issues of the magazine “Kendo World”). Evola was an admirer of the Japanese race and considered their
tradition and heritage to be quintessentially Solar, so Imai’s reflections as an Anglican priest seeking to
integrate the Japanese race-soul with the Christian tradition may be of interest to some.

\end{sffamily}\end{footnotesize}