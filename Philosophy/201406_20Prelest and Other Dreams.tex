\section{Prelest and Other Dreams}

By now, everyone should have figured out the meaning of my dream, which is:

\begin{quotex}
\emph{Sleeping people are full of spiritual delusions. Awake men are more cautious.} 

\end{quotex}
This spiritual delusion is called \emph{prelest}. This cannot be considered a rare condition, but rather it is the universal state of people. It begins with a false thought. Since the nature of error is not to recognize itself as error, a whole worldview is weaved from that thought and all its consequences. Then, one's very identity becomes associated with that worldview, to the extent that any challenge to the worldview is taken as a personal threat. That is why people become so enraged and irrational when discussing or debating worldviews. The alternatives, as in every menace to one's life, are fight or flight.

Yet, it is not death that follows from the destruction of a false worldview, but rather an awakening from a form of sleep. In our recent posts, prelest can be related to rhetoric. Since the worldview born in prelest is necessarily false, it cannot lead to being, but only to a torrent of words. The opposite is self-possession, or said differently, an increase in the level of being.

Self-possession describes facts and phenomena, hence it does not promote one particular worldview over others. Thus, it is more like an invitation to look, not a desire to debate. However, those stuck in rhetoric can only understand it as a worldview to be defeated.

\paragraph{Chopra's Challenge}
Although I am hardly a fan of Dr. \textbf{Deepak Chopra}, and am only vaguely familiar with his thought, I give him credit for offering a million dollar challenge\footnote{\url{https://www.youtube.com/watch?v=Up6GqgBK5Qo}} to atheists, materialists, and skeptics to explain consciousness in terms of science. Specifically, they need to produce a falsifiable, scientifically demonstrable theory of consciousness.

I've since found out that he has often debated the usual cast of characters and in a way, I prefer his approach to those of the protestants who engage in such public debates. I also found out that there are high speaking fees for debaters, up to \$80K for an appearance.

Dr. Chopra is a philosophical idealist, that is, he accepts the primacy of consciousness or, as \textbf{Valentin Tomberg} put it, the subtle rules the dense. Idealism used to be philosophy proper, and was the philosophy of intelligent and educated men throughout time.

As our recent discussions of magical idealism have shown, the World can be explained by idealism, but consciousness cannot be explained by materialism.

\paragraph{Spark of God}
Despite his origins in Hinduism, Dr. Chopra seems to hold new age ideas. This is one of the most common new age beliefs, not just his: He said that he used to be an atheist until he realized that he \emph{was} God. Similarly, men have told me that right to my face. If that is the \emph{prima facie} argument against atheism, atheism is preferable. I do know such a claim is popular with the audiences.

Those who are more modest will say we all have a spark of God. But even that new age idea is derived from gnostic heresies. There is no point to such a vague formula, since whatever it is intended to convey is adequately dealt with through other concepts. Specifically, what of God we have in us is \textbf{consciousness} and \textbf{free will}. Hence to be more Godlike is to be more conscious and to be more free.

To summarize it, however, someone with the ``spark" of God is recognizable from what he wills, since divine wills cannot be in conflict.

\paragraph{Children of God}
Another new age idea is that we are all children of God, as a birthright. I don't know the derivation of that idea, but it is not from the Western religious tradition, and its propagation has led to all sorts of mischief. To be a child of God is a supernatural gift and requires the second birth.

\paragraph{Addiction to Spiritual Experience}
The purpose of meditation is often lost on meditators. Often its defenders will point to it health benefits, such as lowered blood pressure, etc. Those benefits may or may not occur, but the purpose is not material betterment.

Hindus, due to their mechanical conception of karma, consider time spent in meditation as a sort of ransom to burn off accumulated karma. That cannot be true. Rather, the purpose of meditation is to learn how to detach from the contents of consciousness. So during the time allotted for meditation, there may actually be just a few moments of actual meditation going on. Those moments should increase in duration and intensity with practice.

There is also the possibility of addiction to meditation. Many years ago, it was my practice to meditate 30 minutes each morning. At some point, I began to experience periods of intense physical pleasure during these meditations, which are difficult to describe. Of course, this was highly motivating and I looked forward to my daily meditation practice. Over the course of several months, these experiences became rarer and rarer, and eventually stopped altogether.

Looking back, I should have written a book about it and gone on the Oprah Winfrey show. That is because people are impressed with wonderful experiences and want to know how to get them. That is a fundamental aim of new age religions. However, in my case, I knew that the goal is transcending all such experiences, so I persisted despite the resulting dryness. I don't know today if it was a little blessing for me or rather a test of resolve.



\flrightit{Posted on 2014-06-20 by Cologero }

\begin{center}* * *\end{center}

\begin{footnotesize}\begin{sffamily}



\texttt{mohenko on 2014-06-20 at 05:01 said: }

Your posts become ever more apt for my own situation, I follow the crumbs eagerly. Thank you for setting the trail.


\hfill

\texttt{Bill on 2014-06-20 at 10:35 said: }

Your postings are very helpful to me and I find myself constantly referring back to previous posts for assistance, clarification, and inspiration. I believe you had once mentioned that you were going to limit yourself to 1000 postings. If that is correct, has any thought been given to the possibility of publishing a complete set of postings in book form once this blog project is completed?


\hfill

\texttt{Synodius on 2014-06-22 at 17:02 said: }

A version for kindle with a good index would be very helpful…


\hfill

\texttt{Tosti on 2014-06-22 at 18:33 said: }

Prelest is extremely common, as you suggest. Typically the sort of `powers'. siddhis, charisma, which can be seen in various spiritual practices fall into two main categories-those which develop naturally through certain exercises(these may be likened to a psychic muscle if you will), and those which are granted by proper orientation through the Grace of God. The former, which are natural enough to the authentic man, are in principle no different than any other material quality such as a rational mind, a physical prowess, or discernment through wit, and can be as dangerous as those accomplishments due to the error of pride. The latter, bounded by a proper structure and guided by reliable guides(elders), is a route much less dangerous, much less susceptible to the prelest which seems to be a very part of the modern air we breathe.


\end{sffamily}\end{footnotesize}
