\section{New Age and Ancient Wisdom}

There is a loose group of spiritual ideas that have a large influence in Western, particularly American, culture, that go by the name of “New Age” spirituality. Furthermore, they derive their validity from the claim that they are updated versions of Traditional wisdom. While there is certainly some basis for that claim, in actuality they are distortions and simplifications of those teachings. That is what we want to explore in order to illustrate the true meaning of those teachings.

\paragraph{Law of Attraction}
The Law of Attraction is the belief that persons, conditions, things and events come into your life experience because you “attract” them through your mind, either consciously or subconsciously. Although this is the cornerstone of the New Thought movement, it has entered popular culture in the past decade or so, primarily on the basis of a movie called \emph{The Secret}, and the resulting publicity on various television shows.

The principle is based on the idea that the universe will bring into manifestation the ideas, thoughts, or images held in the mind. Since no one is typically aware of this, especially in regard to negative things, the theory is that the universe is responding to the subconscious mind. The proposed solution, therefore, is to consciously direct thought to the desired result. There are two questions: first of all, is this true, and then, assuming that it is, what ought one desire?

This is not an idle question, because the same question comes up in Bhagavad Gita. In that text, the primary question is why some achieve self-knowledge while others don't; this can be secularized to why some have health, wealth, and love, while others are lacking. The accusation is that the Universe, represented here by Krishna, plays favorites. Krishna rejects that charge:

\begin{quotex}
In whatsoever way men approach Me, even so do I reward them. \flright{\textsc{BG 4:11}}

\end{quotex}
Here is probably the first expression of the Law of Attraction. Krishna responds in kind to what people desire. Those who desire pleasure or money will receive them. Those who instead seek for liberation, will gain liberation. In the next verse we read:

\begin{quotex}
Those who desire success in their works worship the gods here; for quickly, in this world of man, comes success from works. \flright{\textsc{BG 4:12}}

\end{quotex}
Success in the world comes quickly and easily, so that is what people act on. Self-knowledge and liberation are much more difficult, so few people seek it. Now I have known people who have focused on money and wealth, usually because of \textbf{Napoleon Hill}'s book \emph{Think and Grow Rich}. It has given them the motivation and the confidence to pursue wealth relentlessly, (i.e., money is a “minor god” to them).

On the other hand, I have witnessed religious science treatments, Course in Miracles adherents, and charismatic prayer meetings, all of which lay claim to miraculous healings. I never saw anyone cured of baldness or near-sightedness. Since there is a valid principle involved here in that the “representation creates reality”, the failure of this law needs to be explained. There is a practical explanation as well as a metaphysical explanation.

\paragraph{Law of Accidents}
Along with the Law of Attraction, there is the notion that “there are no accidents.” This can be understood in two ways.

The first way is that it is a restatement of the Principle of Sufficient Reason, that is, there is cause or a reason for manifested things. This reason may or may not be discoverable, but it is not due to the intention of the person.

The stronger interpretation is that everything that happens in your life has been attracted to you through your conscious or subconscious mind. In other words, it is a manifestation of your essential being. That explanation, however, does not take privation into account. In its essence the person is perfect, but not in existence. Insofar as a person experiences privation, he will be subject to accidents, that is, external forces that do not arise from his own essential being.

I have known many people who have become very distraught because of this teaching. When bad things happen, which they always will, they reproach themselves for “attracting” those things. And rightly so, if this idea is true, since it reflects their essential being. A vicious circle of desire, disappointment, and self-reproaching ensues. As long as the law of attraction is used to attract sense objects (money, sex, power), rather than spiritual enlightenment, there is no way out. To return to the Bhagavad Gita:

\begin{quotex}
When a man dwells on objects, he feels an attachment for them. Attachment gives rise to desire, and desire breeds anger. From anger comes delusion; from delusion, the failure of memory; from the failure of memory, the ruin of discrimination; and from the ruin of discrimination, the man perishes. \flright{\textsc{BG 2:62-63}}

\end{quotex}
Dwelling on a thought or image in the mind brings about a desire for its manifestation. Anything that thwarts it brings anger. Then delusion about reality and a forgetting of one's Real I. That is spiritual death. The delusion is that the manifestation of the desire will bring happiness, when that is not the case:

\begin{quotex}
The man of self-control, moving among objects with his senses under restraint, and free from attachment and hate, attains serenity of mind. \flright{\textsc{BG 2:64}}

\end{quotex}
\paragraph{Spiritual Mind Treatment}
There are two main techniques used by new agers: affirmative prayer and visualizations.

In affirmative prayer, there is the assertion that “I am” already what I desire to be. For example, the person tries to come into the awareness that “I am healthy” or that “I experience abundance.” There are variations on how this is done in the different schools. For example, a science of mind treatment will involve five steps. Christian Science will use a form of denial, such as, declaring, “this disease does not exist.”

The other technique is the practice of the visualization of the desired result. These techniques do have some basis in Hermetic and magical practices. However, in that tradition, there is a long training period necessary, something the New Ager wants to dispense with.

I hope it is obvious, though it is apparently not, that these treatments are not nearly as scientific or as reliable as claimed. Otherwise, the solution to all the world's woes would be a weekend course in the law of attraction. There are two defects with the whole idea.

First of all, there is the metaphysical restriction that only possibilities of manifestation can manifest, and furthermore, they must be compossible with other manifestations. Otherwise, everyone would be a millionaire without servants to do their bidding, all parking spots would be directly in front of the store, and Jennifer Lawrence would be totally exhausted.

Even assuming a real possibility, the other, more fundamental, issue is that man as such is not a united being. Spending ten or twenty minutes thinking of or visualizing a desired result will not overcome the rest of the day in which the mind is wandering all over the place, often thinking and visualizing results directly opposed to the treatment. Even during the time spent on the practice, the untrained mind cannot focus on a single idea more than a few seconds.

Hence, training in concentration is necessary before embarking on these practices. Then the time spent on the treatment may be all quality time. Then it would be necessary to maintain self-awareness throughout the day in order to suppress the random flow of ideas and images in the mind. As \textbf{Patanjali} expresses it:

\begin{quotex}
Yoga is the ending of the oscillations of the mental substance, \flright{\textit{Yoga Sutras 1.2}}

\end{quotex}
Paradoxically, the ending of these oscillations depends on not dwelling on the things of this world, as we read above.



\flrightit{Posted on 2014-09-22 by Cologero }

\begin{center}* * *\end{center}

\begin{footnotesize}\begin{sffamily}



\texttt{andros on 2014-09-24 at 05:44 said: }

Alexander Pope seems apt in this regard:

A little learning is a dangerous thing;

drink deep, or taste not the Pierian spring:

there shallow draughts intoxicate the brain,

and drinking largely sobers us again.

And in the same poem (An Essay On Criticism) he also gifts us with:

`To err is human, to forgive divine'

\& `Fools rush in where Angels fear to tread'

(An Essay On Criticism)


\hfill

\texttt{Logres on 2014-09-24 at 07:55 said: }

My grandmother was Christian Science. Before she died, she basically retracted her beliefs and embraced the Logos, as she could understand it. The fad for Eastern “thought-stuff” makes me wonder if modern Westerners are just trying to do with it, what has already been done in the West: subject it to the personal whims of desire and personality-based distortion in order to find a quick or intellectually lazy path. I say this, because much of the BG is identical in dogma to what is presented in both Old and New Testaments (“God is not mocked, whatsoever a man sows, that he also reaps”).


\end{sffamily}\end{footnotesize}
