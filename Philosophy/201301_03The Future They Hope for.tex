\section{The Future They Hope for}

\begin{quotex}
What inspired Teilhard de Chardin, and inspires his followers, is a certain unitary view of reality, a joining of God and the world, of the spiritual and the secular, into a single harmonious and all-encompassing process which can not only be grasped by the modern intellectual, but can be felt by the sensitive soul that is in close contact with the spirit of modern life; indeed, the next step of the process can be anticipated by the modern man, and that is why Teilhard de Chardin is so readily accepted as a prophet even by people who do not believe in God: he announces, in a very mystical way, the future which every thinking man today hopes for. \flright{\textsc{Fr. \textbf{Seraphim Rose}}}

\end{quotex}
Since we never got to the actual debate between the Chrises yesterday, we can try to mention the main points. Neither of the men are very deep or careful thinkers. As Fr. Rose points out, they share a common vision of the future, something they ``feel" more than ``think", despite their radically different starting points. They are the heirs of currents of thought that began centuries ago, which they now accept uncritically.

The best thinkers, those who laid the foundation for the modern world, dig down to the roots of thinking. By rejecting Euclid's famous fifth postulate, geometers were able to create alternative geometries. In an analogous way, certain thinkers rejected one or another of the fundamental concepts of Traditional metaphysics, and drew out all the consequences thereof. We have provided the examples of Francis Bacon who rejected formal causes, Spinoza rejected final causes, and Nietzsche proclaimed the death of God. They are not necessarily the conscious agents of change; more likely, they simply can't understand, or intuit, the point of traditional teachings.

Over time, the professors pick up these ideas, their students accept them uncritically, until eventually certain ideas enter the public realm; the conclusions are adopted although no one any longer understands how they came about. The common element is the revolutionary idea and opposition to the status quo. Several years ago, a secular friend married a spiritualist. She told me she was an ``iconoclast". I recall that I thought it was quite a strange thing to say. I pointed out that her husband was a real iconoclast, but he didn't have to boast about it. That ended that friendship.

I wonder how she feels now that the revolution is firmly in control, so it makes no sense to smash the idols. Of course, like the ancient Hebrews and Greeks, mere victory is insufficient; every man, women, child, livestock, and edifice must be destroyed. That is why the modern mind is so rabid and intolerant of any opposition to their imaginary future.

When someone argues very emotionally and illogically, you can be assured their point of view has little validity. That is Christopher Hitchens. Unlike Nietzsche's atheism, with its attendant transvaluation of all values, Hitchens' new atheism is rather tame, and actually unnecessary to his point of view. As we pointed out, his anti-clericalism and progressivist view is common to spiritualists, new agers, and others who may not share his atheism.

He does not think deeply, but instead begins in the middle with unexamined assumptions. A frequent theme is that religions are man-made creations. From a certain perspective, that is true especially if history is considered as no more than the examination of written texts as Guenon points out. What Hitchens doesn't mention is that scientific theories are likewise man-made creations, subject to change and refutation. Unlike religious texts, a scientific text can claim no higher authority, so it serves as a shaky basis for a worldview. Of course, that worldview claims that knowledge is progressive; so if it is admitted that what science teaches now is incomplete or even false, why should we accept anything now?

Hitchens' main point is that he adheres to a higher morality. He thinks that because he is under the illusion of ordinary life and simply cannot conceive a contrary point of view. The deeper question is why be moral at all? Morality makes no sense in his atheistic worldview. One can only be moral out of habit, inculcation, or from blindly accepting the imperatives of one's genetic programming. He fails to grasp that his moral and political views have no necessary connection to his atheism and belief in science.

He rants quite a bit, all in support of a progressivist worldview and in opposition to all religions which he lumps together. Of course, he does not allow the opposite, refusing to be linked to the nefarious actions of other atheists. He was obsessed with the treatment of menstruating women in Jewish law, but his point is unclear. While we don't think it right to banish such women from the public square, it is certainly not something to embrace. But that brings up a point. If you accept the premises of evolutionary biology, then feelings of revulsion must have a survival value. Nevertheless, he seems to praise practices that evoke such feelings in normal men.

There is a final point. Science is based on careful study of the past, from which hypotheses are derived. For example, Tycho Brahe spent decades tracking the motions of the planets; that allowed Kepler to formulate his laws of planetary motion. Similarly, the understanding of human nature, as a science, must begin with the study of history, from which an anthropologist can propose general laws. Since the future is everything to Hitchens, he departs from true science by ignoring the past. He proposes, in his utopian vision, revolutionary changes in social arrangements, changes which may never have been tried in the past. A real scientist, it seems to me, would be much more cautious about advocating such changes without first understanding all the ramifications.

Hedges was a non-entity as the foil. The only remnants of his Calvinist background that he retains is a dour personality and a belief in total depravity. Apart from that, the two Chrises agree on the revolutionary future. Hedges sincerely believes that the religious right is the greatest danger in the USA. In actuality, the religious right has been politically ineffective as they lack the intellectual chops to oppose the revolution effectively. For Hedges, they are nevertheless the tattered remnants of throne and altar, or the windmills of imaginary fascism that he thinks he is jousting.



\flrightit{Posted on 2013-01-03 by Cologero }

\begin{center}* * *\end{center}

\begin{footnotesize}\begin{sffamily}



\texttt{Mihai on 2013-01-04 at 10:09 said: }

It seems to me very relevant that the only people from the religious ``side" who are called to such debates are protestants and neo-protestants or at least progressive types in general- such as Vatican II Catholics. 

Since you mentioned Fr. Seraphim, he is very accurate in pointing out that Christians with an evolutionist/progressist worldview are extremely immature in thinking that simply adding God to the whole evolutionist paradigm ``solves" the problem. I've read even the statements of an orthodox monk who thinks there is no problem with darwinian evolutionism as long as one puts God as the cause of the process. 

Certainly, Christians who think like this, especially the ones who spend extended time with Scripture, like the monk above, suffer from a certain ``hardening" of the intellect when making such statements. And I am not saying this as an insult or an exaggeration. Clearly they did not spend a single moment to trully ponder the implications of such a worldview. For example: can anyone explain to me, in the light of evolutionism, the first 11 chapters of Genesis ? Or what is the purpose of the birth of Christ, who came to halt history's decline, if we are to accept no such decline, but on the contrary, an unconscious and impersonal evolution ?

And what about the description of the end times, which are represented as a period of utmost degeneration ?


\hfill

\texttt{Matt on 2013-01-04 at 12:03 said: }

Mihai,

That monk may have been speaking about evolution (genetic change) without the neo-darwinian framework attached to it that disregards formal and final causes. However, if he is speaking about it with the framework in mind, then yes, the problems are merely pushed to the side.


\hfill

\texttt{Michael on 2013-01-04 at 14:56 said: }

Mihai,

Of course you know the problem is that virtually all scientists teach the evolution of man. Do you believe that the theory of evolution will one day fall into disfavor as we learn more about our past?


\hfill

\texttt{Mihai on 2013-01-04 at 16:27 said: }

@Matt: Genetic change is one thing and evolutionism is another. Of course, these two things are confused with one-another, or people falsely assume that genetic changes ``proove" evolution, which is why we have this problem in the first place.

@Michael: The problem is not about the study of the past. It is about the worldview. Evolutionism was the dominating paradigm of the western world way before Darwin formulated his theory. This is not a scientific fact, it is a philosophical formulation. The large public (and indeed many scientists) cannot tell the difference between these two domains and this is why evolutionism is at the very hub of the modern mentality.


\hfill

\texttt{Cassiodorus on 2013-01-04 at 16:33 said: }

I've been reading some interesting material featuring the on going debate between the Intelligent Design community and Thomist defenders of theistic evolution. I finally understand the difference: the ID people have larglely accepted the premises of the materilaist's mechanical philosophy whereas the Thomists staunchly remain true to immanent teleology and Aristotelian formal and final causes.

I keep wondering if their is something about Scholastic moderate realism that made the surrender to nominalism something of a forgone conclusion.


\hfill

\texttt{Cologero on 2013-01-04 at 23:56 said: }

{\textgreater}{\textgreater}"I keep wondering if there is something about Scholastic moderate realism that made the surrender to nominalism something of a forgone conclusion."

Perhaps in the sense that non-Euclidean geometry is the conclusion to Euclidean geometry. There is no logical connection. Nominalism is the forgetting of the ideas. For the Scholastics (and beyond) to know is to know the essence or idea of the thing. When that no longer makes sense to anyone, nominalism is the result.


\hfill

\texttt{Noct on 2023-01-03 at 10:30 said: }

Evolution cannot be real because the greater is not derived from the lesser, it is the same narrative of the tower of Babel and progressivism in general. Pretending to reconcile religion and modern science is an absurdity that only leads to the corruption of traditional understanding by giving ground to materialism.

The ape is not potentially a man, and to involve God as an explanation of the process is to ignore causality for a simple evolutionary bias. God does not magically add new genes, mutations cannot mean a qualitative progression, every actualization depends on an already existing potentiality. The lesser does not contain the greater, but the greater always contains the lesser. Species do not evolve, but degenerate.


\end{sffamily}\end{footnotesize}
