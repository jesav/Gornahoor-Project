\section{Magical Idealism in a Larger Whole}

\begin{quotex}
The keenest sorrow is to recognize ourselves as the sole cause of all our adversities. \flright{\textsc{Sophocles}}

\end{quotex}
Since one aspect of building a system is to integrate a lesser one into a larger whole, what follows are some possible wholes that will accept what is profitable in magical idealism, while accounting for its defects.

\paragraph{Solipsism}
The temptation to solipsism exists in the idea of the Absolute Self. \textbf{Julius Evola} admitted this in \emph{The Individual and the Becoming of the World}:

\begin{quotex}
It has already been seen that the passage of the individual consciousness through the second stage entails solipsism , that is the impossibility of admitting coherently a multiplicity of subjects or I's in the last resort. That immanent certainty – as was said – which alone I can call ``I" is unmultipliable: it is a bare experience that mediates everything and is mediated by nothing – and as to the ``other" I's, they are nothing at all, they are nothing that I can experience directly, but only some hypotheses which I make recourse to in order to explain to myself some groups of representation of my unique unmultipliable experience. In summary, the ``other I's", insofar as they are ``other``, are not ``I" and insofar as they are ``I", they are not others, but me myself. All that is too obvious. 

\end{quotex}
Evola quotes \textbf{Otto Weininger} in support:

\begin{quotex}
the frightened withdrawal in the face of solipsism and the impotence of giving an independent value to being, the incapacity for opulent solitude, the need to chase the crowd, disappearing, plunging into the multitude. \emph{It is shallowness}. 

\end{quotex}
Is there not another path besides solipsism and the fall into the anonymity of the mass man? After all, isn't solipsism a feminine quality?\footnote{\url{http://anarchopapist.wordpress.com/2013/07/09/why-women-are-solipsistic/}} This reminds me of a story told by \textbf{Bertrand Russell} about a woman who approached him after a lecture, expounding her own philosophical vision. Russell listened, finally replying, ``Ma'am, that would make you a solipsist", to which she replied, ``Isn't everybody?"

The clue comes in the concept of privation. To the extent that I suffer from a privation, then there is a gulf between my essence and existence. I lack the power to bring all my possibilities into manifest existence. Since, for God, there is no such gulf between essence and existence, then ``I" am not God. The claim that the Absolute Self will someday manifest is insufficient.

The Hermetist knows the Law of Analogy, viz., ``As above, so below." While magical idealism gives me a great insight into my own consciousness, this can only be gnosis from below. So, by analogy, the philosophy can be applied ``above", i.e., to God, the only Absolute Self. For God, the problems of privation, spontaneity, and lack of power do not exist, since for Him, there is no privation, spontaneity is replaced by freedom, and He is all-powerful.

The Hermetist understands we are created free. Yet this freedom is a task to be accomplished, and the insights of magical idealism can reveal the obstacles for achieving freedom. There is no slavery in the creature submitting to an all-powerful God, no more so than the pledge of fealty of the Knight to his Lord. On the contrary, only the free man can make such a commitment.

\paragraph{Androgyne}
In any case, the goal of Tantra is not simply the Absolute Self, but rather the Androgyne. This alchemical marriage is also the goal of Hermetism and it is not clear to me how that would fit into Evola's system. That is why we introduce other Hermetists, such as \textbf{Boris Mouravieff} and \textbf{Miguel Serrano}, who are more focused on the androgyne

\paragraph{Cosmology and Physiology}
Tantra, as well as Hermetism, incorporate a complex cosmology and psycho-physiology. The former involves a hierarchy of beings who maintain the world process. The latter describes in detail a classification of the inner life. Magical idealism needs to be supplemented in those areas.

\paragraph{Spontaneity and Thinking}
A question arose recently in the Gnosis study group about how to quiet the mind by controlling thoughts. Since it may be of larger interest, I am providing the response.

\begin{quotex}
Of course there are random thoughts arising. The point is to experience firsthand what Gornahoor has been publishing in the commentaries on magical idealism. These thoughts are arising ``spontaneously" and we are not choosing them ``freely".

Only by constantly observing them, can one be convinced of that. Then you can eventually see how the thoughts arise in the first place, and how they link to each other to form a worldview (the lie or confabulation). That becomes the gnosis … you actually know this. By knowing, you become free from the bondage to these random thoughts and can consciously create a true worldview based on revelation.

The closer you can observe the thoughts arising, they will stop by themselves, often before becoming fully formed. That is the ``technique", if you want to call it that.

You will see that thoughts have different sources.\footnote{\url{https://www.meditationsonthetarot.com/the-cosmic-hierarchy}}

\end{quotex}
\paragraph{The Human Person}
Some people believe that we are opposing the Western tradition when, in point of fact, we are really highlighting its main features. For example, the ideas of personality are true to tradition; we just aspire to take them beyond rhetoric by making them real ontological experiences. For example, this is from the Catholic Catechism:

\begin{quotex}
Being in the image of God the human individual possesses the dignity of a person, who is not just something, but someone. He is capable of self-knowledge [and] self-possession. 

\end{quotex}
Even Evola uses the phrase the ``dignity of the person." What this means is that the person has more than just a human nature (``what he is"), but he also has the essence of the I (``who he is"). Self-knowledge is much more than something trivial such as ``I don't like broccoli" or ``I was born a certain way." Those are passive characteristics, and the Hermetic path is an active path.

Those who have been participating in the Gnosis seminar\footnote{\url{https://gornahoor.net/?p=9899}} have come to understand that authentic self-knowledge is a complex and serious business. Nevertheless, efforts pay off; in esoterism, a trophy is not awarded to just everyone.

Now that you know that self-possession is a mark of the person, go back and read everything that Evola has written about possession.

\paragraph{Personality and Character}
Thomist psychology distinguishes between the ontological person and the empirical (or empiriological) person. \textbf{Rev. Victor Warkulwiz} explains it this way:

\begin{quotex}
The ontological person is the person as such. The empiriological person is the ontological person as he manifests himself through his acts, powers, and habits, all of which are accidents. 

\end{quotex}
In other words, the ontological person is what the person is \emph{born} with, that is, it is his essence. The empirical person is what he acquires in the course of his life. This is no different from what Mouravieff writes. Fr. Warkulwiz makes a further distinction:

\begin{quotex}
Personality expresses the person through nonmoral acts. Character expresses the person through moral acts. 

\end{quotex}
Moreover, Fr. Warkulwiz makes this remarkable comment:

\begin{quotex}
The same person takes on a number of personalities. 

\end{quotex}
Again, this is a fundamental teaching of Gnosis, which actually describes and catalogues the different possible personalities.

The ontological person is obviously what we have been calling the real I. It is constant and stable apart from the manifestations of the various personalities. Unfortunately, as long as we identify with the empirical and material world, we remain oblivious to the real I.



\flrightit{Posted on 2014-07-20 by Cologero }
