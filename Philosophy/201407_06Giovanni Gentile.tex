\section{Giovanni Gentile}

\begin{quotex}
This is \textbf{Julius Evola}`s commentary on \textbf{Giovanni Gentile} in \textit{Saggi sull'Idealismo Magico} (\emph{Essays in Magical Idealism}). It has been difficult to translate, not just because of the difficulty of the subject matter, but more because there is no corresponding philosophical tradition in English from which to draw the requisite vocabulary and concepts.

As per his proclaimed method, Evola will point out an unexamined, and unresolved, problem in Gentile's system of \textbf{Actualism} that can only then be resolved in his own system of \textbf{Magical Idealism}. According to him, as long as Gentile remains on the philosophical, rational, abstract level, Actualism will never reach its goal. What is therefore required, as its completion, is a method of self-realization that transcends that level. This is, as Evola points out, the ``entire difference" between the two systems.

Our goal, in going through all this trouble, is the meagre hope that some young philosophy student will become interested in this and be willing to dedicate his career to moving such a line of thinking forward. If you are that one, please contact me in any way for some suggestions.

Now I understand there is a bit of excitement in some small circles among those who seem to have discovered ``neo-reaction" a couple of years ago. Their writings have been described in various superlatives. But, in reading what follows, can any of them be said to have the ability to ``play the game" at such a high level? The question is obviously rhetorical. They often make use of Evola's political analysis, as though it arose fully formed from under a cabbage leaf. Rather, it is secondary and is the fruit of much deeper thinking.

That is the challenge, then, to neo-reaction, identical to Evola's challenge to Gentile. As long as it remains on the plane of the rational, it will do no more than engender a torrent of abstractions. It will splinter into various sub-genres and devolve into a battle of personalities. What is required is a method of self-realization, a spiritual perspective, a religion. In other words, more than a Philosopher, we are awaiting a Prophet. 

\end{quotex}
Contemporary idealism can be defined like this: a deep need for absolute self-realization, which, however, the I does not acquire immediately in its interiority, but rather that it simply knows, assimilating it from outside itself in the phenomena that it causes in the rational order in the abstract. In Giovanni Gentile, this situation appears in a particularly clear way: for him, the effort of embracing and dominating the whole of the world in an immanent principle reaches its perfection; but, on the other hand, this principle remains a simple ideal entity, it is the previously criticized ``transcendental I" and expresses only a dull reflex of that deep individual power that was experienced, for example, in Michelstaedter. If Gentile could really call the I the ``pure act" of his rationalism, then he would appear not as the university professor, whose ``actualism" has the reform of the educational system as its goal, but rather as that cosmic centrality that esoterism shows, for example, in the types of the rishi, the yogi, Christ, and the Buddha. This is the entire difference between ``actual idealism" and magical idealism. Now since the pure rational-in-itself can never have its own justification, it can be shown how Gentile's system is based on an irrational fact, which, however, as such, contradicts his principle: from that a crisis arises, for which no solution can be given, if one does not pass from the first idealism to the second.

Gentile's fundamental principle is derived from Hegel and is expressed in the claim that pure immediacy, pure being is gnoseologically absurd, that the condition of every being is an act that posits it for the I. The priority of the category of the act over any content whatsoever of experience follows from that. Such an act is naturally that of the gnoseological subject, of the impersonal thinking I. Now thinking implies something thought. Regarding the concept of the thought, Gentile falls into a compromise: he distinguishes a ``logic of thought" with its own laws (relative to ``nature") from the ``dialectic of thinking", i.e., from the concrete and actual process of thinking and knowing. If nothing is, unless mediated, the ``thought" is nothing outside a logic of the thought that mediates it, which in its turn is inconceivable outside the concrete process of thinking, that is, from the actuality of thinking itself. So there is no way to truly distinguish a logic of the thought (or ``abstract logic"), supported by its own laws, from the process of thinking (or ``concrete logic"): these laws can appear only as some particular formations of the concreteness of this.

It is hardly worth pointing out that abstract logic is only the comprehension or abstract thinking of concrete logic, since here it would represent the difficulty in the question of knowing how such abstract thinking is possible since it, not thought, evaporates in the nothing; thought stops being abstract, and becomes a type of category of concrete logic itself. The concept of ``thought" and of ``fact" thus ends up affected by the same contradiction of the Kantian noumenon and is resolved in the concept of determination in general of thinking.

Thus we end up in this situation: on the one hand, thinking or ``pure act" remains the only category, capable of reclaiming every phenomena into itself; on the other hand, it is necessary to try to deduce from it a principle of determination, so that it effectively takes into account the multiplicity and diversity of the phenomena themselves, that it goes on to absorb. Gentile then considers the ``dialectic of concrete logic" for such a goal. To think, he says, inasmuch as it is self-positing, is to discern itself, i.e., to abolish the abstract, precise identity of the I in an object or non-I, (i.e., to determine itself) and, in then recognizing itself in this determination or object, mediates itself, to be like the I, like self-knowledge. But the recognition restoring immediate, precise identity, results in its process hurling toward the infinite.

The internal self-separation relative to autosynthesis, to the self-positing of the I as such, would therefore explain the genesis of determination: but the bad thing is that it does not explain itself. Gentile does not in fact give any foundation to that, because the I, in general, separates from itself ``another", and then also because it is not exactly reflected in this ``other", as, for example, in the divine trinity of Athanasius. In other words, there is in Gentile a simple, exposition of becoming, not a transcendental deduction, its meaning does not become obvious in any way; it remains a being of fact, not a being of law and the necessity that is connected to it apodictically can then only derive from an empirical suggestion. That is, it shows that from the Logic there is a problem of value, which, however, is absolutely neglected by Gentile.

Since the concept of the Aristotelian God that is used in the eternal identity of his act and that of the unmanifested Shakti of Tantra do not offer any inconceivability a priori, the question must be asked why the spirit must become (or is becoming). If one responds that a spirit that would not be objectified and therefore is not becoming, would not be self-consciousness, spirit, but rather nature, one retorts that since when was such a reaction something more than question begging; in reality nature would instead be spirit that, as such, being eternally constrained to self-objectification and becoming, would not be dissimilar in any way from the plant that, as such, can only vegetate.

Now every ``nature" presupposes, in its essential definiteness, a law, so that it is that determined nature and not another, but every law is a legislator that, as such, cannot itself be subject to a law. For Gentile, stuck on the plane of abstract speculation, the character of the legislator cannot have any meaning: believing that the problems are resolved through simply transporting them from the ``metaphysics of being" to that of ``knowing", believing that a certain determination of experience should turn out more comprehensible when the place of substance is called ``thought", he then restores becoming as the unavoidable nature of thinking, not realizing that a thousand questionable points then arise in order to substitute for them a thousand more. Does he notice that the objection presupposes the abstract point of view which, making the act an object, precludes the way of understanding it? Then one will reply that such an instance, rather dear to Gentile, has no value, because he gratuitously claims what he assumes as explanatory principle what is made by his opponents the matter of the problem. In the second place, that in the inebriated coalescence of the I to its act the problems are not only not resolved, does not even reach up to them, and that by basking in the obscurity he deceived himself that he has dispelled it.

Since ``actual consciousness" is understood as the form that collects every phenomenon, the adherence to it absolutely means, as we say, to adhere to the same world, letting oneself live from it and therefore to end up in a type of passive mysticism that then, in the concrete, is identified with an absolute, perplexed phenomenalism. On the other hand, according to his own principles, Genitle could well assume the noted objectification exactly because ``abstract" (and an objection, insofar as it is not identified with his doctrine, for the actualist will always, and a priori, be ``abstract") for, on the other hand, taken as the antithesis, it creates itself ``concretely" in a new synthesis. But this is precisely the critical point: when the actualist really does that, he would be constrained to transcend the plane of the rational and to affirm the principle of the arbitrary.

\begin{quotex}
In this section, \textbf{Julius Evola} deals with the nature of thought itself. Thought cannot be the object of thinking, since it would then be just another thought. Rather, there must be something that transcends thinking, the ``non-rational". Nevertheless, the non-rational is not the same as the irrational. Certain philosophers, Schopenhauer in particular, who consider the Will as irrational fall short. They experience themselves as the playthings of irrational forces. Instead, the Philosopher who has is in full self-possession experiences himself as the author, or at least master, of these forces.

\end{quotex}
That can be clarified as follows. For whoever possesses himself in the naked creative center of the I, it appears as the action of unconditioned freedom. Instead, it is experienced as logical necessity by whoever keeps himself on the peripheral plane of the discursive, as the inescapable imperative of a \emph{Sollen}\footnote{German word for \textit{should}, \textit{having to}.}, as the \emph{vis a tergo}\footnote{A force from behind.} of duty and law. The arbitrary affirmation, insofar as experienced so to say from the outside, passively, appears as truth or logical cogency, as apodictic. Gentile understood the formula I = Not-I in conformance to that, in which synthesis the process of concrete logic, as the ``ultimate and unconditioned condition of every thinking" without realizing that a condition can never be unconditioned or else the I is really \emph{its own norm}, and then every law can only be contingent and an ``unconditionally imperative characteristic of the law of concrete logic" becomes an empty sound. Or else there is a limit, or law indifferent to the power of the I, only in virtue of which this is such (as the \emph{Anstoss}\footnote{opposition} of Fichte's philosophy) and clearly postulated by the \emph{impotence} of being or thinking otherwise, if only not properly affirmable in the gnoseological sense; and then one may as well confess himself as a creature and pass on to religion – because one idol is not better than another.

It is not worth objecting, with Rickert, that the denial of the imperative of the law could be a particular case of this in the sense that, to the extent that it is willed, it is recognized worthy of the choice in respect to the opposed alternative of adhesion to the law, since that presupposes agreement with that which is in question, i.e., that it cannot be given a real negation, that it is not possible to be reaffirmed beyond the given facticity of the law \emph{in general} of \emph{Sollen}—that the mode of living the act according to \emph{Sollen} represented an extreme instance. This difficulty reappears from another side and is close to the problem, as the same philosophy of the act it is general possible. Since, of the two choices:

\begin{itemize}
\item Either the I is entirely compenetrated with its activity, and then cannot embrace it with a synthetic gaze and discover its general law, but rather, lost in it, it can only learn from it in the unpredictable, uncoercible contingency of the moment. 
\item Or else the I, almost lifting itself from itself, can know that law, but then otherwise its contingency is conceded, since with that, this becomes a distinct ``thought", and remains almost interiorly surpassed in the principle that makes such a distinction possible—this surpassing, keeping in mind that, by hypothesis, has nothing to do with the internal surpassing of that law and contemplated by it. 
\end{itemize}
Gentile has seen the difficulty and, having rejected the second option, tries to make himself sufficient to the first, saying that if he recognizes he has given with his doctrine not thinking in itself, but a \emph{concept} of thinking, that is, not thinking but \emph{a} thought. He concedes otherwise the relativity and the abstract nature of such a ``concept of the self-concept" (that is, of his entire philosophy) and he is ready to abandon it to the indomitable becoming of dialecticism, his own \emph{Logica} being understood as a simple milestone, and as abstract as all the others. But the parry is mistaken: relativity, the remission to the becoming of the spirit is also the principal content of the Gentilian concept of the self-concept, to that ``thought" in which Gentile—here and now—conceives and kills ``thinking", so that this relativity is the hypocrisy of exclusivity and dogmatism, this pretense comes from itself, this sacrifice does not accomplish anything but a staying firm, in the closed circle of the contingency of the moment, having outside of itself its own reason.

If the solution must be \emph{real}, if the vicious circle must be ruptured, then it is necessary that the passage is not that from one concept to another, from one ``thought" to another, but \emph{rather that of the entire dimension of the logico-discursive consciousness into that higher or deeper plane of absolute freedom}. Then it is necessary to surpass the creatural concept of the identity of freedom and the law, the \emph{vis a tergo} of the \emph{Sollen}, and be recognized in that unconditioned principle in respect to which the same act is a fact and that in truth stands at the beginning of Gentilian philosophy approximately in the same relationship that the Gnostics understood between the spiritual principle and the demiurgic principle.

This is an utterly significant situation: the instance of its own surpassing rises from the same rationale: the same discursive consciousness imposes on the I to go beyond it, if it wants to penetrate, to its foundation, that thinking that resolves the world in an immanent principle. As Abbagnano sharply notes, true thinking, thought \emph{thinking}, it can never become an object of itself, it can never be \emph{thought}, what other meaning, if not that it falls outside the plane of the rational? Thus the effort of contracting all reality in thinking and all life is overturned in emptiness: the very act of celebrating the greatest power of thinking implies that it is transcended in a the non-rational.

Nonetheless here a new question arises: what is the meaning of this non-rational? Or: in what relationship is the I with this non-rational? Such a point is important because it defines the position of magical idealism in respect to irrationalism. When the non-rational (to which therefore the foundation of rational law, or logical \emph{Sollen,} must be pushed back: every rationality should be the effect of a deep non-rational, but arbitrary, affirmation) for the I is like a blind and uncoercible power, that cannot in any way direct and dominate and which is felt as an accident dispersed in an indefinite, unpredictable becoming, it is truly to be thought that the surpassing beyond the rational has not yet entirely accomplished, that the I still looks at the \emph{alogon} from the outside, that it has not penetrated itself and \emph{possesses} at the center of the originary and unconditioned principle of creation.

The irrationalism that proceeds from such a situation is that of Schopenhauer, Bergson, Le Roy, Abbagnano, etc., and is confused with a type of vivified empiricism: it indicates the moment of transit, in which the I is already detached from the hallucination of pure logic and is oriented to the deep power from where logic rises, but on the other hand is not yet sunk into this power, so that it lives it only passively. For those who have instead absolutely possessed themselves, the \emph{alogon} indicates only the unconditionality of his will, autarchy. Therefore, beyond the abstract intellectual, there is still an abyss between the man who lives his own life as lord and master, and the man who feels himself only as a demonic force of a nature that is passive to himself and that has its \emph{raison d'etre} outside himself (certainly, not in the rationalist sense of the phrase, but in the Greek sense, used by Michelstaedter), briefly: spontaneity.


\hfill

\begin{quotex}
This is the third and final installment of \textbf{Julius Evola}`s commentary on \textbf{Giovanni Gentile} from \emph{Essays on Magical Idealism}.

Although it is highly technical, we can cut to the main point. First, there is the distinction between \emph{spontaneity} and \emph{freedom}. In a free act, ``I" make the choice. A spontaneous act seems to happen automatically, seemingly without the awareness or involvement of the I.

The application of this comes in the idea that the \emph{world is my representation}. That is, I create in consciousness the sensation of the world: its colors, shapes, sounds, etc. The naive realist view, on the other hand, believes that there is a real world, out there, right now, that somehow impinges on my consciousness and that is the source of my representation of the world. No one can say exactly how that happens; besides, Kant seems to have demolished that as a possible explanation.

In opposition to that, previous idealists have instead proposed other sources for such representations, as Evola points out. Evola rejects them all. Instead, he claims that the spontaneous nature of the representation of the world is due only to a privation, i.e., it is due to my impotency.

This is important because almost all non-philosophers are naive realists; that is, they believe they are passive receptors of a true representation of the world in their mind. When restricted to the physical world alone, this is not so serious as people can get along in the world with this naive belief.

However, the mischief arises in regard to the human world. Naive realists falsely believe they have been given a true representation of the human, or moralized, world, that is, the world of action, motivation, desire, and so on. They are unaware of the extent to which they are projecting their own ideas, prejudices, errors, ignorance, etc., onto their representation of the world. That is, their representation is spontaneous, created without conscious awareness.

Here, the magical idealist has a way out. He takes possession of his own world representation in the awareness that he has freely chosen it. 

\end{quotex}
It is clear that when the surpassing of the rational occurs in connection with that impulse to persuasion that was manifested in rationalism as the will to dominate every reality in thinking, only at the point of life experienced as autarchy is the principle restored, that can give to logic its justification. Therefore the further dimension in magical idealism must be understood, in which the positions of Gentilian philosophy are integrated.

That said, let's definitively establish what is, in the last instance, the inherent value to modern idealism starting from Kant up to Royce, Weber, and Gentile. The fundamental principle is that \emph{spontaneity} is one thing, and \emph{freedom} another. The former is the activity whose principle one has simply in oneself (which is spontaneity, according to Aristotle's definition); the latter is the activity whose principle the I has in itself, but stands to this \emph{in relationship of possession}. In spontaneity the possible is identical to the real in the sense that the act has the form of an absolute being-tied-to-itself, of a nonconvertible compulsion, of a mindless occurrence: briefly, in it the principle is \emph{passive} in respect to oneself. Instead, in freedom, the possible is \emph{not} identical to the real, a moment of autarchy, of real possibility (not of \emph{dynamis}, but rather of \emph{potestas}) controls the act as the ultimate reason for its being or non-being, of its being this or that. Hence, it is that the real is contingent to the possible—and not through privation, but rather through perfection and through the possession of the principle of actuality. In conjunction: one thing is not having conditions from another (i.e., coerced non-being, \emph{negative} freedom exactly characteristic of spontaneity, of the false Spinozan \emph{causa sui}), another is the not having conditions absolutely, being \emph{positively} free, which also implies the absence of internal determinations and arbitrariness (the non-inconvertibility) of the act. \emph{Now to the extent that a being is an I, to that extent and for that reason, it is freedom and not spontaneity}.

Once that is understood, when the idealist, against any contingency of experience—for example, against an expanse—says that it was he who posited it, it is evident that he is referring not to freedom, but rather to spontaneity. He in fact refers to the simple representing, to that elementary assent so that, in general, this is to be aware of things, assent that, if it is the \emph{necessary} condition for every reality, qua reality, then \emph{for} the I, it is quite far from also being the \emph{sufficient} condition. In fact, in representing there is not a subordination of reality to possibility; the I is passive to his own act, he does not so much affirm things, but rather these are affirmed in him. As passion or emotion, the representation is something of \emph{his}, that is intimate to him and which he draws out of his own interiority (at this point the legitimacy of the application of idealism arrives, moreover satisfied by Leibniz), but it is not \emph{him}, since the I cannot say he gives it freely to himself, since he is not at its determination in regard to \emph{unconditioned} causality and possession.

Consequently, \emph{to the extent that the idealistic reduction of nature appears as a position of the I, it reduces the I itself to nature, i.e., insofar as that I, which is freedom, knows nothing or, better, acts as if he knew nothing, and, with obvious paralogism, it equates the concept of the I with that of the }\emph{principle of spontaneity}—which is then in truth that of nature. So that the meaning of the idealistic claim that ``The I posits the non-I" is, in reality, ``Nature posits itself" or more simply: ``A world is".

The crux of such a doctrine is therefore Spinozism. But Spinozism, dedogmatized, leads to phenomenalism. And this demonstrates the very history of idealism. In Kant if in the transposition of the individual I to the impersonal ``I think" and to ``consciousness in general", there is the first step toward the dissolution of freedom, there nevertheless still remains a residue of interiority in virtue of the opposition of the limit of the thing-in-itself to the activity of the synthetic \emph{a priori}. The progress of the theory of immanence about this dualism is moreover that of the \emph{détente} of individual affirmation. The I of Fichte assimilates the non-I (the Kantian thing-in-itself) only by becoming the abstract egoity dispersing itself in the world of the preconscious. The Hegelian Idea does not reaffirm the logical principle on that totality of concrete determinations that Schelling's monochrome ``philosophy of identity" let fall outside of itself provided that it admits the ``other" in itself, makes itself dialectic, calls rational and free those contingent determinations of things which, being simply \emph{given}, standing to the I in a relationship of \emph{force}, that justly can only be called irrational (in the Greek meaning of the word).

Finally, among the ruins of the ``philosophy of nature" and the heterological of the \emph{Ohnmach der Natur} [impotence of nature] on the one hand, on the other by falling short of the ideal of \emph{a priori} knowledge in the order of positive science (to the reduction of geometry and mechanics by deductive science prior to experimental sciences – so that the presupposition collapses from Kant's criticism) there is the final collapse of the individual in the irrational becoming of phenomena, with which he makes self-knowledge itself coincide, without residue and further mediation. The I does not make itself the form that dominates (gnoseologically) the contingency of the phenomena — Schlegel's ``Chaos outside of the system" — that as the current awareness of Gentile's philosophy, i.e., that as the absolute adhesion of the act in spontaneity according to which things become and assert themselves in the I — however, just as it was said, for a type of passive mysticism that in practice is identified with an absolute, perplexed phenomenalism.

We pointed out in the second essay that beyond the identification of reality and will (= possibility) the question remains, whether the will is the criterion of reality or whether reality is the criterion of the will. That is, if what happens is called real because it is willed (however real only in the measure in which it can be called willed, moreover that correlative of a \emph{privation} of the will remaining not real), or else if it is called willed because it is real, i.e., for the simple irrational fact of its being there (\emph{Dasein}), of its brute \emph{to oti}: it was shown that only the second alternative meets actualism insofar as it is not elevated \emph{to a doctrine of power or magical idealism}. The being that it rightly posits does not posit it that way because it finds it \emph{de facto}: ``Mustness" is the truth and the foundation of its ``shouldness". Analogously, since a great number of the events of experience in general cannot be connected back to a beginning of informed deliberation, of intentional predetermination on behalf of the real I (and this I can prove at every instance to the idealist, since he does not want to deceive himself purposely), from that subjective activity from which things are posited, it is necessary to abstract the natures of finality and conscious predetermination – at long as one does not prefer to pass (as was the logical consequences of the premises of transcendental philosophy that Hartmann precisely dealt with) to a philosophy of the unconscious.

Finally, since one cannot even give \emph{a posteriori} a logical-axiological construction of such a vast complex of phenomena (Prof Krug's pen is still waiting to be deduced), the logicality proper to this cosmogonic function of the I must be reduced to a \emph{minimum}, to an abstract generality, to an empty universal that is equally adequate to the specificity of infinite distinctions – to which the character of logic characteristic of the pure Gentilian act responds exactly, which being able to be called everything indifferently, is as if it is called nothing; it is a sack that can equally well contain everything. Empty indetermination from the logical point of view—however, \emph{materialiter}, irrationality — intentionality, afinalism, passivity, pure spontaneity exhausting itself all in the variety of ``here and now" — such are therefore the marks that define the function, in which the doctrine of immanence believed it recognized its supreme celebration, since it has the courage of thinking it down to the foundations.

In Spinoza that blind spontaneity of that which can be only what it is, of that which is passive in respect to its own nature – so that the I is reduced to a vain and incomprehensible shadow — was God; in Gentile this God is known in its truthfulness and is made explicitly nature, the uncoercible folly of phenomena so that the relevant doctrine is identified, beyond all the logical paraphernalia, with Bergson's. In the one as in the other, the individual does not \emph{consist} but \emph{surrenders}, does not dominate things but loses himself there and is dissolved in an intoxicated coalescence that recedes into a demonical principle. Such is the way of corruption, the self-lampooning of the immanentistic application.

In conclusion. I said that mere representative activity is the necessary, but not sufficient, condition of the reality of things, \emph{since these are related to an I}. I can say I posited things, but insofar as I am spontaneity, not insofar as I am an I, and that is freedom. Now to say that I, as I or sufficient principle, cannot recognize myself as the \emph{unconditioned} cause of the representations (i.e., of nature), it does not at all mean that these representation are caused by ``another" (by real things or existing in themselves) but, simply that I am insufficient for a part of my activity, which is still spontaneity — \emph{that such a part is still not moralized, that the I, as freedom, suffers a privation in it}. So that it is realism, as we said, that must be pushed back for a goal of not receiving. When then will one be able to truly affirm the principle of idealism, that the I posits things? When the individual has transformed the dark passion of the world into a body of freedom, i.e., when he has made the form pass by which he lives the activity represented by spontaneity, by connection of reality and possibility, to unconditioned, arbitrary causality — to potency. In the face of this task, the idealist flees instead: to the real or magical act, to the act that, by possessing them, abolishes things. He substitutes the discursive act that recognizes them and is supported on them. \emph{He calls being his non-being}, he calls real that which, being real that, as the privation of his power, should instead correctly be called unreal, and so confirms this privation, adulates it, and incestuously feeds on it. Insufficient at the point of the I, he abdicates and dissolves himself in things; and rationality, historicity, concrete liberty, the transcendental I, etc., are only so many names of this flight, are only the symbols of his impotency, that the values given by force to that which, in relation to the point of the I and morality (in Weininger's and Michelstaedter's sense of the word) is non-value = death and obscurity: nature.



\flrightit{Posted on 2014-07-06 by Aeneas }
