\section{The Truth is far from Suspected}

\begin{quotex}
Be assured, savants of the world, it is not in disdaining the sacred books of nations that you show your knowledge, it is in explaining them. One cannot write a history without monuments and that of the world is no exception. These books are the veritable archives wherein its deeds are contained. It is necessary in exploring the venerable pages to make comparison between them and to understand how to find the truth, which often languishes there covered by the rust of ages. \flright{\textsc{Fabre d'Olivet}}

\end{quotex}
In our task to identify the sources of the ``formidable mental deviation that characterizes the modern west" [\textbf{Rene Guenon}], I took one for the team and endured a ``debate" on youtube between the two Christophers — Hitchens and Hedges. It was advertised as the contest between two worldviews: the scientific one represented by Mr. Hitchens and the religious by Mr. Hedges, although, in point of fact, they more or less agreed on everything. What is more remarkable, however, is their unexpected, and probably unwelcome, agreement in many respects with séance meeting spiritualists as described in Renen Guenon's \textit{The Spiritist Fallacy}.

There is no value in getting involved in such debates if only for the most basic reason that it assumes a common intellectual plane for such a debate to occur. Nothing could be further from the truth since what would be necessary is the questioning of fundamental assumptions. The religious point of view in that debate involved too many disparate and incompatible notions; to do it properly would require the translation of the religious dogmas to their metaphysical equivalents. Only then could there be a fruitful discussion apart from the obvious difficulty that few would be capable of engaging in such a discussion for a variety of reasons.

The modern worldview represented by the Chrises must start from the assumption that the past is no more than a period of darkness, ignorance and depravity that can only be redeemed by modern ideas. Regarding the study of the past, Guenon writes:

\begin{quotex}
History, as officially taught, limits itself to exterior events, which are only the effects of something deeper; and it sets these events forth in a tendentious manner under the influence of all the modern prejudices. And further, there is a veritable monopoly on historical studies in the interest of parties, both political and religious. 

\end{quotex}
The Hermetic method of doing history as described by Fabre d'Olivet, whose influence on Guenon is undeniable, is quite different. To grasp the world of the ancients expressed through their sacred scriptures requires a deep understanding of their essences which is revealed in their homogeneity rather than simply in the outward events described. To disdain those writings and neglect their truths is to limit oneself to the parochialism of the present. Even more than Guenon, who mined such texts for their symbolic and metaphysical value, \textbf{Julius Evola} made use of myths, legends, and sacred texts to discern the worldview of the ancients. Regarding the deprecation of that worldview or the deliberate ignorance of it, Guenon writes:

\begin{quotex}
This falsification of history seems to have been accomplished according to a set plan; but if this is so, and its essential aim has been to have public opinion consider this deviation as `progress'. everything seems to indicate that it must be the work of a directing will. … in any case, it can only be a collective will, for there is manifestly something that goes beyond the sphere of activity of individuals considered in isolation. 

\end{quotex}
First of all, let's be clear about the more or less common opinions held by the modern mind. It is characterized by appeals to humanitarianism, moralism, or sentimentalism, rather than anything properly intellectual. They envision a world of pacifism, universal brotherhood, various ``rights", liberation, and so on, all to be brought about in a more or less distant future. All of this is couched in the guise of ``progress" and ``evolution" and allegedly backed up by science. Guenon describes its effects:

\begin{quotex}
One can hardly imagine the seduction that grand words offering a false semblance of intellectuality exercise … This is a kind of verbalism which provides the illusion of thought for those incapable of really thinking; it is also an obscurity which passes for profundity in the eyes of the common man. 

\end{quotex}
There is little chance for intellectualism to gain any headway in this discussion. To those accustomed to live only by their passions and their arbitrary ``likes", a commitment to the intellect seems oppressive to them, rather than the true liberation that it is. The appeal to Tradition is associated with the defects of the known past. On an emotional level, that can only fall short of the imagined glories of the unknown future.

What is really curious about all that is how that modern worldview arises as the conclusion of disparate premises. For example, spiritualists and the ``new atheists" come to the same conclusion, although the latter abjure the former. New Agers, liberal Christians like Hedges, socialists, and so on all agree on that worldview. In other words, it matters little the exterior religious or philosophical allegiance, so long as the result is the same. This leads Guenon to this suspicion:

\begin{quotex}
If one does not believe in chance, one is forced to admit the existence of some kind of equivalent of an established plan, but one which evidently does not need to be formulated in any document. Isn't the fear of certain discoveries of this kind one reason for the superstition of the `written document' as the exclusive basis of the historical method? Starting from there, all that is essential necessarily escapes investigation. 

\end{quotex}
Once again, we see that understanding history requires determining the essence behind the appearances. Since that cannot be determined from written documents, it is rejected by ``official" historiography which is based on texts. Hence, any mention of a ``plan" is ridiculed as a conspiracy theory. This is exacerbated by the fact that most people are merely the unconscious instruments for the plan's effectuation, so they can plausibly deny it. Of course, Guenon is not some simple-minded conspiracy theorist. He elaborates:

\begin{quotex}
It is impossible to believe in the spontaneous production of movements of any importance. In reality, things are more complex than we indicated; instead of a single will, we should envisage several intentions as well as several results; there could be a whole special dynamic in this, whose laws would be interesting to ascertain … the truth is far from being generally known or even suspected. 

\end{quotex}


\flrightit{Posted on 2013-01-02 by Cologero }

\begin{center}* * *\end{center}

\begin{footnotesize}\begin{sffamily}



\texttt{Jason-Adam on 2013-01-02 at 23:40 said: }

I'm currently reading Fabre d'Olivet's book right and clearly see the huge influence it had on Guenon \& Evola – though obviously more on Guenon…………

In studying the history of ideas one needs to understand that the person said idea usually had an end in mind beyond inocuous speculation….by discovering the hidden ends and means we unravel history.


\end{sffamily}\end{footnotesize}
