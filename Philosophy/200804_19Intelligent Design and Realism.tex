\section{Intelligent Design and Realism}

\label{sec:IntelligentDesignandRealism}

A couple of years ago I participated in an on-line discussion on the web site of the \textit{International Society for Complexity, Information, and Design (ISCID)}.

The host was a philosopher (Del Ratzsch) who had just published a good book on Intelligent Design from a philosophical, not scientific, perspective. I asked him this question:

\begin{quotex}
Dr. Ratzsch, do you see any relationship between the philosophy of ID and the problem of universals (e.g., if something is designed, doesn't that imply that it instantiates some universal idea)?
\end{quotex}

This was his response:

\begin{quotex}
It certainly, it seems to me, instantiates an idea, and for design theories to do much the idea would need to be recognized as such (even were its content not recognized). I'm not sure that that by itself would commit one to a specific endorsement of universals, or something of that sort. Could you elaborate? 
\end{quotex}

I then elaborated:

\begin{quotex}
Elaboration: I guess I was driving at the issue of metaphysical realism vs nominalism. Is ID neutral in that regard — that is, is it purely an empirical science — or does it commit one to realism over nominalism?
\end{quotex}

He answered this way:

\begin{quotex}
That sounds like a question that could use some further thought, and I'm not sure how I want to answer it. I tend to have sort of a gut sympathy for realism (in this sense), but on the other hand, it was in part a nominalism arising out of theological considerations that got science off the ground initially. At the moment, at least, I'm not sure that ID would force one either way, although my \_guess\_ is that realists would outnumber nominalists among ID-friendly philosophers. If anyone has other suspicions on that, I'd be interested in hearing them.
\end{quotex}

Obviously, my ``sympathy" is with realism and I can't imagine how an Intelligent Design can make any sense at all unless it starts with an idea. When Prof. Ratzsch refers to ``theological considerations", he means the Reformation which rejected the realism of the Scholastics in favour of a formless nominalism. The fact that science allegedly got off the ground under the influence of nominalism is hardly an endorsement. It reveals the limitations of science. The havoc in social structures today is due to nominalism run amuck as so-called scientific theories are applied to politics and society.

Authentic science must be based on realism, that is, it accepts on principle the rationality of the world and the ability of man to comprehend it … in short, a belief in the Logos. Thus, the true battle between neo-Darwinism and Intelligent Design is not scientific, but metaphysical.

This battle has been going on for centuries, and I refer you to \textit{Ideas Have Consequences}\footnote{\url{http://brothersjudd.com/index.cfm/fuseaction/reviews.detail/book_id/1200/Ideas\%20Have\%20C.htm}} by Richard Weaver. This is a book that is on the short list of every so-called paleoconservative, but seldom does it form the basis of any of their arguments.

\flrightit{Posted on 2008-04-19 by Cologero }
