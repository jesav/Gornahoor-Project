\section{The Great Cat Photo Contest}

\begin{quotex}
To be popular, it is advisable to give people what they want, not what they need. Usually, that involves sentimentalism.

Abd al-Qadir al-Jilani relates a story about cats and the whole audience begins to weep from spiritual emotion. They listened with boredom to the brilliant sermon of the great theologian. 

\end{quotex}

\begin{wrapfigure}{rt}{.35\textwidth}
 \includegraphics[scale=.5]{a20140814TheGreatCatPhotoContest-img001.jpg} 
\end{wrapfigure}

The visitors at the cat photo contest look for the cutest cat. However, the judges consider different factors such as originality, technical excellence, composition, artistic merit, and overall impact. The point is that the cat is irrelevant, since the same factors are used to judge every other photo contest. In other words, for the naïve visitors, the cat in the picture is the object of interest. However, for the judges, the photograph itself is the object of interest. The judges see through the photograph to the agent who created it.

Social analysis follows analogous rules. One can be like the cat lover and take what is going on as complete in itself. Then he will choose a position that he thinks is the cutest. The alternative is to understand events as a battle of worldviews and the hidden factors that propel them.

\paragraph{The Marketplace of Ideas}
John Stuart Mill was said to be remarkably intelligent, but the concept of a marketplace of ideas is ludicrous. Allegedly, in a free discussion the better ideas (or better, worldviews) will win out. But that assumes that the ideas in the marketplace are objective, fair, and disinterested. It leaves out of the equation those who create or promulgate certain worldviews and why they do it.

The worst people, who assume they are the most intelligent, are the ones who say things like “I will listen to both sides of the issue and then decide.” That makes them purely passive receptacles of others’ ideas. In a fair market, all parties are privy to all information. For example, if there is a drought in some coffee producing region, everyone knows it and the price takes that into account. That is why sellers try so hard to create unfair markets: e.g., by hiding information, locking competitors out, and trying to establish brand name distinctions.

So how fair is the marketplace of ideas? First of all, unlike news about coffee droughts which are less likely to be ideologically biased, the promulgation of ideas and worldviews is tightly regulated. In the USA, for example, six corporations account for nearly all the news sources. Many ideas, considered beyond the pale, are prohibited from the market place … there is no need to identify them. So the choice is quite limited.

Then there is the issue that in modern times, complex ideas require a deep knowledge of science, economics, mathematics, history, and philosophy to be fully understood. Now I know for a fact that such a breadth of knowledge is not very common.

At university, for example, I signed up for a logic class in the philosophy department. There were 25 students on day 1, but only three of us finished the course. Specifically, the future philosophers, journalists and political scientists all failed to take a formal course in logic. Economics was another class with a high dropout rate. You can probably assume that the talking heads on TV are woefully ignorant of economics. Forget science class altogether, because if you don't sign up, you won't have to drop out. Yet these commentators claim to understand difficult questions in climatology and human genetics.

Most amazingly is that everyone things himself capable of understanding issues of politics and metaphysics. Yet, to be admitted to Plato's Academy, it was first necessary to master maths before considering those issues, as we pointed out several years ago in Maths and Politics.

\paragraph{Development of Thought}
There are two ways to try to move human thought further: the revolutionary and the evolutionary.

\textit{Revolution:}
The revolutionary way is to take the dominant idea and proclaim its opposite. After a while, this gets very easy and predictable. It started with the reformation, enlightenment, and the French revolution. The reformation challenged the dominant spiritual authority, the enlightenment challenged the very idea of a spiritual authority, and finally the French revolution overthrew both the spiritual authority and the political power.

Karl Marx gave the revolutionary impetus a firm philosophical foundation. First of all, he correctly recognized that a “spectre”, or spirit, is haunting Europe. This is not a metaphor or other figure of speech as our cat lovers might suppose. The goal of that spirit is to overthrow the existing social and political order of things. Yet people I speak to who are haunted by that spectre never seem to recognize themselves as Marxists in spirit.

Subsequent developments in the West all follow from this. Marx was interested in the economic-political order, so he proposed that the proletariat would be the agents for the overthrow. However, developments a century later added some complexity. If the socio-political order is understood to be white dominated, then minority races become the new agents for the overthrow. If that order is understood to be male dominated, then women as feminists become the new agents. If that order is understood as hetero-normative, then deviant sexuality becomes the new agent.

With this principle, everything comes into focus. The revolutionary worldview is certainly original, it is promulgated with technical excellence through the mass media, and its overall impact is undeniable. Eventually the revolution itself becomes the established order and it is difficult to see where else it can go from there. Reaction won't automatically follow, since reaction is the opposite of the revolution.

\textit{Evolution:}
The evolutionary way is the way of depth. Previous thought is not overturned, but is understood on a deeper level. There are two ways to initiate that process:

\begin{itemize}
\item Bring out all the logical consequences of earlier thought 
\item Integrate it into a larger whole 
\end{itemize}
Here was need only recap what was written in more detail in recent weeks. From Tomberg to Keyserling, we see the method of depth described. The meaning of things is contemplated, even if it is multivocal. Freedom of the will is the foundation for thought and action.

\flrightit{Posted on 2014-08-14 by Cologero}

\begin{center}* * *\end{center}

\begin{footnotesize}\begin{sffamily}

\texttt{jc on 2014-08-21 at 06:28 said:} 

Lately I have been interested in the works of Kurt Gödel, amongst the realisation that we do live in a universe that is interconnected and infinitely more complex than we initially imagine.

It's only when we understand fundamentals like what was written about Plato's academy, that metaphysics takes on a deep and ontological meaning. There are no short cuts, though there are real and tangible gains.

\hfill

\end{sffamily}\end{footnotesize}
