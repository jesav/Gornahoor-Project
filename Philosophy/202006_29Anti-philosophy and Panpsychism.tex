\section{Anti-philosophy and Panpsychism}

\begin{quotex}
A philosophy is not a court of law. It is not a matter of being right or wrong. It is a sign of great vulgarity to want to be right and yet more to want to be right against someone else. And it is a sign of the same vulgarity to attend a philosophical debate with the thought only of seeing one of the two adversaries be right or wrong. Speak to me only of a philosophy that is more resolute, or more profound, or more attentive, or more pious. Or more unbound. Speak to me of an austere philosophy. Or of a happy philosophy. Speak to me above all of a certain \emph{fidelity} to reality. \flright{\textsc{Charles Péguy}, \textit{Note on Bergson and the Bergsonian Philosophy}}

\end{quotex}
Metaphysics, properly understood, is not one philosophical system among, and opposed to, others. Rather, it is an understanding of Being and, as such, is its own argument. Attempts at refutation end up begging the question, that is, they subtly assume what they are refuting.

\paragraph{Matter and Quantity}
Materialism is an anti-philosophy that considers matter to be the only reality. Scientists like Galileo are said to have created modernity by denying that the ``external world" has any qualities, or qualia, such as colour, etc. Instead, only mathematics is necessary for physics; quantity alone has explanatory power so qualia are unnecessary.

That sounds rather revolutionary except that it has always been known. \textbf{Thomas Aquinas} knew the principle materia signata quantitate, that is, matter is designated by quantity. Therefore, in his view, the material world has no qualities and is constituted solely by quantity. For example, a ball has measure (it is a sphere), number (it has a radius), and it has weight or mass. Yet if I asked Galileo to toss me the red rubber ball, he would not throw the nearby cannonball, since it is of a different colour and texture.

So, we see that even if the so-called external world has no qualities, no one can navigate that world very well without them. Instead of ``external world", let's be specific: it is better to call it the physical or material world or, in other words, the world known to scientific realism.

\paragraph{Qualia}
The common anti-qualia argument is Kantianism reinforced with some concepts of scientific realism. For example, the apple itself is not red; rather, its redness is a purely subjective experience caused by the reflection of light on the apple, which affects the retina which then creates the sensation of ``red" in consciousness. The apple tastes ``sweet" only because our taste buds interpret it as sweet, and so on for texture and other qualities.

So, in this worldview, the object is defined fully by measure, number, and weight. What we call its ``qualities" are not at all attributable to the thing, but are just subjective experiences. That leads to an unresolvable dualism between the objective thing and subjective experience. Unfortunately for this perspective, the qualities are not all that subjective, since observers will usually agree on those qualities.

The scientific worldview has no good answer for that. Although the traditional view agrees that quantity suffices to designate material things, it has a fuller understanding. Along with matter, things also have a form (or essence). What are called ``qualia", then, are provided by the form. (These are called nama and rupa in the Hindu schools.) The form and matter of a thing are not a dualism, since they both constitute the thing; they are nondual.

\paragraph{Animism}
\begin{quotex}
There can in fact be no `inanimate'\footnote{The terms ``psychism" and ``animism" are synonymous (one deriving from Greek, the other from Latin), so we may use them interchangeably.} objects in existence, and also that `life' is one of the conditions to which all corporeal existence without exception is subject. \flright{\textsc{Rene Guenon}, \textit{The Reign of Quantity}}

\end{quotex}

Scientific materialism has no possible explanation for consciousness, since consciousness cannot be measured, counted, or weighed. Similarly, science has no precise definition of life nor a clear understanding of how it arose. The traditional view, on the other hand, understands life to be one of the conditions of manifestation, from the beginning, so no such explanation is necessary. Although the material part of a thing is quantity alone, its qualities are experienced in the psychic element. Hence, there is nothing that is purely mechanical.

The corporeal or sensible world then integrates the material part of a thing with its sensible elements, ultimately derived from its essence. Hence, there is no ``external" world, but rather a world that has both an inside and an outside.

This does not mean, however, that your automobile or android is conscious, since they are artifacts without an essence. On the other hand, what we consider to be natural phenomena may conceal something deeper. In earlier times, the following description would have been perfectly natural:

\begin{quotex}
In flowing and running water, in mists dissolving into water, also in the winds and the lightning flashing through the air, in all these, you have to look for the physical body of Angelic beings. \flright{\textsc{Rudolf Steiner}, \textit{The Spiritual Hierarchies}}

\end{quotex}
\paragraph{The Divided Line}
The physical world does not create mathematics, yet it follows mathematical laws. It is obvious that mathematics cannot be derived from matter. The best physicists, like \textbf{Roger Penrose}, recognize this. Yet that is a half-step; beyond the maths, there is the form or essence.

The best attempt to understand forms in terms of matter is the notion of ``supervenience"; that is, the claim that the whole depends on the arrangement of its parts. But that just begs the question. Who or what recognizes the whole? The philosopher \textbf{David Lewis} offers this example:

\begin{quotex}
A dot-matrix picture has global properties — it is symmetrical, it is cluttered, and whatnot — and yet all there is to the picture is dots and non-dots at each point of the matrix. The global properties are nothing but patterns in the dots. They supervene: no two pictures could differ in their global properties without differing, somewhere, in whether there is or there isn't a dot.

\end{quotex}
A more contemporary example is a JPEG figure: mathematically, it is just a sequence of binary digits. However, ``globally" it is an image of some object. So, yes, the image does depend on the bits, i.e., quantity, but its global property is a quality experienced by a consciousness.

The traditional view is that the whole creates and arranges the parts, not the other way around.

\paragraph{Panpsychism}
It is interesting that secular philosophy has been warming to a similar idea under the name ``panpsychism". Unfortunately, it is seldom properly understood, especially by those who see psychism or animism only as a property of ``physicalism". For reasons already stated, that is not even possible. At least there is the recognition that consciousness and qualities cannot be explained in terms of matter.

The most reasonable attempt, at least in profane philosophy, is that of \textbf{Timothy Sprigge} who integrates a philosophy of Absolute Idealism with panpsychism\footnote{\url{https://onlinelibrary.wiley.com/doi/abs/10.1111/j.1468-0149.1985.tb01122.x}}. There are physicalist philosophers who try to claim that consciousness is itself a property of matter. In poker that is called ``being married to your hand". You convince yourself that your hand is a winner, although objectively your hand will be a loser. Here are two examples, the second worse than the first

\textbf{Philip Goff}\footnote{\url{https://www.theguardian.com/books/2019/dec/27/galileos-error-by-philip-goff-review}} interprets psyche as a property of matter. \textbf{Galen Strawson}\footnote{\url{https://en.wikipedia.org/wiki/Galen_Strawson}} claims that the mental/experiential is physical. Of course not, since the mental cannot be measured, counted, or weighed. As an aside, Strawson the younger denies that there is moral responsibility. That is not a legitimate philosophical position, but rather a characteristic of a psychopath.

Three Nobel prize winning physicists come closer to the right idea:

\begin{itemize}
\item \textbf{Ernst Schrödinger}: the material universe and consciousness are made out of the same stuff 
\item \textbf{Louis de Broglie}: I regard consciousness and matter as different aspects of one and the same thing 
\item \textbf{Max Planck}: I regard consciousness as fundamental 
\end{itemize}

\flrightit{Posted on 2020-06-29 by Cologero }
