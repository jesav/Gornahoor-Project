\section{Everything else is a Pastime}

\begin{quotex}
One can call the whole of experience false, illusory, nonexistent — but whoever experiences and asserts this falsity, illusion, nonexistence cannot himself be false, illusory, nonexistent. Beyond the obliqueness and fluctuation of “things that are and are not”, there is then one single certainty: the “\emph{I}". Only here the individual, with possession, has an absolute and self-evident reality. \flright{\textsc{Julius Evola}, \textit{The Individual and the Becoming of the World}\footnote{\url{https://www.gornahoor.net/?p=187}}}

\end{quotex}
New Age spirituality claims that if we rise — or better said, evolve — to our “higher selves”, our lives will be full of Light and Love, we will all be “One”, there will be no judgment of morals or lifestyle, and we will manifest abundance, perfect health, wonderful “relationships”, and so on. New Agers furthermore claim that the world is in our consciousness and if we would only change our consciousnesses, the world would instantly improve. With such a wonderful sounding and easily accessible agenda, it is curious that the results are so meager.

The New Age is actually based on distortions and misunderstandings of Tradition. The common view raises some unanswered questions.

\begin{itemize}
\item What is consciousness? 
\item If all is consciousness, then who brings about the change? 
\item How does he bring about the change? 
\end{itemize}
\paragraph{Consciousness}
\begin{quotex}
Existence is only real when it is conscious to somebody. \flright{\textsc{Carl Jung}, \textit{Answer to Job}}

\end{quotex}
Consciousness refers to what we are aware of, either virtually or actually, that is whatever is experienceable. In other words, consciousness is the phenomenal world. What we call the objective or external world is what we experience through the senses: sight, hearing, touch, smell, taste. This is the realm of “becoming”. Clearly, then, a change in consciousness is tantamount to a change in our world, that much is true. However, unlike New Age teachings, there is not a “cause and effect” relationship, and it is certainly not so simple as it sounds.

Besides the external world, there is an inner world of experiences, such as emotions, thoughts, imagination. What New Agers really mean is that a change in imagination will result in a corresponding change in the external world. This is not a cause and effect relationship, but considers that the “universe” or the “subconscious” will somehow bring about the change. Clearly, this is impossible, or else everyone would win the lottery every week, or every teenage boy would be having a relationship with Jessica Alba.

\paragraph{Atman}
\begin{quotex}
The “I”, in fact, is not a thing, a “given”, a “fact”, but, essentially, a deep centre of will and power. \flright{\textsc{Julius Evola}, \textit{The Individual and the Becoming of the World}}

\end{quotex}
The next question is “who desires, initiates and brings about the change.” If the world is phenomenon, then the “who” must be outside phenomenon, that is, noumenal, not a “thing”. The “who” is the constant in every act of consciousness, which is in perpetual flux. As such, it is not part of the world, but rather the silent observer of all that is. This is unobservable to scientists, who must necessarily deal with phenomenon. For the philosopher, it can only be an issue for debate, ultimately unresolvable. But for the metaphysician — and this is what distinguishes him from the philosopher — it as a definite and knowable state. In Tradition, this state has been called, \emph{inter alia}, the Unmoved Mover, the Observer, True Will, or Atman.

\paragraph{Will}
\begin{quotex}
The explanation that \emph{magical idealism} demands is completely different: it is \emph{an explanation by means of action, a resolutive explanation}. It is to ex-plicate, or to actuate, to make perfect: to make what is in potential pass into act, what is imperfection into perfection, what is insufficiency into sufficiency, according to a synthetic, creative, primordial process. This is the only true explanation. \emph{Everything else is a pastime}. \flright{\textsc{Julius Evola}, \textit{The Individual and the Becoming of the World}}

\end{quotex}
The next question is how change comes about in the world of becoming. This is the answer to the question of the sufficient reason for existence, or how does essence (the idea) become existence (the thing or situation). Now the idea may be experienced in the imagination, but at a deeper level, the idea is known directly by a process of intellectual intuition, beyond any sensual imaginings. But it is the Will that actualizes the idea in Existence. For the Sage, this Willing is conscious, free, and deliberate. Otherwise, it is unconscious, determined, and spontaneous. As unconscious, the source of the Will is projected onto some other entity, such as the “universe”.



\flrightit{Posted on 2010-08-05 by Cologero }

\begin{center}* * *\end{center}

\begin{footnotesize}\begin{sffamily}



\texttt{Matt on 2010-08-05 at 23:51 said: }

Funny, I was just reading critiques about the relativistic chaos that post-modernism and structuralism/deconstruction falls into and I was thinking to myself I wonder if the contributors at Gornahoor will do a critical post of those views and sure enough you have a post critiquing something that is in many ways similar to those movements. A good post as usual mixed in with some of your good sense of humor! Maybe you'll do a post critiquing those movements in the near future, though I guess you touched on them a bit with your Descartes' nightmare essay.


\hfill

\texttt{Matt on 2010-08-06 at 01:06 said: }

Also, I suppose it would be fair to say that the new-agers also make the mistake of confusing the Self/Absolute Principle with the cosmic process, rather than acknowledging what tradition affirms, which is that the cosmic process – the world process is I guess a more accurate statement since it is not yet a harmonious ordered unity (cosmos) but a disharmonious unity – is part of the “I”, but not the sum total and whole of the I. I think Evola made a good metaphor for the world process and all its multiple states as the I's body of experience.


\hfill

\texttt{Liz on 2010-08-06 at 11:23 said: }

I'm finding that the more I read the less I know — very frustrating, I might add. I also think it's good to open your mind to new thoughts and new theories — want to recommend “Sun of gOd” by Gregory Sams. Look at it this way — the secret of “The Secret” was in recognizing that we live in a responsive Universe. “Sun of God” helps us understand why this is the case, which is unaddressed in “The Secret” itself.


\hfill

\texttt{Will on 2010-08-06 at 11:49 said: }

In my opinion, most of the `post-modernist philosophers' are not worth writing about. Most of them have the annoying habit of taking an idea that could be expressed in a single paragraph and writing a 200 page book about it.

Jean Baudrillard made an interesting critique of Marxism back in the seventies, revealing it as the flip-side of capitalism, but Evola said as much back in the 30s, and said it better. Paul Virilio's works are an interesting critique of technology, but if you've read one, you've read them all, and furthermore, the language is very obtuse. Michel Foucault's works are somewhat worthwhile if you have an interest in history, but his ideas and concepts have mostly been used by the left, as he himself was a far-leftist. Not that that invalidates his ideas in and of itself, but it's a bias that comes through. Pretty much all of these guys are coming from a post-Marxist, post-Freudian, Frankfurt School perspective.

Jacques Derrida is, in my humble opinion, utterly worthless. If you want to understand `deconstruction,' study Nagarjuna.

The only one of the postmodernists I can recommend – and with MANY reservations – is Gilles Deleuze. His book on Nietzsche is excellent, and the introductory essay to A Thousand Plateaus – “Rhizome” – is a worthwhile piece in the way that it outlines two different modes of conceptual thinking. But here again, there is a huge left-wing bias in most of his work, and furthermore, he likes to invent his own language (like Heidegger) and this makes studying him a very labor-intensive process.


\hfill

\texttt{Matt on 2010-08-06 at 15:10 said: }

Will,

Yes, I'm aware of those names and what they believe, and I agree, none of them are worthwhile (besides Nagarjuna of course). Its not just the left-wing bias, but since they believe pretty much everything is a social and linguistic construct, their writings lead to that relativistic chaos I referred to in my earlier post. None of them can really give their own answers to a fundamental question of what is “being” because they define themselves and the rest of humanity by what they are not.


\hfill

\texttt{Tosti on 2010-08-06 at 17:50 said: }

About Nagarjuna-agreed. I'm curious as to your take on the Traditional writers, specifically Schuon?


\hfill

\texttt{Will on 2010-08-06 at 19:08 said: }

Matt, I think you hit the nail on the head in raising the “fundamental question of Being.” The postmodernists all take their cue from Nietzsche. Interestingly, so did Evola, though they went in opposite directions. Whereas Evola saw that Nietzsche's philosophy represented a confused attempt at transcendence, and then sought to remedy the defects in his own work, the postmodernists go with Nietzsche's rejection of Being in favor of becoming, which from a Traditional standpoint is an error.

Beneath Nietzsche's criticism, however, was a deep spirituality which, in my opinion, failed to find a satisfactory form of expression. I don't think the same can be said for most of the postmodernists, who seem bogged down by nihilism and relativism in a way that Nietzsche was not.


\hfill

\texttt{Will on 2010-08-06 at 19:12 said: }

Tosti, I must confess almost total ignorance in regards to Schuon's work. My favorite writers among the first generation of Traditionalists are Ananda Coomaraswamy and Julius Evola.


\hfill

\texttt{Tosti on 2010-08-07 at 08:42 said: }

Will,

Evola, of course. And Coomaraswamy has been most enlightening. His Symplegades (I have Guardians of the Sun-door), lays it out quite nicely. What a wonderful guidepost! Schuon has influenced me quite a bit, casting light on my understanding from perspectives I hadn't dreamed. He was possessed of both a marvelous intellect and a passionate soul. Highly recommended. My only regrets are that I can find only a few individuals in my area who are familiar with these writers.


\end{sffamily}\end{footnotesize}
