\section{A Million to One}

\begin{quotex}
When reason comes out against the reality of life and knowledge with a consciousness of its own supreme rights, it finds
that everything in life is alien, dark, and impenetrable, and it cannot do anything with it. \flright{\textsc{Vladimir
Solovyov}, \emph{Lectures on Divine Humanity}}

\end{quotex}

The British mathematician \textbf{John Littlewood} claimed that everyone might experience a “miraculous” event every
month or so. He assumed that we experience an event every second, which may be exceptional or unexceptional, so that
there will be one million events every 35 days. A miracle is defined as an event with a probability of one in a
million. Somehow, the conclusion is that there are no exceptional events. This has become known as
Littlewood's Law\footnote{\url{https://en.wikipedia.org/wiki/Littlewood's_law}}.

\paragraph{Queen of Hearts}
As an example, someone recently proposed that there should be nothing surprising in being dealt 13 hearts in a game of
Bridge. The reason is that the sequence of cards is no more unlikely than any other sequence of cards. The human mind
simply does not recognize that other sequences are also equally exceptional mathematically.

However, a Bridge player is not concerned with the specific order of cards, just with the number of cards in each suit.
Arguments like that are not uncommon, but their flaw is failing to distinguish permutations from combinations. The
former takes the specific order into account, but the latter views the cards as aggregates.

For example, there are 13! sequential ways to get 13 hearts. That is a large number, but trivial in respect to the
3,954,242,643,910,000,000,000 possible permutations of Bridge hands.

Ignoring the order, there are only 635,013,559,600 possible bridge hands, or combinations. Therefore, a hand with 13
hearts is just one out of 600 million possible hands; that is what makes it unusual. There are many, many more hands
with 3 or 4 hearts, regardless of the order.

It is true that each permutation of cards is equiprobable, but that makes it useless in terms of information content.
Combinations, on the other hand, are information rich and compact, since each hand can be expressed by the number of
cards in each suit.

\paragraph{Lack of Knowledge}
In the above example, if every living human were dealt 100 bridge hands, then there is a very high likelihood that one
of them will be 13 hearts. Uncertainty may result from a lack of knowledge or from an inherent random process\footnote{\url{https://en.wikipedia.org/wiki/Randomness}}.

In the example, we know the rules of Bridge, so we interpret the 13 hearts in that context; a hand with 13 hearts is a
legitimate outcome. But someone, not in the game, who unexpectedly finds a hand with 13 hearts may interpret it
differently. Actually, the sequence of cards is predetermined in the deck; those are “hidden variables”. Therefore, the
cards should be reshuffled after each card is dealt, to eliminate that effect.

\paragraph{Radioactive Rocks}
Radioactive decay is, apparently, inherently random. A physicist cannot determine in advance which particular atom will
decay. Nevertheless, within a 24 hour timeframe, he can predict how many atoms will have decayed. This is another
example of permutations — the order of decay — versus combinations. The combinations provide
useful information about the half life of the element.

This is an example of how totally random events can reveal a pattern.

\paragraph{Abandon Hope}
Suppose you find yourself wandering aimlessly in the forest of life. Every so often you see a sign or a billboard with
Latin letters on it, but they don't seem to relate to any known language. Eventually you come across the
mouth of a cave with this sign above it: “\emph{Abandon hope, all ye who enter here.}”

The hard nosed scientist on your left shoulder tells you, “You can ignore that sign. The sequence of letters has the
same probability as all the other signs that we have seen. It is totally unexceptional, just one event in a million.”

Your spiritual guide on your right shoulder objects: “That warning cannot be random so it must be a serious message.
Under the circumstances, you need to walk around that cave.”

\paragraph{Search for Intelligent Life}
The cave sign is an example of specified complexity\footnote{\url{https://en.wikipedia.org/wiki/Specified_complexity}}. The concept is still valid despite its bad reputation due to its
association with intelligent design. Critics point out that the concept has no formal mathematical definition. Of
course not, because specified complexity has meaning only to an intelligent observer. Mathematically, it is just
another pattern.

However, the SETI project\footnote{\url{https://en.wikipedia.org/wiki/Search_for_extraterrestrial_intelligence}} scans electromagnetic radiation from outer space looking for patterns, i.e., complexity, that
would indicate an intelligent source. At the same time, scientists are broadcasting patterns into space in the hope
that extraterrestrial intelligences will recognize them.

\textbf{Stephen Hawking} believes that it unwise to alert extraterrestrial intelligences to our existence on the grounds
that they may destroy us. On the other hand, extraterrestrial intelligences would likely have discovered
Littlewood's Law, so that they would interpret signals from earth as simple random sequences, of little
significance.

\paragraph{Living in the Real World}
Random events are unrelated to each other. But is that true, as Littlewood's Law requires? In that case, we
are forever doomed to experience the world as a “blooming, buzzing confusion”, as babies do according to
\textbf{William James}. But that is not the case, not even for babies. The world is intelligible so we look for
patterns, combinations, and specified complexity in the events of our experience. If we sometimes get it wrong or
misinterpret events, that does not invalidate intelligibility.

Even events that are not causally related can be meaningfully related, as in synchronicity\footnote{\url{https://en.wikipedia.org/wiki/Synchronicity}}. Is that purely subjective?
So what? Aren't your dreams, your plans, your desires also subjective? That does not make them unreal;
moreover, your life is desiccated without them. Obviously, such experiences cannot be accounted for in mainstream
science, yet have a place in revolutionary science\footnote{See Section~\ref{sec:202207_07A Revolutionary Kind of Science} of this book.}. In practice, randomness may be hard to distinguish from a pattern.
But like the dark side of the moon, the real meaning of an event is usually hidden from plain view.

For event \#1 million, suppose you observe an electron suddenly changes its spin. Is that random, or is it because the
electron is entangled with another one, a million miles away? How does that affect your next choice?

\begin{quotex}
Among the men and women the multitude,\newline
I perceive one picking me out by secret and divine signs,\newline
Acknowledging none else, not parent, wife, husband, brother, child, any nearer than I am,\newline
Some are baffled, but that one is not — that one knows me. 

Ah lover and perfect equal,\newline
I meant that you should discover me so by my faint indirections,\newline
And I when I meet you mean to discover you by the like in you. \flright{\textsc{Walt Whitman}} 

\end{quotex}

\flright{\itshape Posted on 2022-07-19 by Cologero}
