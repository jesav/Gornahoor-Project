\section{The Nature of Things}

\begin{quotex}
Now I a fourfold vision see\\
And a fourfold vision is given to me\\
Tis fourfold in my supreme delight\\
And three fold in soft Beulah's night\\
And twofold Always.\\
\emph{May God us keep}\\
\emph{From Single vision \& Newtons sleep}. \flright{\textsc{William Blake}}

The height of poetry is to speak the language of the gods. \flright{\textsc{George Santayana}}

The greatest art in theoretical and practical life consists in changing the \emph{problem} into a \emph{postulate}; that way one succeeds. \flright{\textsc{Goethe}}

The first \emph{causa} which occurred to his mind in reference to anything that needed explanation, satisfied him and passed for truth. \flright{\textsc{Friedrich Nietzsche}}

\end{quotex}
At a recent gathering of some old friends, one of them, the most literary of the bunch, recommended that I read \emph{The Swerve: How the World Became Modern} by \textbf{Stephen Greenblatt}. It is allegedly about the attempt in the Renaissance to recover a lost version of \textbf{Lucretius}'s \emph{Nature of Things}. Then the argument is that the discovery of that poem then ``created" modernity. To dramatize the event, she claimed that the poem was ``suppressed" as it was a ``threat" to the world order.

When I hear words like that, I already know that the book is tendentious; therefore, I immediately resolved not to bother with it. Nevertheless, in a flash, the following essay came into my mind. Now she might have been able to follow it, but I sadly realized that no one else sitting around the table would have had the least interest.

Now ``threatening" ideas have a certain erotic appeal to literary types, like the way the good girl often prefers the proverbial bad boy biker. Unfortunately, once understood, Lucretius turns out to be quite melancholy. According to St. Jerome, he committed suicide. Some doubt Jerome's account, although it is quite consistent with his philosophy: one has the right to choose the time and manner of death, since it is ultimately inevitable.

Melancholy is thus hidden at the center of the modern world. It is evaded through the pursuit of material goods and ever more exotic pleasures. Death is denied even though popular culture is replete with stories of death and murder. Nevertheless, the West is dying, and through its own choosing. Instead of admitting to suicide, the cause is sought elsewhere: e.g., climate change, the patriarchy, racial degeneration, and so on.

So Lucretius is hardly an unmixed blessing. Rather than being ``suppressed" as a ``threat", the actual situation is that the Medievals were not attracted to, nor interested in, the philosophical positions espoused by Lucretius. For them, the cosmos was a moral cosmos, with forces of light and darkness in combat with each other. Moreover, the movements of the atoms were neither blind nor random, but rather were directed by transcendent forces. Lucretius had nothing to offer.

An example might help. Cows cannot study philosophy because they graze all day with their nose to the ground. Man has long intestines, so he doesn't have to eat all day. The atomist, like Lucretius, would attribute that difference to random variations. The Platonist, however, believes that the purpose of long intestines is to enable man to study philosophy.

Philosophical disputes are usually unresolvable, due to different starting points. What we can do instead is to examine the suppositions of a system and follow where it logically leads. Now most people are inconsistent in their thought patterns, and are often reluctant to accept the necessary conclusions of their worldview. Nevertheless, the spirit will follow that logic, even if it requires centuries to unfold. The full consequences of the birth of modernity are yet to be experienced.

\paragraph{The Hermetic Method}
The Hermetic method is not to try to refute one theory with another, but rather to arrange all knowledge in a more comprehensive system. Hence, there are aspects of Lucretius's vision that we can sympathize with. To avoid Single vision, I rely somewhat on \textbf{George Santayana}'s little book: Three Philosophical Poets: Lucretius, Dante, Goethe. He calls them respectively the Poet of Naturalism, the Poet of Supernaturalism, and the Poet of Romanticism.

Although Santayana ultimately sides with Lucretius, the situation is more complex. That is because we accept the one thing that Santayana denies, viz., the existence of other worlds. Santayana considers the possibility that after death we might wake up into a world in which the atomic philosophy does not apply. He rejects that idea as an idle fantasy, but Hermetists do not. In the lower world, the one created by man, the atomic philosophy does indeed apply. But Dante and Goethe do reveal higher worlds.

\paragraph{Foundation}
Lucretius is the third in a chain that began with Democritus and continued with Epicurus. They embody the philosophy of atomism, materialism, and naturalism, which I shall use interchangeably. Epicurus developed an ethical system based on naturalism, called appropriately Epicureanism. The discovery of Lucretius's poem in the 15th century did bring to light the doctrine of atomism. Insofar as the foundation of the modern world rests on naturalism and a vague humanism as its ethical system, perhaps we can view Lucretius as the harbinger of modernity.

However, if Lucretius was indeed a threat 500 years ago, that is not at all the case now. Rather the threat comes from those few who understand the crisis of the modern world and are in revolt against it.

\paragraph{Atomism}
The fundamental idea is that the world is made of space and atoms of matter. These atoms are in motion through space, arranging themselves in patterns, which account for our experience of things. These patterns arise and decay, so there is life and there is death. Empedocles identified these forces as Love and Strife. Lucretius calls them Venus and Mars, and Sigmund Freud, Eros and Thanatos.

Since the life instinct is so strong, the corresponding death instinct is not always taken into account. Epicurus regarded the denial of death as the root of human neurosis, so the cure is to come to terms with the end of both the life and the soul at death.

The elusive third force is necessarily left out of any philosophy of naturalism, so there is no room here for the Holy Spirit and the reconciler of life and death.

\paragraph{The Soul}
Even the soul is material. Of course, for Tradition also, the animal soul is material, so the human soul is material inasmuch as it comprises both animal and vegetable souls. Hence, for the atomist, the human being is just a type of animal; that is likewise the educated opinion of modernity. Nothing distinguishes essentially human from animal life; the former is merely a more complex arrangement of atoms compared to the latter

Animal life is limited to a few main features: eating, mating, caring for young, avoiding pain and death. Those activities are pre-philosophical, that is, no argument is necessary to engage in them. On the contrary, bad arguments are used to evade animal life, such as reproductive sex or raising children.

The Traditional view is that the human being is more than a material, animal soul. He is a spirit. Santayana understands that perfectly well, when he writes:

\begin{quotex}
A spirit would in any case not be human, but altogether divine.

\end{quotex}
Precisely. Man is either an animal or a god … or both.

\paragraph{Modes of Experience}
There are four modes of our experience of the world: touch, vision, thinking, and consciousness. Tradition ranks that list from least reliable to most reliable, while for naturalism, it is just the reverse.

\subparagraph{Touch}
Democritus asserted that materiality consisted of extension, figure, and solidity. That is, materiality is whatever can be touched. This leads to quantity as the only objective measure of reality. By focusing on these primary qualities, this allowed Galileo to simplify his observations. Galileo readily adopted atomism in his works. That story is ably described by \textbf{Henri Bortoft} in \emph{The Wholeness of Nature}. There is no space there to review Bortoft's account, but it is necessary to focus on his main point. Bortoft shows that Galileo did not \emph{prove} atomism based on his empirical research; rather, he simply assumed it, and creatively drew out its consequences. In other words, he turned the problem into a postulate.

\subparagraph{Vision}
Vision, broadly speaking, refers to all the senses that don't require contact (e.g., hearing, smelling). For the atomists, all qualities beyond materiality are apparent only, and are imputed to things merely by convention. This includes, at a minimum, light, colour, taste, warmth, beauty, excellence, and so on. Only space and matter are real.

Obviously, such a teaching reduces the most important aspects of life to arbitrary conventions. Somehow or another, the ``good news" of atomism has still not fully penetrated modernity. People still insist on arguing endlessly about such ``secondary qualities", simply forgetting that only space and matter are real.

\subparagraph{Thinking}
What should the atomist think about? I suppose he could be a scientist, thinking about space and matter. That brought a certain amount of excitement for several generations of scientists beginning with Galileo and Newton. But science falters when dealing with social life.

For the common person, the thoughts of the physicist are much too difficult. Hence, they are not reproducible in his consciousness. For the Platonist, however, science is easy. The higher realms of human knowledge—politics, metaphysics, theology—are much more difficult. The common person disagrees. Science is indeed hard, but everyone believes his ideas of political issues, philosophy, religion, and God are somehow correct. As Nietzsche points out, humans are easily satisfied with the first explanation that pops into their mind. That is why, for most people, their views on such topics are not much different from what they believed at the age of 19.

\subparagraph{Consciousness}
For the atomist, the soul is the life of the body, and consciousness plays no role. At best, it is an epiphenomenon, and at worst, a delusion. Let this sink in. Our self-identity as a person is due to our conscious experience. However, for the materialist our identity is in the brain which secretes thoughts independent of any belief in an individual will. A leading new atheist assures us that the ``self" is an illusion, if ``assurance" is the right word.

Illusion or not, that is our experience. So the atomist position in a nutshell is this: reality is the random motion of atoms in space, which arrange themselves in patterns, only to have them dissipate. Consciousness is merely the passive, and helpless, observer of that drama. The delusional self is therefore trapped in conscious experience, whether good, bad, or indifferent. This brings us to the Gnostics who rebelled against such a view. They believed that a demon had created the world described by the atomists, so they hoped that the real God would rescue their spirit soul from that trap.

The modern man has lost even that hope. He believes that distractions such as Netflix or TV, along with psychoactive drugs, medically prescribed or not, will save him.

\paragraph{Morality}
The question arises for modernity: if atomism is true, how then should we live? Epicurus developed a moral system consistent with atomism which goes by the name of hedonism. If Lucretius was melancholic, Epicurus hated life. Hence, unlike modern misconceptions, the hedonist sought only simple and moderate pleasures. Withdrawn from social life, he sought the company of a few close friends in his garden.

Santayana held the contrary view. He believed the materialist should love life. Perhaps so, since it made no moral demands on him. Of course, Santayana could believe that since he was blessed with time, money, and intelligence. For many others, their material life holds few attractions, but is mostly hardship, struggle, and lack. I suppose that a saintly hedonist could be indifferent to the atomic world process, but that is hardly the same as ``loving" it.

\paragraph{Human Society}
Modernity has taken the atomic idea one step further. Applying atomism to humanity, the postulate is that each human being is an atom. Therefore, people are quantitative and any attribution of secondary qualities is purely subjective and conventional. The obvious corollary is that qualities such as race, ethnicity, and even gender, are socially constructed. This is indisputable since it is a postulate, not a theory requiring proof.

Leibniz had a similar, yet radically different, conception. He called the atoms ``monads", each of which has its own qualities. This is the polar opposite to Democritus. The monad is real, but space, matter, and motion are merely phenomenal. Modernity has preferred to follow the ideas of Newton rather than those of Leibniz.



\flrightit{Posted on 2018-01-21 by Cologero }
