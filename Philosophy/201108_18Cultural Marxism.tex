\section{Cultural Marxism}

\begin{quotex}
As for the modernists, it is necessary to invert the positions and even tell them, ``they are the latecomers"; that up until their futurism it is a traditionalism or it is condemned to a short expiration date to become such: that which they consider as the future being only of rather elementary and infantile forms of traditions or liberation, in respect to a truly constructed new epoch, whose creator will be neither the petty bourgeois reactionary nor the promoter of cultural Marxism and the new art, but rather the man of the living Tradition. \flright{\textsc{Julius Evola}, \textit{Sui limiti del bolscevismo culturale}, La Vita Italiana, February 1938}

\end{quotex}
Long before anyone had heard of the Frankfurt School, \textbf{Julius Evola} wrote about \textbf{cultural Marxism}\footnote{\textit{Bolscevismo culturale}}. In this essay, which I am summarizing, Evola distances himself with his youthful, pre-Guenon, flirtations with Dadaism, futurism, and the anarchism of Max Stirner. This is his definition:

\begin{quotex}
The expression cultural Marxism designates a whole of manifestations in the fields of literature, art, thought, science and, in general, culture which, even when they lack any relation whatsoever with Marxist and communist political propaganda and even when their creators or promoters don't think they are exercising any political influence, equally measured by their direct or indirect effects, preparatory forms and ferments of ethical and spiritual decomposition are revealed, and their ultimate consequence can only be Marxism.

\end{quotex}
\paragraph{Spiritual Decomposition}
The expression ``cultural Marxism" is very apt since it tacitly implies the conviction that political Marxism would not be possible without a preliminary disruptive action in the heart of culture where, with a sure glance, the germ of Marxist evil is identified even where it still acts only in abstract domains, exterior to the specifically social, political and governmental spheres. But these same germs, that is, cultural Marxism in its contemporary forms of manifestation, are considered more as visible symptoms than as the ultimate cause of the evil.

The undoing of an organism with the liberating of the forces that it had held in a unity and with the appearance of new forms of life tied to its putrefaction is not the cause of its death, but its effect, standing at the center a fact that is spiritual at its base. Equally things stand about what we can call the culture of decomposition and this analogy is not only artificial but offers an important base for the considerations that we intend to develop here: one of the most characteristic aspects of the decomposition is the passage to the free, elementary and impersonal state of force and processes, which, before the crisis of death, had obeyed a higher law and figured in the synthesis proper to knowledge and will of a ``person".

\paragraph{Social Aspect}
There is, first of all, a social aspect consisting in tendentiously emphasizing all the cases in which ideals are presented as illusory, values as pure convention, institutions as unjust and defective. Corresponding to that, this culture emphasizes the higher right of life, instinct, the irrational, the purely ``human" and, in a certain special domain, creates the new superstition of the subconscious, the unconscious, the libido and primigenial eros as the true root of existence and the human person. It is absolutely superfluous to name names. The theatre, the novel, the new psychology, even certain forms of biography and interpretations of history, in any case, cinematography and even the economy are today the organs of action and the dissolving influence in this sense.

\paragraph{Moral Aspect}
This social aspect is naturally inseparable from the moral aspect of cultural Marxism. The constitution of love and sexuality in forms of true obsession is a fundamental point and not just the direct effects should be considered, which are, for example, practical immorality, the sensualization of life, the breakup of the family, the debasement of the institution of marriage to a social convention and utilitarian contract, revocable by a stronger claim of instinct or temperament, and so on. But we must also consider the indirect effect of the reaction, constituted by moralism, whose attention, if only from the opposite side, is equally obsessed with eros and no longer manages to recognize any truly higher point of view.

\paragraph{Main Forms}
\begin{quotex}
Important forms of cultural Marxism are the collectivization of life and sensibility, mechanization and finally the unleashing of elementary forces. There in certain of its aspect, these tendencies lead to an anarchist liberation of the Self and subjectivity, in other of its forms we see a veritable destruction of the human element, of the feeling of all that is individual and tied to its own proper quality. In a certain measure, it is about the repercussions of an environment in which technique has determined, close to a fatal premise of standardization and leveling, the principle forms of life, sensibility, and taste.

\end{quotex}
\paragraph{Decadent Art}
Alongside technique, Evola mentions modern art. Evola points to modern music which

\begin{quotex}
puts the pure symphonic element over the melodic in a way that leads either to the unleashing of pure rhythm (Stravinsky) or the breakup of so-called atonal music (Schoenberg)

\end{quotex}
As for pop music, he singles out African influenced music, or more generally, the savage element, which involves not just the snobbish connoisseurs, but

\begin{quotex}
rather the great masses of the entire world, which in the syncopated forms of rhythm, in the motifs of jazz where precisely the essential is formed by the primitive, by pure sound texture in forms of an automatic and obsessive repetition, we find rather more than in any of their traditional music.

\end{quotex}
How that would apply in particular to pop music is left to the reader, much of which is pure rhythm and no melody, even among those who claim Evola as an inspiration. In pictorial art, he points to the ``liberation of"

\begin{itemize}
\item pure color (post-impressionism) 
\item pure form (cubism) 
\item pure dynamism of pure sensation (futurism) 
\item pure incoherency (Dadaism) 
\end{itemize}
As for architecture,

\begin{quotex}
the elementary and rational are associated in a single front that chooses for its ideal the functional style of the machine and tries to liberate constructive forms from every residue of decoration, of traditional estheticism, and of the artistic will.

\end{quotex}
\paragraph{The Traditional Conception}
In order to understand the deleterious effects of cultural Marxism, Evola first makes clear the basis of Tradition.

\begin{quotationx}
According to the traditional conception, every organization is based on a dualism, on a polarity:

\begin{itemize}
\item
on one side there are the forces of the \textbf{cosmos}, that is of order, form, conscious personality; 
\item
on the other side, there are the forces of \textbf{chaos}, unleashed nature, instinct, the subpersonal. 
\end{itemize}

The symbolism of the ancients attributed to the former a divine, luminous, heavenly character and to the latter, instead, a dark and demonic character. The organization needs both these forces, however in a combination according to which the forces of chaos are constrained, ordered, formed and liberated from their primordial nature by the forces of the cosmos.

Such a transformation has various levels, and such levels correspond to as many planes and, in certain civilizations, to as many castes of the hierarchical ordering. At the highest point, this transformation has the human personality appear directly as the virile, regal, and dominating carrier of pure spiritual authority. At the next plane, there is already a residue of elementariness, but it receives the form proper to the warrior, the aristocrat, the military chief: it is the warrior-aristocratic caste. Still lower, there is the bourgeoisie and finally the working mass, which, in this traditional order, also has its ritual mode of living that, for example, is clearly visible in the ethical and even religious character of the ancient guilds.

\end{quotationx}
\paragraph{The Essence of Marxism}
\begin{quotex}
The essence and meaning of Marxism and communism is lost on those who do not know how to see in it not an isolated phenomenon and a type of unexpected perversion, but rather the last episode of a series of destructions and subversions that, for the successive levels, have overwhelmed the higher formations of the general hierarchical ordering and, reaching the plane of the masses, have led them to a definitive liberation from every superior principle and there to a unleashing that had to make its immediate expression of a bursting out of the substrate of chaos, of elementariness, and of an obscure materiality that previously had been constrained and formed by higher forces. And so that a purposeful light rises on the fact that in Marxism the ultimate products of the western ideological disintegration meet with the primitivistic, confused substance of a people who, like slaves, have never had the benefit of a true formation and traditional articulation. 

\end{quotex}


\flrightit{Posted on 2011-08-18 by Cologero }

\begin{center}* * *\end{center}

\begin{footnotesize}\begin{sffamily}



\texttt{logres on 2011-08-21 at 20:23 said: }

I wonder if Gramsci had read Evola?


\hfill

\texttt{I on 2011-08-22 at 14:19 said: }

When a tree grows old, it begins to rot from the inside. Many trees seem to stand to old age even when they are already dead within. When a tree finally falls and starts to decay in the soil, moths and other forms of dissolutive agents appear to finish the tree. The most dangerous form of a tree is the one that stands even while it is wholly rotten inside, and a friend of the whole will support and even help to hack it down before it falls destructively and uncontrollably.

A new tree rises only after the process of dissolution is finished, when moths and fungus' disappear. Some forms of fungus live in harmony with a tree. A reasonable amount of destructive \& dissolutive agents is an essential part and a requirement for the health of the whole biosphere.


\hfill

\texttt{Logres on 2011-08-22 at 21:17 said: }

Doesn't most cultural Marxism masquerade as a ``new whole"?


\end{sffamily}\end{footnotesize}
