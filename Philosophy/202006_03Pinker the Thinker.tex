\section{Pinker the Thinker}

\begin{quotex}
So, how many people know how to observe? And of these few, how many to observe themselves? `Everyone is farthest from himself’ — every person who is expert at scrutinizing the inner life of others knows this to his own chagrin; and the saying, `Know thyself'. addressed to human beings by a god, is near to malicious. \flright{\textsc{Friedrich Nietzsche}, \emph{The Gay Science} §335}

\end{quotex}
The NY times published an article about a year ago on \textit{Why the Fight Over Abortion Is Unrelenting}\footnote{\url{https://www.nytimes.com/2019/05/29/opinion/abortion-restrictions-politics.html}}. The answer is postulated in the first paragraph:

\begin{quotex}
Why is the debate so bitter, so emotional? Part of the answer is very simple: the two sides share almost no common premises and very little common language. 

\end{quotex}
Our intent, therefore, is not to debate the morality or legality of abortion, but to explore the lack of common premises and language. Nevertheless, prior to 1973, there was a long stretch of time in which there were common premises around the morality and legality of abortion. In any case, whatever changes in attitude have occurred in recent decades, the positive law cannot exceed the general moral level of the population.\footnote{\url{https://gornahoor.net/?p=8578}}

Since Steven Pinker is a well-respected public intellectual today, we want to focus on his contributions to the discussion.

\paragraph{Health}
\begin{quotex}
Health is the state of an organism in which disease and infirmity are absent. In particular, fertility is considered to be the healthy state and infertility is the inability of a person, animal or plant to reproduce by natural means. It is not the natural state of a healthy adult, except notably among certain eusocial species (mostly haplodiploid insects). (\emph{Wikipedia}) 

\end{quotex}
So apart from some insects, infertility is not the natural state of a healthy adult. Hence, artificial means to prevent the healthy state of fertility needs to be justified, not the other way around. Of course, the Times article will justify them because there may be other health concerns beyond fertility in play. Moreover, throughout history the prevention and termination of births was quite unreliable. Now that medicine has transformed that process, attitudes have changed: demonstrating once again that the material conditions of life alter consciousness.

\paragraph{Life}
The pro-life cause has not helped with the common language, due to its insistence of referring to the unborn as a baby or child. Nevertheless, for those who effing love science, the unborn is scientifically a human life. It starts as an embryo, then becomes a fetus. So a human life, in the biological sense, passes through multiple stages: embryo, fetus, infancy, childhood, adolescence, etc. That is the starting point for a common language.

The other mistake is to equate abortion with murder. Legally, there are different crimes associated with the loss of a human life by another human, depending on circumstances. Hence, abortion should have its own legal category, unless you are ready and willing to execute mothers, doctors, nurses, etc. In an earlier article, Pinker makes this point:

\begin{quotex}
Mothers who kill their newborn infants should not be judged as harshly as people who take human life in its later stages because newborn infants are not persons in the full sense of the word, and therefore do not enjoy a right to life. Who says that life begins at birth? 

\end{quotex}
Pinker is using rather loose language here by confounding the notions of personhood and life. Life is a scientific category and begins long before birth. Personhood is a metaphysical category, which leads to the next topic.

\paragraph{The Right to Life}
Once again from Pinker:

\begin{quotex}
To a biologist, birth is as arbitrary a milestone as any other. No, the right to life must come, the moral philosophers say, from morally significant traits that we humans happen to possess. One such trait is having a unique sequence of experiences that defines us as individuals and connects us to other people. Other traits include an ability to reflect upon ourselves as a continuous locus of consciousness, to form and savor plans for the future, to dread death and to express the choice not to die. And there's the rub: our immature neonates don't possess these traits any more than mice do. 

\end{quotex}
Well that is too bad for the biologists who cannot tell with certainty what life is, thereby destroying its claim to be a serious science. What Pinker is actually asserting is the ``right to life" — which depends on being a person — and not really when biological life begins. However, his definition of ``person" is arbitrary, and not at all scientific for someone who claims to be a scientist.

By the way, mice do dread death which should be common knowledge.

\paragraph{The Will to Life}
The \emph{Declaration of Independence} correctly asserts that the right to life comes from God, not from the state or the arbitrary opinions of so-called ``moral philosophers", politicians, talking heads, etc. Every living being derives that right from its own Will to Life.

Life is being discovered in the most unlikely and seemingly inhospitable environments. In every little crack in asphalt or concrete, you will probably find some plant trying to poke through it. Even around Chernobyl, which is unsuitable for human habitation, an entire ecosystem of flora and fauna has been thriving there.

A life is never a ``clump of cells" because it has a central organizing principle. That makes it a life rather than an arbitrary clump. A human artifact, on the other hand, lacks its own organizing principle; it may seem to function as a unit, but that is because its apparent unity is actually inserted by the designer.

The conclusion is that a life has value in itself, not because it may or may not have a value to others.

\paragraph{Genetic Health}
Pinker provides the evolutionary explanation for opposition to abortion:

\begin{quotex}
[It] is straightforward. Humans are unusual among mammals in that men invest in their children, feeding, protecting and teaching them. As a result, a father who invests in his children will have more successful children, which favors any genes that tilt a man toward investing in his children. But of course that only works if they are his children. Evolutionarily speaking, cuckoldry is the worst thing that can happen to a man, because his investment would be wasted in protecting another man's genes. 

\end{quotex}
So Pinker confirms that opposition to abortion is a sign of a healthy genetic makeup. That is, the selfish gene expresses its impulse to pass on and preserve itself. The obvious corollary is that abortion is, from the biological point of view, akin to a deleterious genetic mutation. The genetic urge to perpetuate oneself is so strong that it controls our mental processes. Pinker asserts, without providing any evidence:

\begin{quotex}
the actual thoughts and emotions running through people's brains are not about babies, cuckoldry, genes, investment or any of the concepts that enter into the ultimate, long-term, evolutionary explanation of people's motives. 

\end{quotex}
I don't see how our genes become aware of complex social and political issues, even to the point of controlling thoughts. Certainly Pinker does not demonstrate a causal chain. If anyone other than Pinker had said that, he would be considered a crank. Note how the Times sneakily alters Pinker's point:

\begin{quotex}
The \emph{men} who are pressing to make abortion illegal are unaware of the evolutionary forces motivating them 

\end{quotex}
Pinker said that ``\emph{people}" are unaware, unless the Times is using ``men" in the generic sense to apply to human beings generally, without regard to sex. But I don't think so.

By this logic, unfortunately, the promotion of abortion must be the sign of a deleterious genetic mutation, since the genes of those people fail to get passed on. Moreover, the desires of ``people" to pass on their genes, and \emph{to not pay for the upbringing of other men's children}, must be indicators of biological health. If an evolutionary psychologist can show otherwise, perhaps he should give it a try.

Of course, there may be moral reasons that transcend biology, but then we are leaving the realm of science.

\paragraph{Group Consequences}
\begin{quotex}
A country whose population is stagnating, diminishing, or aging, creates a vacuum for younger, more active, poorer peoples. A country that no longer has children is a country that has lost confidence in itself, its culture, its history and its values. \flright{\textsc{Fr. Gregory Celier}}

\end{quotex}
Decisions that seem best for an individual often have adverse effects on society as a whole. Artificial contraception is an example. It seems so reasonable to gain the pleasures of sexual intercourse without the possibility of an unwanted pregnancy; the alternative seems absurd to most people today. \emph{Humanae Vitae} warned of the long-term consequences, including the decline of moral standards. Another consequence is the reversal of the general understanding of the purpose of sex.

Moreover, when individual decisions are repeated, the social consequence is the decline in the general population. Most European countries have a sub-replacement level birth rate, as individuals deliberately choose not to reproduce. The social upheaval that will result from this will be ongoing. Nevertheless, the individual does not, and cannot, take the societal consequences into consideration.

\paragraph{Anti-Life}
\begin{quotex}
To the woman he said: in sorrow shalt thou bring forth children \flright{\textsc{Genesis 3:16}}

\end{quotex}
When the purpose of sexual activity is pleasure alone rather than reproduction, then pregnancy is no longer understood as an indicator of health, but as a curse to be avoided. The following is a tweet that had more than 50 thousand ``likes", the last time I checked.

\begin{quotex}
WHY IS GIVING BIRTH SO NORMALIZED??? LIKE YOU LITERALLY CARRY A MONSTER INSIDE YOU THAT MAKES U HORMONAL FOR 9 MONTHS THEN SPLIT YOUR BODY IN HALF GIVING BIRTH?? AND LIKE PEOPLE JUST EXPECT THAT OF WOMEN????????!!!!! 

\end{quotex}
This young woman understands Eve's curse, probably better than most. Woman's sex drive is so strong in order to overcome that reluctance to pregnancy. In most if not all materially advanced countries, the birth rate is becoming precipitously low.

\paragraph{The Best Life}
Several months ago, USA Congresswoman Katie Hill resigned after a photograph appeared of her grooming a staffer while naked. She admitted to being bisexual and engaging in three-way relationships. Surprisingly, her own party coerced her to resign.

That evening, I watched a liberal commentator on TV who condemned the resignation. She defended Mrs. Hill with the interesting claim, ``all she was trying to do was to live her \textbf{best life}."

I was taken aback because I had never considered a life of sexual excess and drug use as anyone's ``Best Life". Traditionally, the best life has been considered a life of virtue. Now there may be disputes about the precise virtues, but they have been close enough. Katie Hill's lifestyle would never qualify.

\paragraph{Death and Violence}
One often hears, ``Violence is never the solution." Obviously, that is quite untrue. Violence ends wars, stops criminals, overturns government, provides food. A youthful unwanted pregnancy more likely than not might prevent the woman from her ``best life", however understood. Perhaps she would drop out of law school, or give up dreams of Hollywood stardom. Or she could terminate the pregnancy and get her life back on track.

When it is a matter of convenience or adherence to abstract principles, convenience will usually win out.

\paragraph{Summary}
The lack of common premises and languages have been delineated. One side is arbitrary in its understanding of life, genetics, physical health, pleasure, convenience. The other side is objective in its understanding of life, health, the purpose of life.



\flrightit{Posted on 2020-06-03 by Cologero }

\begin{center}* * *\end{center}

\begin{footnotesize}\begin{sffamily}



\texttt{Tannheuser on 2020-06-05 at 12:15 said: }

I've always felt that ``Abortion is murder" was a cheap and inaccurate talking point. Again there seems to be no distinction between ``human" (biological entity) and a ``person" (a metaphysical category as you point out), which causes a lot of confusion on both sides. Canto XXV from Dante's Purgatorio is relevant here. The below makes a lot more sense intuitively than the ideas in vogue today:

Having become a soul (much like a plant,

though with this difference—a plant's complete,

whereas a fetus still is journeying),

the active virtue labors, so the fetus

may move and feel, like a sea-sponge; and then

it starts to organize the powers it's seeded.

Open your heart to truth we now have reached

and know that, once the brain's articulation

within the fetus has attained perfection,

then the First Mover turns toward it with joy

on seeing so much art in nature and

breathes into it new spirit—vigorous—

which draws all that is active in the fetus

into its substance and becomes one soul

that lives and feels and has self-consciousness.


\end{sffamily}\end{footnotesize}
