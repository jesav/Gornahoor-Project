\section{From Worms to Gods}

In which we review \textbf{Wolfgang Smith}'s writings on Darwinism and biology. 

\begin{quotex}
I am worm, I am slave, I am king, I am God. \flright{\textsc{Gavrila Derzhavin}}

\end{quotex}
Spiritual and mystical writers are often criticized, e.g., as being ``life denying", for describing themselves as ``worms" or ``slaves". Here, however, the Russian poet Derzhavin gives us the complete formula of the evolution of Christian gnosis: subconscious, waking consciousness, the Real Self, and finally God consciousness.

By now, even old men in the forest have heard the news that God is dead. Unlike the men in the marketplace who rejoice because now ``all things are possible", the old men see that nothing is possible. Man is once again a worm, with no higher purpose, and a slave to his passions, his neurons, his genes, and the molecules that compose his body. Consciousness itself is unreal, an epiphenomenon, an artifact of a certain complexity. The new god is Science, although men seldom grasp what that means. They go on with their lives as if the One God still exists.

I recently read a list of items that, it was claimed, Americans are ignorant of. Among indisputable facts about who was George Washington or who fought in the Civil War, they slipped in some disputable items such as neo-Darwinism vis-à-vis Intelligent Design. Of course the ``intelligent" know that Darwin is correct. The intellectual equivalent of cheap grace, it gives the scientifically illiterate and innumerate the illusion of being among the cognoscenti. There are two ways, however, to understand, or misunderstand, the ``theory of evolution".

\paragraph{The Marketplace}
Option 1 is to go on with life as though nothing has changed, as in this vignette about Galileo\footnote{\url{https://gornahoor.net/?p=1109}}. Man goes on just as before Darwin. His self-awareness is still as a person with a moral will, who can create and love, and so on. The only difference is that he assumes that the human race evolved naturally, albeit randomly, from lower forms of life, rather than as a being who is created in the image and likeness of God.

\paragraph{The Mad Men}
The Mad Men, however, understand that the very concept of the human race is radically altered, in conformance with the demands of the neo-Darwinist theory. By this standard, man is an animal, consciousness is at best a by-product or epiphenomenon of a certain brain complexity, and free will is an illusion. There is no moral absolute, since all behavior is governed by genetics and evolved in order to achieve a certain level of social harmony.

\paragraph{Wolfgang Smith on the Refutation of Darwin.}
Smith's arguments against Darwin are spread across different books, so what follows is an amalgamation. In preparation for this review I listened to two debates involving William Dembski: one against a philosopher and other against a biologist. Dembski's demonstration of Specified Complexity\footnote{\url{https://en.wikipedia.org/wiki/Specified_complexity}} is one of the foundational elements of intelligent design theory. We can't go into the maths here, but hope a simple illustration will help. Suppose you encountered a bullseye in the forest with some arrows in the centre. Of course, you would recognize the unlikeliness of that discovery since it is so different from the forest ambiance. The twigs on the ground, for example, are dispersed in random directions, unlike the specific formation of the arrows. It would be absurd to look for natural causes like wind, animals, etc., to explain such a structure in the forest. Dembski's accomplishment was to formalize such events in terms of information and probabilities. The higher the information content, the more likely there was design involved.

Now the philosopher and the biologist were not at all interested in Dembski's equations. They focused solely on the identity of the designer. Working backwards, their assumption of the non-existence of a designer disproved Dembski a priori. The philosopher, for example, went on and on about an ``old man in the sky", showing an absurd level of ignorance for a degreed scholar. The biologist, moreover, claimed that evolution has been fully ``established", so it would be pointless to refute it.

The Wikipedia article on pseudoscience\footnote{\url{https://en.wikipedia.org/wiki/Pseudoscience}} identifies two of its qualities, which apply to both opponents. You can make your own judgment about which is pseudoscience. These are:

\begin{itemize}
\item Having an agenda apart from pure science 
\item Relying solely on confirmation rather than falsifiability 
\end{itemize}
Is Darwinism falsifiable? Well, Darwin himself pointed out some objections that would falsify his theory. Smith focuses on three.

\begin{itemize}
\item \textbf{Transformism}: Darwin conceded that gaps in the fossil record would falsify evolutionism. Certainly, there are gaps in the fossil record. There are arguments to explain this away, whether by denial or by postulating sudden and large changes in evolution, rather than the small accumulations assumed by Darwin. Smith relies on \emph{The Transformist Illusion} by \textbf{Douglas Dewar} for his argument. 
\item \textbf{Irreducible Complexity}: Again, Darwin conceded Irreducible Complexisty\footnote{\url{https://en.wikipedia.org/wiki/Irreducible_complexity}} would be a refutation. The biologist \textbf{Michael Behe} has provided examples of irreducibly complex biological features. He means that the features require multiple parts that work together in harmony, so that no single system component could have evolved on its own. 
\item \textbf{Complex Specified Information}: We described this above. Information cannot be created through random processes. Without design, all the information today would have to have existed at the so-called Big Bang, otherwise we would just see random processes. Since information is the opposite to entropy, and entropy is increasing, there would have been zero entropy at the beginning. Natural selection cannot by itself create information, unless perhaps Nature is quite different from how we conceive it now. This is the position of Thomas Nagel which we reviewed recently\footnote{See Section \ref{sec:MindandCosmos} in this book.}. 
\end{itemize}

\paragraph{Vertical Causation}
One of the objections to Intelligent Design is that the nature of design is left open. Actually, that is proper for a strictly scientific theory. Intelligent design does not deny that evolution has actually occurred, so it does not deny common descent. Smith, however, will deny it since it occurs in time.

The nature of design requires a metaphysical theory. I had brought this up years ago during an on-line discussion with a philosopher\footnote{See Section \ref{sec:IntelligentDesignandRealism} in this book.}. Smith brings up the notion of Vertical Causation to explain design. This is in opposition to scientific materialists who can only conceive of Horizontal Causation, and even naïve religious believers who are not so different. They understand, by Intelligent Design, that a god interferes from time to time in the world process.

Vertical causation, in Smith's description, on the contrary is ``atemporal", i.e., above the flow of time of the world process. He quotes St Augustine that time itself was created along with the world. Besides God, there are two modes of secondary causation:

\begin{itemize}
\item The causality emanating from the angelic realm 
\item Action derived from human intelligence 
\end{itemize}
To put it succinctly, natural causes are ``blind" whereas vertical causation is the ``hallmark of intelligence".

\begin{quotex}
It is the prerogative of an intellectual nature, a being endowed with intellect and free will. 

\end{quotex}
Smith claims that intelligence and creativity are not temporal processes (although they may manifest in time). Thought, however, does take place in time as a psychosomatic process. However, thought is just the means; cognition itself is beyond thought. That can be difficult to see. Even those trained in monitoring their thoughts seem to miss that cognitive moment. It is like a stroke of insight because

\begin{quotex}
We cannot know bit by bit, because to know is necessarily to know one thing. 

\end{quotex}
Human creativity is allied to God's creative act, Smith says. The human artist works through a word conceived in his intellect, in imitation of the Holy Trinity.

Smith refers to the physicist \textbf{Roger Penrose} who showed in the Emperor's New Mind that Gödel's theorem refuted the idea that the human mind is like a computer algorithm. That is because it can know things are true despite not being able — in principle — to demonstrate them logically. This is beginning to gain some traction as in \textit{Your Brain does not Process Information and it is not a Computer}\footnote{\url{https://aeon.co/essays/your-brain-does-not-process-information-and-it-is-not-a-computer}}. However, this article is lacking the proper metaphysical tools for a deeper understanding.

\paragraph{Unexpected Allies}
Wolfgang Smith identifies three factors to consider:

\begin{enumerate}
\item \textbf{Chance}. Random mutations 
\item \textbf{Necessity}. Physical and biochemical laws 
\item \textbf{Natural Selection or Intelligent Design}. The selection agent. 
\end{enumerate}
To sum up, Smith shows how Darwinism is refuted on its own terms. Yet, even if you reject that refutation as do the Mad Men, those complacent consumers of ideas in the marketplace need to consider that intelligence, creativity, and free will cannot possibly be the result of natural processes.

So we can accept factors (1) and (2), so how does the Mad Man prefer Natural Selection to Intelligent Design? Basically, Natural Selection makes sense only on the assumption that there are only natural and physical causes. Even granting that, chance and time alone can cause nothing.

So does Natural Selection refute Intelligent Design or is it just one possible, still unproven possible explanation? Oddly enough, the idea of Intelligent Design is hard to shake, even for hardnosed scientists. We recently described the panel discussion on whether our universe is really a computer simulation.\footnote{See Section \ref{sec:TheOpenUniverse} in this book.} The entrepreneur Elon Musk holds to the same view.\footnote{\url{http://www.independent.co.uk/life-style/gadgets-and-tech/news/elon-musk-ai-artificial-intelligence-computer-simulation-gaming-virtual-reality-a7060941.html}}

Obviously, if we are living in a computer simulation, then our whole world is designed. Fundamentally, if we could prove that evolution by natural selection is true, then any such design would be ruled out a priori. There is more. Richard Dawkins himself believes in the alien design theory\footnote{\url{http://www.theoligarch.com/richard-dawkins-aliens.htm}}. Well which is it? Does Natural Selection refute Design or is Design still a possibility? These bright men and women would clearly fail that test about being the smartest American.

\paragraph{Excursus on Meaning}
Here is an example to try to make some concepts clear.

\begin{quotex}
How did ears get their start? Any piece of skin can detect vibrations if they come in contact with vibrating objects. This is a natural outgrowth of the sense of touch. Natural selection could easily have enhanced this faculty by gradual degrees until it was sensitive enough to pick up very slight contact vibrations. At this point it would automatically have been sensitive enough to pick up airborne vibrations of sufficient loudness and/or sufficient nearness of origin" \flright{\textsc{Richard Dawkins}, \emph{The Blind Watchmaker}}

\end{quotex}
This is clearly rank speculation, starting with the question of how the sense of touch ``evolved" for the skin to detect vibrations. A scientific explanation needs a basis in physical reality. Specifically, the missing link is the DNA that accounts for the ear and how it altered over time.

Yet there is a deeper objection that is hard to grasp for many people. It is that we hear ``meanings" not just sounds. It is easy to prove for yourself: how do you ``hear" your native language vs how you experience an unknown foreign language.



\flrightit{Posted on 2016-06-09 by Cologero }

\begin{center}* * *\end{center}

\begin{footnotesize}\begin{sffamily}

\texttt{Max on 2016-06-13 at 15:35 said: }

Or, more on topic, attempt to locate the ``missing link" proving neo-darwinism in this map: \url{http://biochemical-pathways.com/\#/map/1} 

I have been pondering how it comes that people fail to recognize intelligent design when looking at it. I assume that intelligence and intent is required for that. People who support themselves on others with no thought of contributing to any common good assumes that their means of existence just arose randomly causing no obligations on their part. Anyone who has tried to create something should be able to see the signs of something created. 

Different people sees different things. Now, if only it were so easy as allowing everyone to live in their own private dreamworld. A problem that arises however, is that we also act on false beliefs and thereby risks ruining it for everyone else.

If someone believes that they live in a random hell without purpose, and is then informed that that is not the case, but on the contrary everything is beautifully thought out and ordered with the best of intentions, would it not be a cause of joy?


\hfill

\texttt{Cologero on 2016-06-15 at 07:30 said: }

Max, it gets worse when even man-created events are seen as random, without design or purpose. For example, in the USA, we hear talk about the ``demographic change", as though it were some natural event. Whether the change is good or bad is a separate issue. Either the change was deliberately planned or it could have been foreseen as the inevitable consequence of certain policies.


\hfill


\end{sffamily}\end{footnotesize}
