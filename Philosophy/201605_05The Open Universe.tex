\section{The Open Universe}

\label{sec:TheOpenUniverse}

Once in a while, it is useful to regard current events from the Traditional perspective. This week we ponder about whether the world is a computer simulation engineered by some evil aliens and how that might explain the political system in the USA.

\paragraph{The World as Simulation}
\begin{quotex}
We ought to regard the present state of the universe as the effect of its anterior state and as the cause of the one which is to follow. Assume an intelligence which could know all the forces by which nature is animated, and the states at an instant of all the objects that compose it. for this intelligence, nothing could be uncertain; and the future, as the past, would be present to its eyes. \flright{\textsc{Pierre-Simon Laplace}}

\end{quotex}
The \textbf{American Museum of Natural History}, which claims to be ``on the frontier of scientific discovery", recently held a panel discussion on the question: \emph{Is the Universe a Simulation?}\footnote{\url{https://www.youtube.com/watch?v=wgSZA3NPpBs}}. The panel was ``politically correct" and does not represent any physics faculty I am familiar with.\footnote{See the following webpages: \url{https://www.aip.org/statistics/data-graphics/percentage-physics-faculty-members-who-are-women}\newline\url{https://www.aip.org/statistics/data-graphics/race-and-ethnicity-physics-faculty-0}}. Presumably, a panel discussion on real physics would attract a minuscule audience, whereas the illusion of physics will be a sellout.

While these sophisticated New Yorkers would most likely be appalled to learn that the Universe was designed and created by God, they are perfectly willing to consider the possibility that it has been designed by an alien intelligence. That is, our world is more like a computer program, so we could conceivably be nothing but software. Now the ideas of an algorithm and computability have precise meanings, so the question reduces to this: Is the universe computable? If not, then the premise must be false. I couldn't bear to listen to more than a few minutes of the debate, so I don't know if these issues were even brought up; nevertheless, we will consider these: the Big Bang, Natural Selection, and Consciousness.

\subparagraph{The Big Bang}
The Enlightenment philosopher Laplace claimed that if he knew the laws of physics and the initial state of the universe, he could then predict all future events. Let's concede that we know those laws, so the only issue is the initial state of the cosmos. Unfortunately, there is no ``initial state" of the Big Bang, since there is no ``beginning" of the universe in space-time. Rather, the equations posit a singularity at time $t=0$. So either our understanding of relativity and quantum mechanics is flawed, or the universe is not computable.

\subparagraph{Intelligent Design}
While preparing a review of some books by \textbf{Wolfgang Smith}, I watched a couple of youtube debates about the theory of intelligent design. The opponents of the theory were quite nasty; it seemed that the merits of \textbf{William Dembski}'s ideas didn't matter at all. Rather, they were intransigently opposed to the idea that there may be a designer. They insist that ``natural selection" fully accounts for the variety of flora and fauna that we experience now.

In other words, natural selection is not merely an alternative to intelligent design, it is actually a refutation of intelligent design. If that is the case, then we cannot be living in a simulation, since life forms cannot have arisen from a deliberate design.

\subparagraph{Consciousness}
There are many metaphysical demonstrations that intelligent thought is transcendent to the world, and therefore cannot be the result of some computable biochemical process. We've mentioned some, so there is no point in going into more detail. Hence, we cannot be inside a simulation.

From a different perspective, some may be interested in \emph{The Open Universe} by \textbf{Karl Popper}, which refutes many varieties of scientific and metaphysical determinism. Its concern is to justify the ``\textbf{freedom, creativity, and rationality of man}", which are incompatible with a simulated universe. I suppose that people with an underdeveloped sense of their own freedom, creativity, and rationality may still find it plausible.

\paragraph{The Limits of Rationality}
One consequence of an open universe is that there are limits to rationality. For example, there is Gödel's theorem which denies the computability of the laws of arithmetic. More to the point, all political attempts at central planning are doomed to failure.

Astrology was used in the past as a tool for central planning. An interesting history of astrology, mathematics, and political control can be found here: By fetishising mathematical models, economists turned economics into a highly paid pseudoscience.\footnote{\url{https://aeon.co/essays/how-economists-rode-maths-to-become-our-era-s-astrologers}}

However, there is an important difference between ancient astrology and contemporary economics. The former tried to be deterministic, seeking an exact correspondence between the heavens and earth, while the latter relies on less precise statistical methods. Those methods were not known to the ancients and they are poorly understood now, except by specialists. One philosopher who has integrated statistical understanding into a traditional Thomist framework is \textbf{Bernard Lonergan}. In his magnum opus, \emph{Insight}, he writes that the knowledge of laws can be applied to

\begin{itemize}
\item Individual events 
\item Systematic processes 
\item Non-systematic processes 
\end{itemize}
The knowledge of individual things or events is the most primitive. This is how most people think and it pertains also to animals. From the individuals, the next level of understanding is to grasp the concept that explains the individuals. There is an understanding of the ``whole". For example, by understanding the proof that the angles of triangles add up to 180 degrees, I no longer have to start measuring each individual triangle. In the scale of angelic intelligences, there are deeper and deeper layers of abstraction. For example, the Periodic Table makes clear at a glance the relationships of the chemical elements and the Heliocentric theory explains the movements of the planets with a simple law.

Things get trickier in the Open Universe, since we have to rely on more sophisticated quantitative techniques. \textbf{Robert Nozick}, in \emph{The Nature of Rationality}, discusses game theory, decision theory, and so on. Of course, quantitative methods are quite important in that domain.

There have been many attempts to classify human beings and typology has always been an esoteric science. Without going into any of them right now, we can point out that it is always a matter of trends and tendencies, since the movement of history cannot be predicted with precision. Nevertheless, statistical methods are nevertheless useful. After all, insurance companies and casinos are profitable precisely because of their knowledge of probabilities and statistics.

Unfortunately, in setting policy, it is difficult for decision makers to make the transition from the knowledge of individuals to the knowledge of non-systematic processes. First of all, there is a psychological resistance to any system of classification that entails any essential aspects to diversity (despite diversity itself being cherished). Then there is the resistance to general conclusions, because in a statistical distribution, there are individuals that fall outside the general tendency.


\hfill

\emph{To be continued}

\flrightit{Posted on 2016-05-05 by Cologero }

\begin{center}* * *\end{center}

\begin{footnotesize}\begin{sffamily}



\texttt{James O'Meara on 2016-05-05 at 12:58 said: }

As Shaw saw, people swore allegiance to ``natural selection" not for any evidence but because they wanted to escape from the nosy Calvinist God. Only later did they realize that they were now stuck in a meaningless universe. Since God is still infra dig, I think ``aliens" have taken his place among the bien pensant. Mythology etc. is ``explained" by ``ancient aliens." Well, that's still pretty fringe; but here we see ``smart" New Yorkers taking the idea of alien programmers seriously. Anything but God or even Spirit.


\hfill

\texttt{Wayne Ferguson on 2016-05-06 at 11:24 said: }

Leaving aside the idea of a computer simulation, per se, the idea of a holographic universe is, in fact, being seriously entertained by several top-notch physicists. As I see it, this would be consistent with some form of idealism (e.g. Neoplatonism).

With that in mind, I use the following popular presentation to show that the appearance of evolution (which seems, to me, to be undeniable) would, from the stand-point of a holographic universe, be true as phenomena (in the same way that a sunrise or sunset is) but not ultimately explanatory…

``The illusion of third dimension : the universe as a hologram or holographic universe" [``Nova" excerpt]

\url{https://www.youtube.com/watch?v=16WIlRJxnrY}

The sun apparently circles the earth (from a geocentric perspective). We still acknowledge this appearance — honor it, even — when we speak of ``sunrises" and ``sunsets" (even though we know it is not, strictly speaking, ``the truth" — or at least not the whole truth).

Likewise, it seems to me that I can reasonably acknowledge that my (apparent) body appears to be the result of evolutionary processes without conceding that ``I" am the product of biological evolution, per se. Evolutionary biology illuminates the natural genealogy of the form that I see when I look in a mirror, but there is more to me than meets the eye…

As such, even if we were to confirm that the spatio-temporal world is a holographic image that reflects some sort of transcendent intelligence/idea/datum, we could still point to (and speak of) the phenomena of biological evolution (as we currently imagine it) as taking place over the last several hundred million years), but we would now know that the real cause of this apparent process transcends the flow of appearances (similar to the way in which we now know that the ``sunrise" and ``sunset" are caused by the movement of the earth relative to the sun).

Leaving aside the holographic universe, however, it is also the case that we can't seem to arrive at an understanding of consciousness through the analysis of matter and material processes alone — even Sam Harris acknowledges this (see the quotes, below, following the link to my ``Two Arguments Against Physicalism") 

[Note: One of those ``two arguments", BTW, is that it makes no sense to say that consciousness, as such, is selected for if it is reducible to physical structures and processes–and yet some scientists never tire of generating ``just so" stories that explain why certain modes consciousness, as such, evolved (but I digress..) ]

In short, even though our apparent bodies appear to have evolved over time from non-human species, that does not account for consciousness as such–which, as I see it, is the gift of God (an inexplicable mystery). That which is born of the flesh is flesh" (i.e. those who understand themselves exclusively according to their natural history and/or genealogy and identify themselves exclusively with the form they see in the mirror and with their personal autobiography) and ``that which is born of the Spirit is Spirit (those who participate in the life of Christ which is represented in terms of the virgin birth and incarnation). As such, we are called upon to be ``put to death in the flesh, but made alive in the Spirit" — called upon to take up our cross and enter the kingdom NOW. As Valentin Tomberg explains, our Divine image is in-tact, but because we do not recognize and honor it, the Divine likeness has become disfigured and, thus, it is said that we ``must be born again/from above" (i.e. we must become ``The Hanged Man" — see Letter 12 ``Meditations on the Tarot"– see Letter 14 for the discussion of the Divine image and likeness).

\url{https://jwayneferguson.wordpress.com/2015/05/23/two-arguments-against-physicalism/}

Quoting from chapter two of ``Waking Up", by Sam Harris:

``However we propose to explain the emergence of consciousness—be it in biological, functional, computational, or any other terms—we have committed ourselves to this much: First there is a physical world, unconscious and seething with unperceived events; then, by virtue of some physical property or process, consciousness itself springs, or staggers, into being. This idea seems to me not merely strange but perfectly mysterious. That doesn't mean it isn't true. When we linger over the details, however, this notion of emergence seems merely a placeholder for a miracle" (56).

``The fact that the universe is illuminated where you stand— that your thoughts and moods and sensations have a qualitative character in this moment —is a mystery, exceeded only by the mystery that there should be something rather than nothing in the first place" (79).


\hfill

\texttt{Wayne Ferguson on 2016-07-10 at 22:34 said: }

In this controversial film on Intelligent Design with Ben Stein, Richard Dawkins speculates that organic life on this planet may have been designed and that the designer(s) may have left some trace that will eventually be found. This segment lasts about 6 minutes:

\url{https://www.youtube.com/watch?v=V5EPymcWp-g\&t=87m0s}


\hfill


\end{sffamily}\end{footnotesize}
