\section{The German Conservative Revolution}

In this useful collection of essays, \textit{From the German Conservative Revolution to the New Right}\footnote{\url{https://pancriollismo.com/}}, \textbf{Lucian Tudor}
introduces us to several German political thinkers, loosely grouped under the rubric “conservative revolution”. In the
second part of the book, Mr Tudor tries to link these authors to a more contemporary movement called the “New Right”.
For this review, we will focus on the German authors, perhaps to deal with the second part at a later date. Readers are
advised to skip past the Foreward.

In my notes to the book, I have extracted several extended quotexs (marked off with a blue bar). Therefore, I will
leave the commentary to a minimum and reproduce those here. The intent is to encourage the reader to consult the book
itself and even the German authors themselves, many of whom have never been translated into English. But first some
preliminaries.

The book was published by a Chilean identitarian group, \emph{Círculo de Investigaciones Pancriollistas}. I
don't know what attraction German writers or the New Right have for them, since there is sufficient
material in Spanish and other Romance languages that, in my opinion, is of equal of greater value. Moreover, we can
recall Charles Maurras' ideal of an alliance of Romance language speaking nations, which would have
included Romania (Lucian Tudor is Romanian). That ideal is based on the undeniable cultural, historical, linguistic,
and spiritual identity of those nations, if “Identity” is truly the goal.

Unlike historical Europe, the Americas are much more ethnically diverse. This can perhaps be seen as the fourth
“hyperborean migration”, i.e., the habitation of the Americas, Australia, and so on. It brings up the question of how
exactly does such a group establish a separate identity within such a diverse country like Chile? The German writers
all assume that the nation and geography more or less coincide.

Nevertheless, as Mr Tudor shows us, these German thinkers often reaffirm the same ideas as the French and the Spaniards.
There are actually many valid principles that can be applied to any claimants to be heirs of this line of thought. Mr
Tudor lets \textbf{Edgar Julius Jung}'s definition set the tone for the first part of the book.

\begin{quotex}
By conservative revolution? we mean the return to respect for all of those elementary laws and values without which the
individual is alienated from nature and God and left incapable of establishing any true order.

\begin{itemize}
\item
In the place of equality comes the inner value of the individual; 
\item
in the place of socialist convictions, the just integration of people into their place in a society of rank; 
\item
in place of mechanical selection, the organic growth of leadership; 
\item
in place of bureaucratic compulsion, the inner responsibility of genuine self-governance; 
\item
in place of mass happiness, the rights of the personality formed by the nation 
\end{itemize}
\end{quotex}

At this point, we need to mention an inconvenient fact: the liberal societies of the West have had great material
success and enjoy popular support. People flock to liberal states and no one tries to leave. Hence, it will never be a
question of who has the “best” argument, since the psychological trumps the logical except in rare men. After all, who
is willing to assume his “rank” in society? Mr Tudor deliberately left out any detailed analysis of Julius Evola, so
there is no discussion of the degeneration of castes; that is the real explanation for the state of things. But getting
back, Mr Tudor extracts some common features:

\begin{itemize}
\item the belief in the values of Volk (people, nation, or ethnos) 
\item the recognition of the value of differences between individuals and between peoples 
\item the importance of authority 
\item the value of holism and supra-individual community-feeling 
\item the importance of religious belief 
\item the supremacy of vital and spiritual forces over material and artificial forces in human life 
\item the call to overcome modern nihilism 
\item a revolutionary view of tradition (\textbf{radical cultural conservatism}) 
\end{itemize}
\textbf{Moeller van den Bruck} distinguishes between the “reactionary” who wants to restore lost forms from the
“conservative”. Quoting him, Mr Tudor adds his own conclusion:

\begin{quotex}
“Conservatism seeks to preserve a nation's values, both by conserving traditional values, as far as these
still possess the power of growth, and by assimilating all new values which increase a nation's vitality.
He [the conservative] distinguishes the transitory from the eternal.” In other words, there are values and principles
which are timeless and eternally valid, but the particular forms (institutions, laws, social orders, cultural forms,
etc.) in which they are manifested are temporal, and vary and transform by time and place. This conception of
Conservatism makes it possible to resist undesirable modern developments without unrealistically rejecting everything
in the modern world and to revolutionize contemporary society by regenerating what was valuable in the past, conserving
what is valuable in the present, and accepting positive new ideas for the future. 

\end{quotex}
\paragraph{Conservative Socialism}
Mr Tudor identifies 7 themes addressed by the conservatives: Conservative socialism, folkish integralism, Christian
radical traditionalism, cultural pessimism, Biocentrism, political philosophy, and philosophy of war. What he calls
“socialism” is actually more like a “third way”, more medieval than modern. These are its features:

\begin{itemize}
\item an anti-individualist value of organic community and social solidarity 
\item the reconciliation of social justice with a respect for the inequality of character and hierarchy in society 
\item a corporative organization of the economy 
\item the view that ethics (work ethic, altruism, devotion to service and the whole) are just as important as economics
in defining socialism, 
\item a greater emphasis on national unity rather than class warfare 
\end{itemize}

\paragraph{Integralism}
This includes:

\begin{itemize}
\item Social wholeness (holism) 
\item Cultural particularism 
\item The collective meaning of the Folk 
\end{itemize}
\begin{quotex}

Combined, individualism and total openness harmed the integrity of peoples (Folk) and created uncertainty and alienation
in social and cultural life. The Revolutionary Conservatives advocated the overturning of liberal society and the
creation of integrated, more closed, ethnically particularist, and holistic (anti-individualist, community-oriented)
states which would restore the profound sense of collective meaning in life. 

\end{quotex}

\paragraph{Christian Traditionalism}
Mr Tudor mentions \textbf{Othmar Spann} and \textbf{Edgar Jung} as the most prominent representatives of this line of
thought.

\begin{quotex}
The Christian radical traditionalists advocated the creation of a monarchical state which would also be led by a
hierarchically organized, authoritarian elite that would be open to accepting new members based on their quality, thus
creating a spiritually aristocratic leadership. As Jung described it, the state as the highest order of organic
community must be an aristocracy; in the last and highest sense: the rule of the best. 

Christian radical traditionalists also asserted that their vision of the ideal state was the True State, meaning a
socio-political structure which varies between cultures but which reappears across history, and is thus based upon an
eternally valid model. 

\end{quotex}
Obviously, this line of thought is more appealing than the others to Traditionalists.

\paragraph{Cultural Pessimism}
The pessimist rejects the idea of human progress and recognizes that cultures go through cycles of growth and decline.
Unlike what a Spengler may write, however, there can be no “law” behind that.

\paragraph{Biocentrism}
\begin{quotex}
Biocentrism posited an essential distinction between Seele (Soul) and Geist (Spirit), which conflict each other in human
life. The Geist is the nous, the pneuma, or the logos … what we would call in ordinary speech the intellect, the
reason, the spirit, the mind…. The Seele corresponds to the Greek psyche. It is the living principle, the vital spark …
and one with the body, soma. Biocentrism is a romanticist and anti-rationalist philosophy which poses the Soul as
positive and the Spirit as negative. According to Biocentric theory, human beings had originally in primordial, ancient
times lived ecstatically in accordance to the principle of Leben. 

Biocentric philosophy also attacked Judeo-Christianity as a Logocentric religion opposed to Life and upheld ancient
Paganism (which has a Dionysian character centered around vitalistic, feminine values) as the Biocentric religion of
Life. 33 

\end{quotex}
Biocentrism is the exact opposite of a principle. It extols brute animal life above proper human life. The admission
that biocentrism is explicitly “feminine” says it all. It is the philosophical ideal of the neo-pagan.

\paragraph{Political Theory}
\begin{quotex}
Carl Schmitt's philosophy began with the concept of the political,? which was differentiated from politics?
in the normal sense and was based on the distinction between friend and enemy. The political exists wherever there
exists an enemy, a group which is different and holds different interests, and with whom there is a possibility of
conflict. 

Sovereignty is the power to decide the state of exception, and thus, sovereign is he who decides on the exception. 

\end{quotex}
Another notable argument made by Schmitt was that true democracy is not liberal democracy, in which a plurality of
groups are treated equally under a single state, but a unified, homogenous state in which leaders' decisions express the will of the unified people.

\begin{quotex}
Democracy requires, therefore, first homogeneity and second, if the need arises, the elimination or eradication of
heterogeneity. 

\end{quotex}
Ultimately, the party system displaces the friend-enemy into the bosom society itself: each party sees the other as the
enemy. That inhibits rational political discourse within democracies and makes the quest for the common good more
difficult.

\paragraph{Philosophy of War}
The conservatives viewed universal peace as unrealistic and accepted the inevitableness of war.

\begin{quotex}
[Spengler] warned that if Europeans adopted the pacifist ideal, non-Europeans would wage war and rule the world: Strong
and unspent races are not pacifistic. To adopt such a position is to abandon the future, for the pacifist ideal is a
static, terminal condition that is contrary to the basic facts of existence 

\end{quotex}
Schmitt also critiqued the notion among liberals and Marxists of the claim to fight for universal humanity, for such a
notion dehumanizes one's enemy, essentially declaring him to be an outlaw of humanity; and a war can
thereby be driven to the most extreme inhumanity. Schmitt especially held in high regard the system of limited and
civilized warfare developed by Europeans since the Middle Ages, which allowed the avoidance of excesses.

\begin{quotex}
Werner Sombart wrote of the difference between nations whose dominant character is marked by the Trader type
(exemplified by the English) and the Hero type (exemplified by the Germans). The former is marked by utilitarianism,
materialism, individualism, and commercialism, while the latter is marked by altruism, the willingness to sacrifice,
orientation towards duty, anti-individualism, and contempt for materialism 

\end{quotex}
\paragraph{Arthur Moeller van den Bruck}
In the next part of the book, Mr Tudor provided us with some interesting ideas from the Conservatives themselves. What
follows is a small selection of those ideas.

\begin{quotex}
Young peoples, which included Germany, Russia, and America, possessed a high amount of vitality, hard work,
will-to-power, strength, and energy. Old peoples, which included Italy, England, and France, were saturated, highly
developed, valued happiness over work, and generally had a lower amount of energy and vitality. 

\end{quotex}
Moeller van den Bruck actually proposed an alliance between Germany, Russia, and America. Obviously, that never
happened.

\begin{quotex}
He had a peculiar idea of race which presented a dichotomy between \emph{Rasse des Blutes} (Race of the Blood), which
refers to the common biological concept of race, and \emph{Rasse des Geistes} (Race of the Spirit), which refers to
psychological or spiritual character which is not hereditarily determined. 

\end{quotex}
We have no idea why Mr Tudor considers that view “peculiar”.

\begin{quotex}
Secondly, the nations which Spengler claimed constituted the West had powerful differences between each other,
especially in terms of being young and old, which affected whether they would rise or decline, as well as cultural
differences. Moeller wrote that due to these significant differences there was clearly no homogeneous Occident and for
that reason alone there can be no homogeneous decline … the English and French nations were old? but shrewd and
politically experienced, while Germany was young? and vigorous but had behaved in an inexperienced and impetuous
manner. 

\end{quotex}
Comment: losing has consequences

\begin{quotex}
Revolutions cannot transform a nation because the past customs, traditions, and values of a nation cannot ever simply be
totally brushed aside … Materialism and rationalism “embraces everything except what is vital” … Higher spiritual
forces and ideas guided his actions”. 

\end{quotex}
Comment: How to recognize them?

\begin{quotex}
the proletarian is a proletarian by his own desire. Thus the proletariat in the Marxian sense was not a product of his
position in capitalist society, but merely of “the proletarian consciousness.” 

\end{quotex}
Along with the idea of the spirit of the race, this recognition of the role of caste aligns Moeller van den Bruck with
some of Evola's ideas.

\paragraph{Othmar Spann}
Spann is one of the more Traditional thinkers. Besides his concern with the social whole over the individual, he looked
to the Nordic-Roman Tradition of the Middle Ages as the model of a healthy social order.

\begin{quotex}
Spann essentially taught the value of nationality, of the social whole over the individual, of religious (specifically
Catholic) values over materialistic values, and advocated the model of a non-democratic, hierarchical, and corporatist
state as the only truly valid political constitution 

It is the fundamental truth of all social science … that not individuals are the truly real, but the whole, and that the
individuals have reality and existence only so far as they are members of the whole.? This concept, which is at the
core of Spann's sociology, is not a denial of the existence of the individual person, but a complete
rejection of individualism; individualism being that ideology which denies the existence and importance of
supra-individual realities. 

Because he also believed that the German nation was intellectually superior to all other nations, Spann also believed
that Germans had a special duty to lead Europe out of the crisis of liberal modernity and to a healthier order similar
to that which had existed in the Middle Ages 

Spann attempted to formulate a conception of race which was in accordance with the Christian conception of the human
being, which took into account not only his biology but also his psychological and spiritual being. This is why Spann
rejected the common conception of race as a biological entity, for he did not believe that racial types were derived
from biological inheritance, just as he did not believe an individual person's character was set into place
by heredity… The material or physical substance and appearance is shaped by the immaterial, pre-material, or
super-material substance … Race is not determined by biological inheritance but by the spirit, which holds a social and
historical dimension, and thus is formed by the spiritual community. 

\end{quotex}
Of course not. That is not just the “Christian conception” of the human being, but the actual human being apart from any
conceptions. A human being is a physical, psychological, and spiritual composite; that must be part of any intelligent
thought. If biological race is primary, then there would be no cultural decay or decline. Nations decline spiritually
before they decline materially.

\begin{quotex}
The principles of the True State, on the other hand, were metahistorical and eternally valid, because they were derived
not from material reality, but from the supra-sensual and transcendent reality, from the Divine order. Spann regarded
the Holy Roman Empire as the best historical reference for the True State. 

\end{quotex}
Spann expected the “subordination of the intellectually inferior under their intellectual betters”. With universal
education, however, the belief arises that one man's opinion is as good as another's.

\begin{quotex}
The state would be led by a powerful elite whose members would be selected from the upper levels of the hierarchy based
on their merit; it was essentially a meritocratic aristocracy…Another defining characteristic of the elite of the True
State was its spiritual character. The leadership received its legitimacy not only from its intellectual superiority
and its power, but from its possession of valid spiritual content…Furthermore, the leadership must be guided by their
devotion to Divine laws and animated by Christian spirituality, which inherently rejects rationalistic and
materialistic thought, asserting the primacy of the metaphysical, transcendent reality 

\end{quotex}
Such views made Spann unpopular both with the Nazis and with the liberals. Edgar Jung also took up the theme:

\begin{quotex}
The phenomenal forms that mature in time are always new, but the great principles of order (mechanical or organic)
always remain the same. Therefore if we look to the Middle Ages for guidance, finding there the great form, we are not
only not mistaking the present time but apprehending it more concretely as an age that is itself incapable of seeing
behind the scenes…neither Fascism nor National Socialism were precursors to the reestablishment of the True State but
rather simply another manifestation of the liberal, individualistic, and secular tradition that had emerged from the
French Revolution.? Fascism and National Socialism were not guided by a reference to a Divine power and were still
infected with individualism, which he believed showed itself in the fact that their leaders were guided by their own
ambitions and not a duty to God or a power higher than themselves. 

\end{quotex}
\paragraph{Hans Freyer}
\begin{quotex}
Tönnies's work established a fundamental distinction between Gemeinschaft (Community) and Gesellschaft
(Society), a distinction which Freyer and many other German intellectuals would agree with. According to this concept,
Gemeinschaft consists of the organic relations and a sense of connection and belonging which arise as a result of
natural will, while Gesellschaft consists of mechanical or instrumental relations which are consciously established and
thus the result of rational will…The community (Gemeinschaft) thus designates a social entity which is based upon
solidarity, bonding, a sense of connectedness and interdependence; it means belonging to a supra-individual whole on a
deep spiritual level. 

Hans Freyer's cultural philosophy began with the theory of the Volk (people or ethnicity) as the primary
cultural entity, and the reality and importance of cultural particularism. Drawing from the German philosophical
tradition, Freyer argued that the Volk was the collective entity from which particular cultures emerged, which bore the
imprint of a particular Volksgeist (folk spirit) or collective spirit of the people.

It is here [at the Volkstum] that all the talk of race originates and has its truth. When one objects that this is pure
biology, that after all spiritual matters cannot be derived from their natural basis, or when one objects that there
are no pure races, these objections fail to grasp the concept of race that is a component of the new worldview. Race is
understood not in the sense of mere biology but rather as the organic involvement of contemporary man in the concrete
reality of his Volk,

people in modern times held an awareness of the existence of the multiplicity of human cultures and their historical
foundations. This awareness caused many modern people to feel an uncertainty about the full validity of their own
culture, something which served as a factor in the loss of a sense of meaning in their own traditions and therefore a
loss of a sense of personal meaning in their culture. That is, a loss of that sense of guidance and value in
one's own traditions which was more common in ancient and Medieval societies, where human beings tended to
recognize only their own culture as valid.

His new conservatism also held Christian religiosity at its center, emphasizing the direct experience of the faith in
Christ, rather than the institutional body of the church 

\end{quotex}
\paragraph{Oswald Spengler}
Since Spengler is much better known than the others, I will just provide this quote:

\begin{quotex}
In his critique, Evola pointed out that one of the major flaws in Spengler's thought was that he lacked any
understanding of metaphysics and transcendence, which embody the essence of each genuine Culture. Spengler could
analyze the nature of Civilisation very well, but his irreligious views caused him to have little understanding of the
higher spiritual forces which deeply affected human life and the nature of cultures, without which one cannot clearly
grasp the defining characteristic of Culture. 

\end{quotex}
\paragraph{Identity}
As Virgil noted in the Aeneid, it took a lot to form the Roman people. What this means is the Identity is not given a
priori, but rather it must be created. There is an essence, or possibility of manifestation, and the manifestation
itself which occurs in time and space. Since the mass of people are passive consumers of “culture”, the true creators
and bearers of culture are necessarily few.

\flright{\itshape Posted on 2015-09-14 by Cologero}

\begin{center}* * *\end{center}

\begin{footnotesize}\begin{sffamily}

\texttt{Mark Citadel on 2015-09-27 at 05:06 said: }

Wonderful stuff. I had never heard of Edgar Jung or Othmar Spann, but their views certainly seem to be close to my own
concerning the proper institution of Christian autocracy and the race of the spirit. 

“The state would be led by a powerful elite whose members would be selected from the upper levels of the hierarchy based
on their merit; it was essentially a meritocratic aristocracy”

This should not be unfamiliar to Tudor. From what I have read, Corneliu Codreanu had this exact same conception. He
rejected hereditary monarchy and instead favored an elective position determined by an elite caste. I think
I'm going to pen a discussion of this view to coincide with the date of the Captain's
martyrdom. It's an interesting debate hereditary vs. elective. I'll have to see more of Jung
and Spann.

I really hope Tudor, being Romanian, could help get some translated works of Nichifor Crainic available. He was
extremely influential in Interwar Romania as a political theorist and theologian, but literally nothing is available in
English.


\hfill

\texttt{Political Prisoner on 2015-10-06 at 10:24 said: }

It would be interesting to see how Christians reconcile war or the just war concept with Jesus' sermon in
Matthew 5 on the non-resistance to evil. Matthew 5 makes perfectly clear that all of the militarism and crusading
during the Middle Ages on was inherently anti-Christian, and that none of Jesus' teachings, which were to
be taken literally as with the early martyrs, were to be violated in the least. What has manifested under the Catholic
church therefore cannot be called Christianity but a satanic parody of such. Rather it would seem that “liberal”
pacifism is much more attuned to his teachings than conservative “toughness”.


\hfill

\texttt{Cologero on 2015-10-06 at 21:20 said: }

If the topic of the just war teaching is so interesting to you, then I suggest you review the relevant literature.

In any case, your comment has nothing to do with the post. Comments should follow the rules of cross-examination and
deal only with what was said or suggested in the post.

Unfortunately, you misunderstand Gornahoor's purpose, which is to explore Tradition including its social
structures, metaphysical principles, and esoteric teachings, with the ultimate aim being its possible restoration. We
are not promoting any specific religious teachings, although we will use them to illustrate the manifestations of
Tradition. We focus on the Medieval tradition for the practical matter that it is closer to us in time, customs,
language, etc., so presumably it will be easier to grasp for the modern mind seeking to understand.

Now the authors that primarily interest us — Guenon, Evola, Coomaraswamy — all agree that the
European Middle Ages, along with its spiritual teachings, constituted a valid tradition. If you don't
understand why that is so, then you have no understanding of any traditions at all. Besides, we have expended
considerable effort in justifying that judgment.

I can provide you some material for self-study that may help you on your spiritual journey to liberation from your
self-imposed prison.

We agree with Charles Maurras, at least in principle although not in all specifics:

\begin{quotex}
What our fathers did through custom and feeling, we ourselves pursue it through reason and will, with the assurance and
clarity of science. 

\end{quotex}
What you see as a `Satanic parody’, other, more intelligent pagans see something else. For
example, Evola wrote:

\begin{quotex}
we see Christianity becoming Roman with Catholicism: purifying itself of its original anarchic, universalistic, and
humanitarian aspects, and giving rise, in the Middle Ages, to a civilization that is characteristic of the type we
articulated: hierarchical, tied to traditions of caste and blood, interspersed with initiatic elements 

\end{quotex}
And this one, again quoting Maurras:

\begin{quotex}
All my favorite ideas — order, tradition, discipline, hierarchy, authority, continuity, unity, work, family,
corporation, decentralization, autonomy, organization of workers — had been preserved and perfected by
Catholicism. 

\end{quotex}
Here are some quick bullet points for your consideration (all these topics have been developed in various posts).

\begin{itemize}
\item We reject the doctrine of “Sola Scriptura”, so your out of context biblical quotes prove little 
\item The spiritual authority has the sole prerogative to decide on the final meaning of revealed texts. On this both
Guenon and Evola agree. See Donoso Cortes on why it is necessary for a stable society to settle infinite debates. 
\item A man should normally follow the tradition of his land and ancestors. This is tied in with notions of fidelity,
loyalty, piety, duty and so on. Your implication that your ancestors were idiots who followed a “satanic parody” of
true tradition is a truly contemptible notion. Shame on you. 
\end{itemize}

\hfill

\texttt{V on 2015-10-11 at 10:44 said: }

@Political Prisoner.\newline
This issue regarding violence and Christianity has been addressed over and over again. I think maybe the original
teaching may have been lost somewhere in Protestant deviations and that is why so many people no longer know it. The
teachings of Jesus regarding non-violence and pacifism are meant to be taken as behavior designed for the individual,
not as for the nation or the state or the military. In other words, Jesus's teachings are of a personal
nature, not of a national or collective nature. They are something for individual people to follow in daily
interactions, they are not meant to be followed by nations in war. In any case, I think Cologero makes a good point.


\hfill
\end{sffamily}\end{footnotesize}
