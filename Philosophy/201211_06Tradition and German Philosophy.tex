\section{Tradition and German Philosophy}

The attitudes of \textbf{Julius Evola} and \textbf{Charles Maurras} toward the influence of German thought were
fundamentally different. Maurras opposed it on several points; he regarded the Germans as barbarians and rejected, in
his view, German nationalism, racism, its Protestant outlook. Specifically, he rejected Fichte's philosophy
as the basis of German thought. Rather than an alliance with the Germans, Maurras was hoping for an alliance of the
Romance language speaking nations of Europe. The documentation for this will have to wait for another time, since it
involves pulling together and translating statements from multiple works.

On the other hand, Evola was a Germanophile; he admired the German spirit and regarded the civilization of the Middle
Ages as a joint creation of Europe's German and Roman elements. In his youth, the heirs of German
philosophy were found in Italy in Giovanni Gentile and Benedetto Croce. Evola embarked on a program of self-study of
German philosophy well before his turn to Tradition. This influence colored (or tainted, depending on your point of
view) Evola's exposition of Tradition in some significant ways. Oftentimes, it seemed strained, as he tried
to combine the two streams of thought.

I should add a disclaimer here. I myself have a great respect and admiration for the German people and their
accomplishments in the arts, music, science, and philosophy. Since the thinkers about to be discussed were trying to
come to terms with the fundamental and hidden structures of the world, their ideas deserve to be carefully considered.

German philosophy is a series of footnotes to \textbf{Immanuel Kant}, who tried to reconcile empiricism with traditional
metaphysics. In the \textit{Critique of Pure Reason}, Kant demonstrated that, starting solely from empirical data, the pure
intellect was incompetent to know reality in itself. This is the opposite of metaphysics whose fundamental claim is
that the Intellect knows reality through a direct intuition of the ideas, or forms. In one way, this first critique can
be read as a \emph{reduction ad absurdum} proof that empiricism, and, \emph{a fortiori}, the scientific method, is
false as paths to knowledge. Instead, Kant went in another direction.

He wrote the \textit{Critique of Practical Reason} in response. Since we have direct experience of ourselves as moral beings,
i.e., beings acting in the world, this can be the only source of truth. To make sense of the moral life, Kant
postulated the existence of God, the freedom of the will, and immortality as fundamental truths. However, he did not
intuit these truths as a metaphysician, but saw them as logically necessary axioms.

From the Traditional point of view, the Intellect is prior to the Will; Kant reversed this, denying the Intellect its
priority, and making the Will fundamental. The corollary was that action was the means to knowledge, a point not lost
on Evola. Post-Kantian German philosophy developed this philosophy of the Will.

\textbf{Arthur Schopenhauer}'s system was the most extreme. Since Mind is not the fundamental reality, it
was the Will. But, unlike Kant, for Schopenhauer the Will is itself unknowable; since it is not directed by the Logos,
it is blind and irrational. We do not know that Will directly, but only through its representations or appearances,
what we call the “world”. The Will becomes dual, splitting into the knowing subject and the known object. But this is
all illusion; when the duality is abolished, the individual will dissipates, and there is only the Will.

\textbf{Johann Fichte} did not go so far. In his understanding, the “I” is itself the noumenal reality, not the abstract
Will, but one's own will. As Kant showed, our conception of the world cannot derive from the world, which
is noumenal and unknowable. Fichte, accepting Kant's postulate of the freedom of the will, concludes from
this that our conception of the world must be a free creation of the mind. Since the I is primary (I experience it
directly, although as subject and never as object), it is the phenomenal world that is derivative. A morally acting I
creates that world as its field of action; without something to oppose it, the I cannot be “moral” in any real sense.
This all follows logically from the initial assumptions.

The echoes of this manner of thinking resound in Evola's philosophical works, as is obvious from \textit{The
Individual and the Becoming of the World}\footnote{\url{http://www.gornahoor.net/?tag=the-individual-and-the-becoming-of-the-world}}, and what he writes about the “I” or the Absolute Self. It is the source of
most, if not all, of his divergences from \textbf{Rene Guenon}. Clearly, Evola's claim that action is a way
of self-realization is based on this type of philosophy.

Nevertheless, it is not without danger. It may be hard to accept that the physical world is the creation of my “I”, but
if we restrict ourselves to the social world, it gains much more credibility. Specifically, it reveals itself in the
modern idea that our social world is a construct. Two hundred years later, this idea has become commonplace, even among
those who have no idea who Fichte is, and, in any case, could not even understand him. To the modern mind, then, there
is no objective social reality, as it is simply a free creation of the mind. Hence, by changing our conception of it,
we can mould our society any way we please. Still following Fichte, this is not simply an intellectual exercise, rather
it is a moral quest. Therefore, those who reject the dominant conception are experienced as ignorant, as immoral, as
enemies, as mortal enemies.

Furthermore, the postmodern mind fully embraces Fichte, seeing the physical world itself as a free creation of the human
will. This leads logically to causes such as man-made climate change. Even more, what may seem to be undeniable
physical differences, such as sexual and racial diversity, are themselves regarded as products of the human will. For
those who have followed thus far, this is all too obvious to require further elucidation.

There is a serious consequence: this philosophy cannot be countered on its own terms. It is pointless to mention
biological realities to those who do not even regard them as independent of human conception. You cannot try to create
a counter-conception, which logically makes no sense.

Yet this is the direction of modernity and postmodernity. If follows its own logic and to deny that logic is seen as a
moral failing. There can be no discussion with such a point of view; that is why Evola appreciated Donoso Cortes so
much. Guenon insisted that only an intellectual conversion can overcome that perspective.

I know this will not please many people, since it seems to be too passive. They believe in debate and confrontation. But
those who understand Tradition will agree that one's own intellectual conversion must come first. Then,
instead of debate, an alternate worldview must be presented and events understood in the terms of a new historiography.\footnote{A different way to read Fichte, as well as the influence of Nietzsche, will have to await another day.}

\flright{\itshape Posted on 2012-11-06 by Cologero}

\begin{center}* * *\end{center}

\begin{footnotesize}\begin{sffamily}

\texttt{Kaulaphon on 2012-11-07 at 05:52 said: }

“All I ever talked about was masturbation.” – Derrida. 

The interesting thing though, is that when postmodernism turns everything on the battlefield into volitional
happenstance to be assembled anew, the assembly in most cases ends up with a set of prefabricated moralistic notions
chosen prior to entering the philosophical field. 

Even when this kind of intellectual assault is turned upon itself, so to speak, the end product of philosophoical
deconstructionism is shown to be equal to the deconstructionist's personal pet beliefs. 

Postmodern analysis consists of playing language games in order to give oneself the power to justify ones own beliefs
(this is the essence of ALL philosophy, if one would believe Mad Freddy). In many respects, it is just a complicated
and labyrinthine way of communicating very simple notions. This is why postmodern analysis never produces any novel
results. This is why postmodern analysis never ends up defending the difference between the sexes, or defending
colonialism, or finding that social hierarchy is just. 

Choosing from a set of arbitary constructs, the constructs willed into manifestation are always the same ones. Quaint,
aint it?

\hfill

\texttt{Saladin on 2012-11-07 at 19:24 said: }

I am not an expert on either Kant or Fichte but I have read Schopenhauer and I think that your description of his
philosophy is not entirely accurate. True, Schopenhauer's philosophy was indeed most extreme in its
pessimism. However, he in fact postulates that The Will is knowable and that “Kant had ignored inner experience, as
intuited through the will, which was the most important form of experience. Schopenhauer saw the human will as our one
window to the world behind the representation; the Kantian thing-in-itself. He believed, therefore, that we could gain
knowledge about the thing-in-itself, something Kant said was impossible, since the rest of the relationship between
representation and thing-in-itself could be understood by analogy to the relationship between human will and human
body. According to Schopenhauer, the entire world is the representation of a single Will, of which our individual wills
are phenomena. In this way, Schopenhauer's metaphysics go beyond the limits that Kant had set, but do not
go so far as the rationalist system-builders that preceded Kant. Other important differences are
Schopenhauer's rejection of eleven of Kant's twelve categories, arguing that only causality was
important. Matter and causality were both seen as a union of time and space and thus being equal to each other.”

\hfill

\texttt{Jason-Adam on 2012-11-08 at 02:10 said: }

The Abbe Barruel, exposer of the true powers who were behind the French revolution, wrote that Immanuel Kant was a
member of the Illuminati in volume 4 of Memoirs sur l'histoire du Jacobinisme…..it is a pity he died before
he was able to complete his refutation of Kantianism.

Another fact I am surprised more people on the right are not aware of is how in the “Protocols of Learned Elders of
Zion” Nietzsche is classed along with Marx and Darwin as agents of destruction. Again, we know the Protocols are not as
simple as they appear but I do think they are worthy of being studied, as Evola thought so, and it is puzzling to me
that no one seems to have picked up on the anti-traditional role ascribed to Nietzsche in said document.

Going a bit further, in the Protocols, the “Jews” (not to be taken as meaning Jewish people but something else entirely)
claim that their only worthwhile enemy is the Society of Jesus. Could it be that the Jesuits are, or are the only thing
close to being, the Western spiritual elite discussed by Guenon ? I wish somone would investigate whether there is any
initiatory character in the exercises of St Ignatius as well as the other works of Jesuit mysticism.

\hfill

\texttt{Cologero on 2012-11-11 at 10:44 said: }

Nietzsche is a complex figure. Yes, he is destructive insofar as he opposed the established order, often in false or
unwise ways. On the other hand, he can be read another way, as we showed with Coomaraswamy. There is a small group,
seeing no alternative, that turns to Nietzsche as a source of a new spiritual vision. Whether that leads to anything or
not remains to be seen. I think they are overly polemical and reject too much of Western history to be completely
effective.

The Ignation exercises are similar to the Hermetic meditation described by Tomberg. Their organizational structure and
goal to serve as a transnational spiritual elite point to something initiatory. The cause of their current state of
dissipation is unknown to me.

\hfill

\texttt{Jason-Adam on 2012-11-11 at 22:30 said: }

Someone told me once that he believed the Jesuits had adopted an outward “leftist” political stance as a means of
bringing about the end of the present age sooner by increasing the speed of destruction \& degeneracy, a “ride the
tiger” type scenario. Whether this is true or not I can not say. There is also the possibility that the Jesuits have
been infiltrated and “switched” from a force for good into a force for evil.

Do you know why more traditionalist writers have not discussed Ignatian spirituality ?

\hfill

\end{sffamily}\end{footnotesize}
