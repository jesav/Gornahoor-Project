\section{Social Surgery}

Man as a physical being is subject to gravity and other laws of physics, not to mention chemical laws. As part of biological life, he is also subject to various biological laws. In particular, this includes the two scientific laws of neo-Darwinism (incorrectly called the ``theory of evolution"). Darwin's insight relied on these two principles.

\begin{enumerate}
\item \textbf{Variation} refers to random changes. 
\item \textbf{Selection} refers a method to select or prefer certain changes over others. 
\end{enumerate}
As such, that is not yet a scientific theory. For example, the eminent biologist Richard Dawkins created a thought experiment in which the phrase ``METHINKS IT IS LIKE A WEASEL" could arise through random variations. First, there is a random variation of letters. Then a selection process will select the letters that most resemble the target phrase until it is eventually reached. Unfortunately, this sneaks in two metaphysical principles:

\begin{itemize}
\item \textbf{Final cause}: the selection principle is goal oriented. 
\item \textbf{Formal cause}: the phrase is a possibility of manifestation. If, for example, the letter W was stuck on the typewriter, no amount of time could produce the phrase. 
\end{itemize}
A properly scientific theory, on the other hand, relies only on material and efficient causes. Hence, the following modifications define the biological theory.

\begin{enumerate}
\item \textbf{Genetic Variation} refers to random mutations in the genotype that are inherited by the descendants. 
\item \textbf{Natural Selection} refers to the survival and reproductive success of the phenotype within its environment, thereby preserving those mutations. 
\end{enumerate}
When I was a boy taking classes at the Boston Museum of Science, we were taught that cosmic rays created genetic mutations. Now it is accepted that the copying process itself is subject to imperfections. Interspecies mating may also possibly create genetic variations. Since most mutations are deleterious, presumably ``nature" will select the best ``fit" offspring to survive and reproduce, while the least fit will produce no descendants. The selection process is still in dispute, e.g., the role of group selection vs kin selection. Darwin claimed there was a sexual selection, which we see in the tendency toward assortative mating\footnote{\url{https://en.wikipedia.org/wiki/Assortative_mating}}, especially in humans. Of course, there is artificial selection in which humans create various breeds of certain animals for designed purposes.

As such, there is nothing objectionable to neo-Darwinism, since variations and selection can be observed. However, there are four things that this theory does not account for, although the popular imagination often believes so.

\begin{itemize}
\item \textbf{Completeness}: variation and selection do not account for all the features of the phenotype. Specifically, the process does not explain how consciousness, thought, etc., arise. It just doesn't, no matter what you hear. A scientific theory needs to explain all the steps involved. 
\item \textbf{Descent}: man, for example, does not descend from a ``monkey". No biologist claims that a monkey gave birth to a human. A true descendant will contain the genetic material of the parents. A new species arises, according to the theory, when the variation is sufficiently large to be considered something different; i.e., it is not a descendant in that sense. 
\item \textbf{Complexity}: the theory does not explain emerging complexity, or in other words, there is no ``direction" to evolution, even if it appears that way. There is no good definition for complexity, and ``evolution" could just as randomly produce less complex beings. Actually, half the biomass consists of single-celled organisms and they will continue to exist when multicellular organisms become extinct. Neo-darwinism may offer an explanation for the possibility of the development of more complex life forms, but not an explanation for its necessity. 
\item \textbf{Eugenics vs fitness}: there is no ``moral" basis to survival. ``Fit" just means fit to survive. ``Bigger, stronger, faster" are irrelevant. Certain life forms will survive better as the human population density increases. For example, rats thrive in human cities. Social parasites like dogs and housecats do so likewise. There is a tacit agreement with livestock and poultry that they will be allowed to breed and propagate their genes in return for becoming foodstuff for humans. 
\end{itemize}
\paragraph{Social Surgery}
Although man is subject to the laws of biology, he also transcends biology. Given the knowledge of variation and selection, to what extent then should man direct his own future (biological) evolution? Some geneticists claim that the human race is in genetic decline\footnote{\url{http://iqpersonalitygenius.blogspot.co.uk/2015/08/if-humans-are-recapitulating-mouse.html}}. This is due to the accumulation over generations of deleterious mutations. In previous eras, high childhood mortality presumably would cull the genetically less fit from the population.

That is probably true, but it lacks biological relevance. The selfish gene theory stipulates that the genes strive to perpetuate themselves. Whether the organism is intelligent, pretty, or healthy or not is irrelevant, as long as the organism is able to reproduce itself. Nevertheless, such a prospect is disheartening to intellectuals who overvalue their intelligence, especially if they also misunderstand the point of life. They have various utopian ideals, never compatible with each other.

As a thought experiment, the British philosopher \textbf{Francis Bradley} posed an interesting question about the relationship between science and human progress: to what extent should scientific knowledge influence and direct social policy? Another way to pose it is this: should science be used to oppose the roots of order in favor of an ideological system or should it reinforce that order?

Despite the claims of the educated classes that they rely on science, scientific knowledge plays little role in social policy. Pressing issues in economics, crime, education, and so on, are more tractable than they appear, although there is little will to actually employ effective measures whenever they might conflict with ideological presuppositions. Bradley goes directly to the heart of the issue. In the case he defends, he acknowledges that there will be religious opposition to his proposal.

Bradley proposed what he called ``social surgery", which includes compulsory euthanasia. Writing shortly after Darwin, Bradley noted:

\begin{quotex}
We have the moral code of Christianity … but we do not realize how in its very principle the Christian ideal is false … Darwinism seems destined to intervene. It will make itself felt, I believe, more and more effectually. It may force on us in some points a correction of our moral views and a return to a non-Christian and perhaps a Hellenic ideal. 

\end{quotex}
He was correct about the destiny of Darwinism, but for the wrong reason:

\begin{quotex}
The community, though it may have grown naturally to be what it is, should now more or less consciously regulate itself, and deliberately play its own Providence. 

\end{quotex}
Specifically, given that the struggle for existence has been ameliorated, the inferior types are not weeded out naturally. Hence, the community must take on the ``selection" task itself, since it can no longer rely on natural selection. He realized that certain religious attitudes would cause opposition:

\begin{quotex}
[For Christianity] the individual in the next world has an infinite value; the things of this world, our human ends and interests, are all alike counted worthless … the good of the whole can confer no right to interfere with its members. … Once admit that life in this world is an end in itself, and the pure Christian doctrine is at once uprooted. For, measured by that end and standard, individuals have unequal worth … the community is itself its own Providence.

The right of the individual to spawn without restriction his diseased offspring on the community, the duty of the state to rear wholesale and without limit an unselected progeny—such duties and rights are to my mind a sheer outrage on Providence. A society that can endure such things will merit the degeneracy which it courts. 

\end{quotex}
Of course, from a strictly biological perspective there are indeed no such ``rights". From genetic selection alone, parasitism (no more overtones intended) is often a fit strategy, as long as the host is not destroyed. On the other hand, if group selection is valid, then Bradley's proposal is merely an example of its manifestation.

As such, it is a perversion of Providence. As we have recently pointed out in the essay on \textit{Predestination and Predilection}\footnote{\url{https://gornahoor.net/?p=8195}}, individuals have unequal worth, even from the perspective of Providence. However, the criterion of ``worth" may be quite different. If man is solely biological, then Bradley must be correct. On the other hand, if man's true end is indeed transcendent, then what we consider to be of worth is quite different.

A traditional society will not endure many things that a modern society, and presumably Bradley himself, not only endures but promotes. The degeneracy of the modern world has been courted for much longer than Bradley realizes; there is a moral degeneracy that is more deleterious than any biological mutation.

\paragraph{Nota Bene}
As an unanswered objection, and food for thought, we end with two quotes from the Jesuit paleontologist, \textbf{Teilhard de Chardin}:

\begin{quotex}
How should we judge the efforts we lavish in all kinds of hospitals on saving what is so often no more than one of life's rejects? Something profoundly true and beautiful (I mean faith in the irreplaceable value and unpredictable resource contained in each personal unit) is evidently concealed in persistent sacrifice to save a human existence. But should not this solicitude of man for his individual neighbour be balanced by a higher passion, born of the faith in that other higher personality that is to be expected, as we shall see, from the world-wide achievements of our evolution?

To what extent should not the development of the strong (to the extent that we can define this quality) take precedence over the preservation of the weak? How can we reconcile, in a state of maximum efficiency, the care lavished on the wounded with the more urgent necessities of battle? In what does true charity consist?

\flright{\textit{Human Energy}}

So far we have certainly allowed our race to develop at random, and we have given too little thought to the question of what medical and moral factors must replace the crude forces of natural selection should we suppress them. In the course of the coming centuries it is indispensable that a nobly human form of eugenics, on a standard worthy of our personalities, should be discovered and developed. Eugenics applied to individuals leads to eugenics applied to society.

\flright{\textit{Phenomenon of Man}}

\end{quotex}


\flrightit{Posted on 2015-08-19 by Cologero }

\begin{center}* * *\end{center}

\begin{footnotesize}\begin{sffamily}



\texttt{Sparrow on 2015-08-29 at 09:10 said: }

One thing that lead me to reject Darwinism is the unspoken but taken for granted belief that assumes that all animal life is equal (i.e that all possibilities are open to forms of life). There was an instance some years back where Richard Dawkins defended a quack who believed that humanoid dinosaurs were a possibility. One wonders (with the exception of the Traditionalist) why we haven't seen dog people, cat people, or spider people.


\end{sffamily}\end{footnotesize}
