\section{Philosophic Idealism}

It is easy to be unaware that up until World War II, idealism was the dominant philosophical position of Europe. In the 18th and 19th century, the Germans — particularly Kant, Hegel, Fichte, Schelling, Schopenhauer — took the leading role in its development. Later, the Britons such as Green, Bradley, Bosanquet, Collingwood took it in their own direction. In France, it usually went by the name of personalism or spiritualism. In Italy, while Evola was a young man, philosophy was dominated by the idealists Benedetto Croce and Giovanni Gentile. It's hard to believe that at their peak, they were world-renowned philosophers. Although friends and collaborators, they split over the question of Fascism. Gentile was for a while the education minister and his philosophical system was very influential in moulding the intellectual roots of Fascism.

Yet idealism was not a modern development; the mainstream of Western philosophy is the history of idealistic thought. When the European orientalists began studying, translating, and cataloguing the early Indo-Europeans systems of the Vedas, its compatibility with idealism was noted. So, idealism can rightly be seen as ``our" tradition and competitors such as nominalism, positivism, and materialism are more like a persistent anti-tradition. I would recommend the short book \textit{Nobilitas} by Alexander Jacob for a summary of this tradition from Plato until World War II. Ironically, Prof. Jacob is an Indian who is keeping the Western philosophical tradition alive, when Europeans are abandoning it.

Evola was heir to this tradition and his intellectual development took place in the milieu of Italian idealism in the 1920s. In order to study idealism thoroughly, Evola learned German so he could read the philosophical sources in their original language. Out of his studies, he created his own system which he named ``magical idealism". In The Individual and the Becoming of the World, Evola ties together the main elements of his system. One can perhaps recognize Schopenhauer when Evola speaks of in the world as will and representation, or Stirner in the idea of the Absolute Ego, and Plotinus in the idea of privation and the evil of matter.

But Evola was not content in outlining an abstract intellectual system. Ultimately, there can be no system per se, since what is important is the will and the development of its freedom to create, remake, and define the world. This means that without having made the effort at self-transformation through the various phases of consciousness, then his system cannot be properly comprehended. So ultimately, Evola's system cannot be reduced to a set of propositions to learn and memorize. Nor is there any technique, drug, or practice that will develop the will and lead to higher stages of consciousness.

In our day, when ``tough minded" thinkers are drawn to science and materialism (the neo-Darwinist Richard Dawkins was recently ``voted" the most intelligent man in Britain), the claims of idealism can seem incredible. Since Evola simply assumes a basic familiarity with this tradition, it may be difficult sometimes to see what he is getting at. For those new to idealism, I would recommend The Philosophy of Schopenhauer by Bryan Magee for a clear overview of the presuppositions and methods of idealism. Whether or not Evola adds to this tradition in a coherent and constructive way is for each man — who has made the requisite effort — to decide.



\flrightit{Posted on 2008-06-08 by Cologero }

\begin{center}* * *\end{center}

\begin{footnotesize}\begin{sffamily}



\texttt{Jason-Adam on 2013-03-25 at 09:16 said: }

Does Gornahoor still stand by this ?


\hfill

\texttt{Cologero on 2013-03-25 at 13:21 said: }

Granted, Jason-Adam, that was an early post and it probably would not be written the same way today. In the inquisition, the inquisitor would pick out sentences that required explanation or justification. It seems fair enough. So, instead of a vague question, which sentences would you like explained? Most of it is bland and refers to historical movements or facts. Is there a factual error?

The intent was to relate it to Evola's method in particular. Is that paragraph perhaps problematic?


\hfill

\texttt{Jason-Adam on 2013-03-25 at 13:49 said: }

The words idealism and realism can be construed differently – as Platonism can be described as both by different authors…..so when you describe idealism as our tradition and realism as the anti-tradition….what do you mean ? by realism do you mean Plato ?

idealism not a modern development…..by idealism do you mean the doctrine that earthly forms are merely reflections of heavenly realities or do you mean the system derived from Kant ?


\hfill

\texttt{Cologero on 2013-03-25 at 21:21 said: }

A reader should make a good faith effort to read an article in the best possible light. In that case the article is clear enough, although perhaps I would now describe ``positivism and materialism" as the anti-tradition to make it clearer. I am somewhat satisfied with the wikipedia definition of Idealism\footnote{\url{http://en.wikipedia.org/wiki/Idealism}} as:

\begin{quotex}
idealism is the group of philosophies which assert that reality, or reality as we can know it, is fundamentally mental, mentally constructed, or otherwise immaterial. 

\end{quotex}
The intent is not to define that tradition exactly, but rather to put Evola's philosophy of Magical Idealism within it. That is worth commenting on. Perhaps it should be read in conjunction with \textit{The Great Divide and our Ownmost}\footnote{\url{http://www.gornahoor.net/?p=2244}}.


\hfill

\texttt{Obscure Lifeform on 2014-12-05 at 19:21 said: }

The term `realism' only has a relative or contingent meaning. Idealism and materialism are properly speaking the two sides of the Cartesian equation of `subject' and `object'. For Descartes what constitutes the `objectivity' of an object is the capacity to be measured, quantified and mechanized by a human `subject'. Most materialists after Descartes were usually just mechanical scientists who accepted Cartesian objectivity as the appropriate manner of describing scientific activity and its primacy as regards knowledge. Kant on the other hand locates objectivity in self-consciousness, which for him means moral behavior; thus the Cartesian `subject' is primary since objectivity is derived from self-reference. Kant also lays out the new meanings for terms like `realism’, ‘idealism’, ‘transcendental'. etc. which are understood in these ways by all academics after him. Also, after Kant we see an obssession with applying these labels anachronistically to many diverse thinkers and doctrines of the past. This `categorizing' tendency brings one on to Hegel and all later thinkers who generalize and filter the past for their own ends or as Hegel writes; ``Thinking is generalizing".

Realism in the medieval sense stands in contrasting relation to nominalism. The realist asserts that universal essences (Platonic ideas) are real or fundamentally real and knowable; Thomism is a realist doctrine which asserts that universal essence are real and knowable, but only within the confines of personal experience, hence Thomas is often labelled a `moderate realist' or `Aristotelian realist'. platonic realism would be exemplified by the neoplatonists and medieval Augustinians. An important Platonist who also wrote at the end of the medieval era was Denis the Carthusian of the 15th century; he produced commentaries and histories of medieval thought, often blaming the downfall of medieval intellectuality on Duns Scotus and William of Ockham. Denis admired Thomas Aquinas, but was avowedly a disciple of Henry of Ghent and Pseudo-Dionysius. 

Nominalists on the other hand believe that only the particular essences of individual substances are knowable or real, thus psychologizing all general concepts. For the nominalist, everything is in reality absolutely particular. I'm sure you can see how this leads to pure quantification and Descartes' thought.


\hfill

\texttt{Cologero on 2014-12-21 at 11:44 said: }

I'm not quite sure of your intent, Obscure Lifeform; I presume you are clarifying some classifications. If so, I would add this:

\begin{quotex}
The theory [of the Scholastics] is described not as idealism, but as realism; but this does not imply that they are in conflict with the doctrine of Augustine; it means rather that the ideal principles possess real validity, that as ideas they subsist in the Divine mind before the things corresponding to them are called into existence, while, as forms and essences, they really exist in nature and are not really products of our thinking.

Source: Idealism \url{http://www.newadvent.org/cathen/07634a.htm}

\end{quotex}
I don't think I understand how ``knowing" could be separate from the personal experience of knowing.


\end{sffamily}\end{footnotesize}
