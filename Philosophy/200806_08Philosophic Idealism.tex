\section{Philosophic Idealism}

It is easy to be unaware that up until World War II, idealism was the dominant philosophical position of Europe. In the
18th and 19th century, the Germans — particularly Kant, Hegel, Fichte, Schelling, Schopenhauer — took the leading role in its development. Later, the Britons such as Green, Bradley, Bosanquet, Collingwood took it in
their own direction. In France, it usually went by the name of personalism or spiritualism. In Italy, while Evola was a
young man, philosophy was dominated by the idealists Benedetto Croce and Giovanni Gentile. It’s hard to
believe that at their peak, they were world-renowned philosophers. Although friends and collaborators, they split over
the question of Fascism. Gentile was for a while the education minister and his philosophical system was very
influential in moulding the intellectual roots of Fascism.

Yet idealism was not a modern development; the mainstream of Western philosophy is the history of idealistic thought.
When the European orientalists began studying, translating, and cataloguing the early Indo-Europeans systems of the
Vedas, its compatibility with idealism was noted. So, idealism can rightly be seen as “our” tradition and competitors
such as nominalism, positivism, and materialism are more like a persistent anti-tradition. I would recommend the short
book \textit{Nobilitas} by Alexander Jacob for a summary of this tradition from Plato until World War II. Ironically, Prof.
Jacob is an Indian who is keeping the Western philosophical tradition alive, when Europeans are abandoning it.

Evola was heir to this tradition and his intellectual development took place in the milieu of Italian idealism in the
1920s. In order to study idealism thoroughly, Evola learned German so he could read the philosophical sources in their
original language. Out of his studies, he created his own system which he named “magical idealism”. In The Individual
and the Becoming of the World, Evola ties together the main elements of his system. One can perhaps recognize
Schopenhauer when Evola speaks of in the world as will and representation, or Stirner in the idea of the Absolute Ego,
and Plotinus in the idea of privation and the evil of matter.

But Evola was not content in outlining an abstract intellectual system. Ultimately, there can be no system per se, since
what is important is the will and the development of its freedom to create, remake, and define the world. This means
that without having made the effort at self-transformation through the various phases of consciousness, then his system
cannot be properly comprehended. So ultimately, Evola’s system cannot be reduced to a set of propositions
to learn and memorize. Nor is there any technique, drug, or practice that will develop the will and lead to higher
stages of consciousness.

In our day, when “tough minded” thinkers are drawn to science and materialism (the neo-Darwinist Richard Dawkins was
recently “voted” the most intelligent man in Britain), the claims of idealism can seem incredible. Since Evola simply
assumes a basic familiarity with this tradition, it may be difficult sometimes to see what he is getting at. For those
new to idealism, I would recommend The Philosophy of Schopenhauer by Bryan Magee for a clear overview of the
presuppositions and methods of idealism. Whether or not Evola adds to this tradition in a coherent and constructive way
is for each man — who has made the requisite effort — to decide.

\flright{\itshape Posted on 2008-06-08 by Cologero}
