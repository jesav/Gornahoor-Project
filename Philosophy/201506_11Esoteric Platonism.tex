\section{Esoteric Platonism}

\begin{quotex}
He was awaiting the city with foundations whose architect and demiurge is the God. \flright{\textsc{Hebrews} 11:10}

God bears in the intelligible world to reason and its objects the same relation which the sun bears in the visible world to sight and its objects. \flright{\textsc{Plato}, \emph{Republic}} 

\end{quotex}
Besides Stoicism, Platonism and Aristotelianism were also reworked in the transition from the Philosophical to the Religious consciousness in the West. Again, this process involves seeing the earlier philosophies in a large context while bringing out its deeper meanings in the light of transcendent revelation. The Philosophical consciousness was focused on Thought, whereas the Religious consciousness was focused on a change in the level of Being. There are two claims that can be made explicit:

\begin{enumerate}
\item Plato, Aristotle, and the Stoics together present the highest intellectual teachings in Pagan civilization. 
\item The way the Church Fathers reworked those teachings is an advancement. 
\end{enumerate}

Claim (1) is really indisputable. The neopagans may reject it typically through a philosophy of “vitalism”, which regards the body and soul life as primary, thereby rejecting a higher intellectual or spiritual life. This position can be rejected philosophically, but poses a danger either by distracting those otherwise interested in Tradition, or by coming to power in an orgy of destruction.

Claim (2) speaks for itself. While absorbing the rationalism of Greek thought, it adds the empirical element, a sort of “metaphysical positivism”. However, since it requires a change of Being, it cannot just be “thought”, it is beyond thought. Spiritual exercises are necessary. This combination of rationality and spirituality is rare, either because of lack of a suitable guide or lack of motivation.

Oddly enough, claim (2) is typically rejected by the “super-correct” who reject this introduction of “pagan elements” into the purity of the Gospels. This is often accompanied by an unconscious crypto-paganism that imagines a Zeus-like god residing in a material heaven not much different from Mount Olympus, all while claiming to be Christian.

The following will all make sense to men of Tradition, since Guenon regards it as an authentic metaphysic. What, then, is required, is an intellectual conversion. That is, to begin to see and understand man, the world, and God in the light of these teachings.

\paragraph{Introduction}
\textbf{Constantine Cavarnos} specifically mentions the following topics from Plato and Aristotle as the most influential on the Fathers. The Fathers did not regard the two philosophers as essentially opposed to each other. We will provide a brief introduction to each of them, usually by combing them into a single section.

\begin{table}[h]
\centering
\begin{tabular}{rl}\toprule
\textbf{Plato} & \textbf{Aristotle}\\\midrule
Sensible and Intelligible Realms & Matter and Form\\
God as Demiurge & Conception of God\\
Tripartite division of soul & Immaterial Being\\
Four Chief Virtues & Categories of existence\\
Unity of the Virtues & Moral excellence or Virtue\\
Virtues as Beautiful & Four causes
\\\bottomrule
\end{tabular}
\caption{Concepts from Plato and Aristoteles}
\end{table}

\paragraph{World of Ideas}
Plato distinguished between the sensible realm and intelligible realm. \textbf{Julius Evola} described the same teachings in the first chapter of \emph{Revolt Against the Modern World}.

\begin{itemize}
\item \textbf{Sensible Realm}. Physical order, visible, perceived by the bodily senses, changing, phenomena, appearances, destructible or mortal things, 
\item \textbf{Intelligible Realm}. Metaphysical order, invisible, apprehended by the mind, ideas, unchanging, indestructible or immortal things 
\end{itemize}
While Plato was unsure of the precise nature of the intelligible realm, the Fathers realized that the ideas pre-existed in the Mind of God. This is what \textbf{Rene Guenon} claims in the \emph{Multiple States of Being}.

To be clear, this teaching is not a “theory”, one possible worldview among many others. Rather it is knowledge or gnosis itself. No one is clearer about this than \textbf{Julius Evola}, not even the Christians. He writes in the very first chapter of Revolt Against the Modern World:

\begin{quotex}
Anywhere in the world of Tradition, both East and West and in one form or another, this knowledge has always been present as an unshakable axis around which everything revolved. Let me emphasize the fact that it was knowledge and not “theory”. As difficult as it may be for our contemporaries to understand this, we must start from the idea that the man of Tradition was aware of the existence of a dimension of being much wider that what our contemporaries experience and call “reality”. 

\end{quotex}
\paragraph{Hylomorphism}
Aristotle expressed a similar teaching as the distinction between matter and form, in which matter stands for the sensible realm and form (or Idea) correspond to the intelligible realm. Aristotle mistakenly rejected the independent existence of the forms or ideas. Of course, the Fathers rejected Aristotle's belief that matter and form are uncreated.

This doctrine is called Hylomorphism. Guenon pointed out similar teachings in the Samkhya school of Vedanta.

\paragraph{God}
Plato and Aristotle contributed to the understanding of God, although in different ways. From Plato, came the idea of God as Demiurge, artist, architect, creator.

Platonic ideas:

\begin{itemize}
\item God is beyond being 
\item God created the world as a likeness to an eternal model 
\item The “inexpressible beauty” of God. 
\item The physical universe is beautiful 
\item God is the Idea of the Good 
\item God bears in the intelligible world to reason and its objects the same relation which the sun bears in the visible world to sight and its object 
\end{itemize}
The main objection is that Plato believed that matter was eternal and the Demiurge merely molded it; of course, matter is also a creation of God. These are some corollaries of Plato's conception:

\begin{itemize}
\item If God is beyond being, he cannot be a being Himself. Most people today imagine God as some powerful being out there. 
\item The world did not arise from material processes by chance. Rather it has developed in conformance with a Divine ideal. 
\item As the cosmos is good and beautiful, neither Plato nor the Christians are “world denying”. Rather, the world is understood as an element in a hierarchy of being. 
\item God is not some great being “out there, right now”. Rather, He is like the Sun in our consciousness, bringing the intelligible realm of ideas into our awareness. 
\end{itemize}
While Aristotle misunderstood that God was the Creator and Providential (unlike Plato in both cases), he can still give us a deeper understanding of God:

\begin{itemize}
\item God is a substance 
\item God is immaterial 
\item God is unmoved 
\item God is impassive 
\item God is pure act 
\end{itemize}
Since we are called upon to strive to attain likeness to God, we need to reach the state of passionlessness. Don't forget that this means freedom from “negative emotions”, not from all emotions.

\paragraph{The Soul}
Although Plato and Aristotle seem to have different conceptions of the soul, they are harmonizable. The Fathers accepted Plato's tripartite division of man as body, soul, and spirit as well as his teaching on its powers or parts. However, unlike Plato, they held than man is a body-soul, not just a discarnate soul. They likewise accept the three powers of the soul:

\begin{itemize}
\item \textbf{The appetitive} (\emph{epitheymetikon}). Directed toward sensual pleasure and material gain. 
\item \textbf{The spirited} (\emph{thymos}). Directed toward ruling, conquering, fame. 
\item \textbf{The rational} (\emph{logistikon}). Directed toward the true, the good, and the beautiful. 
\end{itemize}
The Fathers broadened the understanding of these powers. The appetitive function can be sublimated and directed toward what is really necessary to be fully human: to the virtues and to God and His Will.

The spirited function should be directed against inordinate and wrong desires, but also against demons, for we wrestle not just against flesh and blood.

The Fathers include inner attention, meditation, and prayer in the rational function. Thus we see that there is both a lower and a higher aspect to each function.

In association with these functions, there are three possible states:

\begin{itemize}
\item \textbf{Contrary to Nature}. One of the nonrational powers governs the soul, and the rational part is enslaved. 
\item \textbf{In Accordance to Nature}. The rational part of the soul governs the whole soul. 
\item \textbf{Above nature}. One lets God rule the soul its thoughts, feelings, desires, and so forth. 
\end{itemize}
Plato only recognized the first two, which is the limit of the philosophical consciousness, while the Fathers realized the state above nature. \textbf{Mark the Ascetic} describes this state as:

\begin{quotex}
where the mind finds the fruits of the Holy Spirit, which the Apostle Paul called love, joy, peace, and so on. 

\end{quotex}
\textbf{John Climacos} says that:

\begin{quotex}
in this state one has the indwelling God Himself governing him in all his word, deeds, and thoughts. Wherefore through illumination he apprehends the will of the Lord within himself as a certain voice and transcends every human teaching. 

\end{quotex}
\paragraph{Immaterial Being}
Besides composite substances (i.e., composed of form and matter), Aristotle recognized the existence of simple, immaterial things consisting of form without matter. God, angels (which Aristotle believed to be subordinate gods), and the soul are immaterial.

In particular, man's intellectual center is immaterial. That means it can know essences directly, intuitively, without “becoming” the thing in matter. These we know through thinking, or correct thinking, since most of our thinking is either contrary to nature or contrary to God. Here, Esoteric Stoicism teaches us the importance of discriminating our thoughts and rejecting the useless and harmful ones.

Thus, our experience, say, of demons comes through our thoughts, not from artistic, or not so artistic, pictures. Hence, demonic activity in our consciousness is not so easy to recognize, since it is experienced as one thought among others, not in terms of a sensible image of an ugly demon. Au contraire, the thought may appear quite beautiful and pleasing to your self-esteem.

Another way to understand thoughts is the experience of what Guenon calls “possibilities of being”, i.e., they may be experienced as various thoughts or impulses. Of course, the free man transcends this and can decide whether or not to act on such thoughts, while the ordinary man simply accepts most everything that crosses his mind.

\paragraph{Virtues}
Both Plato and Aristotle wrote on the virtues, and those ideas were further developed by the Fathers. Since these sections are rather long and we have written often on this topic, we will save it for another day.

\paragraph{Categories}
Aristotle identified 10 categories of being, which were used in different contexts by the Fathers, as gathered in Table~\ref{tab:201506_11Esoteric Platonism2}.

\begin{table}[h]
\centering
\begin{tabular}{ll}\toprule
\multicolumn{2}{c}{\bfseries Categories of Being}\\\midrule
Substance & Time\\
Quality & Position\\
Quantity & State\\
Relation & Action\\
Place & Passion
\\\bottomrule
\end{tabular}
\caption{Categories of being}
\label{tab:201506_11Esoteric Platonism2}
\end{table}

Fundamental is substance, both material and immaterial. Quality is given primacy over quantity. Things are in relations.

Space is different when speaking of material and immaterial substances. For the former, it refers to physical space and for the latter, mental space. For example, there are three different interior spaces:

\begin{itemize}
\item \textbf{Contrary to Nature}. The soul forgets or ignores God and His justice, an “unholy, demonic, place” rendered desolate by demonic, impassioned, negative thoughts. 
\item \textbf{In accordance with Nature}. The place of clear self-knowledge and repentance. 
\item \textbf{Above Nature}: the soul rises to prayer and experiences the fruits of the Holy Spirit, e.g., love, joy and peace. 
\end{itemize}
Spatial (above) and temporal (after) metaphors and symbols are used in spiritual writings to describe transcendent states. Not recognizing that these refer to inner or mental space, the common mind tries to imagine the transcendent in terms of a physical space “out there” or of a hereafter as a continuation of physical life in everlasting time. Rather, \textbf{Joseph Ratzinger} explains that Eternal Life is:

\begin{quotex}
The kind of life man may graciously come to possess in relationship with God who is life. Eternal life begins in this life through a person's knowing God and entering into communion with Him. 

\end{quotex}
“State” is also a misunderstood category. Particularly if one is attached to sensible images of spatial and temporal metaphors, it may be confusing to regard Heaven, for example, as a state of being. Once you understand that being human is itself a state of being, then it will make more sense. It does not deny that Heaven is a place, just that it is a material place.

The application of categories of being is a large topic. Begin understanding the world in terms of these categories in order to attain the intellectual conversion.

\paragraph{Causes}
Cavarnos does not mention Aristotle's doctrine of causes, but we've added it since it is very important, at least in the West. Again, begin by applying this doctrine to events in your life and in the world. As we've pointed out, Science rejected formal and final causes while retaining material and efficient causes. You will often read that science has “shown” that final and formal causes don't exist, somehow forgetting that was the assumption, not the conclusion. We will say more on this when describing the transition from the Religious to the Scientific consciousness.

\flrightit{Posted on 2015-06-11 by Cologero}

\begin{center}* * *\end{center}

\begin{footnotesize}\begin{sffamily}

\texttt{Alistair Fraser on 2015-06-11 at 04:11 said: }

A quite fabulous compression of high octane provocation to contemplation on the nature of being

\hfill

\texttt{Br. Giles Mary on 2015-07-08 at 21:27 said: }

This looks very good. I'm giving a brief talk to religious sisters tomorrow about masculine worship. I'm using some of Ratzinger's writings on symbols and The First Epistle of John where he writes, “I write unto you, young men, because you are strong, and the word of God abideth in you, and you have overcome the wicked one.” Amongst other things, I didn't know that Church Fathers explicitly taught the re-direction of the spirited function of the soul against demons, but I do now. Thanks, Cologero.

\hfill

\end{sffamily}\end{footnotesize}
