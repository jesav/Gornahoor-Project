\section{The Locust Conspiracy}

In which we explore the outer reaches of scientific knowledge including psychopathy, time travel, entropy, quantum
physics, conspiracy theories, and miracles.

\paragraph{Definitions}

An \textbf{opinion} is a proposition that is neither demonstrably false nor self-contradictory. Hence, it is the least
reliable form of knowledge; it may be falsified in the future. Emphatically, an opinion is not any thought that pops
into your mind.

A \textbf{belief} is a proposition that is actionable, that is, it will lead to action in the world. If I believe it
will rain on Saturday, then I won’t pack the picnic basket. That is how you can tell if someone really
believes what he claims.

\textbf{Not Even Wrong} is an argument that is neither correct nor incorrect because its premises are so off base or the
argument is confused. For example, Amalric of Bene in the 12th century was saved from the charge of heresy because the
investigators determined that his views were pure lunacy.

The opposite of a false proposition is a true proposition. The opposite of a “not even wrong” proposition is still
false.

\textbf{Psychopaths} and \textbf{Sociopaths} have a lack of conscience yet can appear charming to the unwary. Since they
are manipulative, they can do very well in achieving power in politics or business. It amazes me how few people can
recognize a psychopath. Haven’t they ever wondered how some people, usually not as competent or intelligent
as they are, manage to rise above them?

It is pointless to argue with a psychopath, since they get a thrill out of irritating the lesser beings below them. They
even boast about it. Also, the charge of hypocrisy against them is ineffective. Quite the contrary. They revel in
“getting away with it” and even love their hypocrisy to be made public. They often reveal themselves with a wry smile,
usually at inappropriate moments.

A \textbf{conspiracy theory} is an explanation that relies on a secretive cabal of sinister and powerful groups working
together to achieve a result, often over the course of generations. Obviously, it is hard to prove without being one of
the insiders. The irrational belief in a conspiracy theory is considered to be a mental illness: viz., \textbf{illusory
pattern perception}. the Wikipedia article \textit{List of conspiracy theories}\footnote{\url{https://en.wikipedia.org/wiki/List_of_conspiracy_theories}} includes the belief in a \textbf{white racist
patriarchy} to be one such theory.

The \textbf{Open Conspiracy}, envisaged by H G Wells\footnote{\url{https://forcingchange.wordpress.com/2012/01/16/advancing-the-open-conspiracy-h-g-wells-and-the-world-state/}}, offers a better explanation. Its tenets are being implemented
today, but not by a secret cabal, but rather openly by people and groups implementing the policies independently. The
Biblical explanation should be sufficient:

\begin{quotex}
The locusts have no king yet they fly in formation. (Proverbs 30:27) 

\end{quotex}
Keep in mind that there is nothing great about locusts.

\paragraph{The Missing W}
In one of his books, Richard Dawkins created a thought experiment in which he showed how the phrase “METHINKS IT IS LIKE
A WEASEL” could be generated by tossing some tiles and selecting those that would lead to the phrase. But suppose the
letter W was missing from the tiles; then, no number of tosses would create the phrase. The same notion can be applied
to the evolution of the human being. Humans could not have “evolved” from matter unless that there was that
possibility. That is, the biological human is the W tile, what we have been calling a possibility of manifestation. The
goal has to be baked into the entire world process.

\textbf{Reductionism} is the idea that more complex systems can be explained by simpler systems. That requires that the
elements of the simpler system interact in a certain way to create the complex system. But that interaction is the very
definition of the complex system and was not predicted from the simple system.

Theories of everything or the big bang theory have predicted nothing significant. They cannot even predict the daily
activity of a small city.

\paragraph{Social Studies}
Actually, life in the city can be understood to a large extent, if you really try. But the effort will make you few
friends.

\paragraph{The Laws of Physics}
There are no laws of physics that need to be obeyed. As Albert Einstein admitted in the Evolution of Physics, a theory
is the free creation of the human mind. Newton's theory of gravitation was his creation, yet is certainly
false. Einstein's theory of gravity is Einstein's gravitation; better than
Newton's, but most physicists don't believe it is really true.

A sonnet follows the convention of 14 lines with a particular rhyming pattern. A scientific theory has its own
convention: it has to “save the appearances”, i.e., it explains the phenomena we experience. It is not falsified, that
is, no phenomena had refuted it. Ideally, it should predict future phenomena. Relying on “chance” hardly qualifies as
an explanation.

\paragraph{The Theory of Evolution}
Neo-Darwinism $=$ Darwinism + genes. That is the proper name, not the “Theory of Evolution”, because nothing evolves. One
species disappears, another appears. A proper theory will explain in detail the entire process. Neo-Darwinism may be
true, but not as it is currently understood.

It is like attending a seminar on achieving financial independence, and learning that the secret is to play the slot
machines at the casino. Certainly, one of the attendees might hit the jackpot at some point, but that hardly makes the
advice very sound.

\paragraph{The Tunnel}
Imagine that every time you drove your automobile through a tunnel, you didn't know where you would come out
on the other side, because you would create a diffusion pattern. You should be grateful that you don't
follow the “laws of physics”.

\paragraph{Time Travel}
The equations of physics work in both time directions. The equation is the same whether the particle is moving forward
in time or backward in time, so it is hard to tell which is which. I am pretty sure that I could tell which direction I
am traveling. For example, were I going backward in time 20 years, I would be much healthier. So much for that law of
physics.

\paragraph{Unscrambling an Egg}
The one exception is the second “law” of thermodynamics, which states the entropy is always increasing. It means that
disorder is increasing, so things can't be getting “better and better”. So we should be able to determine
the direction of time by measuring entropy.

However, at least one physicist has a better explanation: what we call order and disorder are simply different states of
the system. So if I break an egg and scramble it, the “egg in the shell” and the “scrambled egg in the pan” are just
two different states of the egg. You may have noticed that no two scrambled eggs look exactly alike. You can scramble
eggs, probably daily for the next hundred years, and no two will be exactly alike. Nevertheless, the scrambled state is
much more likely than the egg-in-the-shell state.

The shell state always a possibility, and from the standpoint of physics, neither state is more privileged than the
other. Just as the particle in motion may be going backward in time, the scrambled egg could return to the shell state
without violating any law of physics.

\paragraph{Miracles}
This gives us a better understanding of miracles. A miracle is the manifestation of a not very likely state of matter.
We are deluded into thinking that our common experience is somehow necessary, whereas it is merely the more likely
scenario. It is not contrary to any “laws” of physics for a system to be in one state rather than another.

For example, it is unlikely that you will win the lottery, but it is very likely that someone will win. The winner was
fortunate to be in the most unlikely state of being.

\paragraph{Objective Reality}
For some reason, we think that our experience of the world out there, right now, is “objective” while our inner life is
“subjective”. The opposite is the case. Our knowledge of the external world is a matter of opinion, subject to change.
We create a representation of the world, which may be called the “collective conscious”. Whatever doesn't
fit is our unconscious; e.g., black holes, the centre of the earth.

On the other hand, we are certain of our inner experiences, at least our conscious experience, even if much of our inner
life is unconscious.

\flright{\itshape Posted on 2020-09-02 by Cologero}
