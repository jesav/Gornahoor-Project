\section{The Philosophy of the Future}

\begin{quotex}
The heart has its reasons that reason does not know. \flright{\textsc{Blaise Pascal}}
\end{quotex}

\paragraph{Evola and Gentile}
During \textbf{Julius Evola}'s youth, \textbf{Giovanni Gentile} was the grey eminence of philosophy in
Italy, not just in a university setting, but also close to the seat of political power. He was the epitome of the
cultured European, incorporating the whole of philosophy, art, literature, and history into his comprehensive system,
which he named “actualism”. So Evola's attack on Gentile was an attack on the Italian political,
intellectual, and educational edifice. Gentile never responded personally to Evola's critique, but instead
allowed his student \textbf{Ugo Spirito} to address the issues raised by Evola. But the first issue to consider is the
fundamental aim of philosophy, and this is where the two minds differ. Gentile's first political post was
the Minister of Education, which he used to reform the Italian school system. Evola's goal was much higher:

\begin{quotex}
If Gentile could truly name the I as the “pure act” of his rationalism, then he would appear not as the university
professor, whose “actualism” has the reform of the educational system as its goal, but rather as that cosmic centrality
that the esoteric reveals in the types of the rishi, the yogi, Christ, and the Buddha. 

\end{quotex}

So Evola's real objection is that Gentile sells himself short; the I, in its self-actualization, should have
as its goal to become a rishi, yogi, Christ, or Buddha. This, then, is the logical development of actualism that
Gentile somehow missed. While Evola's system has some defects, the goal is worthy.

To describe that goal, Evola has to incorporate elements of Oriental thought, for example, if Atman is Brahman, then how
does that affect philosophy? The influence of Friedrich Nietzsche is also strong, since it is now impossible to be a
philosopher without dealing with his withering critique of the decadence of Western thought and spirituality. This we
will address in the next section.

Ultimately, Evola never developed the philosophy of magical idealism, certainly not to the point of developing more
Christs and Buddhas. Even during the fifties and sixties, when allegedly there was a stream of young men who consulted
with him, no one arose to carry on that philosophy. By then, I suppose, idealism was a non-starter as the basis for a
philosophical system, and people were looking for less abstractions, turning instead to political and religious
solutions to the problems posed.

Evola himself, having first promoted a philosophy of action, resorted to passivity as in the aristocrat of the soul and
riding the tiger. Of course, while every Tom, Dick, and Harry nowadays claim to be an aristocrat of the soul, the
rishis are still hard to find.

\paragraph{The Philosopher of the Future}
If magical idealism is not the philosophy of the future, then we are still waiting for the philosopher of the future.
Those with a sound intellect should aspire to this, and not be content with the comfortable life of writing clever and
erudite journal papers. Aside from Kant, the great philosophers developed their view of life in their twenties. So
start now, you can always revise it.

Now there are three claimants to the knowledge of ultimate reality: the \emph{Priest}, the \emph{Philosopher}, and the
\emph{Prophet}. Borrowing an insight from \textbf{Valentin Tomberg}, we can say that the philosopher works in the day
through the light of reason, the prophet in the night through direct illumination from God, and the priest is the
mediator between the light and the darkness. The philosopher of the future will probably be in tension with the other
types, while still needing to incorporate their insights.

The philosopher must first deal with facts, then an understanding of the facts, and finally indicate how that affects
our lives. The fundamental facts have been summarized by \textbf{Arthur Schopenhauer}: the world as will and idea. Here
we find the Traditional doctrine of the two worlds of being and becoming. The “world” referred to is that of becoming,
and the “idea” is the world of being. Here are some examples.

\textbf{Plato}, and the lineage following him, called the will “eros”, i.e., the drive or “love” of wisdom. Wisdom, for
him, is to know the world of being. For Nietzsche, this overvaluing of the “other” world in Plato and in the
Christianity which built on Platonic ideas, led man away from his true calling of being fully loyal to the earth. There
is no other worldly afterlife beyond this world, but only its endless repetition. The Will to Power replaced eros. In
denying the world of being, Nietzsche denies God, or better, God, for him, is yet to come.

As a Traditional thinker, Evola opposed Nietzsche's biologism, while incorporating his more important
insights. While not denying the world of being, he changed man's relationship to it. First of all, he
retained Nietzsche's emphasis on will and action; this, as we have seen, brought him into conflict with
Rene Guenon. Now action can be understood in two ways. The conventional way is to see it as “horizontal”, i.e., as
activity wholly in the world of becoming. A deeper way is to understand it “vertically”, i.e., as the actualization of
potentialities. In this way, Evola can claim that it is insufficient to \emph{know the truth}, one must also \emph{will
the truth}. This implies absolute freedom.

The philosopher of the future can build on this. A rishi, or a seer, is more like a prophet than a philosopher. Hence,
he must learn to think with his heart as well as with his head. If the goal of philosophy is to bring God's
presence into the world, then he must learn to do that himself. To be free means to have no sufficient reason outside
oneself, so the philosopher must be free. Since for God, essence and existence coincide. Hence, the philosopher of the
future must actualize all his possibilities. Now we mean the philosopher is God-like in the relative, not the absolute
sense. How that is so, will be the task of this philosopher to explain.

\paragraph{The Religion of the Future}
The religion of the future will be based on gnosis. This is not a new religion, but rather a deeper understanding of
what religion is and means. In other words, it is the actualization of religious or spiritual understanding. This is
reflected in various states of consciousness, both psychological and spiritual. I am not making this up and have amply
documented how this has always been the case.

There are two false claimants to the religion of the future: one is to alter it to bring it into conformism with
modernity, the other is to repeat the religious forms of the past. Now there is no problem with the second option for
those who are satisfied with it. But the prophet of the future will write a large book on the phenomenology of the
soul.

\paragraph{2018 Postscript}
In a recently published collection of letters between \textbf{Wolfgang Smith} and Fr. \textbf{Malachi Martin}, there is
this intriguing comment from Prof. Smith:

\begin{quotex}
If the Greek Fathers could integrate Plato and Neoplatonism into the Christian worldview, and St. Thomas Aquinas could
do the same for Aristotle, why should it not be possible, in our day, to correct and somehow “Christianize” Hegel, let
us say, or Schelling, or even Nietzsche? Is there not in each of these German “Titans” a certain spark of truth that
needs to be brought out, to be “liberated”? 

\end{quotex}
That would be a good task for a young scholar. The starting point, of course, would be \textbf{Jacob Boehme}, the father
of German Idealism. Also, \textbf{Vladimir Solovyov}, who has already adapted Schelling into his system.

\flright{\itshape Posted on 2014-06-26 by Cologero}

\begin{center}* * *\end{center}

\begin{footnotesize}\begin{sffamily}

\texttt{seeker on 2014-06-26 at 13:25 said: }

You bring to mind Fr. Seraphim Rose's book (Orthodoxy and the Religion of the Future)? He gave the term a
decidedly pejorative connotation, referring specifically to New Age syncretism, one of the false claimants, “bringing
it into conformism with modernity”. But he was also critical of Hinduism, saying that the yogis experience during
meditation are mere psychic phenomena and not real spiritual knowledge.

The book was obviously not meant for the “rishi, yogi, Christ, or Buddha”. In this letter he refers to the Kali Yuga by
name, which leads me to believe there is more to his thought than one would gather from the above mentioned book. 

\hfill

\texttt{Ash on 2014-06-27 at 00:31 said: }

Concerning the relationship of the philosopher to gnosis: would such a person have already attained such a high state of
existence themselves? Or would they be showing the way as travelers themselves? I suppose two tests for such a
philosopher would be a) are they practicing a living exoteric tradition, and b) are they old enough to have gained
wisdom over time. So anyone starting now in their twenties ought to be preparing themselves for that.

\hfill

\texttt{Cologero on 2014-06-27 at 00:44 said: }

Seeker, I have not read that particular book, but I had previously read the letter you linked to. Obviously, I am being
a bit tongue in cheek and am not advocating anything like New Age … actually quite the opposite if read carefully.
Don't forget that Fr Rose also warned against “super-correctness”\footnote{\url{http://orthodoxinfo.com/ecumenism/fsr_63.aspx}}, and we have been criticized many times
by the super-correct. The problem today is one of a worldview. There was an earlier time when men could think with
their heart and were acutely aware of the reality of the other world of being. But now, such a mentality is utterly
alien to most men, so an intellectual conversion of some sort is necessary. So the way forward is actually a recovery,
but at a deeper level. That is because a man who has had to work for something appreciates it more than the man for
whom it came without effort.

I have been planning a post on the mental and spiritual states of the yogis, but much preparation is required.
Ultimately, it will be up to the prophet of the future to make such distinctions. Now Fr Rose was totally into Guenon
before he was not, so it is legitimate to speak to the former class of men as well as to the post-Guenon men. Careful
readers will need to sort it out.

\hfill

\texttt{Cologero on 2014-06-27 at 07:45 said: }

As I pointed out, Ash, most great philosophers developed their philosophy in their youth … I suppose we can find
examples and counter-examples. But the philosopher is a just “lover of wisdom” and so is not necessarily “wise”. There
are ultimate facts and the philosopher tries to understand them in the light of his own reason. The best are not
daunted by the complexity and enigmas of reality. In our time, there are new forms of escapism, even more insidious
that the decadent Christianity that Nietzsche or Evola opposed.

These forms of escapism are obvious enough to name. First are the so-called “new atheists”, who believe science and a
narrowly conceived rationality can account for all of reality. The other is new age political correctness which through
shaming and self-deception tries to enforce a worldview involving beliefs that no well-bred and healthy-minded man
could have believed in previous eras.

So, this imaginary philosopher would have to challenge those forms of escapism and describe a worldview in which the
quest for gnosis “makes sense”. So, yes, a man in his twenties ought to be preparing for that, “as if”, before his mind
stagnates.

\hfill
\end{sffamily}\end{footnotesize}