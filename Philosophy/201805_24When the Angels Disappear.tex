\section{When the Angels Disappear}

\begin{quotex}
Excellent personal qualities should beg to be excused or conceal themselves, for intellectual superiority offends by its
mere existence without any desire to do so. \flright{\textsc{Arthur Schopenhauer}}

It is not things that disturb men, but opinions about them. \flright{\textsc{Epictetus}}

True wealth is only the inner wealth of the soul. Everything else brings more trouble than advantage. \flright{\textsc{Lucian}} 

\end{quotex}

\textbf{Aristotle }in the \emph{Nicomachean Ethics} divided the good things of human life into three classes:

\begin{enumerate}
\item Those outside 
\item Those of the soul 
\item Those of the body 
\end{enumerate}

\textbf{Arthur Schopenhauer} described them in more detail.

\begin{enumerate}
\item \textbf{What a man is} and therefore personality in the widest sense. Accordingly, under this are included health,
strength, beauty, temperament, moral character, intelligence and its cultivation. 
\item \textbf{What a man has} and therefore property and possessions in every sense. 
\item \textbf{What a man represents}. We know that by this expression is understood what he is in the eyes of others an
thus how he is represented by them. Accordingly, it consists in their opinion of him and is divisible into honour,
rank, and reputation. 
\end{enumerate}

\paragraph{What a Man Is}
What a man is ultimately depends on his own consciousness. Therefore, attempts to alter the material conditions of life,
e.g., through legal means, so-called “safe spaces”, etc., will have limited effect. Schopenhauer explains why:

\begin{quotex}
Everyone is confine to his consciousness as he is within his own skin and only in this does he really live; thus he
cannot be helped very much from without. 

\end{quotex}
People living in the same environment and political system may still have radically different understandings. Despite
being in the identical material situation, people live in worlds of their own. Schopenhauer explains why:

\begin{quotex}
A man is directly concerned only with his own conceptions, feelings, and voluntary movements; things outside influence
him only insofar as they give rise to these. The world in which each lives depends first on his interpretation thereof
and therefore proves to be different to different men. Accordingly, it will result in

\begin{itemize}
\item
being poor, shall, and superficial, 
\item
or rich, interesting and full of meaning. 
\end{itemize}

For example, while many envy another man the interesting events that have happened to him in his life, they should
rather envy this gift of interpretation which endowed those events with the significance they have when he describes
them. 

\end{quotex}

What a man \emph{is} contributes to his happiness more than what he \emph{has} or what he \emph{represents}. Hence, we
will focus on that essay, which can be found in Volume 1 of \emph{Parerga and Paralipomena}.

Aside from cases of serious misfortune, how we interpret and feel about the events of our life is more important to our
inner well being and happiness than the events themselves. That is why misfortunates that originate from outside us are
more easily bearable than those we have created ourselves. That is why people berate themselves for their errors.
Schopenhauer tells us why: we feel that our luck can change, but it is much more difficult to change one’s
nature.

Therefore, subjective blessings should be pursued more readily than objective ones. Schopenhauer has his own list:

\begin{itemize}
\item Noble character 
\item Gifted mind 
\item Happy temperament 
\item Cheerful spirits 
\item Well-conditioned and sound body 
\end{itemize}

\paragraph{Cheerfulness}
Although Schopenhauer has a reputation for being the “philosopher of pessimism”, in his personal life he valued being
merry and cheerful. Obviously, they are their own reward. Curiously, people tend to be suspicious of cheerfulness and
look for a reason for it. Health is an important factor. He recommends the avoidance of:

\begin{itemize}
\item excesses and irregularities 
\item violent and disagreeable emotions 
\item prolonged mental strain 
\end{itemize}

Of course, movement and regular exercise contribute to health.

\paragraph{Pain and Boredom}
Pain and boredom are the enemies of happiness:

\begin{itemize}
\item Lack and privation produce pain 
\item Security and affluence give rise to boredom 
\end{itemize}
Inner vacuity and emptiness, which Schopenhauer claims to be able to see in the faces of the masses, crave events in the
external world to fill up their minds. He describes the process:

\begin{quotex}
This vacuity is the real source of boredom and always craves for external excitement in order to set the mind and
spirits in motion through something …. The emptiness of their inner life, the dullness of their consciousness, the
poorness of their minds drive them to the company of others which consists of men like themselves. They then pursue
pastime and entertainment in common which they seek first in sensual pleasures, in amusements of every kind, and
finally in excess and dissipation. 

\end{quotex}
\paragraph{Solitude}
The greater our inner wealth, the less room there is for boredom. While inner vacuity results in “the craze for society,
diversion, amusement, and luxury of every kind”, inner wealth is different.

\begin{quotex}
The clever and intelligent man will first of all look for painlessness, freedom from molestation, quietness, and leisure
and consequently for a tranquil and modest life which is as undisturbed as possible. Accordingly, after some
acquaintance with human beings so called, he will choose seclusion and, if of greater intellect, even solitude. For the
more a man has within himself, the less does he need form without and also the less other people can be to him.
Therefore eminence of intellect leads to unsociability. 

\end{quotex}
\paragraph{The Three Physiological Fundamental Forces}
By these, Schopenhauer means eros, thumos, and nous. In the primal state, their originary use was in the struggle
against lack and privation. When that problem is solved for the most part, the forces are underutilized and require
stimulation. Depending on his dominant centre, a man will pursue different pleasures.

\begin{itemize}
\item \textbf{Eros}. The pleasures of the power of reproduction: eating, drinking, digesting, resting, and sleeping. 
\item \textbf{Thumos}. The pleasures of irascibility: walking, jumping, wrestling, dancing, fencing, riding, hunting,
athletic games, and even war. 
\item \textbf{Nous}. The pleasures of sensibility: observing, thinking, feeling, writing, poetry, improving the mind,
playing music, learning, reading, meditating, inventing, philosophizing, etc. 
\end{itemize}
Sensibility, i.e., the ability to respond to intellectual and aesthetic sensations, ranks the human being higher than
the animals, which are restricted to the two inferior forces. Schopenhauer describes the two types like this:

\begin{quotex}
The life of the masses is passed in dullness since all their thoughts and desires are directed entirely to the petty
interests of personal welfare and thus to wretchedness and misery in all its forms. 

\end{quotex}
On the other hand:

\begin{quotex}
The existence of the man who is endowed with outstanding intellectual powers is rich in ideas and full of life and
meaning. Worthy and interesting objects occupy him as soon as he is permitted to devote himself to them, and he bears
within himself a source of the noblest pleasures. Stimulation from without comes to hm from the works of nature and the
contemplation of human affairs and then from the many and varied achievements of the most highly gifted of all ages and
lands; only such a man is really capable of thoroughly enjoying those things for he alone can fully understand and feel
them. Accordingly, for him those highly gifted men have actually lived; to him they have really appealed; whereas the
rest as casual hearers only half-understand something or other. 

\end{quotex}
\paragraph{The Two Lives}
Such a man lives two lives: a personal life and an intellectual life. The latter is his real life and the former is
merely a means. The intellectual life obtains cohesion, wholeness, and perfection, “becoming ever more complete like a
slowly maturing work of art.”

The centre of gravity of such a man is entirely within himself. Schopenhauer makes this rather strange point:

\begin{quotex}
Our moral virtues benefit mainly other people; intellectual virtues, on the other hand, benefit primarily ourselves.
Therefore, the former makes us universally popular, the latter unpopular. 

\end{quotex}
\paragraph{Pain and Melancholy}
Up to this point, we have been emphasizing the positive aspect of the intellectual life. However, it can be a mixed
blessing. That is because, in his words:

\begin{quotex}
Great intellectual gifts may produce a very much enhanced sensitiveness to pain in every form. Further, the passionate
temperament that conditions such gifts, and at the same time the greater vividness and completeness of all images and
conceptions inseparable therefrom, produce an incomparably greater intensity of the emotions that are thereby stirred. 

\end{quotex}
Since there are more painful emotions than pleasant ones, the former can be aroused more readily. For example, I know a
great souled being who sometimes sees too deeply and too far ahead, beyond what she can handle, resulting in sadness.

Others have perhaps noted this ambiguity. For example, \textbf{Aristotle} can assert, on the one hand:

\begin{quotex}
The philosophical life is the happiest. 

\end{quotex}
Yet, he also wrote:

\begin{quotex}
All those who distinguished themselves whether in philosophy, politics, poetry, or the arts, appear to be melancholy. 

\end{quotex}
\textbf{Sophocles} has also contradicted himself in the same way:

\begin{itemize}
\item To be intelligent is the main part of happiness. (\emph{Antigone}) 
\item The most agreeable life consists in a lack of intelligence (\emph{Ajax}) 
\end{itemize}
Even the Bible leaves us in ambiguity:

\begin{itemize}
\item The life of a fool is worse than death! (Sirach 12:12) 
\item In much wisdom is much grief, and he that increaseth knowledge increaseth sorrow. (Ecclesiastes 1:18) 
\end{itemize}

\paragraph{Love and Will}
Although Schopenhauer’s essay focuses on the physical and intellectual aspects of happiness, a brief mention
can be made about the moral and aesthetic life, as they relate to happiness. Thinking is self-limiting, and true Wisdom
lies in a realm beyond thought. If Wisdom is the feminine element, then Love is the masculine element that resolves
contradictions. The intellectual is content, like Epicurus in his garden, to remain alone in his solitary
contemplations. Love, on the other hand, draws him back out of himself to another, albeit on a higher level than the
sociability of the dullards described by Schopenhauer.

On the level of cataphatic theology, or thinking, one can meditate on the Unmoved Mover. However, in Love, one learns to
\emph{be} the Unmoved Mover. In the second appendix to the \emph{Yoga of Power}, \textbf{Julius Evola}, quotes
\textbf{Dante}:

\begin{quotex}
I am as the centre of a circle, to which the parts of the circumference stand in equal relation. 

\end{quotex}
In other words, Love is characterized by \textbf{centrality}, \textbf{transcendence}, \textbf{stability}, and
\textbf{immutability}. Love, whether in the form of an actual woman or, if you prefer, as a metaphor for the feminine
part of a man, brings a huge risk to the philosopher. The wholeness which he thought he had achieved turns out to be
missing an essential element, viz., transcendence.

If he can raise himself to the point of centrality, stability, and immutability, he reaches a level of being that
transcends mere thinking and finds his True Will, i.e., not motivated by worldly concerns, nor even in the service of
thought.

\flright{\itshape Posted on 2018-05-24 by Cologero}