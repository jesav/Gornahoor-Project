\section{Francis Bacon and the Creation of Modernity}
\label{sec:BaconModernity}
It is our contention that great events in world history are deliberately planned; nothing happens at random, and things happen for a reason. To understand such reasons is the very definition of intelligence. The beginnings of the modern mind were found in the anti-metaphysics of nominalism. But it was \textbf{Francis Bacon} who sketched out the intellectual lineaments of modernity and, in the process, redefined what it means to be an “intelligent” man. His definition still holds today among the university educated and is even absorbed unconsciously among the masses.

Bacon lays out his plan in the Great Instauration, an ironic reference to Ephesians 1:10, “to instaurate everything in Christ”. We must not be deceived by Bacon's pious language and Biblical references which are included to mask his true intentions, as the consequences of his method had to have been known to him. To deny that is to deny Bacon's intelligence; Bacon was anything but unintelligent. The consequences of his plan will be the denial the supernatural from any understanding of the world. Bacon reveals his purpose in the Preface:

\begin{quotex}
That the state of knowledge is not prosperous nor greatly advancing, and that a way must be opened for the human understanding entirely different from any hitherto known, and other helps provided, in order that the mind may exercise over the nature of things the authority which properly belongs to it. 

\end{quotex}
\begin{enumerate}
\item \textbf{Problem}: The state of knowledge is not advancing. 
\item \textbf{Solution}: A way of knowing entirely different from anything in the past. 
\item \textbf{Benefit}: The mind exercises its authority over the nature of things. 
\end{enumerate}
The attack on Tradition is made head-on. The Traditional view is totally opposed.

\begin{enumerate}
\item The can be no advancement of Traditional knowledge. 
\item There is nothing new under the sun. 
\item The nature, or essence, of things has its authority in the mind of God. 
\end{enumerate}
Bacon masks his intentions by claiming to be returning to the Primordial State, when Adam was given dominion over the world. But Adam's naming of the animals was an act of recognition, a remembering of their nature or form. Bacon, despite his protests, takes on the Satanic project of replacing God, so that the new man is now the arbiter of forms. This is not a moral judgment, but rather a description. To know what a thing is, is to know its sufficient reason. The Western Tradition lists four causes to explain a thing or event, as shown in Table~\ref{tab:fourcauses}.

\begin{table}[h]
\small
\label{tab:fourcauses}
\centering
\begin{tabular}{lll}\toprule
\textbf{Cause} &
\textbf{Description} &
\textbf{Question}\\\toprule
Material &
What something is made of &
\\\midrule
Formal &
What it is &
What\\\midrule
Efficient &
How it came to be &
How\\\midrule
Final &
What is its purpose &
Why\\\bottomrule
\end{tabular}
\caption{The traditional doctrine of four causes}
\end{table}

Bacon takes his stand against this schema by first rejecting the final causes. He explains why:

\begin{quotex}
I am laboring to lay the foundation, not of any sect or doctrine, but of human utility and power. 

\end{quotex}
Hence we see that everything must be in service to human utility and power. If anything has its own reason for being, that is, its formal cause, then its usefulness for human ends is no longer the prime consideration. He then rejects the notion of formal cause:

\begin{quotex}
Matter rather than forms should be the object of our attention, its configuration and changes of configuration, and simple action, and laws of action or motion, for forms are figments of the human mind, unless you call those laws of action forms. 

\end{quotex}
Can it be said more clearly? Only matter, its configurations, changes, and actions are to taken into account. There is no hylomorphism and hence the natural world is not the reflection of the supernatural. There is a subtle change in the notion of the efficient cause. It no longer refers to the essential and internal relationship between forms, but rather to the accidental and external relationship between material configurations. Now Bacon calls the alleged intuition of forms a ``figment of the imagination''. This is why we insist that debate is pointless. If a man can see the forms, or ideas, it is no figment; otherwise, it may as well be. Philosophical discussion can make the concept plausible, but only gnosis can prove it. Let us make a point in passing. The laws of action or science are not the forms. Scientists often claim they are reading the mind of God; however, this is not true, since the forms are in the mind of God.

We can briefly mention the consequences of the line of reasoning instaurated by Bacon. First and foremost, the function of Reason has been radically altered. In the Traditional view, it is the defining quality of the intellectual soul and its goals is complete understanding of the Logos. For Bacon and his successors, Reason has simply an instrumental value, as a tool used to accomplish human ends. This reaches its ultimate conclusion in Nietzsche when the will to power is pursued for its own sake, not simply for human betterment.

Not only are things considered configurations of matter, but so also is human society. Without a common understanding of final causes, there can be no rational discussion of how to achieve the common good, since the idea of the good is no longer common. Thus, every socio-politico-economic decision becomes a battle for power, with so-called rational discussion being just a means to an end.

Without any understanding of formal cause, the power to define “what is” is left to man, so it too becomes a contentious battleground. The end result is the hermeneutics of suspicion, so that any attempt to explain events is regarded as a deliberate deception with hidden motives. Thus, what started out as the desire to redefine Reason, ends up in total irrationality. \textbf{QED}



\flrightit{Posted on 2011-11-26 by Cologero }

\begin{center}* * *\end{center}

\begin{footnotesize}\begin{sffamily}

\texttt{Boreas on 2011-11-27 at 12:15 said: }

A good and thought-provoking post! It brought to my mind a few things, especially the notion about “a satanic project”. Bacon truly represents – together with Machiavelli and the like – the humanistic ideals of many satanists out of whom many aspire to be “world-embracing geniuses”. When understood in this sense it really does look like that Anton Lavey was right in saying that very many (modern) men are “satanists” without knowing it themselves, that is, they reject the idea of the supernatural and God \& wish to increase man's power and knowledge for it's own sake and over the natural world. The “funny” thing is that there are a lot of those who see themselves as pious christian believers and see this “satanic project” as man's – “the crown of Nature” – manifest destiny appointed to him by God.

This also brought to my mind the fact that historical christianity is the father of God-denying atheism and the progressive messianic belief of transforming (“saving”) the world, which is a historicist and horizontal perversion of the idea of spiritual evolution, salvation and liberation. From sphere to cube?

I just read Bacon's biography and there was a mention of him belonging to the Freemasons, and there were also suspicions of him belonging to the Rosi-crucians. While the first one seems to be an established fact and no cause for wonder, the second one sounds a little strange, at least if were talking about the true Rosi-crucians.


\hfill

\texttt{Cologero on 2011-11-27 at 22:14 said: }

Good insight, Boreas, and the Lavey quote was on the mark. The easy way out is to assume that Satan is pure and easily recognized evil, but that is far from the case. Voegelin described what you call the historicist and horizontal perversion as “immanentizing the eschaton”. Interesting claim about historical Christianity fathering its own demise. I'll try to relate that to Solovyov's discussion of development.


\hfill

\texttt{Boreas on 2011-11-28 at 04:57 said: }

You may also want to take into consideration Gurdjieff's `law of octaves' and Evola's notions about the reason why Christianity has been relatively easy to attack and out-throw, while the more metaphysical (instead of theological / scholastic) systems of Hinduism, Buddhism etc. – despite their internal schisms and modern corrosions – have remained more intact and above modernist \& rationalist criticism. The degenerated Atlantis always resides in the far west.


\hfill

\texttt{Logres on 2011-11-28 at 08:58 said: }

Didn't Bacon even write something entitled “New Atlantis”?


\hfill

\texttt{Count Cagliostro on 2011-11-28 at 17:10 said: }

Excellent post and very insightful comments!

Would you care to elaborate on the statement that “historical christianity is the father of God-denying atheism…horizontal perversion”???

To my limited knowledge, Krishna, Bodhisattva and Christ embody the Savior archetype, which is obviously indespensable to the founding of every religion.


\hfill

\texttt{Boreas on 2011-11-29 at 03:30 said: }

``Would you care to elaborate on the statement that `historical christianity is the father of God-denying atheism…horizontal perversion'???''

As I understand it, Christianity in its historical manifestation was the first religion and tradition that emphasized the historical – that is, temporal – role of Christ as a messiah. Christianity “temporalized God”, so to speak: God became man so that man might become God, the Word became flesh etc.

Christianity was also the first religion to exoterise (emphasis here) the monotheistic conception of the world, which could not lead to anywhere else in the popular mind and among the masses of humanity than to the irreconcilable dualism, antropomorphism, and via these eventually to the deistic, a-theistic, and naturalistic aberrations, the first one denying God's absoluteness and omnipresence, the second one God / divinity, and eventually naturalism denying transcendence and the conception of sacred itself, via “solidification” leading to rationalism, rationalistic science and to the enlightenment – and so on. Yet, mutadis mutandis, the last ones still bear in themselves the conception that the world is to be transformed by man and man can do this, yet in these world-outlooks by rationalistic, mechanistic and scientific means. In this way the Christian conception of spiritual salvation and apotheosis was transformed by the `terror of history' (Eliade) into the liberal myth of progress. Man must conquer the world, because Man is God!

This is a short summary and by necessity a simplification, but I hope this satifies you, Count Cagliostro.

As a further reading I could recommend you Marty Glass' excellent and poetic analysis `YUGA – An Anatomy of Our Fate'.

\url{http://www.sophiaperennis.com/books/eschatology/yuga/}


\hfill

\texttt{logres on 2011-11-29 at 22:51 said: }

Not all Christian theologians have let it go unnoticed – the Lutheran theologian Walter Kunneth wrote Theology of the Resurrection specifically to emphasize the ahistorical “mysterious” pleroma of Christ, as over against both Bultmann (existentialists) and the fundamentalist/neo-orthodox movements.


\hfill

\texttt{Boreas on 2011-12-01 at 10:36 said: }

Yes, there has of course been and are men and women who understand the a-historical nature of the Christ and the corresponding teachings. Neither do my previous posts try to deny the echatological and historical importance of Jesus, it has only been largely misunderstood and distorted very badly.

One thought came to my mind about this “satanic” nature of Bacon \& co. Could it be conceivable that the appearance of the “accuser” and “opposer” of this kind was in a way unavoidable, because (exoteric) Christianity has conceived Satan also quite unilaterally as “the devil”, which is one part of the dualistic problem.

There is also the nagging, unsolved problem of theodicy which theological thought is very badly aquipped to solve with its own means, since it is also still in the grips of unrecognised dualism. This may be one of those reasons why the more metaphysical systems have remained more intact; they acknolwedge the importance of cosmic evil in the world process and in the great economy of the universe. (From this should not be drawn the conclusions that `evil is good because it is necessary'. This leads to a very downward path.)


\hfill

\texttt{Boreas on 2011-12-01 at 10:43 said: }

Typos abound, sorry for that.

I forgot to mention in the previous post that maybe the world is slowly but steadily moving to a more balanced view in this matter, once the age of Saturn starts to loom in the horizon and the Sun moves closer to Capricorn. Most fortunately this is not even a blink of an eye in the cosmic time scale, although in the viewpoint of a temporal consciousness it seems soooo far ahead (Saturn again!).


\hfill

\texttt{Eric on 2011-12-04 at 17:52 said: }

I like the post, it was informative, and I'm sympathetic towards your views, but your conclusion is weak.

If human society is merely a configuration of matter, that in no way precludes various moral theories such as virtue theory (the good is defined as those actions which concord to cultural virtues – which are defined according to that culture), or Kants moral theory, which merely looks for universal action in the light of reason to judge moral acts, even Hume's emotionally inspired moral theory is not dependent upon a notion of God.

Your final comments, “The end result is the hermeneutics of suspicion, so that any attempt to explain events is regarded as a deliberate deception with hidden motives. Thus, what started out as the desire to redefine Reason, ends up in total irrationality”, are particularly weak, there is no necessary connection between a Godless worldview without forms and the `hermeneutics of suspicion'. I'd like you to elaborate on why you think it *must* be the case. As previous said, cultural norms can be the final appeal by which virtues are decided, why do you think that option, as well as the aforementioned Kantian and Humean options are irrelevant?


\hfill

\texttt{Cologero on 2011-12-04 at 20:46 said: }

Eric, there is no claim about precluding “various moral theories”. So who cares about accumulating theories? The point Gornahoor has been trying to make in many different ways is that if there is no objective morality, that is, a legitimate way to determine justice, then power is the only way to resolve disputes of justice. That is far from a “weak” conclusion, it is absolutely logical. If there are multiple moral theories, then how does a polity decide which one to adopt? Obviously, there is no rational way to choose one. Hence, it is personal preference or whim.

If there are multiple moral theories, then each one will have different results in practice. These will result in one group or another benefiting. Hence, the hermeneutics of suspicion because of the necesssary distrust that engenders. If there is no “necessary” connection, it is only because one of the parties is too ignorant or gullible. So, it must be the case; it certainly is the case. How can it be made more clearly?

I don't believe the post claimed that morality was dependent upon the notion of a God, but rather on the notion of formal and final causes. Kant and Hume can come up with all the incompatible moral theories they like, but how do you convince someone to be moral at all? A moral theory is dependent on the notion of good … what is good, what is justice? I know what a good score in golf is, so I can act accordingly. Golf has rules and a purpose. But how do I know what a good life is? or a good polity? That requires I know the purpose of my life, of human life, of social life. 

Relying on cultural norms is circular reason. In a multi-culture, which is the situation nearly everywhere today, it is a recipe for civil strife.


\hfill

\texttt{Leonardo Cavalcanti on 2020-09-08 at 08:49 said: }

Of course it is logical, Cologero, and there is an entire book to demonstrate this point, by Alasdair Macintryre called After Virtue, after the notion of finality was abandoned, the understanding of ethics went downhill. This is obvious, but people become emotional about this, they don't want to find out that the modern institutions cannot be trusted, well, many people are afraid of the truth, that's why they deny it.


\end{sffamily}\end{footnotesize}
