\section{The Most Beautiful Work}

This is part one of a review of \emph{Plato's View of Man} by \textbf{Constantine Cavarnos}, available from the Institute for Byzantine and Modern Greek Studies\footnote{\url{https://ibmgs.org/ancientphil.html}}.

\begin{quotex}
It is within our power to transform our lives, to restore the proper order within ourselves, giving primacy to spiritual over material values. No task is more urgent, no work more glorious than the regeneration of individuals and of states. In the \emph{Euthyphro} \textbf{Plato} calls it ``\textbf{the most beautiful work}", which man can accomplish with divine aid. \flright{\textsc{C. Cavarnos}}

\end{quotex}
We must take care that a teacher of doctrines, which are food for the soul, does not deceive us, like the merchant and the retail-dealer, who sell food for the body. For the latter praise all their goods, although neither they know what is good and what is bad for the body, nor those who buy from them, unless one happens to be a trainer or physician.

\begin{quotex}
In the same manner, those who take their doctrines about in cities and sell them by retail to the person who desires them, praise everything that they sell, although, my excellent friend, probably some of them are ignorant which of their wares is good and which is bad for the soul; and their customers are similarly ignorant, unless one happens to be a \emph{physician of the soul}. \flright{\textsc{Plato}, \emph{Protagoras}}

\end{quotex}
There are three philosophical sciences. Although they have their empirical and profane counterpart, these sciences are meant in the esoteric sense of referring to the inner nature of their subject matter.

\begin{itemize}
\item \textbf{Anthropology}, the science of man 
\item \textbf{Cosmology}, the science of the cosmos 
\item \textbf{Theology}, the science of God 
\end{itemize}
In this short book, Mr. Cavarnos focuses on anthropology as taught to us by Plato. The two essays are followed by a series of extracts from Plato's actual works. As such, it provides a good summary of our understanding of man and his destiny. The point is not simply to explain Plato, but rather to encourage us to engage in ``the most beautiful work", at least those of us who are capable of it.

\paragraph{Man's True Abode}
Man's true abode is a world of absolute truth, goodness, and beauty, which is the conclusion of the philosophical life. The devil is in the details; even after becoming dimly aware of that, there is still the work that needs to be done. There are some obstacles the can impede one's path.

\paragraph{Beauty}
Beauty is understood as order and harmony. Mr. Cavarnos writes:

\begin{quotex}
By contemplating objects that possess measure and proportion and by purifying himself through the practice of self-examination, the lover of wisdom becomes more and more orderly and harmonious, more and more God-like. His whole soul becomes converted from darkness to light, from untruth to truth, from the changing to the unchanging, from becoming to being, from disunity to unity. 

\end{quotex}
Many fall short at this point by confusing the idea of Beauty with the accumulation of beautiful things. The mistake is to forgo the development of one's inner harmony through the attempt to create an orderly and harmonious environment.

Eros is the aspiration for beauty, or love. It is not sublimated sexual desire as in Freud, but rather sexual desire is understood as a lower form of eros; it is the desire for physical immortality through one's descendants. Eros has led heroes to undergo dangers. Love spurs others to create works of the spirit.

The Platonic path of love passes from the concrete to the abstract, from the particular to the universal, from the personal to the impersonal. The true initiate rises from the love of bodies to the love of persons, then to the love of theories, next to that of institutions and communities, and finally to that of absolute transcendent Beauty.

Mr. Cavarnos objects to that on the ground that Christian love is the love of the person, not the abstract. On the other hand, Plato's view is consistent with the increasing levels of abstraction associated with the angelic hierarchy. No man is an island, so it is difficult to conceive of man apart from the institutions and communities that he belongs to. They certainly deserve our love.

\paragraph{Goodness}
A man's fate after death is determined by his way of life on earth. Lust and a contentious attitude are impediments to the good life:

\begin{quotex}
Whoso indulges in lusts or in contentions and devotes himself overmuch thereto must of necessity be filled with opinions that are wholly mortal and … has made great his mortal part. But he who has seriously devoted himself to learning and to true thoughts, and has exercised these qualities above all others, must necessarily and inevitably think thoughts that are immortal and divine. (Timaeus) 

\end{quotex}
\paragraph{Truth}
\begin{quotex}
I live in the midst of a lie, and die for a lie, and the earth is a lie, and rests on a lie, on a stupid derision. \flright{\textsc{Dostoyevsky}, \emph{The Demons}}

\end{quotex}
How does one find the truth in a world dominated by the lie? Man suffers from inner deformity or ugliness, from disease of the psyche. He is a deformed being in need of harmony, of beauty; he is a sick being, in need of therapy. This is the result of ignorance, which is the cause of the vices of cowardice, intemperance, and injustice. There are two types of ignorance:

\begin{itemize}
\item Conscious ignorance, of which a person is cognizant 
\item Unconscious ignorance, in which the person claims to know what he does not know 
\end{itemize}
Unconscious ignorance is more insidious, since such a person is not motivated to seek knowledge and believes he is a better person than he really is. The rational man is painfully aware of his faults and shortcomings. He knows how difficult it is to achieve any knowledge. He understands the efforts required to master one's will.

The unconsciously ignorant person, on the other hand, is convinced of his moral superiority. He has a ``good heart", a ``happy gene", a ``positive mind", and so on. He believes in all the right causes with a self-serving passion. Ultimately, he is ``beyond dogma and ideas", his religion is love. In practice, that means that he is impervious to rational and intellectual thought. Unconscious ignorance assumes the form of pride, conceit, prejudice, and leads to contentious and false sophistry. Basically, unconscious ignorance is a lie told to oneself.

The ``greatest and most efficacious method of purification" is \emph{elenchos}: the process of examining one's beliefs, opinions, and feelings. It purifies ``the soul from conceits that stand in the way of knowledge." The proper use of this method results in intellectual integrity and inner health.

\begin{quotex}
It brings about a harmony between what a person thinks he knows and what he really knows, between what he professes to believe and what he really believes, between what he imagines himself to be and what he really is. 

\end{quotex}
\paragraph{Immortality}
\begin{quotex}
Two associated aspirations raise the human soul above the rest of created nature: the hunger for immortality and the thirst for truth or moral perfection. Either without the other is meaningless … Endless life without truth and perfection would be an eternity of torment, and perfection without immortality would be rank injustice and indignity without measure. \flright{\textsc{Vladimir Solovyov}}

\end{quotex}
Plato offers several proofs for the immortality of the soul, which we need not go into at this point. Life extension for its own sake is pointless. A soul that has become purified and perfected may eternally behold perfect Truth, Goodness, and Beauty. There the soul will live with the gods. Obviously, no one is interested in that today. However, if you declare that the soul will spend eternity with its dogs, the whole world takes notice.

Plato's explanation is that the soul participates in the idea of Life. Mr. Cavarnos clarifies that notion:

\begin{quotex}
That some such view is involved in the Christian conception of immortality is undeniable. But the life in which the soul participates is not for the Christian an abstract idea, but the grace of the Holy Spirit. Only in so far as a man comes to partake of this does he become truly alive and immortal. 

\end{quotex}

\hfill

\begin{quotex}
I do nothing but go about persuading you, old and young, to care neither for your bodies nor for money more than for the perfection of your souls, or even as much. \flright{\textit{Apology}}

There neither is nor will there ever be anything of greater importance than the cultivation of the soul. \flright{\textit{Phaedrus}}

When the soul and the body are united, nature orders the body to serve and be ruled, and the soul to rule and be master. \flright{\textit{Phaedo}}

\end{quotex}
\paragraph{Soul, Body, Money}
\begin{itemize}
\item Care of the soul is fundamental, because a person \emph{is} the soul. Not what a person has, but what he is constitutes his real dignity. 
\item A person who cherishes the body does not cherish himself, but what belongs to him. 
\item The person who tends money tends neither himself nor his own things, and is a stage further removed from himself. 
\end{itemize}
A philosopher needs leisure time and funds to pursue wisdom, though he does not need wealth. He keeps his body healthy for the same reason. He does not use his rational part simply for the pursuit of money. He does not pursue the care of the soul solely to cure the body.

\paragraph{Faculties of the Soul}
\begin{itemize}
\item Rational power

Consists of the intellect and eros (love)

Directed toward the true, the good, and the beautiful 
\item Spirited power

Consists of conviction (or faith) and thumos

Directed toward ruling, conquering, and getting fame

Gymnastics strengthens this power and music moderates it and keeps it in harmony with reason. 
\item Appetitive Power

Consists of conjecture (or opinion) and desire

Directed toward eating , drinking, sensual pleasures, and material gain

There are three types of desire:

\begin{itemize}
\item Necessary desires 
\item Unnecessary and spendthrift desires 
\item Lawless desires 
\end{itemize}
\end{itemize}
\paragraph{Five Forms of the State}
Since there are five forms of individual life, there are five forms of the State corresponding to them. The State arises out of the predominant human characters. These forms are: the aristocratic, the timocratic, the oligarchic, the democratic, and the tyrannic. The first two follow from the rational and spirited powers and the last three from the three types of desires.

This schema can be tried as an analytic tool to understand current events, from the ``third dimension", as it were. The transition from one type of rule to the next documents a process of decline. Of course, any actual society will contain a mixture of the various types and their manifestations may vary quite a bit.

The important lesson here is that it is nearly impossible to communicate across these types, since their aims and their notion of rationality are so different.

\subparagraph{Aristocratic}
The aristocratic man, or the ``best", is virtuous and ruled by reason. He knows the divine ideas, eternal and timeless. Thus he is guided by nature, by the universal.

It is important to note that this definition is not at all what is commonly meant by ``aristocrat" today. For Plato, the aristocrat corresponds more to the priestly or spiritual caste, but without the religious overtones.

Rule by the best is unlikely. The other types are more driven to seize power. Moreover, while the aristocrat sees the life of reason as liberation, those who are dominated by the appetites experience it as oppression.

\subparagraph{Timocratic}
The timocrat is the honor-loving man. His guide is not wisdom, but rather law and custom. He is governed by belief and conviction which are time tested. The timocrat corresponds to what was later called the ``aristocrat" in the Middle Ages and to this day. He is the one that values his breeding and keeps portraits of his ancestors in the castle. These are the dukes, barons, and so on. In other words, this type corresponds to the warrior caste.

The timocrat is guided by custom and law, unlike the aristocrat; in other words, by the particular, not the universal. In the well-ordered state, he submits to the wisdom of the aristocrats. Otherwise, this creates some tension between reason and custom. Nevertheless, since people are always part of a community, even the aristocrats must be loyal to their own traditions. It is difficult to see where patriotism fits in, given Mr. Cavarnos' objections.

The timocrat is arrogant and ambitious, but his aristocratic fathers were rational.

\subparagraph{Oligarchic}
The timocrat may be a general or other high office holder. When such men are attacked, outlawed, or otherwise diminished, the oligarchic type arises. Seeing no value in the principle of the love of honor, he instead enthrones the principle of appetite and avarice. This is his psychic center. The rational and spirited part of him is used solely to acquire money.

The philosopher of this type (if that is the correct word to use) is Ayn Rand. While claiming to be a follower of Aristotle, her use of reason is restricted to the satisfaction of necessary desires. A society dominated by this type will not admire reasonable or honorable men, but rather rich men. The oligarch, like a Mr. Scrooge, may often be temperate in the satisfaction of bodily desires. He follows public opinion in most matters.

\subparagraph{Democratic}
The sons of the oligarchic type gradually seek out unnecessary pleasures. They lack the self-discipline of their fathers. All desires are equally alike and he cannot discriminate between better and worse desires. Hence, he discards moral standards.

\begin{quotex}
He calls insolence `good breeding'. license `liberty'. prodigality `magnificence' and shamelessness `manliness’. \flright{\emph{Republic}}

\end{quotex}
This type is easily recognized in a democratic society. For example, when extended to public policy, prodigality results in public debt. The last of the list exposes how even some self-described ``new right" types are really democrats in disguise.

This type is fickle, bouncing from pleasure-seeking to flirtations with philosophy or even spirituality. But there is little depth to his thinking. He often just says what pops into his head. He is not guided by reason, tradition, or common opinions. Rather, he has an inordinate craving for freedom.

\subparagraph{Tyrannic}
Too much freedom breeds dissension and anarchy. Mob rule results; a leader arises among them, and the appointed rulers are cast out.

The tyrannic type purges out of himself every desire or opinion that is deemed good, discards all self-control and brings in madness to the full. He has no respect for his parents. He is made up of fierce base desires and his life is the furthest from the life of reason. Plato includes murder and incest among the lawless desires, but they are just the peak. There are other such desires that are considered quite normal and respectable in our time.

In a tyranny, assuming the tyrant seizes control, the people are reduced to slavery. The tyrant stirs up wars and impoverishes his subjects by the imposition of heavy taxes. He gets rid of his boldest followers and purges the State of the wise and the brave.

\paragraph{Fifth Virtue}
Plato sometimes mentions a fifth virtue: holiness or piety (both translate the same Greek word). But this virtue belongs to religion, not philosophy, so philosophy is ultimately insufficient. As we have seen, piety was the most important virtue for the Romans. Therefore, the aristocrat of the soul should strive for holiness, beyond justice, wisdom, courage, and temperance.



\flrightit{Posted on 2014-12-14 by Cologero }

\begin{center}* * *\end{center}

\begin{footnotesize}\begin{sffamily}



\texttt{Ezra £ on 2014-12-19 at 21:04 said: }

According to Guenon, the spiritual caste is supra-rational and a traditional society is based on supra-rational principles. He attributes much of the moderna deviation to rationalism.


\hfill

\texttt{Cologero on 2014-12-19 at 23:34 said: }

Thanks for reminding us of that, Ezra, but keep in mind that this is a book review. I picked out key points from the book. As Mr. Cavarnos points out, there is the virtue of ``holiness" which transcends the rational. Mr. Caranovos was the one who reminds us that Platonism (or neoplatonism) is an exoteric teaching. Also keep in mind that the supra-rational is not equivalent to ``irrational"; this is important in an era where simply being rational would be an accomplishment. Here, of course, by rational, we mean knowledge of the divine essences, or being. Guenon makes this claim:

\begin{quotex}
It was only among the neoplatonists that Eastern influences were again to make their appearances, and it is there indeed that certain metaphysical ideas, such as that of the Infinite, are to be met with for the first time among the Greeks. 

\end{quotex}
Now it may or may not be true that this was the ``first" time. I think not, since the Greeks had learned some things from the Egyptians and Persians. Furthermore, in several other places, Guenon compares certain doctrines of Aristotle favorably with Eastern doctrines. So there is some inconsistency there.

I have also mentioned that Indian writers have often noted points of correspondence with Western philosophers of the rationalist type. Perhaps, then, the Easterners have become just as decadent. That is probably true, since much of contemporary Hindu spirituality is replete with fraud and deception. Hence, we have to start from where we are and from what we know, not from where we would like to end up.


\end{sffamily}\end{footnotesize}
