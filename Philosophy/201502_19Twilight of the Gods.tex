\section{Twilight of the Gods}

\begin{quotex}
Zeus and Christ, all of the inmates of the institution and all the gods in the rest home merged into one wildly incoherent supergod, but one so ancient, so grandly senile, so sweetly insane that even the grasses trembled at the very thought of his approach … the senile god shouted his incoherent truth to multitudes, who in turn killed their neighbors and rode in bitter triumph through endless savage wars. \flright{\textsc{Jane Roberts}, \emph{The Education of Oversoul Seven}}

\end{quotex}
In his massive volume, \emph{The System of Antichrist}, \textbf{Charles Upton} expends considerable effort and talent in analyzing several New Age movements in terms of Traditional principles. Unlike \textbf{Seraphim Rose}'s \emph{Orthodoxy and the Religion of the Future}, a similar work which gives no quarter to such movements, Upton is more sympathetic, pointing out where the movements are somewhat traditional and where they are not. Perhaps that is due to flashbacks as a youthful hipster when he took such ideas more seriously.

Although it is a good idea for those of a Traditional bent to comment on popular culture (if they can bear it), the purpose of the book is not clear. Traditionalists do not take such movements seriously in any case. New Agers will hardly be convinced; Upton relates that a New Age group called his views “patriarchal” to his face. Yet, \textbf{Rene Guenon} himself wrote long tracts against Theosophy, spiritualism as a religion, and even Mormonism, so perhaps it is worth the trouble. The more interesting parts of the book are the interludes in which he ably expresses his own views of Tradition.

Unfortunately, the real question of why New Age ideas have taken hold is not answered. In Guenon's views, the New Age can be nothing but a form of degeneration. However, it can only take hold because of the spiritual vacuum in the West. In preparation for this review, I researched some of the movements on Amazon and meetup. I was surprised to see that the Course in Miracles is in the Top 10 list of Christian self-help books and that there are several meetups just in my area. The Seth books, as well as the Course, have hundreds of reviews on Amazon, including many positive stories of self-transformation.

\paragraph{Reality Creation}
While Guenon was looking to recover the primordial Ur-Tradition, New Age writers like \textbf{Jane Roberts} took Nietzsche's idea of the Twilight of the Gods as the point of departure. As her parable in the epigraph shows, the old gods have become senile. It's not that they are merely ineffective, but, on the contrary, when they take action, people die.

Roberts was a sci fi writer, but only became successful with the Seth books. You can think of him as Jane's performance art or as the discarnate voice of a higher being who is obsessively interested in events on earth. I used to own a few books but gave them away years ago. I checked again and noticed that there now seem to be a few dozen books available. Apparently Seth won't shut up.

Upton carefully goes through the material, and whenever he finds something interesting, he points out that the Sufis discovered it first. Seth teaches about the multidimensional god at the top of the hierarchy of beings. God is an idea, bearing in mind that ideas are the most real. Did not Guenon affirm that possibilities are as real as things? Hence, angelic intelligences are really ideas, although perhaps non-formal manifestation would mean the same thing.

Seth also seems to know about the degrees of existence. A being exists in multiple states which can be described as simultaneous. Seth describes this as “reincarnation” but Upton makes a better case for “transmigration” instead. That is a large topic for another occasion. Nevertheless, Guenon also teaches that a being exists in multiple states that are both simultaneous and sequential.

Seth also focuses on creativity over being. Upton objects to Seth's claim that a being strives to become more itself. It seems, however, that this is what Guenon may mean by a being actualizing all its possibilities. It is also a theme common to \textbf{Nicholas Berdyaev} and even \textbf{Julius Evola}.

What Seth is most known for is the idea that we create our own reality. Now as an absolute statement, that is impossible to accept. But in a relative sense, it is important. It is hardly a new idea, since \textbf{Plato} discussed it in the \emph{Republic}. There we read that Er had to choose the conditions of his birth. It comes down to a question of being the active agent in our lives or of being passive in the face of circumstances.

Seth claimed, “It is quite possible to take your normally conscious `I' into the dream state, to your advantage. When you do this you will see that the dreaming `I' and the waking `I' are one, but operating in entirely different environments.” This is called lucid dreaming. I'll leave you with a personal story.

\begin{quotex}
Last week, as I was reading this chapter, I was coming down with a severe bronchitis attack. While asleep, I became aware of my labored breathing and wheezing. Alongside that awareness, I “heard”, or silently “said”, over and over: “You create your own reality.” When I woke up, I found that the bronchitis had cleared up. 

\end{quotex}
\paragraph{Power and Shamanism}
\textbf{Carlos Castaneda}'s books were my favorites at one time. Carlos had to wander in the desert to find the rather dangerous shaman, \textbf{Don Juan}. Nowadays, it seems that shamans abound. I had dinner with a nice lady a few months back. She told me she had her own shaman teacher. Although she was quite interested in Gornahoor as I described it, she seemed disappointed that it doesn't generate any cash. My sister thinks I'm a shaman, so maybe I'll print up some business cards and sell my services. I just need to work on that shape shifting thing.

Upton takes Castaneda's experiences as real and admires his literary skills at describing certain states of non-ordinary reality. Don Juan is a \textbf{Man of Knowledge}, and a master of Power. The key to Power is Will.

The goal of the Man of Knowledge is to avoid death by remaining conscious. This is not as far-fetched as Upton believes. There is value to being conscious at the moment of death. As an exercise, stay conscious as your breathe. Pay attention, in particular, to the exhalation. Consider that it may be the last breath you take. Can your mind withstand that shock of physical death?

Don Juan and Castaneda inhabit a world of psychic events. They are not false although we evade becoming conscious of such things. For example, the experience of familiar spirits was common to our ancestors as we've pointed out several times. Psychic attacks are likewise real; Guenon himself claimed to have been the target of such attacks.

Upton also objects to the idea of a “mold” for man. But that is just the Thomist form. Is that mold God? Perhaps it is just an understandable mistake, given that St Theresa d'Avila claimed that if you could see a soul in all its purity, you might mistake it for God.

Like Seth, Don Juan teaches that we create our own reality to some extent. That is the “\emph{tonal}”, i.e., everything knowable and intelligible. The \emph{nagual} is beyond definition; it is power. As a practice, the shaman can see the arbitrary nature of the tonal, how it is a mental construction, and thereby reach the nagual. Readers will recognize this as the Hermetic test to question all one's assumptions. In my experience, the beginning stages of a spiritual path may feel like incipient insanity as our naïve world conception dissolves, only to be recreated at a higher level.

Upton is somewhat uncomfortable with Don Juan's teaching of raw power. He correctly relates it to \textbf{Shakti}. That is also one of Evola's accomplishments, to incorporate Power into spiritual life. In a Hermetic work like \emph{Gnosis}, by \textbf{Boris Mouravieff}, knowledge or gnosis is just the first stage of transcendence. Ultimately, it will lead to the True Will.

\paragraph{Miracles}
\textbf{Richard Smoley} pointed to three books that appeared last century about what he calls “esoteric Christianity”: \emph{Meditations on the Tarot}, \emph{Gnosis}, and a \emph{Course in Miracles}. Now we have mentioned the first two many times. They each insist on ties to legitimate exoteric traditions; that is what we would expect from a real esoteric teaching. The Course, although it claims to be dictated by Christ, is more ambiguous. Hence, if it has any value, then it must be read as a sort of “third testament”, perhaps along the lines of \textbf{Joachim de Fiore}'s third Age of the Holy Spirit. But that is not how most of its adherents read it.

The course consists of two main parts: a workbook with 365 daily exercises in applying its principles in daily life. The text contains the core metaphysical teaching. I need to admit here that I followed the course decades ago when it first came out. I followed the exercises carefully for a year, carrying around a little card with each daily lesson as a reminder. I am sure it has affected me to this day, not always in conscious ways. Specifically, it was compatible with my own understanding of a sort of Vedantized Christianity. Now Guenon believed he was psychically damaged from his early involvement with certain occult movements in Paris. I don't think I will accept that, since there is no guilt, only forgiveness. Upton admits

\begin{quotex}
There is a great deal of profound truth in \emph{A Course in Miracles}: the uncompromising sense of God as Absolute Truth and Love, deep insight into the convoluted games the ego plays to escape this Truth and Love, and understanding that the subject/object mode of consciousness cannot directly witness Absolute Truth; the doctrine of one and only choice which is completely free, that between Truth and illusion; the primacy granted to forgiveness in the process of metanoia, that total change of mind by which Truth is chosen and illusion dismissed; the doctrine that humanity never really fell into sin, never entered into the illusion of separation from God. 

\end{quotex}
After that glowing introduction, one wonders why the next 30 pages are devoted to debunking it. Like Seth, the Course gives the Self power over reality. One exercise starts with this:

\begin{quotex}
I am responsible for what I see. I choose the feelings I experience … everything that seems to happen to me I ask for, and receive as I have asked. 

\end{quotex}
Of course, if the ego says this, it is false. The goal, then, must be to rise up and experience life from the Holy Spirit. Ultimately, just as he did with the other movements discussed, Upton points out its many inconsistencies, its misunderstanding of true metaphysical doctrine, and its active opposition to the true Tradition.

Nevertheless, in a way it has served as a model for the Gnosis Study Group. We have daily exercises, it is based on sound metaphysical principles, and it is tied to a valid exoteric tradition. I'm afraid it does not generate wide interest, but what interest it does generate is intense.

\paragraph{Conclusion}
These teachings fill a vacuum that the western Tradition has lost. There is the active call to create one's own life. The shaman points out the loss of theurgy. And the course fills the need for transcendence and a concrete spiritual path. So on the one hand, the movements reviewed by Upton sometimes bring out forgotten aspects of Tradition. On the other hand, they all oppose valid Traditions in their fullness. They promise secret knowledge and cheap grace. As far as I can tell, they end up in Dante's seventh circle of hell.

Upton is intelligent enough to understand traditional doctrine and create his own religion that is not found in any typical church or mosque. It is clear that these New Age teachings awakened in him the desire for something more. That they fall short of full metaphysics is quite understandable. After all, what religious documents have clear metaphysical teachings? None. It is up to those capable of it to tease such teachings out of the symbolism.



\flrightit{Posted on 2015-02-19 by Cologero }

\begin{center}* * *\end{center}

\begin{footnotesize}\begin{sffamily}



\texttt{rhondda9 on 2015-02-22 at 13:36 said: }

I read Upton's book a couple of years ago. To me it read like a confused confession. In parts he is talking about Christianity and in other parts about Islam and it was as if he wasn't sure to which one he was committed. Then he would bring in his wife who is a Christian and it just got more and more confusing. Dante was in there too.

When I looked up Seraphim Rose, there was a book called Nihilism. There was a quote that hooked me. I cannot remember it exactly but something about the greatest denial expressing the greatest need. I have ordered the book.


\hfill

\texttt{Cologero on 2015-02-25 at 23:34 said: }

I agree, Rhonda, but I didn't want to make it about Upton. Rose's Nihilism is worthwhile.


\hfill

\texttt{Cologero on 2015-02-26 at 00:03 said: }

There is a Christian gnosis, theosis is ancient Christian teaching, what exactly is the “unitive way”, what is the proper interpretation of panentheism (actually the position of the Eastern churches), why did the Fathers have a high regard for hermetism (I don't accept the modern methods of dating texts), why did the ancient church distinguish between catechumens/faithful/perfect, what are the roles of the angels, and so on and on. Now JP II had high regard for Vladimir Solovyov … what about Solovyov's relationship to gnostics, hermetism, the divine Sophia? There is an orthodox way to understand all these things, apparently and unfortunately they have been ignored or forgotten. Gornahoor as a whole is a commentary, although we prefer readers go back to the original sources. 

\hfill

\end{sffamily}\end{footnotesize}
