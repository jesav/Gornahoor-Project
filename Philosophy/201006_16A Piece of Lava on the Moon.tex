\section{A Piece of Lava on the Moon}

\begin{quotex}
\emph{It would be easier to convince most people to consider themselves a piece of lava on the moon than a self.}

Those who are not in agreement with their self in this sense will not understand any comprehensive philosophy nor to they need to. Nature, whose machine they are, will lead them without their participation in everything they must do. To philosophize, one must be independent, and independence is granted only by one's self. We should not desire to see without eyes, but neither should we assert that it is only the eye that sees.

\flright{\textsc{J. G. Fichte}}
\end{quotex}

Julius Evola, who was a close student of German philosophy, took up this theme, when he claimed that most people are not the active agent of their lives, but they are ``lived" by larger forces. Yet, we are constantly surprised by how often this proves true. While few people believe they are pieces of lava on the moon, there are large numbers of people who readily believe they are pieces of earth matter, or thinking meat, or electro-chemical processes, or the result of some sort of mechanical evolution. There is a remarkable ability to deny or avoid the self. 

As Johannes Tauler wrote: ``If I were a king and did not know it, I would not be a king." So, if I am a Self and do not know it, I am not a Self. So, keeping in mind the identity of Knowledge and Being, I am a Self if and only if I know myself as a Self.



\flrightit{Posted on 2010-06-16 by Cologero }
