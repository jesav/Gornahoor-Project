\documentclass[a4paper,12pt,twoside]{book}
\usepackage[utf8]{inputenc}
\usepackage[english]{babel}
%\usepackage{fontspec}
%\setmainfont[
%  Ligatures=TeX,
%  Extension=.otf,
%  BoldFont=cmunbx,
%  ItalicFont=cmunti,
%  BoldItalicFont=cmunbi,
%  SlantedFont=cmunsl
%]{cmunrm}

%\usepackage{polyglossia}
%\setmainlanguage{spanish}

\usepackage[c5paper]{geometry}
%\geometry{inner=2.5cm,outer=2.5cm,bmargin=3.2cm}
%\usepackage[DIV=14,BCOR=2mm,headinclude=true,footinclude=false]{typearea}

%\usepackage{ulem} %Hace que \emph sea subrayar

%\usepackage[p,osf]{scholax}
%% T1 and textcomp are loaded by package. Change that here, if you want
%% load sans and typewriter packages here, if needed

\usepackage{ebgaramond}
%\usepackage[type1]{libertine} % Linux Libertine for zweispaltige Texte
%\usepackage{textcomp}% Required to get special symbols
\usepackage[scaled=.8]{DejaVuSansMono}% FiraMono Typewriter font
\usepackage{PTSansNarrow} 
%\gilliuscondensed
%\usepackage[sfdefault]{FiraSans}
%\usepackage{bm}% Extra bold faces
%\usepackage[lf]{carlito}

\usepackage{lettrine} %Capital letters at the beginning of a chapter
\usepackage[activate={true,nocompatibility},final,tracking=true,kerning=true,spacing=true,factor=1100,stretch=10,shrink=10]{microtype}
\SetTracking{encoding={*}, shape=sc}{-20} % versalitas menos separadas
% activate={true,nocompatibility} - activate protrusion and expansion
% final - enable microtype; use "draft" to disable
% tracking=true, kerning=true, spacing=true - activate these techniques
% factor=1100 - add 10% to the protrusion amount (default is 1000)
% stretch=10, shrink=10 - reduce stretchability/shrinkability (default is 20/20)

%\usepackage{array,multirow,booktabs,colortbl,chngcntr} % El último es para counterwithout;
\usepackage[strict]{changepage}
%\usepackage{caption}
%\captionsetup{format=plain,labelsep=newline,labelfont={small,sc},
%textfont={small,it},singlelinecheck=false}
\usepackage[Bjornstrup]{fncychap} % Para cabeceras de capítulos sofisticados:     Sonny,    Lenny,    Glenn,    Conny,    Rejne,    and Bjarne.

\usepackage{graphicx,wrapfig,booktabs,multicol} % wallpaper: poner imágenes de fondo; wrapfigure: figuras a un lado del texto
%\graphicspath{{figures/}}
\usepackage{fancyhdr}
\usepackage{emptypage,pdfpages,fancybox} % Para que las páginas en blanco no tengan encabezado;
\usepackage{enumitem} %paralist: para compactenum, enumerate sin espacios
%\setlist[itemize]{nosep} %Espacio entre items en itemize
\usepackage[hyperref]{xcolor}
\usepackage[hidelinks]{hyperref}
\usepackage{xurl}

%\usepackage{minipage}

%\usepackage{quotchap} %Encabezados de capítulos
\usepackage{syntonly,verbatim}
%\syntaxonly

\usepackage{setspace,xspace} % xspace: Da \xspace para no tener que poner {} después de los comandos; pdflscape: páginas en horizontal;

\newenvironment{quotex}{\begin{quote}\small}{\end{quote}}
\newenvironment{quotationx}{\begin{quotation}\small}{\end{quotation}}

%Bibliografía
\usepackage[backend=biber,style=mla]{biblatex}
\addbibresource{bibliography.bib}

\begin{document}

% Por alguna razón, los marginados se creaban al revés. Así los corrijo. Feo, pero eficaz:
\let\tmp\oddsidemargin
\let\oddsidemargin\evensidemargin
\let\evensidemargin\tmp
\reversemarginpar

%\setcounter{secnumdepth}{4} % Para que llegue a numerar hasta las subsubsecciones;
%\renewcommand{\heavyrulewidth}{0.14em} % Grosor de las líneas extremas de las tablas;
%\renewcommand\thempfootnote{\alpha{mpfootnote}} % Símbolo de notas dentro de minipage
%\let\oldcaptionof\captionof
%\renewcommand{\captionof}[2]{\oldcaptionof{#1}{\newline \textit{#2} }}
%\renewcommand{\tablename}{Tabla}
%\counterwithout{figure}{chapter}
%\counterwithout{table}{chapter} % Así la numeración es 1, 2, 3... y no 1.1, 1.2... y no reinicia la num. en cada capítulo;

%\providecommand{\ggl}{\guillemotleft}
%\providecommand{\ggr}{\guillemotright\xspace}
\providecommand{\flright}[1]{\begin{flushright}#1\end{flushright}}
\providecommand{\flrightit}[1]{\begin{flushright}\itshape #1\end{flushright}}
%\renewcommand\UrlFont\sffamily
\urlstyle{tt}

\pagestyle{fancy}
\renewcommand{\sectionmark}[1]{\markright{#1}}
\renewcommand{\chaptermark}[1]{\markboth{#1}{}}

%Portada:
%\includepdf{00Portada}

\frontmatter
%
%\onehalfspacing
%\pagenumbering{Roman} %gobble es como empty

%\include{preindice}

%\fancyhf{}
%\fancyhead[LE]{\small \textbf{\thepage}$\quad$ Índice general}
%\fancyhead[RO]{\small Índice general $\quad$\textbf{\thepage}}
%\clearpage
%\tableofcontents

%\doublespacing
\mainmatter

\fancyhf{}
\fancyhead[LE]{\small \thepage$\quad$ \nouppercase{\itshape\leftmark}}
\fancyhead[RO]{\small \nouppercase{\itshape\rightmark} $\quad$\upshape\thepage}
%\cfoot{\bfseries\thepage}

\pagenumbering{arabic}

\nocite{*}
\onehalfspacing
\printbibliography

\end{document}
