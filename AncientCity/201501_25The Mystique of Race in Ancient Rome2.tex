\section{The Mystique of Race in Ancient Rome II}

\begin{quotex}
This is the concluding part of the essay by \textbf{Julius Evola} titled ``La Mistica della Razza in Roma antica". It was originally published in the journal La Difesa della Razza.

Beyond the specific issues concerning Ancient Rome, I hope some of you will recognize two larger points.

First of all, this essay confirms what we have said about the aristocratic interiority being centered on the higher mind, the mens, the ajna chakra, the seat of intellect and intuition, that commands the lower soul and the body. To be blunt, that describes the true ``aristocrat of the soul".

The second is the idea of destiny and the spiritual impetus behind it. A people, a family, and so on, have certain spiritual qualities at their origin. The purpose of rites and worship was to keep those qualities present in order to project them into the future. The modern mind, including most neo-pagans, cut off the past in order to create a future with no relationship to it. But human nature will not be denied. Instead of being guided by past heroes and sages, able to distinguish good and evil, the deracinated contemporaries will latch onto any spiritual influences at random. Devoid of true identity, they will create artificial identities.

\end{quotex}
The ``presence" of the \emph{genio}, the lares or the penates in the group to which it corresponded, was made aware and symbolized by the \emph{fire}, the sacred flame, that had to burn uninterruptedly in the center of the patristic houses, in the temple placed in the \emph{atrium}, the place where the \emph{pater familias} celebrated the rites and in which the various members of the domestic or aristocratic group were gathered for meals, for example, which itself had a ritualistic significance in ancient Roman and Aryan life. For example, a portion of the food was reserved for the god of the domestic fire, in order to remember the unity of life that connected the individuals to him—a unity of life and also a unity of destiny. In certain aspects, in fact, the \emph{genius}, beyond being the principle that determines the fundamental traits of the individuals arising under his sign, was also conceived as the directing principle of his most important and most decisive acts, like the one who helps and guides him, so to speak, from behind the scenes of his finite consciousness, becoming the ultimate cause of his destiny, both good and evil, that was intended for him. In that way, this being of the ancient Roman racial cult successively gave rise to popular depictions, which however conserve very little of the original meaning: we can for example recall the undeniable relation of the \emph{genius} with the popular Christian conception of the ``guardian angels" or of the good and evil angels, these images that have become absolutely mythological and deprived of the essential and concrete relation with the blood and mystical forces of the race.

The intimate connection existing between the individual and the \emph{lares}, the \emph{genius}, and in general with the divinity symbolized by the sacred fire of a given bloodline, and the living character, assumed to be present and acting in such a divinity, explain the peculiarities of the ancient cult. This entity of the fire appeared as the natural intermediary between the human world and the supernatural order. Starting from the idea of the unity, fulfilled in the bloodline and in the race, of the individual with a force that, as the \emph{genius} or the \emph{lares}, was more than physical, ancient man was convinced of the real possibility of the influence on his own destiny precisely in this way. Special rites had to propitiate and ennoble in order to ensure that a transcendent influence was of help to his strengths and actions through the mystery of blood and race to which he belonged. A specific character of the most ancient cults of the most ancient Aryan societies was its anti-universalism. Ancient man did not turn to a God in general, a God of all men and all races, but the God of a lineage, in fact, of his \emph{gente} and his family. And vice versa: only the members of the group that corresponded to them, could legitimately invoke the divinity of the domestic fire and to think that their rites were efficacious. It is easy to pronounce negative judgments and formulaic stereotypes, like that of ``polytheism"; it is difficult to clarify what that was about in the ancient world, because the meaning of the ancient religion became almost entirely lost in the ensuing centuries. We limit ourselves to make two points.

First of all, there is a visible hierarchy that legitimizes the ancient aristocratic-racial Aryan and Roman cult. In an army, one does not directly address the supreme leader, but rather the hierarchy on which he immediately depends, because of the fact that he, or the individuals closest to him, were able to settle the situation, without needing to go higher up. Likewise, admitting a universal God was not a reason to exclude every intermediary and to condemn any reference to the particular mystical forces that are closer to a folk or race and connected in a concrete unity of destiny and life. Celsus even brought up the hierarchical argument against the accusation of polytheism made by the Christians by observing, by analogy, that whoever pays tribute to obedience to an authority delegated to the government of a given province implicitly pays tribute to the central government, while whoever claims to address it solely and directly, beyond being impertinent, can, in reality, be acting in an anarchic way. And it is well known that Romanity, beyond particular aristocratic cults, also recognized more general cults, parallel to the universality to which the eternal city gradually elevated itself, and also indicates on the level of entities, like the \emph{lares} or \emph{genii} themselves, because there was also a national conception of the \emph{lares}, for example, where they attributed a cult to the \emph{lares militares}, or they spoke of the \emph{lares publici}, or they referred to the mystical force of the imperial lineage, to the ``demigods who founded the city and established the universal empire", or they introduced the idea of ``genius or universal demons".

In the second place, ancient traditional man did not reduce the cult to a mere sentimental disposition for which the rite was only an empty ceremony. Those who considered the relationship between the human world and the divine as real and effective, thought that there existed precise conditions. One of these was race and blood. Even without wishing to enter the complex field of the metaphysical presuppositions of the cult, it appears evident that the force, to which the individual thought he owed his life, that he supposed ``present" in his same body but to which he attributed superindividual and supernatural characteristics, was conceived as the most direct and positive path to return to what is highest in life. The race, as race of the spirit, was therefore a religious value, it contained a sacrament, it was hidden by ``magic", and that for considerations, one must recognize it well, in their positive and realistic mode.

The oath on the \emph{genius} in Roman antiquity was made while touching the center of the forehead, and the cult of the \emph{genius} itself did not lack a relation with that of the Fides, the personification of essentially Aryan and virile virtue, of fidelity and loyalty. The detail related to the gesture of the oath is, for every expert, rather interesting, because it related the \emph{genius} and the entities similar to it back to \emph{mens}, to the intellectual and virile principle of life, hierarchically superordinate both to the soul and to the purely corporeal forces: it cannot be by chance that the place attributed by the Roman tradition to \emph{mens} — the center of the forehead — was that which in the Indo-Aryan tradition certainly assigned the ajna chakra to the force of ``transcendent virility", and to the so-called ``center of command". With that in mind, the suspicion is unlikely, that in the Roman family cult, if not exactly of superstitious personifications, was a type of ``totemism", the totem being the dark entity of the blood of a tribe of barbarians, related to the forces of the animal kingdom. We see instead that the ancient Roman world gave to the gods of the race and family group precisely some supernatural traits, the mind, \emph{mens}, or the \emph{nous} conceived in Mediterranean antiquity exactly as the supernatural and ``solar" principle of man.

Certainly, we must not generalize and think that it is about that in every case. The traditions encompassed in the ancient Roman world are more varied and complex that has been supposed up to now. Both ethnically and spiritually, diverse influences met in the most ancient period of Rome. Some are actually related to inferior forms of cult — inferior either by belonging to a non-Aryan ethnic substrate, or by representing a regressive and materialized form of somewhat more ancient cults, of Aryan and particularly Atlantico-Occidental origin. That is valid also for the cult related to mystical forces of blood, race, and family, that in some cases and phases has, let us admit, ``crepuscular" traits, with special regard to their inferior chthonic aspect predominantly related to that matching instead celestial and super-terrestrial symbols.

One can nevertheless not contest the idea that in the greater number of cases the highest tradition was present in Rome and that in its development Rome was able to ``rectify" and purify to a not negligible measure the different traditions that it had included. So against the myths which, in reference to the cult of the \emph{lares} at Acca Larentia, to the \emph{re plebeo} Servio Tullio, and to the Sabine element remaining at an inferior level, we have the ``heroic" elements of the cult of the lares and penates and such elements assume ever more significance in the events at the time of the Empire. Some think that the same term ``\emph{lares}" comes from the Etruscan \emph{lar}, a word that means leader or chief, which however was related to chiefs and leaders like Porsenna and Volumnio. A very widespread tradition among the ancients, of which it suffices to recall Varrone, identifies the \emph{lares} with the ``heroes", in the Greek sense of demigods, of men who have transcended nature and were made participants of the indestructibility of the Olympics so that it validates, in spite of its generalization, Mommsen's idea through which every gens would have had as one of its heroes, the principle of the people that was venerated precisely in the person of the \emph{lar familiaris}.

The supernatural and ``regal" side of the ancient cult of the mystical forces of blood is emphasized with that. This is not everything. On the one hand, the funereal epigraphs attest to the Roman faith that the principle of immortality for his descendants was the lares themselves: many epigraphs do not indicate the negative ``telluric" possibility of a type of dull and nocturnal post mortem survival in an underworld, but they affirm the higher idea that death is the principle of a superior existence. They put death exactly in relation, to which they were dedicated, with the \emph{lares} or heroes of his people. On the other hand, as previously noted, Romanity would universalize the notion of the \emph{lares}, extending it to the central dominating force of Romanity. We find therefore the inscriptions dedicated to the \emph{lar victor}, the \emph{lar martis et pacis} and finally to the \emph{lares Augusti}. It is already in an environment in which it is not about more of the race as gens and nuclear family, but as folk and political community. Even outside the race so conceived, a divine force, a mystical entity, is presented, connected to the destinies of war, victory, and triumphal peace—\emph{lar victor}, \emph{lar martis et }\emph{pacis}—and connected finally to the ``genius", to the generating principle of the leaders, the Caesars, to the \emph{lar Augusti}.

With that we will now discuss a very different subject which is the Aryan conception of the fortune and destiny of the leaders, the city, and nations. For now, we believe we have brought sufficiently to light the meaning of the mythical figurations and cults typical of the ancient Roman peoples, where unequivocally the consciousness of blood and race resided and where religiosity was not a factor of evasion and universalism, but constituted the most solid cement of the unity of folk and bloodlines. The mystery of blood was a central idea of ancient Roman spirituality and to disregard it means to be condemned to a superficial and profane understanding of the most tangible, noted, and celebrated aspects of the law, custom, and ethics of ancient society.



\flrightit{Posted on 2015-01-25 by Aeneas }
