\section{Priest and King in Rome}

Although we have written about caste in \textit{Caste and Social Order}\footnote{\url{https://gornahoor.net/?p=1698}}, it will be helpful to address it more broadly. \textbf{Georges Dumezil}, whose research into the social order of Indo-European peoples came too late to have any influence on either Guenon or Evola, extends our understanding through common historical, philological and mythological patterns. We intend to focus here on the ancient Roman system.

\paragraph{Functions}
There is often a confusion over the nature of ``hierarchy", particularly for those who can only think in terms of rigid structures of command and control. Instead, we think to think in terms of functions and dialectical relationships. Dumezil recognized three fundamental principles, or functions, which correspond to the inner nature of three different types of men. These are:

\begin{enumerate}
\item The maintenance of cosmic and juridical order 
\item The exercise of physical prowess 
\item The promotion of physical well-being 
\end{enumerate}
Clearly these functions correspond to the three higher castes: Brahman, Kshatriya, Vaishya. In one sense, they are in opposition, but that is resolved in a higher understanding. For example, the ``spiritual authority" of the Brahman and the ``temporal power" of the Kshatriya can be united in a higher principle, viz, \textbf{the will to maintain the cosmic order in the physical realm}. Then, the three castes are united in the principle of initiation, the ``twice-born", as opposed to the Sudra who are not initiated. One could even go further: the four castes from the entire social body and, as such, in opposition to the outcastes.

\paragraph{Mythology}
The social order on earth is a reflection of the order of the gods. For the Romans, \textbf{Jupiter}, \textbf{Mars}, and \textbf{Quirinus }are the gods for the respective three functional castes. Note how this mirrors the functional caste structure. Jupiter represents the highest, or Brahman, caste, then Mars the Kshatriya and finally Quirinus, the Vaishya. Furthermore, since the function of the Brahmin caste has two aspects — one facing the cosmic order and the other facing the juridical order of the city — it is represented by two gods. In the case of the Romans, these are \textbf{Jupiter} and \textbf{Dius Fidius}. These, then, would correspond to the Royal function and the properly priestly function.

\paragraph{Nature of Authority}
It is easy to forget, in our day, that life for the ancients was totally dominated by religious conceptions, down to the smallest detail. Every important action was accompanied by a rite; even the home had its sacred hearth and daily rituals, with the paterfamilias acting as priest. There was no notion of positive law\footnote{\url{https://gornahoor.net/?p=447}}; every law of the city was considered to be of divine origin. We see in the example of the Spartan battle at Plataea\footnote{\url{https://gornahoor.net/?p=2024}}, that their action is utterly irrational from the secular perspective: \emph{while their kinsmen are dying from the attack of the Persian archers, the Spartans passively hold their ground until the diviner determines the proper moment to initiate the battle just from ``reading" the entrails of a sacrificed animal}.

\paragraph{Priest and King}
Just as the father was both head of the family and priest at home, so also, in the ancient traditions, the chief priest of the city religion was also its king (also called \emph{prytanes} or \emph{archon}). This shows that there was no separation between the royal initiation and the priestly initiation. The king would maintain the public sacred hearth, offer sacrifices, lead the prayers, and preside over religious ceremonies. Fustel de Coulanges describes the inauguration of a Roman king:

\begin{quotex}
These king-priests were inaugurated with a religious ceremonial. The new king, being conducted to the summit of the Capitoline Hill, was seated upon a stone seat, his face turned towards the south. On his left was seated an augur, his head covered with sacred fillets, and holding in his hand the augur's staff. He marked off certain lines in the heavens, pronounced a prayer, and, placing his hand upon the king's head, supplicated the gods to show, by a visible sign, that this chief was agreeable to them. Then, as soon as a flash of lightning or a flight of birds had manifested the will of the gods the new king took possession of his charges. … There was a reason for such a custom; as the king was to be supreme chief of the religion, and the safety of the city was to depend upon his prayers and sacrifices, it was important to make sure, in the first place, that this king was accepted by the gods. 

\end{quotex}
Here, if nothing else, we see the notion of the divine right of kings and the role of lightning as a sign from God, even to this day.

Since the Traditional man depended on the gods in all aspects of his life, he certainly depended on the priest who was the intermediary between himself and the gods. His power of prayer, sacrifice and augury made him the clear leader of the city. Thus, the chief priest was also the magistrate, judge and military chief.

\paragraph{Romulus and Numa}
Romulus, the first king of Rome, knew the science of augury and founded the city in accordance with religious rites. He created the cult of Jupiter and build the first temple to the god. Romulus was a warrior king and urged the Romans to cultivate the art of war. While Romulus aggrandized Rome through war, the next king, Numa, aggrandized her through peace. Numa reformed and codified the Roman cult, with its three branches devoted to Jupiter, Mars, and Quirinus. For the bloody sacrifice of his predecessor, he substituted the unbloody sacrifice of bread and wine. He also founded a shrine dedicated to Fides Publica and taught the Romans the oath of Fides, which refers to faith. Dumezil writes on this topic:

\begin{quotex}
When Christianity gave the substantive noun ``faith" and the verb ``believe" the overtones they still have today, it was at the very least rediscovering and revivifying very ancient usages. 

\end{quotex}
\paragraph{Conclusion}
We hope that this will clear up some questions, while raising even more serious ones. First of all, it establishes the rationale for the caste hierarchy. The ruler is not subservient to the priest because, in fact, the ruler is the chief priest. This understanding persisted into the Hermetic Tradition as we saw in Thomas Campanella's \textit{City of the Sun}\footnote{\url{https://gornahoor.net/?p=518}}, in which the metaphysician is also the ruler. Furthermore, unlike the priest of today, the priest had real power (spiritual, not temporal): his prayers and sacrifices are efficacious, his auguries insightful.

In Romulus and Numa, we see the beginning of the dialectic that persists to our day: the warrior-priest vs the mystical priest; the terrible vs the ordered; Dionysus vs Apollo. Numa's successor, Tullus Hostilius, mocked Numa and his institutions, as well as piety to the gods, on the grounds that it made men cowardly and effeminate.

For the ancients, both the laws of Romulus and those of Numa were divinely inspired, so they lived with the contradictions. For the Hermetist, every apparent opposition is a polarity that is resolved in a higher synthesis.

References:

\emph{The Ancient City}, Numa Denis Fustel de Coulanges

\emph{Mitra-Varuna}, Georges Dumezil

\emph{Gods of the Ancient Northmen}, Georges Dumezil

\emph{Archaic Roman Religion} (2 vols), Georges Dumezil



\flrightit{Posted on 2011-04-04 by Cologero }

\begin{center}* * *\end{center}

\begin{footnotesize}\begin{sffamily}



\texttt{Max on 2018-08-30 at 18:06 said: }

The same type of opposition as that between Romulus and Numa is envisioned by Vico in the tension between the Israelites emphasis on law and reason and the divine as accessible interiorly, contrasted to the work of providence active in the gentiles auspicious understanding of embodiment. Neither of them necessarily invalidates the other, and we can even combine the two currents.

The conclusion of an article by John Headley ``On the Rearming of Heaven: The Machiavellism of Tommaso Campanella" was quite revealing:

``The actual indebtedness of Campanella to Machiavelli was more than peripheral, exceeding simply the incidental resort to cunning tactics. By profoundly appropriating the idea of religion's social and political utility, originally a product of Paduan Averroism, Campanella joined many political theorists of the Counter Reformation in judging religion by its effects, its utility, while nevertheless maintaining for himself its claims to truth. Similarly preoccupied with power and its effective exercise in this world, Machiavelli formulated and imparted to Campanella what became the central problem of his life: the empowerment of Christianity. The friar attempted to resolve for his own age the question which the secretary had hesitated to address in his own time; in his quest to achieve the predominance of a viable ecclesiastical state in Italy as well as papal theocracy throughout the world, Campanella in effect took up Machiavelli's challenge to realize a politically militant Christianity. That other interpretation of Christianity, to which Machiavelli occasionally alluded, Campanella would spend a lifetime pursuing in order that heaven might truly be rearmed."

Very few wish to touch upon political Christianity today. Although for most of history ``church and state" were the same. In ancient Israel, the state and its administration was basically the temple. The utility of religion need not be controversial, and does not detract from a higher understanding. It would be more surprising if it were no good for anything. Christian institutions have unfortunately come to downplay their own potential.

Your quote of Rushdoony in the introduction to Pagan Imperialism was intriguing. The word moloch supposedly comes from the combination of the consonants for king, malik, to the vowels for shame. However, Melki-tzedek, who Guénon says is the link to the primordial tradition, is ``The King of Righteousness", and senior to the priesthood of Aaron. Is there anyone who knows what Campanella had to say about Melki-tzedek?

Rushdoony left me confounded, so I am unsure of whether it is worth pursuing him further. It cannot be right that what is deemed heretical if uttered by a king, becomes sacred when restricted to the confines of a book. It might be true that all law, ius, is originally religious in nature, but is not one of the main features of sacred law to avoid having to worship the law book as an idol. Not to criticise law as such, but presumably, all words of God passed through a human of some sort before being written down, notwithstanding that it is also wise to preserve exceptional revealed words for future generations. Those words must however be made alive again continuously.


\hfill

\texttt{Matt on 2018-08-30 at 23:30 said: }

Max,

I have just recently started reading Rushdoony. He was a name I would sometimes hear others mention – usually accompanied with hysterical shock and horror. I decided to finally start reading some of his work, since, like you, the quotation Cologero provided in his introduction caught my attention. Some of his work can be found on online sites like scribd. 

I'm currently reading his work The One and the Many: Studies in the Philosophy of Order and Ultimacy. Rushdoony thinks the relationship between the One and the Many is one of the key metaphysical questions a person needs to consider. The book's main thesis is essentially that all non-Christian religions and worldviews focus on one at the exclusion of the other (the One is primary, or the Many is primary), whereas Christianity – Reformed/Calvinist Christianity, what he really means – brings the One and Many into the Godhead as the primary reality. From there, he attempts to chart all the real world implications (past, present, and future) that arise from the chosen answer.

I'm close to halfway through the text; it's an engaging read. I'll hold off on any definitive conclusions about his worldview until I finish reading it and his other key works, but I will say this: Rushdoony does not come off as the boogeyman of ``Christian Fascism" that was conjured up from the nightmares of a Christopher Hedges. It would seem Rushdoony would answer his accusers' charges that fascism, as opposed to his reconstructionist socio-political order, is a modern manifestation of the pagan belief of the One being primary at the expense of the many, which envisions the state as God incarnate.

It would have been interesting to see a conversation between Rushdoony and Evola. Instead, we got one with Bill Moyers. I guess we get the conversations we deserve.


\hfill

\texttt{Adam J. Franz on 2019-11-16 at 08:12 said: }

Cologero,

What is the source for Numa's sacrifice with bread and wine. Not at all surprising, just wondering if I need to leaf through Livy or something. 

This is my first post, but I love your content and have been reading it for several years.


\hfill

\texttt{Cologero on 2019-11-16 at 08:44 said: }

Try Numa bread wine sacrifice\footnote{\url{https://www.google.com/search?q=numa+bread+wine+sacrifice&oq=numa+bread+wine+sacrifice}}

The purpose was to allow the lower classes — for whom meat was a rarity — to offer sacrifices.


\end{sffamily}\end{footnotesize}
