\section{The Holy Knights Templar}

It is unnecessary to summarize here the entire history of the \textbf{Knights Templar}, which can be found easily enough online, so we will just highlight certain points that are of interest. Ultimately, our goal is to consider the option of a Fourth Order of the Templars.

The Templars conjoined the two greatest archetypes of Tradition: the ascetic and the warrior. As is well-known, the great Church Doctor \textbf{St. Bernard of Clairvaux} created the \emph{Rule for the Knights}, adapted for them from the \emph{Rule of St. Benedict}. That is available as \textit{In Praise of the New Knighthood}\footnote{\url{http://digihum.mcgill.ca/~matthew.milner/teaching/resources/docs/clairvaux_newknighthood.pdf}}.

Such an order of knights is hardly possible today, first of all because there is no desire among the hierarchy for such a venture. The other obstacle is that the means of warfare have changed so much. The various means for mass destruction have eliminated the role of the Knight. Although the ideal of the warrior ethic still exists, it has been incorporated into the special forces in the military services of secular states.

Nevertheless, the romance of the Knights Templar persists, despite or, in some cases because of, their ignominious end. The Traditionalist order of the Society of Saint Pius X\footnote{\url{https://sspx.org/en}} sells an audiolecture \textit{The Knights Templar}\footnote{\url{https://angeluspress.org/collections/church-teaching/products/knights-templar}} as well as a written history, \textit{The Templars: Knights of Christ}\footnote{\url{https://angeluspress.org/products/templars-knights-christ}}.

Hence, the last Templar grandmaster, \textbf{Jacques de Molay} ought to be a candidate for sainthood as a martyr for the faith. Instead, due to his fabricated condemnation by both Throne and Altar, Freemasonry — the enemy of the Church — has claimed him as one of their founders.

\paragraph{Ludibrium}
There are two ways to do history. The ordinary way is two-dimensional, that is, the goal is to describe the relationship of physical events over the course of time. The esoteric way is three-dimensional history which is more concerned with identifying the spiritual forces that underlie material events. Among those factors are ``archetypes" which are mythological motifs that manifest themselves repeatedly, in different forms, throughout history. The motif that we are interested in is that of esoteric and exoteric teachings. The esoteric teaching will assert itself, often in apparent opposition to the exoteric teachings. However, our goal is always to unite them.

Following the breakup of the Templars, some of them were absorbed into the \textbf{Order of Christ} in Portugal, and others returned to the East where they were first formed. The story is that the Templars reformed as the Rosicrucians who emerged publicly a couple of centuries later. \textbf{Johannes Valentinus Andreae} published the \emph{Chymical Wedding of Christian Rosenkreutz}, one of the manifestos of the Rosicrucians. He referred to that document as a \emph{ludibrium}, that is, play. This brings to mind Valentin Tomberg's ideal of turning work into play. Secular historians will only be satisfied with material connections between events, but the third dimension will look for the reappearance of related ideas. The obvious clue is that a ``secret" society will not be publishing manifestos.

\paragraph{The Knights Kadosh}
\textbf{Albert Pike} in the chapter Knight Kadosh\footnote{\url{http://www.sacred-texts.com/mas/md/md31.htm}} from \emph{Morals and Dogma} adds to the mythical history of the Templars. He asserts that the first Knights took their vows before the Patriarch of Constantinople. That is incorrect; it was the Patriarch of Jerusalem. He then claims that the Templars had both an esoteric teaching, the Church of John, and and an exoteric teaching, the Church of Rome. He implies that the exoteric teaching a smokescreen to hide the true teaching. However, there cannot be any conflict between the esoteric and exoteric teachings. After all, the Templars were quite devout in their exoteric practice.

Nevertheless, the destruction of the Templars did have consequences. Pike claims that a remnant of the Templars continued under the auspices of Freemasonry. Their goal was to ``to overturn the Throne and the Altar upon the Tomb of Jacques de Molay". Having ended the French monarchy, they directed their efforts against the Pope. At the time that Albert Pike was writing, the second part of the task was incomplete. However, were he alive today, he may be less sanguine about the ultimate success of the revolution.

\paragraph{A Dilemma Resolved}
Pike implies that the revolution was a Pyrrhic victory, since the successor of the Templars perished, i.e., there is no longer a direct — or real — relationship to them. Therefore, we turn instead to the ideal relation\footnote{\url{http://www.gornahoor.net/?p=8282\#relations}} and ask the same question. Pike makes the distinction between the Johannine and the Petrine current in the Templars. By rejecting the Petrine, he implies that the Rosicrucians and Freemasons are the heirs of the Johannine current. However, there has been revealed a more complete way to regard those currents.

The Russian esoterist \textbf{Valentin Tomberg} met with some students who had studied under Professor \textbf{Gregory Mebes}. In that group there were three grades: Martinist, Templar, and Rosicrucian. Certainly, then, Tomberg was immersed in the Johannine current. Nevertheless, he came to the realization that any doctrine of ``two churches" was false. The Church of John was not a rival to the Church of Peter. Rather, their relationship was described this way:

\begin{quotex}
The mission of John is to keep the life and soul of the Church \emph{alive} until the Second Coming of the Lord. This is why John has never claimed and never will claim the office of directing the body of the Church. He \emph{vivifies} this body, but he does not direct its actions. 

\end{quotex}
Hence, the spiritual current of the Templars belongs with Rome, although only the SSPX seems willing to revivify that connection.

\paragraph{A Fourth Order}
A latter-day Templar revival would not be a first, second, or third order, since it is a lay initiative, but a fourth order. The masonic temptation is to build, and this needs to be avoided. Hence, that would rule out much of what is called an ``intentional community". The alternative is a community that grows organically. Instead of a top-down model, that more traditional idea of subsidiarity\footnote{\url{https://gornahoor.net/?p=2554}} is appropriate. Hence families would develop into phratries, and then tribes, in a natural way.

The Knights Templar was restricted to those already knighted. Moreover, each knight turned over his assets to the community. That would not be appropriate for a lay group. A family corporation\footnote{\url{https://info.legalzoom.com/establish-family-corporation-21136.html}} of some type would be fairer, and it would make it easier to leave without causing legal problems.

\paragraph{Entryism}
An Order needs to have a spiritual centre. One viable alternative is to affiliate to an SSPX chapel. Not only would there be a friendly atmosphere for medieval values, but a relatively small group would be able to exert influence. I will offer some specific suggestions in the mailing list.

The Templars survived for two centuries, and their demise was not the result of any internal tension. That is because they inserted themselves into a larger organisation, which granted legitimacy, cohesion, and spiritual support. Self-existing alternative movements today typically claim a factitious membership based either on a non-existent genetic similarity, or else the adherence to some small set of propositions. The first alternative cannot settle legitimate and important differences of opinion, and the second alternative is no different from the modern world.

Building an organisation from scratch outside of existing institutions is a daunting task. Even in democratic situations, the formation of new political parties – while theoretically feasible – has been totally ineffective in practice.

\paragraph{Distinctiveness}
There need to be outward distinctions, while not appearing to be overly eccentric. Obviously, a certain modesty in dress and the avoidance of much of popular culture would be mandatory. In another post, we will adapt the Rule proposed by St. Bernard of Clairvaux to make specific recommendations for lay groups. Of course, there will need to be some latitude for local groups to determine some of their own customs.



\flrightit{Posted on 2017-07-05 by Cologero }

\begin{center}* * *\end{center}

\begin{footnotesize}\begin{sffamily}



\texttt{James on 2017-07-05 at 02:26 said: }

Thanks for the great post.Looking forward to the next post on the subject.


\hfill

\texttt{Aleksandar on 2017-07-05 at 03:42 said: }

"The Templars conjoined the two greatest archetypes of Tradition: the ascetic and the warrior."

I've been fascinated by this topic for quite some time. This is much needed today. A purpose for a man of traditional values, physically and mentally trained and disciplined individual, perhaps skilled in martial arts, or at least in tune with the heroic pathos, who at the same time avoids, much as possible, modern day degenerate influences designed to numb down, absorb and dissolve. What is left for him in today's world? One who assimilates two paths which are today artificially divided. Unfortunately, at the gym I mostly witness brutes and meatheads. Same thing with others who indulge in a more physical aspect of life. Who says that you cannot do all of that and same time cultivate your mind – not in a pseudo intellectual way, of course, working on your "brain sharpness" with self help books and pseudo yoga manuals, but through active involvement in the entertainment of higher ideas based on the true traditional knowledge?

Knight is now reduced to a mere expendable, mercenary, a cog in the wheel without a Divine guidance and higher purpose except serving the secular order. Modern day ascetics (not counting monks) are in some way NEETS, hiding away in the shadows but devoid of that Divine impulse to overcome their nihilistic state.

This is why I'm also interested in the phenomena of another archetype which reemerged in last few years (or decades) – a paladin. Similar to a Templar, a popular RPG class played by thousands, if not millions of kids (and adults) across the world in different fantasy video games. Paladin is based on Charlemagne's Twelve Peers, offering an image of ideal, heroic knight whose strength comes through religious devotion and asceticism. Sadly, many of these life options are now limited to a mere fantasy game, a role play. An impulse is certainly there, as evidenced in a certain rise of popularity of medieval knights, military orders and history through the online communities (ignited by the current events in the West, no doubt).


\hfill

\texttt{Fred on 2017-07-05 at 06:59 said: }

Very good article, I will mention two things that can inspire a reflection.

In the Parsifal of Wolfram von Eschenbach, without a doubt an initiatic poem, the guardians of the Graal are called Templars and have white capes. Eschenbach himself was a knight.

About the remarks of Albert Pike I really want to point to the fact that the direction of the subversive project of modernity is usually closely tied to degenerated initiatic groups. In fact every error of modernity is a perversion of a traditional principle. For example liberty, equality and fraternity. It is explicitly evident in the pantheism of Spinoza and Giordano Bruno, clearly arised from their own misinterpretations of hermetic and kabbalistic texts. So, freemasonry was the pretorian guard of tha anti-traditonal englithenment, as natural for a subverted esotericism. The same can be said for jews.


\hfill

\texttt{willehalminstitute on 2017-07-05 at 12:18 said: }

@Fred: The ``Templeisen" as they are called in Middel High German in Wolfram's Parzival cannot be the same as the Templars, at most their forerunners, because the Grail events described in Parzival took place in the 9th century, while the Templars only surfaced some two centuries later. I am basing myself on the research by Werner Greub in his book ``How the Grail Sites Were Found – Wolfram von Eschenbach as a Historian" (see Grail Sites at \url{http://www.willehalm.nl}) that I translated and published.


\hfill

\texttt{THAT on 2017-07-05 at 18:27 said: }

It is a commendable initiative, but the fact that we know so little about the true nature of the secret doctrines and initiations of the original Templars makes the appropriation of the Templar identity by modern groups problematic. It will be based merely on the exoteric Templar image; anything more than that would be speculative. 

And if entering into speculation, it is not outside the realm of possibility that the Templars did in fact have secret doctrines and practices, at least for the elite of the order, that were heretical and not merely a harmonious deepening of Catholic Christianity. Evola seemed to believe that much in his `Mystery of the Grail'. The fact that they were exoterically devout Catholics proves only that they didn't really have any other choice than to pay lip-service to it.

De Molay and the Templars took high interest on loans in their role as bankers, and this at least doesn't add credit to the sainthood account. 

What speaks strongly in favour of the Templars is of course Dante's support for them. Charles Upton, in the chapter ``The Templars, the Freemasons and the Counter-Initiation" in `Vectors of the Counter-Initiation'. suggests that there may have been within the Templar complex both a stream of corrupted esoterism and a valid Christian current. The Templar-connected stream inverting the resources of their initiatory current for the sake of worldly power would have gone underground and seeded various conspiracies. Much of this is speculation that cannot be proved, but the most vital resources and secrets of the Templars were saved from the persecutors, and I find it hard to believe that the underground stream seeded from this wealth of initiatory empowerment was ever discontinued. I tend to believe this lineage may still preserved by certain privileged and secretive elitist circles, but perhaps in a corrupted form.


\hfill

\texttt{Fred on 2017-07-05 at 19:46 said: }

While the role of freemasonry as a counter-initiatic source demand that we at least acknowledge the possibility of counter-initiation and errors among the Templars there are several factors that witness in favour of the Knights.

First of all Dante and St.Bernard, then the fact that they were killed by King Philip IV, the man who did more than anyone else to undermine the two universal authorities of the middle age: the Pope he submitted and brought to Avignon, and the Emperor (France was the first of the great kingdom of Europe to declare the authority of his King as the highest and rejecting any nominal submission to the Emperor).

While a French puppet the Pope even intended to pardon the Templars, as shown in the Chinon parchment.

They were killed by the King who was the living embodiment of the ksatriya revolt against the traditional order of the medieval Christendom/Europe.


\hfill

\texttt{THAT on 2017-07-06 at 09:59 said: }

St. Bernard marks the beginning of Templar history. I do not think there is any reason to assume they would have been corrupt from the beginning; so the Templar ideal as partly forged by St. Bernard would be valid no matter what eventually developed at the expense of its original purity within the shadows of the Order. 

One subject for much speculation has been to what extent they were influenced and changed in their inner orientation and agenda by finds (some of which may have accelerated their rise to massive power; did this power go to their heads?) that they made in the East and by their contacts with various gnostic groups. The Order of Assassins, for example, was counter-initiatic in the extreme.

I am aware that the Templars were pardoned, but this could just as well be interpreted as a political move not necessarily based on objective truth, aimed at keeping the exoteric romantic legacy of Templarism as part of Catholic mythology; once the Templar organisation no longer represented a practical problem, one could conveniently forget about the shadowy esoteric affairs and honour the aspects of Templar history deemed good. I am not saying this has to be the right explanation, but it is a reasonable possibility. 

What you say make for a good case, but the fact that corrupt forces are involved in a persecution does not guarantee that the persecuted is not corrupt as well. In all probability the case of the Templars is far more complicated than is allowed for by the one-sidedness often seen both in their accusers and apologists.


\hfill

\texttt{Cologero on 2017-07-07 at 22:58 said: }

Interesting comments. I'll try to integrate them into another post, but here are some loose ideas in the meantime.

When I was a boy, one of my favorite TV shows was about a Paladin: Have Gun, Will Travel\footnote{\url{https://www.youtube.com/watch?v=HAoEOhsGJyU}}. The archetype is surely alive doay, although hardly manifestable in the modern world, maybe like Junger's ideal of the Anarch. Unfortunately, it seems that in today's pop culture, that ideal has deteriorated to a wise-cracking wise guy, nearly indistinguishable from a sociopath.

The widespread image of Jacques de Molay depicts him with long hair and wearing a long flowing robe. St. Bernard specifically rejected that image of the Knight, and insisted the Templars keep their hair short. The lesson: be wary of any historical reconstructions of the Knights.

As for their sanctity, Bernard referred to the Knights as ``holy martyrs". The contemporary witnesses to Molay's execution regarded him as a martyr, and collected relics. So the idea of sainthood is not so far fetched. Usury is more subtle. Perhaps you can categorize the Templars as usurers, if you regard their fees a mere ``loopholes". The argument contra usury, it seems, is based on the misconception that money is constant. If I borrow a cup of sugar, I can return a cup of sugar in 6 months. I can bake a pie now and you can bake one using the same recipe in 6 months. But if I borrow \$5 to buy a bag of sugar now, there is no guarantee that you will be able to buy a bag of sugar when I return the \$5 in 6 months.

I think there is a misunderstanding about the esoteric/exoteric distinction. Without using those specific terms, Bernard alludes to the distinction several times in his ``In Praise of the New Knighthood".E.g., he wrote ``we must not let this literal interpretation blind us to the spiritual meaning of the texts [i.e., scriptures]". He mentioned how the commoners would receive the ``milk" of doctrine, while the Knights would understand the ``meat". The latter would include the recognition of the divinity in man and the spiritual sense hidden beneath the written word. The point is that the esoteric and exoteric are intimately related, even if the former is reserved for the few. Fundamentally, the esoteric is not some new ``secret" doctrine, but rather a much deeper understanding of the exoteric doctrine. Anything else is subversion.

Please consider the distinction made between real and ideal relationships. We are not looking for an ongoing material secret transmission [``real"] of the Knights, but rather the reappearance in space-time of transcendent archetypes [``ideal"].


\hfill

\texttt{THAT on 2017-07-08 at 08:15 said: }

I concur with your understanding of the exoteric-esoteric spectrum. Obviously I didn't mean to imply that `esoterism' amongst the Templars must naturally and inevitably l have been some heretical `secret doctrine' and so on, only that it is quite possible that the Order might at some point have come to secretly house such subversive elements, that would have inverted the original esoteric intention to some degree.

The ``meat of doctrine" to which you refer was most clearly expressed in the case of Meister Eckhart, and that didn't go so well for him; this is the greatest weakness of Catholicism, as it leads exactly to the formation of ``secret doctrines" that may become subversive, when the church doesn't allow for a smooth and continuous transition to open esoterism that is accepted as a deepening of exoteric doctrine.

I am not so knowledgeable about the question of usury, but you—and I apologize if I misunderstand you here—seem to be making too lightly of it; both Christianity and Islam have taken a rigorously strict approach to the question from the beginning, and this is because usury is a means that belongs to the Enemy. Any organisation touching upon it makes a compromise with him.

Speaking of the Enemy—and relating to your mentioning of entryism—it would be valuable to consider how this `order-to-be' might guard itself against the infiltration of agents of subversion, of whom there is a never-ending supply these days. 

I understand what you explain about `real' and `ideal' relationships, and that the ideal and archetypal relationship is intended in the case of this project; I only expressed the suspicion that an actual uninterrupted Templar transmission of some kind may still be running underground as part of ``occult" elite operations. The possibility must be kept in the back of the mind as such forces might take special notice of any rising contemporary `movement' connecting with the Templar archetype.


\hfill

\texttt{Fred on 2017-07-08 at 15:40 said: }

I always felt the ``meat of doctrine" much more in line with Catholicism in the allegorical esotericism of Amor than in Eckhart.

Maybe this is because I'm italian but I see Dante, the Roman de la Rose and chivalric novels more closer to St.Bernard. As a cultural expression it feel much closer.

Eckhart was a german spirit, his emphasis on detachment feel almost buddhist to me. His problems with the inquisition were caused by the words he used: alien words to the culture of Latin Christianity.


\hfill

\texttt{Old Chap on 2017-07-09 at 12:32 said: }

The downfall of the Templars shows us that such Orders and their slavish adherence to certain values is foolish in a world and a time where authentic, not superficial morality is a rarity. Since we live in a world without morals and just laws (where they exist on paper they are not followed in reality), all of these old sacred designs are simply impossible on any substantial scale. The fault lies with human nature which is markedly flawed, rather than a nonexistent elite which will never appear (still waiting for Christ's return and all the other prophesied saviors of mankind). Faith alone does not supersede reality no matter how much one wishes otherwise. Chivalry is dead because there is no longer a use for it, as it would be an undeserved relic for a people more concerned with gossip and ``winning" and consumerism and the endless array of meaningless ``activities" and entertainments. (Just my two pence.)


\hfill

\texttt{Harharkh on 2017-07-10 at 14:55 said: }

You live in fantasy land, old man.


\hfill

\texttt{Cologero on 2017-07-10 at 21:35 said: }

Yes, grasshopper, and when you realize the illusory nature of your own existence, then you, too, may become enlightened.


\hfill


\end{sffamily}\end{footnotesize}
