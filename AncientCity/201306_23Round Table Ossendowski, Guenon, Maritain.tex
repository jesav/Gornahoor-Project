\section{Round Table: Ossendowski, Guenon, Maritain}

These are the notes of a round table held in 1924 in Paris and hosted by Frederic Lefevre. What follows are his notes from that meeting published in the 26 July 1924 issue of ``Les nouvelles littéraires". The original text in French may be found here: \textit{Le Salut qui veint de l'Orient?}\footnote{\url{http://esprit-universel.over-blog.com/table-ronde-1924-ren\%C3\%A9-gu\%C3\%A9non-ferdinand-ossendowski-et-jacques-maritain}}. 

\paragraph{Does Salvation come from the East?}
The arrival in Paris of Mr. \textbf{Ferdinand Ossendowski}, who has been called the Robinson Crusoe of the twentieth century, could not go unnoticed; all critics are unanimous in considering Beasts, Men and Gods, his first book translated into French, as a prodigious book, the most exciting travelogue they ever read. We had the good fortune to talk with Ossendowski, the man who saw the living Buddha, and we had the no less valued opportunity to meet with the three French personalities who seem best designated to compare their doctrines and science to his experiences, and to judge, in the light of Western concepts, the incredible number of observations he reported about his dramatic trip through Asia. I selected the historian of Asia, \textbf{Rene Grousset}, whom we presented to our readers a few weeks ago, \textbf{Jacques Maritain}, the Catholic philosopher, to whom the Thomistic revival in this country is largely due, and \textbf{Rene Guenon}, the Hinduist, whose meditations have earned us this remarkable book, Introduction to Hindu Doctrines, a nice summary, full of substance, both East and West, and who publishes these days on the questions which trouble the European consciousness.

Ferdinand Ossendowski was first of all asked about the political question like all those who were victims of the Russian Revolution. He gave us some interesting reports on Russia and the yellow world. He took the trouble to answer all our questions. The Catholic philosopher Jacques Maritain was neither the least curious nor the least impatient. Let us try, at the mercy of our memory, to reconstruct the back and forth.

\textbf{J. Maritain:} Isn't there an alliance between the Asiatic world and the Soviets?

\textbf{Ossendowski:} In order to answer that question it is necessary to go back into history. The last descendant of Genghis Khan, Amoursang Khan, who was also the emperor of China, took refuge in Russia when the Ming dynasty triumphed over the Yuan dynasty. As a pledge of gratitude, he transferred all his rights as supreme Khan to the Empress Catherine II. That was, you know, in the 18th century. He also gave her the holy stone of prophecy, a type from black agate which was all covered with lichen. As long as it remained at Ourga, sickness or unhappiness would never touch the Mongols, nor their animals. Since they left Asia, the Mongol people slowly began to die.

\textbf{Lefèvre:} Did the czars use the title of supreme Khan and did the holy stone of prophecy grant all its power in Russia?

\textbf{Ossendowski:} Catherine II did not use that title, but Alexander I, aware of the duties that it imposed on him, helped the Mongols diplomatically in their revolts against China and since then, the Mongols and Chinese, follow the titles of czar with ``white Khan" (Tzagan-Khan). Also, when Semionov, general of the army of Koltchak, became viceroy of the Far East, he demanded the title of great Khan of Mongolia from the living Buddha of Ourga. He got it. Baron Ungern was no stranger to such a rapid acquiescence.

\textbf{Lefèvre:} Is Semionov still the great Khan of Mongolia?

\textbf{Ossendowski:} He still is, but no longer exercises an effective power. He is now living in Nagasaki, Japan. He fought against the Bolsheviks in the district of Oussouri on the Pacific coast when he separated from Ungern. Effective power then returned to Ungern whose first move was to have the living Buddha address a bull to all the Asiatic peoples in which he ordered the battle against the ``Bolsheviks, evil devils who are going to kill the morals and the soul of all humanity." Semionov had moreover been contacted by the Dali Lama who had sent a message to the living Buddha ordering him since 1921 to undertake the battle for the defense of humanity.

It is then that the living Buddha had begun the war, Baron Ungern being his generalissimo. They called Ungern the God of the war, and I saw, in his army, representative of all the Asiatic tribes. They fought first against China for the autonomy of Mongolia. They were victorious. Once free, Mongolia proposed to China to take the leadership of this panasiatic movement against the Bolsheviks first and the white race afterwards. But China is in such a state of anarchy that she could not accept. She is totally divided into small kingdoms which are in the hands of adventurer generals. Before her refusal, Mongolia took the command of its allies and undertook a battle hardly crowned with success. Ungern was betrayed by his officers, delivered to the Bolsheviks forces in Transbaikal, and killed. Since then, it is the Soviets who have claimed the rights of the czars over Asia. For five years, there have been in Petrograd eleven panasiatic congresses, and the university of propaganda of Moscow, where there are two hundred thousand students, counting forty thousand Asian students (the most numerous are the Indians from British India, then the Chinese, the Persians, and Turks). The Bolshevik propaganda is moreover very skilled. Communism is not for it a doctrine of export, as the Orient repudiates it absolutely. It presents Russia as the avant-garde of the yellow world ready to swoop down on the white race. But the living Buddha and his wife, and especially his wife, created an obstacle to that propaganda. The Bolsheviks did not hesitate; they poisoned the Buddha's wife about two years ago. Unable to fight openly against the Bolsheviks, the living Buddha limited himself to maintain contact with all the people in order to form, at the right time, a great central Asiatic kingdom. China's monarchist party is in a close relationship with the living Buddha and, if he should triumph, it would force China to join that anti-Bolshevik and anti-white movement.

\textbf{Lefèvre:} Does that mean that you do not believe that the Bolsheviks are capable of organizing a panasiatic movement?

\textbf{Ossendowski:} I don't believe so. They do not base themselves only on purely Asiatic elements.

\textbf{Maritain:} Are there relations between the living Buddha of Ourga and Gandhi?

\textbf{Ossendowski:} Yes, they are in a relation, but if a will to fight against the white race unites them, their systems are different. The living Buddha is a warrior and Gandhi is a pacifist. The movement he created is not yet a warrior movement.

\textbf{Lefèvre:} Do you know Rabindranath Tagore? Isn't he critical of that battle against Western civilization as extreme? In three very significant messages that Cecile Geroge-Bazile just translated into French under the title, Nationalism, he raises some blunt declarations: ``The West is necessary to the East. We are complementary to each other by our different aspects of truth, which is why, if it is true that the spirit of the West has fallen in our countries like a storm, it sows nevertheless here and there some living seeds which are immortal. And when in India we become capable of assimilating what is permanent in Western civilization, we will be in position to bring a reconciliation of these two great worlds."

Tagore is an intelligence nourished by all cultures of the world and besides he declared: ``I believe in the true union of the East and the West. All the glories of humanity are mine. No people can make its salvation while detaching himself from others." If there is an affectionate esteem for Gandhi, he fears the Ghandians.

\textbf{Ossendowski:} I saw Tagore in London. He is an Asiatic. In his black eyes, you can see nothing. When I look into those eyes, I always had the impression of a screen that is necessary to lift up.

\textbf{Guenon:} Tagore is a very Westernized Asian.

\textbf{Ossendowski:} Don't rely on that too much;

\textbf{Maritain:} Did you encounter in Mongolia any traces of Catholic evangelization?

\textbf{Ossendowski:} No. A few Franciscans, but without any importance.

\textbf{Guenon:} However, the Catholic missionaries alone could get a hearing with the Buddhist soul. Unfortunately, they all commit, in my opinion, the great error of addressing themselves only to the pariahs, the non-cultivated castes, and they were despised by that fact. They thus strictly limit themselves in the field of their influence since they neglect everything that constitutes the intellectual vitality of the oriental world.

\textbf{Grousset:} Intellectual vitality, intellectual vitality! But there is no philosophical activity in Mongolian Buddhism.

\textbf{Guenon:} What do you know about it? Don't you know that true wisdom is silence? The virtue of the Buddha is something entirely interior.

\textbf{Lefèvre:} But are you speaking to us, Mr. Ossendowski, of the living Buddha, whom you saw?

\textbf{Ossendowski:} I saw the one from Ourga. There are two others.

\textbf{Lefèvre:} That is the trinity of living Buddhas?

\textbf{Ossendowski:} Exactly, and they each have some very distinct attributes. The Dalai Lama, who resides in Lhassa, Tibet, is like the incarnation, or better said, the realization of the hoiiness of the Buddha. The Lama of Tashilhunpo who lives two hundred kilometers from Lhassa, realizes the wisdom and the science of Buddha. The third, the one whose palace I was in at Ourga in Mongolia, represents the material and warrior force of Buddha.

\textbf{Lefèvre:} I feel moved; I will still keep a more attentive ear. I press the question of the blessed contemplator of the holiness of the living Buddha. Well, is he truly more than a man? What impression did you get during your first interview. Formidable, wasn't he.

\textbf{Ossendowski:} Yes; unfortunately, he is an old drunkard.

\textbf{Lefèvre:} I am aghast.

\textbf{Guenon and Ossendowski together:} That has no importance.

\textbf{Ossendowski:} The power is with the Russian merchants who forced him to drink in order to exploit him better. He lost his sight. The personality of the living Buddha presents the same duality that one finds again in all Lamaism. When he became blind, the Lamas fell into the deepest despair. Some of them were sure that it was necessary to poison him and put in his place another incarnated Buddha; the others valued the great merits of the pontiff in the eyes of the Mongolians and of the faithful of the yellow religion. They finally decided to build a great temple with a gigantic statue of Buddha, in order to appease the gods. This however did not succeed in bringing back the sight of the Buddha, but gave him the chance to hasten his departure for the other world of those from among the Lamas who gave proof of an excessive radicalism, as to the method proper to resolve the problem of his blindness.

\textbf{Lefèvre:} Mr. Ossendowski's statements surprise us. We poorly explain to ourselves the strange morality of the living Buddha.

\textbf{Guenon interrupts:} Don't judge those things with your Western categories. What you call virtue is for Hindu wisdom something exterior and quite incidental.

\textbf{Maritain:} It is because of this that the living Buddha doesn't disregard the help of his brotherhood of poisoner lamas.

\textbf{Ossendowski:} These are the doctors in political medicine. Coming back to the living Buddha, he is not drunk every day and apart from that defect, he appeared to me to be a very superior man. At certain moments, I sensed something pass in him like a true inspiration. He was 64 years old. He is the son of one of the Dalai Lama's squires. He himself has a son who the intrigues of Japan tried to circumvent.

\textbf{Maritain:} What do you think of the mystery of the King of the World, leader of a subterranean humanity with marvelous science and power, whom you regard in your book as the central animating mystery of Mongolian hope?

\textbf{Ossendowski:} I suppose that this legend has a political origin. No nation in Asia being strong enough to sustain temporally the imperialism of the yellow religion, that function was devolved to a subterranean humanity and to its leader. And so the hopes of the Asians had the point of necessary support… while waiting for the new Genghis Khan.

\textbf{Guenon:} The idea of the King of the World dates back In Asia to high antiquity, and it always had an important role in the Hindu and Shaivite tradition which form the base of Tibetan Buddhism.

\textbf{Maritain:} For us, in any case, this name evokes different assonances, ``the prince of this world has been" judged," say the Gospels.

\textbf{Lefèvre:} Of all the strange phenomena which you witnessed, Mr. Ossendowski, and that you relate in your book, is there anything in it that appears inexplicable?

\textbf{Ossendowski:} I have to say that I spent several months in an excruciating solitude, the nerves strained at each instant in a constant battle for life. I was ripe for all suggestions and even auto-suggestions.

\textbf{Maritain:} However regarding the other things that you had the same vision, of your family which was far away and certain predictions made came true. Prediction is not of the domain of suggestion.

\textbf{Ossendowski:} I am a traveler and observer. Therefore, I am leaving soon for North Africa. I am going to visit Morocco (I have a letter of recommendation from Marshal Lyautey), Algeria, Tunisia, Egypt. Next year I will push onto Central Africa. I have the feeling that I will report some things about it. I come in sympathy with the peoples, with the land itself. I have the romanticism of the land.

\textbf{Maritain:} Did you record any teachings given by the Lamas?

\textbf{Ossendowski:} The Lamas are very intelligent and cultivated, but they keep the people in credulity and superstition. At all the bends in the road, it is necessary to offer sacrifices to the evil spirits.

… with a melancholy smile: I myself offered numerous sacrifices. The people live a type of crude pantheism. Everywhere I saw them bent under fear.

\textbf{Maritain:} And it is these Lama teachers who want to bring to Europe the kingdom of the spirit … what strikes me is that in place of these distinctions between the different orders that can be regarded as one of the most precious acquisitions of western civilization, and as a condition of human freedom, one notices over there a universal confusion between the spiritual and the temporal, between mysticism and politics, between the interior hierarchy of holiness and the hierarchy of spiritual government.

\textbf{Guenon:} But there is also over there a deep wisdom that the West does not know how to become aware of.

\textbf{Maritain:} I am quite far from denying it. But in what mixtures does it take place? And what spirit does it come under? It is the task of Catholic theologians to discern it. When will they decide to study that question in the light of their principles? That is urgent. However, Mr. Ossendowski, one thing astonishes me in your book: doesn't it seem, after what you bring back from there, that this wisdom is turned above all on the side of the administration of created things?

\textbf{Ossendowski:} That is true without doubt from what I saw at Ourga. Don't forget however that at the side of the living Buddha of Ourga, and above him, there is the Panchen Lama and the Dalai Lama who turn away from the things of the world, and are absorbed in a pure contemplation.

\textbf{Maritain:} In that the yellow religion remains faithful to one of the deepest truths of the spiritual order. And certainly, they are correct to reproach our civilization for its materialism, and its dissipation in outer activity. If Europe is in its torment, it is because she has failed in her mission. But that is not from them that she must receive initiation into the things of the spirit. It suffices for her to come back to her most authentic tradition which, with a better claim that the oriental tradition, affirms the preeminence of wisdom and contemplation.

\textbf{Ossendowski:} The Asians think that the war of Asia against Europe is inevitable and holy.

\textbf{Grousset:} But Japan, so hungry for material progress and passionately turned towards western civilization, will not follow them on this path.

\textbf{Ossendowski:} the Japanese are now regarded as the renegades of Asia.

\textbf{Maritain:} It is not only force that must be opposed to force, but also spirit to the spirit.

\textbf{Guenon:} Why do you speak of an opposition? It is rather necessary to say alliance and agreement.

\textbf{Maritain:} There is no alliance possible outside of truth.

\textbf{Guenon:} Such is certainly my thought. But the Orient brings us a truth which can agree with the truths of the highest western traditions: the Aristotelian and Catholic traditions.

\textbf{Maritain:} Aristotle's metaphysics will never agree with a thought that it is necessary, so cleverly that you defend it, to call pantheist and which, in wanting to go beyond being, can only smash reason.

\textbf{Guenon:} The word pantheism is a western word which should not be applied to Hindu speculation. There is nothing in common between this and what we call pantheism, nor what we call idealism.

\textbf{Maritain:} As to the Catholic religion, the alliance in question will be for her only an inadmissible subordination and the ruin of the distinction between the natural and supernatural orders, between nature and grace. Theology, supported on the revealed principles of faith, is the supreme science.

\textbf{Guenon:} No, it is only a determinate of metaphysics, I speak of the true and authentic metaphysical wisdom. This goes well beyond.

\textbf{Maritain:} No science goes beyond revealed faith. Moreover, does Hindu wisdom know the complete way, not only the order of morality properly called, what we call merit, sin, etc., but also the order of charity?

\textbf{Ossendowski:} The Mongolian people are honest, peaceful, and deeply worthy; they practice hospitality. But there is in effect no place in eastern religion for charity in the sense of the love of God

\textbf{Guenon:} That is a sentimental element and consequently secondary.

\textbf{Maritain:} Come on then! It is a fully spiritual and fully supernatural virtue: ``God is Love." It is by that alone that man reaches perfection; it is by that also and by the gift of wisdom which is inseparable from it, that true contemplation has its place. It is by that alone that the Spirit can rule among men. That is the main point on which no agreement is possible between absolute intellectualism and Hindu esoterism.

\textbf{Lefèvre:} Is it necessary then, in your opinion Mr. Maritain, to reject en bloc all oriental thought?

\textbf{Maritain:} No way. There are precious and very high truths to gather from them, which keeping oneself from all injustice and all vicious bias, and in evading (on this point I am in agreement with Mr. Guenon), applying to it deficient methods that the critical rationalist applies to Christianity in the West. On condition notwithstanding of subjugating everything by an intelligence faithful to sacred truths which are our heritage! One can ask oneself if the Greco-Latin culture, where the salvation of reason is, is not destined to lose soon its privilege in fact, to cease to be the only form of intelligence, as culture becomes truly worldwide.

\textbf{Lefèvre:} Evidently we cannot prevent Tagore's books, for example, from being translated into all the languages and the oriental conception of life from belonging to cultivated readers in the entire world.

\textbf{Grousset:} The Anglo-Saxons have understood for a long time that this compenetration was inevitable and that it was useless to oppose it.

\textbf{Maritain:} No protectionist barrier is in effect possible for the products of the spirit. That ``expansion of culture" will be for human intelligence a tremendous ordeal. All the more reason to study the Orient with attention and sympathy, but while at the same time without bending the Hellenic, Latin, and Catholic deposit.

\textbf{Ossendowski:} I repeat it to you, I am only an impartial observer, but I will not hide from you that I dream sometimes with anxiety at what will happen if the entire peoples of color, religion, different tribes begin to emigrate into the West. Would that be the last march of the Mongolians?



\flrightit{Posted on 2013-06-23 by Cologero }

\begin{center}* * *\end{center}

\begin{footnotesize}\begin{sffamily}



\texttt{Avery on 2013-06-23 at 12:04 said: }

This should be blatantly obvious, but just in case: this entire conversation is colored heavily by Lothrop Stoddard.

Amarsanaa Khan was the last ruler of the Zunghar Empire which received the title Khan from the Dalai Lama. He sought asylum with the Russians in 1757 and died in Tobolsk (no word on magical black rocks in the historical record). A common legend among the Mongols, current in the 1920s, stated that Amarsanaa would one day return from the world beyond to free them from the Chinese. When Ossendowski was in Mongolia, a self-proclaimed reincarnation of Amarsanaa called Ja Lama was fighting the Communists. No joke. 

The ``white Khan" thing is true, as is Semionov's flight to Nagasaki. Information about the Bloody Baron's reign, of course, cannot be confirmed.

Ossendowski: ``The Dalai Lama, who resides in Lhassa, Tibet, is like the incarnation, or better said, the realization of the holiness of the Buddha. The Lama of Tashilhunpo who lives two hundred kilometers from Lhassa, realizes the wisdom and the science of Buddha. The third, the one whose palace I was in at Ourga in Mongolia, represents the material and warrior force of Buddha."

This offhand statement, which does not correspond to anything in Beasts, Men, and Gods, is quoted and analyzed by Guenon on page 25 of The King of the World (Sophia Perennis edition).

The exchange between Guenon and Maritain is excellent. We are surprised that The King of the World was the most lasting product of the discussion when Guenon was its least constructive contributor. He comes off as simpering and lacking any command of the discussion, especially when Maritain gives him a tongue-lashing about the essence of Christianity. But his insistence on avoiding ``sentimentalism" is in fact a fine understanding of Eastern doctrine. 

Finally, it is amusing to see Tagore's poetry talked about with such fear, when much more obvious sources of art and literature like the entire tradition of Far Eastern painting, haiku, and the Pali Canon have had a far greater impact on the West.


\hfill

\texttt{Ash on 2013-06-23 at 16:07 said: }

One finds Ossendowski's description of the way in which the Lamas and their disciples ruled in the Mongolian and Tibetan lands to be one begging for examination. Guenon's dismissive attitude towards the character of their rule (keeping the populace in fear, although that may of course be Ossendowsky reading his own prejudices into things) when they themselves apparently have corruptions to address.When the Brahminic caste falls into corruption, is it the duty of the other castes to see that the corrupting individuals are replaced? Is it still rebellion when those of a caste acting properly move against a caste which has become corrupt in this way? Or is there no way to avoid the cycle of decay? Perhaps we must try even when it is hopeless, like the ``Roman guard at Pompeii."

Although Tagore is usually viewed as a more ``humanist" writer in the Bengal and broader Indian tradition, he does seem to have the more ``Traditional" point of view here regarding the relationship between ``East" and ``West", at least with regards to the manifestations of Tradition in each. The idea that either the East will take the West or that some sort of war is inevitable may serve well as a warning but is dangerous when it leads to fatalism and barriers which need not exist between spiritual friends. Grousette's comments on Japan are ironic indeed.

Maritain is certainly one who seems worth exploring further. To clarify, what precisely is meant by the ``Aristotelean tradition"? Greek philosophy as a whole or Aristotle and his Thomist successors specifically? It seems strange to leave out the Stoics, Plato, the mystery cults, and a legion of other non-Aristoteleans from the Western Tradition. Maritain's statement that metaphysics comes before epistemology is useful when we look at the nature of the Outer Contender discussed in the previous post. For this current, the reverse is the case. The metaphysics of Tradition are useful insofar as they lead one to accept other political ideas. This reversal of the sources of Truth becomes very apparent in writers like Dyal. The New Right has caught this fatal mistake from the New Left. To educate the willing and combat the error, it is absolutely fundamental to maintain the proper ordering which Maritain described in his works. 

Avery and Cologero, could you recommend some sources for the ``white Khan" claim? I tried looking around online but I couldn't find much to verify this apart from works not citing any primary sources.


\hfill

\texttt{Logres on 2013-06-24 at 00:33 said: }

I noticed, in reading Ouspensky, that he makes the claim that all anger is a waste of spiritual energy. However, when you read the first section of the Philokalia, you find Isaiah advocating a ``holy anger" whereby the practitioner sides with God in being angry at the seeds of sin sown in the soul, and flashes like lightning against it with the prayer of Jesus. This seems to be a fundamental disagreement, not merely in ``theory" (which for Ouspensky is not ``real" anyway) but in praxis, between East \& West. Nevertheless, I still consider that this division must, in some way, not be exhaustive, although it is telling.


\hfill

\texttt{Graham on 2013-06-25 at 23:47 said: }

Guenon's interjections were mostly amusing. `Don't judge him with your Western categories!' On the other hand, to tell the truth, I find it somewhat sinister when he says ``the Orient brings us a truth which can agree with the truths of the highest western traditions", simply because he has no intention of placing the two on an equal footing. 


\hfill

\texttt{Cologero on 2013-07-01 at 23:26 said: }

In 1931, Citroen sponsored a year long expedition (the ``Yellow Expedition") through China and Mongolia desert; Teilhard de Chardin was in that part as the official geologist. The excerpts of the report that I've seen mention the political turmoil in China, but no organized movement against the European race. There was much interaction with the Buddhist lamas, but no mention of anything like Agarttha.


\hfill

\texttt{Presbyter Iohannes on 2021-05-18 at 16:45 said: }

``The virtue of the Buddha is something entirely interior."

Quite interesting remarks by Guénon here. I though he at that time still thought Buddhism was but a hinduist heresy. In fact, it was only in 1939, already living in Cairo, that Guénon had his mind changed on the subject by Pallis and Coomaraswamy (``A Fateful Meeting of Minds" on the book ``The Essential Ananda Coomaraswamy" by Marco Pallis).

Very interesting indeed.


\end{sffamily}\end{footnotesize}
