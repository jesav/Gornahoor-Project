\section{The Science of Prehistory}

In \emph{Il mito del sangue}, \textbf{Julius Evola} has an important chapter on the theories of \textbf{Hermann Wirth}. In particular, Evola is interested in Wirth's thesis of the Polar, or Hyperborean, origin of the Nordic race, a theory that himself accepts. He begins the discussion with a quote from Oswald Menghin, the rector of the University of Vienna: 

\begin{quotex}
More than any other discipline, the science of prehistory has been brought, and even more, should be brought, to the center of the spiritual battle of our times. I don't think I'm mistaken in asserting that general prehistory will be the science that guides the next generations. 

\end{quotex}
Unfortunately, Prof. Menghin's advice is little followed today. There are a few reasons for that. The primary reason is that few people are even aware that there is a spiritual battle in progress; moreover, even fewer, probably, even understand what a spiritual battle is. 

Then there is the problem of methodology. The belief is prevalent that the only approach involves the scientific method. Perhaps this method can provide a little information, but it does not, and never could, offer any insight into the inner nature or spirit of prehistory. Obviously, that would prove useless in a spiritual battle. 

Finally, there is the question of pragmatism. Because of the emphasis on the false doctrine of evolution, few see the purpose of even looking into the prehistorical past in order to gain insight into the present. In this view, it is the future that matters, and the past something to be overcome or forgotten. 

Yet, at the time he wrote, Evola had reason to be optimistic. He continues: 

\begin{quotex}
In recent times there can be found in many circles a significant impulse to return to origins. The origins, here, appear under a special, spiritual light. He turns back to a presentiment that in primordial times he lived in a still pure state with meanings and symbols that were then lost, obfuscated or altered. Prehistoric research was brought from the level of lifeless scientific-archeological or anthropological positivism to a level of spiritual synthesis, guaranteed, thereby, to open new horizons for the true history of civilization. 

\end{quotex}
This impulse apparently ran out of gas. Instead of opening new horizons, we have three major theories of origins that limit what we know about ourselves. 

The first is the biological view, that man is a genetic accident, made ``in the image of animals", or that the Nordic race evolved out of Africa. The individual is no longer a spiritual being with a Will, but rather the resultant of various electro-chemical processes in the brain. 

Next is the return to polytheism. First of all, the purpose of the study of prehistory is not to return to the forms of the past; they are ``the past" for a good reason. Secondly, as we learned from Guenon, polytheism is a sign of decadence and any genuine primordial tradition is necessarily monotheistic. 

The worst, of course, is the negative, whiny, and ignorant rants of some who see nothing but decadence and disease, not only in the present state of Western man, but also in the past. Instead of recognizing a primordial unity, they propose a plurality of unrelated peoples and ideas. Compare that with the grand and inspiring vision —\textit{The Flowering of European Civilization} of \textbf{Donoso Cortes}\footnote{See Section \ref{sec:FloweringEuropean} in this book.}—, that unifies the greatest civilization of the West and relates them to the present.



\flrightit{Posted on 2010-06-28 by Aeneas }

\begin{center}* * *\end{center}

\begin{footnotesize}\begin{sffamily}



\texttt{Cologero on 2010-07-02 at 12:57 said: }

It is important to recognize that the rejection of the ``theory of evolution" is not the same as believing in some sort of so-called ``creation science". The point is that science is simply not competent to deal with the topic; it necessarily remains in the realm of metaphysics. ``Involution" is a metaphysical concept.


\hfill

\texttt{James O'Meara on 2010-07-09 at 10:32 said: }

Re 5:

Yes, the issue isn't `evolution' at all the rather the mechanism of `natural selection’ [the `biologism' of point one above]. Evola was perfectly happy to speak of `evolution of the soul,' for example. The materialists have cleverly gotten everyone to use `evolution' to mean simply `change' which no one, of course, denies; except for some ``5,000 year" creationists. And that's the other clever move: to label anyone who questions `natural selection' as a knuckle-dragging Bible-thumper.


\hfill

\texttt{Cologero on 2010-07-12 at 12:40 said: }

Yes, though the complete statement of the theory is ``natural selection by random variation".

It is not so much the ``natural selection" part of the theory, since every organism must be able to survive in its environment. It is the ``random variation" — which is really a non-explanation — that is in question, since everything must have its sufficient reason.

As Evola and other Hermeticists say, the sufficient reason for existence is Will. That is the point of the True Will, to manifest one's highest possibilities. Manifestation (things that happen) seems random to those who sacrifice their will to external forces.


\end{sffamily}\end{footnotesize}
