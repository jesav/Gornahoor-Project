\section{Catching our Breath}

\begin{quotex}
The fool does more harm by his ignorance than the wicked man by his wickedness. \flright{\textsc{Al-Ghazali}}

\end{quotex}

It is time for a recap what has been accomplished thus far in our project to determine the lineament of the Western Tradition, which we claim is as real and as important as anything arising from the East. First of all, we outlined the features of the religion of the Ancient City, how the city was founded by a man-god, its cults, the importance of spiritual authority and its caste based social structure. We showed its decline due to the degeneration of castes which led to the decline of the ancient pagan civilizations due its forgetfulness of its own tradition.

We turned to the next stage of Western tradition, which was both an overthrowing of its decadent predecessor as well as being in continuity\footnote{\url{https://www.gornahoor.net/?p=3281}} with it; this is called Christendom. Ignoring the rather bizarre conspiracies that abound today, we rely instead on \textbf{Rene Guenon}'s sober and intellectually sounder understanding. Christianity at its inception was a mystery religion, founded by a man-god, just as were all the Greek and Roman mysteries and cities. You can accept this claim or not, it has no bearing on what follows. As it became the dominant religion, it was compelled to become exoteric as the social structures of the Roman Empire unraveled. However, we see maintained the outward characteristics of a traditional society: the distinction between spiritual authority and temporal power, as well as a caste based structure with a strong warrior caste\footnote{\url{https://www.gornahoor.net/?p=277}} with its own set of orders. Nevertheless, Guenon insisted on the esoteric character of the dominant spiritual tradition of Christendom, providing many examples from Gospel stories to \textbf{St. Bernard} to \textbf{Dante}, inter alios.

Next, we pointed to Ananda Cooomaraswamy\footnote{\url{https://www.gornahoor.net/?p=556}}, a man whose immense intellect bridged both East and West. He listed several Western authors whose depth of thought rivals that of the East. It would be risable, if it weren't so tragic, to see well meaning men turning to the East while remaining utterly ignorant of the great Western Traditional minds. Please read that piece over and over.

Even \textbf{Julius Evola} had to concede the Traditional nature of Christendom, as much as he disliked its dominant spiritual tradition. Yet, he did make the important point that it was the result of the collaboration of Roman and Nordic elements, the result of which was the spiritual unity of all of Europe, something which did not exist prior to the Middle Ages nor after it following the various reformations and revolutions. His claim that the best aspect of the Middle Ages was its pagan elements is like saying the best part of a cup of coffee is the sugar. They cannot be separated. Furthermore, the spiritual authority defines what it is, which is why Sola Scriptura needs to be rejected.

We then did a creative re-reading of several important figures of Christendom, understanding them as exponents of an esoteric tradition. This, I believe, was very fruitful and it needs to be extended and developed by others. That brings us to the next step which is Guenon's contention that the esoteric tradition has been totally lost in the West, or else it is deeply hidden. We pose the question somewhat differently. First of all, we deny that it has been lost as Tomberg and Mouravieff demonstrate. Whatever their defects may be, they absolutely show the continuing influence of Hermetism, at least in the East of the West. If you doubt the significance of that, please go back to that Coomaraswamy piece and then look up what the ancient Christians wrote regarding Hermes Trismegistus.

The second point is somewhat different. Now, the religion of the Ancient City is certainly lost, in the very real sense that it is unrecoverable — it was lost even at the time of the Roman Empire when men no longer understood their own traditions. The revival is this paganism requires a revelation from a man-god and can never be the result of an artificial reconstruction or re-enactment. On the other hand, the metaphysical teachings of Christendom have not been lost and they even have adherents today. What we can admit, instead, is that the teachings remain virtual, not actual. It would take a gnosis (not a conceptual understanding) to make them actual. However, since they are virtual and not genuinely lost, there remains the possibility of their being actualized. Thus, any objections to self-initiation are beside the point. Anyone confirmed by someone with valid orders has this understanding virtually; under the right circumstances, it can be made effective. We have recently provided examples of the Infinite\footnote{\url{https://www.gornahoor.net/?p=3261}} and the Intellectual Soul\footnote{\url{https://www.gornahoor.net/?p=3390}} which demonstrates beyond a doubt the understanding of the Middle Ages.

Now, we are about to turn to some modern thinkers to show the way from the virtual to the actual. \textbf{Vladimir Solovyov}, who represents the East of the West, was concerned in overcoming the breach between them. As Guenon points out, there is a relationship between theological and metaphysical language. Gornahoor has never engaged in religious polemics, which are pointless. However, Solovyov has developed a metaphysical system which is compatible with his theological understanding. What Guenon hinted at, Solovyov actually accomplished. That is why it is so important to anyone envisioning the religion of the future. Even those who foresee a return to some sort of paganism have to achieve the same level of metaphysical understanding.

\flrightit{Posted on 2011-11-20 by Cologero}