\section{Hungry Ghosts}

If we don't feed our ancestors, they become hungry ghosts, tormenting, cursing, and frightening us.

\begin{quotex}
Greece and Rome appear to us in a character absolutely inimitable; nothing in modern times resembles them; nothing in the future can resemble them. \flright{\textsc{Numa Denis Fustel de Coulanges}, \textit{The Ancient City}}

\end{quotex}
In his still valuable work \textit{The Ancient City}, \textbf{Numa Denis Fustel de Coulanlges} reveals the rules by which Greece and Rome were regulated, particularly through their religious cult. There is no doubt that the ``conditions of human government" are now much different from theirs (and have changed several times in between). Fustel denies that the changes result from chance or by force as sole explanations.

\begin{quotex}
The cause which produces them must be powerful, and must be found in man himself. If the laws of human association are no longer the same as in antiquity, it is because there has been a change in man.

\end{quotex}
This change, he says, is in the ``intelligence", that is, the ways and patterns of thinking. In other words, besides genetic changes which occur slowly over time, there is a psychic evolution\footnote{\url{https://gornahoor.net/?p=1664\%22}}, a change in consciousness, which takes place more quickly. Any attempt to base society on genetics alone (race of the body), without taking into consideration the race of the soul, will certainly fail. Likewise, any attempt to ``go back" to the past, without grasping the changes in thought and forms of consciousness, will be futile.

Nevertheless, it is Gornahoor's position that there is a continuity with our past, and we cannot understand ourselves without taking this past into account. The past leaves traces in our thought\footnote{\url{https://gornahoor.net/?p=2178}}, whose effects on us are real, even if they operate unconsciously. The Hermetic method is depth, so we need to extract the genetic sources of our psychic lives by digging into these layers of thought, a process which is called ``reading the Akashic record." Usually, as Guenon and Evola do, this involves contemplating symbols and myths of bygone eras, though in this case, we also have written documents.

This is much more than an idle intellectual exercise, since the cult of the dead has been the keystone to the religion of Romanity. Fustel notes that concern for the dead has been part of Indo-European religion. The most ancient generations, long before there were philosophers, believed in a second existence after the present. They looked upon death not as a dissolution of our being, but simply as a change of life. Fustel rightly points out that they rejected reincarnation, even in the Vedas. They recognized two post-mortem states:

\begin{itemize}
\item The ``celestial abode" that was reserved to the few great men, the heroes. 
\item A shadowy underground existence attached to the body, family, and place. 
\end{itemize}
The ancient Greeks and Romans buried the dead with clothing, utensils and weapons. ``They poured wine upon his tomb to quench his thirst and placed food there to satisfy his hunger.``

It was important to bury the dead so they would they would have a dwelling place. The soul with no tomb was a wandering spirit, vainly seeking the repose he craved (we still say ``may he rest in peace"). As a wandering ghost, he could never find the offerings and food it needed. He became malevolent, tormenting the living, bringing disease, ravaging harvests and frightening them. Thus, unlike the moderns whose funereal ceremonies are meant to assuage their grief, the ancients performed them for the benefit of the dead.

The ancients did not so much fear death in itself, but were tormented by the possibility of an improper burial. For example, after one sea battle, the Athenians executed the victorious generals for neglecting to bury the dead.

Periodically, the tomb would be visited by the family, who brought gifts of flowers, wine, and food, again for the benefit of the deceased. In the aftermath of the battle between Sparta and Persia\footnote{See Section \ref{sec:RomansSpartansWar} at this book.}, a contingent of the Spartan army visited the battle site annually for at least 600 years.

Yet the duty of the living was not just to provide for the dead, but also to worship them as gods of a sort. This was common to the Romans, Hellenes, and also the religion of the Rig Veda and \textbf{Laws of Manu}. A neglected soul became a malignant spirit while an honored one protected the family. Every pater familias was priest in his own home and kept a hearth where he would perform the rites. Hence, a man was concerned that he die without a son who would provide for his needs in the afterlife.

Fustel mentions a further development, or a deeper understanding, where the dead were not always confined to their tombs and their souls lived together in an underground abode. Here rewards and punishments were distributed according to the lives men had led in this world. Fustel ridicules the earlier beliefs and sees a contradiction where we see a deepening. But we are not so inclined to ridicule our ancestors nor to accept that such practices and beliefs that dominated so much of their daily life were vain and useless exercises. He also explicitly contrasts it to his Catholic faith in order to protect it from his objections to pagan practices, while, again, we see continuity and deepening.

For Catholics, unlike neo-Christians, had a similar dual outlook. The \textbf{Saints}, like the \textbf{Heroes}, went to Heaven while the rest were relegated to a lesser existence in Purgatory where they are in need of the prayers and offerings of the living. Further, the postmortem state is dependent on the quality of the life one lived, in contrast to neo-Christians who reject any such correspondence.

It is clear that we in the West no longer honor our dead, not just those who oppose the past, but unfortunately even by those who claim to be recovering Tradition. Hungry ghosts are wandering the land in revenge, despoiling our cities, and frightening our people. Steps to take include:

\begin{itemize}
\item Honor and remember the dead. Don't speak evil of them. 
\item When celebrating the lives and accomplishments of our heroes, remember them on the death day, not their birthday. 
\item Have sons and pass on Tradition. 
\end{itemize}
And this\footnote{See Section \ref{sec:FloweringEuropean} in this book.} is the tradition we pass on:

\begin{quotex}
Here has passed the greatest people of history, and the most astounding of human civilisations: that people must have taken grandeur from the Egyptian, brilliancy from the Greek, strength from the Roman, and, beyond the strength, the brilliancy, and grandeur, something more valuable than grandeur, strength, and brilliancy—immortality and perfection.

\flright{\textsc{Juan Donoso Cortes}}

\end{quotex}


\flrightit{Posted on 2011-06-14 by Cologero }

\begin{center}* * *\end{center}

\begin{footnotesize}\begin{sffamily}



\texttt{Graham on 2011-06-15 at 12:07 said: }

Could you clarify: what is the relationship between the Akashic record and Purgatory? If I understand you, to say that the souls of our ancestors dwell in Purgatory is the same as to say they are inscribed in the Akashic record. Therefore, to read the Akashic record means to be in communion with the souls of our ancestors. Understanding and propitiating them, in addition to being a filial duty, will bring their protection. Ignore them and they will wreak revenge.


\hfill

\texttt{Cologero on 2011-06-20 at 21:49 said: }

By ``Akashic record", I was referring to the traces left in consciousness of earlier beliefs and events. The examples of saying ``rest in peace", in bringing flowers to cemeteries, or belief in ghosts are some of these traces. They use to have a positive and objective meaning, but are really meaningless (or subjective) gestures in any contemporary frame of reference.

As you point out, Traditional European pagan societies did indeed see the necessity to propitiate the souls of their ancestors, so those ``gestures" mentioned above made sense in that world view. So it is legitimate to ask whether those societies were deluded and obsessed with meaningless gestures, rites and rituals, in other words, it was a fiction useful for establishing roots, giving meaning, and maintaining continuity and stability.

Or is the contrary correct? To wit, Tradition was correct and our failure to be in communion with and revere our ancestors results in our current state of spiritual upheaval. If so, then how and where was this aspect of Tradition maintained? I say that doctrines such as Purgatory and the veneration of Saints are developments of that Tradition. The souls in Purgatory require the prayers and sacrifices of the living to `rest in peace'. We pray to patron saints for specific purposes just as the ancients would have prayed to the corresponding god or goddess. I am here not interested in theological or biblical justifications for these practices, which may have their place, but whose necessity arises only when a real experience has been lost and the doctrines becomes a mere belief.

The fact is that is what Tradition was in ancient Roman, Greek and Vedic practice. The man of Tradition feels a compelling need to pray for the dead and ask for their intercessions.


\hfill

\texttt{Graham on 2011-07-18 at 10:27 said: }

The book `Purgatory Explained'. by one Fr. Schouppe, confirms some correspondances between Catholic and pagan beliefs. Fr. Schouppe relies on scripture and the testimony of saints and mystics for his exposition. Apparently, the soul in purgatory dwells in the grave, – that is, *locally* and in the `underworld’ – and is potentially the source of many temporal and spiritual benefits for the living. Several chapters dwell on the virtue of filial piety. I haven't finished the book, but there has been no mention of such souls becoming malignant or vengeful.


\hfill

\texttt{Cologero on 2011-07-18 at 12:24 said: }

Thanks for pointing this out. I looked at some of it on Amazon. It says the souls can bring ``blessings". There is another book on the same topic (which I am not familiar with) with the coincidentally fascinating title ``Hungry Souls". Isn't the point of the book is that there is more than a mere correspondence between ``beliefs"? There is a real correspondence between experience, even if the understanding of those experiences differs in some respects.


\end{sffamily}\end{footnotesize}
