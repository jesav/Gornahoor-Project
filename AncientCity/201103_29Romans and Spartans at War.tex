\section{Romans and Spartans at War}

\label{sec:RomansSpartansWar}

A few nights ago, I began to watch a video on-line by a cute and vivacious neopagan young woman. In a voice filled with giggles, she boasted about how laid back she and her boyfriend were. In particular, if he didn't light the candles in just the right way, they didn't concern themselves with it. That's when I clicked the stop button.

In \emph{The Ancient City}, \textbf{Numa Denis Fustel de Coulanges} described in detail, more than a generation before Dumezil, the religious practices and social organization in ancient Rome and Greece (with frequent references to comparable practices in the Vedic civilization of India). He describes how the whole of life was oriented toward the transcendent, and that rites accompanied every activity from meals, to governance, to war. Needless to say, these rites were expected to be done perfectly, according to custom.

The ancients are seen to have subsumed all their activities to the will of the gods; gods who did not necessarily love them, nor offered eternal guarantees. For the ancients, the mystery of creation was not of interest, but rather the mystery of generation. Hence, their religion was intimately tied up with worship of ancestors and with keeping their memory alive. Thus, the attack of the counter-tradition on the family goes much deeper than the sociological ramifications; its real aim is to extirpate the very basis of Tradition.

Very often, the discussion of Tradition has been abstract, particularly in regard to caste. To illustrate the dynamic of the relationship among the castes, I will quote two passages, describing battle scenes, one Roman, the other Spartan. Note how the warriors, in a tremendous display of self-discipline, submit their actions to the judgments of the spiritual leaders.

\begin{quotex}
Let us examine a Roman army at the moment when it is preparing for battle. The consul orders a victim [sacrificial animal, usually a chicken] to be brought, and strikes it with the axe; it falls: its entrails will indicate the will of the gods. An aruspex [or haruspex, a priest specializing in this task], and if the signs are favorable the consul gives the signal for battle. The most skillful dispositions, the most favorable, circumstances, are of no account if the gods do not permit the battle. The fundamental principle of the military art among the Romans was to be able to put off a battle when the gods were opposed to it. It was for this reason that they made a sort of citadel of their camp every day. 

\end{quotex}
This demonstrates that the judgment of the priests was important than any merely tactical considerations. This is illustrated even more dramatically in the case of the Spartans, who await that judgment, even in the face of imminent personal danger.

\begin{quotex}
Let us now examine a Greek army, and we will take for example the battle of Plataea. The Spartans are drawn up in line; each one has his post for battle. They all have crowns upon their heads, and the flute-players sound the religious hymns. The king, a little in rear of the ranks, slaughters the victims. But the entrails do not give the favorable signs [according to the diviner … a type of priest], and the sacrifice must be repeated. Two, three, four victims are successively immolated. During this time the Persian cavalry approach, shoot their arrows, and kill quite a number of Spartans. The Spartans remain immovable, their shields placed at their feet, without even putting themselves on the defensive against the arrows of the enemy. They await the signal of the gods. At last the victims offer the favorable signs; then the Spartans raise their shields, seize their swords, move on to battle and are victorious.

\end{quotex}
Fustel summarizes all this by pointing out the role of religion in the lives of the ancients. There is never a question of the secular powers (kshatriya, vaishya) overthrowing that order, although the King often encompasses spiritual authority along with temporal power.

\begin{quotex}
Thus, in time of peace, as in war time, religion intervened in all acts. It was everywhere present, it enveloped man. The soul, the body, private life, public life, meals, festivals, assemblies, tribunals, battles, all were under the empire of this city religion. It regulated all the acts of man, disposed of every instant of his life, fixed all his habits. It governed a human being with an authority so absolute that there was nothing beyond its control.

\end{quotex}


\flrightit{Posted on 2011-03-29 by Cologero }

\begin{center}* * *\end{center}

\begin{footnotesize}\begin{sffamily}



\texttt{James O'Meara on 2011-03-30 at 09:47 said: }

Ah yes, main criteria for making a religion, or anything, acceptable is: just be cool with it, man! The the only imperative is to have no imperative [just as the only intolerance is against those who are `intolerant'. i.e., do not practice total tolerance for favored groups]. Often confused with the New Age as such, its roots are actually far back in the history of soi disant `Liberalism" and so its effects are more widespread, indeed all pervasive, as James Kalb, for instance, has shown in his The Tyranny of Liberalism. We must attack Libya because Gadaffy is violent! We must kill the Serbs because they practice genocide! You'll notice how the image of Zen in the West is old men serving tea, not Samauri warriors or Kamakazi pilots. Take it easy, doood.


\end{sffamily}\end{footnotesize}
