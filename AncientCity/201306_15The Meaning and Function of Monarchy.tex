\section{The Meaning and Function of Monarchy}

\label{sec:MeaningMonarchy}

\begin{quotex}
This essay by \textbf{Julius Evola} was published under the title \textit{Significato e funzione della monarchia}. It was included in ``La monarchia nello Stato modern" by Karl Loewenstein, 1969.
\end{quotex}

\paragraph{Monarchy as a Possibility Nowadays} K. Loewenstein's essay has provided the reader with an overview of all the various forms of monarchy and the possibilities that, in his opinion, remain for a monarchical regime in the present age. Monarchy, as we have seen, is not taken in the literal sense of the term (government of one man, power concentrated in one man) but, correctly, in its traditional and most current sense, i.e., with reference to a King.

Loewenstein's conclusions are rather pessimistic. In order to exist in our day, monarchy should resign itself to being a shadow of what it had been. It could be conceived only within a democratic framework and, properly speaking, in the form of a parliamentary constitutional monarchy. Apart from England, which would be a special case, the model offered by the monarchies of the small states of northern and western Europe — Sweden, Norway, Denmark, Belgium, Holland, Luxembourg — is what should possibly be kept in mind.

In the analysis of the range of the various arguments adopted in favor of monarchy, Loewenstein tried to be objective, but was unable always to be so. The precise aversion to every principle of true authority is quite visible in him, while an insufficient emphasis is given to the factors of an ethical and immaterial character. Now we believe that if you were forced to conceive of a monarchy only in an empty and democratized form, besides only being possible because it concerns marginal small states, not yet involved in the dynamism of the great forces of the era, we undoubtedly might as well end the discussion in the negative.

It must be recognized, however, that pessimistic conclusions regarding monarchy appear largely justified only if you hypostatize the situation of the current world and believe that it is irreversible and destined to continue itself indefinitely. This situation is defined by a general materialism, the prevalence of base interests, the egalitarian error, the government of the masses, technocracy, and the so-called ``consumer society." Except that we are beginning to multiply the signs of a profound crisis of this world of affluence and counterfeit order. Various forms of revolt are already noticeable, for which it is not impossible that it could reach a state of tension and a breaking point, and that, especially in the face of possible liminal situations, tomorrow different forms of sensitivity may be reawakened, reactions occurring similar to those an organism is capable of when it is mortally threatened in its deepest being.

The supplanting, or to a lesser extent, of this new climate is the decisive element also for the problem of monarchy. In our opinion, it should be placed in the following terms: What meaning could monarchy have in the case that such a change in climate should take place, and in what form could it be a center for the reconstitution of a ``normal" order — normal in a higher sense? Certainly, the presence of a true monarchy in a nation would have a rectifying power, but this is a vicious circle: without the premise that we mentioned, any restoration would have a contingent, not organic and, in a sense, unnatural character.

The disorder present in the political field, everything that it shows of instability, dangerously open to subversion — to Marxism and communism — substantially derives from the deficiency of a superior principle of authority and from an almost hysterical impatience for such a principle, through which certain political experiences of recent times serve at most as a convenient alibi. Speaking of a superior principle of authority, we refer to an authority that has an actual legitimacy and, in a certain way, a ``transcendent" character, because without this, authority would lack any basis, it would be contingent and revocable. A truly stable center would be missing.

It is important to clearly fix this essential point, in order to differentiate the type of monarchy, which this essay deals with, from monarchy in the broad sense of power or government of one man. In fact, spurious counterfeit forms of authority are conceivable, and are even realized. Communist regimes are also based on a de facto authoritarianism that can disguise the crudest and even tyrannical forms which are the justifications that they mendaciously give. One can put the dictatorial phenomenon along the same lines if it is conceived otherwise than in relation to emergency situations as originally occurred in ancient Rome.

On the other hand, the antithesis, so often advanced between dictatorship and democracy, is relative, except that you examine the existential foundation of these two political phenomena, that is, a ``state of the masses". If the dictatorship has not purely functional and technical characteristics (an example is offered currently by the Salazar regime in Portugal), if it is based on pathos as in some recent plebiscitary and populist forms, the same element galvanizing it is activated by every democratic demagoguery. The dictator makes a bad surrogate to the monarch with the appeal to forces that confusingly seek a foothold, a center, whatever it is, just to come to the head of chaos, disorder, situations that have become unbearable. This also explains, however, the phenomenon of possible, abrupt changes in polarity as a result of some trauma that has suspended the cohesive and driving force of the system, as in a magnetic field when the power goes out. The most perspicuous case is perhaps provided, in this respect, by the astonishing change in the collective political climate occurring in present-day Germany, after the almost frantic mass enthusiasm that had characterized the previous dictatorial period. It is significant that, on the contrary, a similar phenomenon of inversion was not produced in Germany after the First World War, because its antecedent was not a dictatorship but a traditional monarchy.

Through the ``transcendence" of the principle of authority characteristic of regality, the monarchical regime constitutes the only real antithesis both to dictatorship as well as absolute democracy. We must indicate the basis of its superior right for that reason. The various forms that it may take and the ideas or symbols that can legitimize this transcendence according to the times, do not touch the essential: the essential thing is the principle. Loewenstein is right when he says that in a world desacralized by the natural sciences, in which religion itself is undermined, there can no longer be a question of the mystique of the monarchy that in other times was supported on specific theological conceptions and a liturgy. But if you take a look at the world of the holders of the crown at all times and in all places, the recognition of the need for a stable center can be seen as a common and constant theme, a pole, something that to be truly stable must have, in a certain way, its own principle in itself or from above, which must not have a derived character. In this respect one can take a look, for example, at F. Wolff-Windegg's excellent work, Die Gekrönten. Someone rightly wrote: ``A purely political royalty — it can certainly be said — has never existed." Not so long ago, the sovereignty of divine right ``by the grace of God," did not imply, in its subjects, specific theological considerations; its value, so to speak, in existential terms, corresponded precisely to the need for a higher point of reference that absolutely does not happen when the king is such only through the ``will of the nation" or ``the people." On the other hand, only under that assumption could those dispositions, those forms of behavior and customs of a higher ethical value develop, in the subjects, in the sign of loyalty, which we will discuss shortly.

So we cannot share Loewenstein's opinion that the ideal argument in favor of monarchy is now invalidated. What he says is true, of course, namely that the decline of monarchy is due not so much to democracy as to the coming of cars and aircraft, the automobile, television — you can say, in general, the technological industrial civilization. But here we have to wonder if, in fact, we are entitled to hypostatize this civilization, we must ask ourselves to what extent man wants to accord to everything a value different from that of a set of simple, mundane means, which in ``consumer society" leaves an absolute inner emptiness. Let us repeat: it is primarily a question of the ``dignity" of monarchy, an esteem and a right that always and everywhere drew from a supra-individual and spiritual sphere: sacred investiture, divine right, mystical or legendary filiations and genealogies, and so on, were only imagined forms in order to express an always recognized substantial fact, namely that a political order, a truly organic and living collective unity is only made possible where there is a stable center and an elevated principle in respect to any particular interest and the purely ``physical" aspect of society, a principle independently having a corresponding intangible and legitimate authority. Therefore, in principle what Hans Blüher wrote is absolutely correct: ``A king who lets his sovereign function be confirmed by the people, admitting thereby that he is accountable to the people — instead of being responsible for the people before God — such a king renounced his kingship. No infamy committed by a king — and God knows if they were not committed — destroys the mystical objective sanction of the sovereign. But a democratic election destroys it immediately."

\paragraph{The bound of loyalty} If in the past, the bond of fidelity that united the subject and follower with the sovereign could be treated as a sacrament — sacramentum fidelitatis — something that was preserved even later as the quite perceptible foundation of a special ethics, an ethic, in fact, of loyalty and honor, which could acquire a particular force in the assumption, just now indicated, of the presence of a personalized symbol. In normal times, the fact that the sovereign as an individual might not always be at the height of the principle, did not matter; his function remained unprescriptive and intangible because obedience was not to the man but to the king and his person had value essentially as a support so that the capacity for super-individual dedication, that pride in serving freely and possibly even the readiness to sacrifice (as in the dramatic moments when a whole people rallied around their sovereign) could be awakened or propitiated, that they might constitute a way of elevation and dignification for the individual and, at the same time, the most powerful force to hold together the union of a political body and to limit in it what it has that is anodyne and disheartened, and in recent times has taken a dangerous extent.

That everything that cannot be achieved to the same extent in another form of political regiment, is quite obvious. A president of the republic can be flattered, but no one will ever recognize in him anything but a functionary, a ``bourgeois" like any other, which only extrinsically, not on the basis of an inherent legitimacy, is vested with a temporary and conditioned authority. Whoever maintains a certain subtle sensibility perceives that ``being in the service of their king", the ``fight for their king" (even the fight ``for their own country," despite the romantic coloring, has in comparison something less noble, more naturalistic and collectivistic), the ``representing the king", all have a specific quality, all of which indicates instead a parodic, not to say grotesque, character when it pertains ``to one's own president". Especially in the case of the army, high bureaucracy, and diplomacy (regardless of the nobility), this appears very obvious. The same oath, when it is not paid to a sovereign but to the republic or one or another abstraction, has something discordant and empty about it. With a democratic republic, something immaterial, but still essential and irreplaceable, is inevitably lost. The anodyne and the profane prevail. A monarchist nation that becomes a republic is, in a certain way, a ``degraded" nation

If we observed that the kind of fluidity that forms around the symbol of the Crown is quite different from what may be related to the exalted ``states of the multitude", which can arouse or favor the demagogy of a popular leader, the difference also exists with regard to any simple nationalistic mysticism. Of course, the sovereign also incarnates the nation, symbolizes its unity on a higher plane, establishing almost, with it, a ``unity of destiny." But here we find the opposite of every Jacobin patriotism; there are none of those confused collectivizing myths that speak to the pure demos and that almost divinize it. It can be said that monarchy moderates, limits, and purifies simple nationalism; which, as it prevents any dictatorship replacing it with advantage, so it also prevents any nationalistic excess; it defends a structured, hierarchical, and balanced order. It is known that the most calamitous upheavals of recent times can be attributed mainly to unrestrained nationalism.

After what we have said, it is clear that we do not share at all the idea that monarchy at this point should be democratized, that the monarch should assume almost bourgeois features — ``must come down from the august heights of the past and present himself and act in a democratic way," as Loewenstein claimed. That would simply destroy his dignity and his raison d'.tre, as we indicated. The king of the north European countries who carries a valise, who goes shopping in the stores, who consents to letting radio or television display his well-behaved family life to the people including his tantrum-throwing children, or else the Royal House that is provided for the curiosity and gossip of the news magazines, and whatever else one thinks, might make people close to the king, including, in the end, a good-natured paternal appearance (if the father is conceived in a bland bourgeois form), all this cannot avoid damaging the very essence of the monarchy. The ``Majesty" then really becomes an empty epithet of the ceremony. It has rightly been said that ``the powerful who, through a badly understood sense of popularity, consents to get closer, ends up in a bad way."

It is clear that take to take all that as firm, means going against the current. But, again, we pose an alternative: it is a question of accepting, or not, a state of fact as irreversible, thinking that only the useless vestiges of monarchy can exist. One of the elements to consider in this regard is the intolerance in our world, for distance. The success of dictatorships and other spurious political forms is due, in part, precisely to the fact that the leader is seen as ``one of us", the ``Great Comrade," and only in these terms is he accepted as a guide and obeyed. In these circumstances the concern for `popularity' and for ``democratic" means is quite understandable. But that, basically, is anything but natural; we do not see why he should be subordinated when the leader, in the end, is just ``one of us" when an essential distance is felt, as in the case of the true sovereign. So a ``pathos of distance" — to use one of Nietzsche's expressions — should be substituted for that of affinity, in relationships that exclude any haughty arrogance on the one hand, and every servility on the other. This is a basic point, in its existential character, for a restoration of the monarchy. Without exhuming anachronistic forms, instead of propaganda that ``humanizes" the sovereign in order to captivate the masses, almost on the same line as the U.S. presidential election propaganda, one should see to what extent traits of a figure characterized by some innate superiority and dignity can have a profound activity in a suitable context. A kind of asceticism and liturgy of power could play a part here. While just these traits will enhance the prestige of the one who embodies a symbol, they should be able to exert a force of attraction on common man, even pride, in the subject. Moreover, even in fairly recent times there has been the example of Emperor Franz Joseph who, while interposing the strict ancient ceremonial between himself and his subjects, while not imitating in the least the ``democratic" kings of the small Nordic States, enjoyed a particular, not common popularity.

To sum up, the main prerequisite for a revival of monarchy, pursuant to the dignity and function which we mentioned, there remains, in our opinion, the awakening of a new sensibility for an order that is detached from the most material, and also the simply ``social", plane, and tends to everything that is honor, loyalty, and responsibility, because similar values in the monarchy have their natural center of gravity; while, in turn, the monarchy will end up degraded, reduced to a simple formal and decorative survival when these values are not alive and active — first in an elite, then in a real ruling class. They are not the same chords that the defender of the monarchical idea and of any other system must make resonate in the individual and in the community. So it is absurd to entrust the destinies of the monarchical idea to propaganda and a praxis that approximately copies the methods of the opposed party in a democratic spirit. Even today being able to ascertain the appearance of tendencies toward an authoritarian center, towards a ``monarchy" in the literal sense (= monocracy) is not enough, after what we said about the profound differences which the various objectifications of the principle of unity and authority may present. The meaning of what is not allowed to be sold, bought, or usurped in the dignity and participation in political life is a decisive factor and escapes like water through their fingers for those who think only in terms of matter, of personal advantage, hedonism, functionality, and rationality. If one must no longer speak of that meaning because of the famous Marxist ``meaning of history", which is claimed to be irrevocable, we might as well set aside definitively the cause of monarchy. This would, moreover, be tantamount to profess the most bleak pessimism in regard to what still can appeal to man of recent times.

\paragraph{Constitutional Monarchy}
After having considered the spiritual aspect of the problem of monarchy, it is necessary to indicate the aspects that are related on the positive, institutional, and constitutional plane. On such a plane, it will be necessary to make clear the specific function to attribute to monarchy and what differentiates a monarchical system from other systems. It is amazing that a comparable problem is almost not faced by the propaganda of the monarchists. In elections there were, even in Italy, discourses by the monarchists who blamed, more or less on the same lines of other sectors of the opposition, the dysfunctions of the republican democratic and partocratic State and the danger of communism, avoiding however indicating, in no uncertain terms and without fear, in which terms the presence of the monarchy would positively eliminate both, or, better put, in virtue of which particular prerogatives the monarchy would be for so much.

If one is really a monarchist, one cannot concede that the monarchy becomes reduced to a simple decorative and representative institution, a kind of nice furniture or, according to the image mentioned by Loewenstein, something like the golden figure that was put on the bow of a galleon; the State, in concrete terms, would remain that of the republican parliamentary democracies, concerning the king only to countersign, as would a president of the republic, whatever the government and parliament decide. The restoration should instead involve a kind of monarchical revolution (or counter-revolution).

The well-known maxim ``the king reigns but does not govern", should be opposed to the other: ``the king reigns and rules" — rules, of course, not in terms of the absolute monarchies of the past, but, in the normal way, in the framework of established law and a constitution. In this regard, the best example was given to us by the previous central European monarchies, for which Loewenstein has not hidden his strong antipathy. Not only a regulatory, moderating, and arbitral power with respect to various political forces should be reserved to the sovereign but also that of a last resort. The constitution and the law should not be made into fetishes. Constitution and law do not fall ready-made from heaven, they are historical formations and their intangibility is conditioned by the normal course of things. When this course fails, when faced with emergency situations, a higher power must assert itself positively, which has remained dormant and inactive under normal conditions; it does not for this reason cease to constitute the center of the system. The king is the legitimate subject of that power. He can and must exercise it whenever it is necessary, saying, ``Thus far and no farther," and preventing every subversive revolutionary movement (preventing it by means of a ``revolution from above"), as well as any dictatorial upheaval whose only justification is the lack of a true center of authority.

It is not said that such power must be exercised directly by the sovereign; he may do it through a capable and decisive chancellor or prime minister who, strong in the support of the Crown and essentially responsible as its face, can deal with the situation. The case of Bismarck in the ``institutional conflict" mentioned by Loewenstein corresponds to this possibility. Certain of the confidence of the sovereign, Bismarck could also take no account of parliament's opposition and by following his path, he made the greatness of Germany, receiving later the approval of his work in a new constitution.

One might venture to say that, in part, there was a similar situation at first sight when the King of Italy supported Mussolini, granting him powers that however, Victor Emmanuel himself, if he had not felt so constitutionally bound, could have exercised, so as to impose an order on an Italy shocked by subversion and the social crisis through new structures, without the need of fascism, and preventing those developments — defined by some in terms of a ``diarchy" — which finally undermined to some extent his position through the presence, almost, of a state within the state. At decisive moments a sovereign should never forget the saying of an ancient wisdom: Rex est qui nihil metuit (The king is the one who fears nothing). Through a badly understood humanitarianism, in extreme cases, even the danger of battles in which blood might flow, he cannot be afraid because this is not about persons, but of making authority, order and justice rule above all things, against possible turmoil by a part of it. The formula, as we have already stated: ``Thus far and no farther." In unexceptional situations Benjamin Constant's conception of the Crown as the ``fourth power", as an arbitral and balancing function, can be accepted. Even the rights recognized by Bagehot for the Crown: the right to be consulted, the right to encourage, the right to warn, are unexceptionable.

Therefore, a shift of the center of gravity should be effectuated with a monarchical restoration. A national delegation may also be chosen by the ``people", according to some modality (which we will return to), but it should be responsible, in primis et ante omnia , over against the king, according to relations of personalized responsibility which would close the door on many forms of democratic corruption. The king should be, therefore, the supreme point of reference, and the previously mentioned values of loyalty and honor should be felt, rather than the representatives being the instruments of the parties and the mysterious, ephemeral entity of the ``people" whom they exploited, and who alone has the power to confirm or repeal according to the system of absolute democracy, i.e., of officially recognized universal suffrage.

On the other hand, for a true renewal of monarchy the ideal of an organic State needs to be present, through which the problem of the compatibility in general between monarchy with the system of absolute parliamentary democracy cannot be evaded. The superimposition of one over the other can only lead to something of a hybrid. It is to be considered that if the hoped for change in mentality will be achieved, the absurdity of the system of representation based on indiscriminate universal suffrage will gradually be recognized, i.e., on the law of pure number, having as the obvious premise not the conception of the citizen as a ``person" but his degrading reduction to an interchangeably undifferentiated atom.

In this regard, it must be remembered that modern democracy in its absolute form is one thing, another is a system of representation, the latter not necessarily coinciding with the former. It is known that a system of representation also existed in traditional monarchical States, but generally as organic representations, i.e., of bodies and orders, not of ideological parties. To want to consider the parties, the best system would be bipartisan, accepting an opposition that acts constructively and dynamically within the system, not outside of it or against it. (For example, that a revolutionary or communist party, whenever it observes certain purely formal statuary norms, can be considered ``legal" and must be allowed in a national assembly even though its stated or implied program is the overthrow of the existing order, is a true absurdity). Apart from the bipartisan solution, already adopted with advantage in England's monarchy, the representative system that through its most organic character should be harmonized with the monarchy would be traditional corporative, in the broadest sense, without reference to the attempt, which was made by fascism with the creation of a corporate rather than partocratic Chamber. Perhaps the current Portuguese system — the Spanish to a lesser extent — will approach the desired order. Loewenstein highlighted the alternative that would occur in the case of a restoration, because either the king is supported on the upper classes who are more inclined to sustain the monarchy, and then he would play into the hands of those who are quick to accuse him of conservative reactionism, or else he goes toward the working classes and, in general, starts acting as the ``king of the people," and then he would dangerously alienate the support of the other part of the nation.


\paragraph{Conclusion} Now, a similar turning point obviously presupposes the retention, the perpetuation, of the state of class struggle, in terms of Marxist ideology. But we believe that one of the prerequisites for a new, organic, and monarchical order must be seen exactly in overcoming this antagonistic division of national forces. Corporate reform should aim precisely at that, which the mentioned alternative is carried out, opposite which the restored monarchy would be found, or would fail to a large degree. Even if within corporations, or whatever you want to designate as the primary representative authority, opposing tendencies were asserted, one is to think that the preeminence given to the principle of jurisdiction would reduce the ideological factor considerably in such differences.

In a sector that has become increasingly important to the system of corporate representatives on the basis of responsibilities could exhibit an actual character of a particular path of development exhibited by almost teratology presented by the technocratic element and, in general, by the economy. We know of the criticism against the technological civilization of consumption in the most advanced industrial society; the destructive aspects that are typical have been shown, the need to put a brake on economic processes that have become almost independent, like the image of the ``unleashed giant" used by W. Sombart. Now it is not possible to envisage a brake on the system, a restraint, without the intervention of a higher political power. The task of adequately restraining and ordering on the strength of a more complete hierarchy of interests and values, the forces in motion in society, obviating also a paradoxical situation that has occurred in recent times, that of an increasingly strong state with an increasingly weak head, would evidently find the most favorable environment for its implementation in a true monarchical state.

Institutionally, the authority could be provided either by a single assembly, but, alongside representatives of economic and productive forces also comprising representatives of the spiritual and cultural life (as there were, in fact, in the `General States' or Diets, similar assemblies of ancient traditional monarchical regimes), or by the bicameral system, an Upper House and a Lower House, the latter being truly corporative, the former making itself felt instead in the higher-level instances. We know that the latest ``conquest" of absolute democracy was to reduce the upper House, or Senate, to a useless duplicate of the other House because even for it the principle of the election of the masses and of interim elections (at least for most of its elements) was asserted. As in the Italy of yesterday, the definition of the Upper House should be, instead, one of the essential tasks of the monarchy, even if only conveniently assisted, continuing the formal nature of the appointment from above.

In this way the Upper House would remain the political body closest to the Crown and it would be natural that loyalty, fidelity, and active impersonality were present in it to the highest degree. It should have power, authority, prestige, and a meaning different from that in the Lower House. As custodian of values and higher interests, it would constitute the real nucleus of the state, its ``head". It would, therefore, have to emphasize its active functional character in the place of the codetermination of the political line, a character that will differentiate it greatly from what had been, in post-Risorgimento monarchical Italy, the Senate: an assembly of worthy people, of ``high intelligent men", of notable personages according to worth, yet as essentially decorative role, without any real, vigorous organic function.

Without dwelling on the details, it is clear that a system of this kind would overcome the aberrations of absolute democracy and the republican partocracy and it would have its natural integration in the monarchy. Here monarchy would not be something heterogeneous, almost the remains of another world, superimposed on the current parliamentary system. Therefore, de rigueur, the problem of monarchy returns as part of a larger problem, that of the ``revolutionary" reshaping of the entire modern state.

But for the functions of the monarchy that we have tried to sketch out, in order to be able not only ``to reign" but also to have an active part — more or less critical depending on the circumstances — in the ``government", it is clear that it would require a special qualification of the sovereign not only in terms of character, pursuant to the strict traditional education of princes, but also in terms of expertise, knowledge, and experience. This is made necessary by the character both of the times as well as the modern state. The ancient regal Far Eastern conception of wei-wu-wei, of ``action without action", is suggestive, alluding not to a direct material action but to an action ``through presence", as the center and quintessential power. This aspect, while maintaining its intrinsic validity in terms just mentioned, when, as in current times and probably still more in those that were predicted, everything is in motion and forces tend to move out of their normal orbit, needs to be integrated, while taking care that it is not crippled in this way.

As we said, in other times in a monarchy the symbol also had preeminence over the person; given the overall climate and given the strength of a long tradition and legitimacy, it was able not to be jeopardized by the merely human aspects of the person who in either case embodied it. If today or tomorrow we were to come to a restoration of the monarchy, this would no longer be possible: the delegate should be at most at the stature of the principle, not through the ostentation of the person, but through the opposite. He should also have the qualities of a true leader, a man capable of holding the scepter more than just symbolically and ritually. Such a qualification to this day cannot be only like that of the ages of the warrior dynasties. The qualities of character, courage, and vitality, while remaining the essential basis, should be united with those of an enlightened mind with the essential political knowledge adequate to the complex structure of a modern state and the forces at work in contemporary society.

The decline of traditional regimes had two causes which acted firmly even before the materialistic climate of modern civilization and industrial society was added. On the one hand, at the top there was in fact a growing inability to fully embody the principle especially when the general structures were beginning to creak; on the other hand, at the bottom, there was the failure, in the people who had become more or less the ``masses", of a specific sensibility, of certain capacity of recognition. Therefore, the possibility of a monarchical restoration is subjected to a double claim, and appears to be conditioned by the removal of both negative factors. On the one hand, rulers would be valued, who owe their prestige not just to their super-elevated position or to the symbol that overshadows them, but who are also able to cope with any situation as exponents of an idea and a higher power. On the other hand, a change in the general mental and moral level of the masses would be required, which need we have not tired of emphasizing.

Nowadays, one or the other conditions appear hypothetical. But if we do not have to come to essentially negative conclusions, to be drawn from studies on monarchy in the modern state, such as that undertaken by Loewenstein; if it must not be considered merely as an institution, a pale shadow of what monarchy was, it is now almost entirely devoid of its meaning and its essential raison d'.tre, there is no other way of laying out the problem. It is therefore worthwhile to repeat that the fate of monarchy appears to be, in a certain way, in agreement with that of the entire modern civilization and more properly depends on what may be the solution to a crisis which, as appears from many clues, is assailing the very foundations of that civilization.

\flrightit{Posted on 2013-06-12 by Aeneas }

\begin{center}* * *\end{center}

\begin{footnotesize}\begin{sffamily}



\texttt{h.ontologia on 2013-06-15 at 11:33 said: }

We reccomend Dr. Erik Maria Ritter von Kuehnelt-Leddihn. His book, ``Leftism, From de Sade and Marx to Hitler and Marcuse", among other material, is available online free of charge.


\hfill

\texttt{Ash on 2013-06-16 at 21:25 said: }

Unfortunately, Evola seems to be sinking into an emotional argument in this part of the essay, particularly with the vague accusation that the Republic has ``something less noble" about it. He may have trouble convincing the Republican-era Romans, the Swiss, the Irish, or many other peoples of that fact. A Jacobin could just as easily make the argument ``is it not far more noble to fight for the ideal and idea of the Republic than the petty wars of a man with a crown?" And those on Gornahoor who place the spiritual as coming before the temporal would have trouble against that argument, even if we're talking about a good King vs. an ideal Republic. He also forgets that the earliest kings were usually not hereditary. Sparta and Rome provide key European examples of this. 

While I am a monarchist with regards to the Canadian/UK/Commonwealth and European monarchies, and with regards to the ``ideal" forms of governance, I must confess I have always found Maurrassian arguments more useful in this sphere: the role of the institution in the State. With regards to its ability to stand above petty day-to-day politics, I must say that our tradition of constitutional monarchy has at least allowed this aspect of the monarchy to become far more entrenched, one might say to an extreme. Certainly, monarchs like Tsar Alexander III, ruling with the ``right and power of Autocracy", are far more beholden to the daily trials, conspiracies, and deals of ruling a State. Evola does a good job of showing how monarchy can be a reflection of the universal order, but he fails to show why other forms of governance cannot be held valid in this regard as well. The Jacobin example can be cited here – should not the foundation of the State be an Idea rather than a position held by a person? (Devils' advocate, of course).


\hfill

\texttt{Avery on 2013-06-17 at 04:48 said: }

…whatever else one thinks, might make people close to the king, including, in the end, a good-natured paternal appearance (if the father is conceived in a bland bourgeois form), all this cannot avoid damaging the very essence of the monarchy… It has rightly been said that ``the powerful who, through a badly understood sense of popularity, consents to get closer, ends up in a bad way."

It is interesting to note that the Emperor of Japan has only addressed the people on three occasions. The first was when Emperor Meiji presented the Imperial Rescript on Education, a Confucian document that all schoolchildren of that era had to memorize as holy writ. The second was to announce Japan's surrender; here the Imperial voice made it possible to believe the unbelievable. The third and most recent was when the present Emperor appeared on television to console the people, as a ``fatherly" figure, after the 2011.3.11 earthquake. This was from beginning to end a well-intentioned idea for an extraordinarily horrific day, but at the same time I hope it is not repeated, for the reasons Evola describes above. It was of course based in the thinking of 20th century Europe, which long ago sunk to such frivolities as the ``Royal Christmas Message" in Britain.

The same oath, when it is not paid to a sovereign but to the republic or one or another abstraction, has something discordant and empty about it. With a democratic republic, something immaterial, but still essential and irreplaceable, is inevitably lost.

Not that hard to put your finger on it; Evola did it himself in other works. When the people are asked to fight for representatives that they elected or for symbols of democracy, that is to say for a mirror of themselves, there is no honor involved at all. (Note that honor encourages heroic, not ``humane", behavior. A ``humane" military is one in which the will to fight will be eclipsed.)

This is such simple psychology that Americans from the beginning have fought to defend the Constitution — that is, a document written by their forefathers which guarantees their freedom — rather than their President, the Union, or anything like that. This provided an artificial ``distance" for the first two centuries of the Republic, and for the most part kept the military above politics.

In reality, though, the laws that grow out of the Constitution can be modified by the people, so the American military has in the past few decades noticeably degraded into a force fighting for ``the will of the people", which now makes it difficult to acknowledge any higher principles. One particularly disturbing example of this could be seen in this month's blogs\footnote{\url{http://townhall.com/columnists/toddstarnes/2013/06/07/conservative-christian-soldier-told-not-to-read-levin-or-hannity-in-uniform-n1615420/page/full}}. The appearance of such internal confusion should make it possible for smarter people to calculate the decade in the near future when the American military will simply disintegrate.

As for the complaint about the large number of democracies, elected monarchies, or at least aristocracies in the Classical period: to associate these with the Faustian political systems is a misunderstanding. The kings of the Classical world were simply the head priests, and where there were no kings a direct appeal to the gods would do. Perhaps Evola will address this later on.


\hfill

\texttt{Jason-Adam on 2013-06-17 at 13:17 said: }

@ Avery, I will need to depend on your judgement as I have never been to Japan, but can we say that the Japanese tradition is still living ? From what I know, it seems as post-WW2 Japan has been Americanised and filled with liberalism, pacifism, socialism and degeneracy not to mention ``guilt" for the ``crimes" of the past, maybe not as bad as in Germany but similar ? 

Two things struck me about the Evola article, I read the full piece in Italian rather than wait for Cologero to finish the translation and twice he mentioned Portugal as offering an example of a normal socio-economic system. Traditional Catholic believe the same and point to Portugal as a superior case to XXth century Italian and German attempts to overcome democracy. A good book to read is this one : \url{http://www.strobertbellarmine.net/books/Derrick–Portugal.pdf}

Another book that Evola's piece reminded me of the THE MENACE OF THE HERD by Erik von Kuehnelt-Leddihn which by the way has some good point s about the USA as well.


\hfill

\texttt{Avery M on 2013-06-17 at 21:32 said: }

The best part of this article so far was in Part 1: A political order, a truly organic and living collective unity is only made possible where there is a stable center and an elevated principle in respect to any particular interest and the purely ``physical" aspect of society, a principle independently having a corresponding intangible and legitimate authority. Even though the article is about monarchy, this actually makes a fine case for a democratic state like Portugal's Estado Novo as long as it preserves a unwavering Catholic character.

You are correct about liberalism and degeneracy in Japan, to an extent that many outsiders do not realize, but actually the conservative side of the intellectual world has put leftist historical views in eclipse, especially compared to Germany. A drive to learn from Japan's excesses but honor the sacrifices of the past is rather mainstream these days. And the Imperial House, as I indicated above, has proven much less willing than its European counterparts to beg the media for attention.

(On a side note, one thing I am concerned about is that Japan's netuyo online reactionaries, while their unconscious intuition may be similar to those on this blog, have no social mores, sense of duty beyond knee-jerk nationalism, or capacity for self-improvement. That's not really related to this article, though.)

\hfill

\texttt{Ash on 2013-06-17 at 21:47 said: }

Agreed. Evola unfortunately does not mention here the ways in which a transcendent foundation could manifest outside of monarchy. A question about your comment on the American military though – when was it *not* an expression of the will of the people? Even at its very foundation it was fighting for states or for the Republic, which originated from a declaration claiming to speak for ``We the people."


\hfill

\texttt{Jason-Adam on 2013-06-18 at 14:28 said: }

It is precisely because I could not find any transcendent ideal in the American military that I chose to abandon my original plans to attend West Point. I could not in good conscience serve something that I knew to be false and beneath me. While Evola probably would have advised me to join the army anyhow and use it as a personal means of development, I just felt like it would be dishonourable for me to participate in what I knew to be an evil and anti-traditional institution. Was I wrong ?


\hfill

\texttt{Ash on 2013-06-18 at 23:44 said: }

Well, I certainly couldn't call the American military a ``traditional" institution in the sense the word is used here. Politically speaking, I also think that it is currently doing great harm in the world, but whether or not serving would be good for ones' own development and whether that can be done in accord with the regime and program one is serving is something one would have to look so a spiritual advisor for. Speaking personally, I would not serve in the American military because of the regime it serves, but I would not call the institution in and of itself evil or dishonourable, and even less so those who serve in it to the greatest degree of honour that they can. I can certainly see Evola giving that strain of advice, but I would say that this would be an example of his tendency to lose himself in the symbol rather than what it represents (thus making a similar error to his monarchy argument). A man could probably develop virtue and soul serving a lesser regime, but if he has a choice in doing so then can that be reconciled?


\hfill

\texttt{Constantine Aetos on 2013-06-18 at 00:36 said: }

This maybe a bit out of topic, but I wonder why Traditionalists have not published a political manuscript that is similar in structure to the Communist manifesto? Or is that too much to ask.


\hfill

\texttt{Avery on 2013-06-18 at 00:57 said: }

You may take your pick of political manifestos: 

\url{https://www.facebook.com/pages/Generation-identity-The-book/497600770293421}

\url{http://www.arktos.com/alain-de-benoist-and-charles-champetier-manifesto-for-a-european-renaissance.html}

\url{https://www.facebook.com/Traditionalism}

But if we mean philosophical perennialism, consider the idea of writing a manifesto to tell people to read Meister Eckhart or ``The Alchemical Wedding" or something like that… it's kind of silly.


\hfill

\texttt{Cologero on 2013-06-18 at 06:53 said: }

The short answer, Constantine, is that they are waiting for George to do it. Other than that, I find it quite remarkable that you were able to pack so many misconceptions into just two sentences. Kudos for that.


\hfill

\texttt{Jason-Adam on 2013-06-18 at 14:23 said: }

Manifesto ? Try Men Among the Ruins for Evola's take. For those more in line with Europe's Mediaeval past try the writings of Plinio Correa de Oliveira.


\hfill

\texttt{Pickman on 2013-06-18 at 20:04 said: }

The last successful attempt at that, Constantine, was the Bible (NT not just the 10 commandments). Practice the sacraments and doctrines of the Church to hammer in the necessary dogma. Exoteric manifestos have not faired well in the modern era.


\hfill

\texttt{Constantine Aetos on 2013-06-20 at 12:52 said: }

Looks like I will be busy reading this summer.


\hfill

\texttt{Ash on 2013-06-20 at 22:38 said: }

The mentioning of the Portuguese Estado Novo is a welcome concrete example of what Evola is imagining. The reign of Frederick the Great of Prussia and his successors would similarly seem to be along the lines he is considering. Unfortunately one of the things that undermined many of these regimes was that the ``revolution from above" undermined its own authority by depriving its citizens of the rights the regime itself claimed to uphold. It is always far better to win an enemy over than to fight him; similarly, the monarchic (or ``conservative") state is more stable insofar as it defuses radicalism by co-opting it. This is something which Bismarck in particular understood with regards to his socialist foes, although much to his chagrin. 

Their other fatal weakness, particularly in the case of Portugal, was that when it came to the actual act of governing, they were sometimes the victims of severe incompetence. Double-digit inflation was a recurring problem in Portugal. and bad planning during Spain's industrialization threatened the social order of the quickly expanding Spanish cities. Corruption was of course always a problem and this is a problem which any true proponent of an ordered State must face. It's not good enough to say ``well look at the others"…we must strive to a higher standard.

As an aside, I'm currently reading Massie's ``Nicholas and Alexandra", his work on the last Russian Tsar. It provides a good picture of the failures of this Tsar to maintain the more direct ``autocracy" of the Russian monarchy and a firm hand on his advisors as revolutionary subversion spread. His father, Alexander III, would also likely be of interest to many on Gornahoor.


\end{sffamily}\end{footnotesize}
