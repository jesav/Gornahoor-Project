\section{Nationalism and Cosmopolitanism}

\begin{quotex}
We are in a world of Generation and death, and this world we must cast off. \flright{\textsc{William Blake}, \textit{Vision of the Last Judgment}}

\end{quotex}
\textbf{Vladimir Solovyov} wrote a massive volume, \textit{The Justification of the Good: An Essay on Moral Philosophy}, which has achieved a modicum of notoriety in the past few years\footnote{\url{https://www.nytimes.com/2014/03/04/opinion/brooks-putin-cant-stop.html}}. There are two sections — on Nationalism and on War — that are worth reviewing, with the goal of understanding the moral and intellectual currents behind current events.

\paragraph{Collective Hostility}
First of all, Solovyov needs to establish the existence of superpersonal morality above and beyond the personal. He begins, not from moral principles, but by identifying the collective evil:

\begin{quotex}
The work of embodying perfect morality in the collective whole of mankind is hindered, in addition to individual passions and vices, by the inveterate forms of collective evil which act like a contagion.

\end{quotex}
There are three aspects of superpersonal or collective hostility, although only the first one is of interest for our purposed.

\begin{itemize}
\item \textbf{National}: between different nations 
\item \textbf{Penal}: between society and the criminal 
\item \textbf{Socio-economic}: between the different classes of society. 
\end{itemize}
He points out the prejudices between the various nationalities in Europe at the time. If Western Europe claims to have overcome such mutual hostilities, it is still not the case in the Balkans. There is open hostility toward Asians and Africans. Nowadays, hostility toward one country in particular is not only tolerated, but is actually considered virtuous. Unfortunately, rapprochement can never be achieved, given such hostility.

\paragraph{National Differences}
Two contradictory ways have arisen to overcome mutual hostility, which makes matters worst:

\begin{itemize}
\item Social violence. 
\item The elimination of all differences. 
\end{itemize}
These manifest, in the international sphere, as Nationalism and Cosmopolitanism, although the latter often goes under the guise of globalism these days. The following definitions are Solovyov's.

\begin{itemize}
\item \textbf{Nationalism}: We must love our own nation and serve it by all the means at our command, and to other nations we may be indifferent. If their interests conflict with ours, we must take up a hostile attitude to the foreign nations. 
\item \textbf{Cosmopolitanism}: Nationality is merely a natural fact, devoid of all moral significance; we have no duties to the nation as such (neither to our own nor to any other); our duty is only to individual men without any distinction of nationality. 
\end{itemize}
Nationalism ascribes an absolute significance to national differences and cosmopolitanism deprives it of all significance. Nevertheless, there is no fault in loving one's own country; on the contrary, patriotism has always been regarded as a virtue. And nationalism does not preclude justice for other nations.

\paragraph{Patriotism}
False patriotism can be recognized in these characteristics:

\begin{itemize}
\item \textbf{Irrational}: does harm instead of the intended good and leads nations to disaster 
\item \textbf{Vain}: based on unfounded pretensions 
\item \textbf{False}: serves as a cloak for low and selfish motives 
\end{itemize}
\paragraph{Moral and Material Goods}
Solovyov shows that the moral content of personal relationships is analogous to superpersonal relationships. To love someone means to strive to obtain for them both moral and material good, but the latter only on condition of the former. Material goods can be obtained by immoral means or used for immoral purposes. Material ends are subordinate to moral ends.

Spiritual goods, on the other hand, can never be achieved by bad means. He emphasizes the point:

\begin{quotex}
One cannot steal moral dignity, or plunder justice, of appropriate benevolence. These goods are unconditionally desirable.

\end{quotex}
Solovyov points out that these moral principles become dim when one's country is involved. Worth quoting in full:

\begin{quotex}
Everything becomes permissible in the service of its supposed interests, the purpose justifies the means, the black becomes white, falsehood is preferred to truth, violence is extolled as a virtue. Nationality here becomes the final end, the highest good and the standard of good for human activity. Such undue glorification is, however, purely illusory, and is in truth degrading to the nation. The highest human goods cannot, as we have seen, be attained by immoral means. By admitting bad means into our service of the nation and by justifying them we limit the national interest to the lower material goods which may be obtained and preserved by wrong and evil methods. This is a direct injury to the very nation we wish to serve. It means transferring the centre of gravity of the national life from the higher sphere to the lower, and serving national egoism under the guise of serving the nation.

\end{quotex}
Solovyov claims that nations prospered only when they served higher and universal ideals. Hence, the nation cannot be the final and ultimate bearer of the collective life of humanity. He then describes the inner history of various cultures and empires.

The examples he provides are for a future post.

\paragraph{Summary}
Solovyov has identified the superpersonal moral principles. Just as humans do not perfectly embody their moral ideals, neither do countries. Nevertheless, by overcoming natural hostilities, there are criteria to apply to international relations. For example, is the country pursuing a moral end, or material ends like price of oil, commodities? Moral ends may include protecting the innocent or eliminating evil ideologies. What are the moral ideals of the country? Although the claims may ultimately be shown to be false, that must not be the starting point.

\flrightit{Posted on 2022-06-29 by Cologero }
