\section{Power and Intelligence}

\label{sec:PowerIntelligence}

\begin{quotex}
The strong man strikes only when it is absolutely necessary, and then nothing stops him. \flright{\textsc{Vilfredo Pareto}}

Modern civilized man cannot endure cruelty, pain and suffering and is more merciful than men of the past, but this is not because he is morally and spiritually higher than they. He fears pain and suffering more than they did; he is more effeminate, less firm, patient and courageous than they; in other words he is spiritually less strong. \flright{\textsc{Nicolas Berdyaev}}

\end{quotex}
The sociologist and economist \textbf{Vilfredo Pareto}, through a positivist analysis of the human condition, went far in uncovering the underlying roots and causes of events. His insights are still valid today, and the most famous is probably the 80-20 rule. He makes the important distinction between logical and non-logical actions. The latter are the most important in determining the social system, since logic and rationality are beyond the capabilities of most people. If you are still not convinced, tell me what proportion of the population could pass even an elementary university course in logic or economics.

Nevertheless, there are those who insist on argumentation, believing they have the perfect syllogism to defeat all opposing views. They just get themselves frustrated or else engage it interminable discussions like a band of intellectual Highlanders, when what is necessary is decision. What follows is a brief review of Pareto's description of the role of the elites within a society.

\paragraph{Distribution of Wealth and Power}
\begin{quotex}
The curve of the distribution of wealth in western societies varies very little from one period to another. What has been called the social pyramid is, in reality a sort of upturned top. 

\end{quotex}
That is, it is like a bell curve except that the base is wide, and the slope to the narrow top is much steeper. Since that distribution is constant across time and space, Pareto concludes it is not by chance, but must be related to physiological and psychological characteristics of human beings. This is not static, since within the system, some men may grow rich while others become poorer.

Similar curves would arise from distributions of various aptitudes or moral qualities. But therein lies the root of the contempt of intellectuals and saints for wealth, since those who achieve the latter do so without excelling in those qualities of intelligence or moral virtues. The Church says that God prefers the poor and \textbf{Origen} said that the intelligent are more congenial to God than the unintelligent, whereas Calvinists regard wealth as a sign of God's election. The Church is making a good bet since there will always be the poor among you, as even Jesus admits. Politicians also benefit since they can boast about soaking the rich and benefitting the poor, while logic says that is a futile endeavor. But it keeps them in power and the logic of power is illogical.

The same diagram applies to those possessing political power with a large overlap with the wealthier elements; these he calls the aristocracy. Once again, the qualities necessary to grab and hold power are not great intelligence and moral virtues. Pareto points out this fact of ``extreme importance": 

\begin{quotex}
Aristocracies do not last. They are all subject to a more or less rapid decline. 

\end{quotex}
The good news is that the apparent stranglehold of the Modern Consensus Reality over political systems will not last, but will necessarily collapse from its own illogic. This god, too, will die. The bad news is for so-called white nationalists who have been displaced for the same reasons. Those who manage lightly read blogs do not have the qualities necessary for the will to power, although they know how to talk about it. But power is not logic, and logical and scientific reasoning will convince no one as long as they can maintain power by other means.

One of the causes of the extinction of aristocracies is their degeneracy. You have heard of the Boston Brahmins? They were the descendants of the Puritans and used to have a financial and political stranglehold over Massachusetts. Yet they put themselves out to pasture by the immigration of the Irish, and then other groups; hence the Kennedy name is now associated with Boston politics. There are no Brahmins left to complain. Some, however, have a longer memory. A Greek woman recently told me that the Greeks used to be blonde and blue-eyed until the Turks polluted them. Apparently, she resents her brown hair.

\paragraph{Circulation of the Elites}
Pareto explains that aristocracies can subsist only by the elimination of the degenerate elements and the adhesion of new ones, i.e., a process of elimination and assimilation analogous to biological systems. If this circulation fails, the animal dies. The same applies to the elite, even if it takes longer to notice. It may happen that as degeneracy increases among the elites, there is an increase in the number of elements of superior quality within the subject classes. In this case, the social equilibrium becomes unstable, and a revolution or conquest may bring a new elite into power. Pareto, writing before 1916, shows great prescience in this paragraph:

\begin{quotex}
For contemporary European societies, conquest by foreign eugenic groups has been of no significance since the last great barbarian invasions, and it no longer exists as a factor in the European social organism. But there is nothing to indicate that it cannot appear again in the future. If European societies were to model themselves on the ideal dear to the humanitarians, if they should go so far as to inhibit selection, to favour systematically the weak, the vicious, the idle, the ill adapted—the ``small and humble" as they are termed by our philanthropists—at the expense of the strong, the energetic who constitute the elite, then a new conquest by new ``barbarians' would by no means be impossible. 

\end{quotex}
This vision alone commands us to consider Pareto carefully. He claims that the new elements indispensable to the subsistence of the elites will come from the lower classes as the crucible in which are being formed, in the dark, the future elites. That made sense in 1915, when Europe was not ethnically diverse and the lower classes had more children, many of them dying young. This produced a hardier stock. However, because of technology, these classes now have many of the benefits that only the elite used to have; furthermore, the new lower classes are now ethnically distinct from the elite. As for the elite, Pareto points out:

\begin{quotex}
The high-minded people who would persuade the rich classes in our societies to have many children, the humanitarians who give no thought to replacing them by others, are working without realizing it for the enfeeblement of the race, for its degeneracy. If the rich classes in our societies had many children, it is probable they would save almost all of them, even the sickliest and the least gifted. 

\end{quotex}
Of course, nowadays all classes are having fewer children since most of them, weak and strong, now survive through medical technology and government economic aid. Pareto insists for various reasons that the elite will be replenished from the rural classes. It is hard to see that as true now, but perhaps the new elite will come from among the survivalists, seed savers, and gold hoarders if some extensive collapse should occur. The collapse of the elite is not merely hereditary. Pareto gives the example of the Catholic clergy:

\begin{quotex}
What a profound decadence this elite underwent between the ninth and eighteenth centuries! The decadence originates from the fact that the elite, in recruiting itself, chose subjects of increasingly mediocre caliber. 

\end{quotex}
Does this sound familiar? In filling its ranks today, do the ruling classes select the most intelligent, the most learned, the strongest, the creative? Seldom, unless such men can force themselves into its ranks. Rather, external characteristics are now more important to promote ``values" that are unrelated to the actually functional needs of governing.

\paragraph{The Decadence of the Aristocracy}
The simple way of looking at things is to presume an elite against the masses, although it is not so simple. Following Pareto, \textbf{A} will designate the elite, \textbf{B} the social element seeking to displace it, and \textbf{C} the rest of the population. A and B count on C for support; in fact, A despises C while B tries to be their champion. In many cases B can slowly infiltrate A, thus avoiding open conflict; in that case, there may be a period of prosperity thus obscuring that infiltration, or at least, the resistance to it.

As B attains power and there is prosperity, most will ascribe its merit to C, ``the people". Since some of the new elite may arise from C, this seems plausible to some. Nevertheless, the lower classes are always and everywhere incapable of ruling themselves. Pareto points out that many among the B

\begin{quotex}
genuinely believe that they are pursuing, not a personal advantage for themselves and their class, but an advantage for the C, and that they are simply struggling for what they call justice, liberty, humanity. 

\end{quotex}
Unfortunately for them, A is also affected by this illusion and many among them betray the interests of their class, believing they are fighting for the realization of these fine principles all to help the unfortunate C. Ironically, the sole effect of that position is that the A simply helps the B to attain power. The few among the A and B who may glimpse what is going on make accusations of hypocrisy. Nevertheless, they are sincere in their delusions, hence not hypocrites.

Only those who are scientific, objective, logical, and detached can understand the entire process. Please try this thought experiment at home: in the USA today, who are the A, the B, and the C, and how is this drama being played out? It is obvious to those who think clearly about it and I need not, and will not, spell it out. Now, I have described the way of contemplation. Those who prefer the way of action will have to become the B without, however, himself falling into the illusion that energizes it. Few men are capable of that. He will have to be hard and strong as well as scientific and detached. Otherwise, we get this as Pareto describes:

\begin{quotex}
A sign which almost invariably presages the decadence of an aristocracy is the intrusion of humanitarian feelings and of affected sentimentalizing which render the aristocracy incapable of defending its position. Violence, we should note, is not to be confused with force. Often enough one observes cases in which individuals and classes which have lost the force to maintain themselves in power make themselves more and more hated because of their outbursts of random violence. The strong man strikes only when it is absolutely necessary, and then nothing stops him. Trajan was strong, not violent: Caligula was violent, not strong. 

\end{quotex}
Pareto makes clear that social phenomena are extremely complex and this is only part of it. I will conclude with some quotes from Pareto, letting them speak for themselves.

\begin{quotex}
When a living creature loses the sentiments which, in given circumstances are necessary to it in order to maintain the struggle for life, this is a certain sign of degeneration, for the absence of these sentiments will, sooner or later, entail the extinction of the species. The living creature which shrinks from giving blow for blow and from shedding its adversary's blood thereby puts itself at the mercy of this adversary. 

Any people which has horror of blood to the point of not knowing how to defend itself will sooner or later become the prey of some bellicose people or other. 

For right or law to have reality in a society, force is necessary. Whether developed spontaneously or whether the work of a minority, law and order cannot be imposed on dissidents save by force. 

In considering successful changes of institutions, persuasion should not be contrasted with force. Persuasion is but a means for procuring force. 

An elite which is not prepared to join in battle to defend its positions is in full decadence, and all that is left to it is to give way to another elite having the virile qualities it lacks. It is pure day dreaming to imagine that the humanitarian principles it may have proclaimed will be applied to it: its vanquishers will stun it with the cry: ``Woe to the vanquished". The knife of the guillotine was being sharpened in the shadows when, at the end of the eighteenth century, the ruling classes in France were engrossed in developing their ``sensibility". This idle and frivolous society, living like a parasite off the country, discoursed at its elegant supper parties of delivering the world from superstition and of crushing ``l'Infame", all unsuspecting that it was about to be crushed. 

\end{quotex}


\flrightit{Posted on 2013-05-01 by Cologero }
