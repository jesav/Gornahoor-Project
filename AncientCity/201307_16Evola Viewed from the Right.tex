\section{Evola Viewed from the Right}

\begin{quotex}
This is the first part of a multipart review of Julius Evola's book Il fascismo visto dalla destra. It will continue sporadically and I truly hope to get to the actual book in one of the subsequent parts. In quoting the text, I will usually rely on my own translations. 

\end{quotex}

\textit{Arktos Media}\footnote{\url{http://www.arktos.com/}} recently provided an English translation of Evola's \textit{Fascism Viewed from the Right}, actually Fascism viewed in hindsight from the point of view of Tradition. To review such a work, we will accept Evola's definition:

\begin{quotex}
Ideally, the concept of a true Right, what we mean by the Right, must defined in terms of the forces and traditions that acted formatively on a group of nations, and sometimes also on supernational units, before the French Revolution, before the accession of the Third Estate and the world of the masses, and before bourgeois and industrial culture, with all its consequences and effects of actions and concordant reactions that have led to the current chaos … 

\end{quotex}
Hence, Evola fails or succeeds to the extent that he follows through on this project, and this requires that he defines the forces and traditions prior to the French Revolution. Specifically, at that time there were the three estates of the realm: the clergy, the nobility, and commoners. So we should expect to see how those estates were reconstituted during Italy's Fascist era. However, Evola adds a little twist with an import he did not consider. He adds:

\begin{quotex}
[the actions that led] to all that that threatens to destroy the little that still remains of the civilization of Europe and European prestige. 

\end{quotex}
If we take ``Europe" in the broad sense to include its extensions in North America and the South Pacific, that is simply false. European culture dominates virtually the entire globe, except for Muslim countries, from political systems, philosophy, law, science, technology, global capitalism, financial dominance, and then to sports, art, and all other manifestations of popular culture. For example, apart from the recent introduction of Asian martial arts (which unfortunately have pushed out wrestling), all Olympic sports derive from Europe. This is not the place to detail this extensively, but European culture and prestige is more dominant than ever, especially since it is freely accepted and not imposed through colonization. At one time, I used to support the Free Tibet movement as it represented a theocratic, patriarchal, hierarchical society; however, the Dalai Lama now plans to introduce European style parliamentary government to Tibet should he get the chance.

To explain the point of this digression: since European culture and prestige are not in decline, there is no sense of crisis and certainly no fear of an impending chaos among contemporary power holders. That would be true only if viewed from the Right, which regards contemporary European culture as anti-Europe, not really Europe. Those on the Left, on the other hand, consider modernity and its fruits of universal democracy, general literacy, science, technology, sexual liberation, free thoughts, to be the true Europe which cast off the ignorance, supernaturalism, oppression, racism, etc., of the previous era.

I bring this up because most today, who claim to be of the Right, seem frustrated that no one seems to notice the ``chaos" that they see. Of course, chaos cannot be ``seen", since it is not a presence, but rather the absence of order. It is that order that the Right seeks to restore, but it must be intuited and yet is hidden in broad daylight from those whose mode of being in the world is precisely to overturn the established order of things. Hence, a dialog is not possible and there is nothing left but the will to power; and we know who today has the superior will to power. This should refute the idea of any nationalism based on zoological ethnicity or ``whiteness"; rather, an appeal to the soul or spirit is necessary. The values of the modern world, hence, need to be understood as the values of a particular estate, specifically, those of the third estate of the bourgeoisie and even of the fourth estate of the undifferentiated mass, as opposed to the estates of the Priests and Warriors.

Before continuing, let us get out of the way a couple of misleading translation errors. The ``idea of force" (p. 78) is the translation of `idea-forza' which is itself the translation of the French `idée-force'. That term is taken from Georges Sorel and refers to a ``myth" that has the power to motivate men to action. Certainly it does not mean the ``idea of force" and perhaps a ``forceful idea" would be closer. The term itself means nothing in French or Italian; that is why the translations of Sorel into English leave it untranslated so that `idée-force' is the better choice.

The other misleading translation, although it is explained correctly in the notes, is the use of `thoroughbred' to translate `essere di razza'. That is certainly incorrect since thoroughbred refers to a breed of horse used for racing. Perhaps the translator meant to say `purebred'. but that is still incorrect, since the biological race is not the critical point for Evola. It is best to let it stand as `the elite is of race', understood as Evola intended. Now Evola has claimed the origin of that expression was from the Romans, but I am not enough of a classicist to know the exact source. However, Oswald Spengler uses the term, although he means race in a cultural, rather than a zoological, sense. So only the elite of a culture were truly ``of race".

In this introduction, I'll also mention some inherent defects in his approach, one because of the nature of Fascism, the other due to Evola's persistent prejudices. For the former, Fascism was not systematic, i.e., it was not derived from any ideological systematization, so there is an effort involved to extract its principles from what it was and not just from its texts. Moreover, I am not as sanguine as Evola about the possibility to cleanly separate the essential from the accidental or contingent.

To the extent that it did have an intellectual framework, that derived from the philosophy of Giovanni Gentile. Unfortunately, Julius Evola had a personal animus against Gentile, attributing ideas to Mussolini which were actually ghost-written by Gentile. Evola came of age within Fascism while Gentile was at the peak of his influence. In many ways the worldviews of the two Sicilians were similar, or at least had a family resemblance; perhaps Evola felt he need to create more distance in order to create his own sphere of influence. However, Gentile was a professor, bourgeois, and part of the high culture of Italy. These were antithetical to Evola. Although an Evolian critique of Gentile could have been of interest, I recommend instead reading this book in conjunction with the several books by James Gregor (who likewise harbored an animus against Evola) to get the full view of Fascism from the Right.


\hfill

Evola points out an obvious, yet forgotten, fact that is more applicable to the USA than to Italy or Europe in general. In the USA, what is called ``conservative" is actually ``liberal" in historical terms. Specifically, it is the political system of the Third Estate, of the bourgeoisie, of ``free enterprise" capitalism. Hence, in respect to the rule of the Priests and Warriors, it is actually on the ``left", a revolutionary idea. Only in relationship to socialism and communism can it be considered to be on the ``right", but only relatively, not absolutely or in principle. The true right, therefore, is confusing to conservatives; on the one hand, it also opposes socialism, but on the other it seems alien. That is why American conservatives are so fickle on the so-called ``social issues", sometimes attached to them, and other times, wishing they would disappear from public discourse. Hence, when pushed, the conservatives eventually side with the socialists against the true right. That is why they always lose and only claim victory when the eventually adopt more and more values of the left.

In the USA, the so-called liberals are actually socialist. Most will deny this, but keep in mind that we are describing real relationships, not nominal. Anyone who opposes not only the Priests and Warriors, but the Bourgeoisie as well in favor of formerly marginalized groups are of the left, in reality, even if not nominally. That is why the left is constantly in fear of an alliance of the Church and the Military, to which they add corporations (the symbol of the bourgeoisie), often with the prefix of ``evil". The irony is that those groups hardly oppose the left any more, in fact, they are often at the forefront of promoting leftist ideas. Rather than a direct attack, the left prefers incentives, regulations, cultural domination, and even outright infiltration to shift the remnants of the three estates of the realm to the revolutionary side; hence, they have become useless as counter-revolutionary counterweights.

Hence, Evola's project will also have to show how the Fascist system dealt with the economic domain. He points out that unified Italy, which had come into existence just a generation or two before Mussolini, was modeled on the French revolution and not any true right. Evola points out that Italy did not have an established aristocracy to represent a true right; on the other hand, there was no reign of terror as there had been in France. Evola does not point out that the right was dominated by the church at that time, which was opposed to the Italian republic.

An excellent point he made is the idea of the ``loyal opposition", which can have some merit within a parliamentary system. The purpose of the opposition party is really to keep things honest, to offer sound criticism. The ruling and opposition parties both still support the state and its transcendent principles. Compare that with contemporary systems in which the various parties agree on no principles and are constantly trying to overturn the other. Hence, there is constant agitation as one party seeks to assert its dominance over the other, not just for power's sake, but rather to alter entire moral and social structures.

Evola wants to distinguish certain principles of Fascism from its specific manifestation in Italy during the twenty years under Mussolini. He explains:

\begin{quotex}
Fascism has undergone a process of what can be called mythologizing. In regard to it, the attitude taken by most people has an emotional and irrational character, instead of a critical and intellectual one. 

\end{quotex}
That may be true, so his appeal to rationalism over emotionalism makes sense. However, in the traditional point of view, there is a perspective that is higher than the ``critical and intellectual one". Perhaps Mussolini is not a mythical figure, but the Leader was often mythologized. If we start from Guido De Giorgio's understanding, he acted quite literally as God's representative on earth. The Leader was revered as such from the Egyptian Pharaohs to Roman Empire. In some cases the Leader could even be understood as an avatar of God, such as, for example, Parashurama, who overturned the revolt of the Kshatriyas.

That is why one should take with caution Evola's assertion that ``the two planes of principle and historical contingency are absolutely distinct." Of course, they are not ``absolutely" distinct as though the plane of manifestation had an independent reality and was not itself the reflection of the spiritual plane. He means here the limited sense that the events of history cannot be understood simply as a judgment on the truth of principles; hence, Italy's defeat does not imply the invalidity of the principles of Fascism, nor the truth of the political systems of the Allies. Hence, Evola intends to evaluate Fascism in a more limited way. He is hampered in this task in that Fascism did not start from a clearly formulated doctrine, but rather it developed in an organic way over time. Evola's task, then, is to extract the main principles while ignoring any stillborn intellectual offshoots.

In chapter III, Evola says that Fascism began as a ``reaction" against the secular masonic state, which Evola describes as:

\begin{quotex}
a state that on the whole lacked a myth in the positive sense, i.e., a superior animating and formative idea that could have made of it something more than a mere structure of public administration. 

\end{quotex}
Against this, Fascism, Evola claimed, revived the idea of the state. Actually, the neo-idealism of Giovanni Gentile provided that revival, and it owed more than a little to Hegel. WWI awakened forces that ``were intolerant of bourgeois Italy". These forces were a fortiori intolerant of the socialists who had opposed interventionism, so Fascism began to define itself as a movement of the counterrevolutionary right. As such, it opposed democracy, plutocracy, freemasonry, in short, the principles of the French Revolution. It is obvious now, a half century after Evola's book, that no major movement (i.e., one with any pretense of political power) of the so-called right would adopt any of that platform and would actually be appalled by it. Hence, the universal opprobrium attribute to Fascism or to any movement that smacks of it. So when Evola tries to extract what is ``good" in Fascism, it won't be seen as universally good.

The first principle that Evola gets into is the idea of the state, which is given pre-eminence, both over the people and the nation. He quotes Mussolini:

\begin{quotex}
Without the State there is no nation. There are merely human aggregates subject to all the disintegrations which history may inflict upon them … only the State gives structure to the people … the nation is created by the State which gives the people a will and thereby a real existence. [my translation] 

\end{quotex}
Evola makes use here of the notion of hylomorphism, which has done so effectively in various contexts. Here, specifically, the ``state" is the form and the people the ``matter", in a relation analogous to that between the soul and the body. The Fascist state, therefore, is the animating and organizing force; this is the opposite of the modern minimalist liberal state which protects the liberties of the individuals who inhabit a particular geographic region.

To perform its task, the Fascist state affirmed ``authority, order, and justice", and appealed to the Roman idea (``Romanity"). Evola describes it:

\begin{quotex}
Fascism recalled the Roman idea as the supreme and specific integration of the myth of the new ``strong and organic" political organism; Roman tradition was not merely rhetoric and tinsel, but an ``idea of strength". 

\end{quotex}
Evola admired that audacious move, since it tried to build a bridge across the centuries, connecting Italy at that time to ancient Rome. Evola points out a problem in that the significance of the state differed on both sides of the bridge. Moreover, the virile ethics, severity, and discipline of the Romans may not have transferred so well to the Italians. This Evola marked as a defect and the attempts to re-assert Romanity were non-existent or ineffective. He points to the Institute of Roman Studies which was never more than mere erudition promoting the study of philology and archaeology, rather than an effective political, spiritual, and ethical force.

Evola concludes that in principle, the Fascist doctrine of the state is traditional. Evola's next task is twofold: (1) to draw out the full implications of that doctrine and (2) to point out its deviations. To recap, the state, in his view, is preeminent to both the people and the nation. He then contrasts the idea of the state to ``society", which represents the physical and vegetative side of the community. Doctrines based on society as such include theories of natural rights, contract theory, and democracy, including the people's democracies of the communists.

Opposed to this social ideal, there is the political ideal which involves transcendence, rather than the materialistic and hedonistic concerns of the society. The problem of transcendence was never satisfactorily solved by Fascism.

\paragraph{The Bovine State}
Evola points out that the ``impulse to self-transcendence can be repressed and silenced, but never completely eliminated, except in the extreme case of systematically degrading people into a bovine state." I don't think he realized how close that bovine state was. He pointed to the ``blind, anarchic, and destructive" revolts of the youth as a sign of the ennui induced by the prosperity and comfort of the consumer society. But those days are long gone as these same youths are now the possessors of power. Yet, they have maintained the revolt as their ideal and have persisted in overturning one by one every previous customary and traditional notion of order.

The destructive aspects of the constant revolt are not immediately apparent, or else they are hidden and kept out of public discourse. Furthermore, it coopts that impulse among the young, since there is nothing left for the young, since revolt is public policy. When post-menopausal women are getting tattoos, nipple rings, and bikini waxes, the youth are forced to take things to absurd levels. Blind, anarchic, and destructive revolts are left to the proletariat who have been shut out of the bourgeois consumer culture and resent it. By the logic of revolt, such public behavior is beyond criticism.

\paragraph{The Mystique of Power}
Similar to Guido De Giorgio, Evola points out that ``in traditional societies, there has always existed a certain liturgy or mystique of power and sovereignty that was an integral part of the system and which furnished a solution to the problem we have been addressing [i.e., the urge for transcendence]," but here a divergence appears. The natural objection is that the system takes on a religious significance, so that a secular state replaces religion, the normal object of transcendence. Evola denies the dualist notion that the state is secular; we can point out that idea is a modern notion. However, Evola traces its source to Jesus' command to render unto Caesar, but would preclude the existence of any such idea of the state in Europe since Constantine.

So, the state depends on a spiritual charism and is thus more than a merely administrative and social system. This brings Evola to the relationship between Catholicism, the majority religion of the Italian people, and the Fascist state. Unfortunately, Evola's distorted understanding of the proper roles of the Priests and the Warriors colors his analysis. He praises Mussolini's occasional references to the Roman ideal, which was necessarily vague. De Giorgio's larger view does not have such a problem since it has a place for the Priests in the traditional system, as the upholders of that tradition and the source of the charism of the state.

Clearly, if the existence of the Fascist state had depended on the replacement of the common religion with one that could only be partial, incomplete, and mostly forgotten, it would have been stillborn. The state, despite his claim, is not ultimate and does not have infinite capacity to form the people. Even by his logic, for the state to have a charism, then the religion necessarily precedes the state, ontologically and temporally. Nevertheless, Evola somehow has to demonstrate the spiritual foundation of the state as opposed to the secular and desacralized interpretations.

\paragraph{Nationalism}
One of the shortcomings, Evola says, is that the Fascist doctrine did not distinguish fully between nationalism and Tradition in the higher sense. He described the former as ``mediocre conservatism", bourgeois, priggish, superficially Catholic, and conformist. As the nationalist elements were absorbed into Fascism, its ideals became blurred, and ``nationalism" became identified as the ``Right". Evola traces the idea of nationalism to the principles of 1789, which were opposed to the traditional structures of the Middle Ages.

Curiously, this syncretism of nationalism, with a lip-service commitment to tradition, has become the hallmark of the so-called right in the USA. Neo-conservatism is often associated with the Jewish element, but that ignores the even stronger Catholic influence, or actually neo-Catholic, since their tradition does not reach beyond Vatican II. This nationalism accepts a large part of the modern liberal state, although it promotes the roles of the bourgeois money powers over the state's more socialist tendencies. Beyond that, the nationalist right supports a vigorous foreign policy, even to the point of being jingoist. Nevertheless, nationalism is not an exclusively rightist tendency, since even the liberal state is nationalist. Although the USA has a federal rather than a national government, there has been an increasing tendency towards the latter. At its inception, a person regarded himself as the citizen of one of the states and owed his allegiance to it, but now schoolchildren pledge their allegiance to the federal government. This nationalist feeling is demonstrated in its holidays, which are mostly associated with wars and political figures; in pre-nationalist Europe, holidays used to be associated with religious celebrations.

Obviously, the emphasis on nationalism was unacceptable to Evola, since he regarded the state as superior to the nation. I'll leave it to readers how this may apply to such contemporary movements like ``white nationalism" and various ``identitarian" initiatives, which often claim some sort of affiliation with Evola. The latter are really nationalist movements since their source of identity is a common nation, in a zoological sense. Their fault is not the desire for a unity, but the belief that that unity will arise from below, democratically as it were, rather than from above, as a spiritual ideal and the state as the organizing principle. As for ``white nationalism", it is simply an oxymoron.

\paragraph{Totalitarianism}
Evola next makes the distinction between the totalitarian, which he associates with Stalinesque type states, and the organic state of Tradition. The central authority can become degenerate when it tries to control everything, as in the Soviet Union. The Traditional system is based on spiritual values, the importance of the personality, and the hierarchical principle.

The traditional state is ``organic", ``differentiated", and ``articulated". That is, the state coordinates forces, making them a part of a larger unity; however, in themselves, they have a certain amount of liberty. The organic state depends on the voluntary loyalty of the hierarchical levels beneath it, so it does not need to maintain an iron control. Nevertheless, the state is all powerful and ignores the ``fetish of the rule of law" in cases of necessity. Of course, that can cut both ways; presumably the positive law of the state is not ``all powerful", since it must be consistent with the transcendent law.

The organic state is divided into regions and smaller subdivisions; however, these must be natural groupings and not merely administrative structures, as apparently was the case in Italy. That would be odd, considering that the different regions had different histories, cultures, languages, and even cuisine.

Evola says the essential task of the true state is to create a general climate in which

\begin{quotex}
liberty is always the fundamental factor that can take form in a way that is virtually spontaneous and which can function in the right way with a minimum of rectifying interventions. 

\end{quotex}
\paragraph{The Ethical State}
So the state has three essential roles:

\begin{enumerate}
\item It is a higher principle or power that gives form to the nation. 
\item It creates a general climate 
\item It is the principle of a new way of life 
\end{enumerate}
The obvious question, therefore, is how does the state perform these tasks? Evola claims that it is more like a catalyst; he has in mind something like this from the Tao Te Ching: `The highest type of ruler is one of whose existence the people are barely aware.' Obviously, that is hardly the case for Fascist Italy, since the state was ubiquitous and unavoidable.

In nations with a long history, such as China, roles (1) and (2) may be very natural. However, (3) is of a totally different order. In ancient China, why would there be the need for a new way of life? However, in Italy during the period in question, when it was still amorphous, (3) was necessarily the most essential role. That goal was to be achieved through the ``Ethical State", proposed by Giovanni Gentile.

Here, it seems to me, Evola reveals one of his irrational prejudices, such as described by Rene Guenon in his correspondence with Guido De Giorgio. The ethical state in this conception is a moral reality that ``realizes itself in the free and ethical will" of the citizens based on a ``religious conception of life." I don't see how this differs substantially from Evola's ideals of liberty and ``spiritual values". Nevertheless, Evola is quite critical of Gentile.

As the first minister of education under Mussolini, it was Gentile's task to reform education; so Evola's complaint about reaching into the schools make no sense. Gentile restored religious education to the schools, returned the crucifix to the classrooms. There is a free ebook from Amazon with Gentile's speech to educators; the tone is nothing like what Evola describes. In many ways, they align, since Gentile emphasized the will, loyalty, the ascetic life, and so on.

Another task of the educator is to maintain the continuity of the cultural life of the nation; for this, Gentile was ideally suited due to his expansive knowledge of Italian art, literature, philosophy, and history. The life of the soul of a nation cannot be neglected. In this regard, Evola was markedly deficient and showed little interest in art, music, literature and all other aspects of High Culture. The notable exceptions were his attachments to Futurism and Dada, both really anti-art and anti-poetry. They are hardly suitable to educate the child or to pass on culture, which is really the meaning of ``general climate". Dada is meaningless in itself and can only be defined in relation to true poetry.

The fundamental point is that if the religious authority and the ethical state do not educate the people, then they will be subject to the random influences of special interests. Evola is inconsistent on this point, both expecting the state to form the nation, but to do it from afar somehow, as if by osmosis.

\paragraph{Sexuality}
Evola objected to the ``Pronatalist Campaign", a Fascist program to increase the birth rate, on the grounds that mere quantity does not make a strong nation. This applies to the proletariat whose only contribution to society is their progeny. As birth rates increased and death rates decreased, population increased leading to increased massification. That occurs to the extent that the masses cannot be integrated into the state, or said another way, the state is unsuccessful in their ``formation". On the other hand, as we discussed in an earlier post about Pareto\footnote{See Section \ref{sec:PowerIntelligence} in this book.}, a strong peasantry keeps the people racially strong. In the years subsequent to Evola's death, which he clearly did not foresee, the decreasing birth rates of Europe is making them weaker, not stronger. This is not unlike the Spartans whose low birth rate led to their slow extinction.

Concomitant with that, Evola brings up the topic of sexual morality. Italy, at that time, prohibited divorce, abortion, pornography, even contraception, all part of the general category of ``sexual morality". Evola attributes that to the bourgeois aspect of Fascism. As the counterpoint, Evola appeals to the ``liberty of the person". If he means that what someone does in private is not the concern of the state is one thing, actually hardly worth mentioning; however, if he means the hypersexualization of society as exists in the contemporary Western states, that is quite another.

Sexual desire is undifferentiated and egalitarian, and when that is made into the standard of behavior, the hierarchical society Evola desires becomes impossible. E. Michael Jones, in his historical studies like \textit{Libido Dominandi}, demonstrates the close relationship between sexual liberation and political control. The sexualization of the West does not lead to more liberty, but precisely to its opposite. Evola seems to recognize this, but without putting it in this context:

\begin{quotex}
A people and a nation will go adrift or be reduced to a labile mass in the hands of demagogues skilled in the art of acting on the pre-personal and most primitive strata of the human being. 

\end{quotex}
Sexual desire certainly falls within that strata, so the ethical state, in its function of forming a free people, will want to channel that sexual energy to avoid the dangers expressed above. Hence, it is not necessarily the reflection of a puritanical attitude. Actually, for the proper warrior attitude that Evola praises so much, we can point to Marcus Aurelius, who characterized the sex act as the rubbing together of membranes to produce a spasm. Moreover, as we demonstrated in \textit{Family Values}\footnote{See Section \ref{sec:FamilyValues} in this book.}, the Germanics that he admired were not so ``sexually liberated", so it has nothing to do with ``bourgeois" values.

\paragraph{The Mass Mind}
Evola concludes this chapter with Plato's notion that the person without a sovereign within, will require one without. The related idea today is the distinction between the notions of ``freedom from" and ``freedom to", i.e., negative and positive liberty. When modern man has cast off the fetters, he does not know what to do, given ``the lack of direction and the absurdity of modern society". He continues:

\begin{quotex}
In truth, personality and liberty can be conceived only on the basis of the liberation of the individual from naturalistic, biological, and primitively individualistic bonds that characterize the pre-state and pre-political forms in a purely social and utilitarian-contractual sense. 

\end{quotex}
Evola makes an interesting distinction between anagogical and catagogical transcendence, the former toward the higher and the latter toward the lower. The anagogical means that the individual transcends himself, moving beyond his own personal interests.

The catagogical possibility

\begin{quotex}
happens precisely in the ``mass States", in collectivizing and demagogic movements which are fundamentally passionate and sub-rational, and can also give to the individual the illusory, momentary sensation of an exalted, intense life, moreover such sensations conditioned by a regression, by a diminution of the personality and true liberty. 

\end{quotex}
With this, Evola exposes the attraction of Leftism: by diminishing the personality in the excitement of the mass movement, the leftist feels alive and with a purpose. I need not go into detail here, but readers can find abundant examples of this in the many mass demonstrations and movements today. They are always vulgar, emotional, illogical, even when the cause may seem just. It is nearly impossible to communicate in depth with anyone who has made that catagogical descent into the mass mind.

\textbf{Julius Evola} asserts that ``a true Right without the monarchy ends up deprived of its natural centre of gravity and crystalisation", since the Crown in traditional societies is the principle reference point. Hence, in any situation in which a Right takes form, its function needs to correspond to that of the system characterized by loyalty to the crown. In the context of Italian Fascism, Evola undertakes this analysis. (For Evola's more general view on monarchy, please see \textit{The Meaning and Function of Monarchy}\footnote{See Section \ref{sec:MeaningMonarchy} in this book.}.

Even for Mussolini, the monarchy was the fundamental element of national unity. Evola brings up an interesting side point. After 1943, the Germans established the Salo Republic in northern Italy with Mussolini as its head. Since it no longer had a relationship to the monarchy, there is nothing of interest for Evola in it. Hence from this comment, we may presume that Evola's fundamental loyalty was to the Crown, and not to Fascism in itself.

For those unfamiliar with Italian history, Mussolini did not seize power, but received it from the King. In the remainder of Chapter V, Evola argues, with historical examples, for the legitimacy of a titular king while the state is actually under the management of a strong leader. We defer to his judgment in this area. He concludes with an interesting personal revelation. After the invasion of Italy by the Allies, Evola escaped to Germany, where he was spent time at Hitler's headquarters in Rastenburg, and was there when Mussolini arrived prior to establishing the Salo Republic. From Evola's perspective, that republic—detached from the kingdom—was a degeneration and a regression, based solely on Mussolini as an individual. Nevertheless, he conceded that there was some basis for those who felt loyalty to Fascism. Interestingly, Evola complained about the vulgar anti-monarchical attitude of the National Socialists, that made them more like revolutionaries than representatives of the Right.

Evola next points out that the role of Fascism, after its initial revolutionary phase, should have been the reestablishment of ``normality and unity". An obstacle was the Fascist party itself, which became a state within a state. A true state is not ruled by parties, which are restricted to democratic, parliamentary regimes. Hence, even a ``one party" state makes no sense, since it asserts that the ``part" wants to be the ``whole". Evola claims that instead of a ``party", what was need was an ``order", like the nobility, that embodied the ``idea" of the state.

Fascism could not overcome its origins as a party with democratic ideas and its desire to become a mass party. Instead of consolidating and purifying itself, making membership a ``difficult privilege", it extended membership horizontally. This allowed in superficial adherents, opportunists, and conformists. Instead, Evola wished that the creation of an Order should have been the goal after gaining power.

\paragraph{Interlude}
\begin{quotex}
There is little point to publishing or reviewing this book, unless we can extract some contemporary lessons from it. First of all, it is clear, as he often insisted, that Evola was not a Fascist since his first loyalty was to King and country. Fascism was merely a contingency that should have led to the establishment of a traditional state, as it faded away as a party in favour of a superior order. Obviously, things did not work out that way.

We also see, in the minor resurgences of most soi-disant rightist groups, often under the guise of identitarian movements, the same party attitude. This is little attention paid to ``quality", as participants are accepted willy-nilly, with wildly varying spiritual, political, ethical, and ethnic world conceptions. The hope, apparently, is that they share a common goal and the differences will somehow work out after the ``revolution". That is quite far from what Evola has described. When personal bonds have been established, it can be difficult to make the decision to exclude certain members, i.e., to purify the movement. However, sometimes there is no place for sentimentality. 

\end{quotex}
\paragraph{Spiritual Conclusion}
Curiously, Evola has left out of the discussion anything concerning the spiritual atmosphere of the country, seeing in the King alone the symbol of its unity. In Evola's conception, the \emph{dux}, or leader, receives his legitimacy from the King, or \emph{rex}. However, he neglects to point out that the legitimacy of the king himself derives from the spiritual authority.

In contrast, we can consider Maurras' solution. Although non-religious and a Positivist, he recognized that the real France was the creation of the 40 Catholic kings of France. Like Evola, he agreed that the King was the symbol of the unity and identity of the nation. As a thought experiment, you could compare how the contemporary Frenchman regards what it means to be ``French", keeping in mind the Republic's commitment to the secular state and universal human rights.

For Maurras, there was little difference between the Catholic and Positivist France, since they were, or would be, based on the natural law. Hence, the laws of the state would generally be acceptable to both groups. Since then, and considering Benoist's early adherence to Maurras and then his subsequent rejection, there has been nothing to replace it. Without a king, without a real religious tradition, without a natural law, on what basis can a new state arise? The only options are some sort of zoological camaraderie or else despair and nihilism. Yes, I suppose there is also ``wishing and hoping".

The closest example today, I believe, is in Iran. There the Supreme Leader is the religious authority, who has the power of veto and sets the general direction of affairs. The President would have the duties of running the government. This is analogous to Evola's distinction between the rex and the dux, except in this case, there is a legitimate spiritual authority. Certainly, this organization would have little appeal in the West. Nevertheless, the Pope used to play a similar role, but that was rejected over time, often in very bloody ways.


\hfill

\textbf{Julius Evola} concludes Chapter VI of Fascism Viewed from the Right with the criticism that the Party was not selective in terms of who was admitted into it. In Chapter VII, he deals with another problem with the system, that of the cult of the leader. First of all, in some ways it is too democratic for Evola since it depends on the ``applause of the people".

Rather, the traditional state relies on an anagogic, or mystical, climate. On the contrary, the Fascist system depends on stirring up the lower parts of the human soul, not unlike a criticism I made recently. He explains (this translation differs slightly from the English text):

\begin{quotex}
Such a [anagogic] climate cannot be obtained with an enthusiasm that can, in certain cases, reach the level of fanaticism and collective passion, although based on the sub-personal aspects of man as mass-man and on the art of firing him up against every other possible form of individual reaction. 

\end{quotex}
Despite the intensity of such feelings, it has only an ephemeral character which differs profoundly from the formative force from above of a true tradition. That is why things in Italy and Germany could unravel so quickly. Evola points out, too, that the democratic nations also rely on propaganda, demagogy, and the fabrication of public opinion. However, he does not explain why the liberal democracies have been able to last longer than the ``ephemeral" axis states of Europe.

In the traditional state, the leader has a natural authority rather than the formless authority of the dictator who depends on arousing the emotional and irrational elements of men. The leader is almost of a different nature from other men. Perhaps he meant here a ``supernatural" authority. Curiously, he points to recent popes who have come down to the level of the people. Here, Evola was referring to Pope John XXIII, not Paul VI as in the notes. Of course, since then, popes have dropped more and more of the regalia and ceremonies that used to set them apart and have opted instead to identify with the ``people". Some commentators make much of this; for example, they claim that more people saw Pope Francis in Brazil than saw the Rolling Stones. Unfortunately, they haven't noticed that the youths at World Youth Day live more like the Rolling Stones than like the pope.

Mussolini believed he was one of the great individuals who would arise, according to Oswald Spengler, at the ``Decline of the West" (the translation oddly says ``Sunset of the West"). The corrective to this situation would require a superior tradition and a different chrism. Yes, that is the point, but Evola does not offer us much. What is the tradition? Which spiritual authority would administer the chrism? These are the issues that need to be discussed at the present time.

Next, Evola addresses the military component of Fascism. Mussolini tried to instill a warrior ethos in the Italian people based on the virtues of obedience, sacrifice, discipline, and dedication. Clearly that is a fool's errand, since you can't ``instill" such qualities en masse. He opposed the bourgeois spirit. This Evola can agree with and he points out that the army and the monarchy have always constituted the essential pillars of the true state before the revolutions of the Third Estate. He left out the spiritual authority as one of the ``essential pillars".

Evola concludes this discussion with the idea of public service as honour. To accentuate this, public officials wore uniforms. Here, also, he gets a dig in against Giovanni Gentile's ideal of the ethical state. Gentile did not write anything much different from the Duce. He, too, advocated discipline, sacrifice, courage, and so on. As the education minister, he had to develop a school program to teach such virtues. Otherwise, by what means could they be ``instilled" in the people? By way of contrast, apparently, Evola describes his own set of values which sound pretty good to certain types, although most people would consider them cold and even inhuman.

\begin{quotex}
They include in the first place discipline, the sense of honour, active impersonality, relationships of responsibility, command, and obedience [not ``responsible relationships …"], dislike of small talk and ``discussions", a virile solidarity with true freedom as its starting point — freedom for doing something that is worth the effort and that carries you beyond the stagnation of bourgeois, ``prosperous", and vegetative existence. 

\end{quotex}

Fascism, as the conscious agent of the state, next had to confront the parliamentary system that it had inherited. Evola put it this way (in Chapter VIII):

\begin{quotex}
The parliamentary system had sunk to a level where the politician had been replaced by the party hack, where everyone could see a system of incompetence, corruption and irresponsibility. 
\end{quotex}

Since parties could replace each other through elections, there could be no stability to the state. Evola names three principles of the problem:

\begin{itemize}
\item The electoral principle in general
\item The representative principle
\item The political principle of hierarchy
\end{itemize}

Evola is quite caustic about the principle representation which is based on universal suffrage and the principle of “one man, one vote.” As such, it is based merely on quantity and the individual as undifferentiated mass man. Evola opposes this individual to the “person” who has “a specific dignity, a unique quality, and differentiated traits”. Interestingly, Evola provides examples in “a great thinker, a prince of the Church, an eminent jurist or sociologist, the commander of an army”, etc.

Such persons should have the public interest in mind, whereas the individual has only personal interests or, at best, the interests of his group. Since in modern democracies, the transcendent is moved out of the public sphere and can be no more than a personal interest. Hence, the appeal is always made to economic interests. In the democratic system, the battle of parties, each of which aims for the conquest of power, leads to a “totally chaotic and inorganic situation.”

In theory, the democratic system sounds fair, since open and free “debate” should lead to the “best” policy. Ultimately, however, quantity always dictates policy regardless of the quality of the debate. This is the representative principle based on egalitarianism, individualism, and quantity.

However, Evola points out that in a traditional society there is a different representative system based not on individuals, but rather on the different bodies or functions within it. Specifically, these would comprise the “corporations” (more like a trade association than a business venture), the nobility, the scholars, and the army. Evola glaringly omits the “Church” or the spiritual authority.

What distinguishes this system is the hierarchical principle which overrides any decision based purely on the quantity of votes. For example, the spiritual authority would demand protection for its role in performing rites and offering education. This is not a “theocracy”, i.e., the rule of priests, since rule is still reserved to the aristocracy or warrior caste. The producing caste rules itself, subject to the warrior and spiritual authorities, through the various corporations. In the principle of subsidiarity, decision making is made as diffuse as reasonable and effective. Of course, Evola does not go into all the details here, since his primary interest is in how closely Fascism adhered to this system.

Like the traditional system, the Fascist system also relied on corporations, which replaced political parties. Instead of endless debate (note the influence of Donoso Cortes here), ideally the corporations should arrive at decisions through coordinated labor as well as technical and objective criteria. In this, Fascism was more or less successful. Italy did, however, emerge from basically third world status when Mussolini assumed control to somewhat of an industrial power twenty years later. The activity of the producing class is ignored by the “traditional” authors, but, despite Evola, it is legitimate for the state to take an interest in questions of the economy and labor relations.

Evola next criticizes the so-called “ethical state”, a coinage of Giovanni Gentile, regarding it as “the dross, the unessential, and invalid part of Fascism.” That is absurd, as Evola lets his personal opinions intrude, since, for better or worse, Gentile was the philosopher of Fascism. The goal of the State was to create the nation, thereby giving the people a will and effective existence. How else does the State do that except through education? Gentile uses the term “ethical”, not as schoolmarmish moralism, but in the philosophical sense of relating to the Will. Hence, education is necessary and of high priority.

Evola was probably objecting to the role of the Church in education. As Minister of Education, Gentile returned the crucifix to the classroom and made religious education part of the curriculum. Yet one of the roles of the spiritual authority is indeed education, so there is nothing at all untraditional about that. Evola often complains about the lack of a full spiritual content in Fascism, because of its relationship to the “dominant religion”, but it is certainly not the place of the temporal power to fabricate religions out of the air.

\flrightit{Posted on 2013-07-16 by Cologero }

\begin{center}* * *\end{center}

\begin{footnotesize}\begin{sffamily}



\texttt{Avery Morrow on 2013-07-23 at 10:27 said: }

``Without the State there is no nation."

This is precisely the inverse of the truth, and if that sounds like a political or loaded statement, try knocking it down a few levels: ``Without City Hall there is no city," or ``Without the Bureau of Indian Affairs there are no Indians." Evola shouldn't have been surprised that Fascism could not engineer itself into being ``organic". Ancient Rome never needed an Institute of Roman Studies.


\hfill

\texttt{Saladin on 2013-07-23 at 13:39 said: }

I agree with you Avery! The only point of contention that I have always had with the Maestro has been his total defense of the State. Granted, his state is more nuanced than what you have described, the Hegalian influence on Evola was not entirely wholesome. Here I side with Nietzsche when in Thus Spake Zarathustra he likens it to a monster.


\hfill

\texttt{Logres on 2013-07-23 at 17:53 said: }

I think he means something more subtle, along the lines of the symbol: ``the eagle is not a piece of metal – the eagle IS Rome". 

``We believe that we invent symbols. The truth is that they invent us; we are their creatures, shaped by their hard, defining edges. When soldiers take their oath they are given a coin, an asimi stamped with the profile of the Autarch. Their acceptance of that coin is their acceptance of the special duties and burdens of military life—they are soldiers from that moment, though they may know nothing of the management of arms. I did not know that then, but it is a profound mistake to believe that we must know of such things to be influenced by them, and in fact to believe so is to believe in the most debased and superstitious kind of magic. The would-be sorcerer alone has faith in the efficacy of pure knowledge; rational people know that things act of themselves or not at all."

? Gene Wolfe, Shadow and Claw 

I agree that in a sense, the Nation brings forth the State. But in another sense (and I think this is the meaning intended by Evola, although I could be wrong), it is the Regime or State and ultimately the Capo or Head who gives a final definitive cast to the political body.


\hfill

\texttt{Cologero on 2013-07-23 at 20:33 said: }

Avery, you are confusing the State with a government. The examples you provide are bureaucratic agencies, mere artifacts, not something natural and organic.

Evola claims the organic state is actually the soul of the nation, intimately connected to the people, of which it is the organizing principle, its consciousness. The state, in this conception, does not exist in a vacuum, but only in relation to the ``matter", i.e., the people. The state, then, is the conscious manifestation of the not yet fully conscious of its identity. That does not mean it lacks an identity.

The modern idea is along the lines of Rousseau's Social Contract. Individuals get together to create a government, regulated by a contract or constitution, that is supposed to safeguard each of their interests. That is not a nation, but rather a congeries of competing interests. Of course, the actual individual is helpless in that situation, so various power blocs emerge further fractioning the government which is forced to expend all its efforts in appeasing the various factions.

The situation Mussolini encountered was a people who only knew a the liberal social contract type of state. That was the matter he needed to work with, mold, and shape; in other words, the historically contingent situation. So, his task was more than the situation in a truly organic society which did not know such a conception. Hence, there was an educative aspect necessary, for which he invoked the idea of Romanity (a word which the English translation avoids using, for no good reason. The Romans did not need an institute because they did not need to be educated, they already embodied those qualities.

This is one of Evola's main points. The historical failure of Fascism was not because of the falsity of its principles, but rather to the contingencies of the time.

As we mentioned in the review, those principles are anathema to the modern mind, which can see in them only totalitarianism and stultification, rather than the call to live life on a superior, virile level. Individuals today, however, prefer to choose their own paths to fulfillment even if the almost always fall short. As we showed in \textit{Kidz, Kulcha, Kreativity}\footnote{\url{http://www.gornahoor.net/?p=1582}}, even the best and the brightest seldom find their way beyond animal existence.


\hfill

\texttt{Cologero on 2013-07-24 at 05:54 said: }

Saladin, I don't believe that Evola was particularly fond of Hegel. Actually, he gave a classical, not an Hegelian, justification for the state as the soul of a nation. If Evola's view of the state is not pleasing to you, then what would you put in its place as the consciousness of the nation? Or do you reject the concept in toto and see nothing but an aggregate of individuals? If not, how is that aggregate organized?

I should also point out that Evola referred to Guenon as the Maestro, so it is preferable to reserve that term for him.


\hfill

\texttt{Saladin on 2013-07-24 at 11:21 said: }

Thanks for your clarification Cologero!

You maybe right in saying that Evola's justification for the State was Classical and not Hegelian, but Evola himself mentions that Hegel was one of his earlier influences.

I do not reject the State in toto (I don't think any Traditionalist would) but to say that ``Without the State there is no nation", as Avery said, is an inversion of the normal order of things. Again, I have to renew my studies of Evola's writings on this topic, but in your reply to Avery, you make the case that Evola's use of the term is quite different from it's modern connotations.

Taking my tradition as an example, ss far as I can tell (and I am not an expert in Islamic history, even though I grew up Iran) in Islam the ``state" is subordinate and secondary to the ``umma". That is to say: the nation (or the folk) is the soul of the state and not the other way around.


\hfill

\texttt{Cologero on 2013-07-24 at 12:39 said: }

Now you are getting to an interesting point, Saladin, which exposes a contradiction in Evola (and also the Fascist state). I was eventually going to get to this, but here is suitable as a starting point.

Forgive me for venturing beyond my competence, but I believe the following to be true. The umma is not really an aggregate of individuals, but rather of a community of persons united in Islam. Correct? So the state expresses the will of the umma and speaks for it in temporal affairs. Nevertheless, the state is subordinate to Islam. This is more in line with how Guido De Giorgio understood the traditional state. Iran, moreover, even has a Supreme Leader, as De Giorgio suggested, and his role is certainly higher than Il Duce Mussolini since it encompasses spiritual leadership.

What Evola wants, and this was true even of his philosophical foil Giovanni Gentile, is for the state to be absolute. That is why Evola had so much trouble with the Catholic church's role in the state. This is an extension of Evola's failure to resolve consistently how he saw the relation between the Priests and the Warriors. Hence, the Italian state as such could not create the spirit of the people as much as Evola expected it. The people were united in their religious tradition, and the state grudgingly had to come to terms with it.


\hfill

\texttt{JA on 2013-07-24 at 13:49 said: }

Cologero brought up the Iranian leader who is a fine example of the Capo.

Is there not though an inherent problem within Catholicism that an absolute state could never be created because the Pope could never tolerate a Duce having temporal authority over the priests (as de Giorgio's Capo does) ? and unlike Islam the Pope is not a temporal ruler as the Caliph or Imam Khomeini was so we are stuck with a dualism at the top – the same dualism that destroyed Frederick II's attempt to build a new empire in the XIIIth century.


\hfill

\texttt{Saladin on 2013-07-24 at 14:56 said: }

Interesting point indeed Cologero! I agree with your assessment of the umma and you are right that it is inline with De Giorgio's understanding of the traditional state.

However, even though we might disagree with Evola's definition, I am still not clear how his is a contradiction as you say?


\hfill

\texttt{Ash on 2013-08-02 at 01:05 said: }

Taking the example of Iran, I would tend to see it as more like medieval Europe in terms of the division of authority. The religious lead is taken by the Ayatollahs of the Guardian Council, but the political Leader in terms of every rule seems to be the President (previously Ahmadinejad, now Rohani). Does the revolutionary Guard fulfill the role of Kshatriya, if in an imperfect form? 

On the State, perhaps we should watch our definitions. Do we mean the bodies which govern the country, or the foundation on which they are based? The Romans lived under several forms of government, from monarchy to Republic to various forms of Imperial rule. So what was the constant in each era which defined the State and people? Even Roman values like severity and simplicity changed throughout the years. The only uniting principle was Rome the idea, even as Rome referred first to a city and then to an Imperium. Rome did not fall when Alaric sacked the city, but when that idea ceased to define its civilization (and that in itself is hard to place. Secessions took place several times…look up the Empire of the Gauls). So then the nation is the ``folk", the people and culture in its diversity and unity; the State in Traditional terms comes to be when the ruling elite among a people uses Traditional principles as their standard. The communist State, alternatively, uses communist ideology as its standard and guide. So the State is definable as when a ruling body has principles to guide and judge themselves, and are not just ruling for personal gain. 

To use Dugin's analysis, fascism was defined by taking the State in and of itself as its foundation (an observation which has problems, but useful nonetheless). This is why it tended, like communism, toward totalitarianism. One of its errors was to abandon the idea of the State as needing to embody the Idea. Others never intended this at all and simply used religious and nationalistic rhetoric to attain power. A State without such a foundation is a ruling gang. And indeed, most States are gangs which are molded over time to higher purposes. Rome was once, after all, a band of tribes just trying to survive on the Italian peninsula.


\hfill

\texttt{JA on 2013-08-08 at 10:35 said: }

The question about deGiorgio I was really to get at but failed to enunciate properly is – does the Capo have the power to determine spiritual dogmas ? In other words can Mussolini decide we should go to Mass on Monday instead of Sunday ? 

The problem I'm having in understanding Evola's idea of the absolute state and his opposition to the Catholic Church as a separate entity not part of the state is – how is his idea any different from England or any other Protestant state where the king rules over the church ?


\hfill

\texttt{Cologero on 2013-08-08 at 20:53 said: }

I believe De Giorgio is very clear that the Priests are the upholders of the deposit of Tradition and the Warriors and, a fortiori the Capo, have the duty of protecting them in that task. Evola's notion is indeed vague, and it stems from his misconceptions as revealed by Coomaraswamy and Guenon. I don't think he ever was clear about his spiritual vision, particularly as it might apply to the contemplative life, as he was always more focused on the way of action as a spiritual path. Certainly, the heroic path and the discipline associated with loyalty, etc., can lead to a certain transcendence, but that cannot be the whole story. In particular, they can't pass on the tradition because they lack the discursive knowledge of it, as we pointed out a couple of months ago. The role of the priests in Evola's conception was like an appendix, i.e., a vestigial organ with no real purpose.


\hfill

\texttt{JA on 2013-08-09 at 10:24 said: }

I find very interesting Fabre d'Olivet's point about mediaeval Japan in his history book that it was a deviation – a revolt of the warriors – from the priest-ruled prior era. The Emperor of Japan was and is actually more akin to the Pope in our culture than to a king. The Shogun would be more like the Emperor.

I also ind it puzzling how writers such as Dugin, taking their cue from Evola's regal supremacy promote Orthodox Christianity in Byzantium and Russia as embodying the Traditional ideal of church-state relations when as I mentioned before de Maistre, Soloviev and others show how the post-Photius eastern church was the same as protestant churches like the Anglicans. 

This speech shows us the way to a good society, I believe 

\url{http://www.traditioninaction.org/History/A04CharlemagneSpeech.html}

PS – this is tangental but were you aware of what Dr Plinio had said about the Nouvelle Droite ? his opinion is that the new right is a false flag resistance created by the revolution, to deceive and give a false hope to those who are of anti-communist mind, and convert them to an anti-Christian philosophy. I agree with Plinio because I find that especially with Dugin's people, it starts of talking bout order, hierarchy, tradition, but when you get in deep there's Crowley, chaos magic, nihilism and communism.


\hfill

\texttt{Ash on 2013-08-10 at 04:06 said: }

JA, I think false flag may be giving a bit too much credit. The Nouvelle Droite and the Identitarian movements which are following it aren't ``false flags" but legitimate expressions of dissent, in my view. That said, they are in dire need of education in proper doctrine. As students of Tradition, I think it would be far more worthwhile to focus on this, at least for the time being. This is how I myself came to return to metaphysical teaching after a journey through the ideas of the Nouvelle Droite and its associates. In time, we may see elements of the Anti-Tradition…I personally have my own ideas about who and what parts of these rightist movements are such manifestations. But only in time. For every Varg Vikernes, runic/chaos magician, and racial ultranationalist, there are many who have found their course their because of their alienation with the order around them and may yet be open to the truth. Could you link to Dr. Plinio's comments? I would be interested in seeing them in full.

As an aside, do you find TIA to be a useful website? I used to read it more myself but I was always rather put off by its rather extreme dogmatism. Forget more Catholic than the Pope, I got the sense sometimes of more Catholic than the SSPX. I'm no bleeding heart myself and more judgmental sometimes than I should be, but an article entitled ``Four Ways to Discern A Man's Soul From His Appearance" which marks his laugh and walking speed as signs? I know body language tells a lot, but come on now.


\hfill

\texttt{JA on 2013-08-10 at 12:11 said: }

The Plinio quote comes from \url{http://www.kelebekler.com/cesnur/storia/gb20.htm}

``A `cocktail' of evolutionism, neo-positivism, scientism, sexual revolution and clearly Masonic doctrines in an `Indo-European' package: in the first place in order to subtly corrupt those young people who escape from social-communist and progressive conformity, favouring their transformation into `anonymous revolutionaries'. in the second place, in order to prepare the pollution of any anti-Communist reaction and to try to satisfy its inevitable spiritual needs in an anti-Catholic and anti-metaphysical sense, in view of a dark and fatal neo-pagan mirage". 

The whole article I linked to is worth reading as it shows the Masonic connections of many important NR leaders such as Christian Bouchet (who is very close to Dugin). More and more I am convinced that Dugin-ism, or a successor ideology that will be derived from Dugin's work, is the Counter-Tradition Guenon said would come in The Reign of Quantity and the Signs of the Times – a communism mixed with false metaphysics .

I used to be a national bolshevik so I understand what you're saying Ash, if it hadn't been for Dugin I'd have never read Evola or Guenon – Dugin being the bridge for me from Marx and Lenin to Tradition. But for others Dugin has become a god-like figure who can never be questioned………

I like TIA alot, I find it defending true mediaeval Catholicism, something that is a very rare fact in this world. Maybe they aren't 100 \% correct all the time but who is ? I'd go to Mass with them if I knew them personally.

For the sake of honesty though my thoughts are moving more towards Plinio/Lefebrve/Williamson/ and away from Evola, Guenon, those types, not that I disagree with Guenon or Evola so much as I just think traditional Catholicism is more on the money and less screwed up than traditionalism. I mean I'm not a liberal and if I'm going to be a Catholic which is what God chose for me to be, I'm not going to say choice of religion is meaningless and can be switched.


\hfill

\texttt{Ash on 2013-08-10 at 17:48 said: }

In my view, TIA is a site which provides some good resources but could be so much more. I thought when reading them at first that it was a satirical site mocking the ``backward Christian". The site seems to be pure reaction in many ways. It is one thing to understand and promote the Catholic social teaching on modern issues, it is quite another to be shocked and outraged at rock music and bikinis as if they are a new fad and haven't been around a good 60 years or so. In some ways they read like a Catholic Jack Chick comic, focusing on the decline of handwriting and titles like Mr./Mrs, not really mentioning that for most of Catholic civilization neither of these things existed in their current form. Perhaps I'm being a bit harsh (and off topic from the post) but I really do think that ultimately the group undermines its own project. 

Most certainly some interesting information in that article. Guenon himself was of course acquainted with Masonry in his younger years so we should of course leave open the possibility for maturing beyond petty ``spiritism", but there is cause to be weary. I take Dugin for what he is: a man who has certainly been influenced by authors of Tradition but is certainly no initiate himself. As such there is good and bad to be found in him. I have neither a natbol background (although I have read some of their writings online) nor do I speak Russian, so perhaps you can judge better than I what he is at his core. I've heard the remarks that he does not take well to dissent before…a true Guru knows that the truth can stand any test, and a teacher who does not allow for such likely has something to hide. Time will tell, I suppose. I studied Catholicism for a long time as a potential neophyte, starting when I read Scott Hahn's Rome Sweet Home and eventually becoming interested in more traditional authors like Lefebrve, before moving on to Orthodoxy and eventually away from the Christian tradition. For my part, although Christian symbolism and teaching still are great sources of learning for me, it is my philosophical issues with anthropomorphic theism keeping me from it as an exoteric tradition.


\hfill

\texttt{Cologero on 2013-08-10 at 18:43 said: }

Ash, you keep mentioning ``anthropomorphic theism" as though you know what you are talking about. I think Gornahoor has dealt with that issue sufficiently, but if you have some particular objection, the comment belongs to the relevant post. If that is still not enough, check out Edward Feser's site and ask mention your ``philosophical issues" there. You will be tarred, feathered, and ridden out of town on a rail.


\hfill

\texttt{JA on 2013-08-12 at 10:14 said: }

Personally if someone can find enlightenment through a non-Christian path as Guenon did, it is for Christ and nor for me to judge. I simply choose not to apostasise from the religion I was raised in as a child, I have little but the little I do have in the form of the beliefs of my ancestors are worthwhile enough for me to keep. I've studied other paths though, Shia Islam, Korean Buddhism, Thelema, but found they were foreign to my soul. 

I first got into Nazbol as a punk, although now he's moved away from that, for a while Dugin – when associated with Limonov (read Sedgwick's book for the history) was appealing to the punk-Crowley set for recruits so that's where they hooked me. Thanks be to God though I've moved far away from that way of life. 

One thing I've learned about initiation recently is that – people keep looking for all these little groups of hoaxsters for the ``secret knowledge" that can enlighten them but they're missing the point – initiation is not a ceremony, initiation is the spiritual change in the interior of a person. Only I can change myself. A guru can help but at the end I have to do the work – he can't do it for me. In that sense all initiations are self initiations. It helps to have a teacher obviously but since I can't find one , that's no excuse not to start working by myself. Now I've focusing my search for books of spiritual exercises I can do to raise my consciousness.


\hfill

\texttt{Ash on 2013-08-13 at 23:55 said: }

Evola's comments on the Germanic society in Revolt provide a good backdrop for the comments in this review about ``the sovereign within" and the notion of freely-given loyalty versus totalitarian control. He describes how the free men would rally around the chieftain, and how the nobles would be defined by their ``free" status. He gives a quote which would certainly apply in the situations of totalitarian states: ``The supreme nobility…does not consist in being a master of slaves, but in being a lord of free men, who loves freedom even in those who serve him." This freedom is both inner and outer, of course. Thus the positive liberty which we find in Maurras also animates the economic ideal of autarky; only a self-reliant nation is a free nation. 

Insofar as the ethical state must find healthy channels for chaotic currents, it may be useful for the Traditionally minded to examine not just the particular public and educational programs for sexual morality, but also their results. Spain, which practiced in the 20th century a hardline Catholic program of public morality, also had a very promiscuous sexual culture in some ways, egged on by the attraction which forbidden fruit holds. Writers would report the high levels of pregnancy outside marriage and the free sexual relationships which were common enough in rural areas. Does that sound like the ends that such a program was pursuing? A good modern equivalent may be the high levels of teenage pregnancy and pornography consumption in ``red" states and politically conservative countries like India and states in the Arab and Islamic world. More ``progressive" programs have in fact resulted in lower health risks and pregnancies, but are of course based on values we find abhorrent. One hopes there may be some third alternative. 

This article has great value in allowing us to see what a Traditional society and State actually look like, by comparison to the excesses and errors of fascism. The manifestation of Tradition in culture (in our day, both folk and even media culture) is something which has been dominated by the New Right (which has had great advances in this area) and was far too overlooked by the intellectual heavyweights Guenon, Evola, and co. I'll definitely peruse Gentile's work on the issue.


\hfill

\texttt{Matt on 2013-08-14 at 18:48 said: }

``One hopes there may be some third alternative."

I think Cologero was probably hinting at that in the post with the Marcus Aurelius quote. Both the puritanical mind and the modern/post-modern mystify sex for different reasons – and neither award it the symbolic value it has that opens up higher possibilities for the select few among humanity. A possible viable education program with influence from the Emperor's teachings would try and demystify the act to the masses; its neither in principle an abominable act in itself (the puritan view) nor is it the path that will lead to the greatest of human happiness and that can adequately satisfy all your desires (the modern/post-modern view). Its all just the rubbing together of membranes (to paraphrase Aurelius) with spasms and grunts where if you actually saw yourself from a third-person distance while engaging in it, you would laugh (I think that was a view stated by a certain renaissance figure might have been Da Vinci). There are more important goals and actions to spend your life engaging in. Teaching the symbolic value of the sex act should be reserved to the few who can understand it and then correctly implement it. Exposing the average individual to this is a waste of time; the message/point completely goes over their head. Miguel Serrano did an excellent job explaining these issues – Gornahoor has a post or two on Serrano's explanation in fact.


\hfill

\texttt{Ash on 2013-10-24 at 17:54 said: }

I've seen Pope Francis praised by certain traditional quarters as an ``imaginative conservative" and heard whispered warnings from others. Many of my progressive, even atheist, friends are very supportive of him. It appears they need reminding that, no, the Church does not now in fact allow abortion and homosexuality. That said, His Holiness' focus on the virtues of charity, faith, and hope, if paired with a cleansing of the Church's own house may well yield much fruit. The Pope Emeritus' vision of a smaller but doctrinally purer Church was constantly marred by his inability to effectively combat the problems which the Church had faced and restore the trust of large numbers of the faithful in it. Do those of our number who are Catholics feel that this Pope will help or hinder the Tradition where it is still present within the exoteric body of the Church? Forgive me if this would be a better topic for the forum or another thread. 

On another note, I've pondered the ``inhumanity" of some of the qualities Evola promotes before, particularly when reading the chapter of Revolt on Man and Woman. It is hard to reconcile the radical detachment which his Traditional Man has from the woman, to the point where even the emotions of affection or love are inherently feminine and have no place in the psyche of the true Man, with the image of the Family or the values of faith, hope, and charity preached by the Saints. Perhaps Gentile would be more useful on some of these ``worldly" manifestations of Traditional principles, or at least in how to live and teach them.


\hfill

\texttt{scardanelli on 2013-10-25 at 08:34 said: }

I think the comment was made somewhere here on Gornahoor that Evola's insistence on masculinity as an absolute ideal rather than a relative quality was somewhat misguided. Perhaps this contributes to these ideals relating to a man's relationship with woman. As Tomberg explains, man has a feminine or receptive soul whereas woman has a masculine or active soul. ``In the lord, neither is woman independent of man, nor is man independent of woman." Of course, Evola was not very keen on Christianity…

\hfill

\texttt{JA on 2013-12-19 at 10:17 said:}

In my opinion despite its flaws Fascism was the best that could be done and can be done now – we can’t simply revert to l’ancien regime as de Maistre hoped for because the people having been roused up by the Illuminati-Masonic-philosopher conspiracy to demand “power” are not willing to give it up easily and force can not solve the problem – a true monarchy is based on the lord-vassal relationship, anything other than that is a dictatorship but dictators are sometimes needed (Cincinnatus from Roman history for one), Fascism offered an opportunity within the modern world to begin the process of healing – it was not an ideal system but it was a start. Rather than criticise fascism the task of the Italian man of reason I think should have been to work within the state (to which he owes loyalty to from birth) to improve it.

Gentile I admire as well as a man who working with modern philosophy managed to re-express ancient principles. It doesn’t concern me what Evola’s personal gripes were with the man, but I don’t think Gentile should be quickly rejected; his education book for one we can still take a lot from.

I am not a fascist but I have no problem saying I prefer fascism to democracy and think fascism can be a prelude to monarchial restoration as per Franco and Juan Carlos.

\hfill

\texttt{Cologero on 2013-12-19 at 14:24 said:}

JA, James Gregor makes a similar case as yours, at least for developing nations. He considers the Fascist model to be better than the pure Marxist models usually adopted.

\hfill

\end{sffamily}\end{footnotesize}
