\section{The Great Migration}

Due to the cooling of the arctic region, the Boreans were compelled to migrate southward. This probably accounts for the legends of wars against the Meridionals. \textbf{B. G. Tilak}, in his book \emph{The Arctic Home of the Vedas} estimated this to have occurred around 8000 to 5000 BC.

\paragraph{Vedas}
Tilak bases his theory on the internal evidence of the Vedas, the sacred books of Hinduism. In particular, Tilak noticed in them astronomical phenomena that are compatible with life in the Arctic. There are four Vedas written as long poems in an archaic form of Sanskrit; they are considered to be revealed scripture, and all orthodox systems of Hindu philosophy are based on them, or the Upanishads, an extension to the Vedas. Tilak considers their origin to be beyond time, hence he traces it to the Hyperboreans. Tilak writes:

\begin{quotex}
There is no reason to doubt either the competency or the trustworthiness of the Vedic bards to execute what they considered to be their scared task or duty, viz., that of preserving and transmitting, for the benefit of future generations, the religious knowledge they had inherited from their forefathers. … what is achieved in more recent times can certainly be held to have been done by the older bards in times when the traditions about the Arctic home and religion were still fresh in their mind.

\end{quotex}
\paragraph{Poetic Consciousness}
As was previously mentioned, caste structure was being established during this period. The tradition embodied in the Vedas was entrusted to the Rishis, which means ``seer". The rishis were still able to remain in direct communion with the supernature. Tilak writes:

\begin{quotex}
We may safely assert that the religion of the primeval Arctic home was correctly preserved in the form of traditions by the disciplined memory of the Rishis.

\end{quotex}
Nevertheless, the lower castes still retained features of the Primordial State. In particular, their consciousness was still motivated by poetry, rhythm, and command, rather than discursive thought. This left them with the power of concentration and great energy, since they were free of the energy-sapping phenomena of boredom, anxiety, daydreaming, and so on. Tilak explains:

\begin{quotex}
The hymns were public sung and recited and the whole community, which must be supposed to have been interested in preserving its ancient religious rites and worship, must have keenly watched the utterances of these Rishis.

\end{quotex}
Now these recitals would have gone on for hours, if not days during certain festivals. Besides the Vedas, there would have been legends and myths, some of which were probably adapted to the Mahabharata and Ramayana. Vedic Sanskrit was a syntactically complex and semantically rich language; because of its difficulty, it is accessible in the West only to a small pool of the highly intelligent and educated. Yet, during the Vedic era, the masses were in rapt attention to these recitals.

We can see the decline in later periods. The Cistercian monks chant 30 psalms per day, completing the cycle of all 150 psalms in 5 days. St Bernard said that was in concession to human weakness, since all 150 psalms should be chanted every day.

Shakespearean plays were intended for popular audiences, yet only high-brows will pay to see one today. Even as late as 19th century England, certain poets were ``rock stars"; people would wait in anticipation for a new poem from Tennyson.

In our time, the closest we have is a rock festival. Yet, sensory overload in the form of sex, drugs, loud noise and light shows are required to hold the attention of an audience. Even at that, the appeal is always to animalistic and emotional states, not to awaken higher intellectual states as in the Vedic recitals.

\paragraph{The Warrior Caste}
Nevertheless, as time goes by, the issue of castes is altering consciousness. Herman separated the venerable men and selected the warriors. For this system to persist, the question of the membership in the castes would arise again. In better times, when men were free of personal ambition and were in organic relationships, the question did not arise. Over time, the rishis were no longer seers, but became mere scribes, repeating the old hymns and texts by rote.

In the Mahabharata, the problem of membership in the warrior caste is revealed. \textbf{Ekalavya} was a man of low caste who, nevertheless, trained himself to be the best archer in the land. When he approached the martial arts teacher, Drona, about being admitted to his ashram, Drona ordered him to cut off his thumb. Of course, without his thumb, Ekalavya could no longer draw his bow. Another element is that Ekalavya was of mixed race, which was probably the reason he was excluded.

Thus we come to the end of the second age.



\flrightit{Posted on 2011-01-13 by Cologero }

\begin{center}* * *\end{center}

\begin{footnotesize}\begin{sffamily}



\texttt{kadambari on 2011-01-14 at 09:10 said: }

``Another element is that Ekalavya was of mixed race, which was probably the reason he was excluded."

Western interpretations of Ekalavya are interestesting. He was not accepted because he was not a Kshatriya. Other characters in the Maharbharata are also rejected, for example, Karna, is of divine origin in the story, but has been raised adopted by a charioteer, so he is unable to duel with Arjuna which is reserved only for Kshatriyas. No one mentions his race really, but humble background as they think he is the son of the charioteer when he is not.

Eklavya probably came from those tribes that had not been Sanskritized. It is not so much as being of ``mixed" race, but of not being of a clan which recognizes the Vedic rites and culture. The story proves that men can be human and flawed and it is Ekalavya who is the hero at the end, and Drona appears mean minded. The fact that they wrote about these things even back then shows that people were aware that caste can be problematic in some instances in excluding worthy people.

Draupadi the wife of the Pandavas is known to be extremely dark skinned so her name is Krsna which means dark, and Krishna is often represented as black. So there are exceptions. Also tribals being people who are not a part of the Vedic clan and do not follow Vedic culture, could be of all colors and variety as Hindu culture encompassed a great area back then.

For example, the royal clan of Rajputs who are known to be handsome were tribals initially and outside Vedic culture, and not original Kshatriyas. They seize power and draw up geneologies for themselves as descending from the original Kshatriyas…So these things are not as simple as people seem to think…


\hfill

\texttt{kadambari on 2011-01-14 at 09:47 said: }

Also I forgot to add that Tilak was a Hindu nationalist and not a scholar. It was not uncommon for Hindus to just accept what was fashionable amongst the indologists back then. Many more things are known today than in his time as a result of scholarship… 

One interesting thing is the people of Kailash. People think that they are the remnants of Alexander. They are not. The are remnants of the original Vedic pagans who were not Sanskritized. There were many such people in Afghanistan, tribals who were not Sanskritized, but these people were forcibly converted to Islam as recent as in the 1800's. However, in Afghanistan, the Hindu royal clans and ethnicity were completely wiped out and the people today are descended from Turko-Mongols and other people from the Middle East, and are not of the same lineage as when Afghanistan was Hindu. Draupadi the princess in the Mahabharata is known to be from Gandhahar (modern day Afghanistan). So was Panini who wrote Sanskrit Grammar. Back then, Afghanistan was a part of greater Kashmir.


\hfill

\texttt{kadambari on 2011-01-14 at 13:14 said: }

``3.Kadambari: I have understood that Krishna's name ``black" is a title which refers to the fact that he represents the unmanifested Self as Arjuna refers to the empirical ego-self. Your interpretation of them seems to be largely historical, am I right?"

I understand that. I am not saying that Draupadi was black, but simply that she is known to be very dark in the Vedas. So all these things were not as strict as one thinks, its not as if people liked only blondes and so on. Obviously in North India in those times from where our culture arises, people were generally fair and sharp featured, but color is more complex than what we understand it in modern times. For instance, it used to be common for Kashmiris to be grey eyed red haired, but no one thinks they are superior just because they are fair and so on. This is what I am saying. I think the idea was to have the Vedic clan values. Many fair peoples in those areas were also tribals and were not Sanskritized, such as the remnants of people in the Kailash who still worship Pagan gods, these people would not be accepted simply because they are fair and so on. I am sure Sanskritization of various ethnic groups took place over a large period of time.

I recently came across people from Andhra. They are black but incredibly educated and gentle people, quite unlike what one associates with blackness in the West, with stable families and values.

Also regarding those simple remnants of tribals that got converted into Islam in Afghanistan in the late 1800, I feel their fate would have been much better if they had come under the HIndu fold gradually.

As for Dalits and so on, as I said, people had come up with religions that gradually tried to absorb these people without destroying the traditional structure such as Buddhism. Unfortunately, after the Islamic invasions, Hindus do not have time to think of their fellows and become incredibly orthodox, this was a means of surviva. One has to understand that the destruction that took place is so unsettling and massive, it is hard for a culture to reassert itself in this kind of circumstance. I think classical civilization survived in the South until the sixteen century before the destruction of the Vijayanagar empire, which is why people from the South even though they are dark can be very superior as they have those values left. The north is barbaric because of Islamic influence and destruction. We say all the thug values came into India from the Middle East.


\hfill

\texttt{kadambari on 2011-01-14 at 13:44 said: }

@Ismo

Also what I mean by Sanskritization is shown by the Rajputs who were famous for their beauty and chivalry. They were tribals who became Sanskritized fairly late in Indian history, they are not the Vedic Kshatriyas, although they claim their lineage to be, they were no less brave. Kind of like what Goethe said about Shakespeare's Julius Caesar, ``They are all thoroughly Englishmen but the Roman toga suits them well…". Rajputs would fight to the last man and when they knew they would be defeated, their wives and children would commit suicide and embrace sandalwood flames they would go into battle with the ashes of their cremated family on their foreheads. Later they become corrupted and gradually give in and begin to give their daughters to Muslims rulers as they cannot expel them but not all Rajput clans do so. They were always brave, but their weakness was that they quarreled amongst themselves instead of uniting in the face of the enemy.

But then most of India's aristocratic classes were exterminated or converted during the invasions.

But still you have to understand that when Islam overran most of Asia, it was not able to convert the Hindus so easily. And today a vast Islamic population is still a hindrance to India's growth by virtue of the fact that the culture is regressive and does not assimilate to the majority like other groups.


\hfill

\texttt{kadambari on 2011-01-14 at 16:57 said: }

Ismo,

I know that this is out of context, but I could not believe it that 100,000 British have converted to Islam so far, most of them women. Recently I read an article in the Times that thousands of British girls were being sold into the sex trade by some Islamic gang. And Tony Blair's sister in law is said to have converted and wears a scarf. Seriuosly, what is wrong with these folks?

As for tribals in Afghanistan I was talking about, the place is Nuristan, formerly called Kaffiristan, land of the infidels! There were still some tall blue eyed pagan non Hindu Afghans till the late 1800's but they all got converted to Isalm. I really feel sorry for some Afghans I see on TV. Many of them just seem simple hill people who ended up brutalized by the wrong religion!


\hfill

\texttt{kadambari on 2011-01-14 at 19:52 said: }

Ismo,

Even the Dravidian Hindu city of Vijayanagar was so magnificent existing as late as the sixteenth century, it was completely destroyed by a force of Muslim rulers surrounding it, and rendered uninhabitable, today one sees only the ruins, and even those are impressive. This marked the end of classical civilization in the South. On the glory of Vijayanagar, a Portuguese traveller writes: ``The city is as large as Rome and very beautiful. It is the best provided city in the world."

When one sees the ugly cities in India today overcrowded and unplanned, it is quite hard to believe that cities like this existed not too long ago…


\hfill

\texttt{kadambari on 2011-01-15 at 11:39 said: }

Ismo,

I mentioned the Dravidian city of Vijayanagar because it was the last Hindu stronghold and a great city. Real Hindus are a tiny minority in India today, thas is, Hindus who understand their civilization. Vijayanagar was a Dravidian city, but it was based on classical Hindu ideals of heirarchy and so on, that is, the role of Kingship and priesthood is that which is defined by classical Indian civilization. This shows that the ``idea" can have different manifestations in different contexts.

Also, it is worth noting that Eklavya is a hero in the story not because he tries to destroy the Drona's order because it is unfair to him, but he shows that he is fit for it, by exemplifying its ideals to the fullest. He is said to meditate on Drona, that is, internalize the ideals which Drona represents, which is what makes him what he is.

Also of interest is the student teacher relationship that is exemplified in the story. Eklayva still considers Drona his teacher (Guru). Now the Guru in the traditional sense is held in the highest regard. He taught his students free of charge and at the end before the students would depart, they would give a gift to the teacher which is voluntary called ``Guru-dakshina". There are interesting legends about how students would value their teacher to such an extent that they would be ready to give anything to the teacher in exchange for the knowledge he imparts, and Kings lose entire kingdoms in the Mahabharata as they wish to keep the word to the Guru and give him what he demands…Now it is in this context that Eklavya gives his thumb as ``Guru-dakshina" as Drona asks for it and he has told Drona that he will give him whatever he desires. It is interesting that in these legends, importance is given to keeping one's word, no matter what difficulties it causes. Truthfulness is valued, there are tales of heros who never tell a lie and so on. The Mahabharata is full of hundreds of tales, and is to Hindus what the Iliad and Odyssey was to the Greeks.

At the end, if anything, the tale of Eklavaya shows that Drona's action was unjust, and that Eklavya is noble. Even Drona ultimately repents for his sin (asking for Eklavya's thumb).


\hfill

\texttt{Will on 2011-01-16 at 12:09 said: }

I am very unfamiliar with the Mahabarata, but I have heard this story before. It seems to me that, like myths in general, it is subject to multiple levels of interpretation, depending on who is being taught, or what is being taught, and perhaps also when someone is being taught, i.e. in what Age. 

What strikes me is that Eklavya is absolutely certain that to be an archer is his destiny and that Drona is his guru, so much so that Drona's rejection of him doesn't change this at all. So then he makes a statue of Drona and relates to that as his guru, such that when he is asked where he learned to be an archer, he does not claim to be self-taught, but rather says that Drona taught him. 

The guru is a personification of a higher power, because it's easier to relate to another human being than to the Absolute as such. But it seems that Eklavya, through exceptionally pure devotion, was able to connect with the higher aspect of his guru, even though outwardly he was rejected.

This is about the nature of devotion, sincerity, and perseverance. It's like the Buddhist story of the old woman who worships a dog's tooth, believing it to be a holy relic, and becomes enlightened. It's said that authentic devotion to a phony guru is better than half-assed devotion to a fully realized master. I don't know if Drona is supposed to be a phony guru or not, but from Eklavya's perspective, it doesn't matter, because his devotion is pure.

In the West, there is a lot of confusion and misunderstanding of the nature of the guru-disciple relationship, because we don't really have a corresponding relationship in our own sacred traditions. (Though there are traces of something similar among the ancient Greeks, such as the story of Plotinus and his master.) And the fact that more than a few Easterners have taken advantage of our naivete on this matter doesn't help.


\hfill

\texttt{kadambari on 2011-01-16 at 13:45 said: }

``I don't know if Drona is supposed to be a phony guru or not…"

The Mahabharats is amazing to read, although ten times the size of the Iliad and Odyssey. One never gets bored reading it though.

No Drona taught the princes the arts of war was was superior in his knowledge. The story just shows his human side, he also does not hesitate to favor his son, although he is not a Kshatriya, and shows other weaknesses. I suppose that these stories are interesting because they show the human character in all its forms, the strong aspects and the weak.

``It seems to me that, like myths in general, it is subject to multiple levels of interpretation, depending on who is being taught, or what is being taught, and perhaps also when someone is being taught, i.e. in what Age. "

I do not think so, people have always been clear on the meaning of these stories, the glorification of a noble character is not hard to miss. The only times they are misintrepreted is when Brahmin haters use it to show that they were cruel in a one sided way, or to justify such actions today. Again the story just shows Drona's weakness, he is a great teacher, but he also has some human failings.


\end{sffamily}\end{footnotesize}
