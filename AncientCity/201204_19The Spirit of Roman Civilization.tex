\section{The Spirit of Roman Civilization}

\begin{quotex}
With this article from the December 1940 issue of La Vita Italiana, Evola takes up the idea of Romanity, and its continuity beyond the Roman Empire itself. While different from the mystical vision of Guido de Giorgio, based on Dante, it is equally spiritual. Following a conception of Spengler, the difference between culture and civilization is noted. Along the way, the features of the Old Right are delineated. Readers are encouraged to contrast them with contemporary pretenders to represent the right. We wonder where the \emph{hierarchical order} will come from, and who is describing the \emph{supreme, divine and transcendent power}, or is even capable of recognizing it. 

\end{quotex}
With the appearance of every new work on Roman Civilization, we experience a certain sense of annoyance: in fact, for the most part, we take notice of books of this type only perfunctorily, they do not reveal any new idea, they repeat the clichés of earlier ``positivist" interpretations, adding only the rhetorical hype of commemoration, thereby producing a pathetic effect, and whatever true meaning it has of our original tradition, it is not so much illuminated by similar writings, but rather trivialized and almost profaned.

We were therefore pleased to have been removed, at least once, from prejudices of that type in reading a very recent book of crystalline clarity written by \textbf{Pietro De Francisci} on \emph{The Spirit of Roman Civilization}\footnote{\emph{Spirito della civilità romana}, 1940}. Above all, beginning with its first chapters, we had to admit: Finally there is an authoritative person who hits the mark and knows what must be considered essential in Romanity. And we also found ourselves totally consenting to the justification of the books, viz., that no constructive revolution is a creation from nothing, but has as conditions the return to elementary principles and factors, which for us can only be those of the original tradition of Rome. And De Francisci also very correctly criticizes those who break our history into two parts: the history of Rome and her Empire on one side, the history of Italy on the other.

As for Corradini, so also for De Francisci, Italianity and Romanity are a single thing, or said better: they \emph{must} be a single thing, on the basis of a decisive choice of their own callings and traditions: that is, we must exalt, consider as our own, and glorify as ``Italian" only what is of value to us in our history, as ``Roman", and not have any lenience or mitigation for the rest. De Francisci correctly says that to bring youth to the awareness of the power and depth of the current of Romanity that spreads throughout all our medieval and modern history, eliminating wrong ideas and destroying old and new prejudices, means to draw on precious nourishment for the ideal strength of our revolution.

Who does not see the abyss that separates similar positions from those which, nevertheless like De Francisci, had to have the direction of the fascist Istituto Nazionale di Cultura [National Institute of Culture]—we mean Giovanni Gentile, who did not hesitate to assert what Romanity is for us, but only in the empty rhetoric of life and content, because for him the true Italian tradition is identified with a series of suspect thinkers and heretical rebels starting with the Renaissance, as if in fascist Italy itself no others should be seen and desired except those involved in the development of Italy of 1870?\footnote{When Italy was unified.}.

As the premise of his treatise, De Francisci, following up on an idea from Spengler, makes the appropriate morphological distinction between \emph{culture} and \emph{civilization}. Culture, both as an intellectualistic phenomenom, as well as refinement of the material conditions of the life of a people, has nothing to share with \emph{civilization}, reality. De Francisci writes this very profound passage:

\begin{quotex}
Civilization is not only a manifestation of the prevalent intellectual activities but the complex and concrete expression of all the energies of the spirit: it is not only the ruler of man in his exterior nature, but is at the same time the dominion of man over his own human nature, the awareness of coordination with other men, of subordination to a certain hierarchical order, and of dependence on a supreme, divine and transcendent power. 

\end{quotex}
It is a unitary and organic construction which, by being such, even permeates the political field, i.e., it also presupposes a political organization as the realizer and promoter of the fundamental values resting on the base of the organization itself. And in this special point, we see the contrast between the idea of civilization and the abstract conception of ``culture", as meant in its modern understanding, in which, culture would be a kingdom to itself, alienated from everything that is ``political", instead of being the highest animating and justifying force of the political, as always happened in all traditional civilizations and, at the forefront, let us admit it now, in the Roman civilization.

Now, De Francisci studies the ancient Roman world exactly in respect to ``civilization" in this precise meaning. Rome was eminently ``civilization" and its greatness must speak to us in the sense of this unitary and anti-intellectualist ideal. What was the specific face of such a civilization? What are the fundamental, typical, and constant elements of its ``style"? De Francisci considers four above all:

First of all, \emph{clarity} and \emph{simplicity}, founded on a precise and certain intuition of reality, and not only of visible reality, but also—it is the merit of our author to recognize it—invisible reality.

\begin{quotex}
While the Romans were realists, they never were materialists: thus few people like the Romans carried with themselves for centuries the conviction of the existence of a will and a transcendent power, to which laws must be adapted and human conduct conformed. But clarity and simplicity are the elements of grandeur 

\end{quotex}
These are reflected—as the echo of something eternal and detached from the small events of individuals, from everything that is \emph{pathos} and sensibility—in the \emph{monumental} element of the Roman world, Furthermore, the \emph{unity} that together is organicity and solidity, founded on a balance of forces and factors, on a wise bond that surpasses and encompasses all varieties, distinctions, complications: unity as formative and organizing power.

An order results from it, which, while ``it was experienced as a transcendent system of principles determined by the very nature of things" (which is the ancient Aryan conception of \emph{cosmos} or \emph{rta}\footnote{\url{http://en.wikipedia.org/wiki/Rta}}), is expressed in a rigorous, definite, and essential style: intolerance for everything that is disordered, uncertain, subjective, scattered. Precision and clarity predominate in the \emph{ethos}, but not as only a human norm, but rather as the rigorous objectification of a supersensible reality.

In that regard, De Francisci rightly opposes those who prefer to portray the ancient Roman as dry, lacking sentiment and imagination. What, alone, remains alien to the Roman soul, was the sterile subjectivism that surrenders itself to the caprices of the arbitrary in which every moral energy is scattered and dissipated:

\begin{quotex}
But not for this reason is his interiority less rich, which consists above all in the adhesion of the spirit to the norms of a higher Order. 

\end{quotex}
This is demonstrated in the three virtues of \emph{pietas}, \emph{fides} and \emph{gravitas}. And, as we ourselves on other occasions have emphasized, the lack of imagination in the Romans is more a sign of superiority than inferiority: it is to be taken in the sense, as De Francisci says:

\begin{quotex}
The imagination of the Romans is not a gratuitous game of intellectual boldness, it is not the creation of a world of images detached from reality, but an instrument to seal this reality in well-defined forms, to frame and organize its forces. 

\end{quotex}
The same thing must be pointed out regarding the accusation made against the Romans of having degraded thought in favour of action. But what thought is this about? No one denies the scarce sympathy of the Romans for theoretical constructions. But action itself, when it proves to be coherent, consistent, and efficacious—De Francisci notes—does that not itself bear witness to a thought, or rather, a higher power of thought? All the history of the Romans stands to demonstrate that they believed in such values and held firm to principles which, through their experience, were defined, made precise, affirmed, and even assumed an ever more universal importance and applicability.

\paragraph*{Political power, spiritual authority, divine law, sin, divination, totalitarianism and freedom in Ancient Rome}

In the order of the structural element, there is a specific element in the ``civilization" of Rome, i.e., a hierarchy, in which the preeminence is reserved to political values: everything is assumed and organized in the operation of the State. But we were pleased to see that De Francisci avoided a double false turn in which, in this regard, he finishes the greatest part of the modern interpretations of Romanity. In fact, in the first place, such a preeminence of the political element is not at all to be understood according to certain modern political pretensions to the primacy of a temporal power over any spiritual authority. The political and religious elements in ancient Rome were in an indissoluble union. The starting point of the Roman was the awareness that divine and transcendent forces exist and act behind human and historical forces. So the highest principle of Roman ``politics", and consequently of every determination of will and action, was that of conforming individual and collective life to the \emph{fas} [divine law],

\begin{quotex}
The revealed divine will, which is the supreme law against which it is not possible to rebel without committing a \emph{nefas} [sin], i.e., not just a reproachable act but has mortal consequences. 

\end{quotex}
After all, De Francisci had already mentioned the religious base of the first Roman law in his earlier \emph{History of Roman Law}. In the new book he recalls the profound significance relative to the fact of the inseparable connection of the \emph{imperium} of the Roman political leaders, with the \emph{auspicium} [divination], that is to say, with a discipline having as a presupposition the possibility of coming into relationship with the divine forces and of presenting the directions, along which they were able to confirm and empower human forces and actions. Even if De Francisci doesn't go beyond an examination deeper into the meaning of the rite in the ancient world, but in that there is quite enough to clearly distance it from those, in this regard, who see only ``superstitions" and ``obtuse fatalism" in order to appreciate, in the Roman \emph{ius} [law], only its positive juridical cadaver.

The other prejudice, which is often fostered in relation to the totalitarianism of Roman political civilization, relates to \emph{libertas} [civil liberty]. But, again, it is impossible to judge the ancient world with modern measures, which then are simply false and misleading. De Francisci clearly points out all the respect that ancient Rome attributed to \emph{libertas}: but it is a concrete \emph{libertas}, comprising in itself the concept of limits: it is freedom as the faculty and the legitimate right of movement, of acting, of disposing oneself, and even within a well-defined space, within a positive hierarchy, where each recognizes his own: \emph{suum cuique}. So the Roman would know an exemplary balance of \emph{auctoritas} [responsibility] or \emph{lex} [law] and \emph{libertas} while disregarding the democratic concept of equality characteristic of Hellenic decadence, in the surpassing individualism with a determination of limits, with an obsession with hierarchy, with a coordination of activity. And this is another of the aspects, according to which Romanity remains, for centuries, the sign and symbol of a higher political and traditional ideal.

\paragraph*{Roman upheavals, Asiatic cults and the end of the first Romanity}

Since we nailed down the truly valuable and, for many, the illuminative, aspect of De Francisci's new work in these terms, let's allow ourselves to make some other points.

First of all, in regard to origins: It is true that, in this respect, one hears nothing said about them today. Nevertheless, whoever has eyes sufficiently trained can recognize and discern what there is of value in regards to race and spiritual forces of the world of the origins. On the Aryan problem in Italy, on the meaning of the crossing or make up of various symbols and costumes—for example the rites of burial or cremation, solar cults and telluric-maternal cults, etc.—the spiritual relations between Etruria and Rome and so on, little or nothing is found in De Francisci's book. Now, if one does not succeed in having a, so to speak, \emph{dramatic} vision of the ancient Italic world, as it concerns both race and spirit, one can in no way grasp the true meaning of Rome, her battles, her mission, her destiny.

In relation to that, what is equally missing in the work of De Francisci is any investigation of what we would call the ``subterranean history" of Rome. In his book, attention remains concentrated on history in the common bi-dimensional meaning of the term, even if examined with undeniable acumen. The analysis of the most profound, spiritual aspect of certain social rifts and certain oppositions of worship in Rome is not made. What was, for example, the influence that acts, in ancient Rome, through the Sibylline Books\footnote{\url{http://en.wikipedia.org/wiki/Sibylline_Books}}? It is a problem, among many others, of the subterranean history of Roma, whose importance is anything but to be neglected.

De Francisci, as we said, saw clearly in the connection of the human will, and therefore of action, to a more than human significance, an element characteristic of Roman reality. And it was emphasized more particularly by others that the Roman perceived essentially the revelation of the divine not in space, as a vision, but in time and in history, like action. Now, can one recognize that, without also recognizing that a history of Romanity will always be incomplete, if it does not become, to a certain degree, a \emph{metaphysics of history}, i.e., if it does not strive to grasp a symbolic content in its objective way in the more important and decisive upheavals of Romanity? The danger of digression and pure personal interpretations, here, naturally, is great. Nevertheless, it is necessary to do something in this direction, if Roman history is to truly speak to us. Does De Francisci know the famous introduction to Bachofen's Legend of Tanaquil? In this old work, even in reference to Romanity, there are methodological ideas that still are particularly important today\footnote{Such as the interpretation of legend as history and the use of imagination or intuition to grasp it. \emph{Translator's note}}.

Also, De Francisci treated various problems of the imperial period, such as the importation of ``Asiatic" cults and their significance, in only an ``historical" way, in the current meaning of the word. The racial moment on the level of the elements of civilization and cult, were not developed. For example: what of the Asiatic cults and forms of the same imperial cult, referring back, in spite of the degeneration of their exterior expressions, to elements of a common archaic Aryan tradition, inasmuch as, for example, certain aspects of the Augustan religious reform, in fact, call back to life some ideas forgotten or obscured by the first Romanity?

\paragraph*{The causes of the decline of the Roman Empire, its rectification in the Middle Ages, and future prospects}

Instead, the best is the analysis made by De Francisci of the various political and social factors and various attempts of the restoration of the late imperial period. He brings to light the true cause of decadence: the universal Empire could only hold on provided that the expansive moment would have a corresponding moment of deconcentration and national-racial intensification. Although indispensable, a unique supreme point of reference—the imperial divine authority—could not be sufficient: it would have been necessary instead to provide simultaneously for the spiritual and material defense of the Italico-Roman race as the matrix privileged by elements destined to govern and command in the world. In place of that, Rome accepted cosmopolitanism, the turmoil of leveling and disarticulation. The Empire presumed to embrace universally the human species without distinction of race, peoples or traditions, on the only basis of the supreme central divine power, and close to a break up and a ``positivisation" of the ancient juridical idea, at this point turning into the natural law.

On such a basis we tend to believe that contrary to the opinions of most and, it can be said, judging by some of his comments, of De Francicsi himself, Christianity, or at least a certain Christianity, assumed the inheritance of only the negative aspects of the Empire. In fact, only in terms of the ``spirit", universalistically, it proposed to unify and gather the scattered peoples in the Empire; and if, beyond that, it created in the clergy a hierarchy and a central power, it was created without any racial presuppositions: the clergy was recruited from all the classes and peoples and, because of celibacy, could not constitute a caste, it could not give rise to a regular tradition, also supported on blood, as instead happens in many ancient Aryan societies.

Only in the Middle Ages, by means of the Aryo-Germanic contribution, a certain rectification of these negative aspects of the legacy of the last Romanity arose. The organic ideal arose. Catholicism itself came to show less the traits of a universalistic religion than those of the faith characteristic of the fighting block of the Aryan and European nations of ``Christianity". And it is in these terms and in forms that, as we have had the occasion recently to note in this journal, today have a curious aspect of current affairs and even of ``futurism", that the purest force of our origins is reaffirmed beyond the decline of the first Rome.

\flrightit{Posted on 2012-04-15 by Aeneas }

\begin{center}* * *\end{center}

\begin{footnotesize}\begin{sffamily}

\texttt{Cologero on 2012-04-16 at 22:41 said: }

Keep in mind that we aren't posting this to teach ancient history. Rather, we do so to reach those few readers who resonate to this description of ancient Rome, and begin to recognize themselves in it. Possibly, they may then remember who they are.


\hfill

\texttt{Cologero on 2012-04-19 at 23:20 said: }

Evola manages to skip over 1500 years in a paragraph and show signs of a hope that would be dashed. It is the latter Evola, the man among the ruins, that English only readers know. These pre-WWII writings paint a different picture.

There are interesting points here. The first Rome was already in decline. It was becoming universalistic, ethnically mixed, and infected with Asiatic cults. But how could it not be? That is the very nature of Empire. Christianity was just one of those Asiatic cults, but it is important because it became the dominant one. Evola, and apparently De Francisci, claims that Christianity (as Catholicism, a ``certain" form of it) inherited the worse aspects of the decline of the Empire. In other words, it was not the cause, but rather the result, the decline of the first Romanity.

Again, Evola points to the rectification of Catholicism under the influence of the Germanic peoples in conjunction with the Romans. But he claims this rectification did not apply to the priestly caste, that is, the spiritual element of Catholicism, something Evola wants to bracket out. However, Catholicism did become the faith of the warrior caste of the nations within Christendom. That is undeniable, but it is hardly obvious how the warrior religion can be split off from its spiritual aspects. Here, we prefer to follow Dante and De Giorgio.

But the final sentence says more than the entire essay. At the time the essay was written, Evola was hopeful another Roman-Germanic alliance, in the form of the Axis powers, was the real future that would reaffirm the forces of the First Rome. After all, let us not forget that the First Reich was the Holy Roman Empire, the creation of the original Roman-German alliance.

Of course, we now know that was not to be. Nevertheless, the essay does indeed have the aspect of current affairs and a futurism, although not in the sense hoped for. A de-Christianized Europe without spiritual moorings and without a faith for its warriors, is following the downward path of the First Rome. Europe is universalistic, ethnically mixed, and besieged with Asiatic cults. A very few see that as a problem, but most do not. The former may resort to desperate tactics.

Is a rectification possible? Do we need another poll question?


\hfill

\texttt{Graham on 2012-04-19 at 23:57 said: }

One observes a similar cosmopolitanism, both racial and spiritual, preponderating in Rome since Vatican II. With the same deletorious effects.

Evola seems to blunder slightly on the issue of priestly celibacy, since this isn't the main reason why Christianity has never had a real priestly caste. Celibacy has only been required of priests since the 11th Century, and even then only in the Latin Rite. In the Coptic Rite, for instance, where priestly marriage is actually encouraged, a caste of priests has never emerged. Many further details could be added but are unnecessary. This discipline was added for the Latins in large part to counteract the incipience of a priestly caste. So there are deeper reasons at work.

Is it Gornahoor's position that Christianity's congenital lack of an hereditary priestly caste is anti-Traditional?

With apologies for quoting at length, I would like to expose the readers here to an argument made by Olavo de Carvalho in a debate against Alexander Dugin. I believe Logres has made reference to this debate before, but unfortunately – in my opinion – only to Dugin, the less intellectually forcible of the two men.

``The strength of the Orthodox Church as a historical agent has penetrated deeply into the mind of Professor Dugin, shaping his ``holistic" notion of theocratic empire. He does not conceive of the empire but as a structure emanated from the Church and united to her, symbolically, in the person of the Czar. In an interview given in 1998 to a Polish magazine[15] he qualifies as ``heresy" the distinction between Church and Empire that shaped Western civilization. But without this separation, the only hypothesis left is that the borders of religious expansion coincide with the map of the empire with pinpoint accuracy. Now, the various empires and imperial nations existing in history have always had well-defined borders that separated them from other empires and independent nations. In this case, the imperial religion becomes only an expanded national religion. What is then the Czar? One of two things: either he is the head of a mere national religion having no possibility of expanding itself beyond its borders and looking with deadly envy at the expansion of her Western competitor, or, alternatively, if he wants his religion to impose itself as universal belief, he has to invade all countries and become the emperor of the world. Both the National-Bolshevik project and its Eurasian version are born from an internal contradiction of the Russian imperial religion. The Eurasian project is the only way out for the Orthodox Church if she does not want to remain confined to the limits of the Russian nation, failing in her declared mission as a universal religion. Meanwhile, the Roman Catholic Church can expand comfortably to the last frontiers of Paraguay and China without the need to carry an empire on its back. And that was, in fact, what happened, while the Orthodox Church, through the medium of Professor Dugin, is still looking for an exit leading to the world and does not see other means of finding it but to constitute herself into a World Empire."

\url{http://www.theinteramerican.org/blogs/olavo-de-carvalho/257-olavo-de-carvalho-debates-alexandr-dugin-ii.html}

It is clear that Dugin backs the same `totalitarian' vision as Evola advances in this article, one which the Catholic Church's understanding of universality has always rejected. And indeed it can legitimately be asked which of the two visions is the more universal – the one that necessitates militaristic expansion, or the one that simply requires adventurous missionaries? I would like to ask which of these two visions, if either, Gornahoor supports – the total fusion of spiritual and temporal powers envisioned by Dugin and Evola, or the somewhat looser cooperation of Mr. de Carvalho and, it seems, the whole of the Catholic tradition. Has this been addressed somewhere?


\hfill

\texttt{Cologero on 2012-04-20 at 00:47 said: }

No, the lack of a hereditary caste is not anti-traditional. Guenon did not think so, since Islam has no such thing. Even by Evola's standards, viz., caste is something transcendent and essential to a man's identity, caste is not necessarily dictated by biology. In may be forgotten today that priests used to have to meet certain standards as far as manliness, physical health, even mannerisms and postures; these should be regarded as the recognition of someone's caste.

Excellent point about Dugin and Evola. Evola never could accept the separation of the two powers in the Middle Ages. We have been preparing the background to bring Solovyov back into the picture. Unlike Evola, he regarded the separation of the spiritual and temporal powers as an advance. So did Tomberg. Since their respective domains are different (albeit hierarchically related), it seems better to separate them. For example, look up the post on the City of the Sun.

Our position is that the only real unity is a spiritual unity based on a valid tradition, not a pseudo-unity based on biology, race, military conquest, heresy, or a social contract. I believe we have made this point often enough. In Penty's Common Mind we read:

\begin{quotex}
From a sociological point of view, the first function of the Church is to maintain in society the acceptance of common standards of thought and morals. This is a necessary condition of any stable social order, because if men are to share a common life they must share a common mind, for there must be a common mind if men are to act together. We have moved so far away from the thought and impulse of the Middle Ages that there are few to-day who recognise the fundamental importance of the common mind to any successful ordering of our social arrangements. Yet it is only necessary to reflect on the general social and political situation to-day to realise that in recognising its importance the Mediaevalists were right while the Modernists in failing to do so are wrong. 

\end{quotex}
You can easily substitute ``Ancient Rome" or ``Spiritual Authority" for ``Church" is this passage.

\hfill

\texttt{Logres on 2012-04-20 at 08:01 said: }

I was just thinking of John Romanides' criticisms of Charlemagne, last evening, \& how perfect they sounded five years ago. It's becoming clearer that the West (in fact) did the ``one thing necessary" in the wake of Rome's collapse. I need to re-read Carvalho \& Dugin's exchange; the Penty article on Right Hand Path is excellent. And I finally understand (after re-reading Soloviev's Law of Development article, here) how the ``devolution" of spiritual/temporal linking doesn't entail the disappearance of the Tradition, but sets up a re-convergence out of Freedom, per Tomberg.

\url{http://www.gornahoor.net/?p=3571}


\end{sffamily}\end{footnotesize}
