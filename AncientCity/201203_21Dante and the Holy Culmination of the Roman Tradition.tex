\section{Dante and the Holy Culmination of the Roman Tradition}

\begin{quotex}
From La Tradizione Romana by \textbf{Guido De Giorgio}. 

\end{quotex}
The traditional gold vein of Rome in the living unity of the two forms supplementing each other in a perfect match and equilibrium, is found again in all its wholeness in Dante who was the first to reveal the mystery of Romanity. The sacred arriving at the creative synthesis of the elements contained in the ancient and new traditions for which he can be called the prophet of Fascist Catholicity [Cattolicità Fascista] in the absolute meaning of the expression.

Poetry in him resumes its place and its sacred destination, it follows ``the footprints" as Boccaccio said, the traces ``of the Holy Spirit", it is no longer psychological and artificially descriptive, but an initiation, revelation, and realization. While theology is expository and proceeds discursively, submitting itself to the limits of reason illuminated by revelation, poetry grasps with supersensible intuition the mystery of the symbols and internalizes them, transforming them into shapes, living them, surpassing their representative exteriority so that to make of them the most suitable vehicle for liberation. In this sense, and in this sense only, Dante is the poet and the Comedy is the sacred poem, vehicle of the divine truth and supreme effort, the highest perhaps that was ever accomplished, to transform the sensible image the reason for the realization of traditional metaphysical principles, grasped in two directions, the ancient and the new, indissolubly unified in the occult name of Rome that is the seal of the divine song.

Thus, the universality of Dante was unique among men and poets, even though the anagogical interpretation of the Comedy had not yet been tried, in order to be completed and realized only ascetically by those who belong to the Race of the Spirit and who are the true key bearers of sacred science and the fasces bearers of divine power.

The Comedy is the supreme pilgrimage of the worlds considered as the only temple of God: if the point of departure is the earth and that of arrival is heaven, this apparent duality shows man only what he must reach when \emph{he is not what he is}, what he must become in order to be what he is, and how his earthly humanity is only a veil, when removed, the Divine Reality is revealed in His original unity, ineffability of the Ineffable. Only at this point poetry ceases and with the realization of the mystery of man who is the Reality of God, the Comedy finishes because the pilgrimage is completed, the end is reached, death is overcome, \emph{fieri} [becoming] became \emph{esse} [being] and \emph{esse} the radical non-existence of the Divine Night.



Heaven and earth are dissolved in the last smile of the Comedy, when the bright axe that dominates the Fasces [Fascio Littorio], resolving the enigma of the two-faced Janus through the plenary universality of the Cross, revealed the occult name of Rome and dissolved the Vestal fire on the lips of the Lord of the last rite. Here the mystery ends and the brilliant flashing ends in the essential tonality of the Silence, lord of Forms and Rhythms, the highest peak of integral realization. The ancient and new tradition led the poet to the secret of the Primordial Tradition, to the ``\textit{letizia che trascende ogni dolzore}" [``joy that transcends every sweetness of delight"]. In the paradisiacal vortex, it is completed and having completed itself it untied the traditional knot, nor can there be anything in what alone is it integrally and completely.

This is the miracle of the occult name of Rome and this is the reality of Dante and the Comedy.

The vision of the peak, insofar as it is so imperfect of the expression that tries to grasp its mystery, permits it to better consider the foundation and the progression and, what is more important to our task, the unification of the two traditions in Rome, i.e., in the Spirit of God. It is not possible to allude to this, unless metaphorically, in this simple introduction to the doctrine of the Roman Tradition, that cannot try to be more than what it is, a vestibule to the Temple, a preparation to the work of integral restoration of Traditional Romanity contained in the Comedy that is the sacred poem of Rome, no longer the ancient and new, but the eternal.

If Virgil represents the ancient tradition and Beatrice the new tradition and if, at the threshold of the Terrestrial Paradise, Virgil disappears before Beatrice, Beatrice also disappears when the divine mystery is grasped by Dante in its immediate realization and what then remains, above and beyond the two traditions unified forever is, climactically, Rome.

Virgil guides the poet through the world of Forms and Rhythms, in the two spheres of bodies and shadows, that he knows perfectly because he belongs to a tradition in which these two domains particularly were meticulously observed and studied, domains that constitute the subterranean and sublunar underworld whose secrets are fully treated in the three Virgilian works ``\textit{sotto il velame delli versi strani}" [``under the veil of strange verses"].


The ancient Roman tradition attached great importance to the knowledge of the immediate and psychic world governed by laws of internal, occult order, that embrace the totality of beings and things considered always with reference to the forces whose expression they are. The so called ``concreteness" of the Romans was based exactly on the precise meaning of these forces that act most visibly in the existence of man inserting there a hidden network of which the events, especially ``chance" events, as the common people believe, are the most significant effects: these forces are either propitiated, dominated, or determined. Virgil represents in the Comedy the knowledge of the two subterrestrial and superterrestrial worlds, the latter term however meant in the much more precise sense that must be given to the third element, the air, which symbolically corresponds to the subtle elements, the Rhythms, more through their ``diffusivity" than for their nature.

In hell, we are present at the extreme concretion of these unchained forces and, so to speak, precipitated in the closed vortex of ignorance, while in Purgatary, we catch sight of them liberated from the formal element in their spontaneous structure of the subtle body, the shadow. Virgil guides his disciple with ``art" up to the threshold of the Terrestrial Paradise from which ascent will begin at the paradisiacal levels, i.e., at the higher states that are forbidden to them because they realize only by means of Revealed Science, Beatrice.

Up to this point the two traditions remain separate even though the one is resolved in the other, that indicates Dante's dismay at Virgil's disappearance in the face of the vision of Beatrice. In the Terrestrial Paradise we have the explanation of the traditional integration, after the theory that leads the symbolic cart in front of the central tree which revives, discovering the reigns of the Silence where only the ascension to the divine states is accomplished. In other words, the second tradition is not opposed to but reveals the first, and completes it, bringing it back to the invisible centre from which everything emanates and to which everything returns as long as it is stripped bare to its original essence.

What in the first tradition is the \emph{Imperium} [empire], is the \emph{Regnum} [kingdom] in the second, while separately they indicate respectively temporal power and spiritual authority, there is an absolute seat in which while converging they merge into each other, and this seat, materially, symbolically, and actually is Rome. So that, while the second tradition illumines and reveals the first, the first precedes, prepares, and exists only for the affirmation of the second; there is an initial necessary opposition that is resolved only in Rome when, i.e., a unifying centre is found that is at the same time the neutral point where the traditional quarrel ends.

It is not easy to express this succession and fusion that must not be considered historically but on a plane where the symbolic values remain such even if unknown or misunderstood until a new light suddenly illuminates them and reveals them. For the two traditions which we discussing, Rome is this light and the Comedy is the poem of sovereign and Holy Rome, the unifier, while Fascism is the operator of the synthesis in which the two forms are compounded in a new revelation of power. The greatness of Dante consists in the statement of these two aspects, the ancient and the new form, of the same tradition that is Roman universality, and, while in De Monarchia he combats, as he says, \emph{pro salute veritatis} [``I engage in battle in this book for the cause of truth"] to reclaim the ancient tradition that had to remain to sustain the second. In the Comedy he arrives at the realization of the unification, at what we will have to call traditional perpetuity, showing the reality of a transhumanization [passage from a human state to a superhuman state] in all its levels that embrace beings and elements, world and afterworld [\emph{soprammondo}], heaven and earth, from Forms to the Rhythms in the Silence.

He is therefore the advocate of Sacred Science in the living, and not theoretical, wholeness of the Roman Tradition, of the \emph{Imperium} and the \emph{Regnum}: the ancient and new traditions mutually sustain each other, avoiding thereby all incongruity of a conflict that would impoverish them, impeding their supreme synthesis which is, practically, the equilibirum of the temporal and the spiritual and, at the centre of realization, the complete transfigurative process, the integral initiation, the real ascent from the earth to the elementary and trans-elementary heaven.



All the symbols of the ancient tradition live again in the creative light of the achievement, the union with god [\emph{indiamento}], and the Argonautic enterprise finds its fulfillment in the revelation of the true face of God with which the last canto and the last canto of the ``Sacred Poem" end. The vein of gold, the vestment of glory, is dressed by Dante in the great light of Rome, highest peak in the radiant circularity of the Ineffable. All the traditional sciences flow together in the Comedy through a dynamic complexity of states and a perfect knowledge of the transitions in the ambit of the three worlds through which the process of the cosmic-human illusion takes place, up to its resolution in the supreme principle in the three phases corresponding to the death, resurrection and transfiguration of man in God. The process of death is slow, gradual, and it embraces all terrestrial experience in its most interior forms to which the vices correspond in the moral sphere, i.e., animality: from here, the \emph{descending} hierarchy of the underworld where the realizing interiority assumes over itself all human development reducing it to a totalizing unity of life integrated in the being that had and lost the light, Lucifer. He represents the maximum concretion in the scheme of diabolic unity, the inverse reflection of the divine unity of which he has only the Trinitarian analogy in the three faces that are turned antithetically while in God they are homocentric and confluent.

Human plurality is resolved in its huge body, congealing itself, solidifying itself, petrifying itself: he represents the fall, the precipitation, the last terrestrial coagulation of the impassable waters, the freeze, the totalization of ignorance and darkness: its night corresponds, according to the reverse analogy, to the night of God, to the precreative indistinction in which all the determinations of being are based, as in him all the determinations of non-being, i.e., of evil. The analogy is perfect even in that Lucifer is the first and the last as God is the alpha and omega, but while in the first case, one has a duality of movement represented by the fall, in the second instead we have the essential unity of the opposites considered as the two confluent points of the divine cycle. Lucifer who \emph{was} the first \emph{is} the now the last: in him the temporal cycle is resolved in the eternity of evil, as in God the eternity of good is resolved. The two main antitheses represent what can be called the highest \emph{critical polarity}, i.e., the terrifying point of active realization, that precisely in which Virgil very painfully creates the \emph{overthrowing} which is a \emph{rectification} where the descendant interiority becomes the ascendant interiority and the place of damnation, the basis of salvation. From the stony precipitation whose symbol is Lucifer the ascendant rectification begins and the stone that is concretion and fall becomes the basis necessary for the flight toward the elementary complexity and the trans-elementary totality.

Purgatory is the place of the second birth from Forms to Rhythms in a hierarchical purification of which the seven terraces [of the mountain of Purgatory] are pointers: it is not a passage from Forms to Rhythms but a resolution of Forms in Rhythms, of the body in the shadow, of corporeity in psychicity where then even this is freed in the spirituality that is the Silence, Paradise. Virgil's ``art" is the perfect knowledge of the two spheres, the Forms and the Rhythms, through which the decoupling from error and from the ignorance of human and terrestrial fallacy is accomplished hierarchically, since reality is one, that of God, but a man has knowledge of this reality only when he integrates it, realizes it, becomes it. Up to what is not completed, it is necessary to traverse the levels of development that, from the human point of view, are the three corresponding to Hell, Purgatory, and Paradise. Dante in the Comedy proposes and exposes all the experience of realization, the complete, integral initiation, through positive knowledge, lived in all the levels that lead from the human to the divine. In the two first kingdoms, the ancient tradition sufficed to lead this path to completion and Virgil represents the science and the knowledge of the laws that govern the subterrestrial and sublunar world. He disappears before Beatrice because he dissolves in her, is completed in her, not because he opposes her as would be the case if Dante had considered the two traditions irremediably different and antagonistic as everyone believes, as much by those who exalt the first as by those who oppose the second. Beatrice appears at the moment when the first guide, Virgil, has completed his work and embraces the gross and subtle manifestations, the Forms and Rhythms. The exercise of human reason in its complete and normal development naturally leads to the sphere where a process of mystical union with God [indiamento] is initiated in the levels of the non-formal, i.e., in the zone of the Silence represented symbolically by the heavens.

Here Sacred Science, Beatrice, carries out the integrative cycles of unity in a flight that is light and salient flame between the Rhythms circularly untying itself in the plenitude of the Divine Being. The Trinitarian scheme is amplified in the assumption of the nine hierarchies, effulgent wings that facet the divine infinity of the joyful love of the Angels, Archangels, Principalities, Powers, Virtues, Dominations, Thrones, Cherubim, Seraphim, where celestiality is engraved in relations of light and splendour in the face of the terrestriality which is overcome, resolved, and dissolved in the divine whirlpool. There still remains, in the first seven heavens, the divisibility of the light with planetary vertices in a progressive self-realization of perfection in unity, a display of radiations in the body of the divine diamond.

In the residence of the sun where the zodiacal band combines with its perfection of the ternary and quaternary resolved in the supreme syntheses of the trinity ($12 = 1 + 2 = 3$), the mystery of the Perfect Man, the Triumphant Christ emerges, the perfection of the divine sonship in absolute assumption of radiance. It comes after the last creative level of the ninth heaven of the Trinitarian perfection, assuming in each of the Divine Persons the seal of the others so to project the mystery of the Ineffable in the creative circularity, then finally the absolute level, the culmination perpetuating itself in the eternal scheme of the worlds, the Empyrean. Here Beatrice disappears, not like Virgil, to permit a progress, an arriving, an end, but to unravel the mystery of the Last Seal where the virginal matrix carries out the cyclic reduction of the light in the very face of God. The last level of Silence is integrated in the same riverbed of the Divine Night where the pulse of the Ineffable vibrates in the realizing unity of God, the Supreme Zero, transcendence of the plenitude itself, darkness of the Ineffable.

The purely exterior literary merits that common men [\emph{volgo}], the \emph{profanum vulgus} [unholy rabble], admire in Dante have no importance and would nullify the value of the Comedy in the very eyes of Dante and of those who can and know how to understand the purpose for which the poem was composed.

It would be necessary to feel ashamed to still speak of, and only of, ``art", ``poetry", ``brilliant construction", in the modern sense of the word when one alludes to Dante's work which is only and eminently sacred in spirit and structure, while the allusions to historical persons are clearly motivated by Cacciaguida at the end of canto XVII of the Paradise. But these allusions hide well dramas other than those that the profane see in it. Regarding these, the central motive, the general orientation, are understood, traditionally speaking, but it is not, nor perhaps will ever be, possible to explain entirely due to the impossibility of retracing the elements of a tradition that, in Dante's time, was entirely oral. As to the strength and the expressive completeness so steadfast in Dante, it is due to the very substance of the topics treated: it is about \emph{poetry of inspiration} in the absolutely sacred meaning of the word and those who know what is meant by such an expression, know the imbued power of the realizing wave that moulds the word in a type of revealing plasma where the specular miracle of the perfect reflection is accomplished. The same Rhythm, the homophony is adequate to the state that tries to be expressed in a way to constitute as many \emph{topoi} or static forms, normative traces in which the transfiguring synthesis from the image to the idea is completed, to substitute for the oral initiatic transmission.



The moderns, therefore, who for centuries have read, studied, and commented on Dante resign themselves to understand nothing of it as long as they persist in not considering him as a prophet, a sacred poet, whose work is the highest expression, perhaps unique, in the Roman Tradition, an eternally new synthesis of the two traditional forms that in Rome, in its occult name, will find their completeness and their perfection. Here is his greatness and his true originality: if the expression reaches a plastic and vibratory perfection never before equaled, that is due to the sacred character of the Poetry which catches the eternal light of revelation in the transience of phantasms and concentrates it in radiant syntheses. In Dante, East and West are balanced in a unique centre that, substantially, is the Primordial Tradition, i.e., the unique most important traditional universality and ultimate realization. Never during the Middle Ages were the relationships between East and West so close: never in those great centuries had the traditional elements completed each other and disclosed each other for oral transmission, direct from master to disciple and from disciple to disciple. Dante appears exactly at the end of this era but in a period in which Dominicans and Franciscans, although already degenerate and hostile, had outlined the two greatest ways of realization of the divine—cherubic and seraphic—homocentric even if divergent by nature and process. He unites these two ways substantially, bundles [fascifica] them without confusing them. And it is necessary to note that when we use the term \emph{fascificare}, we mean nothing that can be considered, even if only vaguely, syncretism or mixture: to bundle in the pure traditional sense means to give to each way, to each element, a unique direction, a centre, an axis without confusing them: this is the novelty of the traditional steadiness.

There is one bond that captures the twelve rods of the Fascist bundle [Fascio Littorio] and there is one lightening power expressed by the double cut axe: the emblem is traditionally the greatest because it represents the confluence in the vertical direction, i.e., that of elevation and conquest. In Dante fascification is supreme, East and West, ancient and new Rome, temporal and spiritual, heaven and earth, world and afterworld, man and God, everything gets becomes more marked, matches up, is unified at a supreme vertex that is Rome. This is Sacred Fascism, the true triumph of justice and truth in man and in the world: if there are quarrels, battles, falls, these have no importance since they take place in the bosom of a traditional society where everything is formed from the supreme balance assured by key bearers and fasces bearers, by the \emph{Regnum} and the \emph{Imperium} forever unified in Rome.

This is the perpetual peace, the universal peace which Dante constantly mentions in De Monarchia and in the Comedy: the reaching of the traditional equilibrium that can only contain and annul in a higher place of harmony the battles and the inevitable disputes in the world, where, since duality reigns, it is not possible to avoid conflict without which the supreme unifying element of Rome would be suppressed. But instead, when this element is restored to its true function and reestablished the bases of the Roman Tradition in their living integrity, a new greatness would rise from under the present ruins of the western world, a new purity of life and thought and the Temple protected by the sword would rise up in the light of Rome for the glory of God in the heavens and the peace of men on earth.


\hfill



\flrightit{Posted on 2012-03-21 by Cologero }

\begin{center}* * *\end{center}

\begin{footnotesize}\begin{sffamily}



\texttt{Matt on 2012-03-22 at 21:05 said: }

Cologero, your hunch (more aptly, intuition) about Fr. Johnson was correct.

Here's Johnson's relatively new podcast on Guenon.

\url{http://reasonradionetwork.com/20120315/the-orthodox-nationalist-rene-guenon}


\hfill

\texttt{Boreas on 2012-03-23 at 12:36 said: }

This text is like poetry in itself. A world of light and life eternal opened before my mind's eye when reading this. Beautiful, thank you.

``Night is preferable to a day."


\hfill

\texttt{Cologero on 2012-03-26 at 08:48 said: }

Apart from the annoying mispronunciation of ``Guenon", this lecture was more about Fr. Johnson than Rene Guenon. It addressed a small, albeit important, aspect of Guenon's oeuvre. A minor point: pace Fr. Johnson, Guenon did indeed make a distinction between Eastern ``religions", such as Buddhism, and Western (so-called Semitic) religions.


\hfill

\texttt{Cologero on 2012-03-28 at 12:36 said: }

Here we see de Giorgio as the reconciling force of the Western Tradition, finding unity and harmony where others can only see difference and conflict. Unlike contemporary men who read into the past their present experiences, this is how the Medievals saw themselves. Like the ancients, the medievals had their High Priest and Emperor in Rome; despite the differences in outward form, their was a continuity in their inner life. Dante is, therefore, the holy culmination, the prophet, the revealer of that continuity.

Poetry in Dante is has nothing to do with the psychological fantasies of the moderns; rather, for him, it is a revelation of the Holy Spirit, the way of initiation, with the goal of realization. The symbolism of the Comedy is intractable to the rational mind and is comprehensible only to those with a developed spiritual intuition.

The Comedy describes a pilgrimage, or a quest, from the three worlds, analogous to those of the Tao. When the path is traversed, death is overcome, becoming becomes being in the Divine Night.

When de Giorgio writes of Fascism, he is not just referring to a limited political movement in Italy of the 30s, but rather to the Roman Tradition, which was the ultimate source of the symbol of the Fasces\footnote{\url{http://upload.wikimedia.org/wikipedia/commons/thumb/7/74/Fasces_lictoriae.svg/224px-Fasces_lictoriae.svg.png}}.

The Comedy leads us to the Primordial Tradition, which has a joy that transcends any little delights we may experience in our ordinary life. The path leads from earth to heaven. Nietzsche exhorts us to ``become what you are". But it requires a Dante to tell us who we are and how we can become it.


\hfill

\texttt{Matt on 2012-03-28 at 17:14 said: }

These translations of de Giorgio have been great. Are you thinking of translating anymore? His approach to Tradition I must say is rather fascinating (and from the description of him in Evola's Path of Cinnabar, de Giorgio the man is fascinating).


\hfill

\texttt{Matt on 2012-03-28 at 17:15 said: }

Never mind, just read one of your earlier comments that states you will provide more. Apologies.


\hfill

\texttt{Cologero on 2012-03-28 at 19:02 said: }

The way to encourage us to provide more translations, which, obviously, we do not require for ourselves, is to engage in a lively discussion in the comments. Readership has gone down in each of the four installments, so, unfortunately, not everyone shares your enthusiasm.


\hfill

\texttt{Cologero on 2012-04-05 at 07:45 said: }

The occult, or secret, name of Rome is an idea that may be unfamiliar to some readers. Besides its public name, Rome had a secret name that was only used in certain rites and rituals. Only the High Priest and his associates even knew that name. Presumably, this secret name has long been forgotten. Some say it is Amor (Roma spelled backwards), which means Love.


\hfill

\texttt{Cologero on 2012-03-28 at 13:08 said: }

``Shadow" here needs to be understood in the sense of shade, or ghost, as the ancients understood the post-mortem state in the afterworld. Vigil leads Dante through the worlds of gross and subtle manifestation (forms and rhythms). These worlds were well understood by the ancients, as Virgil wrote of them in his own inspired poetic works.

The ancient Romans knew the occult forces behind those worlds, which they either propitiated in their rites or learned to dominate. The third dimension of depth, these occult forces, elude the minds of the vulgar, both in the past and even more so in the present.

Hell, then, is the most concrete form of gross manifestation, and Virgil shows the way to subtle manifestation. However, he yields to Beatrice who brings Dante to the threshold of understanding non-formal manifestation. Most men, because they either adhere to the ancients or the medievals, can see only a division and a separation. However, Dante reveals that this division is illusory.

De Giorgio's view is quite different from that of Duke di Cesaro, who regards the ancient (and eventually even the medieval) tradition as something to be overcome and discarded in some evolutionary master plan. No, de Giorgio shows that the ancient tradition is fully revealed and completed in the medieval tradition, which itself would be partial and incomplete without the earlier.

The each have their role in the synthesis. The ancient tradition is the Empire, or temporal power. This is where Evola stops and it colours his understanding of the Medieval synthesis. The medieval tradition is the Kingdom of God, representing spiritual authority. Both wings of the eagle are necessary, and this is the meaning of the symbol of the two-faced Janus\footnote{\url{http://en.wikipedia.org/wiki/Janus}}. Rome is the centre, and the quarrel between the two traditions must end, once this is understood.


\hfill

\texttt{Matt on 2012-03-28 at 17:20 said: }

``Shadow" here needs to be understood in the sense of shade, or ghost"

Kind of like a psychic corpse?


\hfill

\texttt{Cologero on 2012-03-28 at 19:14 said: }

I'm afraid, Matt, that I cannot provide every reference for every conceivable topic, as much as I try. Since Virgil did include ghost stories, it would be nice if readers can fill in the gaps … you cannot continue to expect George to do everything for you.

For those interested, a place to start is this painting by Ary Scheffer\footnote{\url{http://en.wikipedia.org/wiki/File:1855_Ary_Scheffer_-_The_Ghosts_of_Paolo_and_Francesca_Appear_to_Dante_and_Virgil.jpg}}.


\hfill

\texttt{Matt on 2012-03-28 at 19:48 said: }

The question was rather redundant in hindsight.


\hfill

\texttt{Cologero on 2012-03-31 at 12:58 said: }

We see in this segment that de Giorgio is confirming Guenon's teaching on symbols. He write: ``symbolic values remain such even if unknown or misunderstood until a new light suddenly illuminates them and reveals them." The symbol is there, even if its full meaning is hidden, or occult. Dante reveals in the Comedy the meaning of the occult name of Rome.

He prepared the way in De Monarchia in which he revealed the full meaning of and justification for the first Rome. The way from a human state to a transhuman state is revealed in the Comedy. This is understood by very few today, and I suspect this is the first exposure of this occult teaching to most if not all Goranhoor readers.

Those who cannot move beyond the first Rome of the Imperium are unable to progress. Those who don't understand the second Rome of the Kingdom have lost touch with their own Tradition. Clearly we mean here a living connection, as de Giorgio writes, not merely a theoretical understanding.

``Indiamento" is the word used in neo-Platonism for the mystical union with God. Once again, de Giorgio is making clear this connection, at least for those who make the effort to see and are not blinded by their pre-existing prejudices.

The nine hierarchies, which we have described in \textit{Spiritual Beings}\footnote{\url{http://www.gornahoor.net/?p=1941}} can be understood as semi-divine beings or as higher, transhuman states. A fascinating metaphor used by de Giorgio refers to them as ``facets". In other words, the ``Trinitarian scheme" is passed through, as it were, the facets of a diamond, and like a prism, separate the divine Light into the multiplicity of the world. Hence, the angelic hierarchies assist in the creation (or emanation) of the world.

It is worth meditating on the idea of the Perfect Man, that is, the man who realizes all his possibilities as both Guenon and Evola have written about. The end of such a meditation will be the realization of the Absolute Being; but the way to such a realization is only through the Perfect Man.


\hfill

\texttt{Cologero on 2012-04-01 at 13:01 said: }

Like his colleague Rene Guenon, Guido de Giorgio regards the Divine Comedy as an inspired work describing a spiritual path. Vulgar minds, and de Giorgio would include academics in this category, fail to grasp this aspect of the Comedy. Some even see in the Comedy, nothing but racism, anti-semitism, and homophobia\footnote{\url{http://takimag.com/article/the_divine_comedy_funnier_than_ever_taki_theodoracopulos/print\#axzz1qoAHKtSq}}. The many allusions to events in Florence of some 8 centuries ago hide deeper meanings that are lost on them.

Dante rectifies the division between the East and West, as his goal is the Primordial Tradition. The references to the Dominicans and the Franciscans represent two competing spiritual styles. They can be understood as the difference between Apollonian and Dionysian initiations.

Given the human situation, men continue to fight illusory battles: between Apollo and Dionysus, pagan and christian, east and west. At a higher level, these views are unified. When this becomes understood by all the parties, then the Western World will return to its previous vigour.


\hfill

\texttt{Cologero on 2012-04-06 at 11:29 said: }

Why did Dante place Plato and the Philosophers in hell?

Some readers may remember Exit, who until now has honored a voluntary ban on commenting here. However, he did ask a pertinent question about the Comedy. On the assumption his question was sincere, I decided to pass the question on. Perhaps some readers are willing and able to respond to it, keeping in mind all that we have written on the topic, following Guenon and Guido de Giorgio.


\hfill

\texttt{apeiron on 2012-04-06 at 16:08 said: }

While the paganism of antiquity could interpret the divine, it only went half-way and this is discounting the prejudices of catholic dogmatism. Discursive thought is not unifying logos. That is the problem with philosophy. No matter how close the articulations are to the primordial (exempli gratia, Heraclitus) it begins the rationalist discursive cycle after the initial divine act has lost its potency. There is more power in the act.


\hfill

\texttt{Cologero on 2012-04-06 at 22:35 said: }

Thanks, apeiron, for the intriguing comment, but that did not please the questioner. Unfortunately, the vulgar can interpret the Comedy as a travelogue, rather than as a description of states of consciousness on the path to the primordial state.

To clarify a point, the pagan philosophers and poets inhabit Limbo, just outside of Hell. There is no suffering per se (other than, perhaps, the loss of hope). They live in the natural light of reason and a natural happiness. Their post-mortem state is not a negative judgment of Dante, who greatly respects those poets and philosophers. Rather it is a matter of justice, since that is how the pagans themselves envisioned the afterlife. We alluded to this when we mentioned Virgil's description of ghosts or shadows, from Aneas' descent to the underworld.

In an upcoming post, we will deal with Evola's description of the afterlife. While we respect Evola, this will show his limitations since he also cannot conceive of higher states. Evola did not know the occult name of Rome.


\end{sffamily}\end{footnotesize}
