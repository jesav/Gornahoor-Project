\section{Independence Day}

\begin{quotex}
Man is neither the slave of his race nor his language, neither of his religion nor of the course of rivers, not yet of the direction of mountain ranges. A great aggregation of men, healthy in mind and warm in heart, creates that moral consciousness that we call a nation. As long as this moral consciousness proves its strength by the sacrifices that are required by the abdication of the individual to the benefit of a community, it is legitimate and has the right to exist. \flright{\textsc{Ernest Renan}, \textit{What is a Nation} (1882)}

\end{quotex}
\paragraph{The Liberal State}
Renan expresses the liberal position of the 19th century: a nation is the result of Will, not Destiny (natural or familial relationships), certainly not Providence (divine sanction). Every July 4, the USA celebrates its independence day as the first liberal state. It is to nations as Esperanto is to languages. It is instructive to compare its founding to that of traditional societies.

The \textbf{Declaration of Independence} expressed the fundamental principle that governments derive their just powers from the consent of the governed. This is the opposite of the traditional view that temporal power is derived from spiritual authority, or the consent of God or the gods. This gave them remarkable stability and they typically endured for a long time. Of course, that really meant that they were really based on a transcendental principle. But the masses are fickle. As Renan explains:

\begin{quotex}
We have expelled metaphysical and theological abstractions [i.e., any transcendent principle] from politics. What remains after this? There remains man, his desires and his needs.

\end{quotex}


A man's desire is insatiable and his needs vary with circumstances; hence the liberal state will always be subject to such vagaries. Since desires are infinite and resources finite, the liberal state will always experience internal conflicts and its primary task is to arbitrate competing interests. As long as resources are sufficient, stability can be maintained.

Nevertheless, stability is not stasis and power blocs will always try to grab more. This is actually enshrined in the Declaration: In response to a ``long train of abuses and usurpations", the \emph{governed have the right and duty to throw off such government}. That is why political discourse is so often framed in terms of abuses and injustices; that justifies all attempts to radically alter the mores and laws of the country. Pathetic appeals by so-called conservatives to the ``original intent" of the Constitution, therefore, are quite beside the point.

The other useless tactic of conservatives is to argue in terms of transcendent principles, which are irrelevant in a system whose only criteria are the passing whims of the governed. If everyone is equal, then there is no hierarchy. If there is no hierarchy, there is denial of divine order. That is why revolutions are always atheistic, both the French and the Bolshevik. The masses profess atheism, at least materially if not formally. This is the insight of \textbf{Valentin Tomberg}.

\paragraph{Theocracy and Republic}
In his \textit{Hermeneutic Interpretation of the Origin of the Social State of Man}, published in 1822, \textbf{Fabre d'Olivet} provides a panoramic history of the human race, not in terms of facts, but in terms of the inner forces behind the events. Greece and Rome were theocracies under the reign of a priest-king before they became republics as we pointed out.\footnote{\url{https://gornahoor.net/?p=2554}} Yet, they still retained the idea of divine sanction. Plato said in the Laws that is was necessary to consult first the oracle of Delphi. All of the republics of ancient Greece invoked the Divinity at their inception. Rome had a sovereign pontiff (chief priest), although his influence gradually waned. Fabre makes some interesting points about the USA, less than 50 years after its Constitution.

The USA were ``the fruits of a political schism whose principal aim has been to destroy sacerdotal authority. No sovereign pontiff exists in the United States and cannot exist there." He then makes a telling point that nearly 200 years later, conservatives still have not grasped:

\begin{quotex}
By quite a strange inversion, it is possible in this republic that all the citizens are religious without the government having the least religion; that they are all pious, even devout, virtuous, scrupulously upright, without the government having the least piety, the least devotion, the least virtue, the least probity. For the government is a purely political being which adopts the sentiments of none of its members, and which above all in point of religion, affects an absolute indifference.

Now, as this government has above it no spiritual power to which it owes account of its conduct, and that even God does not exist for it [that is, the idea of God is never part of its political acts], although it may exist in different ways for each of its members, it follows from this that it is really without religion in its political constitution and that the law which constitutes it and which emanates from it is atheistic. 

\end{quotex}
This is why conservatives are so muddle headed. They repeatedly point to the religious beliefs of the nation's founders as though they are of significance. The state itself is atheist; as long as the governed, who give it their consent, adhere to a specific creed, no one may notice it. Hence, as the demographic makeup of the population is altered, its whims change and it gives consent to other laws. New power blocs arise each claiming abuses (injustices) and usurpations (unequal sharing of goods), which requires a continual overthrow of the government (hope and change). Conservatives claim foul, when long held moral beliefs are suddenly overthrown. Yet, that is no more than the system behaving as designed.

So it is not Christianity that leads to democracy, liberalism, socialism and egalitarianism as some political commentators of the New Right insist, but rather the atheist state as exemplified in America, France and Soviet Union.



\flrightit{Posted on 2011-07-03 by Cologero }

\begin{center}* * *\end{center}

\begin{footnotesize}\begin{sffamily}



\texttt{James O'Meara on 2011-07-05 at 11:09 said: }

I would recommend James Kalb's The Tyranny of Liberalism for a quasi-Traditionalist [really, neo-Catholic] account of the phony `tolerance' of Liberalism, especially in terms of seeking a phony ``objective" balancing of mere desires.

Americans are so brainwashed by their ``civics" and ``history" classes [if, that is, they remember anything at all] and so ignorant of the rest of the world, that they continue to ``think" that the USA is ``the greatest country on Earth" and ``the envy of everyone else". Why 9/11, for example? ``They hate us for our freedom." I was much amused when Bin Laden himself gave the response I had given for years: ``Then why didn't I attack Sweden?"

Speaking of Sweden, Americans never notice that after almost 300 years not one nation has adopted their absurd and ineffective system of ``check and balances" etc. Not even when the USA imposes a constitution on them [post war Japan or Iraq, for example]. For the rest of the world, `democracy' means parliamentary democracy, a la the superior British system, not the crude cargo-cult version cobbled together by men who, as Jefferson admitted, went into politics because there was nothing else to do in this god-forsaken wilderness.

More to your point, though: the unique features of the American system are entirely religious. Concern with ``founding fathers" [Patriarchs], a constitution written [unlike England] and unlike other written constitutions not subject to more than trivial alteration by the voters [contrary to your presentation, which is more applicable to France or Sweden, say] but rather treated as a Sacred Scripture to be ``interpreted" by a sanctified group of priests in black robes searching out ``original intent" — what is this?

This is certainly Protestantism, and back of that, of course, Judaism, Protestantism representing the resurgence of `primitive Christianity' against the Roman tradition.

It is interesting to note that this Protestant institution, the Supreme Court, now has not one Protestant on it, divided equally between Jews and Catholics. Jews, obviously, and as for the ``Catholics," their leader, Justice Scalia's quoting the Talmud and working to found an Academy of Talmudic Law show where his loyalty lies.

Tradition, of course, is not the source of Liberalism etc. [see the Syllabus of Errors] but once the unity of Church and Empire was broken,the result was Protestantism, the secular state, Liberalism, socialism, etc., back of all of which is the `primitive Christianity' that had been held in check by the ``Roman" part of the Roman Church.

That current Liberalism seems to be increasingly ``officially" atheist [Hitchens, etc.] is irrelevant. Each stage advances, leaving some behind [just as you say, new groups are added to the mix]; 18th century Protestants bemoan the ``social gospel" types, they bemoan the agnostics, they bemoan the atheists, etc. Where, what's the next stage? Only the final abyss of the Kali Yuga.


\hfill

\texttt{Cologero on 2011-07-07 at 07:47 said: }

All good points except for: ``not subject to more than trivial alteration by the voters [contrary to your presentation"

Of course, since that claim is never made in the ``presentation". But the ``consent of the governed" is found in the idea of the ``living constitution" rather than the fundamentalist analog of the ``original intent". So no one bothers with the ``word" anymore. Over the past 60 years or so, decisions by the federal courts, sweeping and vague laws from congress, and the rule of the executive by rules and regulations of federal agencies have resulted in far from trivial changes in actual life.


\hfill


\end{sffamily}\end{footnotesize}
