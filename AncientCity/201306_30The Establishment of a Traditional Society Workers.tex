\section{The Establishment of a Traditional Society: Workers}

\paragraph{Metaphysics of Castes. Workers}
\begin{quotex}
This is the chapter on the workers from \textit{La Tradizione Romana} [The Roman Tradition] by \textbf{Guido De Giorgio}. In this part, De Giorgio reviews the metaphysical foundations of the notion of castes and prepares the groundwork for the discussion of the specific functions of the workers. In his understanding of the degeneration of castes, De Giorgio says here what Nietzsche should have said, had he had the proper background. What De Girgio describes as the ``principle of freedom" is an essential point to understand. 

\end{quotex}

They constitute the third and final caste that corresponds to the domain of Forms in the sphere of the active life. They should really form the penultimate caste, the last begin the Servile workers, but from the median and reconciliatory point of view in regards to a true and proper return to the spirit and traditional form most easily realizable in the current conditions of humanity, we adopted the threefold formula while knowing that in a more perfect traditional society the number of castes must be raised to four in analogous correspondence with the full realization that, as we pointed out, beyond Silence, it includes a state, the supreme, absolutely undefinable, state designated, in the most complete tradition, with the name of ``Fourth" and that is the domain of Absolute Ineffability.

This state comprises and resolves the threefold formula—Silence, Rhythms, Forms—in an integrative indistinction that distinguishes it precisely from them as the mystery of the irreducible unity in its ineffable essentiality. In it the divine cycle is completed totally and integrally. From here the analogous necessity of a fourth caste that would include the Servile workers whose activity assumes the most austere, most elementary, most terminal form.

The constitution of castes must be understood in this way, according to the great law of analogy, i.e., in relation with the divine cycle and not from a purely human and partitive point of view that would have no reason to exist through its fragility and inadequateness to the truth. It is not despotism that creates the castes and especially the servile caste, but rather a necessity of development inherent to the mirroring in the human world of a divine complex that is the one and only reality. The current state of humanity, the apparent abolition of castes, corresponds to the erroneous and confused vision that men make themselves of the divine world: the Divine Principle is currently for men everything they want it to be, i.e., an indecisive and vaguely imminent fog, as men, having abolished caste distinctions, are likewise everything that they want.

The vision of a divine chaotic level corresponds to the leveling human chaos where all pseudo-mysticisms, philosophies, and nontraditional currents flow together, constituting the impure bundle of solidified forms.

But the abolition of castes is unrealizable and they exist even if not explicitly admitted and recognized: their hidden, invisible existence, over which democratism has stratified a leveling haze, creates an incongruence that should strike every person equipped only with good sense. Everyone know that men are not currently in their place and that the division of the active life is arbitrary and unjustified, because everyone is pushed, constrained by circumstances, by his own impulse and not by a conscious force of order, of the state, and of duty. Everyone knows that there are servile workers who dominate and masters who serve, and that even in the two first castes there are atrocious anomalies and that in the third caste these are furthermore still more remarkable: that is due to the absence of the traditional spirit, to democratic leveling which believes it can enslave those who cannot be enslaved, the determinative characteristic of the inalienable and indivertible human person.

This lasts for centuries and the current confusion is only the result of a slow and progressive degeneration due to the purely apparent superiority of man over the divine, of servitude over freedom, of confusion over order, and disorder over true hierarchy. What determined this fall must be ascribed to the incomprehension of that which constitutes the nature and the form of freedom, confuses with the arbitrary, and forces on everything the humanizations of reality that is actually of an absolutely divine order. Now the constitution of castes is based on the true concept of liberty considered in its four essential forms: absolute liberty in the Priests, conditioned liberty in the Warriors, conditionality in the Workers, servitude in the Servile workers. The two extremes are represented by the first and last caste in as much that what is affirmed in the first is denied in the last and vice versa: they constitute the alpha and omega between which the distinctive qualification exist. Reuniting the two extremes in the unification of the absolute principle that is God, one reaches the integral equalization that cannot be realized unless in the divine place, which is definitely affirmed by all traditions. One will easily understand that by reflecting on the true purpose of castes, to arrange the imbalances in a hierarchy where every element is contained in its own sphere in a way to constitute a homogeneous whole in which nothing introduced disturbs the traditional linearity. Seen from above, the constitution of castes is not presented as a scale that goes from the higher to the lower, by as a system of concentric circles around an absolute point that is the traditional unity.

All the castes are therefore to be considered on the same level and is unfortunately the incomprehension of this elementary truth that has produced the democratic illusion and error. Instead of putting themselves at the higher point of view that is the divine order, men have descended into the human level that has no value when it is cut off from the divine and thus the revolt has happened in the sense of the same caste which, going beyond its limits, has flooded like a river that runs over the banks and violently levels what it floods. Then, after having cut off the human from the divine, with a logic of free will whose unconsciousness is truly stupefying, the divine plane is confused with the human and what is equal in the eyes of God, they want to transform to equality in the eyes of man, not knowing that equalization is possible only through a being who is beyond the plane considered and, as such, through His own summit, glimpses a unique level while this level does not exist below. They even go back to the sacred texts in order to justify this error of perspective, interpreting it in the most absurd way, overturning every traditional order, abandoning the true God to create a god in man's own image and likeness. It is seen in the abolition of true order, that distinctive, determinative and determined order of castes, the affirmation of the principle of liberty, not knowing that this in the absolute sense exists only in God, while in man it is conformity to the law of God. According to which every element of creation must remain absolutely in its place in order to be the normal element of nature. But since this democratic leveling was unnatural and restrictive, the castes were transformed into classes and the activity generally understood as a complementary way in the unity of tradition and contemplation, was transformed into work, i.e., into pain, coercion, since such is the meaning of the Latin word.

That explains the class warfare that is the degenerate form of disputes between the castes, which, as we said, is always limited to the first two for reasons of quite another order then those who fuel the claims of the current lower classes. Now ``work", properly called, can be applied only to the caste of Workers if this is considered as the last for the reason already mentioned, and not to the Warriors and Priests whose activity is absolutely of a totally different order. The latter are tutelary of the divine constitution, the former protectors of the human constitution, and neither of the two castes therefore leads by its activity to the satisfaction of its own needs: therefore in a truly traditional society, it is necessary that the last caste provides the maintenance of the first two, neither the Priest nor the Warrior, being able to work, and that in an absolute way, since work would impede the fulfillment of their very difficult task, the maintenance of the traditional order.

Referring to the circular symbolism which we just mentioned, if this represents the traditional world as a circle whose center is constituted by the Priests and the circumference by the Warriors, the very close relationship of these two castes in a perfectly organized society will appear: the Warriors represent the centripetal convergence that binds all the points determining them in the univocal axis. They, with their power, defend the traditional purity, which cannot be accomplished materially by the Priests who are contemplatives and who must be protected in the external order by the Warriors. In the convergence resulting from the harmony between these two powers, the two greatest energies of the traditional world, true, integrally solid, power is achieved, the forest of swords that protect God's enclosure. We recommend to everyone to achieve in its fullness the results that would be obtained from the harmony and the cooperation between the spiritual authority and temporal power through the true destinies of humanity brought back into the great traditional riverbed.

\paragraph{Divine Nature of work and art}

The Worker\footnote{I have translated the root lavor- as labor and oper- as work to be consistent.} caste includes every assignment operative in the individuals who do not belong to the two preceding castes which constitute the base of traditional society: authority and power. When we speak of activity, for this last caste, we mean every type of labor both of the strictly material order like the crafts and of those seemingly more elevated like the professions, since both are remunerated individually. In the two higher castes, we do not speak of truly individual activity since their members work constitutively for the maintenance of the two higher powers, while here, in the last caste, it is a question of an active personal determination whose fruit, however, remains limited to the individual order, while contributing to the general order. We are in the domain of Forms, the last determination of visible reality whose essential characteristic is individualization. In this caste, as in the two higher castes, persons can be included who in reality should not belong to it, which is the indication of the current disorder due precisely to the lack of traditional unity whose most obvious characteristic is the arbitrary distribution of wealth. But in a truly traditional society in which the castes are rigorously established, remunerated labor should be limited to the majority, i.e., to those who neither know how to nor can do otherwise, incapable of the pure contemplation of the Priests and the pure activity of the Warriors.

It is therefore necessary to insist on the analogy between the various castes and the three degrees of reality, i.e., Silence, Rhythms, and Forms, in order to understand their true purpose. If the Forms constitute the most exterior part, that does not mean that their determination does not reflect, in a thousand aspects, the invariable unity that is the one true reality. Every form is a symbol and every symbol is the vehicle of a profound truth whose importance must never be forgotten. So-called objects, things, are likewise mirrors that variously reflect the unity of the creative rhythm in multiple aspects. Man, in ordinary life, makes use of tools that he himself constructs but whose symbolic meaning is currently unknown, while in a traditional society this meaning is precisely what is more important because it provides all the value to the thing without which it would be deprived of universal purpose. Hence, the necessity that the tool be patiently constructed and not serially, with an individual labor that itself is the symbol of the effort through which it is elevated to a higher reality of an absolutely spiritual order.

It will be easy for everyone to distinguish the value of an object, the fruit of patient and assiduous work, quite different from that accomplished by a brutal, external, artificial, and sterile process like an assembly line. Hence, the artistic beauty of the most humble tools of an era and the banality of what is produced in modern times by the machine. Previously, there was art, the profound meaning of symbolic correspondence, and the indicator of this was exactly the care, the commitment with which every object was constructed by the strictly personal work of the craftsman with a diligent technique, in strict analogy with the spiritual renewal wrought in the contemplative and ascetic domain. Every craft then symbolically represented the fixation, in Forms, of a process of an absolutely spiritual order related to a higher reality of which the material world is an appearance, in the strictly etymological sense: man, the creator, makes use precisely of so-called matter—not neglecting the etymology of this word—that is the last, final concretion of reality, in order to redeem it from its apparent blindness and lead it to the transparency of an analogical correspondence with a higher world. So that while in the already made Forms among which man lives, that are the patterns of the Rhythms, are likewise found mirrors of the higher reality, the objects, the things, the tools that man constructs are new forms and represent the work that he must undertake laboriously in order to liberate himself from his humanity and restore in himself the divine state. Ars et labor: art is the knowledge of the analogical relationships that govern Forms, by means of Rhythms, and achieve them, i.e., make them porous, transparent, to the breath of God who expresses them, while labor is exactly this effort of enucleation of the deep, hidden, veiled reality, protected in matter from which it must shine through to reveal itself to man. There is thus a contact between the interior—man—and the exterior—nature—which is resolved in a single reality of realized and experienced expression, therefore abolished in its crude materiality and restored to its true origin and purpose.

Thus understood in its profound meaning that justifies it and makes it necessary, art is a redeeming purification that reestablishes the creative rhythm obfuscated and neglected by the preoccupations of ordinary life. This life, in an absolute way, is a death, not a life, i.e., the avulsion of the world and man from the true reality of the world and of man realizable only if hidden in divine reality whose origin it symbolically expresses.

Tools, the most common objects of use, were not created for the satisfaction of our needs, but only to express the analogical relations between appearance and reality, between what appears and what is, between the world and God: their practical efficiency is of the second order and is of value only for true serfs, i.e., for those whose intellectual myopia is so widespread that it results in the exclusion of every truth beyond the environment of their terrestrial life. If, in a state of primitive perfection, it is necessary to think of the exclusion of every order in order to reach or extend what was already given, if man, in this state, gathered the fruits of the earth and fed himself with them and did not complete, with his work, what naturally surrounded him, in later stages, distancing himself from this original standard, he had to reconstruct the access to the divine from which he had fallen and thus a new necessity: art.

To those who know how to examine in depth what we said, the relationship between art and life will appear evident, initially unified so that art was the life itself considered as a rite, then increasingly more discordant, life was reduced to its lowest function and art limited to those who cannot decide to reject truth, up until the current time in which life is really death and art, deprived of every sacred and realizing character, is an expressive monstrosity wherein all the misery of the world and man is reflected.

Art is not redeemed by making of it a need of the spirit: the spirit, and in an absolute way, is nothing if not the Spirit of God, i.e., the breath that is in life, that penetrates and informs the whole man, that makes him feel, act, think according to God, not according to his own humanity. What is human remains human, therefore purely bestial and inferior, which is the level of this humanity, just as everything that is iron remains iron from the most common utensil to the most refined artistic product, both able to lead back to the unity of origin when their exteriority is removed. In fact, what constitutes art for modern men is precisely exteriority instead of the sacred, symbolic content, what this exteriority expresses by relating to a reality of a transcendent order in the absolute meaning of the word and to a truth of divine order.

Man, in the development of his human faculties, remains man, i.e., nothing: what he feels, lives, accomplishes, thinks, if not beyond the human environment, is destined to perish because it cannot extend beyond time that is succession and beyond space that is materiality.

He remains closed in this prison that he adorns with the most lavish funereal pictures, so that to the extent he embellishes it, so much more will the tomb will exist for him as long as he lives: but, after the dissolution of the body, who is there to take care to lead him into the place of truth, dooming him eternally to that death that he had already experienced in life by rejecting every effort for the surpassing of human limitations. Homo humus: as long as he remains earthbound, he will be doomed to fecundate the earth and to perpetuate the lucifugous illusion that is the detrital, lower world: it is necessary that he clear up the illusory obscurity, due to ignorance, in the light of truth and that he expresses from himself what is hidden, i.e., the other half, which could mean the passage from the archaic word hemi, referring to the general meaning of half, to homo, where the circularity of the o represents the realizing universalization of all the human faculties transposed into the divine and integrated in a essentially superhuman plane. Art is the expression of this transposition that is a true and proper transformation, i.e., a surpassing of the form that is obtained by replacing it in the plane of its normal purpose as symbol of a higher truth. But it is necessary that art is in everything and not in part, that it is not exiled from life and that it does not represent only what we could call the kingdom of fringe utopias, but rather it imprints on the most humble of objects, tools, the seal of its symbolic purpose.

This is craftsmanship, these are the crafts: to depict in every material substance, with a labor of diligent understanding, the intimate, symbolic value, expressing a truth of a higher order, from agricultural instruments and those of the weaving mill, from the most common objects of wood and earth to the construction of house and temple. They are various modes of expression of a unique reality that signify as all the ways lead to God if truly it is God who is sought and not a simple derivative human more or less dressed up and idealized for the use that one wants to make of it.

\paragraph{How Science Displaces Work and Tradition}

Hence, the corporations also originally represented modes of realization of the divine organized in a regulatory body that established its purpose. This is the real and deep significance of art, craftsmanship, professions, corporations, and let us not pretend that they all come to fathom it, but what is important is that the pauci optimi [the few of the best], who ought to participate in restoration of tradition, understand what is hidden under the appearances of ``utility" and ``activity", expressions so dear to those who understand nothing because they realize nothing beyond the illusion of the world and man, both considered without relationship to divine reality. If man is one half, he has to search for the other in order to be made whole, i.e., to be truly man—hemo homo—and one could say, without making a game of words, that he then will truly be man when he ceases to be so. Likewise the world—if it concludes its etymology—that is attired, ornamented, under which the truth of God is hidden, then it will truly be the world, i.e., the place of purification and resurrection, when it is the world, purified from every ignorance, likewise it also will become—the lapis manalis\footnote{\url{http://en.wikipedia.org/wiki/Lapis_manalis}} removed—the wide abyss for the precipitation in the spectral kingdom.

From everything that we have mentioned, one easily understands how and why—not for simple purely aesthetic reasons, therefore negligible in themselves—the machine, in all its forms, represents a veritable profanation of labor, because it removes every sacred character, every profound meaning, violating its secret, denaturizing its purpose, suppressing all those seeds of redemption that constitute the raison d'.tre of manual work and of art.

The return to a traditional society would entail a prudent progressive normalization that would be, for the ignorant mass, a true and proper regression but which could be accomplished gradually without creating catastrophes. It is enough that the pauci optimi understand, to their full extent, the absolutely positive consequences that would result from a comparable return to normality: it is a question here, more precisely, of the human dignity that would be restored and reintegrated in all its hierarchical development, in all its expressions, in all its aspects, in all its fields, in all the castes, in all men.

In order to become sacred, labor must be accomplished by man and not by the machine that, as is clearly observed, takes revenge on man, destroying him in the most blind, bestial and inhuman way, violently and fatally. A renovation worthy of the name entails the return to the norm of human labor, a restitution of those conditions of existence suitable for the development of the great energies that lie buried in man and that can permit him to reach and truly integrate the divine world of reality.

Labor, once connected to its necessary base of meditative concentration, of patient meticulous contact with the world of Forms, in order to perceive the Rhythms and realize the Silence, would again become sacred for all men and each one could, according to his possibilities, accomplish what he is destined to do and that he is strongly removed from that outrageously superficial and profaning modernity.

May the earth, the sea, the sky be restored to their elementary purity, returned to their symbolic purpose, the earth in order to construct house and temple, the base point for the elevation from the human to the divine, the sea to the navigation between the two shores that separate the transient from the divine, the sky to the penetrative flight into the reality of God that is the only reality, and that the machines are gradually, prudently reduced and eliminated in order to make a place for man and put him visibly, without intermediaries, in the face of the difficulties of his celestial task on the earth. Because he cannot escape from the earth if he does not know it, if he does not penetrate it, if he does not make it transparent, letting the holy spirit of the real achievement pass there, returning to labor and art, enriching again the impoverished world, mutilated and profaned by the machine. This was created by the lowest part of man, from what in man is the negation of man because it is the negation of God. We allude to profane science, to that which for so many centuries gives itself the false privilege of redeeming humanity from the chains of materiality.

Here a position of the absolute level that does not admit subterfuge and loopholes is necessary. There is only a single science, sacred science, in an absolute way, i.e., the knowledge of true man restored to his elementary function, to his base, to his centre, to his reason to exist, to his life; to his being, to his purpose, to his perfection, to his universality, to God. This is the only science, the only power derived from the knowledge of his own limitations—it is well established—and of exceeding the limits of terrestriality for the effective realization of the divine possibilities: in this sense only scire est posse [to know is to be able] and not otherwise.

One goes to God with and in the spirit of God, becoming again sons of God, be it formulated in the integrative fullness that does not admit secondary residues and disturbances. All traditions affirm it, with different expressions, in various forms, but with an intentional unity that no one can misunderstand and confuse. Ars una species mille. What is sacred remains such in every tradition of truly divine order and every race is given a tradition conformed to is possibilities, to which it must remain faithful in order to not make the task of traditional integration difficult, while it is permitted only to very few to return to the Primordial Tradition that is in direct contact with the divine plane.

The true science is therefore that contained in the traditional body and has as its goal the return of man to God in all the forms, through all the levels, and according to all the possibilities. Sacred science is the true absolute and definitive knowledge of man and the world in God and comprises various planes of development according to the domain to which it is applied, each one of these planes, however, not able to, nor ever having to, be considered in itself, but all in convergence, in accordance with the unity of the traditional axis, into a single point. This was maintained in antiquity and in the Middle Ages. Science instead is of the exclusively profane order and considers visible reality exteriorly, as it appears not as it is, by integrating itself in such a case in the invisible reality and, taking as the base point exteriority itself, it determines laws.

Therefore, while sacred science is fixed, immutable, eternally constituted and causes stable and permanent traditional societies, profane science is mutable, unstable, progressive and, with its development, gives origin to anti-traditional, precarious, and transient societies.

The domain of sciences is the visible, i.e., therefore, the superficial that, as such, separated from the body to which it belongs, is illusory and as nonexistent; the successive development of scientific hypotheses shows the inanity of an effort destined to remain sterile, unproductive, since truth indefinitely postponed into the future, i.e., never realized, is not a truth. What is concrete and positive for science is truly the ephemeral and the negative, i.e., the phenomenal appearance considered in itself, because as we said, when brought back to its invisible root, it reacquired another meaning, another attribution, and another reality.

This sign in the theoretical field can suffice: how much then is the development of the so-called western civilizations the most conclusive demonstration of the practical advantage derived from the applications of science? One thinks of the misery, the precariousness, the frivolity of current existence and one will understand what results science applied to life can lead to. It is not allowed to insist on that without falling into the realization that every reasonable man can currently make, considering the shortness of the duration of life, its sussultatory character, the rising devirilization of humanity, the precariousness of everything, the spasmodicity of every bond, the insecurity of every system, finally the instability that is the indicator of a permanent abortive process due to the absence of traditional fixity.

Science, having become secular, i.e., popular, became the dominion of everything because it is accessible to everyone through its exteriority, superficiality, and facility and was the principle stimulus to the so-called servile awakening, to the progressive democratization of Europe since, after eliminating every true traditional knowledge and excluding every spiritual surpassing, it leveled hierarchy and, with its practical industrial applications, it amazed the foolish and caused libertarian unrest in the masses, of which the current world is the strongest and most authentic expression. Humanity let itself deviate by easy explanations, by immediate practical applications and has forgotten to ask itself up to what point could it reach such a pronounced deviation from the traditional axis. First of all, science has infinitized so-called nature, divinizing it and exalting its mysteries as if there could be other mysteries outside the divine, then it made a stir on a problematic god who follows with his benevolent eye, approving, the progressive contaminations, and finally, acting autonomous, was proclaimed the guide of research and master of life. Man, destitute of all his power, i.e., the very surpassing of his own humanity in order to reach higher and decisive states, was given instead a false creativity in the world of exteriority that lasts as long as his life and is prolonged as long as he is ignorant.

\paragraph{Conclusion}

Even those who say and believe they adhere to a tradition have accepted science as the expression of an achievement that, according to them, would not collide with the truths of the divine order; more impure and blind than the others, they implicitly repudiate every spirituality and show they do not know the fundamentals of sacred science: this is exclusive of every knowledge that does not replace the elements of creation in the circularity of dependence by a unique center of which they are only appearances. Men, accustomed to the separative visions of things, in considering elements and forces, distanced themselves from integrative contemplation for which everything loses precisely that material and determined consistency that is the foundation of scientific investigation. Science rises when traditional knowledge no longer succeeds in maintaining supremacy through the disintegration of its own unity: philosophy, also secular, has substituted for sacred wisdom.

It is a fateful cycle that is fulfilled and whose origins must be traced back to the very nature and the frame of mind of the people of the West incapable of maintaining intact the traditional solidarity in order for them to be brought to the exteriority of the active life. With the progressive dimming of sacred motifs secular motifs arose and the attainments that man was no longer able to accomplish in the domain of the supersensible were limited to those that were easy and exterior for a childish craving for visible concreteness. While the creative process starts from the internal to the external, the scientific process comes from the external, which does not exist as such, to a purely nonexistent and ideal internal subordinated to that which does not exist except as depending on appearance: hence the scientific hypothesis claims to be the standard of method.

The current condition of humanity and antitraditional blooming are due in great part to science, to the scientific spirit, but also to philosophy that, with science, has substituted for sacred science. Philosophy, subordinated to Revelation, is a preparatory and necessary stage to higher truths that must be integrated, experienced, in order to be known: without revelation, it is sterile and false and is transformed into veritable diverting nonsense. Philosophy is based exclusively on reason, which presupposes a revelatory light in order to be guided to truly decisive ends be it only in the theoretical sense: isolated from this, it cannot find any absolute point of reference, because the absolute is beyond reason and beyond man in the sense that these cannot approach it unless ceasing to be it, which does not prevent them from passing through his terrestrial existence, as everyone will easily be able to convince himself considering that Ascetics and the Saints, those, i.e., those who from this life have reached a permanent state of beatific vision that is fulfilled in eternity, certainly not able to be realized within the limits of time and space. Philosophy is a preparation to sacred science and can constitute its theoretic side only if it is subordinated to it, because otherwise it is nothing: it would not even be of interest to show how and why, from the Renaissance on, modern thought unwound following a direction that is ever more antitraditional in order to be liberated from sacred science that necessarily must precede it because it is focused and productive.

To what extent can Pascal's famous phrase\footnote{God of Abraham, God of Isaac, God of Jacob — not of the philosophers and scholars.} suffice to denounce the incongruity and inadequacy of the god of the philosophers and poets. Divine reality is revealed and no philosophy, if it remains such, can realize it: this possibility is to be excluded in an absolute way: all the acrobatics of human reason will never succeed in grasping that which, being simple, original, and absolute, offers itself from itself through direct realization and cannot actualize itself through a discursive process that is only direct and mediated. Moreover, after what has been said, in the domain of divine unity there is only divine unity itself and what is human is not excluded from it as simply nonexistent, or, better said, as existing only as long as ignorance lasts: this dispelled, and sacred science tends precisely to that, true and proper realization is initiated that, as unitary and passing beyond what is produced and generated, can rigorously be called metaphysical and metarational. We could also call it ``intuition" although no psychological quality is given to this term: the psyche in fact is below the spirit, the intellect, the heart—these three terms denoting, under three aspects, the same type of integrative activity of the divine. The spirit expresses the direct integration whose absolute type is the divine breath, the intellect expresses the cognitive permeation, the heart expresses radiant receptivity: by means of the first, one is elevated, with the second, absorbed, in the third, one is welcomed and realizes himself. Representing here a vertical axis, the spirit is the peak, the intellect the base, the heart the center that gathers the two extreme points and extends them, prolonging them horizontally, hence the Cross as radiant symbol of universality and unifying centrality.

No one will dare assert that philosophy can be elevated to this sphere which is of an integrally revelatory order because it is superhuman in the absolute sense of the word. These truths constitute exactly the traditional body, the sacred science, whose transmission happened in a divine way: this knowledge is the true knowledge for the absoluteness and legitimacy of the efforts accomplished in the same axis of truth.

If the Priests are the depositories of it, all men, even the Workers, must extend themselves by giving to their activity a sacred intention and not considering it as the pure satisfaction of their needs and desires. These cannot be, in the seat the truth, distinct in the lower and higher, material or spiritual, in the absolutely erroneous meaning that the moderns give to the last term when they apply it to that which is called the cultural environment. Culture is a profane thing and does not come close to true spirituality: we admit that it perfects the sensibility and develops the intelligence, but it rests completely outside of that which is sacred and real precisely because it is sacred. It is something empty and superficial and is reduced to a vision of life where all the prejudices of the modern era flow into each other, from so-called historicity to linguistics, all exterior sciences that leave intact the domain of the true reality which evades any separative analysis and false constructive synthesis. As behind the Forms there are Rhythms, so behind the visible there is the invisible and as modern sciences stop only at what is expressed but not at that which expresses. Actually, cultural prejudice contributed to the current decline with a dense network of loci communes [commonplaces] which is applied indiscriminately to the present and the past, all being considered sub specie alteritatis in its own function and not as the reflection of a quid that eludes exterior analysis. This is not the place to insist on modern deviations which have behind them a centuries-old preparation which led to the current madness. Nevertheless many symptoms of nausea, fatigue, and glut bring hope that a radical change can still be produced in a progressive way and without too violent perturbations. This rectification can and must happen from the inside to the outside, i.e., it is necessary that the orientation of thought changes and determines the progressive return to normality: let us say ``return" meaning especially a restoration of the traditional spirit because the exterior forms can never be reproduced; the cycles do not renew themselves and do not return backwards, repeating developments already accomplished. Nor is it possible to predict how this traditional return will happen if first it is not really initiated with stable foundations and with roots that sink into the very core of the integrally restored traditional unity.

The sad anomalies of current life, the discomfort of individuals and peoples, all the previous misunderstandings serious in themselves but fueled by freedom granted with impunity to those who are not worthy of it, the impoverishment, the progressive cover-up of existence, the vanity of facts, the lack of a secure direction capable of satisfying not only the small needs of small men, but above all the needs of integral man in his true revelatory function of the divine essence—all that brings hope that an awakening can and must be accomplished provided that the pauci optimi, conscious of this necessity, does not let itself be submerged by the ignorance of the anonymous and profane crowd.

But if we start from the divine truth as the basis of orientation and we restore the axis of convergence integrally, an endeavor to which the Priests and the Warriors must contribute to, consciously united as being truly the sustainers of tradition, it unites them because the former contemplate and by contemplating realize, the latter because they liberate activity from its contingent utilitarian character developing it in protective adherence to what is sanctified in the Temple through the maintenance of the sacred deposit, then truly the last caste, which is the most numerous and constitutes the great mass, will also reclaim the Sacred Path where every feeling, symbol of a higher purpose, is established in a single direction like a river from a thousand ripples contained within the same banks, and will be reconstituted, with the fascification of all the energies turned to a single end, the new trunk on the old roots which are still alive and ready to regerminate because revived from the eternal breath of God.

\flrightit{Posted on 2013-06-30 by Cologero }

\begin{center}* * *\end{center}

\begin{footnotesize}\begin{sffamily}



\texttt{Graham on 2013-07-02 at 00:12 said: }

There are three points somewhat unclear to me. 

(1) He speaks of a `Fourth' state which unifies the others in ineffability, and of an analogous `fourth' state of Servile Workers. Am I correct in understanding that they comprise those who are `outside' the caste system, the former by being above it, the latter by being beneath it? 

(2) Later he speaks of work versus activity. Am I right in taking this distinction to have some relationship to the principle of freedom? With respect to the circular symbolism he employs, do the rings nearer the centre `act' in a higher degree, while those approaching the periphery `work'. It is unclear to me. AK Coomaraswamy writes that the artisan's activity is composed of both contemplation (in conceiving the work) and servile labour (in executing it). Personally, I have found that purely intellectual endeavours feel more nearly like work (in the sense of pain and toil) than do endeavours where `theory' flows into manual activity.

(3) The meaning of the principle of liberty escapes me. Man's liberty is conformity to the law of God – so far so good. Caste forms a part of this law, self-evidently, so we can say that far from being oppressive, it is most free — a man is most free when he occupies the station conforming to his nature. But can we still say that the upper castes, insofar as they are closer to the centre, have `more freedom' than the lower castes? There is something quantitative about this thought, and which is incongruent with my experience of being more free when operating within my natural capacities.

I do hope that the next installment will have more to say about the role and possibilities of the Workers. As an aside, for those who may have forgotten, Guenon's Reign of Quantity has several excellent chapters bearing on the crafts and their spiritual possibilities.


\hfill

\texttt{Avery on 2013-07-03 at 21:41 said: }

Graham:

(1) By analogy of India, the fourth caste is part of the initial caste system and is not outside it. Arguably the creation of outcastes is unbalancing and therefore undesirable in Tradition.

I wonder if this missing fourth has anything to do with Plato's Timaeus and Critias, where the fourth is a necessary part of stability in three dimensional geometry, even though fourths constantly seem to go missing in those dialogues.

(2) Can we get the original word Giorgio is using here? He doesn't seem to be talking about work (opus) but labor. Labor, in both its masculine and feminine meanings, is a necessary task but one that requires less intellect than work.

Now some writers claim that work is supposed to be inferior to philosophy. Hannah Arendt places both contemplation and politics above work in classical civilization, and only contemplation in medieval civilization. In my opinion, though, an opus is something that, despite being physical, rises above the animal need for labor, as reflected in the ornate work put into all physical objects at the height of great civilizations.


\hfill

\texttt{Jacob on 2013-07-03 at 22:26 said: }

This discussion reminds me of Josef Pieper's ``Leisure, the basis of culture". If you guys have time it's worth picking up. Fairly short, but very good.


\hfill

\texttt{Cologero on 2013-07-03 at 23:54 said: }

RE (2), this is the sentence: ``le caste si sono trasformate in classi e l'attivita' generalment intesa come via complementare, nell'unita' traditionale, di quella contemplative, si e' trasformata in lavoro, cioe' in pena, in costrizione, poiche' tale e' il sense del termine latino"

Rather than the normal word ``lavoratore", he calls this class the ``operarii", a neologism.

Why would Arendt put the nobility with the workers in the medieval era? They didn't think so.

The worker's freedom is conditioned since he works with forms as an individual, unlike the Priests and Warriors (although the difference between ``conditioned" and ``conditionality" is unclear, at least in English.) The Priests tries to transcend form, and the freedom of the warriors is conditioned by his necessary activity in the world, but not by individual forms, they deal with the whole. That does not mean the worker is not free.

The labor of the animal is always natural. The work of man is transcendent and is the reflection of a supernatural reality.


\hfill

\texttt{Avery on 2013-07-04 at 01:09 said: }

Thank you for the original quotation. Giorgio's creation of a neologism shows that he understands and maintains the essential distinction between labor and opus.

``Why would Arendt put the nobility with the workers in the medieval era? They didn't think so."

If the caste system functions as an order then no part can be disregarded; the whole thing is equally traditional and must remain in balance. I made a very bad summary of Arendt on the way to that point. She was not trying to lump workers and nobility together, but rather show how politics was degraded from a formerly sacred status. Classical politics was seen as part of the pursuit of the sacred, while medieval secular power became distinct from spiritual authority, although it aimed for harmony between the two. (Modern politics, of course, is even worse!) The classical philosophers focused much of their efforts on trying to perfect the earthly city, but Augustine told us to abandon the earthly city and seek the divine. Obviously Arendt's argument was way more complex than this but my memory works best with concrete examples.


\hfill

\texttt{Michael on 2013-07-04 at 20:29 said: }

Has there ever been a culture that has gone from being like ours to having castes? I can't think of one in recorded history, but there might be something in mythology that speaks to moving from a dark age to a traditional society.


\hfill

\texttt{Cologero on 2013-07-05 at 09:18 said: }

Following the dissolution of the Roman Empire, there was a concerted effort (documented in Georges Duby's ``The Three Orders") to re-establish society on the basis of the three functions described by Georges Dumezil.


\hfill

\texttt{Logres on 2013-07-14 at 22:23 said: }

This is fabulous; particularly the passage on the cross being the symbol of heart, intellect, \& spirit united. I think there are many ``workers" who know something is wrong, \& are working, with the hope that the Priests \& Warriors will begin to act in time.


\hfill

\texttt{Michael on 2013-07-17 at 21:13 said: }

This is a bit off topic but related since de Giorgio is talking the three orders. To what extent is membership in one of the orders by birth? Evola talks about the priestly caste being twice born, but that would mean that some of the upper caste (or the priests or nobility) are not really of that order if they are not initiates. I think of the nobility of today and it does not seem that they are particularly noble except for birth.

If being twice born is the criteria, then who makes the call? 

I apologize if this has already been answered elsewhere.


\hfill

\texttt{Cologero on 2013-07-18 at 23:39 said: }

Michael, the three castes that Guido De Giorgio wrote about should be ``twice born" as they still are in India AFAIK. Not so long ago in Europe, the priests obviously had their orders, the nobility would probably belong to some order of chivalry, and the workers had their guilds.


\end{sffamily}\end{footnotesize}
