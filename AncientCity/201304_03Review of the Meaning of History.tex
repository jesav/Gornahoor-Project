\section{Review of the Meaning of History}

\begin{quotex}
This review by Julius Evola of Nikolai Berdyaev's book The Meaning of History, in its German translation, originally appeared in Bilychnis, volume XXXII, August-September, 1928.

This book and review was published some 75 years ago, when the events they describe, like the frenetic self-assertion of the ego and the negation of man into some biological being without discernible qualities, were visible to few. Although the two men disagreed on some finer points, they did not disagree on what what happening then in its germinal stage. A couple of generations later, what they saw coming is now in full force. Whether that mystical third stage will arrive over the ruins of the modern world remains to be seen. 
\end{quotex}

History for Berdyaev is an originary spiritual phenomenon identical to the very depths of Being: it is an absolute category, an apriori in respect to it; everything that is outer determinism material forces, phenomenon, or form are only subordinate parts that is includes in itself. Man is essentially history, and only from history—not from psychology, sociology, or physiology—can he be understood in the concrete fullness of his spiritual nature. He has in himself the entire history, as in a microcosm, yet not outside but in the deep strata of his interiority lies the same abyss of the times.

This essential identity of man and history give Berdyaev the way to resolve, actually to not even pose, the vexed question of the ``possibility of history as a science", or, if one prefers this expression, of the ``relativity and phenomenality of the historical consciousness": history for him rises spontaneously as objective knowledge on the basis of a spiritual recollection, almost like a Platonic anamnesis.

Truly a miraculous anamnesis, because it permits us, following Berdyaev, a sudden theologization of history; what would be human history and divine history at the same time, in the sense that it would express the metaphysical in the physical,. God would reveal himself in history, or rather He would be Himself history, since He considers history in an eternal and continuous way, and not in the fragmentary form of man's experience of time in three dimensions (past, present, future). The historical eras are eons of the life of eternity, fluent just to gather in the womb of eternal life, where time is in the form of a transcendent: this is not a new image and that would be difficult to become something more than an image.

In any event, according to Berdyaev the theophany is not closed up in a divine solipsism but has two poles: it is the tragedy of interior free relationships between man and God: a birth of God in man and of man in God, a self-revelation of God to man and of man to God in accordance with a mystery and a dynamism of spontaneity that exactly constitutes the meaning of history.

Berdyaev, descending a little more to earth with a a clear and elegant treatment, then tries to find again in the development of historical epochs the meaning of that theandrophany. Thus the famous doctrine of the three epochs reappears. The first of which would be established by mythology, understood as effective experience, i.e., as an animistic state in which man found himself in relation to a living connection with gods, demons, elementary forces of nature.

A second decisive stage follows, always, according to Berdyaev, from Christianity in which man surpasses this promiscuity, unchains himself from nature and then he opposes it, combats it, tends to dominate it—and so fulfills himself as a free personality while, on the other hand, nature closes itself in on itself, and becomes a material and impenetrable reality. This epoch includes humanism, understood as an orgy of freedom, as an uncontrollable self-manifestation and self-assertion of the ``I" in all fields—a tendency that however would end up with overturning (that would happen in a special way in our days) through the law that every self-assertion of the individual, when the individual puts himself as the end in itself and does not recognize a higher order, he transforms himself fatally into negation. The coming of the machine and socialism would be two consequences and two manifestations of this law; products of the individualist expectation, they lead to an inorganic and exterior relationship, to a depersonalization of man, to an absolute dependence from brute and automatic forces, and to purely economic collective interests.

In this state of crises it may lead to a third era, in which man, in order to feel himself free and individual, will overcome isolation and egoism, and will return to an organic, living, religious connection with the totality, realizing the mystical meaning of history as the reciprocal self-revelation and self-generation of man to God. We said: ``it may lead", insofar as, letting history finally breathe a little, freeing it from that veiled form of historical determinism that is every ``philosophy of history" in its concern to deduce historical facts from a higher law in order to understand them and a free and irreducible nature; he admits, in this solution (that would constitute the Christian, as opposed to the humanistic, Renaissance), the possibility of ``evil", i.e., it would instead bring the world to the depths of individualism and the machine.

The book can be read with interest, the style is rather clear and meaty. Like every philosophy of history, it is necessary to take it more as a work or art, or as a pretext to justify the choice of values of certain objectives for the epochs to come, rather than as a science and a knowing; something that, in our opinion, needs a rather more cynical, and rather less mystical, eye.



\flrightit{Posted on 2013-04-03 by Aeneas }

\begin{center}* * *\end{center}

\begin{footnotesize}\begin{sffamily}



\texttt{Mihai on 2013-04-04 at 04:20 said: }

Berdyaev's book is excellent. The chapters on the analysis of the decline of the modern world are interesting, but more valuable are the opening chapters on the hermeneutics of history, since they can be applied inwardly, to a genuine knowledge. 

The only lacking part is where he brings the typical Russian ``tragedism" into the Supreme Non-Being of God. Admittedly, he does describe it as a mythological template, but it makes me feel somewhat uncomfortable. Guess it's because I'm more philosophical than poetical…


\end{sffamily}\end{footnotesize}
