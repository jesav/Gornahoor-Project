\section{The Valorization of the Middle Ages}

Occasionally the question comes up, as it has again this week, as to why we focus so much on the spirituality, symbolism, social structures, and history of the Middle Ages. Although the answer can be pieced together from various posts, it may be good to summarize it in one place. Clearly, the primary authors who have awakened the knowledge of Tradition regarded the Middle Ages as exemplifying the Traditional spirit. Since it is closest to us, at least those of us in the West, in time, language, and culture, it seems to be the most suitable model for demonstrating tradition in action. This, despite the fact that Rene Guenon points out that the contemporary West is as distant from the Middle Ages as it is from the Asian civilization; after all, isn't to be called “medieval” considered an insult today?

In 1937, \textbf{Mircea Eliade} wrote an essay, “The Valorization of the Middle Ages” in which he mentioned \textbf{Julius Evola}, \textbf{Rene Guenon}, and \textbf{Ananda K Coomaraswamy} among the exponents of those elites who recognize the importance of the symbol and the primacy of transcendence in the Middle Ages. He followed that up with two radio programs on the Secret Language of \textbf{Dante}, the Fedeli d'Amore, and the Holy Graal, themes quite familiar to readers of Evola and Guenon. Hence, those neo-pagans and new rightists who disvalue the Medieval civilization reveal their ignorance, or at least distance, from the intellectual leaders of Traditional understanding.

Beyond its own value, the proper understanding of the Middle Ages is the gateway to understanding other traditions. We can point out three in particular:

\begin{enumerate}
\item The Western pagan tradition was preserved in the Middle Ages, which is made clear in the idea of the two Romes\footnote{\url{https://www.gornahoor.net/?p=3904}}. 
\item Ananda Coomaraswamy mentions several works of the Ancient and Medieval eras\footnote{\url{https://www.gornahoor.net/?p=556}} that are necessary preparations for any Westerner to understand the teachings of the Vedanta. 
\item Guenon, in his book on Dante, points out that Dante was greatly influenced, directly or indirectly, by Islamic and Sufi sources. 
\end{enumerate}
Clearly, those seekers looking at Sufism or Hinduism for an initiation into Tradition, ought to be well-grounded in the practices and literature of the Middle Ages before making that decision.

\hfill

\textit{Addendum:} Since this topic has also come up recently, remember that for most of the Middle Ages, the Latin and Eastern churches were united; even after the schism, the theology was similar; hence, anything of traditional value in the East can also be adopted, or recovered if necessary, by the West. The Greek speaking part of the Empire regarded themselves as Roman, as much so as the West. The breach between the two was to a large extent the result of the Nordic influence on the Western Empire, which was the result of Nordic and Roman collaboration as Evola often points out.

\flrightit{Posted on 2012-09-02 by Cologero}

\begin{center}* * *\end{center}

\begin{footnotesize}\begin{sffamily}
\texttt{Logres on 2012-09-02 at 23:05 said:}

My impression is that the Middle Ages are largely the subject of intense ignorance (to begin with), even Rightists having large misconceptions, or at least preconceptions, and very little patience with changing that. On top of that, they mistrust the populist, “lunar” element of the Middle Ages, the very existence of which (comfortably) actually testifies to the comprehensive and solar “totality” of that civilization; they are unable to see any continuity (for instance) between a cult of a saint or around the Virgin as it relates to the Imperium and more “celestial” patterns. This is an unfortunate situation. Especially in America is the concept embedded of a “Dark Age” with ” the Middle Age”. It is virtually impossible to overcome, or even to suggest, in polite conversation.


\hfill

\texttt{Jason-Adam on 2012-09-03 at 00:16 said:}

I am curious if you men are familiar with the arguement of Aleksandr Dugin, who in his book THE METAPHYSICS OF THE GOSPEL (sadly only available in Russian but I can give you some excerpts if you'd desire) claims that orthodox Christianity is a full and complete Tradition, and that the loss of Tradition in the West stems from the Latin church's breaking away from Orthodoxy in 1054 AD. With this arguement Dugin manages to synthesise Christianity with Evola by saying that the Ghibbelineism that Evola was so fond of (Frederick II etc.) represented a western orthodox backlash against the usurpations of the papacy. Dugin also mentions that the system Dante outlined in De Monarchia is the same as the Byzantine symphonia of church and state which the papacy destroyed in the west with their wars against the emperors.

I would like to know Cologero and/or Logres opinion on this line of reasoning as it will help us get to the heart of the matter in what must be done for the West.


\hfill

\texttt{Cologero on 2012-09-03 at 00:56 said:}

I'm inclined to accept that, Jason-Adam, although Dugin's full work would be most interesting. I believe Evola wrote somewhere that the Council of Trent did not go back far enough in asserting Tradition. I'll take another look at De Monarchia with that thought in mind.


\hfill

\texttt{jwthomas on 2012-09-03 at 01:49 said:}

My Middle Ages history professor often recommended this book, The thirteenth, greatest of centuries, as the best summary of the virtues of that era. I never got around to reading it but I see it's still in print with a new edition out last year.


\hfill

\texttt{Logres on 2012-09-03 at 10:05 said:}

Jason-Adam, if you care to post excerpts, either here or at the Forum or by email, that would be very gratifying. Dugin is a very interesting writer – at least on the points you mention, he sounds quite accurate – the papacy did fracture the West when it supported the Italian city-states against the Emperor. I'm especially inclined to agree since Cologero thinks so as well – I am not sure how any complete Church could be expected to maintain an authentic esoteric tradition if it divorces itself from the ethos and Imperium and folkways of a people, by arrogating such a fractured position to itself as the Papacy did – they essentially cut themselves off from the Nordic spiritual centers, which had repercussions later on, not to mention they threw in their lot with the Normans and their various depradations.


\hfill

\texttt{Jason-Adam on 2012-09-05 at 06:13 said:}

from \url{http://arctogaia.com/public/eng-parad.htm}

Catholicism is a fragment of Orthodox Christianity, because information, before the dissidence the West was as Orthodox Christian as the East; in addition this fragment is distorted and claims priority and completeness.

Catholicism is anti-Byzantianism, and Byzantianism is complete and authentic Christianity, containing not only the dogmatic purity, but also the allegiance to the social and political, state doctrine of Christianity. In the very general outline, we may say, that the orthodox Christian conception of the symphony of the powers (vulgarly called “Caesarean Papistry”) is associated with the comprehension of eschatological significance of not only the Christian Empire. Hence the teleological and soteriologic function of the Emperor, based on the 2-nd message of Saint Apostle Paul to Phessalonicians, in which the question was about the “holding one”, “cathehon”. The “holding one” is identified by the orthodox Christian exegetes with the Orthodox Christian Emperor and the Orthodox Christian Empire.

The defection of the Western church is based on denial of the symphony of the powers, on the rejection of the social and political, but at the same time eschatological doctrine of the Orthodox Christianity. It is eschatological because the Orthodox Christianity links the presence of the “holding one”, which hinders the :advent of son of perdition” (=antichrist), with the existence of just politically independent orthodox Christian state, in which the temporal power (Basileus) and the spiritual power (patriarch) are in strictly defined correlation, determined by the principle of the Symphony. Consequently, the deviation from that symphonic Byzantine paradigm means, “apostacy”, defection.

Catholicism from the beginning – i.e. right after the defection from the united Church – took another model instead of the symphonic (caesarian-papist) one , in which the authority of Roman Pope spread also onto the spheres, which were strictly referred to Basileus's competence in the symphonic scheme. Catholicism broke the providential harmony between the temporal and spiritual dominions, and, according to the Christian doctrine, fell into heresy.

The spiritual crisis of Catholicism became especially apparent by the sixteenth century, and Reformation was the peak of that process. However, we should note, that as long as in the Middle Ages in Europe there existed the tendencies, which had more or less propensity for the restoration of the adequate model in the West. The Ghibelline party of German princes Hohenstaufens was the bright example of “unconscious Orthodox Christianity”, quasi-Byzantian resistance to the Latin heresy.


\hfill

\texttt{as-Sluwfakiyyah on 2015-09-23 at 17:25 said:}

Could you please provide a brief recommendation for some essential titles of this literature that you mention above? Divine Comedy and Summa Theologica are some of the obvious candidates I guess.


\hfill

\texttt{Cologero on 2015-09-23 at 23:44 said:}

@as-Sluwfakiyyah:

There are the sources recommended by Coomaraswamy: Vedanta and Western Tradition\footnote{\url{http://www.gornahoor.net/?p=556}}.

Along with Thomas Aquinas, there is Bonaventura, especially \emph{The Soul's Journey to God}.

On the other hand, the primary literature may be difficult to understand for someone without the background to properly understand it. I'll gather some titles of the secondary literature, if no one else has any recommendations. A good place to start would be Guenon's \emph{Crisis of the Modern World}.


\hfill

\texttt{Logres on 2019-09-03 at 21:44 said:}

As the years creep by, I have found no reason to disagree with your assessment, Mr Salvo, that the Middle Ages is the “latest” and therefore closest traditional civilization to the consciousness of the West. I remain convinced that it is so entirely misunderstood today as to render most of modern assessments and opinions on it “not even wrong”. Medievals seem to have instinctively somehow “known it all”, at least at whatever level they were capable of grasping. Their life is a powerful argument for the need to undo the schism in the Western Church. They even kept baptized “pagan” elements of the Northern peoples inside the pale of civilization. I don't think any there is any “improving” on it, only developing it: democracy seems to inevitably wind up in corrupted oligarchies and degenerated fellaheen mobs. At least in the Middle Ages, a corrupt king could have a hunting “accident” in the forest, and life would move on.

\hfill
\end{sffamily}\end{footnotesize}