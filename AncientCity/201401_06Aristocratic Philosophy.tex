\section{Aristocratic Philosophy}

Several weeks ago I had a conversation with a friend (?) of Gornahoor during which the topic of a ``movement" came up, by which he meant a counter-revolutionary or rightist movement. I had read a post by an alleged leader of said movement that criticized egalitarianism. Following up on that thought, I suggested to my friend that the movement ought to sort itself out by rank. Specifically, if egalitarianism is false, then the different leaders are necessarily unequal. Therefore, it should be possible to determine which are from superior minds and which are inferior. To my surprise, he was appalled by that suggestion. He could only perceive it as some sort of attack, basing his position on the idea that he should be in alliance with those who are ``shooting" in the same direction he is.

But that is begging the question of what is the actual direction of the intellectual bullets. A movement needs to be a microcosm of what it hopes to achieve. Therefore, the movement ought to be clear about its spiritual ideals and its primary exponents, those who would lead, and who are to be the serfs. Otherwise, it is just a mulligan stew, a dish suitable only for intellectual hobos.

Instead of a movement led from above, there is a nebulous mass from below based on little more than popularity, facebook ``likes", and a continuous whir of low quality discussions and arguments. This is the opposite of a hierarchic arrangement. In such an order, a neophyte is not at the same level as an adept. Furthermore, the neophyte is there to learn, not to be ``converted", and certainly not to argue. In order to identify potential members, initiatic organizations have classification schemes to identify various types of men. Certain types cannot qualify for the higher degrees, so it would be pointless to initiate them.

A straightforward classification system is based on worldviews\footnote{\url{https://www.gornahoor.net/?p=3518}}. For example, it is no accident that revolutions arising from the lower classes, as in the French and Russian revolutions\footnote{\url{https://www.gornahoor.net/?p=4235}}, have been explicitly atheist and materialist. Hence, leaders of the ``movement" who are atheist and materialist are vulgar and would most likely belong to a lower caste in a Traditional society.

\paragraph{Philosophic Foundations}
A start in that direction is provided by the short book Nobilitas, by \textbf{Alexander Jacob}, of Indian descent despite his European sounding name. Its subtitle, which describes its aim, is ``a study of European aristocratic philosophy from Ancient Greece to the Early Twentieth Century." It consists of a series of vignettes highlighting the political theories of 20 thinkers, all based on the ideal of the rule by the best. He writes in the Preface:

\begin{quotex}
The superiority of aristocratic government is due not only to its cultural advantage, but also to its solid philosophical foundation. … the term `aristocracy' in my study is not used of a particular class of people so much as of a system of politics devoted to the cultivation of the rule of the best. 

\end{quotex}
By this standard it is eminently fair to demand from the movement an explanation of its philosophical foundation and how the best would be trained and cultivated. Now philosophy, the love of wisdom, is concerned with ideas, hence most of the thinkers mentioned adhere to some form of philosophical idealism. From my point of view, I would have liked to see some medieval thinkers mentioned. Also, since biological racism is not a philosophy, I would replace their representatives in the book with \textbf{Julius Evola} who refuted the myth of blood in favour of the doctrine of spiritual races\footnote{\url{https://www.gornahoor.net/?tag=Sintesi-di-dottrina-della-razza}}. Finally, I would also add \textbf{Rene Guenon}, since his metaphysical teachings would connect Europe to the East.

\paragraph{Gentlemen and Religion}
Now such a short book cannot really describe the philosophical foundations, but only provide some of the more relevant conclusions that arise from them. Not all the thinkers are of the same depth and some are more political thinkers than philosophers in the true sense. However, that is not necessary since the aristocrats would themselves seldom be the philosophers, i.e., the spiritual leaders and educators. Instead, they would be trained in the art of ruling, so the goal of their education is to become gentlemen, not sages. \textbf{Edmund Burke} writes:

\begin{quotex}
Nothing is more certain, than that our manners, our civilization, and all the good things which are connected with manners and with civilization, have, in this European world of ours, depended for ages upon two principles; and were indeed the result of both combined; I mean the spirit of a gentleman and the spirit of religion. 

\end{quotex}
Now, besides the philosophical foundation, we have two new criteria. Which proponents of the movement have the spirits of a gentleman and religion? Are they crude and vulgar, or advocates of a life of sensuality and debauchery? Do they make their case honestly and with integrity or are they full of half-truths and distortions? Is their religious spirit compatible with the history of Europe and the West or do they insist on some mass conversion — or more likely, deconversion — as the precondition for their movement?

\paragraph{Unity of Belief}
Next, is the need for some sort of ``national" unity, however, nation, or nationalism, is defined. Giuseppe Mazzini explains the need for a ``strong national uniformity of thought and faith":

\begin{quotex}
Liberty of belief destroyed all community of faith. Liberty of education produced moral anarchy. Men without a common tie, without unity of religious belief and of aim, and whose sole vocation was enjoyment, everyone sought his own road, not heeding, if in pursuing it, they were trampling upon the heads of their brothers — brothers in name and enemies in fact. 

\end{quotex}
This brings us back to the beginning. Is the ``movement" truly one of real brothers or one of nominal brothers but real enemies? Thus, a movement requires a unity of religious belief. A ``big tent" movement is not organic, but is merely a heap\footnote{\url{https://www.gornahoor.net/?p=1737}}. So what is that belief and how possible is it a force for unity?

So before adopting an intellectual affiliation, investing emotional capital, or even providing financial support, ask if your movement demonstrates these qualities. These are the fruit of a couple of dozen centuries of the aristocratic element in European thought. If you think something ``new" is necessary, or even worse, a new type of man, you will certainly be disappointed. That is why what is required is not a revolution in the opposite direction, but the opposite of a revolution (Joseph de Maistre).


\flrightit{Posted on 2014-01-06 by Cologero }

\begin{center}* * *\end{center}

\begin{footnotesize}\begin{sffamily}



\texttt{William on 2014-01-06 at 10:03 said: }

I've been following Gornahoor for years now, and one example of a fairly pure aristocracy that keeps coming to mind is a proper martial arts dojo. It is easy to settle the question who is better (`aristos’ — let's spar. The sensei is obviously one who knows — he can easily DEMONSTRATE his superior knowledge. The rest of the hierarchy of the dojo is equally easy to settle by the same means. 

Not all dojos are equal, of course — when I started out, I looked for a dojo without a bunch of plastic trophies in the window — competing in local tournaments was not why I was there. And I looked for a sensei with a good heart (a `gentleman’?) — I found one where the sensei had a masters degree in sports physiology and his `day job' was working for the school district developing physical education plans for handicapped kids. Dojos like this are out there.

Looking back on 12 years now, I can say that practicing martial arts has turned out to be a spiritual practice which has deeply changed me for the better. 

In short, the dojo is a living example of a working hierarchy/aristocracy. It has its own Tradition going back millennia. Of course these are small `societies’ — maybe a hundred students at a thriving school? But still, I'm thoughtful what could be learned from them which would apply to the larger questions considered on this site.


\hfill

\texttt{scardanelli on 2014-01-06 at 10:39 said: }

I would say that the major difference is that in physical combat, there is a clear winner and loser whom all can recognize. We can all spar to a greater or lesser degree. When it comes to the esoteric, not all can recognize that which is superior because not all have experience of the esoteric. This knowledge cannot be demonstrated but only experienced within. Therefore those who do not know, do not know that they do not know so to speak.

An aristocracy is based upon the authority of those who have reached certainty within.


\hfill

\texttt{William on 2014-01-06 at 11:30 said: }

I would say that the major difference is that in physical combat, there is a clear winner and loser whom all can recognize

I am not saying that martial arts is a replacement for esoteric work at all. But it is also far more than `physical combat'. All that stuff about `you are your own greatest opponent' is true. There is an enormous amount of `inner' (esoteric) work that a martial artist has to do to progress, including but not limited to overcoming your own ego, overcoming your own fears, being faithful to discipline, humility in the face of a perfection you will never achieve, compassion for students less skilled than you because you need that same compassion from teachers who are your betters, being calm and fully present to your opponent. To me these all sound like spiritual lessons, not physical.

Even in sparring: in a dojo it is not so much about identifying a `winner'. but an opportunity for both participants to try to apply what we've learned. When sparring with a less skilled student, I deliberately emphasize skills I know he is trying to develop. It's a brotherhood.

I'm certainly not suggesting that martial arts is any kind of silver bullet for The Work. But it can be a big help for some, and I still suggest it as a living working example of aristocracy/hierarchy.


\hfill

\texttt{William on 2014-01-06 at 11:38 said: }

For myself, I know I need to submit myself to an authority and discipline greater than myself. So — where to find that? In the context of Gornahoor that would mean the Catholic Church, but where I live the local Catholic Church consists of singing folks songs in spanish to an out-of-tune guitar. And is the priest spiritually developed at all? Having worked as a church musician in churches for 40 years (Protestant and Catholic), I can say that spiritual development correlates poorly to official training and ordination. 

For now, for me, martial arts is the best (though still far from ideal) solution to finding that authority and discipline I know I need.


\hfill

\texttt{JA on 2014-01-06 at 12:04 said: }

``a life of sensuality and debauchery"

How would we explain then Catherine the Great, or the Marquis de Sade, or the Baron von Sacher-Masoch; all of whom were of the best blood ?

I've had the personal honour (or dishonour ?) of being in the same room with members of different royal houses, and honestly brothers – they can snort coke and party all night long as much as any ghetto child……..

Among a certain section of the masses, a careless and libertine way of life is actually associated with aristocracy in their little minds………


\hfill

\texttt{William on 2014-01-06 at 13:49 said: }

I've had the personal honour (or dishonour ?) of being in the same room with members of different royal houses, and honestly brothers – they can snort coke and party all night long as much as any ghetto child……..

Ha ha, precisely! Much the same could be said of cardinals and archbishops (the never ending sex scandals, for openers). Both ecclesiastical AND civil governance is far from `aristos'.

The IDEA of `aristocracy' is great (and all the variations on Plato's philosopher/kings) but how in heaven do we get there?


\hfill

\texttt{scardanelli on 2014-01-06 at 16:02 said: }

William,

I am certainly not denegrating the martial arts. I'm only making the point that the esoteric cannot be demonstrated in the same manner. 

I'm using authority in the sense of Tomberg's fourth letter. Authority is present ``where there is present the breath of sacred magic, filled by the rays of light of gnosis emenated from the profound fire of mysticism." So, where in heaven do we find authority? First learn concentration without effort… This does not necessarily require submitting to the local Catholic Church. At any rate, the whole cannot be judged by the sins of the few. As Cologero has remarked, we get the church that we deserve.


\hfill

\texttt{Michael on 2014-01-06 at 21:06 said: }

I will second what JA says: although I am sure there are exceptions, the aristocrats of the current age show little to no spiritual development.

Will following the path outlined in Meditations on the Tarot lead to a new nobility? Or is spiritual development, at a certain point, incompatible with nobility as we find it in the middle ages?


\hfill

\texttt{Scardanelli on 2014-01-07 at 13:01 said: }

``However, that is not necessary since the aristocrats would themselves seldom be the philosophers, i.e., the spiritual leaders and educators. Instead, they would be trained in the art of ruling, so the goal of their education is to become gentlemen, not sages."

You're right Michael. I should have read more closely.


\hfill

\texttt{JA on 2014-01-07 at 14:32 said: }

amongst many upper class people I know, a pseudo-Nietzschean attitude prevails that personal morality is for the bourgeois, the unenlightened and uneducated folks; and that for those of us of better blood and breeding we are pretty much free to have as much fun and pleasure as we want. In their thinking, the lower classes exist so that we can enjoy ourselves……….and then of course there are those aristocrats who having a ``conscience" and ``compassion" for those less fortunate become leftists.


\hfill

\texttt{Michael on 2014-01-07 at 18:42 said: }

Is the ``pseudo-Nietzschean attitude" the reason we have had the revolution that has resulted in the rule of the shudra? My recollection from Plutarch is that the Roman aristocracy did uphold a higher standard. On the other hand, the ruling class of ancient Greece was immoral in many instances.

The aristocracy of today most likely is just reflecting the degraded morality of the age.


\hfill

\texttt{Matt on 2014-01-07 at 23:25 said: }

``The aristocracy of today most likely is just reflecting the degraded morality of the age."

Or better put, the modern age is just reflecting the degraded morality of the aristocracy. And the term aristocracy is probably not even the right term that applies to the current upper classes, as it means rule of the best; best in virtue and best in intelligence, neither of which the majority of those in the upper class can legitimately claim to represent.


\hfill

\texttt{August on 2014-01-08 at 02:35 said: }

It doesn't matter what one does. Who cares about morality, especially now. The important part is what motivates him to do it, what is happening inside. If you have to persistently restrain yourself from a loose lifestyle, how are you any better from someone who indulges similar impulses? At most you've demonstrated better self-control, in the worst case it's just a matter of time until you relapse or some monstrous repression erupts in one fashion or another. And at least you know what you're dealing with in the presence of the openly indulgent.

When you have crossed a threshold, you know it because even if you WANT to, say, party like you did when you were a 20 year old, there is no longer anything inside you to allow you to do it. Sure, go buy some drugs and head to the club, but you may as well be baking muffins for all the inner participation that is taking place, and going through stupid motions quickly makes one feel stupid. When the inner mania evaporates, the superficial desire quickly goes too, and no restraint is required. Do you see why some people fear a spiritual correction more than death?

If `aristocrats' aren't getting to this point, it could have something to do with the active tendencies, and vanity, of their class, which in our bland surroundings find convenient expression in the artificial drama-world of the drug-sex-party social complex. That they can't see how pathetic it is to get bogged down in that and actually enjoy it is also perhaps to be expected, given that they historically relied on spiritual authority for direction. It also seems to be, originally, a signature British weakness.


\hfill

\texttt{Scardanelli on 2014-01-08 at 15:36 said: }

In what sense are you using morality august? Perhaps I'm misunderstanding you. Morality isn't just an arbitrary law imposed upon ones ego from without and masking ones true depravity, but an expression of an inner state reflected in the outer. Thus morality is relevant in that when we act in a moral way, our wills are aligned with providence and impose this upon destiny, so to speak. Therefor it does matter what one does… To be moral is to redeem the flesh, to spiritualize matter.


\hfill

\texttt{August on 2014-01-08 at 18:09 said: }

Though it should have been obvious from the lines following, I was referring to morality in the Protestant sense, an arbitrary behavioural convention, a residue, given excessive importance and divorced from real ties to the spiritual, eventually failing in its only potentially useful role of maintaining a kind of public order.


\hfill

\texttt{Cologero on 2014-01-08 at 21:11 said: }

On the word of a modern day Greek aristocrat (Joy is a state of mind), drinking, debauchery, and sports competition are sources of joy if done with good breeding, style, and manners. A gentleman understands moderation, so he doesn't go to the extremes of a saint, yogi, or even philosopher.


\hfill

\texttt{Michael on 2014-01-09 at 18:17 said: }

Cologero, Taki is a charming guy, and I think he is in many ways a good model, but how does debauchery fit in with theosis? Isn't the goal still to ``be perfect as your heavenly Father is perfect?"


\hfill

\texttt{Gina on 2014-01-17 at 11:49 said: }

Engaging post. This statement about the favoring of spiritual races over biological ones rings true in my opinion. I've felt it creates a problem in the modern day. Good breeding, higher education, and a deeper appreciation of sciences and art are all something that leaders should possess, and these are qualities that are not blood inherited, although money helps when on the road to spiritual attainment (in my opinion). 

Many people can legitimately trace their family lines back to royalty. I am one of them. This doesn't make them great leaders or worthy to live off of tax payer money. I'd say this is a gross misuse of funds. 

In Kabbalah, for instance ``House of Jacob" is somewhat equal to saying ``soul of balance". The middle way. An aspect coming from Tiphereth. I also do believe that Jacob was historical, however, if you can trace your lineage back to him, you're just as likely to be a King in Saudi Arabia as you are to be an accountant. Certain ``Houses" meant spiritual grades in old times. We've become quite Pharisaic maybe in our interpretation of these hierarchical systems.

Good genes or potentially good epigenetic conditions are not hoarded in royal houses or amongst those in the elite bourgeois. However, our society seems to have destroyed the tenets by which purposeful mating was made important. I do see that. This is a bit of a religious and social crisis where the spirituality has been sucked from the materiality.


\hfill

\end{sffamily}\end{footnotesize}
