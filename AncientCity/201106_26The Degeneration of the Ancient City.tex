\section{The Degeneration of the Ancient City}

The traditional sacred king was himself of a divine nature and the ``gods" were his peers; he was, like them, of ``celestial" stock, he had the same blood as they; he was thus a centre, an affirmative, free, and cosmic principle. \flright{\textsc{Julius Evola}, \textit{Pagan Imperialism}}

We assume the readers of Gornahoor prefer scientific instruction to popular propaganda. The path to gnosis requires the first trial, which is to purge all opinions, beliefs and propaganda from one's mind, at least provisionally. Only then can things be seen clearly.

Next, we plan to show how the Medieval period re-established the City as Empire, which events had to happen as they did, and how the degeneration of the Middle Ages followed the same sequence. Only then can we be in a position to look for possible ways forward.

\paragraph{Tradition, Family, Property}
Whenever \textbf{Julius Evola} refers to Paganism, he means the Religion of the Ancient City\footnote{See Section~\ref{sec:ReligionAncientCity} in this book.}, whose tripodal support was \textbf{Tradition, Family, and Property}. The City was hierarchically arranged under the leadership of the King, and under him, the leaders of the tribes, clans, and families. The religion was based on ancestor worship. The citizen's day and year was structured around the religious rituals of the levels of the hierarchy, which included his own role as priest in his own household. The nature gods that we incorrectly identify as paganism were secondary, although there were temples to Zeus, Apollo, and so on, depending on the city. But even then, they were restricted to the city and had no universal significance.

Pietas (piety) was the highest virtue and meant the duties owed to the gods, the cities, the king, the heads of the hierarchies, which had to be scrupulously observed to avoid becoming ritually impure. A citizen did not make a major decision without first checking the omens. Evola describes their spiritual attitude:

\begin{quotex}
There is no room in it for passion, nor for its antithesis, nor for ``effort", and even less for ``humanity" and ``feeling". It starts from absolute centres without hatred, without craving and without pity; from a calmness which terrifies and immobilises; from a level of ``creative indifference" superior to every opposition.

\end{quotex}
The families maintained these traditions generation after generation and the purpose of the laws of the city was to ensure the continuation of the traditional rites. Wealth was measured in terms of property, not money.

\paragraph{Revolt of the Aristocracy}
The Greeks themselves referred to the period of the Kings as their ``\textbf{Golden Age}``. Evola emphasizes that the king himself was a semi-divine figure, which was not an abstraction or a mere honorific title:

\begin{quotex}
the principle of divine right must be understood concretely and not in a formal and conventional manner: it must be understood in the sense that an actually deified being, as person — beyond any convention and any exterior acknowledgement from another authority — showing an extra-human nature, has the true and legitimate right to rule.

\end{quotex}
As such, the king was priest, since he was the link to the gods; he was also chief warrior and magistrate, since only he had the passionless indifference to arbitrate the claims of competing tribes or families. Over the course of centuries, perhaps because the king no longer displayed such priestly powers, the aristocracy began to challenge the power of the king. The king was stripped of his political powers, but he could not be removed from his role as Chief Priest. This led to the development of a separate priestly caste while the rule of the city was transferred to a Senate formed by the leading oligarchs.

\paragraph{Revolt of the Clients}
Only the families who could trace their source back to a pater had a religion, that is, they had the prayers, rites, sacrifices and so on, which were held in secret. However, there were other families, called clients (vassals), which had no pater, hence no religion. They would participate in the rites of the family they were attached to. They did the labour for the family, though they had no rights to property of their own, certainly no civil rights as we would understand them today.

Over time, the role of the clients changed. They may have lost their religion, as their priest may have lost his power, the omens proved false, the sacrifices ineffective. Horizontally, the clients of different families could congregate, forming a separate class within the city. Then vertically, when the army was divided into centuries rather than families, clients would find themselves in different divisions from the pater, diminishing his authority. The client, thus, became a power bloc. At times they could align with the king against the aristocracy, at other times with the plebs.

\paragraph{Revolt of the Plebs}
Around the city, there developed a population of outsiders, with no family, tradition, or religion. These plebs consisted of alien migrants looking for work (the clients were not numerous enough to do all the servile work of the families), bastard children of the families, clients who escaped from the family, exiles from other cities, and so on.

Following the stages of the degeneration of the city, the status of the plebs changed due, in part, to their mere numbers. The change from a property to a money based economy also changed their situation, since they could become wealthy as artisans, traders, and so on, apart from landed property. Wealth began to replace family, blood and tradition as the symbol of status in the city. Whereas property was sacred and could not be mortgaged in the ancient city, money could be loaned at interest. 

This represents the ultimate decline of the city. Although the families continued their private worship (Evola claimed that even in the Rome of his day, there were still families that maintained a hearth), they effectively lost control of the city. It was no longer hierarchically arranged, but stratified into classes with constantly shifting alliances. The king was no more, money was supreme, the meaning of the ancient laws forgotten. Democracy became the accepted form of government and the plebs even led religious rites, thus debasing the hereditary priesthood.

\paragraph{The Religion of the Plebs}
Lacking fathers and ancestors to worship, the plebs are without religion. That is why Tomberg could write that the lowest classes are atheists. A pleb may try to make himself a god, redolent of the enthronement of the goddess Reason of the French revolution. More typically, lacking hearths of their own, they would attach themselves to the temples of the nature gods. Thus what we today want to recognize as paganism, viz., the cult of the Olympic gods and goddesses (or their Roman counterparts), was actually the religion of the underclass. This is why Evola held the neo-pagans of his (and our) time in utter contempt\footnote{\url{https://gornahoor.net/?p=2007}}.

While the ancient religion continued to be practiced, it was really an empty formalism. The ancestral gods were forgotten, and the nature gods became the norm; they belonged to the universe, not a particular family or city. Wandering poets spread this new religion.

In this milieu, the philosophers could be more public. Thus, \textbf{Pythagoras} brought back the idea of a Supreme Being, \textbf{Anaxagoras} understood the God-Intelligence reigning over all men, the Logos as the cosmic order was taught by \textbf{Hearaclitus}. \textbf{Zeno}, the Stoic, believed in a universal God of the entire human race and the idea of a State for them. \textbf{Plato} and \textbf{Aristotle} sought to understand the structure of man and the state. Such self-doubt would have had no place in the Ancient City.

\paragraph{Summary}
We have made the barest presentation, since our purpose is not to write history, but rather to understand the principles behind events; we encourage our readers to check out the history books reading them in the light of their understanding of metaphysics and tradition.

The errors of the so called new right, alternative right, neo-pagans, and similar anti-traditional movements that claim to represent the ``right" should be quite obvious at this point. The degeneration of castes is a process inherent in any civilization and is not caused from the outside. We have absolutely demonstrated that the decline of the city happened centuries before Jews or Christians were in any way implicated. These are some of the errors of the non-traditional right:

\begin{itemize}
\item Money as status symbol and the introduction of usury was not a Jewish injection into the City. 
\item Ironically, neo-paganism was the religions of the plebs, outcastes and pariahs, and pre-dated Christianity by centuries. 
\item Monotheism was the understanding of the most intelligent and educated of the pagans, not an innovation introduced by Christians. 
\item Long before the first Christians, the pagan Stoics entertained notions of Universalism and the Brotherhood of Men. 


\end{itemize}
\flrightit{Posted on 2011-06-26 by Cologero }

\begin{center}* * *\end{center}

\begin{footnotesize}\begin{sffamily}



Here is a quick list of points to ponder, though hardly exhaustive.

\begin{itemize}
\item It is an error to regard the pagan civilization as unitary. That is the point of describing the successive degenerations of the tradition. By the time of Empire, Rome was already degenerate and miscegenated. 
\item Your definition and understanding of ``monotheist" is leading you to a gross misunderstanding. Please use the word in the way we have used it. 
\item The Vedic civilization is Tradition, not Hindu India which is something else. One cannot be initiated into a Traditional city, had you bothered to read the post. One had to be born into it. Danielou can be nothing other than an outcaste. In the Ancient city (including the Vedas), one achieved immortality by having sons (or else risk becoming a hungry ghost), so its too bad for both you and Danielou. You should be disappointed with the pagans, I'm just the messenger. 
\item We have dealt with the issue of Christian polytheism, which proves our point. 
\item We are interested in the origins and destiny of one people: the Hyperboreans, their descendants, migrations, and eventual unification. That requires a spiritual unity, not genetic, as has been made clear. There is an anonymous text floating around, whose actual source is Fabre d'Olivet. But they left out the concluding chapters on the necessity of a ``supreme pontiff", that is, the high priest of all Europe. 
\item I know my English isn't so good, but didn't we make the point that the city was composed of a hierarchy of families, clans, tribes? So would be Empire under the high priest be the same? Haven't we over and over said that for Tradition unity doesn't mean non-differentiation, but rather hierarchy and harmony. 
\end{itemize}
The whole exercise, therefore, has been to draw out the principles, not just to repeat mindless propaganda that appeals to the emotions of some manly past. Alas, there is no one paganism to return to, which, as Evola pointed out, was just one of the Aryan civilizations. If we prefer to mine the third such civilization to discover those principles, it is because we have some deep reasons for doing so.


\hfill

\texttt{Matt on 2011-06-27 at 23:57 said: }

It is interesting that you state that the olympic gods was the religion of the underclass, instead of the cthonic/underworld gods (are you thinking of writing a post on that subject in the near future) as what Evola and others suggested. I thought the post was good. Did a nice job of pointing out that the process of degeneration starts within the civilization.

Looking forward to your post on the Middle Ages.


\hfill

\texttt{Cologero on 2011-06-28 at 00:38 said: }

The cult of the Ancient City was based on ancestor worship. The nature gods of Olympus also had their temples, but they were secondary. The ancients were neither tolerant of other gods nor syncretistic, despite what the new right and neo-pagans claim. The cult of Zeus in one city was different from another, so there was no commonality.

When the plebs came into positions of power, they had no family gods, so the only way to integrate them into Rome was to allow them entry into the temples of the Olympic gods. Rome had the three main cults of Jupiter, Mars, and Quirinus corresponding to the three main castes. As the memory of the ancestors began to fade, and the rites became lax, paganism came to be exclusively associated with the cult of these gods and the other nature gods.

The problem with the neo-pagans and the new right is their tendency to conflate centuries, or longer, into a single anachronistic panorama, whether they deal with pagans or Christians. Instead of communicating in terms of principles, they become evangelists for some sort of imaginary neo-paganism that is scantily related to the ancient city. They also fail to deal intelligently with the Middle Ages, considered by both Evola and Guenon to be the last example of traditional civilization in the West. In reality, I don't really care to refute them, since they do that pretty well on their own. My only goal is to dispute their claim to be so-called ``Traditionalists", otherwise I wouldn't waste much effort.

The fundamental objection is that they are really partisans, that is, propagandists, and not interested in objective, metaphysical, or scientific knowledge. That is why, for example, they whine about Charlemagne so much, since their simplistic formula is ``pagans good, Christians bad" and this partisanship colours all their commentary. We, on the other hand, prefer to analyze events in terms of principles such as how can spiritual unity be extended from the City to the Empire? What are the roles of the Pope and Emperor, and how are they related to the High Priest and King of the ancient city? How was the caste system manifested in the Middle Ages? and so on.


\hfill

\texttt{Matt on 2011-06-28 at 03:55 said: }

Cologero, re-reading the post again and taking some things into account, I think I now understand what you mean by using the word nature in reference to gods. Placing the olympic gods in that category (for lack of a better term) along with the cthonic ones now makes sense. Would the term ``supra-natural" (above the sphere of fate and necessity) be valid for the ancestral heroes?


\hfill

\texttt{Cologero on 2011-06-28 at 08:24 said: }

The Olympic gods had their domains over nature: the wind, thunder, the sea, crops, and so on. They are, therefore, related to Destiny or Fate, and are portrayed thus in the myths..

The gods of the city are associated with Will, so yes, in a sense that is opposed to fate and necessity. The City is where men act, apart from the natural forces (fate, necessity) to which man is subject. The rites were designed to placate the gods to gain their favour. Note that purity is necessary, the ancient city was not tolerant and did not allow ``false" or alien gods into their city or cult. Those without family were excluded from the rites of the city.

Cities with a common ancestor would be ``friends"; cities with an incompatible genealogy were ``foes". No one had to wait for Carl Schmitt to figure that out. That is a factor that gave Rome a great advantage. Some 20 cities between Asia Minor and Italy claimed descent from Aeneas. Thus they were by necessity allies to Rome, despite any differences in ethnicity or language.

This is the pagan way, based on family, friends, foes, tradition, exclusiveness, reading omens, rites, rituals, animal sacrifices, and so on. We see none of that from the neo-pagans, who nonetheless claim to be their spiritual heirs. Apparently they are learning this from ``academic" philosophers of the new right. If that is so, I want to reassure everyone that I have absolutely no academic credentials.


\hfill

\texttt{Ismo on 2011-06-29 at 15:26 said: }

A few points from concerning the spiritual (ideal) and concrete battle, Evola, and this article:

1. Todays battle is fought primarily on the plane of ideas, and there is absolutely NO point or hope in trying to fight the material battle, since the whole existential atmosphere is against it. The cyclical wheel and the Kali Yuga is only too close to its ending for a `grand scale correcting action'. no matter how far away it may seem to be from the human stand point. This is the whole point of `riding the tiger'. I think you and other New Rightists are aware of this also. This should NOT be a call to inaction or fot the lack of principles, but absolutely the other way around, however. However, in todays world it is possible to wake up only those few individuals who still have some sort of a foundation, no matter how confused it may be. It must be realised that what you, Cologero (and myself / others of the same inclination) are doing is planting the seeds of the coming age. The world and its masses are too dead or deaf to wake up other than via massive catastrophes and other existential destructions.

2. As regarding the jews, I think most of those who have studied history know their negative role as instruments of destruction etc., yet again, what we are right now facing is the fact that most of those who are physically or otherwise `aryan' are in many ways not really acting any differently. Evola also noticed this fact in his Path of Cinnabar and said that there's no point in making the separation into jews and aryans today in larger scale; he also made it clear that a jew can have an aryan soul and vice versa, and this is really the only thing that matters and is nowadays as plain as daylight. My point in this is the following: The degeneration of Aryans by jews is an overt simplification (though an attractive one, if one wishes to find an enemy to point ones finger to), because, if the Aryan would have been alive and well in his soul and spirit, he could have resisted the seed of degeneration in its germinal form. Of course this takes nothing out the instrumental role the jew has played in all of it, but the cause itself is much deeper and is partly if not wholly in the cyclical conditions. Evola also warned that making faux targets is one of the weapons of the counter-tradition / black forces, and blaming the jews about everything is precisely one of the faux targets; just see what has happened via Nazis and their fundamentalist anti-semitism and distorted aryanism! ``It is important not to give weapons to enemies." – Evola

3. As for Family and Property, these are not a priori ``bourgeois" institutions. The bourgeoisie as a class (and class mentality itself) didn't even exist before modernism. Evola always recognised the important instrumental role of the bourgeoisie, although it can easily be interpreted from his writings that he despised them (and most of humanity, to be precise, aha). This is, however, a mis-interpretation: Evola only wished to make it clear, that the Vaishyas (=bourgoisie) and Shudra need to be kept in a tight leash by the spiritual and intellectual elite, and that every great civilisation is and has been made / being given birth by this small minority and not by the merchants and servants who lack the truly creative potential. This has been recognised, by the way, as one of the very reasons the jews have been at the forefront as instruments of the Kali Yuga = They, as a collectivity, cannot create anything of their own, they can only steal, plunder and distort. History proves this, so this is not an opinion or a personal prejudice.

Lastly: the greatest of mistakes we can do is to fight petty, personal and egoistic fights. This is precisely what the real enemy wishes, to divide and conquer.


\hfill

\texttt{Perennial on 2011-06-30 at 01:20 said: }

Ismo, Brilliant! Brilliant! Brilliant! Excellent post, and exactly the right type of position we need to take. I have often pointed out that the forces of Tradition should reserve their internecine battles until AFTER victory, and not before. We can shoot it out after the modern world is nothing but a heap of ashes. In any case, you are right, we must build the foundation of the inner resistance before any talk of broader engagement can arise, because at this point awareness is too low to allow for a possible victory. Time, patience, and building strong fortifications and foundations with ideas, customs, and understanding will lay the road for our ultimate, and even inevitable, triumph over this decadent age.


\hfill

\texttt{Ismo on 2011-07-01 at 08:48 said: }

Thank you both for your kind words. Perennial already said the main points about Evola's stance and how it should been looked upon. I think the most important thing in it, with which every one of us can easily familiarize with whether we disagree or differ with his extremist anti-bourgeois elitism or not, is that spirituality, culture and true intellectuality (and not the modern academic `intellectualism'. should be the defining values over economic or social questions in a narrow sense.


\hfill

\texttt{Ismo on 2011-07-01 at 08:58 said: }

Oh, and one thing that I forgot as an example: Leonidas of Sparta and many other Spartan warriors had both families and property. If one dared call them ``bourgeois" they'd penetrate one's hearts with the spear of destiny!

The point is that it matters nothing if one has children or not or money / property or not, it is the inner bourgeois mentality with its vulgar and cowardly ``values" one needs to erase for good.


\hfill

\texttt{Cologero on 2011-07-01 at 10:02 said: }

Thanks for your points, Ismo, but I'll make some further clarifications. There is a difference between the property (meaning land estates) class and the money class. in the Ancient City, land was sacred, therefore it could not be mortgaged; it could not be exchanged without a religious rite. The money economy was different, since money could be easily exchanged. The underclass and out-castes could participate in the money economy; that eventually undermined the City, and more so, the Empire. I was important to the aristocratic class to pass on the religious traditions exactly, which required sons. This is Tradition, not engaging in a costume drama at the feet of an Asian master. 

One must not confuse personal choices made in the open society of today with the traditional outlook. In the Ancient City, there was no choice. One participated in the religious rites of the city, one followed its laws, and so on. The whole of life revolved around Tradition. To continue the Tradition, a man needed to have sons. If anyone sees in this ``bourgeois values", then his understanding is deficient. In the modern world, there is no such way of life. Guenon chose to go to Egypt where he raised a family. (He did not go to India because of the birth-right caste system, which he could not have participated in. Nothing to do with passport issues.)

Since he did not regard the Catholic church as traditional enough, Evola made the deliberate choice to live WITHOUT tradition. He hoped the political zeitgeist of the time would eventually lead to the restoration of the pagan civilization, similar to what I have described; it was not to be. Therefore, one should NOT use him as an example of following tradition. There are many ways to live in the modern world, but only one in the Ancient City. His was just one way to cope. Therein lies the absurdity of calling oneself a ``Traditionalist"; a man is living a tradition or he is not. We are not.

A further point about his personal life. Evola lived simply, his needs were few. He had book royalties and a small salary from his regular writing for the journal La Vita Italiana. He lived with his mother, although not in the basement. After the war, he relied on benefactors.

A further point, Ismo. The greatest danger is a traitor in the midst. The 300 were defeated not by the Persians, but by one of their ``own". It is best to expose them early.


\hfill

\texttt{Cologero on 2011-07-02 at 09:18 said: }

It's interesting that Kerouac, like Evola, lived with his mother. So what happens when everyone is a rebel? We get the pathetic sight of 60 year old men in pony tails dancing with jerky movements to Credence Clearwater Revival. If their goal was to epater la bourgeoisie, what happens when they become the bourgeoisie? The Western ideal has been the puer aeternus, and in this deracinated, feminized, homesexualized world, he now runs the show. It is an easy thing to be an anarchist, and much more difficult to follow the cosmic order. This rebel without a cause is tolerated, even encouraged, in this world. It turns out they are easy to control, because a man without a tradition, the man who rejects his tradition, has no culture and is easily propagandized and advertised to.

So who is the new rebel for our age? Men with guns and natural sons who follow in the ways of their ancestors, the ways that every well bred man considers sane and normal.


\hfill

\texttt{James O'Meara on 2011-07-02 at 11:31 said: }

Even if Evola didn't intend everyone to live as he did, our evaluation of him depends on how he managed it; extreme poverty, like Simone Weil or bank robbery, makes a difference. I have read that after the war he lived in an apartment in Rome ``provided" by a Prince who was a fan. Reports of his last days indicate a maid of some kind. When did Evola live with his mother, Colgero, may I ask? Who supported whom? When Kerouac finally started making money his mother was old enough to need his care, so one way or another he wound up spending most of his life with her. 

Of course, ``living with one's mother" takes on a different connotation if you are, say, Prince Charles. This raises the vexed, for me at least, question of ``Baron" Evola's nobility. I haven't found any trace of such a family among the Sicilian nobility, although there was a Natale ``Joe Diamond" Evola (February 22, 1907 – August 28, 1973) who was a New York mobster who briefly became boss of the Bonanno crime family. Was it common for Sicilian nobility to enter the Mafia? Was Joe a black sheep sent abroad? Or was it Julius who, like Michael Corleone, tried to leave the Family?


\hfill

\texttt{Cologero on 2011-07-02 at 12:37 said: }

There is no trace of the name ``Evola" in the Sicilian nobility registries. But, then, Evola is the Italianized spelling of a Norman name. In the 1400s or so, the Sicilian nobility was replaced by outsiders in order to consolidate the power of the invaders, and the Evola line may have been a casualty. The first Salvo arrived from Tuscany, presumably to fill one of those roles; that is probably why ``Salvo" is in the registry.

We here at Gornahoor are interested in the actual dharma of the traditional civilizations, not in the lifestyles of modern rebels. The moment now has a different quality; were it still the time to ``ride the tiger", I wouldn't bother with a public blog that brings me no benefit (I have no red thermometer or tip jar; we don't sell books or t-shirts or journals. I have asked for help with work, but I think most men today would prefer to send me \$20 than to actually do something.)

Dharma, piety, duty: that is how a man evaluates his fellows. Evola supported the fascists, but wouldn't commit. He ran away when the allies reached Rome (sneaking out the back, while his mom kept them at bay), while Giovanni Gentile, a far nobler man whom Evola criticized unfairly, stood by the Salo Republic against the wishes of his family. Evola knew and accepted what the popes wrote against democracy, liberalism, socialism, yet opposed them publicly. He praised the ``Catholic philosophers of authority" (de Maistre, Donoso Cortes) while rejecting any authority over himself.

Apparently he believed he was beyond caste and beyond tradition, but when the bodhisattva returns, he nevertheless pretends to play the game.


\hfill

\texttt{Izak on 2011-07-02 at 19:10 said: }

I didn't care for the beginning of this debate at all, but this conversation just got very, very interesting. 

" It is an easy thing to be an anarchist, and much more difficult to follow the cosmic order. This rebel without a cause is tolerated, even encouraged, in this world. It turns out they are easy to control, because a man without a tradition, the man who rejects his tradition, has no culture and is easily propagandized and advertised to."

This has always been one of my basic maxims on approaching life in the 21st century. I also am of the opinion that Evola has a very distinct danger to him, which is that he can be romanticized much more easily than the other traditionalists like Guenon or Burckhardt, or whoever. Schuon allowed himself to be turned into a demigod of some sort, and he led people down the wrong path. Rudolf Steiner wound up being the romantic hero of the new age, having schools created in his name — I've seen Waldorf schools, and I recognize nothing of a traditional character in them. Krishnamurti has led many men astray. They all have magnetic presences.

All of this is why I get wary very, very quickly with those who cite Julius Evola's opinions on things that don't really concern the primordial tradition, or which aren't really connected to deeper issues not of the current times. I think ``Ride the Tiger" was probably the most important book he wrote because it concerns the strategy for the here and now, but it is also very problematic. I don't believe I've ever read a book in which I agree with so much, and at the same time so little!


\hfill

\texttt{Ouranoi on 2011-07-02 at 19:38 said: }

Yes one of the more interesting threads that have appeared here. It's always important to note the practicality of any traditional method applied to living it in reality. In these cases romanticism and dogmatic ideology can no place here although in the midst of time they feed on each-other.There is a reason people did things in the past the way they did due mostly to opportunity and circumstance.


\hfill

\texttt{Ouranoi on 2011-07-02 at 19:39 said: }

while we're at it I could not resist with this interlude on the subject of Evola's nobility and his mother (Concetta Frangipane) which is surname of one of Sicily's oldest aristocracies (the crest is distinctly Norman in character):

\url{http://www.regione.sicilia.it/beniculturali/bibliotecacentrale/mango/franchi.htm}

\url{http://www.sicily-genealogy.com/cognomi\_defghi.htm}

It's all well and good to try and knock Evola's character but he only fled when the allied police came literally knocking on his door (his mother distracting them) giving him precious seconds to flee out the back. Did anyone question William Joyce's tip off from MI6 insiders when he too fled Scotland yard by minutes to make his infamous career in Berlin?


\hfill

\texttt{Perennial on 2011-07-03 at 03:35 said: }

Fair observance Ouranoi! You apparently are better educated on Evola's personal life then most. I will note that in my readings of Evola's life, all that is said is that he was a Baron from Sicily, not a Sicilian noble. The Regno delle Due Sicilie may not have, in fact, been a source of his nobility at all. In any case, I accept Evola more as my philosophical benefactor, my spiritual godfather so to speak. I do not feel compelled to accept his views without exception (and I do not) and mostly use him as a foundation for building my own edifice, which, I believe, is what being traditional is all about. Tradition focuses our lives away from the unimportant, through ritual and customary practices, and gives our minds the freedom to focus on higher things. It is after people stopped living by custom and tradition that they began to pester their minds with other ludicrous nonsense, not having the ability to discern what is, in fact, actually important and worth meditating on. Real freedom is in knowing who you are, and where you are from, and from there where you can go. The idea today that people need to build an edifice from scratch within themselves and without is too much to ask of the average person. We all need a foundation to build from, for few individuals are in the position to build it for their own. Thus Tradition.


\hfill

\texttt{Matt on 2011-07-03 at 04:10 said: }

Cologero, your comment on baby boomer hippie rebels was funny and on the mark. Unlike many young people my age who think having parents like that is a mark of pride, I personally would be downright embarrassed if I was raised by parents like that (fortunately I wasn't).

And yes, it does seem a little odd, to say the least, that Evola, who considered himself to be apart of the warrior caste, chose to flee fascist Italy instead of staying to fight for something that he defended in his political writings at that time(for the most part). I think that is what is truly more revealing of the unsavory side of his character (not saying that his unsavory side encompasses all of his character), not his general support for fascism, but his lack of will in defending it when the time came. 

As for the talk of fathers and sons, I hope that we are not quick to disregard the way of the monk as a viable path of revolt (and a path to true knowledge) to the current age.


\hfill


\end{sffamily}\end{footnotesize}
