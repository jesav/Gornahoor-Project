\section{Human, All Too Human}

Tradition cannot be understood from the human, all-too-human, perspective. 

One criticism was based on something called ``owness", as though we pick and choose our gods on the basis of their suitability to our human condition, instead of rising above and seeing our human condition \emph{sub specie eternitatis}. A second criticism arose from the discussion of the I. The critic could only see subjectivity in this and the workings of an arbitrary will. Instead, it must be seen in the light of the One of \textbf{Plotinus}, the Atman of the \emph{Vedas}, the True Self of \textbf{Meister Eckhart}. This fundamental misunderstanding led him to flee, following some unkind words.

Lately, we seem unable to understand matter from the super-human perspective shown in the obsession with the human body. There is no point in debates, rather the goal is to ``see". There is a superior form of knowing (gnosis, episteme, intuition), that is more akin to ``seeing", properly understood, than to debating. There are forms of meditation, such as Hermetic meditation, and conditions for Hermetic initiation that are unrelated to any book-learning. There are three trials which are only the preliminary steps. Has anyone attempted them? Nothing here will make sense until they are at least attempted.

I call your attention to the Foreward to \emph{Revolt} (which is actually Evola's Introduction to the third Italian edition. Evola writes: (beginning with a response to his critics, who are bringing in the same accusations 40 years later. What irony!)

\begin{quotex}
In my perspective there is no arbitrariness, subjectivity, or fantasy, just as there is no objectivity and scientific causality the way modern men understand them. All these notions are unreal; all these notions are outside Tradition. Tradition begins wherever it is possible to rise above these notions \textbf{by achieving a superindividual and nonhuman perspective}; thus, I will have a minimal concern for debating and ``demonstrating". The truths that may reveal the world of Tradition are not those that can be ``learned" or ``discussed"; \textbf{they either are or are not}. It is only possible to \emph{remember} them, and this happens when \textbf{one becomes free of the obstacles represented by various human constructions}… in other words, \textbf{one becomes free of these encumbrances when the capacity for seeing from that nonhuman perspective, which is the same as the traditional perspective, has been attained.} 

\end{quotex}
So to anyone who wants ``proof", I can only say, ``Take a look and see for yourself. Then you may remember what you have forgotten."

\flrightit{Posted on 2011-03-21 by Cologero }
