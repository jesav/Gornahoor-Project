\section{Archaeology of the Soul}

\begin{quotex}
What tradition can remain to us of those generations that have not left us a single written line? … Fortunately, the past never completely dies for man. Man may forget it, but he always preserves it within him. For, take him at any epoch, and he is the product, the epitome, of all the earlier epochs. Let him look into his own soul, and he can find and distinguish these different epochs by what each of them has left behind.

\flright{\textsc{Numa Denis Fustel de Coulanges}, The Ancient City}

\end{quotex}
What we see described here is a method to uncover tradition. Like an archeologist at a site, we dig deeper and deeper into the soul to uncover more and more layers. Mr. Fustel studies legends and rites preserved by later Romans in order to extrapolate backwards to more ancient times. He also makes connections to the Vedas and Laws of Manu, which have been preserved until our time. Mr. Fustel was neither an esoterist nor a Hermetist, yet he understood this method. How much more, then, can a Hermetist uncover in a similar manner.

A simple Internet search will display incessant arguments about whether or not such and such a writer (e.g., Aurobindo) is a ``traditionalist" or which ``tradition" is more suitable. These discussions are along the lines of which tattoo to buy; different tattoos may appeal to different types of men, yet it will never be more than skin deep. Gornahoor is not interested in such debates, since there is only one Tradition: call it the Hyperborean Tradition\footnote{See Section \ref{sec:HyperboreaPrimordial} in this book.}, or the Indo-European, or La Tradizione Romana\footnote{\url{https://gornahoor.net/?p=1275}}. The way to understand it is to return to the Primordial State, deep within the soul. Hermetists do not recognize each other through their erudition, but rather through their depth. 

Thus, Gornahoor wants to change the terms of the discussion. Whatever your exoteric path, your ``tattoo" as it were, there are some common elements. The Hermetist must undergo the three trials\footnote{\url{https://gornahoor.net/?p=1902}}. He must master the four elements\footnote{\url{https://gornahoor.net/?p=1161}}. He understands the cosmic and juridical order\footnote{\url{https://gornahoor.net/?p=2053}}. He practices Hermetic meditation\footnote{\url{https://gornahoor.net/?p=898}}. He must have drunk the silence\footnote{\url{https://gornahoor.net/?p=1082}}, by calming the oscillations of the mind\footnote{\url{https://www.meditationsonthetarot.com/moral-purification-of-the-will}}. He understands the esoteric meaning of power and his True Will\footnote{\url{https://gornahoor.net/?p=1993}}.

In short, this path is not for the man with a pedigree but rather for the Twice-born, or noble man\footnote{\url{https://gornahoor.net/?p=1216}}. This path is not for the man with a degree, but rather for the man of power\footnote{\url{https://gornahoor.net/?p=1856}}. This path is not for the specialist, but rather for the \textbf{genius}\footnote{\url{https://gornahoor.net/?p=319}}.



\flrightit{Posted on 2011-04-13 by Cologero }

\begin{center}* * *\end{center}

\begin{footnotesize}\begin{sffamily}

\texttt{HOO on 2011-04-14 at 13:01 said: }

`According to this second and new version of history, civilization breaks down into epochs and disconnected cycles. At a given moment and within a given race a specific conception of the world and of life is affirmed from which follows a specific system of truths, principles, understandings, and realizations. A civilization springs up, gradually reaches a culminating point, and then falls into darkness and, more often than not, disappears. A cycle has ended. Perhaps another will rise again some day, somewhere else. Perhaps it may even take up the concerns of preceding civilizations, but any connection between them will be strictly analogical. The transition from one cycle of civilization to another—one completely alien to the other—implies a jump, which in mathematics is called a discontinuity.' [Evola, J. ``The Hermetic Tradition" p. 13]


\hfill


\texttt{Cologero on 2011-04-15 at 07:59 said: }

Yes, Hoo, but you omitted the following paragraph, in which Evola continues with his own version of the archeology of the soul. Otherwise, there could be no possibility of recovering the Hermetic tradition, which may be latent in some of us, those of us who ``remember". Evola claimed that there still existed some Roman families who maintained a hearth and continued to practice the ancient rites.


\end{sffamily}\end{footnotesize}
