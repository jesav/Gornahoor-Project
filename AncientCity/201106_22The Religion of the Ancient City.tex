\section{The Religion of the Ancient City}

\label{sec:ReligionAncientCity}

\paragraph{City States}
Each family in the Ancient City had its own religion, gods, and rites based on its ancestors, with the head of the family as its priest. Families could unite, called a phratry or curia, in a common worship without eliminating the family rites. They would create a new hearth, offer a sacrifice to the new god, and share the meal in communion. A tribe would consist of several phratries, thus uniting families. Once formed, it was closed to outsiders as it had its own form a worship. A city, then, comprised several tribes, each maintaining its own cult, while establishing a new cult for the city.

\paragraph{Condition of Unity}
So the city was a hierarchical confederation of families, curias, and tribes, not an assemblage of individuals. A higher entity could not interfere in the interior of a lower. A child was ritually admitted to his family within the first week, into the curia at a later age, and into the city as a teenager at which point he became a citizen. At each point he was initiated into the new cult. It was the spiritual unity created by the highest cult that united them as a people. It was the hierarchy of religious ideas that inspired and organized the city.

\paragraph{Religious Practice}
Each city had its own rites, liturgies, gods and forms of worship. There was nothing in common between cities, the priests maintained the sacred books in secrecy, and aliens were not allowed to participate or even enter a temple. No deviations or innovations were allowed.

\paragraph{High Priest and King}
When a new city was created, by some splitting off from the city or uniting with other tribes, the new city followed the same model. The leader was the High Priest who led the prayers, performed the rites, offered the sacrifice, and interpreted the omens. Thus the founder of a city was a holy man before he was a king. His primary virtue was piety and he exhibited a cold and lofty impersonality, making him more than a man, a god.

The High Priest of the city was also its King and his principal duty was to perform religious ceremonies. The ancients felt themselves to be totally dependent on their gods, and their religion was intertwined with every aspect of daily life. The king derived his authority from maintaining the city's hearth and was appointed by the gods. The allegiance was thus freely given and not dependent on force. The temporal power of the King did not extend to all the citizens. Rather, it went from the King to the heads of the tribes, who ruled over the families and clients of the tribe.

\paragraph{The Alien}
The citizen participated in the religious ceremonies of the city and if he failed to do so, he would cease to be considered a citizen. The alien could not participate in worship and was outside the protection of the gods of the city. The presence of the alien in a temple would defile it, so rites of purification would be necessary and a new hearth rekindled. Only rarely could an alien become a citizen, and then, only if he was a citizen of some other city. Although citizens may welcome a stranger, he was still restricted from the benefits of citizenry.

\paragraph{Autonomy}
Each city was autonomous, united by its common worship. Thus larger alliances with other cities was inconceivable. One city could not rule a nearby city by sending a governor, since that governor would have no spiritual authority in the other city. Warfare between cities was also a spiritual battle between the respective gods. The victors would assassinate every man, women and child of the losers, or else sell them into slavery. Sometimes the cities could form an alliance and share some forms of worship. The spiritual element counted for everything and genetic, linguistic, and cultural ties did not matter.

\paragraph{Conclusion}
The citizen had no freedom of religion; either he participated in the religion of the city or he was banished. The hierarchy of family, tribe and city introduced the idea of a wider and wider influence of the gods, but the knowledge of the one god had been lost. Philosophers and initiates in the mystery schools would have known otherwise, yet they also participated in the sacred ceremonies of the city. The esoteric did not preclude the exoteric, a point emphasized by Guenon and Tomberg.

The practices common to the Ancient City recur in different ways in the Medieval period. Priesthood, prayer, sacrifice, communion meals, liturgy, reading the signs of the times, piety, dispassion, the spiritual integrity and unity of the polis, insider vs outsider recognition, spiritual authority, the primacy of spiritual unity over genetic similarity: these are all aspects of Tradition of the past. The Tradition of the future will look similar.


\hfill

References

Numa Denis Fustel de Coulanges, \emph{The Ancient City}

Fabre d'Olivet, \emph{Hermeneutic Interpretation of the Origin of the Social State of Man}

Julius Evola, \emph{Pagan Imperialism}



\flrightit{Posted on 2011-06-22 by Cologero }

\begin{center}* * *\end{center}

\begin{footnotesize}\begin{sffamily}



\texttt{Jupiter on 2011-06-22 at 02:50 said: }

It's a good thing civilized religions like Buddhism and Christianity do away with quite a bit of this blasphemous pagan nonsense.


\hfill

\texttt{Matt on 2011-06-23 at 01:26 said: }

``It's a good thing civilized religions like Buddhism and Christianity do away with quite a bit of this blasphemous pagan nonsense."

What exactly makes Buddhism and Christianity civilized and paganism uncivilized nonsense? Both Christianity and Buddhism incorporated most of the religious practices of the regions that each doctrine spread to (in different ways) which the last paragraph of this post makes note of. Its about whether one sees a continuity between the classical/ancient era and the medieval era (the view of Tradition) or one sees little to no connections at all between the two eras (the view of the modern mindset).


\hfill

\texttt{Jupiter on 2011-06-23 at 07:35 said: }

There's no continuity between the degeneration at the end of one cycle and the enlightened brilliance at the zenith of another.

The kinds of things in this article have more to do with the Brahmans and Pharisees than the Buddha and the Christ. Superstitious plebeian savagery that assumes the Absolute gives a damn whether we make sacrifices to this dead relative or that piddly little nature spirit.


\hfill

\texttt{Matt on 2011-06-23 at 13:31 said: }

Jupiter,

That is an interesting take. But if you see what is stated in this post as an example of a degeneration in the Greek and Roman Era, wouldn't that therefore mean it happened right at the founding of Greek and Roman states, and therefore, a degeneration can't even be spoke of, since in your view, the ways of the Greeks and Romans were wrong from the start? After all, the religious rites and practices of the households had been around since the founding of those two civilizations. And it wasn't limited to just the plebs. The patricians had the rites and ceremonies of their households as well. And to connect this with the east, the rites of the Brahmins had been around since the start of the Vedic Era (as seen in the scriptures).

I do agree that there was a degeneration in India before the Buddha (over speculation by the Brahmins, breaking down of how the rites were performed), as well as one before the Medieval Civilization of Europe (hedonism, the rites largely became an empty formalism, possible nature animism, etc.), but I don't think the rites themselves are examples of the degeneration.


\end{sffamily}\end{footnotesize}
