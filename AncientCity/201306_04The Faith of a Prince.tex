\section{The Faith of a Prince}

\begin{quotex}
Europe arouses pity in the heart of the thinking man and horror in the heart of the virtuous man. \flright{\textit{Source unknown}}

\end{quotex}
In the century preceding Rene Guenon and Julius Evola, there was also a revolt against the modern world equal in depth and insight to what followed. To a large extent, this we spearheaded by several Catholic thinkers: \textbf{Joseph de Maistre}, \textbf{Vicomte de Bonald}, \textbf{Juan Donoso Cortes}, \textbf{Louis Veuillot}, not to mention several popes.

We shall, from time to time, bring to light their ideas, beginning with \textbf{Prince Clemens Metternich}. First of all, it is necessary to be clear what the ``Right" means in a specific and objective sense. If the Left represents the state of permanent revolution, then the Right must be, in de Maistre's expression, the ``opposite of a revolution". Revolution seeks to overturn the established, natural, and organic order of things through the secular state. Hence, the counter-revolution wills a hierarchical order and the state founded on true spiritual principles. In short, the Right is the party of the Logos and the Left, that of the anti-Logos. This will be the standard not only to judge the world, but a fortiori to judge those who claim to hold a more traditional view.

Hence, Metternich held that society must be governed by a true understanding of God and Man, or else it will devolve to the administration of passing needs. Unlike the modern mind for which everything is political, for Metternich the ``social principle" held primacy; this is the rights of the family, or, more generally, the principle of preservation and continuity. The following comments are derived from his personal memoirs (1847). He lists the symptoms of disorder in these words:

\begin{quotex}
Kings have reached the stage of wondering how much longer they are going to last; passions are let loose and are in league to overthrow all that society has hitherto respected as the basis of its existence: religion, public morality, laws, customs, rights and duties; everything is attacked, confused, overthrown, or made a matter of doubt. The great mass of the people looks calmly on, in the face of so many attacks and upheavals, against which there is an utter lack of any sort of protection. Some of them are lost in vague dreams, while an overwhelming majority desire the maintenance of a public order which no longer exists, the very first elements of which seem to have been lost. 

\end{quotex}
Metternich then asks two questions: (1) what is the cause of these disorders? And (2), are there any means of halting the growth of this disorder? We will let him speak of the causes. The answer to (2), more than a century and a half later, is still elusive. Those who are new to this line of thought are too often sanguine about the prospects and need to take the long view espoused here. Furthermore, it is fair to ask whether those claiming to be opposed to disorder are in fact contributing to it. ``Nothing is as fatal as error," he tells us, so, following his lead, we need to abandon the temptation to fall into it.

\paragraph{The Causes of Disorder}
Institutions are uncertain in their origin, go through periods do development and perfection, and then fall into decay. At the peak of their strength, there are two foundational elements: (1) the precepts of religious and social morality and (2) the local needs of man. When men rebel against that base, society moves to a state of unrest and eventually upheaval and bloodshed. Metternich admits that is not new in human history, but that in his time, it was more extensive than ever before. Of course, the succeeding century proved to be even bloodier and even more disorderly.

In a snapshot view of European history, Metternich points to the rise and fall of Rome, followed by the period of darkness. He then claims that the Christian religion dispelled that darkness, reestablishing civilization on a new basis of a pure and eternal law. Although some critics of the modern world may want to deny that, or at least deny that it would be applicable today as Metternich held, they have the burden of suggesting another source for the ``pure and eternal law".

The breakdown of the Medieval civilization was due, he says, to the perfect storm of the inventions of the printing press and gunpowder, the discovery of the Western hemisphere, and the Reformation. He writes:

\begin{quotex}
The march of the human spirit was therefore exceedingly rapid throughout the last three centuries. This march having progressed with a more rapid acceleration than the course of wisdom—the unique counterbalance of passion and error—had been able to take, a revolution, prepared by false systems of philosophy, and by fatal errors into which several sovereigns, the most illustrious of the second half of the eighteenth century, had fallen, at last broke out in that country [France] which was one of the most advanced in intelligence, the most weakened by a love of pleasure, in a country inhabited by a race of people who can be considered the most frivolous in the world, considering the facility they have in understanding, and the difficulty they experience in judging an issue calmly. 

\end{quotex}
\paragraph{Presumption}
In short, the disorder can be expressed in one word: presumption, the natural result of such a rapid progress of the human mind in material improvements. The influx of false ideas, the new riches extracted from the New World, and the revolution in the moral order resulting from the Reformation, led to the presumptuous man. Metternich expands on this idea:

\begin{quotex}
Religion, morality, legislation, economy, politics, administration, everything seems to have become common property, accessible to all. People think that they know everything; experience does not count for the presumptuous man; faith means nothing to him; he substitutes for it a so-called personal conviction and feels himself dispensed from any examination or course of study in order to arrive at this conviction, for these means seem too lowly to a mind which thinks itself powerful enough to take in at a glance a general review of problems and facts.

Laws are of no value in his eyes because he did not help to make them and because it would be beneath the dignity of a man of his caliber to recognize the milestones traced by brutish and ignorant generations before him. Authority resides in himself; why should he subject himself to what is only of use to a man deprived of intelligence and knowledge? What had formerly, in his view, been sufficient at a tender age non longer suits a man who has reached the age of reason and maturity, that degree of universal perfection which the German innovators designate by the idea, absurd by its very nature, of the emancipation of the peoples? Morality alone is not openly attacked, for without it he would not be sure of his own existence for a single moment; but he interprets it according to his own fancy and allows everybody else to do the same thing, provided that the other man neither kills nor robs him. 

\end{quotex}
Of course, Metternich could not imagine how far ``morality" would come to be ``reinterpreted" today and how the ideal of emancipation would be extended. Metternich says that what he described can hardly be called ``society", since the social elements have all been individualized.

\begin{quotex}
Each man is the head of his own dogmas, the arbiter of laws according to which he can deign to govern himself, or allow others to govern him and his fellows, in a word, the only judge of his faith, of his actions and the principle according to which he means to regulate them. 

\end{quotex}
As a proof of this, Metternich interestingly points to the fact that nationality—one of the most natural sentiments in man—is lacking in the liberal ideology. Nationalism, or its contemporary analog ``identitarianism", was never denied by the religious and spiritual systems that Metternich followed. There is the idea, today, that the opposite, and hence enemy, of nationalism is something called ``universalism", allegedly taught by the spiritual authority prior to the French revolution. Quite the contrary, according to Metternich the natural sentiment of identity is abolished by the spirit of individualism, which by its nature denies such ties. A collection of individuals is easier to control. Metternich explains:

\begin{quotex}
The true aim of the idealists of the [liberal] party is religious and political fusion and, in the last analysis, is not other than to create in favour of each individual an existence which is entirely independent of all authority and all will, except his own—an absurd idea, which is contrary to the spirit of man and incompatible with the requirements of human society. 

\end{quotex}

\hfill

Suggested reading: \emph{Catholic Political Thought 1789-1848}, ed Bela Menczer 



\flrightit{Posted on 2013-06-04 by Cologero }

\begin{center}* * *\end{center}

\begin{footnotesize}\begin{sffamily}



\texttt{Ash on 2013-06-05 at 00:26 said: }

It's ironic that the Left itself, seeing the failure of ``emancipating" man from his social order, would go on to create ideologies like socialism and Marxism which attempted to correct against the detrimental effects of old liberalism. It's often missed by the Right that the Left itself saw these failures in their predecessors. Even the rights of freedom of speech and association, previously trumpeted as fundamental, are spat upon these days in the respectable academic left, such as in doctrines like the ``safe space", where speech must be carefully controlled so no one feels threatened. Of course, this usually becomes a way to control against the ``threats" of ideas which are outside the accepted ideological and metapolitical bounds of those who rule the ``space" in question. 

It will be interesting to hear from Metternich, someone whose work I myself have not yet studied much; thank you for posting him. Unfortunately, a major theme in these and later generations of Rightist thinkers is that they seem to regard ``reinterpretation" of morality and faith as bad in and of itself. At least, they don't examine much what a positive reinterpretation could look like. Students of Tradition found on this site accept that the myths, spirit, and morals of Rome were reinterpreted over the centuries it took for Christianity to give a new life and meaning to old traditions in and outside the Empire. It would be useful to see how more recent developments can harnessed for similar ends (such as Alan Watts attempted to do with science and the modern market). Clearly, any reinterpretation that strengthens or improves upon old traditions (as opposed to Tradition), is desirable. This is especially true when the old myths have long lost their power, save for those who know how to hear them. Frederick the Great of Prussia, admired by Evola, may be a more aristocratic example of such a re-interpreter (though his ultimate success may be questioned indeed).


\hfill

\texttt{Cologero on 2013-06-05 at 07:24 said: }

Interesting points, Ash. As regards reinterpretations of morality, I believe Metternich was referring primarily to individuals inventing private moral codes rather than communal mores which are somewhat more adaptable. I did leave a small part out, where he wrote about variations due to the material and environmental factors in which a community arose and exists. Nevertheless, Metternich accepted a higher law, or as said nowadays, the ``permanent things."

I suppose verbal ``threats" to one's feelings are now considered analogous to bodily threats. It seems that new types of group identity are now in play, based primarily on behaviors rather than on the natural sentiments of the ``nationalism" that Metternich referred to.


\hfill

\texttt{Logres on 2013-06-05 at 13:38 said: }

``In short, the disorder can be expressed in one word: presumption, the natural result of such a rapid progress of the human mind in material improvements. The influx of false ideas, the new riches extracted from the New World, and the revolution in the moral order resulting from the Reformation, led to the presumptuous man…"

Brilliant. 

``Quite the contrary, according to Metternich the natural sentiment of identity is abolished by the spirit of individualism, which by its nature denies such ties. A collection of individual is easier to control."

J.Nisbet's phrase for this is ``intermediate institutions". They provide a buffer between the Nomos of the State and the local self-government that is proper to the state of fallen (and even un-fallen) man. The naked Self is what the Left wants to ``unveil" as their grand finale, a naked Self devoid of rule by General Law, the hierarchies of heaven \& earth. Althusius made one of the few Calvinist attempts at formulating a real political theory which took this into account: http://en.wikipedia.org/wiki/Johannes\_Althusius.

The Church made a big mistake in losing the idea that even un-fallen man (Adam) would have had a form of government; once one believes that one has overcome Sin and achieved the emancipation of peoples, then if no need for government remains (having achieved Eden), the ideas of the Left follow naturally. 

Phillip Rieff is recommended reading for anyone wanting an expose of Fanon and the trendy ultra-modern deconstructionism which logically results from emancipating the naked Self. Mouravieff's doctrines of General Law (discussed also by Ouspensky in The Fourth Way) can help the Right and the Church re-establish an esoteric basis for exoteric doctrine. ``Force \& Right rule the World – Force till Right is ready" (Bishop Joubert). Even dictums like this can be ``re-interpreted" by the Left. 

How to keep man from thinking he is ready when not?


\hfill

\texttt{Timotheus Lutz on 2013-06-06 at 02:27 said: }

A good book on him is `Metternich 1773-1859: A Study Of His Period And Personality' by Algernon Cecil, uncommon but definitely available from online sellers for reasonable prices. It's one of those gems that fell through the cracks.

`You-know-who' liked it, if that means anything.


\end{sffamily}\end{footnotesize}
