\section{Prolegomena to Noble Philosophy}

\begin{quotex}
Nobility, the quality which I call true aristocracy, requires of a man the recognition of his guilt. In its depths, conscience, which is frequently covered up and suppressed, is always a consciousness of guilt. The necessary thing is to take upon oneself as much guilt as possible and to put as little as possible of it upon other people. The aristocrat is not one who is proudly conscious of himself as first, as a privileged being, and who safeguards his position as such. The aristocrat is the man who is aware of the guilt and sinfulness of this first place, this privileged position of his. \textit{The sense that one is being continually affronted is on the other hand, precisely a plebeian feeling}. \flright{\textsc{Nicolai Berdyaev}}

\end{quotex}
I've been away with severe bronchitis, probably resulting from a pneumothorax I suffered two months ago. I tried the auto-suggestion exercise, but was unable to achieve any lasting results. I found it difficult to reach any transcendent state as gasps for air brought me back. Of course, on the physical plane, possibilities may be exhausted in a process of densification. Despite trying ayurvedic and Santeria home cures, (everyone had an opinion), I eventually turned to heavy doses of white man's medicine. Interestingly, while the body is still subpar, there was a specific spiritual healing which I shall keep to myself.

In that time, I considered several ways to return to the topic. Specifically, what can we glean from the documentation we have on the formation of Tradition in the middle ages? This will require a review of the historiographical method as well as a refresher on the \textbf{Ancient City}. Although I like \textbf{Alexander Jacob}'s term ``aristocratic philosophy", the title of his book, Nobilitas, better suits the purpose. The original term may be misleading, since it is a philosophy for aristocrats, not from aristocrats. After all, an aristocrat is a gentleman and therefore ill-suited to the nitpicking of critical philosophy. So while a \textbf{Socrates} is debating the meaning of courage, piety, justice, etc., in thought, only to end up in aporia, the aristocrat is effecting those virtues in being. He strives to ``be" courageous, pious, just, and so on. Although I would prefer to use the Sanskrit word for noble\footnote{\url{http://www.behindthename.com/name/arya}}, the English ``Noble Philosophy" will suffice for obvious reasons.

Yet even the word ``philosophy" is misleading, since profane philosophy has come to mean something critical or speculative rather than a way of life. That is why Guenon distinguished between metaphysics and philosophy, yet ``Hermetic philosophy" is actually metaphysics. Profane philosophy represents a decline, since it attempts to do in thought what had been accomplished in being.

In Professor Jacob's study of aristocratic philosophy, there is a gap between ancient philosophy as represented by \textbf{Plato}, \textbf{Aristotle}, and \textbf{Cicero}, and the Renaissance and later thinkers. In a sense, profane philosophy represents decadence and a decline from more traditional thought. Thus, Plato created a republic in thought, whereas the ancient Greek city created them in reality. Those cities were established by a godling or man with divine qualities. They were initially ruled by a priest-king and religious rites organized the life the city and its inhabitants. Aristotle presumably had access to those ancient constitutions from which he developed his political philosophy. Cicero admitted he could not understand the ancient Roman customs and laws.

\textbf{Ulrich Valange} observed that High Culture requires culture creators and culture sustainers. So in the Ancient City, the founders were the creators and the nobility were the sustainers, so long as they kept the hearth burning. The clients, or plebes, had no culture of their own and simply followed the culture and rites of their particular ruling family. In this sense, therefore, the culture was ``imposed" on them. This relates back to Berdyaev's opening quote.

The esoteric method is neither argumentative nor empirical. Rather, it is descriptive, since to know requires an intuition, that is, something more like seeing. It is based on principle, so we don't reason from the many to the One, but instead we strive to understand the many from the One. Hence, the principle of the Ancient City, as it relates to High Culture, manifests as creators, sustainers, as well as the passive elements. We are not interested in the faux debates among these groups, but rather in observing how they manifest the principle.

The true aristocrats are acutely aware of their role as the culture sustainers and strive to fulfill that role. Whenever there is a failure in some regard, they become ritually unclean and need to restore the balance. Initially, this was exteriorized, but becomes interiorized in the Middle Ages as we shall see. They are motivated by honour and duty, not by their ability to dominate, although they exercise the power of command as necessary.

Lower than them, are the plebes. They can neither fully understand nor participate in the culture. Hence, they become prone to feelings of ressentiment. They feel that something has been ``imposed" on them. We see that especially in certain neo-pagan circles, who complain that medieval Christianity was forced on them. Understand, from our perspective, that it is pointless to engage in some mock interminable debate; rather, we simply see what type of person holds what type of worldview. As a reminder, from the esoteric point of view, this is what \textbf{Rene Guenon} wrote, (regarding whether Dante was a Catholic or a pagan):

\begin{quotex}
For our part, we do not think that such a point of view is necessary, for true esoterism is something completely different from outward religion, and if it has some relationship with it, this can only be insofar as it finds a symbolic mode of expression in religious forms. Moreover, it matters little whether these forms be of this or that religion, since what is involved is the essential doctrinal unity concealed beneath their apparent diversity. 

\end{quotex}
Those who feel victimized, then, simply do not and cannot understand this. The prime example, as we have repeatedly pointed out, is the legend of the Virgin Spring\footnote{\url{https://www.gornahoor.net/?p=1416}}; there we see the aristocratic and plebeian characteristics fully revealed. These are not judgments; rather they are descriptions of certain states of consciousness.

Next: The Foundation of the Middle Ages.

\flrightit{Posted on 2014-02-13 by Cologero }
