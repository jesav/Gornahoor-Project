\section{The Son of Duty}

\label{sec:SonDuty}

In \emph{Men Among the Ruins}, \textbf{Julius Evola} writes:

\begin{quotex}
Ancient Indo-European traditions regarded the procreation of a son as a ``duty": because of this, the firstborn was called the ``son of duty," in distinction from any subsequent children. 

\end{quotex}
This involves both a distortion and an omission. The son is not a duty of the father, but rather the duty the son inherits. Furthermore, the distinction mentioned, but not specified, is not without import. \textbf{Fustel de Coulanges}, in describing the Aryans, quotes the entire sentence that Evola relies on:

\begin{quotex}
The oldest son was begotten for the accomplishment of the duty due the ancestors; the others are the fruit of love. 

\end{quotex}
Hence, Evola's criticism of those who choose family life is without merit. The ensuing children are not sons of duty; hence they are not bound to follow Evola's path that sometimes more resembles a Gorean fantasy than that of a serious man. The idea that children are desired as the fruit of love and not the result of impulsive sexual urges is ignored by Evola. In the \emph{Aitareya Brahmana}, which Evola was familiar with, the value of a son is described by \textbf{Narada}\footnote{\url{https://en.wikipedia.org/wiki/Narada}} in no uncertain terms:

\begin{quotex}
 \includegraphics[width=4.53cm,height=6.392cm]{a20120403TheSonofDuty-img001.jpg} 

The delights of in the earth

The delights in the fire

The delights in the waters of living beings,

Greater than these is that of a father in a son

A son is a light in the highest heaven.

The gods said to men:

A sonless one cannot attain heaven,

All the beasts know this

This is the broad and auspicious path

Along which men with sons fare free from sorrow. 

\end{quotex}
\emph{This is how our ancient Aryan ancestors actually lived}. The \textbf{Aryan gods} advised a man to have sons; this is not just an aberration of the Catholic god as Evola insinuates. Yes, but not just the gods, even the beasts know as much. So in criticizing the family, Evola stoops lower than the animals.



Besides love and joy, there is a third factor: a sonless one cannot attain heaven. It is not just a question of continuing a biological line, but more important is the continuation of a spiritual tradition.

Now a man may sacrifice his family life to focus on a spiritual life or even, perhaps, a secular order (which Evola claimed existed) of unmarried men. Of course, once a man has fulfilled his ``duty", he may choose the celibate life of a sannyasi. But in Evola's conception, these choices hardly count as a sacrifice. Now this is counter to what he writes in \emph{The Metaphysics of Sex} about marriage. This is summarized by his quoting \textbf{Louis Claude de Saint-Martin}:

\begin{quotex}
If mankind knew what marriage is, it would have at one and the same time an extraordinary desire for it and a tremendous fear of it; for by means of it man could once again be made like God or could end in total ruin. 

\end{quotex}
Pace Evola, we believe Saint-Martin understood perfectly well what he wrote, but that is a topic for another time.



\flrightit{Posted on 2012-04-03 by Cologero }

\begin{center}* * *\end{center}

\begin{footnotesize}\begin{sffamily}



\texttt{apeiron on 2012-04-05 at 19:29 said: }

Leonidas picked the famous 300 Spartans on the predicament that they had a ``son of duty" to their name. So some of the most celebrated heroic band of solar men in Western history, procreated. Today, the act of familial procreation is seen as a bourgeois ``facist" institution by the left and as ``judaic familial values" from the ``new-right". How is this a healthy or even functional view of our ancestral destiny if this is the layman's view and not that of an exceptional and vocational spiritual type? 

I believe Evola also deliberately confuses the old testament quote of ``go forth and multiply" to the lower type of materialist man. In saying this however, we're left with the beings of sacrifice, the celibate saints, priests, holy-men and mystics who actively chose the sterile life to center their energies entirely towards god/godhood. What are we to make of that tradition (even though Evola actively chose not to be traditional as demonstrated on this site) ?


\hfill

\texttt{Angolmois on 2012-04-11 at 05:44 said: }

Marriage and children is a university / school of spirit, as the ancients knew it. For many people it would be a very good school to learn essentials of spiritual and religious life, such as (self-)sacrifice, commitment, focusing of sexual and erotic energies (for the wellbeing and integrity of one's own constitution), and so on.

There's is nothing inherently bourgeois in family life and having / raising children, but I do understand why spiritually inclined people don't marry and / or have children; especially in this age of decline and anti-tradition family – society, civilization at large – is not what it was in the past, and it can be both a spiritual and a practical, financial burden because of reasons too many to even mention in a short comment; our crumbling civilization does everything in its power to kick people with children and healthy family values to the drain. Of course there's always the possibility of keeping the flame of tradition (and bloodline) alive in one's own part, but is this the best way to do this, and not just the most wishful thinking that one can engage in?

This is why I understand Evola's views about marriage and children very well. It should also always be remembered what Evola meant by his words and writings, and to whom they were first and foremost aimed at: for men, warriors and sages of tradition.

I believe Männerbunds were not homosexual in nature – they were a-sexual – beyond sex.

If someone discards the bible as a whole on the basis of ``semitic values", they certainly have not read their New Testament through.


\hfill

\texttt{Cologero on 2012-04-11 at 07:42 said: }

I wish I could accept that, Angolmois, but this shows Evola's naïveté. First of all, he was no warrior and no sage. Granted there may be some men who choose an ascetic path early in life, but in the normal course of things, that choice is made after having fulfilled one's duties. In ancient Greece, the indomitable phalanx was manned by father and son, uncle and nephew, fighting beside each other, not by celibate warriors.

Despite all his talk about loyalty, fidelity, and authority, Evola committed himself to nothing; not to his own nation, not to his leader, not to any specific Tradition. He unfairly criticized the great philosopher Giovanni Gentile who, unlike Evola, did not philosophize in a vacuum. He raised a family, had a worldwide reputation as a philosopher, and participated in forming the Italian government, specifically in the reform of education (something Evola only talked about).

When the Nazis kidnapped Mussolini and formed the ill-fated Salo Republic, Gentile followed him, although he had nothing to gain; the Allies would have treated him gently. Against the advice of his family, he understood what loyalty meant. After the war, Gentile was gunned down in the streets of Florence by Communist partisans. Meanwhile, Evola remained behind in the safety of Rome. When the allies came looking for him, his mother helped sneak him off to Austria. Try to convince me who was the real man.

He advocated birth control for the women to be abused by the Mannerbund. Where did he expect them to come from, if a woman was under the protection of fathers, brothers, and then husbands? Apparently, from the Bohemians, or perhaps slaves from other nations and races. Apparently, he couldn't conceive that birth control used in that way led to the feminism he decried. Paradoxically, it also gave every Western man the same privileges with women that he would have reserved to the mannerbund.

The final effect of Evola's anti-family tirade is that it has brought some unsavoury figures into the sphere of Tradition, the sort of male that would never have been allowed in Traditional leadership.


\hfill

\texttt{Angolmois on 2012-04-12 at 04:08 said: }

You are right about many things, Cologero. I if anyone do not look well many Evola's views about women, which were quite harsh, dangerous and abusive. He certainly was ``too solar" in his emphasis, and this shows everywhere in his thinking and in even in his fate.


\hfill

\texttt{Sparrow on 2015-04-06 at 00:40 said: }

While your criticism of Evola is harsh, I can't help but find myself in full agreement, Cologero. A lot of these same observations came to me when reading his works, before finding Gornahoor. Despite his intelligence, or possibly because of it, Evola lacked good sense, and could be an incredibly argumentative, hypocritical, and selfish person. His sexist views on women (I don't mean ``sexist" in the liberal sense, where acknowledging difference between the sexes and supporting gender roles makes one a ``misogynist.") seemed ironic to me, considering it was certainly not to his detriment that his mother distracted Allied agents long enough for him to sneak out the door. He couldn't be pleased no matter what anyone said or did (e.g he didn't hesitate to mockingly criticize Guenon).

It is incredibly pedantic of me, but I feel the need to point out that Gentile was not killed after the war. He was killed in 1944. Ironically, he had been arguing for the release of anti-fascists (guess that's like petting a rabid dog, only for it to bite your arm off). Goes to show what happens when you attempt to show compassion to those of the Leftist persuasion.


\end{sffamily}\end{footnotesize}
