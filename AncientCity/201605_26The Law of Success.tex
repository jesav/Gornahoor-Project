\section{The Law of Success}

\begin{quotex}
Those who make plans will be born to carry them out. Those who make no plans need not be born. \flright{\textsc{Nisargadatta}, \emph{I Am That}}

You are the ``master of your fate" and the ``captain of your soul," by reason of the fact that you control your own thoughts, and, with the aid of your thoughts, you may create whatever you desire. \flright{\textsc{Napoleon Hill}, \emph{The Law of Success}}

\end{quotex}
The question of why some men are successful and others are not has always intrigued me. I am speaking, of course, only of those men who seem capable of great success: they may possess intelligence, creativity, charm, education, good breeding, yet never reach certain heights. In many cases Fortune plays a role, since chance can never be eliminated from life as a factor. Although we typically write from the perspective of the first caste, this post will focus on the second and third castes, that is, those involved in political and economic activity.

\textbf{Napoleon Hill}, in \emph{The Law of Success}, published his study of the characteristics of several successful men of his era\footnote{I am not particularly recommending Napoleon Hill, but am using him as a resource to understand the Kshatriya and Vaishya castes in the contemporary worlds.}. Success, according to him, begins in the imagination, with an idea. Obviously, this is the active imagination, not the passive imagination of daydreaming. Before dismissing this work as just pop psychology, this is just one of the Hermetic and/or spiritual teachings that he adapts to worldly success. A lot of contemporary New Thought or Law of Attraction teachings are distorted or incomplete renditions of Hermetic ideas.

The epigram above could have been written by Julius Evola, for example, who stressed the same inner state of self-mastery. It is no coincidence that Evola was concerned primarily with the Kshatriyas, not unlike HIll. Hill lists several negative emotions or habits—including suspiciousness, jealousy, uncontrolled sexual desire—that are obstacles to success. Enthusiasm, or thumos in our view, must be balanced with self-control. He mentions the Law of Mental Telepathy by which other can ``tune in" to your thoughts. Your self-confidence will be sensed unconsciously by others. These examples can suffice for now.

An important quality for success is the ability to secure cooperation from others. This is called leadership. The main motives that impel men to action are:

\begin{enumerate}
\item The motive of self-preservation 
\item The motive of sexual contact 
\item The motive of financial and social power. 
\end{enumerate}
Those who understand those motives will have greater ability in motivating others. Note that the motive of rationality or intellect and the motive of spiritual development are not at the top of the list. The Leader will try to gain cooperation through group harmony, and endeavoring to convince them so subsume their own interests to the group identity. Hill writes:

\begin{quotex}
Find a motive around which men may be induced to rally in a highly emotionalized, enthusiastic spirit of perfect harmony and you have found the starting point for the creation of a Master Mind. It is a well known fact that men will work harder for the attainment of an ideal than they will for mere money. In searching for a ``motive" as the basis for developing co-operative group effort it will be profitable to bear this fact in mind. 

\end{quotex}
Some leaders know this by chance, while others understand the laws of motivation consciously. What Hill says can be verified: watch the news, read web sites, and try to see them in action. See how group identity and cohesion is formed, for example.

\paragraph{Attraction and Repulsion}
Hill asserts that snap judgments and hunches come from a telepathic connection. Since men are not interiorly united, their audiences will pick up conflicting signals. That is why one man can be associated with widely divergent emotional reactions. Of course, the receivers of the telepathic communication have their own inner limitations, constrained by their three chief motives, which will distort the message. Unfortunately, most people never get beyond this reactive stage, particularly in the political realm. Facebook's success is based fundamentally on clicking ``like", proving the point.

Yet this is to live at the very lowest state of stimulus/response of the etheric body, the pranamayakosha. This is the intellectual life of an amoeba. At least for the amoeba, attraction and repulsion can be a life or death decision, but for humans it just gets in the way of intelligent discourse.

\paragraph{White Collar Criminals and Success}
I will add my own law of industriousness. Few men will actually act on their idea. A major motivator for this blog was my encounter with two white-collar criminals, primarily through their ex-wives. I got to know a lot about them, in great part because the wives held a great admiration for the men, despite the lies, betrayals, and financial decline that came to them.

These men were actually quite industrious in developing their schemes, maybe because it can be more difficult to live a lie than the truth. I got to see a large part of it. Denying their own moral culpability for their crimes, they were quite libertarian in their views. By that I mean that, in their own minds, they never coerced anyone, and their marks voluntarily played their roles in the scam, either out of greed, ignorance, or stupidity. The criminals felt no responsibility for the voluntary actions of their victims.

So I resolved to be industrious, since I have plenty of imagination and self-confidence. I have the first law of a ``definite chief aim", although I doubt anyone has figured it out. Most people who write me privately seem to think all these posts are random and independent, mostly for entertainment purposes. That is because they cannot grasp anything new, but can only interpret new things in terms of what they already know.

I must be low in a ``pleasing personality" and the ability to ``elicit cooperation". The most inane facebook posts get several dozen, if not hundreds, of ``likes". Gornahoor, after weeks of research, and hours of writing from the heart for each post, might get half a dozen ``pity likes". I do appreciate them. We never get invited to speak at conferences or Internet podcasts.

So we are down to our last 50 posts. We could reveal the chief aim in \#1001, or we may just fade away.

\paragraph{The Elusivity of Excellence}
\begin{quotex}
Imagine yourself suddenly discovering that most of your philosophy of life had been built of bias and prejudice, making it necessary for you to acknowledge that, far from being a finished scholar, \emph{you were barely qualified to become an intelligent student}! \flright{\textsc{Napoleon Hill}}

\end{quotex}
The number of men who are able to live by pure reason, the higher intellect, or logos, is very small. The majority are dominated either by eros (sex, desire) or by thumos (anger, ambition). Hence, there will always be an irrational component in political and social life. In Napoleon Hill's conception, reason is to be used for self-preservation, sex, and money. In his own self-development, he came to realize three things regarding worldviews:

\begin{itemize}
\item That you may learn how and where you acquired your philosophy of life, in general; 
\item That you may trace your prejudices and your biases to their original source; 
\item That you may discover, as I discovered, how largely you are the result of the training you received before you reached the age of fifteen years 
\end{itemize}
Even Hill's outlined plan for a more peaceful world cannot overcome that irrational component because it requires the manipulation of emotion.

By a life of pure intellect, I mean one that is dedicated, but not limited, to meditation of Truth, Goodness, and Beauty. This is rare to achieve, but well-worth the effort. Contrariwise, even intelligent and nominally educated people are focused on particulars.

\begin{itemize}
\item They are intrigued by true facts, but not by Truth 
\item They desire good things, but not the Good 
\item They are attracted to beautiful things, but not to Beauty 
\end{itemize}
These all have their deformations.

\begin{itemize}
\item They are intrigued by falsities that confirm their prejudices and worldviews 
\item They desire the pleasurable, even if it is not ultimately good 
\item They are attracted to unusual and abnormal 
\end{itemize}
I was educated in a Boston area school system that tried to teach excellence. The most intelligent, the best informed, and the cultured would be rewarded. However, the leaders of organizations attained, and retain, their positions, not through intellect alone, but because of their skills at organizing, not to say manipulating, others.

\paragraph{Patrician Rulers}
\begin{quotex}
Foolish consistency is the hobgoblin of little minds. \flright{\textsc{Ralph Waldo Emerson}}

\end{quotex}
Since rule by the philosopher-king is unlikely, the best kind of regime is the rule of the gentlemen, that is, of the patricians whose wealth comes from land, not from commercial activity. Because such men are guided interiorly by thumos, there will always be an irrational element in political life. This should be obvious, not just in theory, but also is easily verified empirically. Those who aspire to a life of action should pay close attention.

Whereas the masses are passive and contribute to society by procreation, the gentleman class redirects eros into more creative channels. The desire for fame feeds their ambition to lead and to rule. Such men are necessary despite the risk of bad rulers, who are not fully rational. Napoleon Hill makes the following point, based on his studies of successful men:

\begin{quotex}
All of the great leaders, in whatever walks of life they have arisen, have been and are people of highly sexed natures. 

\end{quotex}
In colloquial terms, these are the ``alpha males". They have an excess of sexual energy that gets sublimated to fame, fortune, and ambition. This separates the successful in worldly terms from the merely intelligent.

A successful system of rule was the Roman consul in the early years of the Republic. There were two consuls, each with veto power over the other. They were taken from the Patrician class and limited to a term of just one year. The idea of being a professional politician, such as we see today in the so-called Western democracies, was anathema. Today's professional politicians become skilled at just one thing: the learn how to manipulate the political system. The Patrician, on the contrary, had demonstrated his skill at being successful in the world, as was committed to the welfare of the nation, not to a political faction or party.

To jump ahead a couple of millennia, we can look at the phenomenon of \textbf{Donald Trump}. Let's detach from the spontaneous arising of either attraction or repulsion, often associated with him, and consider him as an archetype. Since the plantation system ended some time ago, Trump is the closest we have to a Patrician, having achieved financial success primarily through real estate rather than buying and selling, or worse. Moreover, he is an ``outsider" to the political system. Like him or not, you would have probably felt the same way about particular Consuls. If his personality of off-putting to you, consider that a man cannot achieve such worldly success by being just like you. A young mother of four sons despises him, so I posed a question to her. Suppose a gypsy fortune teller offered these predictions for your sons:

\begin{enumerate}
\item One would get an MBA degree from Wharton, one of the top 3 business schools in the country. 
\item The second son would make a billion dollars developing real estate around the world. 
\item The third would become a big TV star.\footnote{I have never watched an episode of the Apprentice.}
\item The fourth would run for president and defeat 16 other professional politicians 
\end{enumerate}
Objectively that is quite a resume and what mother would be displeased with that prognosis? Now I would prefer a Patrician to have more of that Roman dignity than we see in Trump. Nevertheless, in a democracy we can only vote for what is available. The question is still open: what kind of a man (or woman) do we admire? Someone who became wealthy by capitalizing on political connections? And excelling at party politics is hardly a skill worth bragging about.

Learn from this, all who aspire for leadership. It is never a matter of just winning intellectual arguments on abstruse issues. And minor contradictions don't matter. The intelligentsia – that is, those who earn a living solely on words like journalists, lawyers, and so on – get exercised by so called ``gotcha" questions: a minor inconsistency, an offensive remark, etc. In that way, they become totally oblivious to the actual world, mistaken their purely verbal formulations for reality itself.

\flrightit{Posted on 2016-05-26 by Cologero }

\begin{center}* * *\end{center}

\begin{footnotesize}\begin{sffamily}



\texttt{Thomas Blanchard on 2016-05-26 at 11:20 said: }

I'm curious on what you base your assertion that Napoleon Hill is primarily concerning himself with Kshyatrias. His goals and attitude seem quite bourgeois – after all, you note that he puts reason at the service of self-preservation, sex, and money. You note that you're referring to both the Kshyatria and Vaishya castes, but what differentiates them? Is it the role of thumos dominating the soul versus eros/material appetite?


\hfill

\texttt{Cologero on 2016-05-26 at 12:20 said: }

I need a little poetic license here, Thomas, and I thought it was clear I was not recommending Napoleon Hill as an authority. At the end I wrote: ``to understand the Kshatriya and Vaishya castes in the contemporary worlds". So the intention was to describe the contemporary manifestation of the castes, not their traditional roles. To give him his due, however, is it bourgeois to achieve self-mastery over your own thoughts? In describing the situation of the modern world, I took as a source someone who has studied successful people in that world. The laws of power have not changed over the centuries, however.

Hill is not putting reason at the service of …, he is describing what people actually do. The source for the thumos comment is actually Plato: to the extent that rulers are led by thumos, there will be a residue of irrationality in politics.

The roles of the two castes are hard to differentiate today. Consider how much debate there is about the influence of commercial interests over the political process.


\hfill

\texttt{Pickman on 2016-05-26 at 13:50 said: }

What good will it be to reveal the chief aim in the last post when it can no longer be debated? Why not now?

The reason this blog gets so little attention is simply because of its lack of marketing. It has great content but it's aesthetic presentation and outreach are rather muted. Whereas slick `all style/spin and no substance' propaganda gets the masses attention. It has always been the case (perhaps an inversion of recognisable quality) but that is just the way the world media works.


\hfill

\texttt{Jacob on 2016-05-26 at 19:38 said: }

Cologero, I'll be very sad when the posts on this blog end. I have always felt that you were the most insightful author I have ever read. I say this not as flattery, but because I believe it to be true. I often feel so alienated from the world today with its constant focus on desire and nothing more. At the same time I realize that I do need to be more industrious with the knowledge I have. Recently, I have begun to treat your writings in a way many treat sermons. Get a message that confirms my worldview and continue on without any deep reflection. In any case, thank you for your time. 

On Trump: I am deeply conflicted because I really like how he is willing to bravely deny the tenets of the Left. However, he seems to be sign of our age as well. The man who is the most entertaining and outlandish gets the most attention. It's possible that he is using the tenets of post-modernism to achieve higher goals (riding the tiger), but I don't see any evidence that he has any knowledge of anything higher than power. Still it's better to take a bad leader who establishes order than complete chaos right? I'm almost to the point that withdrawal from the world (especially in politics) sounds like an appealing option.


\hfill

\texttt{Matt on 2016-11-09 at 05:15 said: }

All those pundits trying to predict the presidential election would have done themselves a service by reading this post.


\hfill


\end{sffamily}\end{footnotesize}
