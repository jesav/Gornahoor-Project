\section{Solidarity and Continuity}

\textbf{Auguste Comte} was a 19th century French philosopher, although seldom read today. He called his philosophical system \textbf{Positivism} and he is rightly regarded as the founder of sociology as a science. Dismayed by the French revolution and its aftermaths, Comte was motivated to put the civilization of the Middle Ages on a foundation secured by science rather than faith. His efforts should serve as a model for all contemporary secular reactionaries.

He regarded the social structure of the Middle Ages under the influence of the Catholic Church (``what every well-born person considered sane and normal") as the ideal for humanity; however, in an age of unbelief, only an appeal to science rather than faith can lead to that conviction. He began by ordering the sciences in their natural sequence: mathematics, astronomy, physics, chemistry and biology. Beyond that he proposed sociology and the moral and political science. Towards the end of his life, he attempted to found a non-supernatural religion that he expected to be at the top, though it met with little success.

The Positivist is concerned only with facts and the laws that can be derived from them. This precludes any non-empirical assumptions of a philosophical, ideological or metaphysical nature. In particular, his system doesn't presuppose the doctrines of materialism and determinism which undergird science done today (even if only as methodological presuppositions). Since each science in the hierarchy has its own set of facts, he also avoids reductionism.

Thus, his science of sociology is qualitative since that is the most appropriate way to understand man. This is in sharp contrast to the sociology of today which attempts to apply quantitative methods to describe human behavior. It is a fortiori opposed to the reductionist efforts that go under the name of sociobiology which replace observed facts with the alleged unobserved acts of genes. No, by direct observation we see the three primary elements human nature: \textbf{Feeling}, \textbf{Thought}, and \textbf{Action}. This is where the sociologist starts.

Now, just as the astronomer studies the movements of the planets, and so on for each science, the science of man must being with History, ``regarded as a continuous whole". That provides the set of facts for the study of man, hence, that must be our starting point (not genetics, dialectical materialism, etc.).

\begin{quotex}
The principal feature [of Positivism] is a theory of history which enables us appreciate and become familiar with every mode in which human society has formed itself.

\end{quotex}
As an example of a social law, we can look at the basis for the cooperation of effort in political and social action.

\begin{quotex}
The consensus of the social organism extends to \textbf{Time} as well as \textbf{Space}. Hence the two distinct aspects of social sympathy: the feeling of Solidarity, or union with the Present; and of Continuity, or union with the Past. Careful investigation of any social phenomenon, whether relating to Order or to Progress, always proves convergence, direct or indirect, of all contemporaries and of all former generations… In our thoughts and feelings such convergence is unquestionable; and it should be still more evident in our actions, the efficacy of which depends on cooperation to a still greater degree.

\end{quotex}
In an age where social sympathy is lost or disrupted, we see clearly the path to is recovery.

\begin{description}
\item[Solidarity ]

This depends on formulating a common basis for thought and feeling. Obviously, this cannot arise simply from a common genetic heritage or race, as History demonstrates time and again. Rather, this depends on a common moral and spiritual outlook. This seems impossible today when every man is his own pope, but we now have the lineaments of this task based on the works of the Traditional authors. 

\item[Continuity ]

Like the Old Pagans of the Ancient City, who venerated their founders and continued their ways, we need to do the same. We have an example of this in the Medieval idea of the Nine Worthies\footnote{See Section \ref{sec:NineWorthies} in this book.}. However, those opposed to the Old Order work overtime to portray the past in the most negative light, or else they ignore it totally. Yet what is more insidious is the neo-right, the self-proclaimed defenders of that order who portray the past in even more vulgar terms than their opponents. Furthermore, rather than using History as the raw material to lead to the understanding of Tradition and social laws, they hold a tendentious and superficial view of that history. 

\end{description}
For further reference, please consult these works of Auguste Comte:

\textit{General View of Positivism}

\textit{Appeal to Conservatives}



\flrightit{Posted on 2011-07-15 by Cologero }
