\section{The New Intellectual}

\begin{quotex}
\emph{Noli foras ire, in te ipsum redi: in interiore hominis habitat veritas.}

Don't go outside yourself, turn back into yourself; the truth resides in the interior of man.

\flright{\textsc{St. Augustine}}

\end{quotex}
The following passage is taken from \textbf{Ultima Thule} by Arthur Branwen\footnote{\url{https://gornahoor.net/library/UltimaThule.pdf}}, where he is describing the beginning of Evola's interest in prehistory. [My translation.] 

\begin{quotex}
The central idea of this Evolian study, that along with other of his writing of that time shows the guiding influences of a new cultural perspective different from this earlier interests, was that in Germany of those years, there was emerging a type of culture intended to promote a spiritual and political rebirth that takes as its starting point the study of the most archaic roots of the German, and European, people. A culture that does not stop at pure erudition and tends instead to create a forma mentis, a reflection of a new state of mind tending to abandon the Spenglerian \emph{Decline of the West}, not assimilable in any way to the various sciences of antiquity studied in an ``antiseptic" manner in the universities which, on the other hand, in themselves do not demand any interior change and do not change in any way the relationship between man and the world. 

\end{quotex}
In other words, one cannot hope to understand these ancient myths without oneself becoming spiritually like the people who created those myths. One has to know the way they thought and felt, that is, their forma mentis. Man cannot be studied from the outside as though he is nothing but a chemical reaction.



\flrightit{Posted on 2010-06-30 by Aeneas }

\begin{center}* * *\end{center}

\begin{footnotesize}\begin{sffamily}



\texttt{Robman on 2010-07-01 at 10:21 said: }

This nordicist stuff just seems like the flip side of afro-centrism.


\hfill

\texttt{Cologero on 2010-07-01 at 12:15 said: }

I have no idea what afro-centrism is, nor what its ``flip side" is. Doesn't sound like much of an argument.

Bear in mind, we are not here defending any particular position. Just describing the theories and motivations for them from a particular era.

Nevertheless, we do accept the idea of a Primordial Tradition, so it is a legitimate question to discern its nature and origins. Wirth and Evola used myths, sagas, legends to try to answer that question. So, it is really the methodology we are dealing with, and its suitability for the subject matter — not a particular conclusion.


\hfill

\texttt{Robman on 2010-07-01 at 12:46 said: }

``Flip side" means reversal of, and afro-centrism is an movement by black ``scholars" in America who claim that blacks were the font of all culture, religion, and civilization. This ``primordial tradition" traditionalists like Guenon and Evola are fond of seems to be based on the argument there is a single metaphysic, and somehow they postulate that it must also have originated terrestrially in one place and then been dispersed through migrations. I don't believe the preponderance of historical and archaeological evidence bears this out, but of course then the counter-argument would be that that's because I'm a positivist rationalist who is limited in his scope. Yet I don't think the tenuous symbolism and legends the traditionalists use to buttress their argument are persuasive or point to any metaphysically super sophisticated nordic culture at the north pole either.


\hfill

\texttt{Matt on 2010-07-01 at 15:36 said: }

Robman,

Keep in mind that a significant amount of what Guenon and Evola talk about in their writings on the primordial tradition is meta-history. To them, the metaphysics that you speak of did not have a terrestrial origin. 

And although I'm not sure about Guenon, but Evola never believed all the human races and civilizations have a single origin (though if a lot of that is incorrect due to mistranslation of Italian or lack thereof, Cologero should not feel wrong to make the correction if I have made a mistake). He postulated that the indo-europeans and a few other races have a common origin (whether of a physical or non-physical nature), but they don't share a common origin with what he referred to as the southern races. From what it appears to me, Afro-centrists believe all races and civilizations have the same origin in the African race.


\hfill

\texttt{Robman on 2010-07-01 at 16:51 said: }

Matt, 

It seems to me that while they argue that the metaphysic is of a transcendent super-human order, if was nordic polar inhabitants who first gleaned it and then disseminated it south during their migrations. Otherwise, why else would Evola have been influenced by Wirth and decided that the primordial tradition was ultimately Nordic as opposed to mediterranean? Their view seems to have something of the hermetic ``as above, so below", insofar as terrestrial history should reflect meta-history.

But even if it is as you say it is, only a meta-history, then what is it we have here? Just another creation myth of a religious character?

As to human origins, they never seem to have clarified this point that I'm aware of. I know they reject evolution and hold that everything has devolved from a primordial golden age, including man himself, but what does this mean? How did man come to be then? Did he materialize into human biological form from spirit as theosophists claim?


\hfill

\texttt{Cologero on 2010-07-01 at 18:53 said: }

If you deny a single metaphysic, then you deny the possibility of knowing. The ``flip side" of positivist rationalism is that there is no possibility of knowing the truth; just a never-ending series of scientific postulates always in danger of being overthrown. As to the question of origins, science has shown itself to be noticeably inept, in particular, of explaining the origin of the universe, life, and thought. Yet we experience the world around us, our life, and our thinking. You can leave the question there, in the vain expectation that a future scientist will explain it all someday. Or you could be, like a very small group of men, driven to discover a more incisive methodology that can lead to an understanding of the answers to such questions.

So it is not unfair to claim that the positivist rationalist is limited in his scope, since he himself denies the possibility of transcending that limit, whereas the metaphysician does not. It's your life and your choice (assuming you accept the possibility of ``choice"). Live it the way you prefer.


\hfill

\texttt{Cologero on 2010-07-01 at 18:56 said: }

Yes, leaving out the details, that is more of less Evola's position. There is a book published about 100 years ago by an educated Hindu titled, ``The Arctic Home of the Vedas", which brought much evidence to the same thesis. It has not been disproven, as much as neglected, since such a view is now regarded as ``politically incorrect".


\hfill

\texttt{Cologero on 2010-07-01 at 19:06 said: }

As the article pointed out, the questions you ask cannot be answered in a satisfactory way to a positivist rationalist; it requires a change of consciousness. As a starting point, ask yourself the following questions:

1) How did you come into existence? Not just as a body, but as ``Robman" with a sense of identify and will.

2) When did you first become aware of yourself as an ``I", a separate being from your social milieu.

3) How do you accomplish anything? For example, dinner tonight. Did you plan it, think about it, envision it? Then did you act on it, perhaps by shopping, cooking, or eating out? How exactly does an idea in your mind — or spirit, English hides the equivalence of the two words — materialize itself into physical form?


\hfill

\texttt{Will on 2010-07-01 at 19:12 said: }

Part of the importance of myth is its function as `neuro-linguistic programming.' As Cologero wrote, ``one cannot hope to understand these ancient myths without oneself becoming spiritually like the people who created those myths."

To some extent, whether or not a myth is literally or historically true is of secondary importance. Or rather, the literal is only one level of meaning in myth, and a rather superficial one at that. When we are young, we believe the stories about heroes and gods to be literally true. Then we get older and more skeptical, and we doubt that these things could have actually happened. But if we persist in engaging myth, deeper levels of meaning begin to open up, such as the principles or ideals that myths can express.

So I'm not particularly concerned with whether or not human beings came from the North, or from Africa. Each is a myth that serves a particular function for particular people. I think Robman's observation about Afrocentrism and Nordicism has some truth to it. But I don't think this poses a problem. I sympathize with Afrocentrism to the degree that it can create ennobling myths for people of African descent. All peoples benefit from such myths – it gives you something to live up to.

These days, we're all supposed to believe that we are the chance mutations of monkeys. That's not very ennobling. Indeed, it tends to give people license to give in to their lesser natures – after all, we're just mammals with `selfish genes.' Maybe in the future human beings will figure out a way to make this mythos work for us. But as for now, we seem to be stuck in the ``world where God is dead" that Nietzsche and Evola wrote about.


\hfill

\texttt{Cologero on 2010-07-01 at 22:08 said: }

You cannot compare the \emph{use }of myth to come to a spiritual understanding with the artificial creation of a myth, or better said, ``noble lie"; they are of a totally different and incomparable order.

The fundamental reason, as we have endeavored to make crystal clear, is that this spiritual understanding is available only to the few; in particular, only for those few willing to undergo the spiritual transformation necessary for such an understanding. This is quite different from ``noble lies" intended to influence the masses.


\hfill

\texttt{Will on 2010-07-01 at 22:56 said: }

There is a story from the Buddhist tradition in which a man is traveling to India. His mother, a very devout Buddhist, learns of this and makes her son promise to bring her back some sort of relic or holy object from the land of the Buddha's birth. The son promises, but in the course of his travels, forgets to get anything. So on the way back, he sees the corpse of a dog laying in the street, and he pulls a tooth from it. He tells his mother that he has brought her one of the teeth of the Buddha. She believes him and venerates the tooth as a sacred object. Through her faith and devotion, she generates real merit and ultimately becomes enlightened.

In this story, the distinction between a noble lie and an authentic myth is blurred. It suggests that what one brings to one's spiritual practice – one's general disposition and intention – is, if not the most important factor, then at least a very important one.

This is not to say that we cannot distinguish between authentic, traditional myths and modern counterfeits (like that book The Shack, for example). But I think we have to keep in mind that what works for one person will not necessarily work for another.

This relates to Burckhardt's exposition of the relation between knowledge and will. If one is fortunate enough to encounter good doctrine, including good myths, then presumably this can expedite one's spiritual growth. But if one is not so fortunate, and is born into some strange cult or without access to good teachers, one can still progress if one's intention is pure.

I have to take issue with your statement that spiritual understanding is available only to a few. While I certainly agree that only a few achieve profound realization, all traditional doctrines state that what is realized is in fact common to all. Even Evola acknowledged this, and they don't come much more anti-egalitarian than him.


\hfill

\texttt{Robman on 2010-07-02 at 01:33 said: }

Having to choose between a potentially spiritually transforming ``noble lie" and the existential quandary rationalism leaves one in is an unsettling one to have to make, but personally, it would probably be my nature to choose the latter. I'm asking myself what would be the purpose of traditional metaphysics? For one to reintegrate himself with metaphysical reality? If so, how does a ``noble lie" of a nordic polar realm, if it were only a fiction, help achieve this transformation?


\hfill

\texttt{Will on 2010-07-02 at 10:28 said: }

Robman, I appreciate your question and your perspective. In my opinion, rationalism and ``noble lies" relate to what the Buddhist tradition calls nihilism and eternalism – the two extremes. Nihilism says that nothing means anything and we can't know anything, and eternalism vests everything with some kind of grand meaning. The right view is said to be neither of these.

Corresponding to these are extreme skepticism on the one hand, and extreme naivete or gullibility on the other. If we veer too far in either direction, our view becomes skewed. I've met plenty of new agers who are willing to believe anything that sounds vaguely `spiritual,' so long as it makes them feel good in the moment. That would be naivete. I've also met skeptical, `scientific' people who refuse to consider anything at all beyond a reductionist, materialist worldview.

The notion of a Nordic polar realm is related to the myth of the Golden Age. The Tibetans believe that the Kingdom of Shambhala exists either on this earth in a `hidden land' or in another dimension. Christians believe that the Garden of Eden still exists, but entrance is barred. I think these myths convey the notion that the divine is real and knowable and realizable, but I'm not entirely sure how important it is to `believe' in such things, particularly if one is more skeptically inclined. Perhaps other writers will have more to say on that. There are styles of meditation and practice that work with skepticism and the analytical qualities of the mind, such as the Madyamaka tradition in Buddhism. (That's the tradition I've studied the most, so sorry if my examples keep coming from there.) I think this relates to the earlier distinction between different types of yoga: jnana yoga (knowledge), bhakti yoga (devotion), etc. People have different inclinations and talents.


\hfill

\texttt{Cologero on 2010-07-02 at 12:51 said: }

Evola writes in Pagan Imperialism\footnote{\url{http://www.gornahoor.net/PaganImperialism/ThoseWhoKnow.htm}}:

\begin{quotex}
We, on the contrary—basing ourselves on a tradition much more ancient and real than the one which can be claimed by the ``faith" of Western man, on a tradition which is not proved by doctrines, but by deeds and works of power and clairvoyance—affirm instead the possibility and the concrete reality of what we have called ``Wisdom". We thus assert the possibility of a positive, direct, methodical, empirical knowledge in the ``metaphysical" field, just as the one science strives to gain in the physical field, and, just like science, it remains above any moral or philosophical belief of men. 

\end{quotex}
So you need not surrender your hard-headed rationalism, nor believe in any doctrines, myths, or ``noble lies". However, you do need the tools to investigate the metaphysical field, through direct and empirical experience.


\hfill

\texttt{Cologero on 2010-07-02 at 12:54 said: }

Thanks for posting this. We are definitely not talking about myths, allegories, or ``noble lies". The answer to the question of origins involves a knowing, not a believing.


\hfill

\texttt{Will on 2010-07-02 at 14:12 said: }

Thanks, everyone. I'm grateful to be able to discuss this with intelligent and civil people.

Gottling, that quote from Charles Upton is great, and also gives me some insight into Plato's doctrine of the Forms or Ideas, which I'm still trying to wrap my head around. Upton has a book on romantic love, which I haven't read yet, but I suspect it would inform the earlier discussion on chivalry and Tantra.

And the quote from Evola brings us back to the fact that what we are talking about is the potential to effect qualitative changes in our state of being. I'm curious to know what people think or have experienced about how this works. Does it relate to the feeling of inspiration and a kind of natural connection one has with certain ideas or myths? Since this discussion began with a question about Hyperborea, how might that myth or idea work in relation to gnosis?

Also, as an aside to Cologero, do you plan on finishing the translation of Pagan Imperialism?


\hfill

\texttt{Robman on 2010-07-02 at 14:50 said: }

Will, I don't mind the Buddhist references, since I too have somewhat studied it in the past decade and meditate regularly. My degree of ``enlightenment" is quite another matter, though. Buddhism as it is interpreted today is also another story altogether. I think it was Coomaraswamy who stated that Buddhism today is known for all the things it originally wasn't. Evola's book ``The Doctrine of Awakening" delves into pre-sectarian original Buddhism along those lines.

As far as the Nordic polar mythos, I'm not sure how it can be spiritually useful, except to a select few. This mythos, to my understanding anyway, seems to be directed at raising the spiritual esteem of those of Nordic or ``Aryan" racial type. I think Evola's idea was that spiritual race ultimately manifests in biological race, and that the Aryan type, as the highest caste, was some kind of spiritual elect. Though this idea is flattering if you're white, doesn't it ultimately fetishize race? Doesn't the Polar mythos foster the notion that metaphysical insight is the preserve of a select few? In this, it would become something akin to Judaism. And doesn't this ultimately lead to a biological attachment which hinders detachment and spiritual advancement?


\hfill

\texttt{Robman on 2010-07-02 at 15:03 said: }

Gottling, that was a difficult quote you provided and I'm not sure I understood it fully. However, I have read Upton's Atlantis and Hyperborea essay, and my reading of it was that Upton doesn't at all take it literally but merely symbolically.

Cologero, I read the words, and yet I have no idea what they really mean. Whether people speak of wisdom, metaphysical insight, experience of the noumen, gnosis, enlightenment, satori, mystical union, etc., to me it ultimately remains an abstraction. I haven't the foggiest what this experience entails, what it's aims are, or what imaginary polar nordic realms have to do with it. Even meditation for me is nothing more than a few stolen moments of psychological peace- far from any direct experience of niravana or the void, whatever that even means. Maybe I am just a profane modern westerner, but that's where it stands.


\hfill

\texttt{Matt on 2010-07-02 at 18:10 said: }

Robman,

Evola's quote is speaking about the way knowledge of the metaphysical/transcendent is obtained. Evola explains that its not obtained from mere faith that modern western Christianity preaches, nor can it be obtained from the rational discursive thinking. It can only be obtained by a direct experience. Evola has described it as a type of positivism (or spiritual science), a higher positivism, similar in a way to the lower positivism of the physical sciences that obtain knowledge (though be it of a relative nature) of the physical dimension from a direct empirical way. And since Evola sees the way to knowledge of the metaphysical as a science, it is therefore beyond typical moral trappings. This is the method of the long past western initiatic tradition, and to a lesser extent seen in the Eastern Orthodox doctrine of theoria that teaches a direct empirical experience of God's uncreated energies, which is the only concrete way of knowing the divine (unlike Catholicism's idea that only faith and/or rational discursive thinking are the ways to know the divine, or the protestant idea that the divine can only be known from naive faith). And for Evola, only the few have the ability to successfully follow this way, and the modern age makes it that much harder for western man since the initiatic tradition has been almost entirely lost in the west.

And Will, I share the same sentiments. Its nice to talk about these ideas in a civil manner, which is rarely seen today, especially on the internet. It reminds of one of Cologero's earlier posts on the respect shared between certain thinkers of the western tradition, who had a respect for each other even if they disagreed notably in some areas. Unlike most men, they did not view ideas as weapons to be used against each other.


\hfill

\texttt{Will on 2010-07-02 at 20:57 said: }

Robman, I agree with your comment about the fetishization of race. I am not happy about the co-optation of Traditionalism by people whose main agenda is identitarian politics. According to traditional doctrine, the only identity that ultimately matters is the Atman, or tathagatagharba, or God. This is not to say that racial and cultural traditions don't matter at all, but if we are going to discuss them in light of Tradition, I think we should keep our priorities in line. Otherwise, the kind of biologic and egoic attachment you mention comes into play. Fortunately, the `negative theology' of the Buddhadharma, which ruthlessly analyzes all phenomena as impermanent and not-the-Self, is a good antidote to such thinking. NA ME SO ATTA!

That being said, I do find the Polar mythos interesting, and I certainly don't care that it isn't politically correct. Nowadays, there is a tendency to view these types of studies through the lens of Nazism, as if that was somehow the only possible outcome of such a line of thought. But to dismiss a whole generation's scholarly and philosophical work because of some notion of guilt by association is really throwing out the baby with the bathwater. It's wrong to attach a stigma to Germanic myth and ancient history just because the Nazis liked it. 

Evola has a great line in Men Among the Ruins: ``The Idea, only the Idea must be our true homeland. It is not being born in the same country, speaking the same language or belonging to the same racial stock that matters; rather, sharing the same Idea must be the factor that unites us and differentiates us from everybody else." Personally, this is the view I subscribe to. I am most impressed by people who display admirable qualities such as wisdom, skill, strength, self-control, modesty, determination, perseverance, and other such virtues.


\hfill

\texttt{Will on 2010-07-02 at 21:11 said: }

Matt, I like your exposition of the Evola quote. The notion of ``higher positivism" seems to me an apt description of what that first and greatest generation of Traditionalist writers were doing when they ingeniously laid out the objective metaphysical unity behind the seemingly disparate religious traditions of the world.

In regards to Evola's views on methods – how to go about actually striving for this direct knowledge that can manifest ``deeds and works of power and clairvoyance" – he seems to have focused most on that in Introduction to Magic. And as you said, magic for Evola is a science – a `science of the I."


\hfill

\texttt{Robman on 2010-07-02 at 21:57 said: }

I don't care about the political incorrectness of the polar myth either, and that's not why I question it. I'm just curious as to its veracity and the coherence of traditionalist claims- whether it is historically true, or if it's only archetypal symbolism.

As to Evola, I get mixed signals from reading the man. On the one hand, he emphasizes the idea as you note above, yet at the same time seemed to stress Aryanism as the highest of karmic human destinies within manifestation, not just as a caste and spiritual type, but as a corresponding physical type as well. In one of his books, which I don't believe has been translated into English yet, he delves into physical anthropology and provides numerous pictures of European types with his own typological classifications. Which is it?

As far as his race of the spirit and line about ``The Idea", this sounds eerily like what modern liberals are trying to construct in modern ``proposition nation" America. The results, in my opinion, are negative. So again, Evola often confuses me.


\hfill

\texttt{Robman on 2010-07-02 at 22:19 said: }

Since we've been discussing Herman Wirth and Hyperborea, here is an article you might or might not find somewhat interesting. 

\url{http://heatherpringle.wordpress.com/2010/02/23/herman-wirth-and-the-origins-of-writing/}

It seems this work might back up some of Wirth's ideas. But what is especially striking to me is where it states:

``Indeed, she proposes that the ancient sign language may have originated in Africa and arrived in Europe with modern humans–a proposal that would have horrified Wirth."

I can't imagine that Chinese, or Africans, or Central Americans, or anybody else, for that matter, would be so generous and anxious to ascribe cultural developments found on their own soil to Europeans, for example. Yet western scholars often seem so frivolous with the western legacy in letting others take credit for it. The vast bulk of modern westerners do indeed seem to fetishize race, but other people's races rather than their own. What strange people we are.


\hfill

\texttt{Will on 2010-07-02 at 23:23 said: }

The observation about the similarity to the `proposition nation' concept is good. Realistically, I don't think you can make a nation that way; certainly not one as large as the United States. But a tribe or an organization or small community, perhaps.

Also keep in mind Evola's notion of the two ways of being outside the caste system: as a transcender, or as a pariah. In his call for an elite based on spiritual vision and principle, he is calling for transcendence of the caste system, whereas the liberal notion of the `proposition nation' seems to be more about the lowest common denominator.

I can't comment on the Evola text that you mention as I haven't read it. I can only say that I don't regard him as an infallible source, although obviously I respect his work a great deal. In his autobiography, he describes his work on racism as an attempt to introduce superior, spiritual principles into the discussion, which was widespread at the time, and which was overrun with the biological reductionism that he considered an error. I think Evola saw the Aryan idea as a suitable Ideal for the West, but in his writings it's clear that he respected the accomplishments of many cultures – not in a modern, PC relativistic way, but based on merit and objective evaluation. I'd like to see more of that. In modern politics, it seems like you have to choose between being a relativist or a chauvinist.


\hfill

\texttt{Robman on 2010-07-03 at 12:02 said: }

Will, here is the text I was referring to, but it's in German, and though my German is limited, I was able to glean the gist of some of the captions below the pictures provided. But it does look like Evola was delving into physical typology here:

\url{http://www.velesova-sloboda.org/misc/evola-grundrisse-der-faschistischen-rassenlehre.html}

For example, if you look at plate no. 8 of Buddha, Evola writes, ``classical Nordic Aryan features are visible".


\hfill

\texttt{Will on 2010-07-03 at 23:06 said: }

Robman, thanks for the link. I don't read German, so I'm afraid I'm still at a loss here. I did look at the pictures, but without being able to read the commentary, I don't know in what context they are being displayed.

A couple thoughts come to mind on this.

In regards to the picture of the Buddhist statue (it doesn't look like Saykamuni – maybe Padmasambhava?) the tradition holds that there are certain `marks' of a Buddha; among these, many physical characteristics. I don't have a list of them offhand, but these characteristics provide guidelines for artists in the tradition when they are depicting Buddhas in their artwork. The long earlobes are one such mark.

Within the Buddhist tradition, the three aspects of being – dharmakaya, sambogakaya, and nirmanakaya; or essence, energy, and form – are considered inseparable, and so it follows that the outer form would have some relation to the spiritual essence of a person.

So Evola may have held, in accordance with traditional doctrine, that certain physical characteristics correspond to certain psychological or spiritual traits (Yukio Mishima has some similar ideas in his book Sun and Steel.) Or he may have been adapting his ideas to the already existent German discourse on race, in his attempt to influence the course of events there. I don't know. Interesting that the photos are all of faces, not whole bodies. Mishima wrote instead of bodily traits and their correspondences. Can one be obese and still be `Aryan?' (According to the Tibetans, yes: Dolpopa, the founder of the Shentong school, was so fat he had to be carried around on a door. He was also the most enlightened teacher of his time.)

I find it ironic that people today condemn Evola for his views on race, when the fact of the matter is that it took more guts for him to propound these views directly to Fascist and National Socialist authorities – views that were very `politically incorrect' according to the standards of the time – than it does to condemn him today. Indeed, anti-Semites at the time condemned Evola because they noted that, according to his doctrine, a Jew could have the soul of an Aryan, and vice versa, which makes it very difficult to persecute Jews!

As Socrates discovered, the rabble hates a man who makes them think.


\end{sffamily}\end{footnotesize}
