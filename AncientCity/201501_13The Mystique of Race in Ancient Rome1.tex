\section{The Mystique of Race in Ancient Rome I}

\begin{quotex}
This essay by \textbf{Julius Evola}, which will appear in two parts, was originally published in the journal La Difesa della Razza.

By ``race", Evola means both something less and something more than is meant by the word today. ``Less" in the sense that a race refers to any group of people of a common stock or lineage. ``More" in the sense that he includes the transcendent, spiritual, and mystical aspect of race, not just biological or physiological characteristics. In this essay, he describes the mystical element of race in ancient Rome. A lineage was founded by a spiritual father, not necessarily the common biological ancestor; this father determined the cult, laws, and customs of his lineage.

Note on translation: I have left the words lares, manes, penates, genius (NOT a clever person), gens, and gente untranslated, following the model of \emph{The Ancient City} by \textbf{Fustel de Coulanges}. The interested reader should consult that work for the proper definitions.

Go to Part II $\Rightarrow $ 

\end{quotex}
The literature on racial theory has not failed to emphasize everything that shows the importance attributed to lineage, people, origin, and ancestry in ancient Romanity at that time, and has also conducted research to recover the Aryan or Nordic-Aryan element and type in Romanity and to follow its destiny. Because of the predominant interests in modern racial theory and in the very nature of its development, this research is therefore almost always focused on the basically exterior and subordinate elements: thus it remains on the level of ancient law and custom, on certain aristocratic traditions, on the direct or indirect evidence in respect to a give physical type and, somewhat less often, is conveyed in the field of the most noted and widespread certain cults and myths. It is curious that, as far as we know, it is instead almost systematically neglected a series of sources that, in regard to the higher aspects of the doctrine of race, present a special meaning and are richly documented. The reason for that is in the predominance of the prejudice—which we previously reported in this journal—to consider the whole of what in Roman antiquity had a super-rational and properly traditional character as fantasies, imaginations, superstitions, and finally, as something unserious and negligible. In this way a great part of the ancient Roman world still waits to be explored and this exploration, if conducted possessing the right principles and suitable qualification, is destined to yield valuable results, not just in regards to a spiritual and religious consciousness of the forces of the race.

The lares, penates, manes, genii familiari, the archeget heroes and so on are notions well known to anyone who has made even elementary studies of ancient Roman history. But known to what degree? Also, like the equivalents of dead and mute things that are conserved in museums, like the verbal residues of a world that is felt as foreign and ``dead", as much to leave us indifferent, at least, for whatever technical and academic reasons, they are not compelled to make special studies of sources and traditions, in place of mere culture, resulting in a worthy monograph. To integrate such signs, including pulling sufficient elements from them to make us understand the meaning and fundamental truths of ancient Roman and, in general, Ario-Mediter\-ranean, humanity is a task that, with very rare exceptions, is not at all felt. However, even by this we understand the most precise and significant racial profession of the faith of ancient Rome, not a ``philosophized" profession of faith restricted to any cultured circle, but alive and active in the most original, most widespread, most revered traditions.

The notions of lares, penates, genies, heroes, etc., are in good measure interdependent. In various ways, they all refer to the ancient Roman awareness of the mystical forces of blood and race, to the lineage, considered not only in its corporeal and biological aspects, but also in its ``metaphysical" and invisible, but not ``transcendent", aspects in the limited dualist meaning that has come to prevail for such terms. The single, atomic, deracinated individual does not exist. When he presumes to be a being in itself, he is deceived in the most pathetic way, because he cannot even name the last of the organic processes that condition his life and finite consciousness. The individual is part of a group, a folk, a gente. He is part of an organic unity, whose most immediate vehicle is blood, and is extended both in space and time. This unity is not ``naturalistic", is it not determined and called to life solely through natural, biological, and physiological processes. Such processes just constitute his exterior side, the necessary but not sufficient condition. There is a ``life" of life, a mystical force of blood and folk. It subsists beyond the forces of the life of the individuals that are dissolved in it at death or that are given by it through new birth: it is therefore a \emph{vitae mortisque locus} [a place of life and death]—a place that encompasses life and death and that for that very reason stands beyond both.

To maintain a living, continuous, and deep contact with this profound force of the race is the most direct and essential form of \emph{pietas}, religiosity, the basis and condition of every other, the principle canons of family laws are its consequences and applications, even in relation to the earth, that it itself—as the notion of the \emph{genius loci} shows—maintains mysterious and ``mystical" relations with the blood and the original strength of the people or \emph{gens} that possesses it and lives there. Looking toward the origins, there is the sense of a ``mystery"—there is the myth both of beings having come from above, and of men who transcended humanity, to loosen their life from their person and to thus constitute it as the superindividual force of a folk, of a lineage, of an ancestry that will see its origin in it. Ideally, there is a contact and a perfect match of the individual with this power, to be able to signify through it the apotheosis, i.e., the conquest of the privilege of immortality, and to confer on it the right to be considered even a ``son"—in a higher sense—of the being of the lineage, if even a type of new manifestation of this being itself.

This is the essence of the mystical-racial creed of ancient Ario-Mediterranean and, particularly, Roman, humanity. The significance that it gives to the race as spirit, beyond that of the body, is an irrefutable fact and constitutes the base of the belief of the entities indicated and of the meticulous worship that was dedicated to them. We will put forward some evidence that will also be valid to highlight further aspects of the central ideas we succinctly exposed.

According to a noted work of Macrobius (Sat., III, 3) the lares for the Roman were ``the gods that make us live: they nourish our body and govern our soul". Naturally that must not be understood in an ingenuously literal way, but in reference to the mystery of the ultimate forces of our organism. As we pointed out, not one of the most important processes that are at the base of our organic and psychic-physical life depends directly on our power and is illuminated by our consciousness. Ancient man, while he was uninterested in the exterior, physical work of such processes, which are studied by modern positive sciences, instead focused all his attention on the forces that were presupposed by them and that precisely—in a higher and symbolic sense—``nourished" and ``governed" our life. Macrobius' testimony, among many others, is the most explicit in indicating that the ancient cults of lares, manes or penates were indeed related, above all, to such forces.

These moreover were brought back to a single origin in close relation with the idea of race.

\begin{quotex}
The most ancient documents of the cult of the lares give us mainly their divinity to the individual and embodies it in the \emph{lar familiaris} [the family spirit], the sole, but ideal, father, of a given race; this word, in reality, means not that he created materially the race at its origin as the forefather, but that he is the divine cause of its existence and duration. (Saglio, Dict. Des Antiquités grècques and romaines, III.) 

\end{quotex}
The \emph{lar familiaris} was also called \emph{familiae pater}, father or root of the family or of the \emph{gens}, under this aspect identified with the \emph{genius generis}, the genius [spirit] of a given lineage. Now the word \emph{genius} was still meant more distinctly as the hidden and ``divine" force that generates—\emph{genius nominator qui me genuit}—the creator of a given race is \emph{generis nostri parens}, the word \emph{genius} already in itself is related to the words \emph{geno}, \emph{gigno}, i.e., to the idea of generating, that lies at the base of the same word \emph{gens}, \emph{gente} [folk]: here it is still a question for the real power that acts beyond physical generation, in the union of the sexes (\emph{a gignendo genius appellatur}, Consorino, \emph{de die nat}. 3), through which the nuptial bed has also the name of \emph{lectus genialis} (bed of the folk) and every offense to the sacredness of aristocratic marriage and to the lineage was considered as a crime above all in the face of the \emph{genius} of the lineage.

The ancient writers relate \emph{genius} not only to the \emph{geno}, \emph{genere} (to generate), but also to the word \emph{gero}, so that, by being etymologically inexact it is not less significant in relations of the idea that they had of the entity in word. This reconciliation in fact brings to light the conviction that the force constituting the mystical origin of a given lineage and the matrix of every generation, remains as a ``presence" in the group corresponding and by way of principle governs, directs, and sustains the life of the individuals (Hartung, \emph{Die Religion der Romer}, I). Our language still has the word ``\emph{geniale}" [brilliant, inspired], but just to designate a rather different thing, also opposed to the most ancient conception. The ``inspired" individual, as commonly meant, is more or less the one who invents, who has some ``bright ideas", on the rebellious, disordered, individualistic basis. In the ancient conception, geniality could be conceived only as a special inspiration or inspiration that the individual enjoyed not in that way, but essentially in relation to his race and blood, to the \emph{genius}, to the divine element of his \emph{gens} and the tradition of the \emph{gens}.



\flrightit{Posted on 2015-01-13 by Aeneas }

\begin{center}* * *\end{center}

\begin{footnotesize}\begin{sffamily}



\texttt{scardanelli on 2015-01-14 at 10:24 said: }

Sedir on race, etc in ``On Dreams":

Around this flame, the immense organisms of the human spirit circulate as an army of asteroids around its sun. Here, let it be understood that each of our bodies, each of our fluids, our magnetic

properties,-our sentiments, our mental faculties our powers of action are just so many individual organisms, classified in hierarchical autonomy.

Each of these parts of our selves is free, yet is drawn into the evolutive line of the total self. In turn, this self is free, yet it is physically drawn by the planet, socially by one's race, and spiritually by one's religion. Thus, our ponderable body seems to be the media by means of which the terrestrial forms raise themselves to the invisible, and how the objects of the immaterial worlds lower themselves to the visible world.


\hfill

\texttt{rhondda9 on 2015-01-15 at 12:48 said: }

I want to thank you for the tip to read ``The Ancient City". Interestingly, I have been reading Prolegomena to the study of Greek Religion by Jane Harrison. It was first published in 1903 and again in 1955. When I looked up The Ancient City and perused the table of contents, I was struck by the themes similar to Harrison's book.Thus, further investigation is required.


\hfill

\texttt{Raúl on 2016-10-17 at 10:06 said: }

The reason behind the ritual of the Manes is, in hindu words, that the pretas become pitris, ie. that the soul impresión on the corpses enter in the generation cycle for others being. In this way not only the ancestors `born again'. but the living are in peace.


\hfill

\texttt{Raúl on 2016-10-17 at 10:11 said: }

If the ancestors are of people who is `born again'. the initiatics qualifications pass to the family, and in the other case, shudra and dalits can do the inferior possibilities that belong to they and the society is in order.


\hfill

\texttt{Raúl on 2016-10-17 at 10:54 said: }

I used the expression reborn in two ways, one as transmission of hereditary characteristics, which comes from the dead, and another as the state of the patricians having been initiated.

In my first comment, the first, and in my second comment, the second.


\hfill


\end{sffamily}\end{footnotesize}
