\section{Family Values}

\label{sec:FamilyValues}

There is a rather oddball notion floating around that the traditional family arrangement is actually unnatural to Europeans and reflects—and this is intended as an insult—``Jewish family values". This should be of no interest to us, except that it is an idea promoted by self-defined ``Traditionalists" and for rather ignoble purposes. Unfortunately, in understanding this matter in its depths, we cannot count on Julius Evola who, while disparaging so-called bourgeois values, actually proscribed important aspects of the Roman tradition. Also, as this is a matter of principle, we cannot appeal to contingent factors such as the ``disorder of the world" or ``riding the tiger" to justify a truly un-Roman attitude.

\paragraph{Preliminaries}
We can immediately and unequivocally dismiss the notion on a strictly factual basis: ``Jewish family values", under the Mosaic Law, included polygamy and easy divorce. On the contrary, Roman family values required monogamy and divorce was forbidden or rare.

To demonstrate this both historically and principally, we will make use of two works, one on Ancient Rome, and the other on the Second Rome, describing what well-bred men considered mentally healthy and normal; these are respectively \textit{The Ancient City} by \textbf{Numa Denis Fustel de Coulanges} and \textit{The True \& Only Wealth of Nations: Essays on Family, Economy \& Society} by \textbf{Louis de Bonald}.

I want to clarify the method once again, since it may not yet be clear to everyone despite our many posts on the topic. First of all, after gender, the next differentiation in man is caste\footnote{\url{https://gornahoor.net/?p=1698}}, since it refers to fundamental orientation of a man towards the conditions of his being. This is prior to subsequent differentiations such as race, nationality, or religious affiliation. In our day, a man's caste is not immediately visible to the senses, nor even necessarily understood by the man himself, so they play very little role in social analysis today, except perhaps in its degenerate form of economic class. Yet, it is a fundamental notion, certainly more important than racial or religious characteristics, despite the false importance attributed to these latter distinctions by most ``defenders" of Tradition.

The second point is that we regard the Roman Tradition as a spiritual unity. \textbf{Rene Guenon} often remarked that the Christian Tradition was unusual in that it did not contain a law in its scriptures. Instead, the Medieval spiritual tradition adopted the old Roman Law; that is why the Mosaic Law was never part of Christianity. Of course, there is a clear and obvious reason for this. The law of the more ancient traditions, including those of the Ancient Cities, comprised detailed requirements for the performance of rites, the conditions of ritual uncleanliness, various taboos, and so on. To an outsider, including men today who think they can return to such pagan traditions, these laws would seem arbitrary, even unreasonable. In fact, that is the case; although in general the principles are true, in specifics, the exact relationship is difficult to discern.

As Guenon pointed out, in is origins Christianity was esoteric, hence its primary emphasis was on the interiorization of the law, not on outward observance. Morality, that is, outward observances or law, was secondary. By that I mean nothing more than that morality could be derived from metaphysical and theological principles, and therefore did not require its own revelation. I will expand on this at another time.

This is not at all to deny that the Ancient Tradition lacked an inner or esoteric dimension. The quotes that follow are taken from the two books mentioned; it is easy to see the spiritual continuity between the two.

\paragraph{The Ancient City}
We have previously indicated, in \textit{The Son of Duty}\footnote{See Section \ref{sec:SonDuty} in this book.}, the importance of family to the ancient Greeks and Romans. For a man, especially for a man claiming to be recovering the ancient Roman tradition, to deliberately fail in this task, indicates a severe fault and is not at all normative. This is what I mean by a ``dead end", it is the end of a tradition, not a revolt against the modern world which, in any case, is similarly, and radically, opposed to the Traditional patriarchal family. This is how the ancient Romans actually did consider it:

\begin{quotex}
[The Ancient City] rendered marriage obligatory; celibacy was a crime in the eyes of a religion that made the perpetuity of the family the first and most holy of duties. But the union which it prescribed could be accomplished only in the presence of the domestic divinities; it is the \emph{religious, sacred, indissoluble union} of the husband and wife. [my emphasis]

Civil marriages were not valid. Adultery was a crime. For the sacred fire should be transmitted from father to son.

The son born of adultery annihilates in this world and in the next the offerings made to the manes. 

\end{quotex}
Of course, this applies to the higher castes, as the plebeians did not have a sacred marriage. The proletarian mentality even today eschews marriage as the high illegitimacy rates attest to. Even so, there are remnants still alive of the old idea as we see in divorce statistics.

A few days ago I saw a famous actor, whom I did not recognize, on a TV interview show. The hostess asked him (an obvious set up question) when he planned to marry his attractive blonde gf named Kirsten. He said they will when their ``friends" are allowed to do so. ``I don't want to perform a ceremony as long as they can't." There you have it; for the prole mentality of an actor\footnote{\url{https://gornahoor.net/?p=4338}}, marriage is not a ``sacred union", but only an arbitrary and unnecessary ceremony. Yet, the power of the ancient ideal of marriage persists as we see in these ceremonial elements that go back thousands of years: father gives away the bride, bride wore a veil, white clothing, groom carries her across the threshold, sprinkled with water, bride and groom share a cake. 

By law in Greece and Rome, no family could become extinct, and it was the duty of the leaders to prevent this. The modern world, on the contrary, is intent on making the family itself extinct. So, you rebels against the modern world: where do you take your stand? Which path do you intend to take?

\paragraph{Germanic Monogamy}
\textbf{Tacitus} in \emph{Germania}, sections 18 and 19\footnote{\url{https://www.gutenberg.org/files/7524/7524-h/7524-h.htm}}, describes the Germanic women of his time. Tacitus notes that the matrimonial bond is quite strict among the Germans, with monogamy being the norm. The woman is not the weak link in that relationship; rather she shares fully in their common life:

\begin{quotex}
That the woman may not think herself excused from exertions of fortitude, or exempt from the casualties of war, she is admonished by the very ceremonial of her marriage, that she comes to her husband as a partner in toils and dangers; to suffer and to dare equally with him, in peace and in war: Thus she is to live; thus to die. 

\end{quotex}
Their society is based on chastity, with no ``seductive spectacles", adultery is extremely rare and immediately punished. Young couples are expected to be virgins, so marriage is lifelong. They don't try to limit children through birth control, nor through infanticide as in other cultures.

That opposition to divorce, promiscuity, adultery, and birth control was very influential on the Roman church. Even the Greeks allow divorce and have been inconsistent on contraception. Of course, the neopagans believe in none of it.

\paragraph{The Ancien Régime}
The concrete situation in the medieval and subsequent era is different from the ancient city. First of all, the specifics of the rites and duties of family life are not spelled out in detail, but must be inferred. Then, there is the separation of the spiritual authority and political power, so the latter is entrusted with enforcing the law determined by the former. Bonald distinguishes between the domestic professions, which we would call the Third Estate, or the professions of the Vaishya caste, and the public professions of the higher castes of Priests, Administrators, and Warriors. In the book referenced, Bonald discusses the ideas of marriage and divorce. Implicitly rejecting the notion of the social contract as the source of social order, he writes:

\begin{quotex}
Domestic society began with monogamy and the indissolubility of the conjugal bond. 

\end{quotex}
For Bonald, this is not just a religious, or even Catholic, notion, but is fundamental to any Traditional society. Aware of the Roman Tradition, and in agreement with Fustel, he informs us:

\begin{quotex}
For several centuries the Romans fought against divorce. It appeared among them only very late. 

\end{quotex}
Interestingly, he considers the Roman attitude as indicative of the ``highest wisdom" and understands the role of Christianity to apply this wisdom to society. In his words:

\begin{quotex}
The highest wisdom made itself heard, the Christianity, which is only the application to society of every moral truth, began by constituting the family, the necessary element of ever public society. 

\end{quotex}
Bonald also pointed to the customs of the Germanic tribes, which he admired:

\begin{quotex}
The general march of society toward civilization was no less constant and continuous, and the peoples of the north, who in the end came to renew the worn-out body of the Roman Empire, received the Christian religion from the vanquished wherever they settled in exchange for the monarchical constitution that they brought to them. 

\end{quotex}
As for their attitude toward divorce, ``[divorce] was not so among the people [Germanics] whose martial way of life was chaste and simple.

For Bonald, marriage is understood in its organic wholeness, and is farthest thing from a ``right" or to the personal whim of just the two parties involved. He explains:

\begin{quotex}
[Regarding divorce], the government will have fulfilled all its duties toward religion when it will have seen to it that the bond of marriage, formed by the mutual consent of the parties, guaranteed by the civil power, and consecrated by the religious power, cannot be dissolved by law.

Marriage is at once a domestic, a civil, and a religious act which, in the public state of society, requires for its validity the concurrence of the three domestic, civil, and religious powers: In the consent of the two parties, authorized by their parents, in the intervention of the civil power, and in the concurrence of the religious authority. Once the bond had been formed by this triple knot, and the family that it has founded has taken its place among the families that compose the State, the legislator should consider it as an integral part, inseparable from the great political whole, composed itself of families, religion, and the State. 

\end{quotex}
This is absolutely incomprehensible and absurd to the modern mind which understands society to be composed of individuals, not families, and religion to be a private, rather than public, matter. Even at the time Bonald was writing, in the generation following the French Revolution, marriage was considered a mere civil contract, requiring only the consent of the two parties. Clearly, this the dominant view today, among the chattering classes. Their ill-equipped opponents simply sound ridiculous to educated people today. That is because they sense something is wrong with the modern view, but they lack the intellectual tools to oppose it; a fortiori, their viewpoint is actually consistent with what they oppose. For example, in the USA, the dominant Protestant ideology sees no sacred or sacramental character in marriage, they oppose any public spiritual authority, and their view of society is itself individualistic and atomic, rather than organic. That they are shocked by recent developments is itself shocking.

That the civil power would have any say in marriage is rejected today. Here is Bonald's view on that notion, although his sanguinity is unjustified, given subsequent developments:

\begin{quotex}
The right of the civil authority to establish impediments to marriage will not, I am sure, be contested. Politics, sometimes more stern than religion admits some that religion has not been able to recognize. 

\end{quotex}
For example, from a strictly theological perspective, parental approval and certain levels of consanguinity may not be absolute impediments to marriage; the Church is more interested in the parties sharing the same religion. Nevertheless, the State itself can require parental permission; it can also prohibit marriages both to close relatives and even to those whose bloodlines are too different from each other. Such laws would be impossible today.

In continuity with the requirement of the Ancient City to preserve the family, Bonald provides his own understanding of it, along with a disturbing prophecy:

\begin{quotex}
Let us make bold to say it: The State has no power over the family except to affirm its bond, and not to dissolve it. And it the State destroys the family, the family in its turn will avenge itself and will silently undermine the State. 

\end{quotex}


\flrightit{Posted on 2012-07-28 by Cologero }

\begin{center}* * *\end{center}

\begin{footnotesize}\begin{sffamily}



\texttt{Andrew on 2012-07-28 at 16:29 said: }

While I agree with this in principle, realizing such a marriage is the difficulty. Is this something that would be worth expatriation and moving to a ``healthier" society?


\hfill

\texttt{Matt on 2012-07-28 at 16:49 said: }

On a related note, recent statistics also show that on average, those in the upper-classes are more religious than those in the lower classes. Not too surprising considering marriage and religion seem to be linked….though it is surprising to the current assumption of today that the lower-classes are more religious and committed to marriage than the upper-class. This then gets to the question (for Gornahoor readers its rather rhetorical) that is really interesting….since the modern assumption of marriage/religion and its relation to the proles does not align with reality, where does this assumption originate from?


\hfill

\texttt{Izak on 2012-07-28 at 16:53 said: }

Good work.

Although the guy responsible for the theory you're refuting makes a few good points here and there, this essay needed to be written nevertheless, if not only to help establish the basic universal quality of human heterosexual partner-bonding. My guess is that the ``mannerbund theory" (or whatever) will blow over fairly quickly.


\hfill

\texttt{Cologero on 2012-07-28 at 17:06 said: }

``While I agree with this in principle, realizing such a marriage is the difficulty."

I'm afraid I have no advice to you, Andrew, as all we can talk about are principles. The required social structures no longer exist. Nevertheless, action follows thought.


\hfill

\texttt{Matt on 2012-07-28 at 17:18 said: }

Izak,

Yes, the whole ``Mannerbund theory" is an interesting case.

I'm reminded of what Evola said in his book on eros. In the chapter on homosexuality, Evola stated that pederasty may have had a noble origin as a purely platonic relation (``only a love of the boy's soul" to quote Evola, if my memory serves me well). And I think that is probably the case with the mannerbund/male warrior phenomenon….in that it most likely started out as a chaste, brotherly love relation and then eventually declined to where it turned into a relationship of an erotic nature.


\hfill

\texttt{Aghorable on 2012-07-29 at 05:26 said: }

Turning now to the question of conjugality…

Our good host asks where we rebels take our stand, what will be our path; I fear that more than a few readers have, after lobbing a few stones at the forces of the modern world, long since passed beyond what we might call the rampart of rebellion. They now trek through the deeps of the Detachment Forest, soothing their wounds with help from the nymphs and gnomes who dance and play at the edge of the waters of the Lagoon of Indifference, and seeking the uncharted lands beyond. 

One Ms Moretti recently wrote an article published on another website, where she eloquently summarised the issue: ``So stop intellectually masturbating to Evola and go make your Nordic gods proud." And this for a relatively `activist' audience – imagine the bombs she would drop if she partied with us!

Cologero comments, in response to Andrew: ``The required social structures no longer exist. Nevertheless, action follows thought."

I think it is worth mentioning the interesting writings of a Karlo Z. Valois, wherein he treats of behavioural orientations (his `style elements'. as they pertain to those, including women, who open and seek to follow the way of a superior, vertical development in our current unfavourable milieu. He does this without collapsing into trivialities. The writings can be found on his dormant but still online blog.

Indeed, the fact that the social structures no longer exist means that the highest reasons for getting married and having children in the West have sadly evaporated, and so we see (and will continue to see) appeals to surviving but subordinate reasons from interested quarters. Will this be enough to seduce students of spiritual perfection who hide in their hearts visions of love and order so rarefied that, as Cologero says, the modern mentality would find them incomprehensible and ridiculous? 

Perhaps a cultivated detachment will eventually motivate some of us to take a woman (or three) for wife precisely because we have jettisoned all fear about the outcome, the glorious Me ne frego! 

Yes, de-detachment. Oh the lure of Mephistopheles!


\hfill

\texttt{Michael on 2012-07-29 at 11:24 said: }

Outstanding post, but this statement raises a question: ``Of course, this applies to the higher castes, as the plebeians did not have a sacred marriage."

How does one know whether he is a member of a higher caste or a plebeian? It sounds like it makes a difference as to what type of marriage one should pursue.


\hfill

\texttt{Cologero on 2012-07-29 at 12:13 said: }

``It sounds like it makes a difference as to what type of marriage one should pursue."

A plebeian would not even bother to ask himself that question.


\hfill

\texttt{Izak on 2012-07-29 at 19:30 said: }

Matt: I wouldn't be surprised.

Aghorable: Over on corrupt.org, which I think was originally meant to be a nihilist blog (?), the last few entries were about how wonderful it is to get married in one's mid-twenties, completely oblivious and indifferent to the decay of the surrounding world. It only makes sense that the blog was discontinued soon thereafter. 

Similarly, there was another blog called In Mala Fide, which was also intended to be a nihilist blog, and the last few entries there from the blog's curator were about the virtues of abstaining from meaningless sexuality, masturbation, and rejecting the basic onanism of society at large. That blog also was discontinued soon thereafter.

All of this makes sense to me. These people are probably doing more good without blogs than with them.


\hfill

\texttt{Aghorable on 2012-07-30 at 21:35 said: }

I wrote above ``…the fact that the social structures no longer exist means that the highest reasons for getting married and having children in the West have sadly evaporated…"

Inspired by an excerpt on marriage from the writings of Bô Yin Râ, I feel it necessary to state that just as there is nothing in principle preventing a man from attaining full spiritual integration even in our day, there is nothing in principle preventing a man and woman from coming together (no pun intended!) and likewise spiritually reintegrating one another, this being the highest reason for getting married. Indeed, the first case will always involve a `marriage' of sorts (i.e. not necessarily to a person), for every complete spiritual integration must resolve the male-female binary. 

Therefore, the highest reason for getting married actually cannot `evaporate' entirely; the important social reasons can (and largely have), and the possibilities of realisation may not be great, but let it clearly be said that there are likely many cases of lovers who even today have forged commitments that carry them beyond the profane to a greater or lesser extent, so it is not vain to seek to make this the basis of your personal spiritual opus.

May the Eternal Lover light our path.


\hfill

\texttt{Sparrow on 2015-05-22 at 19:04 said: }

No doubt most, if not all, of those peddling this nonsense about traditional family values being Semitic, are the same people who fawn over the Vikings and pagan societies as (alleged) proto-feminists. Such folk dominate the White Nationalist scene.


\end{sffamily}\end{footnotesize}
