\section{Recovering Tradition}

\begin{quotex}
Gradually there will revive in Christendom the forgotten, deeply sleeping, and perished treasure of wisdom and sacrificial deeds of the past — right back to the primeval revelation and the paradisiacal state of mankind. Thus all truth and all love of all times will have their home in the Church of Christ, which will then be the all-embracing (catholic) unity of all things and all beings who are striving for timeless values — in the sense of realizing the ideal of one Shepherd and one flock. The words of the Creed: ``I believe in one holy, catholic and apostolic Church" receive their full significance when the Church becomes universal (catholic) not only in space but also in time, i.e., when it embraces not only all peoples of the present, but also times of the past. \flright{\textsc{Valentin Tomberg}}

\end{quotex}


In recent posts we have been delineating the characteristics of a Traditional society, so that those so inclined would know what to work towards. Moreover, we emphatically deny that there is such a movement or school as ``Traditionalism" or that a man can be a ``Traditionalist", \emph{sensu stricto}. ``Traditionalism" is a term without content, unless its content consists of the valid Traditions. In other words, one cannot be a Traditionalist, that is, a man who follows Tradition in general; rather, one must follow a particular Tradition. Of course, in a Traditional society, that means one follows the Tradition he is born into, the advice given by the French magician Eliphas Levi. Guenon held the same position, except for those rare men who are beyond caste and hence free to follow the Tradition of their choice.

For the Westerner of today, that Tradition should be Catholicism, although it is now barely recognizable as Traditional. A case can be made that one of the European paganisms is the natural Tradition of the West, and this seems plausible to many since paganism has been in the unconscious of the West and is now beginning to reawaken; or, as Stephen Flowers puts it, ``Odin has been sleeping." He is well read in this field and has done much to reconstruct or resurrect the ancient Germanic religion. I shall be speaking of the former, since I am more familiar with it; nevertheless, the basic points will apply to both identically.

First of all, the question recently came up about whether Christianity contains the seeds of its own destruction. We need to point out that every human movement, even divinely inspired, will gradually run down; this even happened to paganism, which yielded to the newer religion. This is a spiritual law. After the initial conscious impulse, a movement then continues mechanically, declines, and eventually may even become the opposite of how it started. Periodically, conscious impulses must be applied to it, to set it back on its course.

Unfortunately, what is called Traditional Catholicism does not go back far enough, since its tradition stems from the counter-reformation at Trent, when it really needs to go back much further. For example, it needs to rehabilitate the likes of \textbf{Dionysius the Areopagite}, \textbf{Clement of Alexandria}, \textbf{Origen}. The idea of \emph{gnosis} needs to be restored. The idea that \emph{theosis} is required for salvation, not faith alone, must be taken seriously once again. The philosophy of Plato and the Neo-Platonists must form the human side of divine revelation (not a replacement for it). These ideas are not novelties, quite the contrary, they are the true Tradition.

Of course, due to the differences in human nature and the law of castes, not everyone will be capable of understanding things in this way. But the elite will do so, even in not a public way. The warriors will need to apply these teachings to particular circumstances, not because they are more manly, but because that is the law for their caste. This is why a Traditional society is not a theocracy: the elite teach the principles; the warriors and producers apply them.



\flrightit{Posted on 2010-03-16 by Cologero }

\begin{center}* * *\end{center}

\begin{footnotesize}\begin{sffamily}



\texttt{Comprehensor on 2010-03-18 at 18:14 said: }

Though many Westerners are born into Catholicism it is a dead religion (i.e. it no longer possesses the spiritual influence) and so what is it they are really born into? Something which is beyond their understanding and reduced to an egotistic historicism, moralism, sentimentalism, and messianic marxism. Even if there were some of the church hierarchy who had any real knowledge of tradition other than their theological hogwash the church has such a bad reputation that to fix it one would have to purge it of nearly all its ordained members. And all of this impossible work would be for naught because quite simply Christianity is far to sentimental and Semitic a form, which is in direct conflict with the Aryan spirit. What the traditional man should be focused on today is in taking the essential aspects of Aquinas, Eckhart, and Boehme, mixing it with the Vedanta and Yoga, and applying it to their own ethnic tradition, that is, reconstructing the old ``pagan" paths. Above all, one needs to receive an initiation into the lesser and greater mysteries to have any qualification at reconstructing a tradition, and this initiation can only come from living traditions, either Sufism, Hinduism, or Buddhism.


\hfill

\texttt{Cologero on 2010-03-18 at 21:31 said: }

Yes, I agree with your project in outline, but the devil is in the details.

I would add that the core of Western esotericism has been Hermeticism … both Guenon and Evola agree on this. Evola, however, insists this is really a pagan movement. Of course, its origin is pagan, if we follow the progression

Hermes Trismetigistus $\rightarrow$ Pythagoras $\rightarrow$ Plato $\rightarrow$ Plotinus

Nevertheless, this trend buried itself in the dominant religion of the West, as we see from the grail legends and various thinkers such as Dionysius, Clement of Alexandria, Origen, Ramon Lull, Ficino, and so on. Two recent attempts, both by Russians, have attempted once again to integrate Hermeticism, viz, Valentin Tomberg and Boris Mouravieff. It remains to be seen if they create a lasting impact.


\hfill

\texttt{Comprehensor on 2010-03-19 at 00:21 said: }

The devil I would say is the industrial-capitalist civilization which is in no way compatible with tradition and culture.


\hfill

\texttt{Cologero on 2010-03-19 at 05:04 said: }

I'm afraid that is just a symptom, the result of the degeneration of castes. Only a regeneration is the cure, not a revolution.


\hfill

\texttt{Comprehensor on 2010-03-19 at 05:50 said: }

Whatever it is it is here. Modern education, technology, machines, spoonfed consumerism–all of this ruins the mind physically and psychically, and it is only on course to become much worse. The masses aren't going to bend a knee to any initiatic elite, nor are they interested in initiatic regeneration. Most people think that initiatic rites are either satanic or imaginary. This being so how much further removed are they from intellectual principles? And as far as the people who have assumed the functions of the rulers, warriors, and priests, they are all rather pathetic–with very few exceptions–and no elite would waste their time trying to teach those who are corrupted to their very core. A restoration of tradition would take the miracle of all miracles.


\hfill

\texttt{Cologero on 2010-03-19 at 09:43 said: }

The masses will believe any old thing and can change in an instant … they are not burdened with gnoseological questions. It would be better that they believe something true rather than false.

The elite do not kvetch. The elite, and I mean the warrior elite, have the will to power, the power to impose their will. After the disintegration of the Western half of the Roman Empire, the Occidental world was in worse shape than today. The elite went back to the sources and modeled themselves after the great pagans: Hector, Alexander, Julius Caesar, Vegetius, Scipio. Where is that elite today? Who today knows anything about Charlemagne, Arthur, Boucicaut, Godfrey de Bouillon? Forget the principles, all the principles anyone needs are on this blog. That is really for the tiny few, who are ineffectual in the world without the Ksatriyas to actualize those principles in the world. Forget the paleo, neo, new, alternative rights … go back to the beginning.


\hfill

\texttt{Comprehensor on 2010-03-19 at 17:38 said: }

Principles are NOT for a tiny few. They are the backbone of the entire civilization. For a return to tradition we require a complete change of everything modern. That is a mammoth task and no will to power can achieve that. I do not consider Alexander and Caesar great pagans, but rather the enemy. They turned away from the sacred to wage destructive wars and to create a universal empire. I am more interested in preserving unique ethnicities and unique cultures rather than making one great ``European Empire." Moreover, I am opposed to the exoteric/esoteric model which limits the esoteric to a tiny few. We need to do away with that entirely and move to a more metaphysical and participative system like Hinudism which doesn't have this distinction.


\hfill

\texttt{Cologero on 2010-03-21 at 02:01 said: }

Despite many efforts, several misconceptions persist.

We agree, principles the backbone of the entire civilization, but the fact is that different people will relate to them differently.

Some — a tiny few — will understand them directly. Others will implement the principles in the political and economic realms. The remainder need to be taught the principles in the form of stories, imagery, myths and dogmas — just as I pointed out: ``It would be better that they believe something true rather than false." This is the nature of things and it does not imply any defects in the last group.

The way ideas become actualized in the world is through Will … that's just the way it is. The modern world you apparently detest was a creation of the will. That leads to a few questions. Are sound ideas being actualized or are false ideas? Is this being done consciously or spontaneously? What forces are in play?

By definition, Empire preserves local ethnicities and cultures. If you want to preserve yours, then go right ahead.

Alexander and Caesar prerseved tradition, and can't be judged by ordinary standards since secondary causes come into play. They made possible the free exchange of ideas. But I don't need to dwell on this at this time, since that was not the point.

I brought it up for a few reasons. First they were effective. That is why the medievals studied them … to learn to be effective.

Second, we don't see things as isolated events, but rather as the unfolding of a larger scheme. Alexander made possible the spread of Hellenic culture and philosophy.

The Roman empire made possible the spiritual unity of Europe. That is how the medievals saw it.

To be clear, the purpose of this blog is educational — not political nor religious — and its perspective is twofold: first of all, to expound Traditional ideas in modern languages for those able to grasp them. Second, to indicate specifically their relation to the situation of the West. As for the latter, this requires the recognition of the continuity of a certain spirit appropriate to Western man. There are many today who would deny it, and prefer revolution to continuity, both within the West and without.

But the worst are those who in the name of Tradition reject huge periods of Western civilization, making the dispossession complete. I bring up the medievals because they made a conscious decision to study the ancients and appropriate whatever was of value to them in their situation. The same needs to be done today, and to condemn the great heros of the West is hardly helpful in that task.


\end{sffamily}\end{footnotesize}
