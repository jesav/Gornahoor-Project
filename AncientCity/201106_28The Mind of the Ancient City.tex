\section{The Mind of the Ancient City}

\paragraph{The Primordial State of Mind}
It is necessary to recognize that the citizens of the \textbf{Ancient City} were not academics or philosophers, they were not converts to the religion of the city, nor were they immigrants. The only way to be part of the city was to be born in it (or to be a client of a citizen); we have already described the initiation ceremonies. Their state of consciousness was different from and incompatible with the modern mind, and was closer to the Primordial State\footnote{See Section \ref{sec:HyperboreaPrimordial} in this book.}. In that state, a man's concern was his duty and to do it perfectly. Such a man's consciousness was undivided, something hard for moderns to grasp when everyone today is so cantankerous.

The rites of the family, clan, tribe or city were oral, not written, and passed down from generation to generation. This is why it was so important to have a son, although adoption was also an option. Apart from the performance of his duties, a man didn't suffer from a divided mind; he was, so to speak, beyond good and evil. If a client failed in a task, a chief could punish or execute him with no sense of guilt. The enemies of a city would be slain — man, woman, and child — or perhaps sold into slavery. That is what it means to not ``love your enemy". Of course, your enemy does not love you either … it certainly focuses the mind. Your enemy was not someone of a different race, or an adherent of a rival political system; rather he was someone who worshipped different gods. The gods of the city were jealous gods.

The performance of rites was critical. If a \textbf{Vestal virgin} allowed the hearth fire to go out, or lost her virginity, she was buried alive. At least in the \textbf{Inquisition}, a heretic had a fair trial before being tortured to death. The destiny of the city depended on the sacred fire. Only citizens could take part in a sacrifice at the fire. The heroes were worshipped in the city, sometimes for a thousand years, and to possess their relics was a sign of good fortune. Some today still save the bones of their heroes.

\paragraph{Fate}
Although the ancients were very skilled at their arts, such as farming, generalship, and so on, there was always a factor outside man's control that oftentimes caused the most skilled to fail. This was the element of chance, or fate. In the city, everything that happens is the result of man's will. Lacking the conception of abstract thought and principles, by analogy, fate is attributed to the will of the gods. Thus, the men of the city sought good fortune for their families and the city through their rites, sacrifices, invocations, and auguries, all designed to bring on good fortune from the gods.

Nevertheless, over time, they had to notice that their rituals were sometimes ineffective. Also, with the multitude of gods, each with their own requirements, there could arise conflicting claims of duty, both binding. These competing claims were a theme of Greek drama. The general policy was to imitate the gods. However, the gods themselves often fought each other. Hence, one should follow the oldest god. This led to Zeus, or ultimately to Uranus. But Uranus mistreated his children which went against the entire practice of the city.

\paragraph{The Philosopher}
Eventually such conundrums had to awaken in some minds the ``knowledge of good and evil", or discursive thought. Once things could be questioned, the blind and unquestioned system of duties holding the city together began unraveling as we have seen.

If true knowledge was the knowledge of the unchanging, the necessary, and the whole, then the gods themselves were ignorant. That is why they were in conflict. Hence, it is insufficient to follow the gods, one must develop an understanding of the ideas. This is the metaphysical meaning of polytheism: the gods are ignorant of the whole and hence cannot serve as a reliable guide for life. They fight because they are ignorant. So the man who tries to follow the gods will be drawn into their own inconsistencies and conflicts, thus can never be a whole man or know his true will.

The philosophers, formerly restricted to mystery schools become more public. They debate the meaning of duty, justice, the city, nobility, piety and so on. For the old guard of the city, they understand that if the philosopher follows the ideas, then he has no need for the gods. This is what got Socrates into trouble. Yet Socrates has his own conundrum. Society is impossible without the sanctity of following ancestral customs, which is the duty or piety to the gods, who seem to be unnecessary.

On the one hand we have, as fundamental, a blind \textbf{love} (that is, without knowledge) for the gods, because justice is whatever is willed by the gods, without question or external ground. On the other, we have \textbf{knowledge}, or wisdom, as fundamental and we love whatever we know to be true and good, which are superior to the gods. In the dissolution of the Ancient City, this question motivated the next Traditional civilization.



\flrightit{Posted on 2011-06-28 by Cologero }
