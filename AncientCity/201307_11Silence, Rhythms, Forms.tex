\section{Silence, Rhythms, Forms}

\begin{quotex}
It is a monstrously ironical fact that the only civilization which professes to discount heredity and to put all its faith in environment is unique in having no positive environment to offer. \flright{\textsc{Martin Lings}}

\end{quotex}
\paragraph{Speaking from the Heart}
Since we started the translation of \textit{La Tra\-di\-zio\-ne Ro\-ma\-na} in the middle, we left out the explanation of some concepts he alluded to. \textbf{Guido De Giorgio} lists the moments of the Divine Cycle as Silence, Rhythm, Forms, and the Primordial Tradition. That is how we ultimately want to see it, not as it appears to man in the merely human state. I don't know yet if I'll make De Giorgio's chapters available, but in the meantime I will provide my own thoughts. The fundamental issue is that the negative moment has passed. The crisis of and the revolt against the modern world have, by this point, been fully exposed, at least to those few men with an affinity to that point of view.

Hence, it is no longer necessary to rely solely on exoteric teachings, e.g., ancient Hopis beliefs, or some Etruscan fertility goddess, and so on. We are convinced, quanto basta? We are so accustomed to think discursively, to talk about something out there, to debate interpretations. Instead, it is necessary to unthink, to be in the Silence. As Guenon pointed out, the Silence is not a possibility of manifestation, we cannot point to it or see it, certainly not hear it. Rather, it is a clearing, wherein something can appear.

Of course, there is no end to the thoughts that vie for that space, as nature abhors a vacuum. Who, however, takes this exercise seriously? Namely, to be watchful, to be on guard, to be careful which thoughts are allowed in. The dominant thoughts are loud and insistent, they agitate rather than calm. The subtle thoughts come from further away, they are attenuated, soft, and they can be missed without the Silence, until we can vibrate at that same level.

\paragraph{The Rhythm of Time}
\begin{quotex}
Just as today's most obsessive notion is that of the material reality of time, self-existent time was the first lie of social life. As with nature, time did not exist before the individual became separate from it. Reification of this magnitude—the beginning of time—constitutes the Fall: the initiation of alienation, of history. \flright{\textsc{John Zerzan}, \emph{Elements of Refusal}}

\end{quotex}
Man is not yet under the thumb of quantity, as there is a natural rhythm to his life. Time is cyclic, not linear. Time is not independent or ``outside" him, i.e., dominating him. Rather, it is experienced as the ``unfolding" of being, its revelation and reflection in manifestation. The future is in the present, not some ``promised land" of scientific knowledge or societal evolution.

He performs his rites and ablutions on a cosmic scale, with the rising and setting of the sun, the motions of the planets. He eats when hungry, sleeps when drowsy. He does not need a clock to tell the time. Similarly, he knows the seasons by their different qualities. He has no need of calendars to tell him when to expect spring or winter.

When time becomes quantified, it becomes linear, because there is no geometry of time (alone). Yet even physicists measure time through rhythms, as the second is defined as so many cycles generated by the cesium atom. Since time slows down the faster something moves, there is no absolute standard of time. If one physicist stays put and another takes a high speed trip, for each of them, a second will still count the same number of cycles from their cesium clocks. However, when the traveler return, they will find they each counted a different number of seconds.

So, there is no way to avoid rhythm in understanding time. The modern choice is quantitative and exterior, i.e., to count something ``out there". But human time is also based on a rhythm, but it is qualitative and interior. It is only a modern prejudice to suppose that physical time is superior. There is, or at least was, an echo in this in Islam. A month does not end with the passage of a certain number of days, but rather by the appearance of the new moon. Hence, the beginning of the month is locale dependent and not universal. If the night is cloudy and the moon not visible, the current month may get an extra day.

Hence, in a fully human world, we determine cycles and ages by qualitative rhythms not by counting. For example, as was pointed out, the Hindu scriptures on the length of the yugas is accessible to all (although the actual numbers are in dispute). If all that is true, it cannot be the real teaching, which is always accessible only to the few. If the relative lengths of the yugas differ, that is not because there is an absolute ``objective", exterior, standard to count the days. Rather, it is because in later yugas, time is compressed as human change speeds up. This is the esoteric meaning.

\paragraph{Artifacts}
Beyond the natural things, which are reflections of ideas in the divine mind, man creates his own artifacts, which arise from his own mind. You often hear people casually remark that technology is ``neutral", things can be used for good or evil, but they themselves are neither. That is not quite true, as there is no point of neutrality. As man creates artifacts, he alters his environment, the sphere in which he lives and operates. An artifact will turn out to have uses beyond what it was designed for, so there will be unintended consequences.

In the transition to a sedentary lifestyle, changes necessarily occur beyond the expected. A division of labour becomes necessary and a hierarchy results. Since, for example, the land flooded by the Nile needs to be known with precision, quantification becomes more important; this knowledge becomes known to the few. The increase in food production leads to the ability to horde (i.e., saved labour or capital), and the population will increase even as their lifespans get shorter.

In later ages, man has difficulty is distinguishing between the natural and the artificial. He begins to assume that everything is conventional or a ``social construct", without reference to anything transcendent. The success of the artifacts leads men, in their hubris, to believe they have conquered nature. As the experience of the transcendent is lost, man becomes the centre.

Humanism, which purports to magnify man, actually means the abolition of man, i.e., the elimination of his transcendent nature (\textbf{Martin Lings}: Ancient Beliefs and Modern Superstitions). Lings points out how modern medicine, while extending individual lives, ends up weakening society:

\begin{quotex}
Modern medicine means, in the long run, the abolition of health through degeneration of the species caused by the development of a system which allows man, and therefore in a sense forces him, to flout on an enormous scale the law of natural selection which is nature's antidote to decadence. To say that we live in a world where everybody is half-dead because nobody dies is clearly an exaggeration, but that at least is the trend. 

\end{quotex}
A medical doctor shows that general IQ has been declining\footnote{\url{http://iqpersonalitygenius.blogspot.com/2013/05/approx-1-sd-decline-in-general.html}}, the more learned we believe ourselves to be.

\paragraph{Primordial state}
Man in the primordial state, in the "Golden Age", lives in a state of nature, but by that we mean nature as created, not despiritualized nature, the flattened two-dimensional world of appearances of modern man, which is random, aimless. Rather, nature is three dimensional, ``seen" as the reflection of higher states (i.e., ``formal cause") and with purpose (``final cause") in an interconnected whole. This is the meaning of organic, ``organized", all the parts ordered to the totality. In three-dimensional nature, there can be no perversity. These are not ``theories" but rather the result of a direct perception; that is why he has no need for such theories.

There is no need for hierarchy: a man is the priest in his own household, he hunts his food, he manufactures his tools. He is in direct communion with the gods, devas, angels, who were not yet relegated to some celestial abode or remote mountaintop. His mind is not yet cluttered with his ``personal" thoughts, fantasies, concerns, and so on, exactly what modern man considers the source of his very identity: ``I think, therefore I am."



\flrightit{Posted on 2013-07-11 by Cologero }

\begin{center}* * *\end{center}

\begin{footnotesize}\begin{sffamily}



\texttt{Scardanelli on 2013-07-16 at 19:00 said: }

``We are so accustomed to think discursively, to talk about, to talk about something out there, to debate interpretations. Instead, it is necessary to unthink, to be in the Silence."

Precisely. I find that I've grown tired of debates and discussions and seek only a transformation of experience. I read and seek knowledge of course, but ultimately what I seek is instruction on operative or ``practical" issues if that is the correct wording. I especially appreciate these posts that clarify the way, the daily struggle…I wish it was easier to discuss things of this nature.


\hfill

\texttt{muaddibbr on 2016-07-12 at 09:42 said: }

``You often hear people casually remark that technology is ``neutral", things can be used for good or evil, but they themselves are neither. That is not quite true, as there is no point of neutrality. As man creates artifacts, he alters his environment, the sphere in which he lives and operates. "

Pretty good. What do you recomend for further readings on this point (on technic)?


\end{sffamily}\end{footnotesize}
