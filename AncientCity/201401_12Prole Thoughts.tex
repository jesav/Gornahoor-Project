\section{Prole Thoughts}

Since I had been taking notes about various news programs over the past week, I was intending to comment on them. That turned out to be distasteful and probably futile, so instead I prefer to continue commenting on European aristocratic philosophy as succinctly summarized by Prof \textbf{Alexander Jacob} in his book \emph{Nobilitas}; since it seems to be unavailable, there is all the more need to do so.

Many contemporary historians have lamented the lack of direct documentation provided by the ``common" people of past generations. Hence, they go looking for letters to soldiers, or diary entries, etc., to fill in that gap. Today, there is no such problem at all. On the contrary, ``prole thoughts" are all too available through the so-called ``social media" of facebook, twitter, huffington post, and so on. Their interests, concerns, activities, and forms of thought are available for the next great historian of our age to digest. This, however, must be much more than a mere statistical study.

A half step up are found the poseurs. Not content with common interests, they use social media as a debating platform for politics, philosophy, religion and related erudite topics. Since it is impossible today to master all human knowledge, the days of the Renaissance Man are long gone. Nevertheless, most people feel competent to make definitive judgments on complex issues of science, economics, ethics, sociology, and psychology. You would think a lifetime of average grades and standardized test results would discourage people, but it seems not to be the case. Without a teacher, everyone imagines himself capable of ``A" work.

That is likely because at adulthood, his worldview is firmly set, for better or worse, and any challenge to it results in psychic distress. That is why such social media discussions often turn nasty. Someone once said that most men hold opinions that can each fit onto a placard. So a half dozen or so disconnected placards contain the framework for their entire life of the mind. The details don't matter, since any content can be reframed within that framework, like some Procrustean bed. \emph{Nobilitas} is intended to reach the one in two hundred men who are not satisfied with placards.

\paragraph{Rule of the Best}
Based on some recent comments, let us be clear what this means. The best are well born and well educated. Well born does not mean having the best parents, so strictly speaking it is not always a matter of a bloodline. Furthermore, to be ``educated" means to have wisdom developed from within. In short, the raw matter must first be there and then it must be developed; obviously, a healthy social structure will develop such talent. This is contrary to the tabula rasa ideal of today in which everyone is considered to be born equal, and the task of education is to implant the right ideas into that empty space. I should mention at this point that the Greek ideal can also be found in the East, for example, in \textbf{Shankara}'s \emph{Crest Jewel of Discrimination}.

Starting with the Greeks, we see the themes of the One, the Good, and the Logos. The wise man is one who can see the unity of spirit in the multiplicity of manifested things. The highest idea is the idea of the Good and the Logos or cosmic order is an object of contemplation. As we saw last time via Giovanni Gentile, the intellectualism of the Greeks gives way to the voluntarism of the Christian era. In the latter, there is the same understanding of the One; however, the Good becomes Love and the Logos is no longer object but becomes subject. Moreover, Jung claimed that ``individuation" was possible only in certain cultures. Keep these thoughts in mind in the historical manifestations of aristocratic philosophy.

Since modern ideals are so ingrained in nearly everyone today, it may be difficult to understand or come to terms with aristocratic philosophy. For example, the first principle is that the superior rules the inferior; in particular, the lower desires are to be ruled by the dictates of reason. On the contrary, today's standard is that the purpose of reason is to lead to the satisfaction of lower desires. In particular, the conventionally intelligent\footnote{\url{https://www.gornahoor.net/?p=1582}} mark themselves by being sexually risqué, as though they can't find any other purpose for their intelligence.

In the social realm, the same principle holds. Plato claimed that untrained passions are found in children, women, and lower classes. Apart from children, perhaps, such a claim is now rejected, even as passion itself is celebrated. In particular, any movements that claim to be recovering the aristocratic philosophy ought not be run and organized by women.

I have noticed the tendency of some self-described elitist movements that have disdain for lower classes. On the contrary, the Greek model was the common good; society was to be organized even for the benefit of the lower classes. Actually, the philosophical elite, which corresponds to the traditional idea of the Brahman class, has little need for what pleasures the rest of society. Hence, the ruling elite have no need for sexual or financial exploitation, unlike what is found in all modern social systems. \textbf{Plato}'s \emph{Republic} would be ruled by a philosopher-king. Prof Jacob quotes Plato:

\begin{quotex}
Unless either the philosopher become kings in our states or those whom we now call our kings and rules take to the pursuit of philosophy seriously and adequately, and there is a conjunction of these two things, political power and philosophic intelligence, while the motley hordes of the natures who at present pursue either apart from the other are compulsorily excluded, there can be no cessation of troubles for our states nor for the human race. 

\end{quotex}
This is somewhat different from \textbf{Rene Guenon}'s view of the separation of spiritual authority from political power. The philosophers are the spiritual leaders, advisors, and educators, but few have the talent, or even the interest, for ruling. Hence, kings must be educated in the proper philosophy and willing to follow the direction of the philosophers. This system as such was not fully tried in the West until the Middle Ages in which the role of the philosopher was that of the Church and the king's legitimacy depended on the spiritual authority.

As for Plato's prediction of troubles for the human race if his ideal was not followed, can there be any doubt? Since, in the West, many have found pockets of stability, they are unaware of such problems and have scant sympathy for the actual victims of such troubles.

Of existing forms, Plato insists that kingship is the best provided there is rule of law. This is contrary to \textbf{Julius Evola}'s opinion\footnote{\url{https://www.gornahoor.net/?p=6548}} that leaves no space for a separate spiritual authority in the monarchy and allows the monarch absolute power. However, for Plato and the Medievals, the positive law must be consistent with the cosmic law.

\paragraph{Rule of the Gentleman}
The other Greek thinker treated by Prof Jacob is \textbf{Aristotle}, who has a slightly different conception. While Plato's system goes to extremes, Aristotle recommends moderation. For Aristotle, the national good is superior to the merely individual good. Such a position is difficult to defend today in the West since the idea of a nation is mostly discredited. Hence, Western political systems are organized around providing goods and services to particular individuals based only on their political influence. However, this is only partly true, since various subnations, based on race, ethnicity, religion, or even behavioral preferences, vie for goods apart from the good of the body politic as a whole.

Although Aristotle believes everyone should strive for the mean, there are still those who are superior and those who are inferior. A gentleman would be magnanimous, virtuous, just, and prudent in knowing how to apply higher principles to political life. He develops friendships among men of similar character.

Now, this is a sound corrective to Plato. A ruler ought not be as extreme or fanatical as Plato's guardians. This was the position in the Middle Ages, in which the nobility held the place of the gentleman. While a St. Francis of Assisi may be a model for the spiritual class, it is not intended as the proper way to rule a society. Unfortunately, fanaticisms of various types have been commonplace in the turmoil of the modern world.


\hfill

Next: Rome and the Renaissance



\flrightit{Posted on 2014-01-12 by Cologero }

\begin{center}* * *\end{center}

\begin{footnotesize}\begin{sffamily}



\texttt{JA on 2014-01-13 at 09:53 said: }

This series is very informative and much needed, I look forward to the subsequent chapters.


\hfill

\texttt{Avery Morrow on 2014-01-13 at 11:39 said: }

``While a St. Francis of Assisi may be a model for the spiritual class, it is not intended as the proper way to rule a society."

Honestly — even if someone happened upon this proper way to rule, who of this era would want to listen to them? Perhaps all we have left at this point, lacking any acceptable guidance from the old social elites, are pointers back to the unrefined Tao.


\hfill

\texttt{Michael on 2014-01-16 at 22:01 said: }

I think the book is an excellent introduction to the topic, but it leaves me wanting to learn more. Does anyone have any suggestions for reading how aristocracies are established in the first place?


\hfill

\texttt{Cologero on 2014-01-16 at 22:57 said: }

Yes, Michael, that is the point and it will be covered shortly. By way of introduction, I can make the following points.

First of all, it must be stipulated that the civilization of the Middle Ages represents in the West the closest approximation to a Traditional civilization as described by Guenon. Anyone who will give the matter at least serious consideration can hardly deny it.

So we can focus on its beginning stages so we can learn from it. The knights represent the ideal of the gentleman and the monks were closest to the philosophers, i.e., those dedicated to a higher calling above material and sensual pleasures. Plato's further division of the Republic into warriors and producers mirrored the Middle Ages.

The creators of the civilization of the Middle Ages were the \emph{true} men among the ruins, i.e., of the Roman Empire. For them, it was a matter of life or death, not an effete intellectual theory as it has become today.

The social hierarchy was both consciously and unconsciously created and arose organically for the most part. Strong leaders such as Clovis and Charlemagne, established and protected that hierarchy. Certain intellectual currents, originating from the poets and religious authorities, provided the justification. We addressed that topic specifically over a year ago in \textit{Pray, Fight, Work}\footnote{\url{http://www.gornahoor.net/?p=5198}}, even if no one then seemed to grasp its significance.

Now there are many other explanations for this, but they do not take into account the true nature of a Traditional civilization and social organization.

As for the ``how" … a Traditional civilization depends absolutely on an authentic Tradition. That is the ineluctable foundation which the Middle Age civilization was blessed with, and that foundation cannot be based on vague imaginings or unsupportable speculations. Without that, there is nowhere to go.


\end{sffamily}\end{footnotesize}
