\section{From Virgil to Dante}

The Middle Ages are so called because it represents the era between Antiquity and modernity. We can also regard it as the Traditional society between the Ancient traditional world and the coming Traditional society of the future. That the Middle Ages represent Tradition is beyond dispute. Now, there may be some contemporary Europeans (or their descendants) who do not feel part of that tradition and identify with some alleged northern tradition that has no relationship to either the Ancient or Medieval worlds. That, however, does not change the facts; furthermore, we accept that Northern tradition as an integral part of the one Western, or Hyperborean, tradition, which has taken several forms.

To make clear the reasons, we shall rely on the wisdom of Rene Guenon. The reader can then proceed as he likes, whether to seek to renovate and renew the Medieval tradition, or to use it as a model for the next great Tradition. In the end, they may very well amount to the same thing. Let us begin at the beginning.

\begin{quotex}
Christianity originally had both in its rites and doctrine an essentially esoteric and thus `initiatic' character… the earliest Christian church would have had to be a closed or reserved organization. 

\end{quotex}

So Guenon comments on the impenetrable obscurity that surrounds the origins and early stages of Christianity, which points to a deliberate design. Why, then, did the Church become open and exoteric? He explains:

\begin{quotex}
If we consider the state of the Western world in the age in question, it is easy to see that, had Christianity not `descended' into the exoteric domain, this world would soon have been deprived of all tradition, for the traditions that had existed until that time, especially the Greco-Roman tradition, which naturally was predominant, had reached an advanced state of degeneration heralding the imminent end of their cycle of existence.

The conversion of Constantine implied, by a sort of official act of imperial authority, a recognition of the fact that the Greco-Roman tradition had thenceforth to be considered extinct. 

\end{quotex}
Nevertheless, the initiatic tradition with Christendom continued. Back to Guenon:

\begin{quotex}
From Pythagoras to Virgil, and from Virgil to Dante, the `chain of the tradition' was undoubtedly unbroken on Italian soil. 

\end{quotex}
This continuity is not unknown. The monk \textbf{Dom Odo Casel} recognized a continuity between the rites of the pagan mysteries and those of the early Church. Guenon admits the possibility of both

\begin{enumerate}
\item Spontaneous initiation 
\item Exceptional cases in which a virtual initiation that had remained attached to the sacraments might have become effective 
\end{enumerate}

He recognizes the certain writings from the Middle Ages were “manifestly initiatic in character”. He mentions the \textbf{Order of the Templars} and the Chivalric Orders, for example, and the \textbf{Fedeli d'Amore}, which included Dante as well as several other Italian poets, whose origins can be traced back to the earlier Sicilian poets and the Troubadours. Guenon points out that it was a requirement that initiates write love poetry, which is really an allusion to the \textbf{Divine Sophia}. This is also found among the Sufis such as Attar, Hafiz, or Rumi. At a certain point, the trail goes cold, ending with the more secretive Rose Cross. This does not mean necessarily the total end of initiatic rites in the West, just that they went deep underground.

Nevertheless, the Eastern Church did maintain a valid initiation in \textbf{Hesychasm}, whose “initiatic character is indisputable”, according to Guenon. Guenon had always recognized, even in an early book such as his Hinduism, that there was a valid metaphysic in the Alexandrian fathers in the East and in neoplatonism in the West. As Gornahoor has pointed out, the French follower of Guenon, Albert Gleizes\footnote{\url{https://www.gornahoor.net/?p=1430}} recognized in \textbf{Plotinus}, \textbf{Augustine}, and \textbf{Boethius}, the founders of Western Christendom.

Hesychasm teaches a technique of invocation which is called in Greek, mneme, that is, remembering, on which any reader of Gornahoor can recall our emphasis. So, we can observe a closed and secret Church that became exoteric, while retaining an initiatic tradition. That then went underground. Now should we really be surprised that the esoteric teaching should now make itself visible? We have mentioned the “coincidence” of two specific events\footnote{\url{https://www.gornahoor.net/?p=334}}. In Mouravieff, there is a claim to an initiation on \textbf{Mount Athos} itself, along with an instruction to make its teaching public. In \textbf{Valentin Tomberg}, again beginning in the East, we witness a sudden conversion to the Roman Church. The two books, different in tone, but similar in depth, reveal an esoteric teaching. Won't it suffice to repeat what Tomberg writes about remembering and evocation.

\begin{quotex}
Memory is the magic, in the subjective domain, which effects the evocation of things from the past… The present remembrance is the result of a magical operation … where one has succeeded in evoking from the black void of forgetfulness a living image from the past. 

\end{quotex}
Nothing is ever lost. If we can only learn to remember, the mysteries of the past will be revealed again. As the Nordic adage states, the divine sleeps in the rock\footnote{\url{https://www.gornahoor.net/?p=5}}: Forgetting is to remembering as sleeping is to waking.

Back to the earlier point, why was it necessary to forget? It was to re-establish the traditional arrangement of spiritual authority and temporal power, Christ and Caesar, the Pope and the Emperor, Jesus as Priest and King. Jesus as prophet, represented in the Tarot as the unnamed Hermit, and the Church of John, had to be submerged, even to the point that Evola regarded its revival as an Utopian dream\footnote{\url{https://www.gornahoor.net/?p=43}}. Yet there was an erstwhile monk who, a century ago, was aware of it\footnote{\url{https://www.gornahoor.net/?p=2917}} and expected its return.

The more important reason for the exoterism of the Church was so it could provide a path for salvation. This may mean nothing to many people in our day, but perhaps the need to meditate on the Four Last Things\footnote{\url{https://www.catholictradition.org/Classics/4last-things.htm}} to be convinced otherwise. For those who insist on going deeper, for whom `Paradise is still nothing but a prison’, the answer is less clear.

They say, when the student is ready the master appears. So to trust in that is to make oneself ready — by study, prayer, meditation, spiritual exercises, and right action — and trust in initiation will come, whether by ordinary or extraordinary means.

Some may chose a valid initiation elsewhere, for example, by a Tibetan lama or a through a Hindu mantra. It is a mistake to believe, as apparently some readers of Gornahoor do, and many others that I have read about, that such an initiation commits one to a specific path. Quite to the contrary and for a Westerner it is almost always an error and hindrance. Guenon is quite clear about this and explains it in detail.

\begin{quotex}
The question was whether Dante was Catholic or Albigensian. For others it seems rather to be whether he was Christian or pagan. For our part, we do not think that such a point of view is necessary, for true esoterism is something completely different from outward religion, and if it has some relationship with it, this can only be insofar as it finds a symbolic mode of expression in religious forms. Moreover, it matters little whether these forms be of this or that religion, since what is involved is the essential doctrinal unity concealed beneath their apparent diversity. \emph{This is why in the past initiates participated in all forms of worship, following the customs established in whatever country they happened to be}. [my emphasis] 

\end{quotex}
This is why one should be suspicious of those who make a big show of donning Buddhist or Hindu garb in a Western country while reciting foreign texts. A true initiate will follow the customs of his home, recognizing legitimate spiritual authority and temporal power, honoring his ancestors, participating in its rites, and building solidarity with his kith and kin. He can accomplish more by explicating the symbols and dogmas of his own Tradition rather than by introducing an alien vocabulary. The mark of understanding is to be able to rephrase things in one's own words rather than in repeating someone else's wisdom.

\hfill

All quotes are from \textit{Insights into Christian Esoterism} and \textit{The Esoterism of Dante} by \textbf{Rene Guenon} unless otherwise indicated.


\flrightit{Posted on 2011-10-29 by Cologero}

\begin{center}* * *\end{center}

\begin{footnotesize}\begin{sffamily}

\texttt{Will on 2011-10-30 at 10:48 said:}

Excellent post.

A couple tangential points. In Guenon's book on Dante, he mentions the Spanish author Miguel Asin Palacios who wrote a book claiming that Dante borrowed or stole his vision of paradise from the Sufi Ibn Arabi. Guenon's perspective, of course, is that such correspondences are not indicative of theft, but rather of confirmation, as when two scientists performing the same experiment achieve the same result. If anyone is so inclined, it would be interesting to learn more about the possible connections between Dante and Ibn Arabi. If indeed there was `cross-pollination' between the Templars and the Assassins during the Crusades, it seems possible that Sufi initiatory chains may have made their way into Christendom, along with the re-introduction of Greek wisdom that was occurring at the same time.

Second, the Nordic adage which is quoted reminded me of the Tibetan tradition of terma teachings. Padmasambhava, the Buddha of Tibet, supposedly hid teachings to be discovered by later disciples. Some he hid in physical locations like rocks and mountains, and some in more subtle locations such as in the mindstreams of his students, to be discovered in a later lifetime. These terma teachings, which are complete systems of spiritual practice, are supposedly still being discovered in the present day, all over the world.

\hfill

\texttt{Charlotte Cowell on 2011-10-30 at 11:29 said:}

I concur, excellent post and very well timed also.

Just a few spontaneous thoughts on this. Firstly the emphasis on memory I agree is crucial with respect to remaining conscious once awakened, and this is made well known to any initiate, who is urged to remember to drink from the lake of memory! (although I also think that at some point – perhaps for some not all – there is also an exercise in forgetting required, perhaps as needs must and perhaps as practice for certain potentials available following Resurrection?)

In any event I agree with VT's assessment that with memory we are in the realm of the supernatural. Funnily enough I had another dream last night in which I was urged to remember something. Most of the dream was an average dream, but at the very end a man who I have met once in a supernatural context (two years ago), appeared standing at a table where I had just finished eating and handed me a book. It was a smallish book, very old, with gilded covers that at first glance was engraved with a number of circles and a man sat in one corner looking up at them. `Do you remember this?' my visitor asked. I looked back down at the book and recognition hit me like a thunderbolt. I made a dramatic, gasping `GADZOOKS!!' type sound, which made him chuckle, then said `yes I do, from two years ago, but I can't quite remember what, why…’. ‘Try to remember’, he told me, then disappeared. In both the dream and upon awakening I realised that the circles represented the Sephiroth on the Tree of Life, which of course I've seen many times, but the addition of the man in the corner seemed to be new and it was evident I was being asked to recall something from a very specific point in time, a couple of years previously as I mentioned. This is probably only relevant to me and no-one else, I'm simply reinforcing the point that our memories can contain the history of absolutely everything – there being nothing new under the sun – and only need activating by whatever means. 

Memory of the love and light of Christ especially, and how it feels in that first eternal moment when one `turns around' and sees Him for the first time….

For the rest, I agree with the value and authenticity (for want of better terms) of the Hesychasm practice and also, while we're at it, with what you say towards the end about initiation and the notion `when in Rome do as Rome does' or indeed when in Bali do as the Balinese do, etc etc. This is reasonable, polite and wise practice that reinforces the idea of universal divinity. This notion is definitely helping me to settle into a firmer form of practice and discipline. While I've travelled widely and feel a natural compassion and affinity with all native peoples and traditions, at the end of the day I am a Celtic Christian living in England and my natural path and responsibility lies here for as long as I'm here. As a matter of fact a week or so ago I found myself a spiritual supervisor in this tradition and I'm just getting the house in order before starting on November 1. (please wish me luck, I would most dearly love to return to my childhood adolescent state of rigorous self-discipline and will power!)

Finally, seeing as you mentioned Dante, here is my very favourite passage from Paradiso:

The glory of him who moves everything penetrates the universe and shines in one part more and, in another, less.

I have been in the heaven which takes most of his light, and I have seen things which cannot be told, possibly, by anyone who comes down from up there.

Because, approaching the object of its desires, our intellect is so deeply absorbed that memory cannot follow it all the way.

Nevertheless, what I was able to store up of that holy kingdom, in my mind, will now be the matter of my poem.

Blessings to you all this fine day Cxx


\hfill

\texttt{Charlotte Cowell on 2011-10-30 at 11:34 said:}

Will, seeing as you mention another poetic master, here are a couple of extracts from Ibn Arabi's works that I found very inspiring and comforting, from the Journey to the Lord of Power:

And in the first states of trust, four miracles befall you. These are the signs and evidence of your attainment of the first degree of trust.

These signs are crossing the earth, walking on water, traversing the air, and being fed by the universe. And that is the reality within the door.

After that, stations and states and miracles and revelations come to you continuously until death.

*

And if you do not stop with this, He reveals to you the surface signs, you will be admonished with terrors and many sorts of states will befall you. You will see clearly the apparatus of transformations; how the dense becomes subtle and the subtle dense.

And if you do not stop with this the light of the scattering of sparks will become visible to you, and there will be a need to veil yourself from it. Do not be afraid, and persevere in the remembrance of God, for if you persevere in the remembrance of God, disaster will not overcome.

——

Perhaps we can admit that great minds think alike to some extent at least, truth being universal, but that doesn't necessarily mean one plagiarises from the other, simply that they're speaking of the same essential reality?


\hfill

\texttt{Charlotte Cowell on 2011-10-30 at 12:03 said:}

This is the picture on the cover of this book, which as it happens I haven't read (not that I know of, not in this life at least!). I think I will now tho…

\url{http://books.google.co.uk/books/about/Gates\_of\_light.html?id=u6fXjw7ogtgC}

\end{sffamily}\end{footnotesize}