\section{The Esoteric Path}

\begin{quotex}
In this case it would be better, although not absolutely necessary, for the elite to be able to take as its basis a Western organization already enjoying an effective existence. It seems quite clear that there is now but one organization in the West that is of a traditional character and that has preserved a doctrine that could serve as an appropriate basis for the work in question, and this organization is the Catholic Church. It would be enough to restore to the doctrine of the Church, without changing anything of the religious form that it bears outwardly, the deeper meaning that is truly contained in it, but of which its present representatives seem to be unaware, just as they are unaware of its essential unity with the other traditional forms. \flright{\textsc{Rene Guenon}, \emph{The Crisis of the Modern World}}

\end{quotex}
\paragraph{Western Tradition}
It was Guenon's position that, at least for those at a sufficient level of development, the choice of an esoteric path was a matter of personal choice and opportunity. In \textbf{Boccacio}'s \emph{Decameron}, which Guenon regarded as the work of an initiate in the same Tradition as Dante, there is the story of the three rings. In that tale the true religion, between Judaism, Christianity, and Islam, was represented by a golden ring. In the tale, two identical copies were made making it impossible to tell which of the three was the true religion. There are three possible interpretations of the story.

Nevertheless, Guenon thought that the Catholic Church was the natural basis for the re-establishment of Tradition in the West. Obviously, not as it was constituted, but presuming that a spiritual elite might arise who could restore the deeper meaning contained therein. In several of his early works, he provided clues to what they may look like.

We have mentioned Guenon's writings on the Crown of Creation \footnote{\url{https://www.gornahoor.net/?p=7768}} as the basis for metaphysics and the Social Reign of Christ as the model for social organization. This is not the place for details, but it is certainly sufficient as a sound beginning for the restoration.

Certainly, it is a sound project to investigate the Medieval period as a model since all writers of Tradition regard that age as the most recent Traditional civilization in the West. Although Guenon saw its demise in the Renaissance, Evola referred to what ``every well born man considered sane and normal prior to 1789." The spiritual basis of that sane and normal time is irrefutable.

Guenon believed that the revival of Thomism following Pope Leo XIII's encyclical left out an important element. Specifically, Thomism became too dependent on Aristotle, while leaving out the more important influence of Neoplatonism, Augustine, and Dionysus. In this he was opposed by Jacques Maritain who even lobbied to get Guenon's books place on the Index while he was the French ambassador to the Vatican. Therein lies a clue to the possible restoration, which should be further enhanced with a dialog of more the more ancient Vedic teachings.

\paragraph{Adoption}
The three Western traditions are in continuity. We have pointed this out several times: Christianity is esoteric paganism\footnote{\url{https://www.gornahoor.net/?p=7152}}, Augustine referred to the one Tradition, Dante built on Virgil. More recently, \textbf{C S Lewis}, \textbf{Valentin Tomberg}, and \textbf{Joseph Ratzinger} regarded Christianity as the fulfillment, not only of the Hebrew religion, but also of paganism.

The idea of a new people founded by a God-man was not unknown at that time and would have colored the understanding of early Christians. In his studies of the spirituality of Ancient Rome, Evola focused exclusively on a given bloodline. However, birthright was not the only means to participate in a spiritual stream. A man who had no male heir was able to adopt one into his family. From the spiritual point of view, that served the same purpose.

For example, Romulus was the demigod son of Mars; this is not totally fictional since the martial spirit of Rome certainly existed, however you want to understand that. However, Jesus Christ was the Son of a virgin and of the Holy Spirit, hence, not a particular god of a certain spirit, but the Absolute God of all creation. This started a spiritual race, not of blood, but from adoption. We become sons of God by adoption. Sons share in the nature of their fathers and are entitled to an inheritance. This is much more than mere creaturehood. In our time, this notion has been sentimentalized to turn believers into perpetual children.

\paragraph{Initiation}
Guenon insisted that a valid initiation was necessary for an initiatic path; the validity was guaranteed by an unbroken chain of initiates. Certainly, this is not sufficient since Guenon himself pointed to several Hindus and Sufis who proclaimed heterodox doctrines despite their initiations.

Evola criticized Guenon in that regard for the bureaucratization of initiation. More important than the horizontal was the vertical, so Evola proposed that there was an initiation from above. He never really identified that. Tomberg, however, wrote that all initiation is from above and is the second birth in Christ.

Thus, in the West, Hermetic groups appear and go dormant, without a visible continuity. That is so it does not become as sort of counter-Church.

\paragraph{Buddhism}
\begin{quotex}
Guenon has made a commendable effort to interpret the true spirit of Hindu culture to the West in his many works … The form of regeneration consists not in a fusion or synthesis of the two cultures, but in the West regaining, as the result of a dynamic turn in its present trend, those springs of true spirituality through the help of the East. It would be hazardous to forecast the time of the change or the precise manner in which it would be brought about. \flright{\textsc{T R V Murti}, \emph{The Central Philosophy of Buddhism}}

\end{quotex}
However, about 20 years ago, I took Guenon's point very seriously. If it was true that there was no available initiation in the West, I searched for another, which came done to taste and opportunity. I had a romanticized vision of Tibet. There were the tales from Alexandra David-Neel, the Roerich's, not to mention the various claims of the occultists. Tibet had been the last patriarchal, hierarchical, theocratic society. Since there were a few Tibetan centers locally of different lineages, I decided to get initiated at one of them, in the tradition of the \textbf{Dalai Lama}.

So for a few years, I followed that path. It was not so different. There was a priest, an altar, flowers, incense, a rite, veneration of relics, icons, prayers, meditations, heaven, hell, and a period of post-mortem purgation. I followed the precepts of the vows, which were very much like the Tao described by C S Lewis. A minor transgression, or ``venial sin", could be rectified by suitable prayers. A ``mortal sin", such as stealing a freight train or having anal intercourse with your wife, put you outside the sangha and could only be rectified by retaking the vows.

Nevertheless, virtually every Westerner I met told me how they had surpassed the religion of their upbringing. From a practical point of view, I couldn't see much difference. Clearly, they were ignoring several points of the vow. Instead, becoming a vegetarian or rescuing a dog from the pound became the main focus.

I never heard anyone of them advocate the desirability of a patriarchal theocratic Buddhist state in the West. Even the Dalai Lama ended the reincarnations of future Dalai Lamas and came to prefer a Euro-style parliamentary system for any future free Tibet. Hence, it was time to leave.

Can it be said that the initiation really made any difference, beyond a post hoc, propter hoc argument? Certainly the many hours spent in meditation with mala beads and the study of Madhyamika must have accomplished something. At the very least, it was the ``help from the East" that Guenon claimed was necessary.

I am not recommending this choice, since I see that the same, or more, can be accomplished within the Western Tradition. Buddhism is a paganism and thus can be superseded while absorbing its best practices and metaphysical teachings. Then there is the matter of recognizing the real ring.



\flrightit{Posted on 2015-02-09 by Cologero }

\begin{center}* * *\end{center}

\begin{footnotesize}\begin{sffamily}



\texttt{Cologero on 2015-02-12 at 20:54 said: }

I appreciate your comments, Mr Obscure, but I need to keep making the same point over and over again. The Tradition is one thing, the current management is another. That is a distinction to always keep in mind, especially given Guenon's claim that the current leaders are not likely to be aware of the full Tradition. As I've tried to show , this is not an idle opinion … Guenon did indeed try to reinvigorate Tradition. There is also the model of Guido De Giorgio with his ``Vedantized Christianity" as Evola called it.

I agree with your points, but what I reject is that we have to wait for George to do it\footnote{\url{http://en.wikipedia.org/wiki/Let_George_Do_It_\%28radio\%29}}. We do not need more kibitzers and complainers, but men of action. This is George's motto:

\begin{quotex}\footnotesize
Personal notice: Danger's my stock in trade. If the job's too tough for you to handle, you've got a job for me. George Valentine. 

\end{quotex}
Is that clear enough?


\hfill

\texttt{Cologero on 2015-02-12 at 21:03 said: }

I'll iterate these points since new readers may not have encountered them yet.

Evola developed an interest in Idealism (it was the dominant Italian philosophy at the time), so he taught himself German so he could read all the German Idealists (and Nietzsche). From that, he created his own philosophical system while still a young man. He was not concerned about any ``suspicion" that his system would arouse.

A second point. Do you think that God chose Joan of Arc, a young woman, to save France? Not at all. First He went down the list but all the men, one after the other, refused the task. As He neared the bottom of his prospects, she was the only one not to refuse. I bet there is a lesson there.


\end{sffamily}\end{footnotesize}
