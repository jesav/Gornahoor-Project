\section{Donoso Cortes, Carl Schmitt, Bonald}

Nowadays, ``discussion" has become commodified, the saleable product of cable news and journals of opinion; there is no longer even the pretense of a ``search for truth". Learn the reasons why.

I don't believe that \textbf{Carl Schmitt}'s monograph on \textbf{Donoso Cortes}, mentioned by \textbf{Julius Evola}\footnote{\url{https://gornahoor.net/?p=4432}}, has been translated into English. However, the last chapter of Schmitt's \textit{Political Theology} is titled \textit{On the Counterrevolutionary Philosophy of the State (de Maistre, Bonald, Donoso Cortes)}, so it can serve as a summary to Schmitt's understanding of Donoso.

But first, we need to point out an interesting anomaly. The Catholic Schmitt, along with the three Catholic counterrevolutionaries, have had absolutely no impact on the contemporary church. Their ideas on Tradition, authority, and opposition to the modern liberal world were rejected by the modernist Vatican II church. So why was the anti-Christian Evola so entranced by the figure of Donoso Cortes? As Evola points out in his review of Schmitt's monograph on \textbf{Thomas Hobbes}\footnote{\url{https://gornahoor.net/?p=4444}}, what counts is the ``manifestation of a principle and a higher order" and the implementation of a ``type of truly spiritual and traditional hierarchical organization". Since there is no longer any institutional support for such ideas, it is the task of the few to understand and development them.

As we saw, Evola rejects Hobbes' claim of the social contract as the foundation of the Leviathan state, as that would serve to provide it with a veneer of legitimacy. Rather, it is the devolution of the Traditional state, the result of the loss of the sense of a transcendent truth, while still retaining its authority. Such a state, left to its own devices under the ``influences of a deeper elusive nature, always capable of infiltrating wherever they do not find the way barred by the presence of authentic principles and in a firm, abiding truth", will continue to devolve into the ``individualistic breakup of the State". So what does Donoso tell us about that liberal crackup and its opposite?

The liberal ideal is ``free inquiry", the ``marketplace of ideas", ``everlasting conversation". Schmitt writes:

\begin{quotex}
Catholic political philosophers such as de Maistre, Bonald, and Donoso Cortes … would have considered everlasting conversation a product of a gruesomely comic fantasy, for what characterized their counterrevolutionary political philosophy was the recognition that their times needed a decision. 

\end{quotex}
For Bonald tradition offered the sole possibility of gaining the content that man was capable of accepting metaphysically, because the intellect of the individual was considered too weak and wretched to be able to recognize truth by itself.

\begin{quotex}
The antitheses and distinctions that Bonald was so fond of contain in truth moral disjunctions… Such moral disjunctions represent contrasts between good and evil, God and the devil; between them an either/or exists in the sense of a life-and-death struggle that does not recognize a synthesis and a ``higher third." 

\end{quotex}
We see that the counterrevolutionaries reject the liberal ideal that every man's opinion counts. ``Error has no rights", and few men are capable of overcoming intellectual error. Donoso was even more emphatic. According to Schmitt: Donoso Cortes'

\begin{quotex}
Contempt for man knew no limits: Man's blind reason, his weak will, and the ridiculous vitality of his carnal longings appeared to him so pitiable that all words in every human language do not suffice to express the complete lowness of this creature. … The stupidity of the masses was just as apparent to him as was the silly vanity of their leaders. 

\end{quotex}
In essence, Donoso is describing a man at the ``first stage", who is moved ``by forces, ideas, and objects outside of himself. Lacking a will, the man in the first stage can do nothing without being directed by outside forces." Donoso also is contemptuous of bourgeois society. Schmitt writes:

\begin{quotex}
According to Donoso Cortes, it was characteristic of bourgeois liberalism not to decide in this battle but instead to begin a discussion. He straightforwardly defined the bourgeoisie as a ``discussing class". This definition contains the class characteristic of wanting to evade the decision. A class that shifts all political activity on the plane of conversation in the press and in parliament is no match for social conflict. 

\end{quotex}
How prescient was Donoso? Senators, TV ``journalists", etc., drone on about ``having a conversation", ``dialogue" and the right to ``free speech". Their goal is to win an argument, while the goal of the opposition is to win.

\paragraph{The Futility of Debate}
Here is the heart of the matter. The bourgeois class, although nominally in power, has no understanding of, nor even belief in, any principles of a transcendent order. Hence, there are no longer any convictions. With no grasp of true principles, there is no argument against the destructive forces of disorder. When the opposite of the truth is given equal footing with the truth, there can be no resolution. The debate will continue without resolution, like some eternal Highlander battle\footnote{\url{https://en.wikipedia.org/wiki/Highlander_(franchise)}}, but without the heroic elements. Instead of an either/or, there is now an and/or. The masses themselves have no way to resolve the issue, and hence consider themselves free to choose either alternative based on nothing but whim. Thus, the choice for disorder is just as valid as the choice for order.

If a child touches a hot stove, he gets immediate negative feedback and learns never to do it again. In the course of his life, a man will make many mistakes. Often, the consequences are not clearly experienced until years later when the entanglements he created become difficult to escape from. On the societal level, the full negative effects will not be noticed for many years, often exceeding even the span of a man's life. This is why they persist, to the intense bemusement of those who are still capable of seeing the true causes and consequences of events.

\paragraph{Decision}
The liberal, bourgeois state lacks the will to make a decision. Schmitt explains:

\begin{quotex}
Donoso Cortes considered continuous discussion a method of circumventing responsibility and of ascribing to freedom of speech and of the press an excessive importance that in the final analysis permits the decision to be evaded. Just as liberalism discusses and negotiates every political detail, so it also wants to dissolve metaphysical truth in a discussion. The essence of liberalism is negotiation, a cautious half measure, in the hope that the definitive dispute, the decisive bloody battle, can be transformed into a parliamentary debate and permit the decision to be suspended forever in an everlasting discussion. 

\end{quotex}
We see, however, that the negotiation keeps moving just in one direction. For example, suppose I offer \$50 for a product and the seller demands \$100. We negotiate for \$75. The seller then knows my limit. So the next time we negotiate, we start at \$75 and he demands \$125. If, through indecision, I lack the will to hold firm, you can see the price will continue to increase. Thus, social conflict continues to be resolved just in one direction, despite the intentions of conservatives to maintain the status quo, which, in any case, continues to move in the same direction.

The opposite of discussion is decision. In a Traditional society, the ultimate decision was made by the High Priest or the King. At that point, the decision was final and discussion ended. In the religious sphere, the spiritual authority is infallible; in the political sphere, the King laid down the law. In this view, conflict cannot be negotiated away and must be dealt with in a more primal way. This was always recognized. For example, \textbf{Dante} recognized the duel as the ultimate arbiter\footnote{\url{https://gornahoor.net/?p=3636}}. \textbf{Joseph de Maistre} praised the Executioner as the hidden force behind order. That is why the revolutionaries always oppose capital punishment, at least until they gain power themselves.

For Donoso, who lived at a time when Kings no longer held power other than in a nominal sense, the solution was a dictator, or a ``man of destiny". This is the logical conclusion. This is a step back, in the direction of counterrevolution, but it is still authority without truth or legitimacy. While not rejecting this notion, this is only

\begin{quotex}
the stage at which authority suffices and truth is superfluous; in which myths, and not true principles, are the best instrument to capture and organize collective forces; in which the miracle of an exceptional personality, of a ``man of destiny" saturated with the ``numinous", and not a pure ``divine right", founds a legitimate sovereignty and command and confers a transcendent character to the idea of the State. 

\end{quotex}
Ultimately, this stage must itself be surpassed:

\begin{quotex}
we will enter into a new phase in which the Leviathan, so to speak, will become the body formed to make possible the incarnation and the manifestation of a principle and a higher order: with that, the collectivistic and irrational aspect of the principle of totalitarianism and authority will be surpassed and will again implement a type of truly spiritual and traditional hierarchical organization. 

\end{quotex}


\flrightit{Posted on 2012-07-19 by Cologero }

\begin{center}* * *\end{center}

\begin{footnotesize}\begin{sffamily}



\texttt{logres on 2012-07-20 at 01:42 said: }

Almost all ``classical liberals" of the older, well-bred school were suspicious of democracy, yet today's group denominating themselves that have accepted it as indisputable, literally. It seems they have an either/or as well, in a twisted way.


\hfill

\texttt{logres on 2012-07-20 at 01:45 said: }

I also found this: there are first stirrings (in a quiet way) of the coming realizations:

\url{http://www.legends-and-myths.com/40\_1.cfm?f=22-legends-myths-initiation-romance-of-the-rose}

I am not even sure he is a ``traditionalist".


\hfill

\texttt{Andrew on 2012-07-20 at 04:27 said: }

Posts like this are the reason I read Gornahoor daily. Simply giving voice to these ideas makes Gornahoor one of the most valuable websites that exist. I cannot name one other blog that describes Tradition so consistently.


\hfill

\texttt{Avery on 2012-07-22 at 11:30 said: }

``For example, suppose I offer \$50 for a product and the seller demands \$100. We negotiate for \$75. The seller then knows my limit. So the next time we negotiate, we start at \$75 and he demands \$125."

A useful metaphor, since economic metaphors are easily understood these days. We can say that modernity is a seller's market. The buyer has no choice but to accept the constantly raising social ``price", or else walk away. And what's being sold? Power, reputation, fame… all the stuff that your friends with a foot in tradition can warn you about.


\end{sffamily}\end{footnotesize}
