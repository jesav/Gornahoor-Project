\section{Birth and Essence of the Modern Myth}

\begin{quotex}
This essay, Nascita ed essenza del ``mito" moderno by \textbf{Julius Evola}, originally appeared in the March 1936 issue of \textit{La Vita Italiana}, and will be published in five parts. There are a few points to think about in this first segment.
\begin{itemize}
\item 
Once again, Evola, following \textbf{Rene Guenon}, brings up caste as the most fundamental division of human types. Hence, caste is more fundamental than religion or race. 

\item
The degeneration of caste is the first step to understand the movements of history. The attempts to interpret history in terms of race or religion (e.g., pagan vs Christian) are all for naught unless the roles of the various castes are first taken into consideration. 

\item
The different castes have differing ways of thinking and being. Specifically, not every caste is capable of understanding metaphysical principles. That is why there are so many unsound interpretations of Tradition. 

\item
In a manner reminiscent of \textbf{Plato} in the Republic, Evola likens the social organization of the castes to the organization of human consciousness. 

\item
Myth and Symbol are objectively designed to represent metaphysical principles. Of the various contemporary movements\footnote{\url{https://gornahoor.net/?p=3853}} that claim to be based on Mythos, can any of them make such a claim? 

\item
Myth and Symbol are created by the spiritual elite for the benefit of the lower castes. It also provides a link between them, without which the elite would be isolated. 

\item
Evola brings up the notion of the elite as being the living law on Earth, and mentions Emperor Frederick II\footnote{\url{https://en.wikipedia.org/wiki/Frederick_II\%2C_Holy_Roman_Emperor}} and Dante in relation to it. Similarly, the Roman popes also claimed to be the final arbiter of the cosmic and natural law, a claim in full compliance with Traditional civilizations. Who today would still accept that claim either by the popes or by some other pretender to the spiritual authority? 

\end{itemize}\end{quotex}
For the correct orientation in the matter that we want to treat, we should begin from some general doctrinal premises that do not have for us the significance of personal philosophical construction, but rather that of a rigorously objective matter of fact. It is about the doctrine of the hierarchical quadripartition and the understanding of recent history as the process of descent from one to the other of these four hierarchical levels. To make such a view intuitive, let us stop here, first of all, at its social aspect. The quadripartition is therefore presented as that which in all traditional ancient civilization gave rise to four distinct and hierarchically ordered castes: serfs, bourgeoisie, warrior aristocracy, and the bearers of a pure spiritual authority. We do not mean by ``caste" something artificial and conventional, but rather the level that united individuals who have the same nature and vocation in common. In every caste a specific mode of existence took form and expression.

The hierarchy was normative when there was the natural dependency of the lower modes of life on the higher, i.e., to those adhering to a purely spiritual point of view and metaphysical reality. Only in such a case did they have the right relations of subordination and participation, that was analogous to the human organism. In the human organism, in fact, we do not have a normal and healthy condition, in those cases when the physical element, i.e., the servile stratum, or the vegetative life, i.e., the bourgeoisie, or finally, the uncontrolled and impulsive will, i.e., the warrior caste, assumes the direct and deciding part. Instead there is a normal condition when the spirit constitutes the central and ultimate point of reference for the other faculties, and this is why a partial autonomy is not denied to them, but that, on the contrary, they remain strengthened and transfigured in the totality of overall unity.

And this is effectively the image of existence corresponding to those primordial times that today in common parlance are called ``myths". Such times were not those of a half-animal or prepersonal life, but rather those of the greatest metaphysical extent. They were times, in which everything in man that transcends the humman is manifested in all its basic vehemence, similar to the free wind from the hills, as an absolute organizing power, as a force stronger than life and death. Superman or golden age, wild immanence or primordial theocracy, all those are only a pale formulae, polluted by the conceptions of a decadent epoch. Those times were superbly Olympic and heroic; a sole purpose animated every activity, freed it and organized it inexorably around a metaphysical axis. One cannot even speak here of religion. Religion is too little, it is a subsequent appearance, a thing already conditioned and already human. Religion, \emph{religio}, means reconnection. But the question of a reconnecting does not arise where a \emph{presence} exists, where man does not know the mutilation of individualism, where the ``superlife" is, directly or indirectly, the deepest vein and the justification of everything that is life.

We said: directly or indirectly. This distinction introduces us to the meaning that \emph{myth} and \emph{symbol} had in this primordial and normal type of civilization. Symbol and myth were not in the least fantastic creations, poetic images, or superstitious transpositions of confused naturalistic representations. Symbol and myth were instead ways of approximating and of participating in metaphysical reality, formulated according to rigorous and objective laws of analogy, by representatives of this same metaphysical reality for the lower strata of the traditional systems, whose Leaders they were. Symbols and myths included two aspects. The first was constituted by an image suited to produce a galvanization and dynamization of the imagination, through which the appearance of profound energies was produced, an inner shift of the psychic level. The second part of the myth was to give the right orientation, consciously or unconsciously, to this dynamized energy through a higher point of reference, as much to start presentiments, illuminations, or actions beyond everything that has form and that is materially and humanly conditioned. The etymological origin of the very term ``myth" corresponds to that.

\textbf{Rene Guenon} recently pointed out that the Greek word \emph{mythos} comes from the root \emph{my}, which is found in the Latin \emph{mutus}, mute, and in the verb \emph{myo}, to be silent, to keep one's mouth shut, but moreover in \emph{myeo}, which means to initiate, and specifically to initiate into the mysteries. In fact, the very term mystery, \emph{mysterion}, does not have a different origin and reveals the meaning of teaching silently, i.e., exceeding the limit of everything that is nameable, sensible, and tied to a form. Initiatic myths and poetic myths, heroic myths and physical myths, theological myths and myths descending down to the discipline of the most humble corporative community, were only applications of this sole principle to the various domains of knowing and acting. Myth therefore constitutes the articulation and metaphysical potentiality of every form of traditional life. It saw to it that knowledge and action would develop according to the meanings and possibilities that have been known for centuries, but are now systematically denied. But by virtue of myth and symbol, the necessary contacts were also established and strengthened so that those elites would not constitute an isolated and hidden vein, but the royal and solar vein of those who \emph{know} and who \emph{have being}, and who, as such, present themselves, all the way up to \textbf{Emperor Frederick II} and \textbf{Dante}, according to the traditional expression, as ``the living law on earth", \emph{lex animata in terris}.

\begin{quotex}
Here we see \textbf{Julius Evola} repeat familiar themes, even verbatim, but in today's climate, it is well to do so. But besides the outer events of the degeneration of castes, Evola points out how it corresponds to states of consciousness. Not only are the religious, political, and economic structures overturned, but new forms of consciousness arise. No longer aware of the supernatural or the third dimension of history, men become dominated by the subconscious: merely vital, pre-intellectual, and instinctive impulses.

Note, too, how Evola defends reason in an essay devoted to myth. As \textbf{Rene Guenon} points out, Mythos and Logos, properly understood, cannot be in conflict. But reason without the guidance of transcendent first principles is a boat without an anchor, drifting at random. Its use becomes instrumental, good for limited purposes or to achieve irrational ends, but incapable of revealing any higher truth. In an allusion to Kant, Evola notes how reason itself is used to diminish it.

Please pay attention to how Evola's and Guenon's conception of the decline into modernity differs radically from the New Right. Alain de Benoist famously rejects reason as inadequate to lead to truth. This merely reveals his inability to grasp the supra-rational and supernatural. His explanation is superficial, the work of a dilettante, and little more than piling up names and dates, in the hope of providing a comprehensive understanding. He correctly identifies modernity as a process of secularization, but presumes it to somehow be the logical outcome of the spiritual authority rather than its overthrow, as Evola asserts.

There is no organized Right along Traditional principles. Who in the West, today, is ready to renounce individualism and embrace obedience and subjugation to a spiritual authority? Yet that is what Evola calls for, and is comparable to what the Hermetist \textbf{Valentin Tomberg} says about obedience. Likewise, Tomberg recognizes the same cause of the modern world, pointing out that atheism and the denial of the supernatural are the marks of the lower castes. 

\end{quotex}
The comprehension of the history of culture as involution appears today as the repercussion of a period of crisis in the mind of certain dilettantes and literati. That does not deny that such a conception also corresponds to a knowledge found with great uniformity and impersonality in the traditional teachings of the most various peoples. As far as what interest us here, it is not an opinion, but a fact, to ascertain that authority and power were passed progressively into the hands of lower castes, especially in the West. Metaphysical and aristocratic-sacred types of State and civilization are thus taken over by monarchical-warrior States already secularized to a large measure. Later on, the true political directive function passes, under the cover of democratic-liberalism, to the capitalistic oligarchy, which corresponds to the ancient Third Estate and the ancient castes of the bourgeoisie and merchants. In the end, phenomena like marxism, communism, collectivism, and above all bolshevism in their original forms came to show the tendency of power passing to the lowest caste, the ancient castes of the serfs, which corresponds precisely to the mere mass.

This political aspect is naturally only the external and consequential side of an internal involution, for which each one of these four phases is accompanied by a corresponding change in the inner meaning of every institution, of every interest, of every ideal of knowing and acting. Thus not only for the State, but also for the concept of nation and family, ethics, law, architectonic types, speculation, art, the meaning of war, and so on one could be outline a phenomenology and notice the same change, the same rigorous regressive passage through the four stages:

\begin{enumerate}
\item metaphysical or solar 
\item warrior alone 
\item bourgeoisie and humanistic 
\item finally collectivistic and plebeian. 
\end{enumerate}
But, for our purposes, we are interested only in emphasizing the aspect, through this involution, that corresponds to a collapse of the personality, to a progressive failure of the inner tension that makes possible a supernatural organization of all the powers of the human being, and, finally, to a literal descent from the superconscious to an out-and-out subconscious. Man can be truly free and himself only when he maintains the center of his being on a metaphysical plane. When he detaches himself from such a plane and concentrates on practical goals, on temporal realizations and, in general, on whatever was the domain only of the lower castes taken in themselves, he abdicates, disintegrates himself, opens himself to hidden forces, whose instrument he will soon be destined to be unless he takes account of them.

And it is precisely this direction that modern enlightened persons have glorified as evolution. The point of rupture is constituted by individualism. With individualism, man constitutes himself as an illusory center outside the center, he considers as conquest what is only his shameful mutilation, and he believes he can substitute those traditional certainties, those ``non-human" principles from which he has now fallen, with constructions based on the actions of purely human and natural faculties. We can define this phase both as humanism and rationalism; it is of brief duration. Separated from every higher point of reference, reason cannot preserve its autonomy. Yet it, with an auto-corrosive critical labor, undertakes to refute the dogmatic and legislative claims of pure reason and to limit the validity of reason to an inferior use, i.e., practical, experiential, and utilitarian. In a subsequent moment the same weight of this diminished reason seems too heavy: in the forefront, the irrational and the anti-rational now emerge as the substance of every reality and produce a reversal: the guide to reason is now what is effectively inferior to reason, reason is now an instrument at the service of various impulses that can cover over political or religious, sentimental, or practical-activist forms, yet always retain a subpersonal character and that in principle are merely blind self-assertions.

All that today is the religion of life, faustianism, pragmatism, the new naturalism, the religion of immanence and pure becoming, doctrines of the unconscious and new mysticism, belong to this level: far from having the value of a reaction, all of that corresponds only to the last phase of a process of erosion.

The same development is verified in the social arena. The individualistic usurpation automatically invokes the collectivistic limitation. The Traditional man in his hierarchy and his caste could be free, i.e., himself: he had his law, his organic function, his dignity, whatever stratum he belonged to. Modern man, prideful of having swept away every caste and every true tradition, finds himself facing the mass of others without caste and without tradition. Even here the rationalist and humanist endeavor, i.e., the endeavor to arrange an unstable and chaotic mass of atoms by means of abstract juridical and political formulas. But even here the standard is not regained, nothing arises, for which obedience is also consent and subjection is also identification and elevation. Only unstable and tottering forms appear, against which the flood of new forces press to the end. These forces now belong less to the individual than to the masses, i.e., to what in individuals is the mass, to the purely vital, pre-intellectual, and instinctive part of their being. Atomism is outdated, and with it, too, everything that is liberalism and individualism galvanically settles into a new superindividual cement under the sign of a radical intolerance for every principle of a higher order and every true discipline. It is a new climate, saturated with enormous tensions, great forces in motion that no longer have a center.

\begin{quotex}
Far from eliminating myths, modernity, starting from the Enlightenment, introduced a new series of myths. Traditional myths were intended to lead men beyond words and thoughts, to a realization in the silence, the brilliant light of transcendence. Instead, the myths of modernity are grey; rather than awakening to superconscious reality, the modern myths can only arouse the subconscious and subrational forces. This is part III of V: 

\end{quotex}
For the symmetry existing between the superrational and the subrational, the myth gets a new life in this new collectivistic climate. As in primordial times, today the myth, more than the idea or the concept, goes from the directing center to the new forces. But its structure and function today, naturally, are totally different. The myth in recent times appears as a simple image deprived of the metaphysical and superrational potentiality that it had traditionally, exaggerated instead in its mere aspect of ``force-image", i.e., of an image that on one hand hypnotizes the higher faculty of the conscious personality of individuals, and on the other, dynamizes the energy belonging to the passionate, irrational, and subterranean part of their being. The modern myth corresponds precisely to the critical stage, in which the lucent peaks of the superworld were lost in the grey distance, in which the same dogmatic and sentimental surrogates proposed by positive religions\footnote{A religion with a historical founder.} have lost their strength, in which the mirages of rationalism are revealed in their fundamental inconsistency, yet an unbearable tension weighs on individuals; therefore an authoritative need of a discharge ecstatically freeing itself from the weight and limit constituted by their Self and exalting itself in great collective currents. The myth releases this function. It is an illusory center that permits the discharge of this tension because, so to say, it is substituted for the Self. Precisely saturating itself with an attractive power and becoming ``force-image", it acquires not only an appearance of truth and absolute validity, but in the end almost its own autonomous life, which often substituted for the personality of their evokers. In brief, we repeat greatly the same process through which, for the hypnotized, the image proposed by the hypnotizer becomes true, independently of every one of its contents and the Self of the hypnotized, his mind, his senses enter into a state of absolute passivity in the face of it, while from the corporal subconscious profound forces surface also capable of extra-normal effects.

If it is at this rate necessary to measure the inconceivable dynamic spread among the modern masses by the myths, the multitudes fall into a serious error when they believe that they have a sign of spiritual rebirth in the face of a manifestation of positive vital force. Once the energetic side is eliminated, it can be said without paradox that the substance of the modern myth is exactly that which the century of the Enlightenment supposed in the ancient myth, believing that the ancient myth is reduced to a fairy tale, fantastic images and personified abstractions, and that such a myth, which previously never existed, even today takes on life.

In fact, the domain of myth in modern times is quite far from being restricted only to the ethical and social fields. Science, Evolution, Positive Method, Historicity, Objectivity and so on, in the intellectual fields are in fact superstitious myths when Democracy, Liberty, Popular Sovereignty, Progress, Civilization, Race, Universal Peace, the New Age, the Worldwide Revolution, and all the other great substantialized words, written emphatically with capital letters, which have played so much a part in the social life of the last century and are the center of irrational crystallizations of very powerful currents. Therefore, for what concerns their internal content, modern myths should be called fairy tales rather than myths. Etymologically the term \emph{fable}, from \emph{fari}, indicates exactly a mere speaking, the word without a necessary reference to a content: while we saw that the term \emph{mythos} included in itself the idea of an abolition of the word and the idea of a silent and essential realization. On this basis, we can say that if there is an age of fairy tales, it is not that of the ancient mythologies, but the current age, in spite of its presumed positivism.

On the contrary, we can point out that even the so-called positive spirits are those who more ingenuously succumb to myths, accepting as truth ideas that are less than shaky hypotheses, instead are suggestions from the environment. The evolutionist myth, which one knows the part it has had in recent science, belongs to this type. The same can be said of the myth of the unconscious of recent psychology, of the myth of modern civilization as crown and end of every civilization in the domain of historiography, a myth that has devastated every right vision of our past, and so we could continue a long time. Naturally, we are ready to concede that there are types in the social area among some modern myths—for example, the myth of the super-race, the myth of empire, the myth of return to the origins, etc.—that would be susceptible to a higher content and meaning. But we point out that it is not in the least for this possibility that such myths today have value and operate, rather uniquely for their suggestive capacity, through their capacity to offer a point of support for the ecstatic liberation of the personality and for the discharge of irrational and subpersonal powers of the collective psyche.

Thus from a higher point of view, one myth is equivalent to another, there are not any among the modern myths that are susceptible to produce a true reconstruction or even only to serve as its base, since already the simple condition, by which any myth whatsoever can act on the individual, represents a negative and paralyzing condition. On the other hand it would be rather easy to show that even in the cases in which the modern myth is not reduced to simple words, outside of which there is no reality, and to pure hypnotizing images, the higher elements contained in this type of myth undergo a fundamental deformation, for which the concrete result is an action opposed to that which normally such elements should produce. Finally, we already pointed out that the analogy with the phenomena of suggestions, or the emergence of profound, sometimes materially supernormal, forces is the counterpart of the collapse of the personality and momentary paralysis of the conscious faculties, that this analogy permits us to judge, in accordance with their true significance, the extraordinary capacity of impetus, courage, enthusiasm, sacrifice and dedication which the masses often have demonstrated under the strength of myths. We noted that these capacities can deceive the superficial observer and make him believe that he sees a phenomenon of rebirth, of spiritual revival, and of new youth: while it is a question of the last apparitions in a cycle of decadence, of the hidden flood of spiritually destructive elementary forces after the fall of our hierarchical and traditional edifice, they are deprived of every true center and push men and things in the most unpredictable directions.

\begin{quotex}
In this segment, Evola speculates on how myths are created and then on the necessary steps to reverse the regression of the castes. The creators of myths must come from an elite, since the masses certainly cannot do so. Yet he points out the the ultimate source of myth must come from beyond the human, all too human, sphere, so these elite are really just the visible agents of these higher powers. Presumably, he includes both the modern myths, originating from demonic powers, as well as myths that lead to a true transcendence. This would seem to preclude the notion of myth as just a \emph{noble lie}, created by visible leaders to keep the masses enthralled.

The way out, then, is to mentally discard all the myths of the modern world. Freed from these illusions, the path back to the re-emergence of a warrior caste mentality is laid out. This will require a cold resoluteness of spirit, no longer subjected to the lower impulses of instinct and the subconscious. But this is as far as the natural man can go. The final stage of a true spiritual authority, will require an ``invisible action from above". Evola avoids details, as he must, since the spirit blows where it wills. All these higher men can do is prepare and dispose themselves for its reception. However, Evola does believe it will be a revival of the spirit of ancient Rome, the first Romanity. However, since he claims the modern world is the result of the regression from the Medieval spirit, it is difficult to see why the spirit would not appear as the second Romanity of the Middle Ages. We have in Dante, inter alia, a complete initiatory path which would serve the purpose of preparation. 

Since the final few paragraphs deal exclusively with the question of the ``elite", I decided to make it a separate entry.

\end{quotex}
All that that we have now said, nevertheless still leaves the problem of the true genesis of the myth unquestioned. It is certainly necessary that whoever, at a given moment, invented the myths, since we cannot admit that the masses are capable of the spontaneity necessary to shape and propose them and, moreover, in their origins, myths almost always introduce a rationalistic element, an ideological residue. In the second place, when we treat the type of social myths, it is always necessary for someone to directly or indirectly sustain and feed the collective fascination that makes the myth true and active, even if we must admit that, afterward, the myth often ends up assuming kind of its own life. Here, we can elaborate on such a complex and delicate problem since even evoking the image of the tamer, as someone did, leaves us on the exterior side of the process, does not enlighten us on the deepest conditions of its possibility. We will limit ourselves to say that, as we do not believe in a spontaneity of the crowds, so we do not even believe in a true autonomy of creators of myths and collectivistic directors, since these, in their turn, appear to us almost always conditioned by forces apparently directed by them and will seldom demonstrate the qualities of true spiritual leaders. Except for rare exceptions, those who press to the coming of the lowest social strata with their visions and their myths seem the initiators of the truth and of new currents, yet are only centers of crystallization and of diffuse collective states, to be considered, in their turn, less as determinate causes than as the results of more distant and less graspable influences.

To elaborate on the true genesis of modern myths and their power, it would therefore be necessary to ask oneself the general question of the true causes of modern decadence since its various elements are without doubt in agreement; it would be necessary to create a gaze capable of penetrating into what is hidden behind the apparent events; it would be necessary to have a foreboding of the secret, and not simply human, influences which, without taking them into account, the ``creators" of myths obey and which constitute the true basis of irrational attractive forces of ideologies formulated by them. It is useless to say that an investigation of the type sharply transcends both the possibilities and the methods of any sociology and applied psychoanalysis whatsoever; nor would it be to advise those who in a clear vision of the reality prefer confused hopes and disordered attempts of reaction.

From the complex view of history we exposed, it turns out, in every way, that the new mythological age corresponds to the last of the four phases of the process of the involution of the castes. The myths accompany and animate essentially the era of the servile caste, i.e., of the masses, even if with some connections with the preceding era of rationalism. So that it can be said without hesitation that even when some myths today present an aristocratic content, their true substance and their dynamic do not cease for this reason to be plebian. We also pointed out that the involutive phenomenon is not manifested only in a leveling, \emph{but also in an overturning of polarity}. With this overturning the inferior, i.e., the irrational, the vital, the collective, absorbs and directs the remaining superior elements of the human and social structure: forces of ``merchants", forces of ``warriors", forces of a new pseudo-aristocracy. Reaching the zone of the spirit, these currents from below project and exalt themselves precisely in the myth and so reach their highest potential. It is thus that our time appears to us all too saturated with mysticism, but this mysticism, let us repeat once again, has nothing of the supernatural, it is deprived of a true center, and the relative atmosphere is infinitely more dangerous, destructive, and paralyzing than that of a pure materialism or collectivism.

For this reason, if we should indicate the possible ways out of the present situation, we will say without hesitating that the first task is a purification of everything that today is called spirit, an absolute surpassing of all the contemporary forms of mythology and mysticism. Spirit today means fog, sensation, evasion, exaltation. Against that, it is necessary to battle for a new era of clarity, coldness, form. With all the means, it is necessary to lead man back to himself, block him from the insane need to believe and abandon himself, to reduce himself to an essential simplicity. It is necessary to reach a world in which there are only men and things and the pure, absolute connections between one and the other, without images, without feverishness, passionate leaps, words, ideologies, gestures. We will want to call this task the destruction of romanticism and idealism in the name of a new Doric and classical epoch, of a new and Roman dignity. As preliminary conditions for a rebirth, we do not see another way. In such a world, the weak will fall, but the strongest find themselves again. They will learn to remain upright without supports and make themselves active in the higher sense. Ceasing to be moved, they move themselves. This atmosphere of clarity will signify death for every myth. Resolute borders and hard discipline will block the paths of human interiority from the violence of the collective. Slowly, in a silent inner ascent, he will thus initiate the reintegration of the personality, and new forces, forces of an ascending new cycle, will then be able to guide knowledge and action. In fact at this point we will find ourselves already beyond the era of the masses and the era of the bourgeoisie, we should have a civilization and a style of life corresponding analogously to the visions and values of the second caste, i.e., of the caste of warriors and as such transcending also the plane of every humanistic and rationalistic scheme.

But in its turn, this world composed of virility and of form, this severely personalized, Doric, substantial world will become only a starting point. To bring this world back to a transcendent point of reference, to open it to metaphysical contacts in the sense not of an alteration, but of an integration and a transfiguration, is the ultimate task. However, the resolution of such a task remains problematic. In fact, the spirit does not let itself be constructed. There are no disciplines that lead to it and today it is not preset, as there was with the secret sacred content of the first, virile, Romanity. The action of every culture remains limited to a preparation and a disposition. But because the spirit is manifested as the absolutely organizing and animating power what, is needed is a type of invisible action from above.


\hfill

\begin{quotex}
In the conclusion to the essay, \textbf{Julius Evola} brings up the question of the future elite who will restore Tradition. This will take place in several steps. First of all, Evola tries to clear up the inevitable misunderstandings when people first here the word. He is not speaking of what are commonly considered the elite in the world. Evola singles out the so-called intellectuals who are active even today in formulating global policies in politics, finance, or war. The true elite will understand metaphysical doctrine.

The first task is to initiate a process of ``psychic detoxification", which will include the elimination of the myths that drive the modern world, which have already been described. The reintegration of the personality will reverse the lower impulses of the subconscious, putting them under the higher parts of one's being. A smaller group will work in private, developing their spiritual understanding. They will not, at this point, take part in any direct political action.

The transformation of this group in its depths will act as a catalyst, or perhaps leaven, in the environment, gradually infusing new transformative ideas. Once again, Evola makes clear that the will to carry out this project cannot be found on the human level, but only from a higher source. When the will is developed, the elite will make themselves known. True to form, Evola invokes the scepter, the regal sign, to identify the elite. Traditionally, it is the priest who is the bridge between the human and super-human worlds. We have to assume he means a Priest-King as existed in the early paganism of the Ancient City. The action of this priesthood will lead more men to metaphysical understanding, thus re-establishing a hierarchy. Note that ``wind" is also the symbol for the Spirit.

Obviously, Evola cannot provide the specifics, since he is not part of the elite. Tradition doesn't end with Guenon and Evola. Hence, it is now more important to be very clear about metaphysical doctrines and not get lost in the fog of idle speculations instead of the clarity of true knowledge. The silent paths must be trodden, even if covered over with growth. Once again, Evola points out that this elite will be intransigent, intolerant, without compromise, unlike the libertarianism of the New Right. After all, truth cannot compromise with error and still remain the truth. 

\end{quotex}
We now bring up the problem of the true elite. An elite of spiritual leaders, not of hypnotizers and arbitrary rulers of irrational collective forces, will be those who will be able to preside over a cycle of a still higher order, a metaphysical cycle. But in such regard, the possibilities of misunderstandings are endless. Today, in fact, we hear everywhere talk of the elite, and isn't there perhaps a dauber of printing paper of a certain rank who is presumed qualified to take part in it and thereby guide the endangered ship of Western civilization? So it is necessary to declare in the most pedantic way that it is not in the least about what today can be imagined and proposed as elite by most people, and less than ever is it about pale international group of intellectuals more or less in style, as rich in narcissistic and exhibitionistic complexes as poor in true doctrine. While on one hand in the widest strata of Western nations, we should foster a general job of psychic detoxification, of destruction of the myths, of reintegration of the personality and start toward a new active realism, and, on the other hand, a smaller group should make itself capable of treading the silent, secret paths of traditional metaphysical knowledge. Outside of any visible organizations and independently of any immediate goal, this group will enucleate those qualities, will prepare the forces, and will reach a new seriousness, a new depth, a new solar virility, in the face of which every art will signify vanity, every speculation a pointless game of impotent abstractions, every vulgar politician an obscure mechanism, every variety of frenzy of action, which recent times have produced, a deleterious opiate. Only on such a basis will the word \emph{nobility} reacquire in this elite its supreme and original significance, its hard, arid, absolute splendor.

The action of this elite, first of all, will be what the chemists call catalytic, i.e.:  \emph{a transformative action through its presence}. It will act as an invisible ferment and as the center of a crystallization of influences of a new type. The whole problem lies in seeing if these influences will succeed in saturating an environment, i.e., of compenetrating and orienting the new world of form, clarity, personality, and active realism in its deepest roots, preventing it from making itself the principle and end in itself. By that, we clearly mean something different from the passage from one ``vision of the world" to another, the fruitless turning of a sick insomniac from one side to another. We mean a change fulfilled in the blood, or rather more in one's depths than in the blood, as a remembrance, as an awakening, as a transcendent liveliness. The will to something higher will then arise, and will make itself absolute, even if it contains exteriorly rigid and tempered forms like steel, a will higher than anything things and men alone can ever produce. It is at this point that the elite will also manifest themselves visibly, they will take on with sturdy hands the direction of every force, they will be the bearer of a scepter. Bridges will be thrown beyond the humanly conditioned. New links will restore the action and knowledge of beings who live below their metaphysical potentiality. The strongest forces of life and death will turn to be present at the center, as the axis, flame, and light of new solar civilizations. An integral hierarchy will reestablish itself. The free wind from the heights will pass once again over new great currents of men of races.

These, naturally are liminal points, a bird's eye view—and no one should believe that with them the practical importance of reconstructive movements have risen against the final forms of collective decadence and demagogic mysticism—above all the fascist one—whether in the least disowned. But from a higher point of view, i.e., in the super-political sphere, in a world of fogs and uncertain flashes, as in the current one, only the pinnacle that towers over above contingencies and tempests can, in spite of their distance, provide the right orientation. In truth, only intolerance for every compromise and every limitation and the spiritual courage of absolute perspectives will distinguish the new group, the group of those who can bear witness to a spiritual vocation and to whom the tasks and the responsibility of the future reconstruction will be restored.


\hfill



\flrightit{Posted on 2012-05-16 by Aeneas }

\begin{center}* * *\end{center}

\begin{footnotesize}\begin{sffamily}



\texttt{logres on 2012-05-16 at 23:52 said: }

I think I've asked before, but would Old Testament prophets represent a movement out of religion back to the ``presence"?


\hfill

\texttt{Cologero on 2012-05-18 at 00:13 said: }

Good point to emphasize the `presence'. We can often forget, with the flood of ideas and doctrines, that Tradition is limited to that. The real point, however, is to reach a higher state of consciousness, beyond mere reason and beyond thoughts.

Nevertheless, Gornahoor is not a guru, ready, willing, or able to pronounce a judgment on any question that pops up; we are responsible only for our own texts. Readers are welcome to comment on the question if it so interests them.


\hfill

\texttt{Egemen on 2012-05-18 at 03:34 said: }

Thanks to this another review of the situation of ``present", filtered by the frame of ``the castes". Nowadays, I am wondering and so think about inner dynamics of the traditional castes. Main hierarchical structure of the four-folded castes can be seen also in layers of it, one by one. For example, when you look to the main picture of the ``Brahman caste", you can see some leaders as spiritual peaks (in arabic ``kutb"), and their disciplines, and followers of that dicsiplines and so on. Similar leaders, similar sub-rules, similar following masses etc. are also can be seen in other castes. There is a big question is growing in my mind and that is ``what is the nature and dynamics of being a leader in any castes?"…


\hfill

\texttt{Cologero on 2012-05-18 at 08:58 said: }

Progress: The two most serious sins, ignorance and pride, assert themselves in regard to so-called progress.

\flright{\textsc{Guido de Giorgio, Aforismi e Poesie}}


\hfill

\texttt{logres on 2012-05-18 at 23:13 said: }

\url{http://don-colacho.blogspot.com/}

Worth reading his aphorisms, as an antidote to Nietzsche, like the above.


\hfill

\texttt{Cologero on 2012-05-20 at 08:45 said: }

As a concrete example of the process described here, we can point to the \textit{Declaration of the Rights of Man}\footnote{\url{http://avalon.law.yale.edu/18th_century/rightsof.asp}}, arising from the French Revolution. Article 3 says:

\begin{quotex}
The principle of all sovereignty resides essentially in the nation. No body nor individual may exercise any authority which does not proceed directly from the nation.

\end{quotex}
This explicitly excludes the idea of a spiritual authority above and independent of the state.

That the French Revolution was a bourgeois, and not a proletarian, revolution is revealed in the difference between the ``active citizen" and ``passive citizen". The rights of man applied only to active citizens: they must be male tax payers not of the servant class.


\hfill

\texttt{logres on 2012-05-21 at 08:48 said: }

This really makes Tomberg's connection with Evola crystal clear – thank you. It also bespeaks Evola's ``Catholic" sense of grace in a metaphysical context.


\hfill

\texttt{Cologero on 2012-05-21 at 12:18 said: }

It is interesting, although I would proceed with caution. I think part of the similarity may be due to their mutual background in Hermetism, certainly not from a direct connection. What Evola wrote sure does sound like grace, but I doubt he would approve of that word.


\hfill

\texttt{Charlotte on 2012-05-22 at 09:09 said: }

From the little I've read of Evola, his insight into using the Left Hand Path as a means of dealing with the constraints of the Kali Yuga is certainly intriguing – at least, it enables one to `swallow' the humiliation that in some respects seems necessary. Perhaps this is what Tomberg also knew because otherwise I don't see massive similarities between the two authors – I see far more between Guenon and Tomberg, but perhaps I do not know Evola well enough yet to notice any great depth of comradeship between them – certainly Tomberg is not stuck in time….

I also can't help feeling that Evola is somewhat blinded by his own exalted situation in life – could he POSSIBLY be a Roman aristocrat!?!

For example, this statement here – `it is necessary to battle for a new era of clarity, coldness, form’ – whatever one's opinion of its veracity, far more clearly points to classical Greece rather than Rome, the empire of which was notable for its extremes of decadence and decay. Fair enough Dante was Italian, perhaps with all the princely sentiment that entails, but the Renaissance was really one of the classical Greek era not the Roman – of course everyone knows this, but it still seems necessary to point it out in this context.

Personally I feel that a lot of his material has passed its time now, the Spirit is fresh and the `masses' (how archaic!) who live on the land, wherever they're allowed to tend to be more in touch with the natural elan than their self-professed superiors, who are hampered by inflated egos, the wrong sort of living and generally too much `knowing' of the kind that comes from books. 

Spiritual regeneration comes from the simple virtues of faith, hope and love, the triumph of the heart centre. This is something Dante knew well, as he would have accomplised nothing without Beatrice. Sure enough she enlivened his mind, but first she had to set his heart on fire.


\hfill

\texttt{Cologero on 2012-05-22 at 12:57 said: }

You seem to be missing his point, Charlotte. First of all, he lived a monkish life with few possessions. The attitude he described was that of Rome, not of Greece. The former was not always decadent and decayed. The very notion implies it had to start from a higher point. A more apt analogy would be the UK, which used to have the ``stiff upper lip", but is now decadent, atheistic, crime ridden, and hopeless for a large segment of the youth.

The Renassaince was the restoration of Greek culture, but Evola regarded it as an ersatz version.

In the Force arcana, Tomberg writes: It is terrestrial electricity that we make use of in the technical field of our civilisation but also in hypnosis, in demagogic propaganda, in movements of revolutionary masses … for electrical energy has its analogous forms on various planes — physical, psychic, and even mental.

You see, Evola and Tomberg has the same fundamental worldview, whatever the difference in details. Each of those elements mentioned by Tomberg are also found in Evola, as even a cursory reading of the most recent translation will reveal.

Tomberg also writes: The Reformation, rationalism, the French revolution, materialistic faith of the nineteenth century, and the Bolshevik revolution, show that everywhere mankind is turning away from the Virgin.

Once again, did not Evola make the identical list of those forces opposing spiritual renewal?

Tomberg writes: ``Napoleon understood the direction which Europe had taken–the direction towards the complete destruction of hierarchy." Tomberg also wrote that Napoleon should have ruled by the sceptre rather than the sword. Did not Evola say that the sceptre would be the sign of the return of the spiritual elite?

Tomberg describes the French revolution, the communist revolution, and nationalism as perverse collective intoxications. Did not Evola use similar language? Do you now understand that a stage of detoxification is the necessary first step?

Where did Evola say that you have to read more books? The ``knowing" he is talking about is identical to the gnosis of Tomberg.


\hfill

\texttt{Charlotte on 2012-05-22 at 18:02 said: }

well it's a matter of opinion I guess, and I have only just started reading Evola so can't pretend to be an expert. However it seems to me that he focuses far more on class and notions of elitism than I ever heard Tomberg openly express. But we live now in different times.

I agree with their assessment of the French revolution and I would be the last person to advocate blind anarchy – whenever I speak to revolutionaries, and believe me I know plenty, I urge them to consider carefully the consequences o their ideas and actions. Never mind the first step, I will say, it's the step you have to take once power is assumed that ALWAYS causes the problems…. 

I think the difficulty with the attitude you (and Evola) advocate nowadays is that we're in a situation where it's very difficult to trust those in charge – who I suppose are the `elites'. the ones with money at any rate. We may or may not be past the stage of a `peasants' revolt'. but I think it is perfectly justifiable for ordinary people nowadays to question the motives of their `superiors' (especially political) and to wonder whether the people who've been appointed to govern are in fact up to the task. It seems to me that the ones who truly have the qualifications to rule – on the moral scale in particular – lack the desire to go into the governing machine, which in pretty much every country is corrupt at some level. And yes, since Tony Blair's era – he is the classic example of someone who is apparently a member of the `elite' but in fact is the antithesis of such a word in real terms – England went well on its way to the dogs. This is the whole point. Those with the moral fibre necessary to really good judgement – the true followers of the Virgin – want no part to play in the charade that is global politics.

Of course I understand that true `elites' are not the politicians, they are the ones who possess the gifts of the Holy Spirit, healing, prophecy and so on. However, there is a world beyond that of formation, and sooner or later the ideas must be put into practice. How shall this be done, who shall `bite the bullet'. as it were? Universal salvation – something the virgin would certainly wish for us – entails saving one and all, not just the `better ones'. My concern nowadays is that practical solutions are required – we have all been thinking now for aeons and find ourselves irrevolcably on Earth for at least the duration of the present life. Correct `tactics' at an occult level are indeed necessary to understand – and this is something I think both Evola and Guenon can help with, maybe even more than Tomberg in some ways, at a certain junction, but I still feel that the former's understanding of `elite' is outmoded. This is in part a problem of time – those who came before don't have the advantage of our hindsight!


\hfill

\texttt{Charlotte on 2012-05-22 at 18:18 said: }

on the subject of the Virgin, I don't think that people really are turning away, in fact I think that ordinary people are embracing her more than ever, just not necessarily in a way that fits with the Roman Catholic mold. I think this is something you may have to consider. Fair enough it may be manifesting (at the moment) as a more primitive form of `goddess worship'. but the basic principle of turning to the divine feminine is getting stronger by the day. I agree there can be a lack of understanding with respect to what `divine' and `feminine’ – plus the two together – REALLY mean, but all the same I think we are going in the right direction. 

Even Tomberg had to admit that the `true Church' is a spiritual body – we know this; there has to be freedom.


\hfill

\texttt{Cologero on 2012-05-22 at 19:59 said: }

In the Arcana of the Emperor, Tomberg does indeed advocate the precise same hierarchy that Evola endorses. These are not in the least ``matters of opinion", these are the fruits of gnosis. Tomberg writes:

\begin{quotex}
A pyramid is not complete without its summit; a hierarchy does not exist when it is incomplete. Without an Emperor, there will be, sooner or later, no more kings. When there are no kings, there will be, sooner or later, no more nobility. When there is no more nobility, there will be, sooner or later, no more bourgeoisie or peasants. This is how one arrives at the dictatorship of the proletariat, the class hostile to the hierarchical principle, which is the \textbf{reflection of divine order}. That is why the proletariat professes atheism. 

\end{quotex}
This has nothing to do with social class, an artificial construct, which has never been mentioned on this blog. What Evola and Tomberg are describing is the Divine Order, the reflection of the supernatural on earth. These castes are not arbitrary, the are due to the inherent differences in people, and their respective roles, duties, or dharma, in their incarnation. It is not a question of someone getting an extra privilege; how can that be conclusion be reached from anything written here?

Of course we are in different times; that is the entire point! Tomberg says the West is getting old, in other words, it is at the end of an age. A few retrograde movements you describe are no more than small eddies in the downward current, of little ultimate import. There is no right direction, since the direction now is towards dying of old age, rejecting legitimate hierarchy, being blind to the supernatural, and following any number of false movements because they make us ``feel good". The only direction is a radical reversal, a new beginning.

Power is not the problem; we have to look beyond our immediate experience. Those in power have the proletarian mindset, a mass mind, that wants to force entire populations into spiritual darkness by eliminating any and all distinctions, including the most fundamental one, that of male and female. But there is no point in my repeating what has already been written here over and over. Arguing opinions is futile; it is really a question of learning to ``see", to see in the depths, and not be misled by appearances. Some few will begin to ``see"; the future will belong to them. Everyone else are floating flotsam and jetsam, carried along in the currents of the modern world, whatever illusions they privately hold to make that trip tolerable.


\hfill

\texttt{Cologero on 2012-05-22 at 20:00 said: }

Freedom is not a given; it needs to be earned.


\hfill

\texttt{Charlotte on 2012-05-23 at 03:49 said: }

Oh I can see it alright! And I have always thought that a `divine kingship' is the ideal form of government. My protests centre on the fact that the hierarchy has succumbed to total corruption, those in charge either did not prove worthy, or have been usurped and power seized by the wrong people. Get to the top of the earthly pyramid – the manifestation of the hierarchy, and I can guarantee you wouldn't like what's there. The war in heaven has produced unimaginable horrors here on earth – who is watching the watchers, who can reach past their veils?

Apart from that, my issue all along has been one of solutions – it's alright pointing out the problem but what is to be DONE? high ideals are fine, but when you've been witness to real horror and comprehended the real and present problems of the world, like starving children and millions of sex slaves, it sheds rather a different light on things, the rose-tinted glasses no longer cut it. The fact also remains that most countries have done away with their royalty and won't ever go back to them. The question for `spiritual elites' is one of how to alleviate the burden of suffering for people here on Earth – this is why we come down from the Holy mountain, in order to help others. This is why the Buddha, king of overcoming karma, preached compassion – united we stand and divided we fall, as we are ALL sparks from God, not just the few at the top,be they priests, artists or politicians.

At the very end of the spiritual exercises Tomberg warns us most stringently that the biggest test of all for the `mage' is to go back down to Earth after reaching the pinnacle – it is the ultimate test of one's Spirit over one's ego, for surely it's a better view from the top of the pile!! it's SO MUCH NICER to be directing the world of formation and clairvoyantly determining the course of evolution than trying to implement change in the realm of creation. I don't underestimate the trials of this, of course, having battled with them for decades, but I know it takes much greater courage to risk everything in order to see the way it looks from the lowly sheep pen. And in any case, who here is NOT part of `the masses'. I don't imagine any of us are driving round in rolls royces or are free from the long arm of the law, even if we're lucky enough to have divine protection in other spheres? Got to be realistic about our actual influence in this realm, although a more organised system would reap more tangible results.

I also seem to remember Tomberg speaking of how these hierarchies will, at some stage, necessarily have to move into the spiritual realm precisely because of the very problems I speak of. In particular the office of Pope. The Emperor is a spiritual fact, but where does he really reside? Not in the Roman region, I know that much, he isn't the head of any earthly scheme.

As for freedom, you're wrong about that one, it is our sovereign gift from God and a terrible sin for one human being to remove the freedoms of another. To say that some people do not deserve to be free but are born to be slaves of others is devilish – can you imagine the place where they kept the slaves on the coast of Africa? You know there is still a metre high tide mark inside the building where the excrement level stained the walls – so called human `elites' kept other human beings standing in a metre of shit for days on end. Try telling me that the slave masters are somehow superior and I'll fight you until one of us dies.


\hfill

\texttt{Charlotte on 2012-05-23 at 10:36 said: }

So, in the interim a senior tutor from my alma mater dropped by to pass on a message, which I dutifully convey to you now:

Christian Hermeticism itself can only be knowledge of the universal which is revealed in the particular. For Hermeticism there are no `principles’, ‘laws' and `ideas' which exist outside of individual beings, not as structural traits of their nature, but as entities separated and independent from it.

For Hermeticism there is neither a `law of gravitation' nor a `law of reincarnation'. there is only the attraction and repulsion of beings (atoms are beings also) in so far as gravitation is concerned, and only the attraction of beings to earthly life, with its joys and sorrows, in so far as reincarnation is concerned.

Laws are immanent in beings, as logic is immanent in thought, being part of the very nature of thought. And true progress, true evolution, is the advance of beings from life under one law to life under another law….it is thus that the law `an eye for an eye and a tooth for a tooth' is in the process of gradually being replaced by the law of forgiveness. It is thus again that the law `the weak serve the strong, the people serve the king, the disciple serves the master' will one day give way to the law shown by the Master through the act of the Washing of the Feet.

According to this higher law, it is the strong who serve the weak, the king who serves the people, the master who serves the disciple – just as in heaven, where Angels serve human beings, Archangels serve Angels and men, Principalities serve Archangels, Angels and human beings, and so on. And God? He serves all beings without exception.

Thus the `law' of the struggle for existence that Darwin observed in the domain of biology will one day cede its place to the law of cooperation for existence which exists already in the cooperation of flowering plants and bees, in the cooperation of different cells in an organism, and in cooperation in the human social organism. The end of the `law' of the struggle for existence and the future triumph of the law of cooperation for life has been foretold by the prophet Isaiah:

The wolf shall dwell with the lamb,

And the leopard shall lie down with the kid,

And the calf and the lion and the fatling together,

And a little child shall lead them.

This will be, because the new `law’ – ie, a profound change in the psychic and physical structure of beings – will replace the old `law'. firstly in consciousness, then in desires and affections, then lastly in the organic structure of beings.

`Law' succeed one another and change. They are not immutable metaphysical entities. It is the same wiht respect to `principles' and `ideas'. The sabbath was made for man, not man for the sabbath; so the Son of Man is lord even of the sabbath (Mark ii, 27:28) – here is the relationship between beings, on the one side, and laws, principles and ideas, on the other.

Meditations on the Tarot, Letter IX, The Hermit

So now we have reached the time of change and of the returning Spirit. I am called upon to point out that the `masses’ – the sheep, the children – are NOT naturally atheists, but have had their faith, hope and love assaulted by those in whom they placed their supreme trust. In much the same way as an abused child may become hardened in his or her heart, or a kicked dog might want to bite, have the `plebians' revolted against tyrannical overlords. The tower falls.

Furthermore, those selfsame plebians could hardly have carried out their revolutionary acts without effective leadership from the ranks of the `elite’ – the Illuminati in the case of the French Revolution, or Thule society in the case of the Nazi party, for example. It might also be pointed out that the kingship of France was in itself held by usurpers at the time of the revolution, the Merovingians having been sold out to the Franks centuries before. At some stage the cycle of closed circles – the materialism of the mages – must be broken in order for those within to be free to follow the true Master.


\hfill

\texttt{Cologero on 2012-05-23 at 23:54 said: }

We have already address this here, Charlotte: Ideas and Occult History:

\begin{quotex}
We need to see the idea in its relationship to other ideas horizontally and also vertically in its relationship to higher and more general ideas. These ideas do not exist on their own in some ethereal realm. … ``The bearer of an idea … must be a person." 

\end{quotex}
So you can't use Tomberg to refute what has already been expressed here. A fortiori, you can't rely on Tomberg to make some point, while ignoring what directly contradicts you. If you have a beef, it is with Tomberg who claimed the ``proletariat professes atheism". You have also ignored what he wrote about the need for hierarchy. Please tell us how deluded he is.

I don't care if you agree or disagree with anything Tomberg or Guenon wrote, but misunderstanding is not the same as disagreement. You persist in using the term ``elite" in your own way, rather than as it has been defined; try to be fair in your discussions. The whole point, which you are apparently missing, is that the ``elite" who led the French or Bolshevik revolutions were themselves from the masses. How can you possibly believe some spiritual elite would institute the Terror, the massacres at Vendee, the Gulags? If the masses are so naturally God loving, then how can they go along with this.

Being a ``leader" is not the same thing as being an ``elite". First of all, the elite, or ``elect", are actually chosen from above. A ``leader" is chosen by the people. Of course, the plebeians carried the acts under a leadership; that is the whole point of the occult war. The other point is that they did not resist. Did not the masses choose Barabbas over Jesus?

Charlotte, I don't understand your perpetual need to quibble, oppose, and debate, when a conversation would be more pleasant and useful.


\hfill

\texttt{Charlotte on 2012-05-24 at 05:51 said: }

well to my mind there was a very valid point to be made and as far as I can tell Tomberg DOES refute it quite clearly, perhaps Evola and Guenon are sometimes contradictory or have left loose threads? To me this seems very likely. Almost every occultist leaves loose threads and Tomberg is valued so highly because he has done such an awesome job of completion. 

The work of Tomberg is the most complex – it reaches further in every direction – than any of the other authors being discussed. I assume part of the point of Gornahoor is to draw correspondences between him, Guenon and Evola, but he can't be used to bolster everything they write, even if they do concur in places. 

If nobody else sees the point I'm making about slavery then so be it – and I really do NOT agree that freedom is something that should be earned, I know it was granted by God to all human beings. The elite who made possible the mind set of French Revolution and Nazi movement were philosophers – certainly Hitler would not have been possible without the Thule. I completely and utterly refute the idea that the masses are atheistic, I think this is an appalling misrepresentation of people who over the years have been abused by overlords yet for the most part manage to display a truer sort of faith, based on love not mind. The heart is all in matters of God. 

And yes, the `elect' are chosen from above, but they do not necessarily occupy earthly bodies – this is another fact Tomberg at least hints at. Yes the masses chose Barabbas but it was the priestly caste of Jerusalem who incited them to do so and who ensured that Jesus did not have a `free run'. as it were. He came to overturn their money tables and to correct them, to rid them of their corruption, their worship of caesar and money. It was the decision of the Roman emperor to commit genocide against the Jews, not his foot soldiers, they simply followed orders. 

You see the fish rots from the head down and there's no denying it. The masses – the peasantry – are (or at least were) devoted to their superiors and piously in awe of the hierarchy that they truly DID believe was divinely ordained. Humble men and women, for all their roughness and shortcomings DID love their kings and queens with a passion and would die for them. Of course when the lords, ladies, kings and queens become so utterly corrupted that even the most simple-minded pig-herder can see it, then something has to be done. THe former cannot be relied upon to do the right thing, so what happens next? People get unhappy, they feel that the smoke of satan has entered the divine order and they feel compelled to uproot it.

Others with an agenda see that there is unrest – powerful enough for an uprising – and they use that energy to take the power for themselves. And so human history goes round and round – the will to power pollutes the entire pyramid scheme again and again and again. In the end, however, cream rises and it is at least possible that a `good king and queen' might somehow reign again – but they'll have to prove their worth and credentials. If they turn out to be legitimate then it will work. If not then people will have to keep on waiting.

As for conversations, well there are important points at stake and I'm not one to let false doctrines sail by unchallenged – I could do for my own popularity's sake, I know how men prefer to be `managed'. but this isn't the place for that. I've seen MANY heated debates on this website – between yourself and Exit for instance – so I don't see myself as going overboard. One has a right to an opinion after all and we're all out of the first school – maybe we can all learn something from each other…


\hfill


\end{sffamily}\end{footnotesize}
