\section{Ideas and Hidden History}

One of the problems philosophers have long debated concerns the ontological status of the Platonic world of ideas. Aristotle proposed a solution that is fruitful as far as it goes, although it does not go far enough. His solution was that the idea is immanent in the object, a doctrine called hylomorphism. That is, the object is not material (hyle) only, but is also informed by the idea (morphe). Hence, when we see a material object, we know it by grasping its form.

Metaphysical insights from \textbf{Benedict Spinoza} and \textbf{Vladimir Solovyov} may show a way to rethink the problem. For Spinoza, ultimate reality is Substance, beyond space and time. Although infinite and beyond direct experience, Substance is known by two of its modes: \textbf{Extension} and \textbf{Thought}. These modes are appearances. Ultimately, the \textbf{Logos} is the source of ideas.

\paragraph{Extension}
Extension refers to things in Space and Time. Hylomorphism, therefore, accounts for our knowledge of ideas in Space and Time. So, when I look at a banana tree, I experience its material aspects through the senses, while judging it to be a banana tree through the Intellect. Discussions of this sort usually stop at the spatial aspect. However, a deeper seeing will unveil the temporal aspects. I then understand the banana tree to be a living being, unfolding in time. I can see the processes of germination, growth, and fruition. A real seer, or rishi, will see even deeper into the interiority of the plant, its nutritive soul.

\paragraph{Thought}


Since thought is temporal, but not spatial, we can grasp ideas in the mind apart from any objects in space. The ideas form higher and higher interconnected systems. Hence, grasping an idea in isolation falls short of understanding its full reality. We need to see the idea in its relationship to other ideas horizontally and also vertically in its relationship to higher and more general ideas. These ideas do not exist on their own only in some ethereal realm. An idea becomes manifested in a person.

\paragraph{Logos}
The Logos is beyond space and time, i.e., it is part of formless manifestation. Relating this back to \textbf{Rene Guenon}, we would say the possibilities [ideas] of manifestation become manifest in the modes of Extension or Thought. The possibilities of non-manifestation, or non-being, are ideas in the mind of God, since the Logos is the Thought of the Father.

\paragraph{Hidden History}
So now we have a clue to the hidden aspects of history. In the material world of Extension, we need to see beyond Space into Time. This brings us into Thought. From the ideas arising from sense experience, we need to dig deeper and understand it in both in its broader and more general aspects. In this way, we move beyond outward appearances to understand seemingly contingent and unrelated events as manifestations of a larger system. Ultimately, it must relate back to the Logos itself.

Solovyov insists that Ideas are forces, so Action follows Thought. Hence, by understanding Thought, we will understand the world. Nothing happens by accident, but rather as the result of a deliberate act. The primary causal agent is always an idea become person, with a will that pursues determinate ends.

Traditional teachings considers the agent to be the work of angels or devas, since so-called natural forces have personalities. As an aside, we should point out that \textbf{Valentin Tomberg} asserts the same thing, If nothing else, this leads credence to Solovyov's claim that Hermetic philosophy is the foundation of Christian theology.



\flrightit{Posted on 2011-11-23 by Cologero }

\begin{center}* * *\end{center}

\begin{footnotesize}\begin{sffamily}



\texttt{charlotte on 2011-11-24 at 07:37 said: }

I like Origen's theory of universal salvation, which instinctively seems true to me and ties in with Buddhist and Islamic ideas as well. BUt I heard Origen became a self-imposed Eunuch because the sexual demons were too great for him to muster. I wonder how much of a problem this is for other people with spiritual callings….


\hfill

\texttt{charlotte on 2011-11-24 at 07:38 said: }

master not muster, Freudian slip [D83D?][DE42?]


\hfill

\texttt{Perennial on 2011-12-02 at 02:35 said: }

Good question, Charlotte. I note St. Anthony of Egypt and St Augustine as being among those who had a long struggle with sexual demons. It seems to crop up in the lives of certain types of saints.


\end{sffamily}\end{footnotesize}
