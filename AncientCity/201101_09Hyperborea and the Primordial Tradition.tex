\section{Hyperborea and the Primordial Tradition}

\begin{quotex}
In general, the men whom pride makes melancholy, always discontented with the present, always uncertain of the future, love to reflect upon the past from which they believe they have nothing to fear; they adorn it with smiling colours which their imagination dares not give to the future. They prefer, in their somber melancholy, superfluous regrets without fatigue to real desires which would cost them some efforts. \flright{\textsc{Fabre d'Olivet}}

\end{quotex}
\paragraph{Introduction}
There exists an egalitarianism of thought, whose presumption is that human consciousnesses are fundamentally alike, both across time and also across the different strata within society. But this is irreconcilable with some fundamental metaphysical principles\footnote{\url{https://www.gornahoor.net/?p=823}}: the doctrine of cosmic cycles presumes a difference over time (diachronic) and the doctrine of castes presumes a difference within time (synchronic).

Previous writers on Tradition have not always drawn out fully the consequences of these doctrines. Their view has been static and treated the various traditions as though they were mushrooms arising spontaneously after the rain. The consequence has been that ``tradition" often devolves to a type of comparative religion with debates about which tradition is the ``best" or more ``suitable", and so on, based on little more than contingent historical events or polemical ``sound bites". Instead, these doctrines should throw light on why certain forms appeared where and when they did and what is the relationship to each other. The Western or Roman Tradition is the evolution of the Hyperborean Primordial Tradition\footnote{\url{https://www.gornahoor.net/library/UltimaThule.pdf}} (understood as the unfolding in time due to the forces of cosmic cycles). Thus, we trace the development from the North to the East to the South and to the West. Then, towards the end of a complete cycle, the full effects of the Kali Yuga being to manifest. There are reasons for all this.

\paragraph{Diremption and Man}
The appearance of man brought about a diremption in the cosmos. This split Heaven from Earth, with Man as the reconciling force. Initially, the consciousness of man had direct intuitive knowledge of both Heaven and Earth; the soul, or psyche, was experienced as a life force or elan vital, rather than the complex of thoughts, feelings, likes, dislikes, images and so on, that trap man in subjective isolation.

\paragraph{Primordial State}
In contrast, the Hyperboreans in the Primordial State would have lived in purity (undivided mind and will) in the light of the sun followed with long nights interspersed with a dim light on the horizons where the sun is struggling to rise or set.

There is abundance to satisfy all needs, the Hyperborean knows when he is satiated. The mind is free of fear, worry, anxiety. Death is experienced, not as destruction, but rather as a transformation to a different state.

There is an intuitive awareness of God and the universal order by clairvoyant vision, just as we now recognize a tree or a mountain by direct experience. Thinking is regarded as another sense, along with sight, hearing, touch or taste. Discursive thought — of the ``yes" and the ``no", of the ``good" and the ``evil" — is not present. Instead, thoughts are experienced as voices from the celestial hierarchies (gods or angels) or communications from ancestors or as commands from rulers.

Knowledge is passed on by myths, symbols, and rites, in poetic forms.

Temporal power is exercised without force and obeyed willingly; the structure of society is held together by bonds of loyalty, family, and love. Commands are experienced, not as an imposition from the outside, but rather as the direct revelation of spiritual authority in one's own consciousness.


\hfill

Recommended reading

\begin{itemize}
\item Anonymous, \emph{Meditations on the Tarot} 
\item Antoine Fabre d'Olivet \emph{Hermeneutic Interpretation of the Origin of the Social State of Man} 
\item Arthur Branwen, \emph{Ultima Thule: Julius Evola e Herman Wirth} 
\item B G Tilak, \emph{The Arctic Home in the Vedas} 
\item Julian Jaynes, \emph{The Origin of Consciousness in the Breakdown of the Bicameral Mind} 
\item Rene Guenon, \emph{The Great Triad} 
\item Rudolf Steiner, \emph{Spiritual Beings in the Heavenly Bodies \& in the Kingdoms of Nature} 
\end{itemize}


\flrightit{Posted on 2011-01-09 by Cologero }

\begin{center}* * *\end{center}

\begin{footnotesize}\begin{sffamily}



\texttt{James O'Meara on 2011-01-09 at 21:36 said: }

``Previous writers on Tradition have not always drawn out fully the consequences of these doctrines. Their view has been static and treated the various traditions as though they were mushrooms arising spontaneously after the rain. … Instead, these doctrines should throw light on why certain forms appeared where and when they did and what is the relationship to each other."

Oddly enough, I was re-reading Revolt Against the Modern World this weekend, and it occurred to me that this is exactly what makes Evola distinctive. Part One constructs a model of ``Tradition" derived from a survey of world traditions, using a method of synthesis [``the Traditional Method"] which I've previously compared to such methods as [here] to Spengler's ``physiognomic tact" and [on my blog] Clive Bell's artistic sense. Part Two then surveys actual [well, according to him] historical developments, identifying various `cycles.' as Tradition interact, declines, recoups, etc. 

By contrast, Schuon is the anti-Evola, taking each tradition as a `mushroom' or a-synchronic whole [``The Christian", ``The Taoist" etc.]. H. Smith took this model and calcified it further, if possible, into a cookie-cutter eso/exo model for all world religions, taken as such with no analysis at all.

This is exactly what made Evola so interesting, both theoretically as well as in practice [his attempts to influence contemporary events], after coming from reading Guenon etc., who by contrast seem like armchair theorists. 

Schuon take everything today as given, and accounts for change or conflict by the all-purpose `divine adaption.' Evola, and even Guenon, are able to dig beneath the surface and explain, say, the Roman tradition as an ongoing reconquest of telluric influences by a Nordic cycle etc.


\hfill

\texttt{Mark on 2011-01-10 at 16:00 said: }

I will post here to comment on Kadambari's comment that Putin has declared Russia an Orthodox country. The reality is that the adoption of Christianity by the Northern peoples has caused a rupture in their spirit. Their pagan imagination has a tendency to want to go to deeper myths.

\url{http://onlinelibrary.wiley.com/doi/10.1111/j.1469-8129.2008.00329.x/abstract}

``ABSTRACT. As in all post-Soviet states, the Russian intelligentsia has been preoccupied with the construction of a new national identity since the beginning of the 1990s. Although the place of Orthodox religion in Russia is well documented, the subject of neo-paganism and its consequent assertion of an Aryan identity for Russians remains little known. Yet specialists observing the political and intellectual life of contemporary Russia have begun to notice that the development of references to `Slavic paganism' and to Russia's `Aryan' origin can be found in the public speeches of some politicians and intellectual figures. This article will attempt, in its first section, to depict the historical depth of these movements by examining the existence of neo-pagan and/or Aryan referents in Soviet culture, and focusing on how these discourses developed in different spheres of post-Soviet Russian society, such as those of religion, historiography, and politics."

\url{http://www.springerlink.com/content/j30056125185344h/}

Perun's revenge: Understanding theduxovnaja kul'tura

The one thing that it seems that Kadambari has not been able to appreciate is the fact that a Christian conscious and a ``Pagan" unconscious has affected the people of the North in a different way than the people of the South.

It has interested me as to why the synthesis of Paganism and Christianity has caused a rupturing in the person of the North to the point where if we take someone from Scandinavian descent, if they take their religion seriously, then they become either a fundamentalist Protestant or Asatru, whereas this does not arise as much in the Greek or Italian.


\hfill

\texttt{Cologero on 2011-01-10 at 18:47 said: }

That is an interesting point, Mark. I have a post in mind that explores that very topic, though I've delayed it because it is inconclusive. Perhaps, I'll post it anyway, after a little preparation.


\hfill


\end{sffamily}\end{footnotesize}
