\section{Lightning Strikes from the East}

\begin{quotex}
Ignorance pure and simple is far preferable to false ideas. \flright{\textsc{Rene Guenon}}

Lightning cometh out of the east, and appeareth even into the west. \flright{\textsc{Jesus Christ}}

\end{quotex}
In \textit{East and West}, \textbf{Rene Guenon} outlines a plan for the recovery of Tradition in the West. Obviously, the plan is complex and requires some unique skills as well as intellectual depth to succeed, and it is something he doesn't expect to happen next week. Nevertheless, it is well worth elucidating in detail, if only for the boons it will bring to those who make the effort to understand. However, we take it seriously as we heed Guenon's warning\footnote{\url{https://www.gornahoor.net/?p=35}} about the consequences of not undergoing a consciously led social transformation. These are either a fall into complete barbarism or else the assimilation to another Traditional culture, most likely following a period of social upheaval. The plan must take place on three levels:

\begin{enumerate}
\item The knowledge and understanding of metaphysical principles 
\item The recovery of traditional sciences 
\item The transformation of the social order 
\end{enumerate}
\paragraph{Return to Intellectuality}
First of all, it is necessary to understand the goal, or what exactly defines a traditional civilization. Guenon explains:

\begin{quotex}
What we call a traditional civilization is one that is based on principles in the true sense of the word, that is, one where the intellectual realm dominates all the others, and where all things, science and social institutions alike, proceed from it directly or indirectly, being no more than contingent, secondary and subordinate applications of the purely intellectual truths.

\end{quotex}
To be blunt, the intellectual truths he is talking about are not debating points; they are either understood or they are not. The thinking, willing, and feeling functions must be understood in their proper relation. This is especially true in our age with its emphasis on action for its own sake and sentimentality. Guenon writes:

\begin{quotex}
The pure idea has no immediate relation with the domain of action, and it cannot have the direct influence on outward things that sentiment has; but the idea is, none the less, the principle, the necessary starting point of all things, without which they would be robbed of any sound basis. Sentiment, if it is not guided and controlled by the idea, brings forth nothing but error, disorder, and obscurity; there is no question of doing away with sentiment, but of keeping it within its legitimate bounds.

\end{quotex}
The dominance of sentimentality is expected in the modern mind, but it seems to be none the less present even among some segments seeking a return to tradition. So we must insist on this initial understanding, as Guenon explains:

\begin{quotex}
Thus a return to tradition and a return to principles are in reality just one and the same thing; but clearly the knowledge of the principles, where it is lost, must first be restored before there can be even a remote thought of applying them; it is quite out of the question to build up again a traditional civilization in all its fullness without first having the supreme and fundamental knowledge that must preside over the work.

\end{quotex}
\paragraph{Return to Tradition}
Since the time Guenon's book was published, much has changed in the West. On the one hand, the knowledge of Eastern systems such as the various schools and sects of Hinduism as well as Buddhism, has become much more prevalent, even to the point of attracting many adherents. On the other, there are the many alternative, or New Age, types of spiritualities, primarily deriving from Theosophy and similar movements. The latter consist of a strange mix of not fully understood principles heavily infused with mushy sentimentality. The former, although they show some promise, are too often incompatible with the Western mindset and style of life. The adherents either adopt alien styles of dress and mannerisms which are quite secondary to the intellectual principles or else end up accommodating the Eastern teachings to Western prejudices and, in the process, compromising what is authentic in them. That is why Guenon suggests:

\begin{quotex}
In our eyes, the most satisfactory solution for the West is the return to her own tradition, completed where necessary as to the domain of pure intellectuality.

\end{quotex}
Now, this requires some explanation since the Western mind is so imbued with egalitarianism, even among those most loudly calling for a return to Tradition. This domain of intellectuality is only suitable for the small minority, or elect, who are capable of understanding. For the rest, there needs to be an exoteric religion that bears these principles in a more palatable form. Hence, we witness the many discussions, often heated, about a return to Christianity or some sort of neopagan revival. That debate is not of interest to the elect who understand that metaphysics and religion need not be in the least incompatible. Guenon writes:

\begin{quotex}
Even if the West repudiates sentimentalism (and we mean by that the predominance of feeling over intelligence), the western masses will retain none the less a need for satisfactions derived from sentiment, which the religious form alone can give them, just as they will retain a need for outward activity.

\end{quotex}
There are certainly some manifestations of Western thought that are of a high intellectual caliber. Unfortunately, they are ignored by contemporary leaders of the Western religion who falsely believe they are in continuity with that tradition. Or else, they are depreciated and rejected by others because the Western tradition is ``\emph{too high for them}", as Guenon writes. The latter go searching in vain for an elusive ``authentic" tradition, when they can't even understand the one that is closest in time to them. We will bring what Guenon has to say on this topic at a later date.

\flrightit{Posted on 2012-01-17 by Cologero }
