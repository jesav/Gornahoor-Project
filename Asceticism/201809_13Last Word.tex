\section{Last Word}

The next task is also the last task and therefore the most difficult one. Four things are necessary to fulfil it.

\paragraph{Live by Principles}
Research the principles of \textbf{Bernard of Clairvaux} for a good life and live by them.

That has been your task from the start. Everybody forgot. Do not forget that amongst these principles most important one is to pledge your life and everything that one owns to the cause, defence, honour and further knowledge of the Christian religion and the Holy Church. That means both Catholic and Orthodox Church each with their own dogmas and respective holy centres and patrons. Neither defamation nor furthering of the schism belongs to the life one has been expected to lead. This task also does not have anything to do with common modern day politics and enmities, daily life and journalism, as such are regarded to be realities of a profane nature, so one should relinquish any attempt to conjoin your task with latter. Such is permittable only in one case, that of utmost necessity and solely that — indeed the same you have been called up for — defence, honour and further knowledge of the Christian religion and the Holy Church.

\paragraph{Basic Training}
To be able to live up to these principles, you need to acquire a basic training.

The first steps are: the control of thinking, feeling, willing, control of speech, control of action, all these combined. Not in a 10 minute meditation, not in an exercise, one is also not expected to perfect this as everybody comes to a situation sooner or later where one fails this and shows weakness and doubt. In such situations remember also that Our Lord suffered in Gethsemane, find strength in remembering that nevertheless he completed his Mission in the end for all of us. Try then to accomplish what is asked from this training in situations when it is needed and important.

Correct thought, correct feeling, correct word, correct action.

\paragraph{Mutual Respect}
Observance and Excellence, Education and Culture, Ethics and mutual Respect of other Confessions, as well as always when necessary and possible, mediating a peaceful dialogue between them; furthermore, abiding all written and unwritten laws and mores, national and International — is what is asked from all who decide to follow this path. To achieve these, one should school oneself also in the art of oratory — as a proper expression — so that a man can be considered wise and truthful whether with friends or in the face of adversary, and at both tables he will be respected and well received.

\paragraph{Service}
All these you must not do to excel and take pride in your achievements when you compare yourself with your neighbour. You do all this to serve. Christ washed the feet of his disciples and he was the First among all. Love your neighbour\footnote{Updated 13 IX 2020.

This is the last task assigned to me by a dear spiritual friend. As for the materials sent to me, I have been using them, not always explicitly. Those who are supposed to manage them, can't manage.

But I need to improve German vocabulary so I don't get slowed down looking up words.

As for oratory, I have new audio equipment. But it seems a lot of work for little reach. I'm not Joe Rogan.}.



\flrightit{Posted on 2018-09-13 by Anon }

\begin{center}* * *\end{center}

\begin{footnotesize}\begin{sffamily}



\texttt{Cologero on 2018-09-13 at 22:18 said: }

Very moving, but we have to take into account human capacities. For example, you alluded to some exercises suggested by Rudolf Steiner. As he points out, however, even 10 minutes may be too strenuous for most people. In \emph{Knowledge of Higher Worlds}, he makes this suggestion:

\begin{quotex}
The student must set aside a small part of his daily life in which to concern himself with something quite different from the objects of his daily occupation. The way, also, in which he occupies himself at such a time must differ entirely from the way in which he performs the rest of his daily duties. But this does not mean that what he does in the time thus set apart has no connection with his daily work. On the contrary, he will soon find that just these secluded moments, when sought in the right way, give him full power to perform his daily task[s]. Nor must it be supposed that the observance of this rule will really encroach upon the time needed for the performance of his duties. Should anyone really have no more time at his disposal, five minutes a day will suffice. It all depends on the manner in which these five minutes are spent. 

\end{quotex}

\hfill

\texttt{Anon on 2018-09-14 at 03:19 said: }

When you finish the material I sent you, tell me what you think about then about this same matter. Who cant manage, cant manage. Things are how they are.


\hfill

\texttt{Cologero on 2018-09-14 at 12:22 said: }

Well, Anon, if I have to finish the material first then you have not yet said the Last Word.

The goal is to help serve others (i.e., those who read this blog), as you wrote, not to display one's own superiority.

So let's discuss Steiner's exercise from a book that you recommended to me as part of the “material”. Why do suppose that 5 minutes do not suffice?

The material also includes this passage:

\begin{quotex}
Pay attention to your ideas. Only think important thoughts. Learn to separate in your thoughts the essential from the inessential, the eternal from the transient, the truth from mere opinion.

When listening or talking to one's fellow man, one tries to become completely silent within oneself and to give up all approval, in particular all disparaging judgments, both in thoughts and feelings. 

\end{quotex}
Does that mean we should withhold disparaging judgments, at least until we are sure we understand?

How does one separate the eternal from the transient? Certainly comparing the human condition from 800 years ago to day might be a prelude to that.

When we consider correct speech, the circumstances need to be taken into consideration. What is appropriate in one setting may not be in another. And when is it necessary to speak out? Would it not be correct speech to point out crimes and injustice? Silence is Consent: \textit{Qui tacet consentire videtur, ubi loqui debuit ac potuit}.

You may have forgotten Point 5: forgiveness. Things are not simply how they appear, as you imply. A person does not acquire all those qualities on a balmy weekend. It is the task of a lifetime, and there will be failures on the way, many failures. Forgiveness ought to follow faults, but it seems to be much easier to err than to forgive.


\end{sffamily}\end{footnotesize}
