\section{The Persona and Ego}

\begin{quotex}
In the beginning the world was nothing but the Atman, in the form of a man. It looked around and saw nothing different to itself. Then it cried out once, `It is I.' That is how the word `I' came to be. That is why even at the present day, if any one is called, he answers, `It is I,' and then recalls his other name, the one he bears. \flright{\textsc{Brihadâranyata-Upanishad}}

\end{quotex}
In \emph{The I Problem and Genius}, \textbf{Otto Weininger} writes about the realization of the sense of the ``I", that is, the experience of being an independent centre of awareness. Here are some descriptions he provides:

There has been no famous man who, at least some time in the course of his life, and generally earlier in proportion to his greatness, has not had a moment in which he was absolutely convinced of the possession of an ego in the highest sense. Let us compare the following utterances of three very great geniuses. \textbf{Jean Paul} relates in his autobiographical sketch, \emph{Truths from my own Life}:

\begin{quotex}
I can never forget a circumstance which, so far, has been related by no one – the birth of my own self-consciousness, the time and place of which I can tell. One morning I was standing, as a very young child, at the front door, and looking towards the wood-shed I suddenly saw, all at once my inner likeness. `I' am `I' flashed like lightning from the skies across me, and since then has remained. I saw myself then for the first time and for ever. This cannot be explained as a confusion of memory, for no alien narrative could have blended itself with this sacred event, preserved permanently in my memory by its vividness and novelty. 

\end{quotex}
\textbf{Novalis}, in his \emph{Miscellaneous Fragments}, refers to an identical experience:

\begin{quotex}
This factor every one must experience for himself. It is a factor of the higher order, and reveals itself only to higher men; but men should strive to induce it in themselves. Philosophy is the exercise of this factor, it is a true self-revelation, the stimulation of the real ego by the ideal ego. It is the foundation of all other revelations; the resolution to philosophise is a challenge to the actual ego, to become conscious of itself, to grow and to become a soul. 

\end{quotex}
\textbf{Schelling} discusses the same phenomenon in his \emph{Philosophical Letters upon Dogmatism and Criticism}, a little known early work, in which occurs the following beautiful words:

\begin{quotex}
In all of us there dwells a secret marvelous power of freeing ourselves from the changes of time, of withdrawing to our secret selves away from external things, and of so discovering to ourselves the eternal in us in the form of unchangeability. This presentation of ourselves to ourselves is the most truly personal experience upon which depends everything that we know of the supra-sensual world. This presentation shows us for the first time what real existence is, whilst all else only appears to be. It differs from every presentation of the sense in its perfect freedom, whilst all other presentations are bound, being overweighted by the burden of the object. Still there exists for those who have not this perfect freedom of the inner sense some approach to it, experiences approaching it from which they may gain some faint idea of it. … This intellectual presentation occurs when we cease to be our own object, when, withdrawing into ourselves, the perceiving self merges in the self-perceived. At that moment we annihilate time and duration of time; we are no longer in time, but time, or rather eternity itself, is in us. The external world is no longer an object for us, but is lost in us. 

\end{quotex}
Finally,

\begin{quotex}
Every great man knows this phase of the ego. He may become conscious of it first through the love of a woman, for the great man loves more intensely than the ordinary man; or it may be from the contrast given by a sense of guilt or the knowledge of having failed; these, too, the great man feels more intensely than smaller-minded people. It may lead him to a sense of unity with the all, to the seeing of all things in God, or, and this is more likely, it may reveal to him the frightful dualism of nature and spirit in the universe, and produce in him the need, the craving, for a solution of it, for the secret inner wonder. But always it leads the great man to the beginning of a presentation of the world for himself and by himself, without the help of the thought of others. 

\end{quotex}
\textbf{Miguel Serrano} has his own take on this in \emph{Nos: Book of the Resurrection}.

\begin{quotex}
Where is this persona when the child still has no sense of the individual ``ego"? In my case, I remember, when I was a year old or perhaps less, I was leaning out of a tower holding my grandfather's ring tightly in my hand. The women of the house ran to take hold of me, because they were afraid that I would let it drop. But, I remember, that child felt itself to be a persona, it knew the importance of the ring and knew that it would never let it drop. It felt deeply offended by this lack of trust. That child was a very old and wise man. And when the ``ego" became defined, it was a philosopher who asked himself the question. That is the difference, I believe … and this is the ring. I have recovered it. 

\end{quotex}


\flrightit{Posted on 2018-05-31 by Cologero }
