\section{The Bernard Option}

\begin{quotex}
And Raphael said to Tobias: As soon as thou shalt come into thy house, forthwith adore the Lord thy God: and giving thanks to him, go to thy father, and kiss him. And immediately anoint his eyes with this gall of the fish, which thou carriest with thee. For be assured that his eyes shall be presently opened, and thy father shall see the light of heaven, and shall rejoice in the sight of thee. Then the dog, which had been with them in the way, ran before, and coming as if he had brought the news, shewed his joy by his fawning and wagging his tail. And his father that was blind, rising up, began to run stumbling with his feet: and giving a servant his hand, went to meet his son. And receiving him kissed him, as did also his wife, and they began to weep for joy. And when they had adored God, and given him thanks, they sat down together. Then Tobias taking of the gall of the fish, anointed his father's eyes. And he stayed about half an hour: and a white skin began to come out of his eyes, like the skin of an egg. And Tobias took hold of it, and drew it from his eyes, and \textbf{immediately he recovered his sight. And they glorified God, both he and his wife and all that knew them}. And Tobias said: I bless thee, O Lord God of Israel, because thou hast chastised me, and thou hast saved me: and behold I see Tobias my son.\flright{\textit{Book of Tobit}}

\end{quotex}
\paragraph{Madhyamika}
To be influenced by Madhyamika does not simply involve the use of Tibetan and Pali terms. It is not necessary to don saffron robes and smile all the time. Rather, we adhere to the Rosicrucian principle of blending in with one's country of residence: adopt the local dress, customs, language, and even religion. The Rosicrucians were the scattered remnants of the Knights Templar. They had learned the lesson of becoming too publicly successful.

So Madhyamika needs to blend in. Exoteric Buddhists won't effect much. Here are common themes.

Points of view are relative, conventional, and factitious, little more than a change of view affects the feelings. Depending on the thought system, this may be a feeling of well-being, an agitation of the incensive function, sexual arousal, and so on. That is what brings conviction, seldom the intellectual content itself.

The self is impermanent, artificial creation. Actually, careful inner observation will reveal multiple selves vying for control, each a pretender to be the Real Self. Yet the Real Self cannot be found among those competing egos.

Suffering ends with the end of desire not the satisfaction of desire.

All appearances are transitory and illusionary superimpositions of reality created by fallen man. These superimpositions are also called confabulations. When detachment from the world of becoming becomes deeper, and more permanent, the glamour of the world no longer maintains such a strong attraction. And then a man learns to discern the true causes of things.

If there is no rational system, then a man must become clever, i.e., he must use skillful means. He needs to pick and choose, from here and there, by trial and error, to find the way out. If he is fortunate, he may find another man willing to give him personal instruction to get him started.

\paragraph{Get a Life}
I've heard it said recently that Boomers spend their money on things, while Millennials prefer to spend it on experiences. In other words, the focus of life can be on material things or on exciting or unusual experiences. Those are both traps, because both things and experiences are fleeting.

A step higher — open to those with sufficient intelligence — is to chase ideas. That, too, becomes a trap when the real appeal is the emotions arising from expounding, defending, arguing such ideas.

Few are those who follow the life of virtue, to become a Sage or a Saint.

\paragraph{The Death of Ivan Ilych}
Ivan Ilych was a nice guy. He provided a home for his shrewish wife. He performed his job punctiliously. He strove to fit in with his social groups. All, then, should have gone well in his life. Unfortunately, a freak accident had long term consequences. His seemingly minor wound should have healed quickly, but instead he became sicker. Initially, he was in denial. In the end, he came to accept his imminent demise.

So why should Europe die? Haven't Europeans created the science, political systems, technology, art, music, sports that have so enchanted the entire world? Isn't Europe like the nice guy, with his hand out and his wallet open? Perhaps a mistake here and there, but aren't the consequences all out of proportion to the results? Spengler showed that culture rise and die. So why is anyone surprised?

\paragraph{Utopias and Dystopias}
The Medievals did not write Utopias because, from their point of view, their societies were already built on the divine plan. Valentin Tomberg explains in detail:

\begin{quotex}
As it is suitable that the institution of a city or a kingdom be made according to the model of the institution of the world, similarly it is necessary to draw from divine government the order \emph{(ratio)} of the government of a city— this is the fundamental thesis advanced on this subject by St. Thomas Aquinas\emph{ (De regno} xiv,1). This is why authors of the Middle Ages could not imagine Christianity without an Emperor, just as they could not imagine the Universal Church without a pope.

Because if the world is governed hierarchically, Christianity or the \emph{Sanctum Imperium} cannot be otherwise. Hierarchy is a pyramid which exists only when it is complete. And it is the Emperor who is at its summit. Then come the kings, dukes, noblemen, citizens and peasants. But it is the crown of the Emperor which confers royalty to the royaJ crowns from which the ducal crowns and all the other crowns in turn derive their authority. 

\end{quotex}
This is not to say that the Middle Ages were perfect materially and politically. They had a better understanding of human nature then: fallen men would always have political conflicts. There is nothing in that model that was opposed to technological progress. Certainly not insofar as it concerned genuine goods: healthcare, dentistry, food production, sanitary conditions, and so on.

The Renaissance era, on the other hand, focused instead on material and political progress. Thinkers at that time envisioned that scientific discoveries and new political arrangement would usher in a period of human perfection. Hence, Renaissance thinkers wrote utopian novels such as \textbf{Francis Bacon}'s \emph{New Atlantis}, \textbf{Tommaso Campanella}'s \emph{City of the Sun}, and, of course, \textbf{Thomas More}'s \emph{Utopia}. Eventually, activists began acting out utopian ideas, so there arose the Jacobins and later the Marxists.

Our age, having rejected Medieval hierarchy and having seen the results of the utopians, has lost hope. Therefore, our age specialises in dystopian novels. The most well-known are perhaps \textbf{Aldous Huxley}'s \emph{Brave New World}, \textbf{George Orwell}'s \emph{Nineteen Eighty-Four}, \textbf{Robert Hugh Benson}'s \emph{Lord of the World}, or \textbf{Anthony Burgess}' \emph{The Wanting Seed}.

Contemporary dystopias are often portrayed in films, the latter-day equivalent of stained glass for the masses. Of course, in Hollywood films, there is usually a happy ending in which a small band of youths, especially women, mange to restore civilisation. There are three possible outcomes:

\begin{itemize}
\item Accept the death of civilisation with dignity 
\item Continue on the Kali Yuga path deeper into undifferentiated matriarchy 
\item A small elite will continue civilisation in a different form 
\item A more vigorous people will take over what remains of the decaying civilisation 
\end{itemize}
\paragraph{The Bernard Option}
I would like to propose the ``Bernard Option". \textbf{St. Bernard of Clairvaux} created the Rule of the Knights Templar. This required the fusion of two seemingly opposed paths: that of the warrior and that of the ascetic. Actually, they are the two options of the right hand path or pillar.

The new Knights Templar will be ascetics. They will have developed skills in the martial arts. In keeping with Tradition, the new Knights will be hierarchically ordered.

However, the Bernard Option goes further. The original Templars had the support of their society, but the new Knights are counter-cultural. So the creation of a new Order within the system is no longer possible.

\flrightit{Posted on 2017-06-14 by Cologero }
