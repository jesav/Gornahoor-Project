\section{Observe your thoughts}

It appears that Julius Evola was unclear about what Guenon called “intellectuality”, and could only see in Guenon a form of “rationalism”. 
I suspect quite a few readers of this blog are equally unclear. 
As a pedagogic aid, rather than focusing on the words, try to observe what is described directly. 
Ideas, or possibilities, arise in the mind. 
Observe how they arise, such as certain thoughts in similar circumstances. 
See how one leads to another similar to a meshed chain. 
Observe if they provoke emotional reactions; if so, that is an indication of bondage. 
This applies also to ideas that “pump you up”; it is still an addiction.

Observe which thoughts are real possibilities of manifestation, and which are idle fantasies or wishful thinking, that could never manifest in any possible world. 
Of those few good thoughts, determine which of them will lead directly to action which, after all, means to bring the Potential into Act.
With some serious self-reflection, honestly evaluate if you are living up to your true potential or selling yourself short. 
This is a painstaking process, requiring the courage to face up to the hellish aspects of one’s being. 
When conscious attention is lacking, the mind is taken over by telluric and lower forces of disorder. 
Through many efforts, a man can bring his mind closer to a state of order. 
But first he must be perfectly clear, through study and instruction, about what constitutes order and disorder. 
Guenon refers to the “Christ principle”, or Logos, as the force of order\footnote{From \url{https://www.gornahoor.net/?p=4527}.}.