\section{Observe your thoughts}

It appears that Julius Evola was unclear about what Guenon called “intellectuality”, and could only see in Guenon a form of “rationalism”. 
I suspect quite a few readers of this blog are equally unclear. 
As a pedagogic aid, rather than focusing on the words, try to observe what is described directly. 
Ideas, or possibilities, arise in the mind. 
Observe how they arise, such as certain thoughts in similar circumstances. 
See how one leads to another similar to a meshed chain. 
Observe if they provoke emotional reactions; if so, that is an indication of bondage. 
This applies also to ideas that “pump you up”; it is still an addiction.

Observe which thoughts are real possibilities of manifestation, and which are idle fantasies or wishful thinking, that could never manifest in any possible world. 
Of those few good thoughts, determine which of them will lead directly to action which, after all, means to bring the Potential into Act.
With some serious self-reflection, honestly evaluate if you are living up to your true potential or selling yourself short. 
This is a painstaking process, requiring the courage to face up to the hellish aspects of one’s being. 
When conscious attention is lacking, the mind is taken over by telluric and lower forces of disorder. 
Through many efforts, a man can bring his mind closer to a state of order. 
But first he must be perfectly clear, through study and instruction, about what constitutes order and disorder. 
Guenon refers to the “Christ principle”, or Logos, as the force of order\footnote{From \url{https://www.gornahoor.net/?p=4527}.}.

\section{Double-mindedness}

On the individual level, the first temptation of sinister forces is doubt — double-mindedness. Like every negative thought, detach yourself from the thought, find an anchor point, and observe it. THERE IS NO INTELLECTUAL ARGUMENT TO MITIGATE YOUR DOUBT. It is an existential problem, not an intellectual problem. Recall Tomberg's discussion of the fall, for reference.

\section{Prayer of the Heart}
The Prayer of the Heart was practiced by the early Egyptian Desert Fathers as a method of the purification of the Mind and the Heart. The Prayer of the Mind is the inner recitation of a prayer. When the prayer runs on its own, apart from the Mind, then it becomes the Prayer of the Heart. Ultimately, one can pray even during dreams.

The Prayer of the Heart is most often associated with the recitation of the Jesus Prayer. However, the unknown author of The Cloud of Unknowing recommends some simpler prayers like the repetition of ``God" and ``Love". If it is difficult to bring attention from the Mind to the Heart, then attention can focus initially on the hands or another external body part. The Prayer of the Heart bring attention ``in" the heart. In other words, the heart becomes the center of awareness, not the object of awareness.\footnote{From \url{https://www.gornahoor.net/?p=16036}.}

\section{Centering Prayer}

Years ago, I learned centering prayer from Fr. Thomas Keating, not personally, but from some cassette tapes. In this brief video, he asserts the necessity for daily meditation. For our purposes, note particularly his explanation that engaging with spiritual friends is an adequate substitute for a spiritual director. That — if you haven’t figured it out that by now — is why we choose to work in groups. Hence, regular attendance is important, not just for yourself, but also for the commitments you’ve made to others.

Meditation, just like riding a bike or swimming, cannot be explained intellectually. One must simply begin. However, once the practice is established, you can get feedback from your spiritual companions. Just as your biking or swimming can improve, so can your meditating. After all, it is the first step to becoming a Bodhisattva.\footnote{From \url{https://gornahoor.net/?p=9067}}

\section{Cartesian Meditation}

So, back to the young Rene: following his example. I will often lie in bed pondering some issue. Of course, Cartesian meditation is not the source. Rather, the real meditation reaches in the depths, often murky depths, not for the clarity of the atmosphere. Nevertheless, clear ideas are floating in the darkness of those depths, and they need to be coaxed out. Obviously, the discursive mind is required in order to turn those vague intuitions into text. That is the purpose of Cartesian meditation. Ultimately, however, there is not a shortage of ideas, but rather its opposite. There is actually an abundance of ideas, so cutting and pruning is necessary. Much more is discarded than is ever published.

If you allow your intellectual life to be nourished by the real nutriments hiding in the darkness, you will no longer be satisfied with dazzling baubles, word puzzles, or intellectual trivia. The goal of the intellectual life is to be a Sage, so seek the higher things like virtue, the life of reason, aesthetic beauty, the path of salvation, and the attributes of God.\footnote{From \url{https://gornahoor.net/?p=9067}}

\section{Middle Path between Spirituality and Intellectuality}

Tomberg points to the Whirling Dervishes and Zen monks as those who have abandoned the intellect entirely. Some of this lies behind the Hesychast controversy. What concerned Barlaam was the ignorance and credulity of some of the monks, so, in compensation, he overemphasised the side of the Intellect. The monks, on the other hand, pointed out that the first disciples were simple men, not advanced scholars. Now that may be true in the Synoptic Gospels, but John’s Gospel explicitly identifies Christ with the Logos behind the creation of the world. We take the middle path between Barlaam and Palamas.

We can be Holy Fools, yet still be intellectually competent. Our meditations should be on the life of Christ or something analogous; that is, something that requires an Active Imagination, not the passive imagination of a dream-like state. We concentrate without effort and have mastery over what thoughts and emotions are allowed to take hold in our consciousness.\footnote{From \url{https://gornahoor.net/?p=9067}}

\section{Attitude towards Life}

First of all, you need to learn to live without expectations, i.e., to be indifferent to results. You live and act, the best you can, come what may.

To find the perfect woman, you yourself would have to be perfect. The Cabbala explains that all pre-existing souls have a male and female part, that separate at birth. Only if the human being is pure and his deeds are pleasing to God, will the union fall to his lot which he possessed before birth. In the opposite case, he receives for a spouse a soul that does not correspond to him.

On a more pessimist note, Adam found the perfect woman for him, yet things did not work out the way he had hoped. You can look it up if you don’t believe me.

But on a more practical note, don’t make any woman the centre of your life. Find an interest … a destiny … and pursue it with all your heart and energy; that should be your central focus. Good things will follow.\footnote{\url{https://gornahoor.net/?p=11075}}

\section{Remember}

True being is beyond both subtle and gross manifestation, but in the process of manifestation modern men forget their real being. Instead, their self-identify lies in the astral body, which is interpret to be the center of interiority. They falsely attribute their essential nature to it, and thus deny the possibility of psychic heredity. Or else, they admit psychic heredity but then conclude that being is totally determined by environmental factors.

So, for all practical purposes man is indeed limited to hereditary and environmental factors. In his forgetfulness of being, he is then subjected to the influences of the cosmic environment unconsciously. He experiences this negatively as a limitation to his being, whereas in reality it is simply a privation or restriction excluding certain possibilities. In sum, he is passive in respect to the environment.

Try, then, to remember. Remember who you are. Your environment, family, body, mind were willed by you at the deepest level to be in conformity with your nature and destiny. The people you are in relationship are there for a reason, because you chose them. A chance encounter, book or idea is intended for you. Be attentive and conscious at all times to their meaning or purpose. They are not a limitation on your freedom, but rather an opening to being.\footnote{\url{https://gornahoor.net/?p=1664}}