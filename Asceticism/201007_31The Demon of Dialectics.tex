\section{The Demon of Dialectics}

\begin{quotex}
The premise from which the Buddhist \textbf{Doctrine of Awakening} starts is the destruction of the \textbf{demon of dialectics}; the renunciation of the various constructions of thought and speculation which are simply an expression of opinion, and of the profusion of theories, which are projections of a fundamental restlessness in which a mind that has not yet found itself its own principle seeks for support.

\end{quotex}
\flright{\textsc{Julius Evola}, \emph{The Doctrine of Awakening}}

This premise is the exact opposite of what people believe. They believe in the search for the right opinion, the correct view, the salvific belief. The purpose of this search is to do the impossible: to build a castle in the air. The man engaged in this quest is unaware of this elementary thesis: 

\begin{quotex}
We accept the proofs of Hume, Kant, Herbert Spencer, Fuller, and others of this thesis: The Ratiocinative Faculty or Reason of Man contains in its essential nature an element of self-contradiction.

\end{quotex}
\flright{\textsc{Aleister Crowley}, \emph{Equinox Vol 1 No 2}}

To this we would add, from \textbf{Gödel}: if a system of thought contains no self-contradiction, then it is incomplete. 

While for Beings in the human state, there is necessarily a perspective (see Hermit Crabs and Nietzsche\footnote{\url{https://www.gornahoor.net/?p=708}}), it is equally necessary to understand that it is only a perspective, something arbitrary and ephemeral, and unrelated to real knowledge, True Will, Awakening, the Solar spirit. All our opinions on politics and popular culture, our likes and dislikes, our style or dress, are all equally arbitrary and stand as obstacles to Awakening. 

\begin{quotex}
Opinion, O disciples, is a disease; opinion is a tumour; opinion is a sore. He who has overcome all opinions, O disciples, is called a saint, one who knows. (Majjhima-nikaya 2:38) 

\end{quotex}
Crowley once proposed an interesting exercise, which we adapt. For one year, be a political liberal. Subscribe to their journals, read their web sites, attend their rallies. The following year, read conservative journals, watch their networks, participate in their demonstrations. Try to understand the worldviews of each, how that affects their opinions on issues, and blinds them to deeper truths. The same could be done with attitudes toward pop cultures or personal styles. Such exercises will loosen the hold that opinions have over us; they will give us more options for action; they will begin the process of awakening our consciousness from the turgidity of ordinary life. 

Man has two aspects: his \emph{essence} and his \emph{personality}. His \textbf{essence} is who he truly is, what he is born with, beyond the superficialities of opinion. \textbf{Personality} is what accrues to him during his life, his opinions and imaginings, or “social constructs”. You will find that what is essential and what is a social construct are polar opposites from what is commonly believed. 

“Do I contradict myself? I embrace multitudes.” (Walt Whitman) Go ahead, embrace multitudes. Only thought can contradict itself, “I” cannot.



\flrightit{Posted on 2010-07-31 by Cologero }
