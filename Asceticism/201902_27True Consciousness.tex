\section{True Consciousness}

\begin{quotex}
He seems to be a proclaimer of foreign gods. \flright{\textsc{Acts 17:18}}

\end{quotex}
\paragraph{Natural Man}
In \textit{Ascent to the Divine}\footnote{\url{https://gornahoor.net/?p=10801}}, we described Augustine's ladder of ascent. The three lower stages describe natural man. Although the I or ego of the natural man will move among the three lower stages, each man is centred predominantly in one of the three. This centre colours his grasp of reality.

Medieval man was spiritually oriented, so he regarded thought as an opening to a reality beyond himself, either beneath him or above him. For example, concupiscence and malice were attributed to effects of the Fall, not as part of a man's very essence. That is why such impulses and fantasies were seen as originating from subhuman forces, underground, beneath the earth. So medieval man attributed the origins of such ideas to another being, the devil. Rather than being hypnotized, this was rather a truer grasp of reality. For him life was a battle, a spiritual combat, between the higher and lower forces that he experienced in consciousness.

On the other hand, as modern men have fallen deeper into materialism, the sense of thought as the revelation of something transcendent declines. While remaining oblivious to higher forces, he regards the lower impulses as his true nature. Even for the religious, spiritual combat has become pro forma, without any real commitment or understanding.

\paragraph{The Union of Opposites}
\begin{quotex}
Like a bridegroom Christ went forth from his chamber …. He came to the marriage-bed of the Cross, and there in mounting it, he consummated his marriage. And when he perceived the sighs of the creature, he lovingly gave himself up to the torment in place of his bride, and joined himself to her forever. \flright{\textsc{St. Augustine}, \emph{Sermo Suppositus 120}}

\end{quotex}
\textbf{Carl Jung} quoted that passage in \emph{Mysterium Coniunctionis} to illustrate the union of opposites. \textbf{Edward Edinger} in \emph{The Creation of Consciousness} offers an explanation:

\begin{quotex}
The coniunctio of opposites is not generally a pleasant process. More often it is felt as a crucifixion. The cross represents the union of horizontal and vertical, two contrary direction movements. To be nailed to such a conflict can be a scarcely endurable agony. \flright{\textsc{Edward Edinger}, \emph{The Creation of Consciousness}}

\end{quotex}
That explains why such inner conflict is assiduously avoided. People therefore will gravitate to one side or the other. However, Edinger explains the effects of the union of opposites.

\begin{quotex}
The union of opposites in the vessel of the ego is the essential feature of the creation of consciousness. Consciousness is the third thing that emerges out of the conflict of twoness. \flright{\textsc{Edward Edinger}}

\end{quotex}
In spiritual practice, inner conflicts are deliberately courted precisely in order to enable the ego to attain to greater consciousness.

\paragraph{Dueling Selves}
In a series of lectures delivered in 1939 in Rotterdam, \textbf{Valentin Tomberg} speculated on the inner meaning of the Crucifixion of Christ. He draws two opposing, but reconcilable, conclusions:

\begin{itemize}
\item Christ descended into Hell 
\item ``I am my Father are one" 
\end{itemize}
That is, there are both a descent and an ascent. These two movements resulted in a separation between the lower I, or ego, and the higher I, or conscience. In descending into Hell, the interior of the earth, he redeems the ego from the devil as described above. The higher I was then opened up to experience the spiritual world. This is felt initially as conscience, that is, as Christ a judge. However, the requisite purification of the mind and the will is not an easy task. As long as the lower and higher I's are disunited, the sense of conscience will be resisted. Its psychological effect is described by Edinger.

\begin{quotex}
The experience of being a known object, being seen by the Eye of God, can be a fearsome experience because unconscious contents cannot stand to be observed. \flright{\textsc{Edward Edinger}, \emph{The Creation of Consciousness}}

\end{quotex}
Just as medieval man experienced lower impulses as alien forces, modern man often experiences this higher self as alien. Hence, he may describe it as a ``Semitic" imposition or world-denying, in contrast to the lower I which is focused on the world. In that case, it would be impossible to unite the two I's.

\paragraph{Transformation in Christ}
\begin{quotex}
Man's task is to become conscious of the contents that press upward from the unconscious. Neither should he persist in his unconsciousness, nor remain identical with the unconscious elements in his being, thus evading \emph{his destiny, which is to create more and more consciousness}. \flright{\textsc{Carl Jung}, \emph{Memories, Dreams, Reflections}}

\end{quotex}
In his book, \emph{Transformation in Christ}, \textbf{Dietrich von Hildebrand} describes a spiritual path based on phenomenology; hence, it is mostly free of metaphysical arguments or sentimental devotions. For example, there is a chapter on Self-Knowledge and one on True Consciousness. In the latter, he claims:

\begin{quotex}
The inward progress in the Christian's life is linked to a process of awakening to an ever increasing degree of consciousness. Conversion itself is comparable to an emergence from a state of somnolence. In rising from self-contained worldliness towards the reality of God, in experiencing the metaphysical situation in which God has placed him and the new light in which all things and his own self are now appearing, \emph{the person attains to a new level of consciousness}.

\end{quotex}
Hildebrand warns against contemporary schools of thought which strive to reveal the hidden motives of thought. This is the technique employed by the so-called ``Masters of Suspicion" – Marx, Nietzsche, and Freud. In this task, the Medievals were more correct than those masters.

A second form of false consciousness is that of the man whose sole goal is to master a system intellectually. Hildebrand elucidates:

\begin{quotex}
He is not filled with a genuine longing for participation in being. Knowledge is not from him a road to such participation but a mere submission to the immanent logic of an unlimited process divorced from the goal of possessing the truth. Such a man cannot even truly understand the primary function of the intellect, with the participation in being which it embodies by itself. To such a man the process of acquiring knowledge has become a self-sufficient purpose.

\end{quotex}
The unconscious remain submerged in the lower I, as though in a state of nature. Hildebrand describes this state:

\begin{quotex}
The behaviour of unconscious persons is dictated by their nature. They tacitly identify themselves with whatever response their nature suggests to them. They have not yet discovered the possibility of emancipating themselves, by virtue of their free personal centre, from their nature.

\end{quotex}
As a man awakens to the higher I, the fourth stage described by Augustine, this is the result:

\begin{quotex}
A truly conscious person has so far advanced over his nature that he no longer agrees implicitly to all its suggestions. Should an impulse of malice or envy surge up in this mind, he, actuated by his free personal centre, will seclude himself from the impulse and disavow it.

\end{quotex}
Unconscious man lives from moment to moment and is thus incapable of understanding events in a larger context. On the other hand, it is different for conscious man.

\begin{quotex}
Wakefulness means to live [in the sight of God]; to interpret everything in the context of our eternal destiny, in its nexus with all our previous valid experience. Conscious man avoids being submerged beneath things or living among them in the interstices of reality; he incorporates everything int eh objectively valid order of ultimate reality. Only the Christian can be truly conscious in the full sense of the term. For he alone has a true vision of reality proper and a true conception of God and the supernatural realm from which everything derives its ultimate meaning.

\end{quotex}
In describing wakefulness, Hildebrand comes close to Tomberg in the latter's understanding of conscience.

\begin{quotex}
True consciousness implies an intimate recognition of our defects. A person who is thus conscious, who has emancipated himself from his nature and no longer agrees automatically to its suggestions, who is awakened to a sense of his free personal centre and of the essential, express, and lasting response which God demands of him, has also cast off his illusions concerning himself. His own being is illumined by the light of God and he allows that light to penetrate into all corners of his soul.

\end{quotex}
Unconscious man is discontinuous and ununified. Hildebrand describes him this way:

\begin{quotex}
Frequently we come across people who reveal entirely disparate aspects of character, of which now one and then another prevails, so that on different occasions such a man or woman may almost strike us as a different person. According to the varying elements of his environment, which their fluctuating appeal to this or that strain in his mental composition, a person of this kind may seem again and again to change his identity.

\end{quotex}
The life of the conscious man is integral and he always remains himself. The more he suffused with the light of truth, the close he comes to the Absolute I.

\begin{quotex}
We see now through a glass in a dark manner; but then face to face.

\end{quotex}


\flrightit{Posted on 2019-02-27 by Cologero }

\begin{center}* * *\end{center}

\begin{footnotesize}\begin{sffamily}



\texttt{Jack on 2019-02-28 at 23:11 said: }

``True consciousness implies an intimate recognition of our defects."

Yes. I think this can be the basis for a critique of another kind of false consciousness, which is that promoted by new age interpretations of Eastern doctrines. The whole ``we are infinite consciousness and primordially pure" idea which is seized upon by contemporary urban ``yogis" and the like is often an evasion of real consciousness of self, which necessarily includes conscience and hence a recognition of sinfulness. They want resurrection without crucifixion, or heaven without purgatory.


\hfill

\texttt{argusandphoenix (Logres/MS) on 2019-03-01 at 21:27 said: }

``Hence, he may describe it as a ``Semitic" imposition or world-denying, in contrast to the lower I which is focused on the world. In that case, it would be impossible to unite the two I's." 

From reading what I've gotten to in Plotinus, which isn't all of him by far, he discusses ``lower" things that men or beings sink into, and which they certainly have enough light to struggle against. It's not phrased the same way, but the idea is the same as hamartia, or unrighteousness. Have they gotten around to condemning Plato and Plotinus of being Semitic? I used to have the impression that Plato's theory of participation was arbitrary (one always introduces another intermediary), but if one stops thinking of the structure of reality as a ``thing" and begins to attempt experiential relation to personal and supra-personal centers, then there is no arbitrariness at all: two persons or Beings too far apart require an intermediary, and there is no inconsistency if the intermediary is apt (no need for an intermediary of the intermediary, etc., etc.). Thank you for this concise, and fitting post.


\hfill


\end{sffamily}\end{footnotesize}
