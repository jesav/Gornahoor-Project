\section{Excerpt from Forming the Soul}

\begin{quotex}
The following is an excerpt from a tract circulated within the \emph{Orthodox Word}; it was written by the Sisters of St. Xenia's Skete. A good friend of mine gave it to me a long while ago.

There is food for thought here, as the writer(s) argue that old Western culture can be a sort of \emph{pre-evangelium} to form the soul in preparation for higher spiritual discipline. I find it interesting on the grounds that the Orthodox Church is officially strongly anti-Western (at least theologically) \& that the argument here is a case of sophisticated and wise application of difficult doctrines which the body of Christians ordinarily use to ignorantly argue against “higher learning”.

\end{quotex}
To regard everything Western as immediately suspect argues a profound insecurity, a \emph{legalism} more rigid than any sect, a scholasticism more arid than any \emph{summa}. We have not inherited Western culture at all. That is precisely our trouble. We have simply grown up on the degenerate and decaying vestiges of that culture. We live, not in the West, but on the fading memory of the West. Our present culture is an absence of culture, a vacuum that has left our souls shrunken and our spirits stifled. Before trying to plunge his spirit into the deeps of Orthodoxy, today's man must first feed his soul, for its paucity will not permit any profound growth of spirit. Modern Western man is like a plant with the shallowest of possible roots, and he naturally cannot support any great growth. His spirit is no longer capable of soaring, because a lofty spirit must rise out of a deep soul which has the maturity, the sensitivity, to feel noble things and become ennobled by them.

Without this, it is difficult to attain sobriety, fruitfulness, authenticity, depth, in our spiritual lives, simply because an unformed, uncultivated, undisciplined soul seldom has the discernment and balance to see clearly and honestly, nor the sensitivity to feel deeply, nor the intensity to strive wholeheartedly, nor the idealism to reach uncompromisingly for what is truest and best. This capacity of the soul is not in and of itself spiritual. Rather, it serves as a preliminary to spiritual things, and it acts as a sort of spiritual tuning fork, often helpful in preserving a well-grounded, sober, discriminating, yet exalted habit of mind. It will be both restraint and prod. We must regain the capacity of soul which can only be developed by proper and careful cultivation.

It is psychologically impossible for us suddenly to become “non-Western”, even if such a thing was desirable. It is spiritually unwise to try to feed a spirit from a dead and dormant soul, and it is intellectually irresponsible to offhandedly reject the treasures of hundreds of years of Christian culture in hopes of escaping the taint of Westernism. If we reuse to nourish ourselves on what is edifying and elevating, we will inevitably be fed by what is not, as the popular culture of America, in all its shallowness and falseness, seeps into our unguarded hearts daily. If we do not counteract it, if we fail to set the loftiest things before us, we will inevitably let our souls remain choked with artificiality and cheapness. We will remain mired in the fatal shoddiness of world and ourselves, and we shall not be able to touch the depths of our own hearts, nor answer the needs of our neighbor's.

The Orthodox Christian who wishes to avoid crippling even further the already spiritually-underdeveloped soul of today's man can find his guide in the actions of the early Church, and he can also find there his defense against those who would wish to regard all artistic and intellectual pursuits as worldly or useless. When the Church denounced pagan culture, She denounced only those aspects of it which were based on or nourished by the demonism of pagan religion or the hedonism of pagan art. Those aspects of Hellenic culture which were useful and healthy, She not only refrained from denouncing, but even transmuted into a profoundly convincing missionary statement. Then, as now, many decried the use of secular art and learning as a means of cultivating and educating the soul, defending their position with the Apostle's warning to \emph{beware lest any man spoil you through philosophy and vain deceit, after the traditions of men, after the rudiments of this world, and not after Christ} (Col. 2:8). In answering them, the Fathers of the Church formulated the response which has remained the Orthodox position on the issue, a response expressed in the teachings of men such as Clement of Alexandria, St. Basil, \& St. John Damascene.

In his \emph{Stromateis}\footnote{\url{http://www.newadvent.org/fathers/0210.htm}}, Clement teaches that the Apostle's warning applies only to those who have turned back from spiritual things to the things of the world, from the ultimate truth of Christ to the partial truth of secular learning, “philosophy being most rudimentary compared to Christianity, and only a preparatory training for the truth” (\emph{Strom}, VI, 7).

It is quite true that the study of poetry, history, art, fiction, is indeed a “most rudimentary” one. It is indeed not a spiritual study. But we, in our modern condition, are in dire need of the rudiments, not only of spiritual life, but of simply humanity. It is absolutely imperative for us to elevate our souls before trying to participate in any sort of authentic spiritual life. To try to ignore or to overleap this primary step leads only to spiritual sterility or deception. We do not live in a normal age, and we are no longer normal people, even by the minimal standards universal to previous ages in the lief of fallen man. The Apostle warns us not to mistake the lower life of the soul for the higher life of the spirit, and warns us not to turn back from the fullness of Christ to the emptiness of the world, but he does not tell us to ignore the development of the soul altogether…The Fathers have always taught that the higher, spiritual part of man's nature is founded on the first level of the soul, that which is sensitive to and best developed by the study of virtuous, noble, and beautiful things. Our faculties and responses, distorted by the Fall, must be restored, through virtue, to normalcy, and after that we can begin to progress in spiritual things…

[...]

He treasured hierarchy because it was for him a reminder of God. His whole world was an endlessly unfolding, interlocking allegory of the majesty and love of God. He rejoiced in the delight of obedience and walked fearfully in the humility of command, because both were images of profound spiritual realities. He revelled in the color and beauty of the physical world, because they were fore-shadowings of the even greater splendors of the Kingdom of God. He could spend his entire adult life building a cathedral, and never forget the transcience of the the temporal world.

His whole world acted continuously to form his soul in an elevated way. His literature, didactic and moralistic, reminded him of the beauty of virtue and nobility, and the brevity of life, and his poetry sang of his delight in the created world and his awe of God…His society taught him to feel the reality and proximity of the spiritual realm almost more intensely than the physical. His churches, resplendent, delicate, glowing, lifted his soul on traceries of stone, and set his spirit on the heights.

The spiritual impetus of that world carried over for almost a thousand years more. The nourishment provided by a thousand years of Orthodoxy was the spiritual ground in which grew all the best in Western thought and art, and that impetus remained largely intact until the Enlightenment, was eroded greatly during the Romantic Age, and finally crumbled entirely in our own time. The best that was done, was done in this spirit, springs from this world. The community of feeling and intent which marks the best of our writers, artists, musicians, springs from this source. Regardless of social, political, and religious changes, Shakespeare and Dickens, Bach and Mozart, Donne and Hugo share the same world, and it is to this world the Orthodox Christians of today must look for formation of soul. There are lessons we must learn from our own past before we can possibly hope to go on.

We must recover the feelings and sensitivities which were once the common property of all civilized people. Those works of art, of literature, of music, which are pre-modern are of essential value for us. They can teach us, as will nothing we ourselves now produce, what nobility is, what virtue is, what honor and purity are, what sacrifice and loyalty are, what is worthy and what is not. Poetry, music, art, fiction, are not spiritual food, but are rather the milk and bread we need to strengthen ourselves to live on the meat of the spirit.



\flrightit{Posted on 2012-10-05 by Logres }

\begin{center}* * *\end{center}

\begin{footnotesize}\begin{sffamily}

\texttt{william zeitler\footnote{\url{http://www.williamzeitler.com/}} on 2012-11-03 at 11:28 said: }

“He treasured hierarchy…” Who is the `he' to whom this blog refers?



\texttt{Logres on 2012-11-03 at 14:35 said: }

The average peasant, who had a finer sense of what is actually real than 90\% of our America. Listening to Americans reminds me of those old movies where the mayor comes out to say a “few words” about the community, which invariably ends up being the “best”, “finest”, “happiest”, etc., etc. in all kinds of ways, in the world. And I am an American. Why are we this way? I think I have a few clues…


\hfill


\end{sffamily}\end{footnotesize}
