\section{Purification of the Sex Act}

\begin{quotex}
In this section, Evola offers his comments on the purification of the generative act. Drawing primarily on the tantric writings and German idealism (especially \textbf{Fichte} and \textbf{Schelling}), but also \textbf{Clement of Alexandria} and \textbf{Plotinus}, Evola equates the woman with nature. And nature is that which is still dark or impervious to the spirit. The goal, as he sees it, is not detachment from the body, but rather its transformation into spirit, the “resurrection body”. The desire to eternal life lies in the depths of man. 

\end{quotex}
Since an act is impure or imperfect if it does not attain actuality from itself, it is evident that the generative act is like that, and in a typical way—\textit{and it can be said that the woman in a transcendental way is nothing but the symbol of the impotence of the I to give itself a body from itself}. Now, in order to understand what meaning purification has in this case, it is necessary to keep in mind what was said in principle, and that is that the imperfect act only apparently resolves the insufficiency of the agent, in reality it reaffirms it: whoever drinks, whoever asks for water that is “another” and not the eternal water that is “pure act”, and who, with Christ whom we referred to, will never have his thirst satisfied for all eternity. It follows from this that as long as the “I” asks the woman for the conditions for a generative act, he will remain in the state of privation and impurity: and the dyad and the “other” that he presupposes as such (in the duality of the sexes) can with justice only be reaffirmed in the result, i.e., it will make it so that the act is worthless as self-generation, but rather as \textit{heterogeneration} (generation of the son), whence the destiny of death.

What gives life to the son, \textit{kills} the father, which makes the “Only One” mortal, an individual [See Clement, Stromata, Book III, Chapter 9, para 63]. The act escapes from the agent and makes his life pass away. To explain it more clearly: in the depth of the individual there is an originary power that \textit{wants} life beyond temporal limits; this power, at the level of normal and \textit{extroverted} human existence, is the desire “to look outside” (\textit{bhavamukha}).

It strikes on “another” (the woman) and so the impulse for continuity degenerates, the act that should have been the affirmation of oneself (autogeneration) becomes the affirmation of the other — of the son. The resulting continuity is not then that of the individual but \textit{that of the species} — and man is drawn into the wheel of finite and discontinuous existence, dominated by the law of generation and corruption — mortal, eternally thirsty, and eternally disappointed.

This impurity of the generative act is connected to that which inheres in the very existence of a physical body. In the normal man, the conscious power falls to a great degree outside that deep principle that rules the various processes of his organism. For this reason, he does not know how to give himself a body from himself; for this reason, he is impotent against the law of corruption. As Leibniz expressed it, the “flesh”, corporeity, represents simply the indistinct and unconscious \textit{quantum} (or better: passion and privation) that is in the I, and only in this sense is it to be understood as an imperfection. And one could say that such a region of privation in the I is the transcendental foundation of the woman, since we indicated that the correlative of the generative act as imperfect is conveyed precisely in the woman.

This is, to use alchemical symbolism, the “salt” that encompasses the active and central principle of “fire” or of “sulfur” and which “mercury” must release until reconciling this same sulfur only with itself, in the flame of the divine (\textit{theion} = sulfur = divine), of that which is pure possession or perfect act and whose eternal life, whose eternal autogenesis, is precisely symbolized by the Phoenix pulling himself out of the fire.

This is the “Great Work” that has, as its direction, the construction of a “body made of freedom”, of a body spiritually transparent to itself. We are led to the doctrine of the “immortal or cosmic body”, traces of which are found in almost all the religions and is therefore based on the following premise: corporeity is only whatever is passive in the spirit, the not yet expressed, the virtual, “in potential”; it does not constitute a distinct principle, but instead is a state of privation (\textit{sterema}), a shadow in the unique reality of the spirit.

Thus it is clear that liberation cannot consist in detachment from the body, but in its \textit{resolution}. Here purification means precisely the realization, as a function of potency in act, that the material body is experienced as a function of passion — and as such, the construction of the immortal body or “resurrection body”, “body made of spirit”, “apparent body” (\textit{mayavirupa}: recall that \textit{maya} in the tantric and mahayana schools means both appearance and magical power) or body “of flames” [according to Plotinus, the body of flame is sufficient in itself and does not require food], it is the dissolution without residue of the material body in pure activity, in \textit{shuddha-sattva-guna} — it is the individual in which the negative aspect of \textit{rajas} and \textit{tamas} has entirely vanished.

It is called immortal because, depending entirely on the I, the I can make it appear or disappear, maintaining it or destroying it when he wants, for the time that he wants, in a way that the law of life and death is vanquished. It is called the “cosmic body” for this reason, it is admitted that the metaphysical principles or “divinity” that rule nature are present in the body, although only under a darkened form as if sleeping, a form which the I experiences precisely as nature, i.e., as the “other”, and not in itself as spirit. But at the point in which corporeality is entirely conquered in conscious actuality, this form, however, is less identified in the various principles of the cosmic hierarchy, and the individual goes on to feel that his true body is the universe.

Now if the impurity of the generative act is the cause of finite and mortal existence; and if, on the other hand, this existence is defined in the differential of obscurity and privation that the body as such represents; one understands how at the construction of that supreme purity that is the “immortal body” — through which the very “other” of exterior nature is dissolved — a conversion of the force of generation is associated. This is not the place to expound on the technique of such a process.



\flrightit{Posted on 2013-11-13 by Aeneas }

\begin{center}* * *\end{center}

\begin{footnotesize}\begin{sffamily}



\texttt{X on 2013-11-13 at 04:30 said: }

Sometimes Evola is deadly accurate and sharp.


\hfill

\texttt{JA on 2013-11-13 at 09:45 said: }

it's like he's trying to transcend corporal sex instead of seeing sex as the transcendent as Crowley did.


\hfill

\texttt{Jacob on 2013-11-13 at 22:08 said: }

It's great reading the same thing in different language. I feel like I understand slightly better what Serrano was getting at the whole time in El/Ella. 

I'll have to think on this some more.


\hfill


\hfill

\texttt{n0e on 2013-11-16 at 04:58 said: }

in modern terms one could say that Evola shows here metaphysical underpinnings of astral projection


\hfill

\texttt{Cologero on 2013-11-17 at 19:27 said: }

Jacob, there is an Italian saying “tradurre è tradire” (translation is betrayal). So, of course, there is a certain amount of “interpretation” along with a translation. That is, there is no univocal correspondence between a word in one language and another, so the meaning has to be grasped in order to choose the correct word or phrase. In a sense, Evola is doing something similar when he points out that a concept in one tradition is expressed in a different way in another, although the meanings are the same.

It is certainly worth the effort, as much for me as for the readers here, and it is part and parcel of learning how to read, I mean to truly read. Now a parrot has no understanding, so there is no point in just quoting long passages from Guenon, Evola, Nietzsche, etc. That is something that will get you a lot of “likes” on facebook, but no respect in this quarter.

I presume you are reading El/Ella in Spanish? Now that my health has been cleared for the time being — I've seen every specialist but a psychiatrist — I am thinking of reviving the El/Ella discussion list. I just need to decide on the best format to keep it on track. Gornahoor's time is running short and the more important work will be done in discussion lists.


\hfill

\texttt{Jacob on 2013-11-19 at 11:23 said: }

No, I was just using language in its figurative sense. I had only read some of it in English thanks to our discussion. 

But Evola expresses it at least to my mind more clearly, or maybe they were both equally clear and being able to connect that the two are nearly one and the same allows greater understanding. 

I'm really glad you're going to restart the discussion. I often feel like I'm trying to learn to paint beautifully without even seeing the colors and just hearing the sound of the brush. It's pretty much as you said grasping the concept intellectually vs. actualizing it.


\hfill

\texttt{Scardanelli on 2013-11-19 at 21:54 said: }

@jacob

I know what you mean about theoretical knowledge vs. experience. There is knowing and there is knowing so to speak.

There are a wealth of books to read but very few who can direct you regarding practical experience. I think the discussion list format is the best next step. I'm looking forward to it.


\hfill

\texttt{Thomas on 2013-11-21 at 12:21 said: }

Thank you for this very interesting extract from Evola. Is there somewhere that we can find ” the technique of such a process” either in Evola or elsewhere?


\hfill

\texttt{Sigurd on 2017-06-20 at 19:06 said: }

To what noe has replied with: is this the Great Work of Hermetic thought? To project the astral body? This passage from Evola seemed quite nebulous to me, being truthful, before I read the comments. The quasi-philosophical wording makes it quite difficult.


\hfill


\end{sffamily}\end{footnotesize}
