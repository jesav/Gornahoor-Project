\section[Some Observations Concerning Guenon...]{Some Observations Concerning Guenon, Initiation, and Spiritual Exercises}

\begin{quotex}
All that exists potentially is advanced to actuality by the agency of something which is actually what the other is potentially: the partially potential by that which is actual in the same partial respect, and the wholly potential by the wholly actual 

\flright{\textsc{Proclus}, \textit{Metaphysical Elements}, Proposition 77} 

\end{quotex}
Every so often one hears of conversations, or reads essays contrasting \textbf{Rene Guenon}'s views of initiation and spiritual practice against those of other initiates, such as \textbf{Julius Evola}, or \textbf{Frithjof Schuon}. In certain of these, one tends to find Guenon characterized as overly ``conservative" and ``bureaucratic", or perhaps less flatteringly, as ``cold", ``dry", and exceedingly ``logical"; his emphasis on initiatic doctrine, and regularity of a chain, appears to cause many readers of Guenon to develop an opinion that he only regarded a rigid, formal, inflexible initiation rite as necessary, and of sole importance, to the exclusion of any other possibilities, or efforts. While undeniable that Guenon stressed the regularity of chains, and meticulously sought to differentiate the doctrines of traditional initiation from ``occultist" and ``spiritist" pseudo-initiations of his own era, he did so in the hopes of preventing further decay of what did remain of a traditional character, and yet never without fundamentally maintaining that regardless of just how regular an initiation might be, it must always accompany an inner work. Mincing no words on the matter, he remarks in \textit{Perspectives on Initiation} that ``Initiatic teaching, outward and transmissible by forms, in reality is and can only be—we have said this before and stress it again—a preparation of the individual for acquiring true initiatic knowledge by personal effort (203)".

The notion that Guenon was ``too heady" and therefore ``cold" likely arises from those who would take the written word of Guenon, as tantamount to the personhood of Guenon. Indeed, he wrote with staggeringly logical precision—but was no logician, frequently explaining that though metaphysics could be presented in a logical fashion, it is superior to logic; that the representation of metaphysical teaching discursively, is not the same as realization of that teaching, although serving as an advantageous departure point, yet insisting that ``Whoever clings to reasoning and does not free himself from it at the required moment remains a prisoner of form, which is the limitation by which the individual state is defined (supra, 209)".

Others have argued that Guenon offers few solutions to the problem of degeneration and disappearance of regular and effective initiation in the West, outside of either Freemasonry, or Compagnonnage—and more so, that he quickly condemned any attempts at esoteric knowledge outside of those parameters. But again, it would seem Guenon was exercising a precise caution, while attempting to ingrain reliable ``landmarks" in those either already on, or seeking a ``path". Could it be believed for example, that Guenon, while living in Egypt, as devout Muslim and ardent Sufi would have been unfamiliar with the Sufi practices of certain orders, who perform rituals and vigils at tombs of prophets and saints, in effort to invoke barrakah, and participate in a silsilah-chain with deceased masters, and moreover, that such rituals often entailed an ``istikhara", that is, ``dream incubation"? This would all seem unlikely, but his reluctance to discuss such things was probably due to reservations that such doctrines would quickly, especially in his era, be misunderstood and rapidly associated with ``spiritism"—which they are not. Still, in his chapter from ``Perspectives", on ``Initiatic Centers" these sort of practices are alluded to when Guenon is speaking of a ``double hierarchy" within initiatic orders, particularly when an order has reached a phase of becoming externalized to a greater extent, what stands above and behind the physical leadership (who may have forgotten their true roles), is the presence of ``invisible" and ``unknown superiors". While perhaps unbelievable to some students of Guenon, he elaborates:

\begin{quotex}
All this permits us to glimpse among the many possibilities of spiritual centers certain means of acting which are quite different from those ordinarily attributed to them, and which are especially evident in abnormal circumstances…that do not permit…apparent regularity…a spiritual center of any kind may thus also act outside its normal sphere of influence, whether in favor of individuals particularly qualified but isolated in a milieu where the darkness has reached a point that almost nothing of tradition remains and initiation is precluded, or…as reforging an initiatic chain that has been accidentally broken…it is essential to remember that even if an apparently isolated individual succeeds in gaining a real initiation, this initiation is spontaneous in appearance only and will always involve some kind of attachment to an effective center (supra, 65). 

\end{quotex}
For an individual having received a regular initiation (virtual or effective), or such a ``spontaneous" initiation as described above, Guenon indicates that there are means of support for the being's continued progress, even if an order has essentially lost or forgotten its ``operative" methods. In ``Initiation and Spiritual Realization", Guenon in a brief chapter turns our attention to the doctrine of ``upaguru", an influence that triggers or elicits a spiritual or initiatic process, remarking that this function might manifest as another person, or present itself as a situation, circumstance, or even some object. When arising as a person, it matters not if the person fulfilling the role, realizes what it is they are doing because ``…in reality the true cause is found in the very nature of the one upon whom the action is exercised". The upaguru is not though, something entirely random and ``objective", as it is an ``auxiliary" and ``prolongation" of the being's actual guru—and although upaguru might manifest on numerous occasions in the being's life, each manifestation elicits something specific, after which it is no longer upaguru (104-105).

What though might be said though of the initiate who has no guru by which upaguru could be extension or prolongation? In one type of case, as in orders that operate through ``group work", the role of a human guru is replaced by the ``spiritual influences" behind the work, which of course relates to what was said above concerning the influences of ``spiritual centers". While Guenon insists that a guru (either human, as the influences behind group work, or the influences behind the rare ``spontaneous initiation") are indispensable for the initiate, such is only true so far as the ``first stages"—meaning, the conferral of an initiation to begin with, initiation as ``beginning" proper, which is to say another way, the establishment of communication with the higher states. Provided that one or more of these conditions have been met (which as noted above always also implies a continuous attachment to a spiritual center, with or without attachment to a physical center) the continued process of spiritual realization need not necessarily be dependent upon a human guru since ``…the human guru is in reality only an outer representative and `substitute'. as it were, of the inner guru…whether or not there is a human guru, the inner guru is always present, since it is one with the very `Self'. whether in order to manifest itself to those who are not yet capable of having an immediate consciousness of it, it takes as support a human being, or a `non-incarnated' spiritual influence, is in the final analysis, only a difference of modality that changes nothing essential (124)".

With all of the foregoing considered, it becomes clear that consideration of the inner work does frame Guenon's views governing initiation far more than are generally acknowledged, to summarize what this entails, he remarks ``Indeed since all knowledge is an identification, it is evident that the individual as such cannot attain to knowledge of what lies beyond the individual domain…any knowledge that can truly be called initiatic results from a communication consciously established with the higher states (207-208)". This therefore accurately describes both the dilemma of Guenon's initiation, and the solution. Knowledge is the unification of knower-knowing-the known; but, the individual ``as such", that is proceeding no further than beyond the faculties of individual mode (of which mind is the limit), as he mentions, cannot enter into knowledge of super-individual states; yet, it is by means of initiatic transmission (regular or spontaneous), the presence of a guru, or a ``spiritual influence", that must usher the ``first stages", after which, in tandem with inner work, the chasm is crossed, since it is not mind, but Intellect, the noetic faculty which enters into said communication.

Another debate that still seems to go back and forth about Guenon and possibilities of a Western initiation, involves his rejection of the Christian sacraments as initiatory—the reasons for which he précised in both ``Perspectives on Initiation", and in the article ``Christianity and Initiation"–but he said more than just that concerning the sacraments, which tends to get short shrift. Yes, he argued that once Christianity ``exteriorized" as religion, from tariqa, the sacraments (although efficacious in the religious domain), could not in any event remain efficacious initiatically; so, although remaining beneficial to the human being in individual mode (``securing" and prolonging the human state post mortem, as opposed to possible disintegration of that state), they could not of themselves any longer take the being beyond the human state. What tends to gets ignored though, is that he added to this that they could however become initiatory, if a qualified being has the ability to ``transpose" them beyond the domain of religion, in a reversal of the process leading to their exteriorization so to speak, returning them to their principle, noting ``The truth is that the sacraments cannot indeed have such effects by themselves…but…the exoteric rites can, in a certain way, be transposed into another order in the sense that they will serve as a support for the initiatic work itself and that consequently their effects will no longer be limited to the exoteric order (17)". Naturally, their use as such ``supports" is contingent upon, as in the various foregoing scenarios, that the person is, in one or more of the senses outlined earlier, already an initiate.

Regular initiation for Guenon, is then inclusive of additional varied nuances, often glossed over. In ``Studies in Masonry" he defines regular initiation, as ``orthodox", by which he means, not something ``static" or ``mainstream" (a common, modern reaction to the word), but ``correct teaching", in so much that it is correct, because it is whole, complete, originates in metaphysical unity, and points back to it. Far from suggesting something awkwardly rigid, and smacking of the ``letter of the law", employing the same argument opposed to the possibility of repetition within the Absolute (total possibility), he declares

\begin{quotex}
One can then say that it is impossible that, for two different individuals, there should be two initiations exactly alike, even from the outward and ritual point of view, and all the more so from the point of view of the inner work of the initiate…This is why we have said that initiate teaching can never take a `synthetic' form, but on the contrary must always remain open to limitless possibilities in order to preserve the prerogative of the inexpressible (Perspectives, 203-207). 

\end{quotex}
Something of a reciprocal relationship exists between Freemasonry and the Christian sacraments. Excluding for now the variable nuances and ``spontaneous" initiations, Guenon does see as one of the only regular initatic orders surviving in the West existing in Freemasonry—although its character has become virtual; at the same time, the sacraments while effective, are limited to the exoteric. The rites of Masonry lead toward the primordial and super-individual states, yet are latent or deferred; the sacraments pertain to the human individuality and ``save", but do not ``deliver". The former must be made effective; the latter need be transposed.

Hellenism has transmitted a great deal to the West in general, and not surprisingly both Masonry and Christianity have been heirs, especially of Hellenistic spirituality. In Masonry for instance, where traditionally practiced, a special emphasis is placed upon the ``Chamber of Reflection". In the Western and Eastern Churches there is the sacrament of ``Confession", ``Penance", or ``Reconciliation". Algis Uzdaviny's explains in his ``Philosophy as Rite of Rebirth" that beginning with Egypt, and flowing into Hellenistic Philosophy, especially among the Stoics, that a particular pagan practice, as part of their method for initiation, ascent, and ``returning to the primordial state", included public confessions of their ``sins" (for practical purposes, let us substitute the word ``privation" for ``sin", as ``sin" too has lost most of its meaning, signifying to contemporary minds silly sounding infractions against wanton formalism, such as eating meat on a Friday; whereas the Church had, and the Eastern Church still mostly does, regard sin on a ``sliding scale"–that is, the spiritual damage of an act, or act of omission is proportional to the way in which it limits ones spiritual progress, which is why there once were ``confessors"–elders advanced in spiritual gymnastics, and learned in life, who could assist one in assessing the impact of such matters. Ironically, while such a notion is repugnant to modern minds, they will spend fortunes on ``therapists", ``life coaches", and ``councilors" to express their ``inner demons", in exchange for nothing but more profane information, and drugs that provide no remedy!).

Quite akin to Guenon's explanation that fundamental to initiatic science is transition from individual mode, toward increasingly less restricted states, Pierre Hadot describes the purpose of spiritual exercises and Philosophy in antiquity as promoting ``The movement of the soul from individuality to Universality". Hadot explains that for our Hellenistic Philosophers, this movement from individuality to Universality is marked by three key concepts and objectives:

1. Coming to see the insignificance of human affairs, profane affairs

2. Developing a certain ``contempt" for the notion of death, or how death is understood by the profane world

3. Attaining to ``Universal vision" characteristic of ``Pure thought"

In his book ``Spiritual Exercises", Hadot examines and discusses the various types of ``askesis" or methods employed by the Hellenistic Philosophers. Drawing upon the record of Philo of Alexandria, he relates seven major askesis employed in the ascent, and then consolidates them into three essentials:

1. Prosoche

2. Dialog

3. Learning to die (Philosophy itself)

While certain of these askesis are engaged through rhetorical and dialectical teachings of persuasion, skill in rhetoric and dialectic contribute to mastering one's inner dialog and concentration, thus lending itself nicely to prosoche. Prosoche, which is ``self-attention", or ``mindfulness", is closely connected with nepsis, spiritual watchfulness, and ``Guarding the Heart" (Center)–well enough known to students of Philokalia. It is also the significance behind the Masonic ritual ``Chamber of Reflection", as well as sacramental ``Confession"—neither of which, in their deeper meanings constitute a ``one time" symbolic rite, nor an occasional ``penance" to ``meet ones obligation"—but intended as habitual activity, with increasingly greater strides in success. The choice of beginning here with sacramental confession and the chamber, in connection with inner work aimed at the ``making effective" on one hand, and ``transposing" on the other, further in relation to prosoche, is deliberate. Confession is the sacrament following baptism; while second in order, it is really only the first requiring an active participation, as baptism, even when performed on an adult, is more of an acquiescence. Confession then leads to reception of the next sacrament, which is Eucharist or Communion—union with the Body of Christ as Church, and union with the ``Real Presence" of Christ in the Eucharistic species—which is (among other levels of meaning), an Anamnesis (``Do this in memory of Me"); if ``transposed" to the metaphysical (as an aseksis) it becomes Anamnesis of the ``Real Presence" of the central ``Self", which previously we observed Guenon identifying as the ``inner guru" itself. In this way we might come to understand how a sacrament can ``become" initiatic—how, as Guenon wrote, something exoteric can be taken as a ``departure point", and ``foundation", to be ``transformed" (another observation that can be drawn from this, is as Guenon often pointed out, that there never can really be any ``contradiction" between the exoteric and esoteric orders as the lesser is always included in the higher, and proceeds from it—an important observation, as so many seeking an ``esotericism" imagine it to do with meanings that are somehow repudiations of exoteric meanings). As was also noted among Philo's list, Prosoche is the bedrock of the Philosophical exercises toward ascent.

Hadot explains that for self-attention to begin, it presupposes ``examination"; again, what is found in the sacrament under consideration, and occurring in the ``Chamber", where the candidate surrounded by images of death, left alone as if in a tomb, is instructed to consider their life, and prepare a will. And indeed, the chamber is a tomb, in which one dies to the profane world, to be ``born" in the Lodge room (emblem of Kosmos/Universality)–which Guenon refers to as ``the first death" and ``second birth" (second birth for obvious reasons that it occurs after natural birth, and death to the profane world). Per Guenon, this phase of the Initiation process entails a ``psychic regeneration", the re-collection, recollection, anamnesis of the ``intellectuality" of primordial man, by ``gathering what is sown". Still, like confession, this stage is but ``preparatory" for, as Guenon notes ``the second death and third birth" which is a ``resurrection". In a certain way then, although not exact, these relationships drawn between the use of an askesis (prosoche in this account) for the purpose of ``transposing" a sacrament, and ``actualizing" an initiatic doctrine, might analogically be expressed thus–prosoche/confession is to memory of the Real Presence as prosoche/chamber of reflection is to initiatic birth.



\flrightit{Posted on 2013-01-05 by Frater M }

\begin{center}* * *\end{center}

\begin{footnotesize}\begin{sffamily}



\texttt{David on 2013-01-06 at 00:42 said: }

This was very insightful. The last two paragraph are very interesting, but I will need to read more on the subject to understand everything. Could you tell us where next to read about this ? Thank you.


\hfill

\texttt{Pickman on 2013-01-06 at 02:48 said: }

Soundly observations my good fellow Mercurius, welcome aboard (the ship that still requires an anchor some might say). Now outside of sterile academic method, will it ever be possible to achieve a living order, not derivative of subversive Islamic or masonic currents for a re-invigoration of European man?


\hfill

\texttt{Mihai on 2013-01-06 at 12:49 said: }

You wrote quite a lot here and most of it is sound. 

I would, however, want to point out one issue I have with Guenon's view on initiation, apart from questions of ``spiritual bureaucracy", that is the particular question of the Christian rites. 

Guenon's arguments that the ``exteriorizing" of the Christian faith led to their becoming inefficacious in the esoteric domain, like Schuon rightly pointed out, do not hold water. If " what God has joined together, let not man separate"(Matthew 19:6), then there is no reason to suppose that historical contingencies can cause a God-established rite to ``withdraw" its efficacy. Rather their ``transposition" on the esoteric plane depends on the quality of the person receiving them and on the guidance (or lack of it) that certain person receives. We can say that, in Christianity, the degree of exoterism that one attaches to a rite is a reflection of the degree of his understanding (by this not meaning mental understanding). 

Reading pages from the Philokalia, one finds there ample teachings of spiritual methods and scripture interpretations that do not fall short of any esoteric ``requirements". 

One more proof: the fact that in the Hesychast tradition of the East, there is no other requirement for its practice apart from Baptism (which is truly an initiation) and the will and inner disposition of the person undergoing it, plus a truly qualified spiritual guide- this last requirement can prove quite a challenge nowadays.

I have seen that some, following some remarks made by Guenon, claim there exists a ``hesychastic initiation", that is a further rite that one has to undergo in order to practice the Prayer of the Heart. Such a claim is entirely false, as there exists no such additional rite, as the testimonies of many followers of this path attest.

Also, if Guenon admits that the Mysteries where, in their initial period, completely esoteric and than become exteriorized, then one has to ask: where did that additional rite come from ? And if it was there from the beginning, but hidden, then what was Baptism for, if it was supposed to correspond to an esoteric initiatory rite (which, by the way, the works of Dionysius the Areopagite attest) ? Why did these two rites function in parallel ? 

Finally, one cannot fail to notice that Guenon puts the Christian rites and those of Masonry to a double standard.

He rejects the claims that Masonry has become totally ineffective due to its complete degeneracy, arguing that it still holds, at least ``virtual", initiatory power, but then goes on to declare the Christian rites as ineffective because of their ``exteriorizing", even though Christianity has never fallen to the degree of degeneracy that Masonry has.

I do regard Guenon's defence of the alleged validity of modern Masonry to be the lowest point of his teachings, considering that, not only the masonic lodges fell to a level of complete spiritual ignorance, but they also played an outright subversive role that contributed to the destruction of traditional Europe. To think that one can find any trace of living spirituality in a masonic lodge today is an illusion. 

Other than these observations, Guenon's works regarding initiation are very helpful in clearing the smoke and dispelling the many delusions promoted today in ``occult bestsellers" and the like. They should be a ``must read" to anyone interested in engaging on an esoteric path, lest one falls victim to the numerous counterfeits that circulate today.

\hfill

\end{sffamily}\end{footnotesize}
