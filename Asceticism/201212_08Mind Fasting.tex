\section{Mind Fasting}

In this time of year, many are making sacrifices such as giving up cupcakes or even añejo tequila. While there is a benefit to intentional suffering, it cannot happen in a mechanical way. Too often, it is thought of as the function of “will power”, that is the opposition of one desire (cupcakes) against another (sacrifice). What is really needed is to understand the relationship between personal effort and spiritual reality as we learn in the first Arcanum of Meditations on the Tarot. This is essential because 

\begin{quotex}
if one does not understand it (i.e. take hold of it in cognitive and actual practice), one would not know what to do with all the other Arcana.

\end{quotex}
Tomberg reveals the first and fundamental principle of esoterism:

\begin{quotex}
Learn at first concentration without effort; transform work into play; make every yoke that you have accepted easy and every burden that you carry light! 

\end{quotex}
This is also the key to Yoga as Pantajali tells us:

\begin{quotex}
Yoga is the suppression of the oscillations of the mental substance. 

\end{quotex}
These oscillations are automatic and mechanical: they arise from sensory impressions, inner desires, negative emotions, and thoughts that run on their own and usually have origins from unsuspected sources. Concentration is a free act and must be distinguished from obsession, which mimics concentration, but is not free.

\begin{wrapfigure}{rt}{.25\textwidth}
\centering
\includegraphics[scale=6]{a20121208MindFasting-img001.jpg}
\end{wrapfigure}

Intentional sacrifice then is training for concentration: we learn to distinguish the state of calm and freedom from the disorder of “desires, the imagination and discursive thought”. To be effortless, the will cannot oppose directly the sources of these disorders. Instead, we need to learn to detach from them, to observe them from a point of silence. Only then will they dissipate. This is the rational order of things, where the intellect is higher than, and dominates, thought and desire.

The season of Advent is the time of preparation for the incarnation of the Logos. Every physical event is the reflection of a spiritual reality and, since spiritual reality is timeless, it returns eternally. Hence, we can revisit the meditation on the second Arcanum as the Word is made Flesh.

There we see that the Logos is made incarnate by the Holy Spirit through the Holy Soul. As we are, the waters of our souls are full of perturbations. An event may create anger or anxiety, resulting in whirlpools that are hard to climb out from. There are tides coming in and out, so one day we say, believe, or vow one thing, and then the next day, just the opposite. Storms blow across the waters leaving them rough and choppy. In an unperturbed soul, the surface of the water is smooth line a pane of glass. Only in that condition will the Spirit be reflected clearly in the Soul. Otherwise, it gets mixed up with our fantasies and desires, which we too often take to be real expressions of the Spirit.

On the Feast of the Immaculate Conception we are reminded that only the Soul without sin can fully accept the Spirit. That Soul is totally free from perturbations. Can we even imagine what that is like? The exercise is worth the effort.

So either instead of, or in addition to, a physical fast, there is a Taoist exercise called mind fasting. If anyone has bothered to observe his thoughts, he will find abundant material to fast from. Perhaps there is a vulgar fantasy he indulges in. Perhaps he replays an event or conversations over and over in his mind. Perhaps he has a persistent anxiety or a worry. Maybe he has some daydream of success or power. Choose one and give it up. But give it up without effort.



\flrightit{Posted on 2012-12-08 by Cologero }

\begin{center}* * *\end{center}

\begin{footnotesize}\begin{sffamily}



\texttt{Senko on 2012-12-08 at 22:07 said: }

A most excellent post Cologero. Full of theory AND practical advice for one's everyday spiritual life. I find mental fasting much harder than physical fasting. I believe Advent is a most excellent time to prepare ourselves for the birth of Christ in our Heart. Blessings in Christ and Mary!\footnote{\url{http://senkosmos.blogspot.com.ar/}}


\hfill

\texttt{Mihai on 2012-12-09 at 14:43 said: }

Great post ! 

I would say that fasting has three different levels on which it must occur (like everything in the spiritual life, for that matter).

1. Physical: this is known by everyone and, unfortunetly, this is where it begins and ends in the mundane mentality of our age.

2. Psychic: just what you described here as “mind fasting”.

3. Spiritual: Without increasing prayer, meditation, study of Scripture etc. the whole thing collapses without purpose. 

I believe that when the fast turns into a petty legalism, limited to abstaining from certain foods and drinks, while ignoring the other two parts, it becomes even diabolical in character. I have seen a lot of people who do nothing but a physical fast and are most irritable and agressive during such a period.


\end{sffamily}\end{footnotesize}
