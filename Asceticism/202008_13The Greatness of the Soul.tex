\section{The Greatness of the Soul}

In \emph{De Quantitate Animae}, the \emph{Greatness of the Soul}, Saint Augustine describes a path that leads the soul from “its vivifying, perceptive, rational and contemplative powers that enable it to move close to God”. It is Augustine's understanding of the seven stages of the ascent to God. This work served as one of Dante's main influences.

This path is certainly consistent with other descriptions, so I have modified the names of some of the stages for clarity. It cannot be emphasized enough, this is not theory. In other words, these stages can be experienced, a fortiori, they must be experienced in order to be properly understood.

\begin{enumerate}
\item Animation 
\item Sensual Life 
\item Rational Life 
\item Autarky 
\item Ataraxia 
\item Theoria 
\item Theosis 
\end{enumerate}
\paragraph{Animation}
This refers to the vegetative soul, common to all life. Its characteristics are the nutritive, growth, and reproduction functions. Plants, fungi, and perhaps some lower animals have only the vegetative soul. It lacks consciousness. For higher beings it is the source of will and desire.

\paragraph{Sensual Life}
The sensitive or animal part of the Soul includes the powers of perception, sensation, and (willful) movement. It includes the external (sight, sound, smell, touch, taste) and internal (imagination, common sense, estimation, memory) senses. It is also the seat of emotions.

\paragraph{Rational Life}
While the vegetative and sensitive souls are connected to the operations of the body, the rational or intellectual soul exceeds the corporeal nature. As such, it is unique to human beings. Its functions are reason, knowledge of universals, ability to choose the good.

Since the rational soul is not a corporeal function, it cannot be natural; hence it is above nature. This, however, does not mean that there is no dependence on bodily functions or brains. It does mean that there is no natural process, e.g., genetics or neuronal activity, that cause the knowledge of universals or virtue. Specifically, it could not have “evolved” by any biological process.

\paragraph{Interlude on sensual and spiritual man}
\begin{quotex}
But the sensual man perceiveth not these things that are of the Spirit of God; for it is foolishness to him, and he cannot understand, because it is spiritually examined. But the spiritual man judgeth all things; and he himself is judged of no man. \flright{\textsc{1 Corinthians 2:14-15}}

\end{quotex}
The three aspects of the soul do not work harmoniously in humans anymore. People are centred predominantly in one of the three parts of the soul. Even the rational soul is used in service to the needs and activities of the body, not for higher things. These are the characteristics of the three types of sensual people.

\begin{itemize}
\item \textbf{Vegetative Soul}. This being is attached to sensations, thrills, sex, food, etc. They are usually men of action. His main fault is not using the will for higher things. 
\item \textbf{Sensual Soul}. This being is dominated by emotions, and reacts emotionally. He is likely to be sentimental or romantic. At its worse, he will be angry, malicious, anxious, etc. It is difficult to get him to think logically. 
\item \textbf{Rational Soul}. His focus is thinking, calculating, researching. He is on a constant search for the next big idea, the next book, where he believes he will find the ultimate answer. He will debate various verbal formulations. His fault is that he is unwilling to commit to a course of spiritual development; he would rather “know about” it rather than actually do it. 
\end{itemize}
Those embedded in the sensual life don't have a stable I, since it is driven by external factors: new experiences, emotions, new ideas. Most people prefer to live this natural life, unless some boundary situation or conversion experience pulls them out of slumber. At that point, they may be ready to embark on the true spiritual journey.

\paragraph{Autarky}
\begin{quotex}
It is necessary to be one in oneself (concentration without effort) and one with the spiritual world (to have a zone of silence in the soul) in order for a revelatory or actual spiritual experience to be able to take place. \flright{\textsc{Valentin Tomberg}}

\end{quotex}
Once a man has decided to transcend his natural life, the next two stages are critical. If he falters or fails at these stages, he will most likely be worse off than if he had never even begun a spiritual journey.

Autarky means to have dominion over one's soul. The consciousness of the natural man is driven by forces external to him, so his life is inconstant. The task, then, is to be able to concentrate on one's True Self. Then the psychological forces originating from below can be observed dispassionately. In the natural state, one's attention is hijacked by impulses, emotions, ideas, that arise spontaneously. In the concentrated state, one then learns to monitor and detach from such impulses.

Now, the rational soul can exercise its true function. No longer in submission to corporeal life, it turns to knowledge of the universals and virtue. The practice of the virtues, which is basically inner power, will lead to the strengthening of the I.

\paragraph{Ataraxia}
After Autarky has been achieved, attention can be directed from the world of the senses to the spiritual world. This requires the “Zone of Silence”, which is the reduction, or even elimination, of the perturbations of the soul. These include the automatic movements of thoughts, images, passions, personal desires, and so on. In other words, this is the purification of the mind and will.

Ataraxia is the inner state of tranquility and mental calm. Complete trust in God replaces trust in the world. When the soul is silent, then authentic spiritual experiences may take place.

\paragraph{Theoria}
This stage is inaugurated when interest in the things of the world is replaced by the desire of knowing what is true in the supreme degree. Meditation is the means. Once skilled in Concentration, Meditation is the next step. Learn to focus on an idea, or image, or some spiritual writing. Block off some time for it. However, you will find that, throughout the day, the meditation will spontaneously come back to you. You may even receive deeper insights about the object of meditation.

\paragraph{Theosis}
\begin{quotex}
Contemplation is the science of love, which is an infused loving knowledge of God, and which enlightens the soul, and at the same time kindles within her the fire of love, till she shall ascend upwards step by step unto God, her Creator, for it is love only that unites the soul and God. \flright{\textsc{Saint John of the Cross}, \emph{The Dark Night of the Soul}}

\end{quotex}
When even the words and images used in Meditation drop away, the next stage is contemplation. Saint Augustine describes it as the vision and contemplation of truth. As such, it is reason united with divine revelation. Tomberg describes it this way:

\begin{quotex}
Contemplation — which follows on from concentration and meditation — commences the very moment that discursive and logical thought is suspended. Discursive thought is satisfied when it arrives at a well-founded conclusion. Now, this conclusion is the point of departure for contemplation. It fathoms the profundity of this conclusion at which discursive thought arrives. Contemplation discovers a world within that which discursive thought simply verifies as “true”. 

\end{quotex}
This is why we insist that discursive thought is insufficient to understand the highest ideas of our Tradition. At best, discursive thinking gives the illusion of spiritual knowledge and at worse, it degenerates into a Wittgensteinian language game.

Beyond this, I am not in a position to explain more. That is reserved to the Saints that Dante places in the Empyrean: Augustine, Benedict, Francis. They should be your guides.



\flrightit{Posted on 2020-08-13 by Cologero }

\begin{center}* * *\end{center}

\begin{footnotesize}\begin{sffamily}



\texttt{Alan on 2020-08-14 at 01:05 said: }

“Detract not the king, no not in thy thought; and speak not evil of the rich man in thy private chamber: because even the birds of the air will carry thy voice, and he that hath wings will tell what thou hast said.” = unceasing silence.


\end{sffamily}\end{footnotesize}
