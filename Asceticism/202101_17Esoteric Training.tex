\section{Esoteric Training}

This project was conceived some 15 years ago in response to a challenge posed by Rene Guenon: To wit, the recovery of the Western Tradition. As exemplars, there are the civilizations of Medieval Europe and various Eastern Traditions. Hence we endeavored to reinterpret medieval doctrines in the light of Tradition with the aid of the corresponding notions from Oriental Metaphysics.

There are two types of texts: metaphysical and initiatory. Guenon's works are predominantly of the metaphysical type. Specifically, they describe doctrines and ideas in an abstract, impersonal way. Knowledge of doctrine is important, but in itself it is insufficient. It is only potential knowledge.

Initiatory texts are more personal and less explicitly metaphysical, since they describe the pathway from potential knowledge to actual knowledge. Since Guenon mentions Dante's \emph{Divine Comedy} as such an initiatory work, we have used it as the prime exemplar.

So in addition to discursive texts, we have also inaugurated personal training programs without which traditional doctrines are usually incomprehensible. It is usually obvious, when reading about Tradition, which writers have actual knowledge and which do not. What follows is a brief outline of the type of training that might be helpful in lieu of an esoteric school. Strictly speaking, due to individual differences, an initiation is not absolutely necessary, but group work will speed up the process.

Tradition describes the world of Being, essences, the necessary, which alone is fully real. The rest can only see the world of Becoming, the accidental, the contingent.

\paragraph{The Second Birth}
Concentration, meditation, prayer, participating in rite or sacraments, thoughtful reading, and moral purification of the will are preparatory exercises.

Guenon likens initiation to the second birth, aka, being born again. That is at the heart of the Western Tradition, as it was Jesus' nocturnal revelation to Nicodemus. No human individual can make you born again.

\paragraph{Facing the Shadow}
The starting point and ending point are the same, since it is a process: \emph{Know Thyself}.

You will learn exercises to facilitate that process. Initially, the focus is on negativity. Negative emotions, thoughts, and fantasies need to be recognized. The very act of becoming aware of them will attenuate their affects. Nevertheless, few people are willing to confront the Shadow, which is the necessary start to moral purification. Consider these two quotes from anonymous sources:

\emph{Each person has a black mark on his heart when he is born. Once he finds his calling in life, only then does the mark fade away. It is a sin, the sin people hide in their chests. A deep, hidden sin.}

\emph{Find the first truth that terrified you.}

What we fear and what we dislike are more important to self-knowledge than what we love. People reveal what they dislike on the false assumption that it reveals something about the object of their revulsion rather than about themselves.

It is often easier to notice that feature in others than in oneself. People often drop out of group work whenever they come near to discovering their black mark.

\paragraph{Primordial State}
Guenon recognizes three phases on the path to metaphysical realization:

\begin{enumerate}
\item The development of the possibilities of the human state. 
\item The development of supra-individual but conditioned states. 
\item The highest objective is the absolutely unconditioned state, free from all limitation. 
\end{enumerate}
The first step is to create a stable ``I" in the human state. This is Guenon's description:

\begin{quotex}
This realization of the integral individuality is described by all traditions as the restoration of what is called the ``primordial state" which is regarded as man's true estate and which moreover escapes some of the limitations characteristic of the ordinary state, notably that of the temporal condition. The person who attains this ``\textbf{primordial state}" is still only a human individual and is \emph{without effective possession of any supra-individual states}; he is nevertheless freed from time and the apparent succession of things is transformed for him into simultaneity; he consciously possesses a faculty which is unknown to the ordinary man and which one might call the ``sense of eternity." 

\end{quotex}
\subsection*{Supra-human States}
As described in the \emph{Symbolism of the Cross}, we can represent the human state as concentric circles on a plane. In that way we can visualize the first phase as the development of the I at the center. At the center, there is the vertical axis that can be ascended to higher states of being. This is how Guenon describes this phase.

\begin{quotex}
Its second phase corresponds to supra-individual but still conditioned states, though their conditions are quite different from those of the human state. Here, the world of man, previously mentioned, is completely and definitely exceeded. It must also be said that that which is exceeded is the world of forms in its widest meaning, comprising all possible individual states, for form is the common denominator of all these states; it is that which determines individuality as such. The being, which can no longer be called human, has henceforth left the ``flow of forms". 

\end{quotex}
In the \emph{Divine Comedy}, these states are represented by the angelic hierarchies up to the Primum Mobile.

\paragraph{Metaphysical Realization}
Ultimately, there is the Empyrean, which is beyond the conditions of time and space.

\begin{quotex}
The final goal of metaphysical realization; this end remains outside being and by comparison with it everything else is only a preparatory step. The highest objective is the absolutely unconditioned state, free from all limitation; for this reason it is completely inexpressible, and all that one can say of it must be conveyed in negative terms … The only things which have disappeared are the limiting conditions, which are negative, since they represent no more than a ``privation" in the Aristotelian sense. Also, far from being a kind of annihilation, as some Westerners believe, this final state is, on the contrary, absolute plenitude, the supreme reality in the face of which all else remains illusion. 

\end{quotex}
This state cannot be described. Even a poet with the skills of Dante has trouble doing so. He admits that he can't recall the particulars, but merely brings back impressions. To the human mind, God is incomprehensible, so only when he surrenders is his will aligned with God's and Dante knows that the created universe is bound by Love. Only in this state, is there a true Self. In Guenon's words:

\begin{quotex}
Action, whatever it may be, is not opposed to, and cannot banish, ignorance which is the root of all limitation; only knowledge can dispel ignorance as the light of the sun disperses darkness, and it is thus that the Self, the immutable and eternal principle of all manifest and unmanifest states, appears in its supreme reality. 

\end{quotex}


\flrightit{Posted on 2021-01-17 by Cologero }

\begin{center}* * *\end{center}

\begin{footnotesize}\begin{sffamily}



\texttt{gianthedgehog on 2023-02-04 at 06:34 said: }

The first thing that I remember to have absolutely terrified me was when, around the age of 5, I started to think about the eternity of time. My dad told me that after we die we'd be in Heaven forever, and I tried to imagine forever. Thinking about going on and on and on and on, without ever a ``point" of ending, of destination, of settlement, almost broke my mind. Yet, the alternative – ultimate annihilation and oblivion in the abyss – seemed no better, for, then, what even is this infinitesmal life that is literally nothing compared to eternity? Where does it come from and where does it go?

The only thing I could do then was to take my mind off the notion and engage in something that put me back into the here and now. I never really got over this issue, other than learning to skillfully evade this train of thought, which, I imagine, is what most people who have similar thoughts do.

On a sidenote, 2 years ago I had a mushroom trip (this was the event in my life that sparked my interest in metaphysics), and during that experience, in the absence of discursive thought, eternity felt natural and needed about just as much ``comperehension" as the act of seeing the colour red.

So, the question I probably have to answer is, what is it about my individual structure that, when faced with something as factual and simple as the eternity of time, feels so terrified that it would rather flee back to ignorance?


\hfill

\texttt{Arthur Konrad on 2023-02-04 at 18:27 said: }

@gianthedgehog 

I always felt that compared to the extremely sad fact of mortality, the question of eternity seems more like a bureaucratic concern


\hfill

\texttt{Kaukomieli on 2023-02-05 at 06:49 said: }

I think there might be a confusion between eternity and infinity in the concept of ``eternity of time".

Mortality itself is neutral in my view, it the related things such as illness, sickness, disease etc. that makes death un unpleasant fact of life. Death in itself is sacred and a necessary transformation in the kaleidoscopic metamorphosis of the universe.


\end{sffamily}\end{footnotesize}
