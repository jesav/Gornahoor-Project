\section{What must I read to be saved}

January is the open enrollment period for training in self-awareness and concentration. Please read to the bottom for more information. 

\begin{quotex}
What must I do to be saved? \flright{\textsc{Acts 16:30}}

\end{quotex}
In spite of that, I constantly get asked this one question, via all the communications platforms I am on:

\textbf{What must I read to be saved?}

\begin{wrapfigure}{rt}{.3\textwidth}
 \includegraphics[scale=.4]{a20220102WhatmustIreadtobesaved-img001.jpg} 
\end{wrapfigure}

Reading alone may be necessary for preparation, but it is not sufficient. When I took physics and chemistry classes at university, I had to give up Saturday mornings to do lab work. It wasn't always easy to get up early on weekends, but without the lab work, physics and chemistry are incomplete.

Nevertheless, everyone is looking for the ``magic book". And a month later, they will look for the next magic book because the first one had no effect on them. There is no such book, although many have claimed to benefit from a revealed text like the Bible or the Quran.

I have tried to recommend texts that will provide a firm foundation. Over the years, have I not recommended hundreds of texts by philosophers, poets, metaphysicians, theologians, mystics, scientists, and other thinkers? If you resonate with one or more of them, then read the texts yourself.

\paragraph{Self Study}
If that is not very systematic, then this list by Ananda Coomaraswamy is an excellent resource for self-study:

\begin{quotex}
A European can hardly be said to be adequately prepared for the study of the Vedanta unless he has acquired some knowledge and understanding of at least

\begin{itemize}[nosep]
\item Plato 
\item Philo 
\item Hermes Trismegistus 
\item Plotinus 
\item Gospel of John 
\item Dionysius the Areopagite 
\item Meister Eckhart 
\item Dante 
\end{itemize}

Eckhart, with the possible exception of Dante, can be regarded from an Indian point of view as the greatest of all Europeans. 

\end{quotex}
Carl Jung has a nice list of what he calls the \textit{Ten Pillars of the Bridge of the Spirit}. Even if you believe Jung is a danger or a fraud, the list is still valuable.

\begin{itemize}[nosep]
\item Gilgamesh epic 
\item I Ching 
\item Upanishads 
\item Tao Te Ching 
\item fragments of Heraclitus 
\item Gospel of John 
\item Letters of Paul 
\item Meister Eckhart 
\item Dante 
\item Faust 
\end{itemize}

\phantom{.}

Those two lists of texts will provide a solid grounding in metaphysics. If that is not enough then check out this nice list extracted from Valentin Tomberg:

\begin{itemize}[nosep]
\item Bernard of Clairvaux 
\item Thomas Aquinas 
\item Meister Eckhart 
\item Johannes Tauler 
\item Theologia Germanica 
\item Heinrich Suso 
\item Jan van Ruysbroeck 
\item Nicolas of Cusa 
\item Cornelius Agrippa 
\item Paracelsus 
\item Valentin Weigel 
\item Jacob Boehme 
\item Angelus Silesius 
\item St John of the Cross 
\end{itemize}

\phantom{.}

For a firm foundation of traditional metaphysics, I recommend these books by Rene Guenon:

\begin{itemize}
\item Man and his Becoming according to the Vedanta 
\item Symbolism of the Cross 
\item The Multiple States of the Being 
\end{itemize}
\paragraph{The most privileged men}
Haydar Amuli established the hierarchy of threefold division of thinkers.

\begin{enumerate}
\item The common people or the men of reason 
\item The privileged people or men of intuition 
\item The most privileged people or the men of both reason and intuition 
\end{enumerate}
The lowest level, men of reason, refers to profane philosophy. The middle level refers to mystics, who have the intuition of the Unity of Existence, but not the understanding. Those at the highest level have gnosis; they both know and understand the doctrine.

So where do you want to be in that hierarchy? Or are you content to be among those who have ``gone astray" and don't even bother to get started?

You need to be honest with yourself. How well do you actually read? Can you follow a difficult text? How well did you do on the reading section of standardized tests? You should realize at some point that you need some help to understand these texts.

Do you even understand what is meant by ``intuition"? Are you able to concentrate on an idea for an extended period of time?



\flrightit{Posted on 2022-01-02 by Cologero }

\begin{center}* * *\end{center}

\begin{footnotesize}\begin{sffamily}



\texttt{De Bosit on 2022-01-02 at 04:01 said: }

Beliving in the unity of all and being humble when reading is important.

Just assuming you know instantly… You will just reinforce your existing views.

Cryptic Alchemical texts like Boehmes Signature of All Things will not make sense, but trying to understand, believing there IS something there and pondering passages that stood out to you might get you far.

Also you have to fall in LOVE with what you are doing. It's called philosophy for a reason.

If that spark isn't there, just live your life like you are.


\hfill

\texttt{Santiago on 2022-05-03 at 19:08 said: }

This is a great post, although I believe that it's also appropriate to label and recommend material based on ones own progression. I base the following recommendations largely off of my own errors and mistakes, in the hopes that it might help someone else. 

The writings of Guenon, and to a lesser extent, Evola, are very useful, but they can easily lead a person astray if taken too far, too seriously, or without a broader perspective. In retrospect, I feel one of the largest appeals of those authors is their biting critiques of modernity and their intellectual summary of traditional metaphysics. But these are not enough to present someone with a path (much less a healthy worldview, in some cases) and past a certain point, they may even present an obstacle.

Mouravieff classifies the stages of ones progression as being divided into 3 primary stages (see fig. 56, vol. i of Gnosis): `exoterism’, ‘mesoterism'. and `esoterism' proper. Dividing each stage is a `threshold' which transitions one from one stage to another. Perhaps we should recommend material based on this scheme.

While it might sound strange, perhaps it's best to treat the writings of the `Traditionalists' as being suited almost exclusively to the `exoteric' domain in this progression. Reading them can do very little for ones spiritual work, outside of providing an initial impetus. Alternatively, it might be recommended to concentrate on the Bible, the writings of the saints, church doctors and fathers, etc. for this stage.

After crossing the `first threshold'. reading should take a definitive backseat in favor of personal development. Speaking for myself, this cannot be stressed enough. Mouravieff posits this stage (that of the mesoteric) as being pictorially represented by an ascending case of three stairs. Fittingly, I will recommend to concentrate almost exclusively on three authors for the duration of this stage. The decision is immense because one will spend years upon years in this zone, trying to improve oneself, so the following authors are selected with extreme prejudice.

We can apply a correspondence between the three stairs mentioned above with the traditional understanding of the human personality: that is, of the person composed of intellect, emotion and of bodily will. 

First, I will recommend the writings of Vladimir Solovyov (in particular, Russia and the Universal Church), for it is this book which will present a healthy social outlook, a progressive goal to hope for in the religious and political realm, a great history lesson, a comprehensive metaphysic, and a teleology for all of ones personal efforts along the Way. It is for these reasons that Solovyov can guide the intellect of the the hermetic neophyte.

For development of the heart, it is without a doubt the writings of Valentin Tomberg (and in particular, MOTT) that can serve as a `textbook' guide. The passages of this book should be taken individually, with reading encompassing no more than a few pages at a time. Such is the demand of their emotional and spiritual depth, and they will produce great inspiration.

Lastly, for a program of the will, Mouravieff presents his comprehensive system of interior work and personal development, and plots out every step along an individuals path, and all major points of consideration as regards the readers individual `type'. This is the one author for whom it might be useful to take `notes' as one reads.

After crossing the second threshold, and entering the domain of the esoteric, no reading can be prescribed, as ideally the student would already possess it in spirit.


\end{sffamily}\end{footnotesize}
