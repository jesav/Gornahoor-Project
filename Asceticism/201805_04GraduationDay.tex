\section{Graduation Day}

Here are some suggested spiritual practices and how to know when success is achieved.

\paragraph{Detachment and Spiritual Combat}
The fundamental bases of our esoteric practice are detachment and spiritual combat. A sense of detachment needs to be developed. That is, you become the observer of your own life. You watch yourself make coffee, you watch yourself shave. Whenever you can, you wake up and observe.

In spiritual combat, you guard against the thoughts that enter your mind. You don't control what thoughts arise and when, but you do have some measure of control over what thoughts and images you allow to take root in your mind. If you are not diligent, then they can become obsessions, making it more difficult to root them out.

\paragraph{Intentional Suffering}
You must be willing to make sacrifices when necessary. The first thing to sacrifice is negativity. Yes, people love their negativity and are reluctant to give it up.

Suffering out of love is of a different character. If you give up your meal so your child does not go hungry, your pain turns to joy as you see his hunger satisfied.

Abstinence is the intentional denial of certain beneficial foods for a higher purpose. A common one is to give up meat. We see that in the story of Cain and Abel. Ask yourself what they had for dinner.

Cain had a loaf of bread, since that is all he had. Abel also ate a loaf of bread because he had sacrificed his meat earlier. That is why Abel's sacrifice was accepted; Cain didn't really sacrifice anything.

Purgatory is often misunderstood as a form of punishment. Rather it is a purification, because that process was neglected on earth. According to St. Catherine of Genoa, the souls in purgatory are happy because they know they are doing God's will.

\paragraph{The Persona and Conscious Acting}
The soul life contains many persona, which are the roles we play in our family and social lives. As such, they are necessary. However, there is a problem when the Ego starts identifying with one of those roles, or Persona. We need to be aware of the roles we play. Eventually, you can start playing them consciously and deliberately.

\paragraph{Collective Unconscious}
The task of individuation is achieved when there is harmony and equilibrium between the spontaneity of the unconscious and deliberate action by the conscious Ego. To maintain this equilibrium, unconscious processes compensate for the conscious ego to balance the psyche as a whole. These unconscious processes will correct for the egotistical bundle of personal wishes, fears, hopes, and ambitions. However, self-knowledge will ameliorate the effects of these processes, which manifest as irrational and uncontrollable impulses. As consciousness is widened, the actions of the ego will be freed from these unconscious corrections.

Clearing away the fog of this personal unconscious will reveal a deeper layer, called the collective unconscious. Man's mind is not a blank slate. Rather, the common heritage of the human race is contained in the mind as archetypes, which manifest themselves endlessly in history. Jung gives as examples the parent archetype, which is the reflection of the parent in the child. Of course, the collective imago is intertwined with a personal imago, since the child cannot fully understand the parent. There is also the imago of the child itself. This allows for the spontaneity of play. One can become child-like in this regard, which is not at all the same as being childish.

A man will have an interior image of a woman (anima) and a woman, that of a man (animus). The Sage is another one, since everyone tries to be wise to some extent in his own life. The human mind is prepared for psychic life just as the body is prepared for material life. It is inborn in him. Jung writes about it:

\begin{quotex}
The whole nature of man presupposes woman, both physically and spiritually. His system is tuned in to woman from the start, just as it is prepared for a quite definite world where there is water, light, air, salt, carbohydrates, etc. The form of the world into which he is born is already inborn in him as a virtual image. Likewise parents, wife, children, birth, and death are inborn in him as virtual images, psychic aptitudes. These a prior categories have by nature a collective character; they are images of parents, wife, and children in general, and are not individual predestinations. … They are in a sense the deposits of all our ancestral experiences, but they are not the experiences themselves. 

\end{quotex}
\paragraph{Psychology and Esoterism}
When Jung writes as a scientist, he must necessarily limit himself to the phenomena of psychic (i.e., psychological) events. Whether there is something beyond the phenomena is opaque to the scientific method.

On the other hand, the esoterist is definitely interested in what is beyond the merely psychic layer of life. Now Jung in his personal life did indeed believe in the reality of a noumenal reality beyond phenomena. Nevertheless, we are less interested in his spiritual writings, but his immense psychological knowledge can be used profitably.

\paragraph{Megalomania}
There are two main obstacles to spiritual development: the tendency to megalomania and sexual obsession. The former manifests as an unjustified high opinion of oneself. Jung write about several such cases. Such an individual will underrate, belittle, or criticize others, especially those close to him. He shows how unconscious processes will compensate for those feelings.

However, on our path, we don't rely on such unconscious processes. Rather, our method is deliberate and conscious. We notice that this megalomania manifests as “taking accounts”, that is, we are keeping track of all the alleged snubs and mistreatments we have received from others. By observing this tendency in a detached way, the Self will compensate consciously and we make find that such feelings simply dissipate.

It is necessary to “wake up” somehow in order to catch such manifestations “in the bud” as it were, before they can catch hold.

\paragraph{Song of Innocence}
It is accepted that childhood is a time of innocence free from overt, \emph{pace} Freud, sexual thoughts. This is followed by the time of experience. Once again, the notion that sexual activity in itself is somehow bad needs to be dispelled. Rather the issue is the misuse, or better said, the reuse of sexual energy. After the son of duty, different options present themselves. Like the faster, one can give up sex as a form of deliberate suffering, not because it is bad.

The goal is self-mastery, but it soon becomes clear that the desire is quite intense. It doesn't suffice not to engage in sex, because once the idea of it comes into the mind, it will recur, usually in an even stronger way. So it is really a spiritual rather than a material battle. It is a matter of guarding the thoughts.

Perhaps one can remain in the state of innocence for an hour, for a day, even for several days. One becomes a child again, spontaneous and innocent. You stop looking at women with that leering look; you know what that feels like.

Freud wrote of the state of polymorphous perversity, in which sexual energy is not confined to the genital region. The entire body can be on edge. Sometimes it can be difficult to even look at a pretty woman, because the body feels so intense; you may need to shield your eyes in that case.

Often, you just feel really good all day long.

\paragraph{Anima}
The imago of a woman in a man is called the Anima. Jung describes it:

\begin{quotex}
Woman, with her very dissimilar psychology, is and always has been a source of information about things for which a man has no eyes. She can be his inspiration; her intuitive capacity, often superior to man's, can give him timely warning, and her feeling, always directed towards the personal, can show him ways which his own less personally accented feeling would never have discovered. 

\end{quotex}
I want to focus on the esoteric teaching rather than the psychological, although they are compatible. Originally the human being was whole, but such as we are (speaking from a man's point of view), there is a gap in consciousness where the woman belongs. There is one, and only one, who fits perfectly. She may not be found, so there are others who fit less perfectly; some are closer than others.

The two who find each other are called Polar Beings. From an experiential point of view, it suddenly seems as though another person is sharing the same body. Her thoughts become mixed up with your own thoughts, her feelings are your feelings. Times of separation are an ache, times together a joy.

There are states beyond this one, provided certain tests of purity are met. I don't know those states personally, so I'll hold off for now.

\paragraph{Self}
The end result of individuation is the birth of a Self, or Real I, beyond the conscious Ego. The Self is not scientific because it can never be an object of thought. It is noumenal and beyond the psyche. It includes the Ego as well as all the unconscious processes. However, the Self integrates them all into a whole, eliminating the multiple independent little selves that seem to exist.

From the psychological point of view, Jung claims that the experience of the Self and the experience of God are the same.

St. Catherine of Genoa goes beyond that when she writes:

\begin{quotex}
My Being is God, not by some simple participation but by a true transformation of my Being. 

\end{quotex}
\paragraph{Graduation Day}
Quite some time ago, I had been participating in a group for a few years. I noticed the repetitive nature of it and no one seemed to be progressing. There were the same stammerings, the same excuses, the same struggles. One woman expressed the joy she felt communing with the squirrels in the morning.

I suspected most were lying, because my descriptions of the inner life were more vivid — and shocking — than theirs. I had assumed that the two group leaders were following some special program. I learned later that they were just winging it. Moreover, once you moved beyond their own limited frame of reference, they sounded lost. And there was no metaphysical teaching sustaining it.

Not that it wasn't helpful to me, just the opposite. One night I asked the group leader, “When do we graduate?” His response was simple, “Just graduate,” he replied. So I never went back after that.

You graduate when the burden becomes light, as St. Catherine described. You have achieved more self-awareness. You are liberating yourself from the lower forces that enthrall you with anxiety, despair, worry, uncertainty, obsessions. There is no need for any of that.



\flrightit{Posted on 2018-05-04 by Cologero }

\begin{center}* * *\end{center}

\begin{footnotesize}\begin{sffamily}



\texttt{Lyon on 2018-05-06 at 16:10 said: }

This was brilliant on many levels. Thank you! 

One question for you…

You mentioned Freud in your post. My surface-level take on him is that he is a charlatan of the highest order and that his ideas are deleterious. What is your general assessment of him? Apologies if your explicit views on Freud have already been covered in your writings. I am a rather recent visitor to this site.

And a comment…

The St, Catherine of Genoa quotes were profound and sublime. The insights in her quoted words, used judiciously in your post, I found both deeply rooted and uplifting. I never heard of her before visiting this site (…I recalled you mentioning her previously in a post, so at the time I did a quick internet search of her; I will go back to read up on her further).

Thanks again!


\hfill

\texttt{Dennis on 2018-05-08 at 15:12 said: }

My problem isn't getting to “graduation day”, but knowing where to begin the journey properly in the first place in order to even chart a path to “graduation day.” There are just so many books, resources (such as this blog – which itself would take forever to read and digest systematically from beginning to end), etc., that just knowing where to begin and how to separate the wheat from the chaff, so to speak, is daunting. I find myself often flitting from one book to the next, one thinker to the next, with no clear plan or goal, and in a rush of enthusiasms that soon fade, then I move on, hoping the next book or thinker will be the that changes everything for me…and on and on… Any ideas or recommendations from anyone on where and how best to begin?


\hfill

\texttt{F on 2018-05-10 at 09:29 said: }

With the right desire the Bhagavad Gita and Boehmes Way to Christ should be enough to get you started.


\hfill

\texttt{Sean on 2018-05-11 at 12:29 said: }

Not sure how to reply to someone but this is for Dennis.

Dont worry about reading the right books. Start with simple actions in your daily life. I'll give 5 I started with. 

1. Make your bed every day

2. Do some physical activity, a walk, a few push-ups, anything. No need to go to the gym if it feels daunting.

3. Quit all drugs/drinking save special occasions like wine at a wedding.

4. Sacrifice some sort of food. I started with processed carbs and made my way to all meat except wild caught fish. (still have difficulty with sugar)

5. Think positive thoughts about all you people you encounter. It's easy to judge that obese lady who buys nothing but Pepsi at Walmart but try and only let yourself think something positive about her. Even if you can't confirm it. Sometimes I'll think “I bet she's a really great aunt” or something.


\hfill

\texttt{lenamaria5 on 2018-05-15 at 19:36 said: }

@Dennis, Sean gave some good suggestions. I have also struggled with flitting from book to book. Perhaps that's a product of my (our?) generation's digital consumption habits and general lack of focus. It is much better to concentrate one's efforts on a single work or thinker for a longer period, aiming for depth and not breadth. Read meditatively, rather than just seeking information. Set a much longer timeline than you think you need for a work. If you've read through it before the time is up, you've read too quickly. Gornahoor has a post on Lectio Divina that's very good, if you're interested in Holy Scripture.


\end{sffamily}\end{footnotesize}
