\section{The Purification of the Breath}

\begin{quotex}
The way of the spirit is such that it does not exist for those who do not want to follow the path. \flright{\textsc{Julius Evola}}

This is the concluding section of Evola's essay on the metaphysics of purity. Once again, he expresses Eastern doctrines not only in Western terms, but also relates them to actual experiences rather than empty doctrines. He referred to this elsewhere as “metaphysical positivism”. This puts him at odds with those who deny the possibility of higher states, but, more importantly, against those of a new age bent who tend to obfuscate these doctrines due to their incomprehension.

Since there now seems to be an audience that can benefit from such writings, I may translate the related chapters of \emph{Essays in Magical Idealism} provided my energy holds up. 

\end{quotex}
It is sufficient to mention the \textit{kundalini} yoga of the \textit{Shakti tantra}, that will permit us moreover to say something about the “purification of the breath”. Roughly speaking, its meaning is the following: there is a force (\textit{kundalini}) in man that is the root of his individual unity and a principle superior to every polarity or duality. It is the Logos (\textit{shabda brahman}) in the body. However in the common man, this force appears only in an extroverted and impure form (impure because it is turned to the other) of the power of animal generation. It is a matter of realizing kundalini in consciousness, of grasping it, of separating it from that extroverted direction and then folding it on itself in a point of mastery and sufficiency. Then kundalini, restored to its true nature, makes itself the instrument for the reaffirmation of the I over all those principles that rule his physical, biological, and mental being and that previously fell outside his power. Now in order to achieve such a conversion it is necessary that kundalini be invested by something already pure and entire in order to communicate these characteristics to him. That may be \textit{prana}.

\textit{Prana} for the Orientals is the life force connected to the breath and, as one, transcends the material breach with which it can be said that it is in the same relationship as the soul to the body. Moreover, it is necessary to remember that in the metaphysics connected to such disciplines, the highest meaning of creation would be given by the mantra Hamsah: Ham would be the inhalation, Sah the exhalation, therefore hamsah is the simultaneity of inhalation and exhalation. Naturally, exhalation is symbolic of \textit{proodos} [progress], of the act of pure, demiurgic creativity; inhalation is the further power, through which the central principle “originated”, is reaffirmed, recognized, and mastered. \textit{Hamsah} has the meaning of an eternal, simple sudden intuition, the synthesis of being and non-being (sadasat) in which the Absolute makes use of a pure self-revelation or self-giving to itself. Now \textit{hamsah}, which is that in which all beings “live, move, and are” [Acts 17:28], is present also in man, but in an impure and divided form: in man, life (\textit{prana}) is no longer fixed and concentrated simultaneously, but rather it comes and goes with alternating inhalations and exhalations, in a fluctuation and a contingency that reflects the ultimate of the first inhalation of the newborn and the last exhalation of the dying man.

The yogi, through the appropriate discipline (\textit{pranayama}), directs himself to dissipate this impurity. The I must seize prana, pulling it out from fluctuation, keeping it fixed in its own body. With such concentrated and total power, he then proceeds to empower kundalini; then it is “awakened”, it detaches itself from the extroverted direction that craving desires and is related in itself; it no longer flows downward, but “upward” (\textit{urdhvareta}) [a yogi who has accomplished perpetual sublimation of semen]. In that animal generation (heterogeneration) yields to that of the gods or spiritual generation (autoctisi) [in the philosophy of Gentile, the act in which the spirit creates itself]. After making itself pure and individual activity, kundalini progressively empowers various subtle centers and changes the dualities in them into actual simplicity; that means: it realizes in the I a relationship of identity and mastery with those spiritual powers that, close to the state of privation and obscurity of the body, is opposed to it as physical nature. At the limit of the process there is the purification of that which was sexual coupling—i.e., pure auto-generation—and, in that way, the supreme liberation (\textit{paramukti}). It is said that for anyone who is elevated to this point neither the body, nor the “other”, nor dissolution, nor destiny of rebirth exist any more (\textit{kaivalya}); he lives in accordance with pure activity that previously he felt as obscurity and privation: the various functions are reawakened and intensified in the original and glorious nature of cosmic power. In particular, the I can generate, he can give himself from himself a body while also maintaining and changing it at will. Lord of the laws of life and death, he is satchitananda, i.e., conscious actuality (\textit{cit}) and insofar as perfect (\textit{sat}), blessed (\textit{ananda}).

\paragraph{Conclusion}
We believe that this brief essay on one of the most important initiatic doctrines is not entirely lacking interest; and we would certainly be pleased if anyone succeeds in getting from it some idea of the possibility of a consideration of similar topics, that go beyond both the limited attitude of those who, enclosed in rather restricted horizons, only know how to scorn and ridicule, as well as those who, like an octopus, love to muddy the waters and take for mystery and the “occult” what they are unable to understand, which on the contrary they only know how to deform with a great deal of prejudices.: and in this group, we can include almost the totality of those who speak of “occult science”.

There is instead a way to consider initiation, through which it presents a perfectly intelligible content valid in itself, proceeding moreover from a concept of man (\textit{en sarki peripolon theos}) [a god walking in the flesh], of his value, and his task, lofty and magnificent like few others. We will certainly not stop here on this question of the real possibility of similar paths. In any case, we should not neglect how much metapsychics today is little by little verifying as effectively possible for man. On the other hand, it remains in fact that a number of things are impossible only because we do not believe that they are such; and that \textit{the way of the spirit is such that it does not exist for those who do not want to follow the path}.



\flrightit{Posted on 2013-11-15 by Cologero }
