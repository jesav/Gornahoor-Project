\section{Immaterial Intimacy and Material Catastrophes}

\paragraph{Introduction by Cologero}

\begin{quotex}
and directing his gaze from now, on towards beauty as a whole, he should turn to the great ocean of beauty, and in contemplation of it give birth to many beautiful and magnificent speeches and thoughts in the abundance of philosophy. \flright{\textit{Diotima to Socrates in Plato's Symposium}}

\end{quotex}
Diotima was a prophetess and philosopher. In the Symposium, she explains the ladder of love, which comprises the following stages.

\begin{enumerate}
\item In youth one Is attracted to the physical beauty of the other. 
\item At this stage, the spiritual beauty of another person is more important. Therefore, love is now directed towards this person because of her moral, intellectual, and spiritual qualities. 
\item The beauty of knowledge itself becomes the focus. This is the love of Wisdom, or philosophy. 
\item Ultimately, one reaches the appreciation of Beauty apart from any individual, to consideration of Divinity, the source of Beauty, to love of Divinity. 
\end{enumerate}
If Diotima is speaking as a philosopher, those stages make sense. But not if she is speaking as a prophetess of the future since precious few manage to get beyond the first stage. However, in the timeless realm, the notions of past, present, and future are relative. In the Future of Prophecy, Saint Gregory the Great\footnote{\url{https://www.meditationsonthetarot.com/the-future-of-prophecy}} explains the three tenses of prophecy: future, present, past:

\begin{quotex}
prophecy is not a matter of prediction, but rather of revelation. That is why there can be a prophecy of the past and the present; prophecy uncovers hidden truths, truths concealed by time or by present circumstances. He writes: “Prophecy is present when something is concealed, not by the spirit, but by the absent Word, which however is laid bare by the Spirit.” 

\end{quotex}
Sibylle has done her own experiment in making the past present, through the act of remembering. She had been working on those memories for a long time and until recently never figured out what to make of them. Suddenly, their meaning was revealed to her. That is how she became Sibylle.

You can find her experience in the New Decameron at Immaterial Intimacy and Material Catastrophes\footnote{\url{https://gornahoor.net/decameron/immaterial-intimacy-and-material-catastrophes/}}.

As you try your own experiment, notice the extent that you have contributed to the events of your life, whether out of ignorance, by accident, or deliberately.


\hfill

\paragraph{The Chewing Gum Boy}

She really didn't like the boys in Kindergarten. They constantly bullied her, said mean words, started laughing in her presence. There was one boy especially cruel, quite obscene for a five-year-old. Very blonde, strong and he kept singing Nena's “99 Luftballons” using a huge wooden building brick as a microphone. One day when she went to the supermarket with her mother, they encountered this boy with his mother. The two mothers started talking and he was too embarrassed to look at her. She didn't care because she thought she didn't like him. All of a sudden, the boy reached out and handed her a chewing gum. When he finally looked at her, she knew that cruel boys would never pose a danger to her. They revealed their true feelings by their first glance.

\begin{wrapfigure}{rt}{.3\textwidth}
\includegraphics[scale=.5]{a20210215ImmaterialIntimacy-img001.jpg} 
\caption{Diotima}
\end{wrapfigure}

\paragraph{Ending up in the Boys Corner}
On her first day of elementary school, the girls of her class promised each other to stick together in order to avoid sitting next to a boy. When the teacher announced everybody should choose a seat, her best friend ran away to sit with another girl. So she picked a nice window seat next to a long-haired, fair looking girl. Suddenly the girls started laughing at her. When she looked around, she realised she was surrounded by boys. The fair looking girl turned out to be a boy — she spent the rest of the schoolyear by his side, but never figured out what he was either thinking or feeling. He remained completely alien to her.

\paragraph{The Teacher}
When she was thirteen, she switched violin teachers. Her father had taught her since she was four and made his daughter the winner of a several local music competitions. Now that she was supposed to win the national youth competition the stakes were high. A new coach was needed: a motivating, inspiring presence who would make her leap to the next level. The Teacher was indeed motivating and inspiring — mainly because she was terribly in love with him. She would have lessons at his home, eat lunch with his parents — he only touched her slightly, she told me that it felt like a breeze on her neck and arms. So she won the competition, everyone was happy except for her. Why? The Teacher was a decent man: winning was nice — but she wanted Love.

\paragraph{Love gone unnoticed}
After she dropped out of High School when she was 17 to advance her career, she went on a concert tour in the Czech Republic with two very sweet boys, a cellist and a pianist. They were part of a group of prizewinners of an international music competition playing concerts in Prague and Bohemia. She fell in love with the pianist, spent her evenings kissing him and enjoyed herself until one morning she discovered the ashes of a letter on the doorsteps of her motel room. She had no clue what that was all about. It turned out that the cellist had fallen in love with her, planned to open up to her in a love letter, but found out about her and the pianist. She hadn't noticed anything. She didn't feel sorry for him then. Maybe she does now.

\paragraph{The Tragedy of Napoli and Thanksgiving in Chicago}
During her first year at Indiana University, she met the Italian. He started calling her by his own last name as if they were already married. In fact they had barely exchanged a kiss. During the summer she went to visit him in Naples. Her parents had rented a house near Pisa for the holiday season and she took the train down to the South. She felt confident, had learned Italian in six weeks and was looking forward to meeting her future husband again. He was very passionate upon her arrival, but fell short of making love to her. Instead he called his best friend, a manic-depressive art student from Bloomington, who was spending his summer travelling around Italy. For the rest of her stay she spent more time with that best friend than with her supposed future husband. Back in Bloomington she kept hearing stories about him — that he had broken so many hearts, sleeping with every girl in town. Just not with her.

On Thanksgiving Day she decided to change the Matrix. Instead of letting things happen to her she allowed a student from Venezuela to take her virginity in a motel room in Chicago. He got mad at her for not telling him that it was her first time. She had simply assumed he wouldn't care and was stunned that he felt responsible for what she had made him do. Her first encounter with wrongly guided willfulness.

\paragraph{Mr. Nice Guy}
She did not love him at first. But she thought it was time to settle with a “nice” guy and found a seemingly willing partner in a German cellist she studied with in Amsterdam. They came from the same prude-protestant background and did everything right. Meeting both parents, spending time with his ersatz father — a highly skilled psychologist who kept emphasising how good she was for him after his ill-fated relationship with his former girlfriend. Still: he loved his ex-girlfriend and no other woman could make him happy. A couple of years ago the nice guy's ex-girlfriend committed suicide.

\paragraph{Another Change to the Matrix}
She tried another change to the Matrix. She felt trapped and pushed by external expectations and wanted to be left alone. So fifteen years ago she went online, picked herself a husband from a dating site, became pregnant and got married. Her husband was her best friend for a while. But he always seemed emotionally detached. When she finally managed to catch a glimpse of his Anima, she got scared. He was totally unaware of her. Everything she actively achieved by \emph{free movements} seemed like a catastrophe to her.

Some time ago he confessed to her that he is still trying to cope with his first girlfriend breaking up with him 35 years ago. If he had told her that fifteen years ago, she would have never married him.

So mere willfulness is not a strength in itself? I think she knows that by now. She has tried to change the Matrix twice and just created a Matrix inside the Matrix.

\paragraph{Sybil}
I was conceived spontaneously. Almost accidentally, not willingly. I could have remained silent forever. I don't know why I didn't. She made me write and delete and then rewrite a comment before eventually sending it. She wasn't really expecting an answer, she was just fed up with the passivity of men in her life, had a weak moment and made me comment on some really beautiful and deep thoughts on Love.

But when I think about it — there was more to it than her weakness. She was intrigued by the thought that skin would become an obstacle to intimacy. Since she had experienced her most intimate moments with little or no physical contact throughout her life, she seemed puzzled by the idea that this could actually be a Path and not a flaw.

Me: But you used to be so sceptical of Diotima's talk.

Her: Yes, but that was before.

Me: Before what?

Her: Before he listened to what I meant instead of what I said. He didn't close the door even if he really wanted to. But he said “Stay with me” instead. By knowing me better than I knew myself he acted like a true Knight.

Me: Now you sound like an overly romantic scriptwriter.

Her: It's better than watching the same old film over and over again.



\flrightit{Posted on 2021-02-14 by Sibylle }
