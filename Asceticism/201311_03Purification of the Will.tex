\section{Purification of the Will}

\begin{quotex}
This is a section from an article titled “Purity as a Metaphysical Value” published in Bilychnis, the journal of the Baptist Theological School of Rome. It is dated June, 1925, volume XXV. \textbf{Julius Evola} was around 27 years old at the time, which is about the median age of Gornahoor visitors. The section that preceded it is called “Purification of the Mind” and the following sections deal with the purification of the word, sex, and breath.

There are some interesting points here. First of all, there is the admission that he underwent psychoanalysis; this should not be surprising given that there were two psychoanalysts who participated in the UR group. The other is his rather odd experience of the purification of the will, wherein he felt he could not move. Has anyone else felt that? Obviously, he means something different from mere indecision, or does he?


Perhaps he did not make himself clear, but his pairs of opposites do not seem to derive from Plotinus, since the primal will is for the Good. True, the One transcends all opposites. Nevertheless, if one is not motivated by pleasure or desire, one must transcend them, i.e., become “superior” to them. The important point is that Evola is here advocating a rather austere ethic: choose the necessary over the pleasurable, the difficult over the easy, i.e., choose the path of greatest resistance. A man needs to become master of his domain, self-motivated and self-ruled. But what self? 

\end{quotex}
The impurity of the will consists in heteronomy, i.e., in its being determined by something other than itself. In western culture, because of the prevailing extroversion, the conviction that every action must have a sufficient reason is widespread, or that there must be a reason or a cause for its happening or not happening, for its happening in one way and not otherwise, and this has joined with thinking that things do not happen in a different way through a divine act. It is precisely such a mode of action that is called impure. In fact, in it, action draws its own impulse not from itself, but from a motive, reason, impulse, or object by way of appetite or aversion, etc. The will to the thing desired does not want only and nakedly itself, but something else, so that it is properly said that it is willed by that other. That is the \textit{sakama karma} of the Orientals: action based on desire, action that is not through itself, but through what proceeds from it. Purification in this case is instead connected to the conviction that \textit{the sufficient reason of an affirmation can be the affirmation itself}, rather than to the concept of an act that is done only from a pure creative impulse. One can also find the best expression of this in a passage from Eckhart:

\begin{quotex}
From this deeper principle you must do your works, without a why. I affirm it decisively: even if you do your works for the kingdom of heaven, for God, or your sanctity, although motivated by the other, even then you will not really be in the right. If you ask a true man, a man who acts from his depth: “Why do you do all your works?” he will answer you rightly only if he says: “I act only for the action itself.” 

\end{quotex}
Here it is rather important to note that the need for purification assigns both the pure and the impure, the good and the bad, in a word: not particular terms but the combination of pairs of opposites. The purity in question means full autonomy, pure possession of oneself, and in respect to that, the link to the good, the sacred, etc., is no better than any other link: if that which is called good or pure by men binds the will, that is likewise to be called impure. Hence in such order, they appeal to expressions like cleansing oneself, baring oneself. \textit{Afele panta} [from \textbf{Plotinus}, “forsake everything”]: it is necessary to cleanse oneself of everything — the “high” as the “low”, the “spiritual” as the “material” — it is necessary to reduce the will to its naked essence resting only on itself. Once that point is reached, \textit{everything} becomes equally pure, just as prior to that \textit{everything} is equally impure. And that in such an order the pure must not be said of things in themselves, but of a way to live them, the measure of which is autonomy and autarchy, so that in being compelled to call something impure, only the proof of its own impurity is conveyed.

Here a particularly subtle discipline is necessary for the fulfillment of the requirement. In fact, how to guarantee that what one wills proceeds truly from the unconditioned and not from an obscure, incomprehensible complex of inclinations and impressions rooted in the subconscious? The answer is an in-depth self-analysis, to make everything progressively emerge into the light of consciousness that previously had been taken away. Even among us today we begin to work in this direction with psychoanalysis. Beyond that, there are methods of control based on the principle, that depending on whether action is conformed or not to a hidden inclination, it will produce pleasure or aversion. Along these lines, it is not enough to believe that the alternative is indifferent, or to put aside one's own will and to leave decisions to chance; for example, by the flip of a coin. In the sentiment that results from it and extending this discipline to a topic that always more closely concerns us, we will have a real means of indicating the progress or decline along the path of the purification of the will.

In general: it is necessary to renounce everything once one feels that it becomes necessary, or once one uncovers a desire or satisfaction for it; \textit{it is necessary to do, on principle, not as one pleases, but what is required, to always take, on principle, the line of greatest resistance and, thereby, to make the will ever stronger and purer, to make self-possession ever more energizing}. Hard discipline, which one would hardly know how to adapt to unless one succeeds in feeling in the naked will, in autarchy, a stronger motive and a more intense and vaster pleasure than what things in themselves can ever offer us.

In any case, it leads to a rather difficult point, whose reflection is precisely the difficulty that common knowledge meets in conceiving an action where there is no longer a “because” to arouse it. One feels as if the entire inner being were crystallized, so that no movement is any longer possible: it is like a paralysis, an absolute aphasia, that contrasts painfully with the sense of inner possibility. Almost as if one had something to say but the mouth remains mute and inert to command.

The experience of such an inner state provides the sign of purification and for that reason the individual knows how little what he called his action was truly his, how much a real impulse was absent from his ordinary life, “higher” or “lower”, and he being not the author, but a puppet, a medium blown about by alien forces. He knows however the I, and beyond that, how to find an excess of strength, he knows in spite of all acting, that he has achieved in himself the principle of a higher life, a power that stands beyond his being made from dependency, contingency, and finitude. And the door for that higher accomplishment, which is connected to the remaining purifications, is disclosed to him.



\flrightit{Posted on 2013-11-03 by Aeneas }

\begin{center}* * *\end{center}

\begin{footnotesize}\begin{sffamily}



\texttt{n0e on 2013-11-04 at 08:22 said: }

Psychoanalysis in Evola is, i suppose, just a modern way of achieving alchemical result, uniting Sun and Moon, conciousness and unconciousness. Which would be a necessary precondition for developing maximum control by the Will of the Self.


\hfill

\texttt{JA on 2013-11-04 at 08:57 said: }

This is very similar to Crowley's teachings in Liber Aleph. I've always seen C and E as quite similar personae, I think if E had not made the fortunate encounter with Guenon's writings Baron Evola would have wound up as the Italian version of AC. 

For clarity, let's point out that this written before the Baron discovered Guenon, at the time he was still working within a modernist framework of spirituality; as he described himself in The Path of Cinnabar, hence the positive discussion of psychoanalysis which he would later condemn after becoming a Traditionalist.


\hfill

\texttt{scardanelli on 2013-11-04 at 10:16 said: }

In light of the content of the post, there seems to be a tendency amongst traditionalists to create another pair of opposites, Tradition:Modernity. There was a comment previously about Jung not being “Traditional” or in this context, psychoanalysis not being “traditional,” as if there were a list of authors one is authorized to read when one becomes a “traditionalist.” 

But it seems less a case of something being traditional or modern and more a case of “everything in its place.” If one seeks to see the self and the world through the lens of psychoanalysis (or Freudianism, Marxism, Feminism, etc) this is obviously an error, but when seen as integrated in a hierarchical fashion, as a possible tool to analyze the “inclinations and impressions” of the soul, it is neither modern nor traditional, but either a more or less effective tool. This, I believe, is what Tomberg refers to as the spirit of free research. Everything is “traditional” in its proper place. It is the breakdown of this hierarchy that is not traditional.

It seems that this paralysis Evola speaks of is only necessarily so from the perspective of the ego, and this contrasts with the inner freedom of the Spirit or Self. When one realizes this interior freedom, he sees the exterior as arbitrary and dualistic. To act in the name of good is to evoke the bad. To act for the sake of the Kingdom of Heaven is to evoke The World. So this paralysis is effectively a binding of the ego and elimination of arbitrary and conditioned action, in favor of unconditioned action, whose source is the Spirit or self. And the “[t]hief will be obliged to flee with his artifices of iniquity.”


\hfill

\texttt{scardanelli on 2013-11-04 at 10:28 said: }

Further, it seems that Evola is saying that we must not only recognize these opposites, but use them for the sake of purification of the will. I don't think Evola is advocating that we be manly men, and do what is difficult for its own sake, but to recognize our attachments and use their opposite to bring about detachment. We choose the difficult to detach from ease and comfort, not to create some perverse masochistic attachment to pain and difficulty.


\hfill

\texttt{Avery on 2013-11-04 at 10:38 said: }

“Jung does not simply return a Christian to Christianity, a Jew to Judaism, a pagan to his paganism, as he claimed. More importantly, by rendering the suppressed underground emotions conscious and acceptable, Jung legitimates the demonic and destructive as having rights of their own on the strength of their therapeutic potential.”

Philip Rieff, The Triumph of the Therapeutic


\hfill

\texttt{JA on 2013-11-04 at 15:49 said: }

It's for me to be the voice of humour and add, in reference to dear scardenelli's mention of masochism, that the practice's nomenclature derives from Baron Leopold von Sacher-Masoch; an Austro-Hungarian aristocrat, and that experimental and different tastes in sexuality are quite the norm among those of us of a higher breeding……………………


\hfill

\texttt{Scardanelli on 2013-11-04 at 22:04 said: }

@Avery, forgive my mental laxity, but I'm not sure what your point is. Is this an argument against Jung as “non-traditional”?

@JA, I suppose I'm lucky then to be the low born descendant of dairy farmers and coal miners.


\hfill

\texttt{Jacob on 2013-11-05 at 19:52 said: }

@Scardanelli: 

I assume you mean my comment. I agree with what you wrote. The Scholastics always said truth is one. 

However, I'd like to understand where is Jung's place. Some (most notably Charles Upton) seem to imply Jung is subvervise and that following his teachings would lead to confusion between the psyche and spirit. I always assumed Jung viewed God as changing and at least partly evil himself. This may be a complete misunderstanding and extremely incorrect, which is why I commented that if Cologero (someone more knowledgeable than I) believes there is something redeemable there I would take a second look. 

Still I don't believe reading half truth would be beneficial if I can get access to the full truth. To kind of illustrate my point: Giles Deleuze is sometimes said to have based his metaphysics on Hermeticism and early Christianity. As I investigated though it seems that the word desire was being interpreted differently than Deleuze probably meant it. Instead of making an attempt to reconcile Deleuze to Christianity it seems far better to simply bypass Deleuze altogether since I have far better resources at my fingertips. Even though he has intelligent points to make, I'd like to understand where he is wrong first in order to fully understand the issue. 

Anyway, sorry for the long post and I hope this did not sound combatitive, because it was not my intention a lot of communication is lost over the Internet though.


\hfill

\texttt{Cologero on 2013-11-05 at 21:04 said: }

Jacob, you can safely ignore Jung, there is nothing necessary in his works. In the “Answer to Job”, Jung describes God as unconscious, requiring man to make him conscious. Nevertheless, when Jung restricts himself to empirical psychology rather than metaphysics, he has some interesting things to say. This may be useful in our times since the predominant method of argument these days, even among neo-traditionalists who should know better, is to attack the motives or impugn the psychological states of one's interlocutor. The truth or falsity of a statement is barely considered.

At the least, Jung had an encyclopedic knowledge of religions, religious experiences, history, symbols, etc. Hence, I find reading his works of interest just for that reason. By the way, Jung had high regard for Evola's book on Hermetism.


\hfill

\texttt{scardanelli on 2013-11-06 at 11:56 said: }

Jacob,

I wasn't necessarily arguing for or against Jung. I apologize for not clearly expressing it, but I was trying describe a way to approach truth as a process that unfolds within, rather than something we search for in books. In this sense, Jung is either more or less helpful in this process. I think that for many of us, books serve as a starting point. But for too many they are also an ending point. So we become dogmatic about who we read, who is traditional, who is modern, and in the end we change very little. Reading is instrumental, but really becomes a distraction at a certain point, when perhaps more concrete efforts are required.


\hfill

\texttt{scardanelli on 2013-11-13 at 20:54 said: }

One can see an interesting parallel with this series on the purification of the will, word, sex, etc in St John of the Cross. In The Ascent of Mt Carmel, regarding practical advice for the active night of the senses in which one purifies oneself of the passions, he gives this advice:

1) Seek to imitate Christ in all things

2) Renounce any sensory satisfaction in things that does not strictly serve to honor God

3) Harmonize the passions by always choosing:

{}-the most difficult over the easiest

{}-the most distasteful over the most delightful

{}-the less pleasant over the most gratifying

{}-hard work over rest

{}-the unconsoling over the consoling

{}-the least rather than the most

{}-the lowest and despised over the highest and most precious

{}-to want nothing rather than something

4) seek to carry out these practices and overcome the will's repugnance for them

5) desire to act, speak, and think with contempt for oneself, and desire others to do likewise.

So we clearly see an accordance with Evola's discussion of opposites, and the way in which one can use one to eliminate attachment and dependence on the other.


\hfill


\end{sffamily}\end{footnotesize}
