\section{Purification of the Mind}

\begin{quotex}
This is the second section from the article titled “Purity as a Metaphysical Value” by \textbf{Julius Evola}. It was published in Bilychnis, the journal of the Baptist Theological School of Rome. It is dated June, 1925, volume XXV.

Here he relies on Patanjali's yoga sutras to describe the stages in the purification of the mind. Interestingly, he describes a further stage beyond Patanjali's highest stage of Samadhi. Not surprisingly, Evola regards Patanjali's stages as negative and passive, so he describes an active and positive stage beyond that. This he relates to the “intellectual intuition” as understood in the West from Aristotle to the medieval scholastics, and then to German idealism. This stage is beyond discursive thought, and the lack of that type of knowledge accounts for most of the common misconceptions about Tradition. 

\end{quotex}
The impurity of the mind proceeds from the passive character of common perception that (not from the gnoseological point of view, but from that of empirical conditions) it is a feeling, an impulse from the outside to the inside pursuant to the coercion of the sensible object from the outside—so that the I cannot avoid perceiving or feeling what it perceives or feels depending on various times and places. Purification comprises two phases in this order: 

The first refers to the domain of the powers of the senses, to the capacity of detaching the mind from external objects, of withdrawing it onto oneself and fastening it at will. In other words, it is a matter of the maximum platonic catharsis: “To detach one's eyes and, in general, the soul, from sensible things”—taken however not in a metaphoric or moral sense, but literally: it is necessary to liberate the various perceptive faculties that the objects enchain, violate, and contaminate and to make them free of perceiving and not perceiving—and by that, in fact, unadulterated: pure. \textbf{Patanjali} indicates the stages that can realize that way through ordered discipline, of which, in any case, one can hope to progress only through exceptional inner energy:

\begin{enumerate}
\item \textit{Pratyahara} or control and mastery of the various impressions and accidental processes of associations and thought. 
\item \textit{Dharana} or concentration on a single object or sensation, excluding all the rest. 
\item \textit{Dhyana} or absorption in a not more sensible object, but produced by the mind itself. 
\item \textit{Samadhi} or elimination of the mental object itself and conjunction of the mind with its sole naked power. 
\end{enumerate}
But this negative phase does not constitute the final instance. One does not have true purity of the mind through the capacity of detachment, but rather through its power whereby the “other”, the correlation to the material object ceases to be a condition for perception, so that the “I” can, besides not perceiving, give himself, create from himself his own perception at will. It would be a matter, then, of substituting for the form of sensible and passive perception another that is active and positive, no more receiving but producing the object from one's interiority. Such is the virtue of the Aristotelian \textit{nous poietikos}, or of the “intuitive intellect” of the scholastics, Kant, and Schelling.

Only one must note than such a positive perceiving must not be \textit{another} faculty limited to the ideal order and juxtaposed to material perception, but rather its transformation and full resolution. Elsewhere [in \emph{Essays on magical idealism}], I considered the phases of such a path: here, as a simple suggestion, we can refer to so-called “supernormal knowledge”: modern psychology has ascertained that two distinct modes of perception are possible and real, having in equal measure the character of objectivity.

One is the normal connection to the physical organs, which—from the physiological, not the gnoseological, point of view—can be called centripetal, proceeding from the outside to the inside, starting from physical impressions transmitted from the afferent system to the central brain. The second way is, instead, independent of the physical organs and has an opposite direction: the point of departure is not the peripheral physical stimulus but rather an interior apperception, which then goes on to result in terms of physical perception and also in images, according to a centrifugal course analogous to that of hallucinatory processes.

Now while the first knowledge is limited by spatial-temporal and physiological conditions, the second is in a large measure freed from them and, what would result from the latest research, tends to participate in the nature of an omniscient principle. One could therefore connect to that the meaning of perfection, of the teleion, of the “purification of the mind”. From the philosophical point of view, it would then be a knowing that would originate no longer from a particular sense object to be subsumed to a discursive concept, but instead \textit{from everything}, in order to give the particular thing its place in this—just as a limb has a place in the organic unity of the body.



\flrightit{Posted on 2013-11-05 by Aeneas }

\begin{center}* * *\end{center}

\begin{footnotesize}\begin{sffamily}



\texttt{scardanelli on 2013-11-06 at 11:42 said: }

I have been thinking about this concept of widsom being “centrifugal” and “centripedal” when considering Tomberg's second letter. It seems that there is no reason why knowledge gained from the senses, from discursive thought should have to remain “limited by spatial-temporal and physiological conditions.” This thought can be a starting point for meditation, wherein one deepens the thought. So meditation can, in a manner of speaking, raise up what is below and bring down what is above by giving form to experience of the Spirit. Wisdom spiritualizes the material and materializes the spiritual. This is only possible once one has undergone the negative phase as described by Evola.


\hfill


\end{sffamily}\end{footnotesize}
