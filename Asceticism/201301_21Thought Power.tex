\section{Thought Power}

We have often heard from writers on Tradition that the mind needs to be dominated by something “higher”. This is not commonly understood by those who take the standpoint of ordinary life, since, for them, the flow of thoughts is itself the core of their self. Yet that is not how Tradition understands it. The reality is that there is something that transcends, observes, and controls the flow of thought. However, that is not “natural”, rather it is supernatural. It is a skill that must be learned and developed.

Careful observation will reveal several layers of thoughts. There are thoughts related to our desires and their fulfillment. Beyond that, there are thoughts that seem to relate to higher things, but are still indicative of the human state. For example, this may be exoteric religious beliefs, political positions, scientific theories, and so on. Then there are more subtle states beyond those, which begin to approach the “super-human”. The become visible only as the mind becomes less agitated. Rene Guenon refers to some of these states as angelic. That would be an interesting project for someone to take the angelic hierarchy of Dionysius and relate them to such metaphysical states of being.

Below, you will find several texts from \textbf{Swami Sivananda}`s book Thought Power. This is not an endorsement of the Swami, if only because I don't know enough about him. However, what he says is helpful and similar ideas can be found in \textbf{Lorenzo Scupoli}`s Spiritual Combat. I removed explicit religious references to avoid confusing things or bringing in devotional elements. This spiritual battle with the chaotic mind is not optional; victory is seldom quick so it will come only to the persistent. There is no time to waste and no excuses will be accepted.

\paragraph{Thinking and Courage}
If you have no courage to face the results of your thinking, to swallow the conclusions of your thinking, whatever they may mean to you personally, you should never take the trouble to philosophise. Take up devotion.

\paragraph{Destiny}
Man sows a thought and reaps an action. He sows an action and reaps a habit. He sows a habit and reaps a character. He sows a character and reaps a destiny. Man has made his own destiny by his own thinking and acting. He can change his destiny. He is the master of his own destiny. There is no doubt of this. By right thinking and strong exertion, he can become the master of his destiny.

\paragraph{Power}
There is no limit to the power of human thought. The more concentrated the human mind is, the more power is brought to bear on one point. Cultivate concentration, a serene mind is fit for concentration. Keep the mind serene.

\paragraph{Clear Thinking}
The mental images of the common man are generally very distorted. He does not know what deep thinking is, his thoughts run riot. There is a great deal of confusion in his mind.

It is only thinkers, philosophers and yogins who have well-defined, clear-cut, mental images. Those who practice concentration and meditation develop strong, well-formed mental images. Through right thinking, reasoning, introspection and meditation, you will have to clarify your ideas. Then confusion will vanish. Think clearly, clarify your ideas again and again. Introspect in solitude. Purify your thoughts to a considerable degree. Silence the thoughts.

\paragraph{Deep Thinking}
Most of us do not know what right thinking is. Thinking is shallow in the vast majority of persons. Deep thinking is given to few. Thinkers are very few in this world.

Hard thinking, persistent thinking, clear thinking, thinking to the roots of problems, to the very fundamentals of the situation, to the very presuppositions of all thoughts and being is the very essence of Vedantic Sadhana.

You will have to abandon an old idea, however strong and ingrained it may be, when you get a new elevating idea in its stead.

\paragraph{Likes and Dislikes}
The self-controlled man, moving among the objects with senses under restraint and free from attraction and repulsion, attains to peace. The mind and the senses are naturally endowed with the two currents of attraction and repulsion. Therefore, the mind and the senses like certain objects and dislike certain other objects. But the disciplined man moves among the sense-objects with a mind and senses free from attraction and repulsion, mastered by Self, and attains the peace of the Eternal.

\paragraph{Spiritual Combat}
Control thoughts, avoid imagination or day dreaming. The mind will be annihilated. Extinction of thoughts alone is Moksha or liberation.

The experience of the world illusion is due to your imagination. It vanishes away when imagination is completely stopped.

Victory over thoughts is really a victory over all limitations, weakness, ignorance, and death. The inner war with the mind is more terrible than the outer war with machine guns. Conquest of thoughts is more difficult than the conquest of the world by the force of arms. Conquer your thoughts and you would conquer the world.

\paragraph{Random Thinking}
All are victims of random thinking. All sorts of loose thoughts of diverse kinds come and go in the mental factory. There is neither rhythm nor reasoning. There is neither concord nor discipline. All is in a state of utter chaos and confusion. There is no clarification of ideas.

You cannot think of one subject even for two minutes in an orderly and systematic manner. You have no understanding of the laws of thoughts. There is a perfect menagerie inside. All sorts of sensual thoughts fight amongst themselves to enter the mind of a sensualist and gain the upper hand. Many cannot entertain a single, sublime, divine thought even for a second. Their minds are so framed that the mental energy runs in sensual grooves.

\paragraph{The Last Thought}
The last thought of a person will be the thought of God only, if that person has disciplined his mind all throughout his life and has tried to fix it on the Lord through constant practice. It cannot come by a practice in a day or two, in a week or a month. It is a life-long endeavor and struggle.



\flrightit{Posted on 2013-01-21 by Cologero }

\begin{center}* * *\end{center}

\begin{footnotesize}\begin{sffamily}



\texttt{scardanelli on 2013-01-21 at 22:16 said: }

An interesting and timely post. I just ordered Mouni Sadhu's works on concentration and meditation as an aid in this endeavor. Lately, when practicing, I have noticed the ability of one's stream of thoughts to take the form of “likes” or “interests,” those things which one would be the most tempted by, and these are used against us to break our concentration. Victory is seldom quick indeed…


\end{sffamily}\end{footnotesize}
