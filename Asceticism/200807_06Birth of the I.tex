\section{Birth of the “I”}

In \emph{The I Problem and Genius}\footnote{\url{https://www.gornahoor.net/?page_id=12944}}, Weininger writes about the realization of the sense of the “I”, that is, the experience of being an independent centre of awareness. He proceeds to give examples (from Jean Paul, Novalis, and Schelling) where they describe their earliest experiences “of the possession of an ego in the highest sense.”

Some men seem to have had a strong experience of that, often from an early age, while others don't even seem to understand the question. Readers may want to consider this an exercise in self-realization and try to remember their own such experiences. Here are two of mine:

\begin{quotex}
The first shocking memory I recall is when I had just finished a BM. I opened the bathroom door and called to my mother to take a look at it in the bowl. I was dumbfounded when she did not want to look at it and told me to go flush it myself. At that moment I woke up to myself and realized I was on my own.

\end{quotex}
That had to have been around two and a half years old. The following one was probably a year later:

\begin{quotex}
I tied all my toy trucks and cars together and made a train. As I pulled it around the house, my sister was crawling behind the train and following it. This is my first recollection of the sense of having a “will”.

\end{quotex}


\flrightit{Posted on 2008-07-06 by Cologero }
