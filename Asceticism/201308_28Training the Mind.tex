\section{Training the Mind}

\begin{quotex}
II,31: “If Power asks why, then is Power weakness.”

It is ridiculous to ask a dog why it barks. One must fulfil one's true Nature, one must do one's Will. \flright{\textsc{Aleister Crowley}, The Law is for All}

\end{quotex}
A dog does not need training to fulfill its true nature, so why then does a man? That is because man knows good and evil, so his mind is divided:

\begin{quotex}
the Will is single, the direct expression as “The Word” of the Self. The mind must inform the Understanding, which then presents a simple idea to the Will. This issues its orders accordingly for unquestioning execution. If the Will should appeal to the mind, it must confuse itself with incomplete and uncoordinated ideas. The clamour of these cries crowns Anarchy, and action becomes impossible. \flright{\textsc{Aleister Crowley}, The Law is for All}

\end{quotex}
So here, according to Crowley, we have this sequence:

\begin{enumerate}
\item The mind presents an idea to the Will 
\item The Will acts without question 
\item One's true nature is fulfilled 
\end{enumerate}
Hence, a man's mind must be correctly ordered in order to fulfill his own nature; or, in terms of Tradition, the Intellect is the ruler and action follows from contemplation. Anyone who observes his thoughts will immediately see how changeable they are, how he thinks one thing now but its opposite an hour later, how he vows one thing now and breaks it soon after. That is duality but his mind must be One. In Hermetism, the overcoming of duality to the One is called the return to the Primordial State. At age seven, I learned the answer to the fourth question in the Baltimore Catechism:

Q: What must we do to gain the happiness of heaven?

A: We must know, love, and serve [God].

The creed begins with “I believe in God”, but we see that faith alone is insufficient for salvation. Belief must be transformed into knowledge, or gnosis. That has been my quest since first communion, the preparation for the ultimate communion.

Whatever impedes the return to the Primordial State (and beyond) is called “sin”. In the Act of Contrition, there are three reasons to be contrite, as shown in Table~\ref{tab:ActofContrition}

\begin{table}[h]\small
\centering
\label{tab:ActofContrition}
\begin{tabular}{ccc}
\toprule
\textbf{Reason} &
\textbf{Aspect of God} &
\textbf{What it contradicts}\\\midrule
Pains of Hell &
God as Judge &
To Serve\\\midrule
Loss of Heaven &
God as Creator &
To Know\\\midrule
Deserving of all my love &
God as beyond being &
To Love\\\bottomrule
\end{tabular}
\caption{Act of Contrition}
\end{table}
It will become clear in this series what it means to know, to love, and to serve. But first, we return to the training of the mind. Now Evola reverses Crowley's original order. The latter began with “daring” and then moved to the training of the mind. Evola puts daring after the training, which is its natural position.

\paragraph{Space, Time, Causality}
Crowley proposes a rather rigorous scheme of education for training the mind, starting with Mathematics, then the study of the Classics, preferably in the original languages, and then to Science. These are related respectively to space, time, and causality, the preconditions without which conscious experience would not be possible. The fundamental goal is self-knowledge. Mathematics and Classics are comparable to the study of anatomy and physiology in medicine, since mathematics is the understanding of the anatomy of the mind and Classics that of the physiology of the mind.

\subparagraph{Mathematics}
The study of maths will reveal the laws of reason as well as its limitations. This reveals the anatomy of the conscious self; its abstraction, apart from personality or desire. This is why a book like Gnosis by \textbf{Boris Mouravieff} relies so much on numerical representation; this prevents the imagination from trying to envisage things and one is forced to focus on the mechanics of mental phenomena. Crowley relates in to the numerology of the Qabalah which reveals hidden connections based on numbers alone. Mathematics teaches the nature of necessity, i.e., logical necessity, and the knowledge of forms; hence, it is related to Space.

\subparagraph{Classics}
In studying the writings of antiquity, a man can discover the history of the structure of his mind. Such books may bring to light one's subconscious memories. Memory is the “mortar” of the mind without which the structure of the mind could not hold together. Thus, this study leads to the comprehension of one's own nature in the dimension of Time.

This is what I was trying to get at in the study of El/Ella, viz., the attempt to get at the roots of one's mind by going back in time to one's earliest memories.

\subparagraph{Science}
Science is the study of causality in the world and can lead to the comprehension of the Variety, Harmony, and Beauty of the Universe. The true method of the advancement in knowledge is the observation of the like and the unlike, which leads to the proper understanding of Magick. Proper science is non-dual, but Trinitarian, and does not stop at the like and the unlike which must be resolved. As Crowley writes, in man there is both God and dust, and by Magick these are united in one flesh.

Materialistic science cannot explain consciousness, knowledge, or true will. For it, freedom is impossible as man is always subject to the laws of material causality. The magical view, however, understands it differently. Matter is condensed energy, and this is not in dispute by physicists. But energy is condensed will and this can be verified in one's own consciousness. By willing, a man can bring certain energies to his activities. But “purity of heart is to will one thing”, otherwise the will works against itself. The first step to purity of heart is training the mind.

\paragraph{Daring}
In Hermetism, daring follows being silent, knowing, and willing. Crowley quotes the by now overworked line from Nietzsche: \emph{Live dangerously}. Due to the inertia of the mind, men fear Light and persecute those who bring it. In particular, one should “analyze most fully those ideas which Men avoid”. In times such as ours where the forces of consensus reality are so strong, simply to consider an unpopular thought is indeed a courageous action. However, let us keep in mind that some things which Crowley considered as courageous thoughts in his time are to a large extent commonly accepted opinions today.

Hence, in some areas we must leave him behind in order to challenge consensus reality. For example, Girolamo Savonarola \footnote{\url{https://www.gornahoor.net/?p=2744}} lived dangerously in his day and to follow his path of sanctity would likely be just as dangerous now. The attempt to recover the writings of antiquity to restore tradition seems to bring out the heresy hunters today. Actually, to follow the right hand path nowadays would be counter-cultural.

There is still a lingering romanticism with the so-called left hand path, as though sexual excess, drugs, and similar practices can lead to anything higher than the way of Chastity and Holiness as Crowley calls it. If all experience is samsara, and our preferences for one sort of experience over another is arbitrary, then we could just as easily choose the right experience. Actually, the overcoming of the elements as we described in the Qualifications for Initiation\footnote{\url{https://www.gornahoor.net/?p=1161}} are far more effective, and certainly more dangerous, than the practices recommended by Crowley. 

To return to the point at the beginning, man's nature is revealed to his Intellect before his Will. Like the barking dog, the natural man will effortlessly follow the right path once he grasps his own nature.



\flrightit{Posted on 2013-08-28 by Cologero }

\begin{center}* * *\end{center}

\begin{footnotesize}\begin{sffamily}



\texttt{JA on 2013-08-29 at 10:27 said: }

a problem that I have had, in my practice, is what do you do if your earliest memories, your inner sense of who you are going back to when you were a child – is incorrect ?

I've had a problem where basically I've had to create an artificial personality and through force upon my will subjugate my true will to the artificial will I created, in order for me to get closer to the ideal person I want to be.


\hfill

\texttt{francismercuri on 2013-08-29 at 14:33 said: }

“Purity of heart is to will nothing” vs. “to will one thing”. It really at a certain point becomes perspective dependent, and perhaps semantic. To will nothing whatsoever is a dissolution into an undifferentiated, akin to a “cosmic consciousness”; its “passivity” tends more toward “mysticism” than magick (as a science), or esoterism, broadly speaking. The Absolute though, as “total possibility” can be limited neither by “nothing” (which when considered literally, can not exist), nor even true Non-Being (possibilities which exist but are non-manifest). In Traditions such as Shaivism “Maya” is not “illusion”, but real forms that the Absolute “takes” without ceasing to remain “Absolute”; is this not similar to Western doctrine regarding the embodiment of the Logos “without confusion”? And while traditions such as Taoism characterize the Work as an “actionless-action”/wuwei, man is also defined as having arisen, as the result of the interaction of Heavenly/Yang Qi, and Earthly/Yin Qi–the Sage is taught to be “mediator”, as “child of Heaven and Earth”, attaining to various “Grades” of which retention of an operating individuality is necessary.


\hfill

\texttt{William on 2013-08-30 at 09:51 said: }

“Purity of Heart is to Will One Thing” is of course a reference to the book by Kirkegaard (well worth reading). Isn't aspiring to 'emptiness' to still Will One Thing? After all, to achieve emptiness (in any of its varieties) requires a lot of Will! As Crowley says: “One must fulfil one's true Nature, one must do one's Will.” Isn't the goal to purify our hearts of all detritus, so all that remains is One Will?


\end{sffamily}\end{footnotesize}
