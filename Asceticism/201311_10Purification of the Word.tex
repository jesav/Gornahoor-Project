\section{Purification of the Word}

\begin{quotex}
In this rather interesting section, \textbf{Julius Evola} writes on the purification of the word. This is the fruit of his study of the Tantras, which he learned from the writings of \textbf{John Woodroffe}. We need to reemphasize that his goal is not to convince everyone to become Hindus, but rather to use those teachings to elucidate metaphysical principles. Hence, Evola in this series always relates those principles to the corresponding Western ideas. There are three consequences:

\begin{itemize}
\item It shows that such principles have been known as well in the West 
\item It points the way to the recovery of Tradition by deepening our understanding of those principles 
\item Even those who choose to look in a different direction, e.g., the re-paganization of the West, must nevertheless incorporate this understanding into their own worldview 
\end{itemize}

All things are created through the Word, i.e., it is the principle of all created things. In the process of creation, the original creative impulse passes through a series of stages arranged hierarchically\footnote{\url{https://www.meditationsonthetarot.com/the-cosmic-hierarchy}}. 

The concept of the mantra, although it goes into deeper detail, is akin to the Western notion of hylomorphism. The “idea”, or “word”, are non-dual, but as they manifest, we experience the effects as a duality between the idea of the thing and its particular manifestation. That is why to know the real “name” of a thing is to know it in its essence. We came across this in the discussion of Occult Phenomena in the case of Adam, who had the power to name things. 

Note, too, how the power of the mantra is related to the devata and matrikas, which in the West are better described as “angels” or angelic intelligences. They are in the “causal” state, which is beyond form. Since the discrediting of the works of \textbf{Dionysius the Aeropagite} by academics, the role of the angels in creation has been neglected and they have become merely sentimentalized. Here, we see Evola relate them to an idea or word, a teaching we have alluded to many times previously. \textbf{Valentin Tomberg} also advocated for a better understanding of the role of angels.

Finally, I don't know that the phenomena mentioned in the final paragraph can be empirically verified. 

\end{quotex}


At this point, it is necessary to refer to the Indian doctrine of the “mantra” and, first of all, to indicate the metaphysical principles it presupposes.

The Word (\textit{shabda}) — according to the \textit{mantra shastra} — is the principle of the totality of created things. In the system of reality and beings, there is the manifestation of an originary power of expression, a manifestation that is structured in various hierarchical grades. In order to understand that, we note that a duality is implicit in the Word: on the one hand there is the word properly called (\textit{vak = vox}), on the other, the meaning or the object that the word itself expresses (\textit{artha}). Now in the first, supreme power of the Word, called \textit{shabda brahman}, word and meaning are one and the same thing, the expression is pure self-revelation, absolute transparency of the eternal meaning in itself. However such a unity becomes altered at the point of expression properly called. In fact, in the concept of manifestation there is implicitly a duality, or a proceeding, a going toward the other (\textit{bhavamukha}). Thus what as \textit{shabda brahman} is a second absolute, individual simplicity, is articulated and distinguished in further powers of the Word.

The Word, in being incarnated, germinates, through its own proceeding, what was a meaning resolved by it and is made objective in an ex-istence. In this dichotomic process, the “supreme sound” (\textit{para shabda}) assumes two aspects. The first is called the “subtle or causal state” (\textit{sukshma, karana}) of the sound and corresponds to “nature naturing”, to the Logos in its properly creative function (\textit{hiranyagarbha shabda}): object or meaning and word here are distinct; in the second place it no longer has a unique, concise meaning of the whole, but a unity that is dispersed in a multiplicity. Nevertheless here the distinction and multiplicity are still included in the unity of a productive function; although distinct, object and word are still not external to each other. In this second power of “sound”, there is therefore an ensemble of cosmogonic functions, corresponding to the \textit{logoi spermatikoi} of Greek speculation and to the “letters of light” of the Kabbalah and which are precisely called “letters to the causal state” or \textit{matrikas} (goddess mothers) and connect symbolically to those of the Sanskrit alphabet. All things in the world would proceed from the “combination” of these letters, but not such as they appear in their effects to sensible perception, but rather what are in their causes: such are the “Names” of things. Now, in these functions of the Word, the aspect of the meaning corresponds to “\textit{devata}” (Vedic deities), and the aspect of the word or expressive corresponds to the “mantra”. The mantras would therefore be the Names of the \textit{devata}, or the various “bodies of power” that rule the productive process of things; and, vice versa, the devata would be the transcendental meanings that the mantras incorporate and bring to light.

Beyond this subtle state of the Word, there is a third material state (\textit{sthula}), corresponding to the audible spoken voice (\textit{vaikhari shabda}). That is, one stands at the level of worn-out manifestation, where the scission between \textit{artha} and \textit{shabda} is complete: on the one hand, there is the spoken language, on the other, material objects, whose relation to it is exterior: the “Name” or the world no longer has an objectively expressive or creative value, but only a conventional, allusive value of materiality, not of the inner meaning of the object. Besides, while the “natural name” or mantra of things is universal, the name which appears at this level is particular and contingent, it depends on time, place, individualization, race, etc. But beyond the various languages of men there would be, or would have been, (according to some initiatic traditions, in the period prior to the “confusion of tongues” alluded to in the Biblical story of the Tower of Babel) a type of universal language, in which each thing and each being would have its natural, original or essential Name (\textit{bija mantra}).

The word, such as it is known by the finite being, is therefore “impure”: impure first of all because it does not have in itself, in its own power, but outside of itself, the object that it expresses, because it no longer provides the real nature of the object in a productive function, but only the simple subjective image; in the second place, through the contingency and particularity of this same image, which depends on place, time, individualization, etc. Now the requirement of the practices that reconnect themselves to the mantras is exactly toward a “purification of the word”, which is: to bring the I from that language that is the evocative faculty of simply subjective images, to that other language that is the power of evoking the things themselves, that is, to the language that produces things in their causes, being identical to the its supernatural productive process.

The mantras are the “natural approximate names” of things; by means of a living compenetration in them, the yogis seek therefore to return to, or better said, to identify themselves with, the various causative powers or \textit{devata}. That is the awakening of mantras: to awaken a mantra means to evoke, regenerate, render into act, the subtle function of the Word connected to it. It is a matter of a true putting in relationship, of a real identification. The I of the order in which the word is simple evocative discourse of vague images passes on to that in which it is a creative spiritual power, however, from the plane in which the perceiving is a passion to the plane in which it is a putting (whence the connection to the “purification of the mind”). The mantra is therefore nothing if it is not “reawakened”: one can repeat it a million times—it is said in the texts—but as long as it is not known, it remains a mere flapping of the lips. The mantra must be actualized, “made to bloom” (\textit{sphota}) in its essence “made of light” (\textit{jyotirmayi}): only then does it “work”. Its pronouncement is therefore an inner act, whose material expression acts only as the vehicle. The \textit{sphota} can happen only by means of the force of will: but more often the vital force (\textit{prana}) or the force of generation (\textit{kundalini}) is undertaken as auxiliary.

Now when one realizes a state of identity with the individuating principles of things with the mantra, it is evident that, by vibrating the same will in a “reawakened” mantra, the related act has magical value (that which corresponds to it is directly realized) since what one wills is as if the thing itself wills it. So the virtue attributed by the texts to the mantra—to the “pure word”—is astonishing. Having reawakened the mantras of various elements, the yogis acquired power over them, he can, for example, make fire flare up where he wants or to go into the middle of it without being affected by it; through mantras, he can produce the well-known phenomenon of the growth of a seed into a plant in a few minutes; he can place around himself a circle that nothing can pass through: a lance or a projectile hurled against him rebounds instead against the one who threw it; he can conceal himself from the view of others, in order to provoke visions, thoughts, and feelings in them; he can kill or cure at will. In the \textit{Vishnu Purana}, the power of procreation by means of the mantra is even considered.



\flrightit{Posted on 2013-11-10 by Aeneas }

\begin{center}* * *\end{center}

\begin{footnotesize}\begin{sffamily}




\end{sffamily}\end{footnotesize}
