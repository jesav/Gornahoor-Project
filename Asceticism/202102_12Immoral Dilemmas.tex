\section{Immoral Dilemmas}

These are the results of my first experiments in remembering. Behind the temporal succession of the events of life, there is an eternal principle that gives them meaning and an intemporal I that is unchanged throughout.

The common view is that memory has something to do with the past, as when friends or family get together to relive their common nostalgic moments. Phenomenologically memory is bringing the past into the present. Or to put it another way, the seemingly disparate and unrelated events of life cannot be understood as merely temporal succession. As Rene Guenon expressed the goal in \emph{Oriental Metaphysics}:

\begin{quotex}
The person who attains this “primordial state” is still only a human individual and is without effective possession of any supra-individual states; he is nevertheless freed from time and the apparent succession of things is transformed for him into simultaneity. He consciously possesses a faculty which is unknown to the ordinary man and which one might call the “sense of eternity.” This is of extreme importance, for whoever is unable to leave the viewpoint of temporal succession and see everything in simultaneity is incapable of the least conception of the metaphysical order.

\end{quotex}
The exercise in remembering should lead to seeing one's life in its simultaneity. The events in their totality suddenly reveal a hidden meaning to life. This meaning is constant, eternal, and above time. What follows are some notes from this exercise. Keep in mind that there are four levels of memory of increasing depth:

\begin{enumerate}
\item \textbf{Intellectual memory}. The unadorned memory of a past event. 
\item \textbf{Emotional memory}. A memory that elicits an emotional reaction. 
\item \textbf{Volitional memory}. A memory that incites to action. 
\item \textbf{Moral memory}. A moral aspect is added to the memory. 
\end{enumerate}
Forgetting is analogous to death so remembering is life giving. Some memories arise spontaneously. Others take some effort. One should not stop at intellectual memory. It may take more effort to expand that memory to deeper levels. This essay has been some time in planning. The events mean nothing to you, but each time I've relived them, there is more intensity. Humor may be used to hide it. Nevertheless, over time, a common thread has been revealing itself to me.

You can avoid remembering now, but your entire life will be exposed at death. There is no point in evading the inevitable.



\flrightit{Posted on 2021-02-12 by Cologero }

\begin{center}* * *\end{center}

\begin{footnotesize}\begin{sffamily}



\texttt{Michael M on 2021-02-12 at 10:09 said: }

When it comes to viewing memories and the addition of the moral element and a common thread throughout life, the more we untangle those knots looking back and going further and further long those lines does give a different type of perspective or Gestalt of the whole. 

Utilizing that theme, would it be more advisable then to look at present / future events in the light of such a theme? Or is that too close to trying to get a specific result from a type of effort? Thereby destroying the whole effort to begin with. The building of a foundational “center” with a common theme that has always been present in life would reasonably seem to be the only true way to have a viewpoint or post for life that would never be carried away by external influences.


\hfill

\texttt{Tannheuser on 2021-02-12 at 13:24 said: }

““Say it, say it, say it.” Now I knew perfectly well what she wanted me to say, but I just couldn't say it. I was her one and only love but she was just one of many to me.

Tant pis pour elle. How much pain has unrestrained desire caused in the world? And why does it have to feel so good?”

—–

The girl who I couldn't “say it” to was actually the first woman I had ever slept with, and the most beautiful. Our relationship had started because I wanted to make the woman who I really loved at that time jealous, which worked. Still, for a while she sucked me into a world of immense pleasure that was hard to step away from. She had a very pleasing personality and the sex was incredible, but the whole thing was spiritually suffocating – I felt like Odysseus on Circe's island. 

In the end I only ever said “I love you” to three women, who corresponded in order to the types of ame-soeur, mistress, and wife.

—–

Michael M: If your life is a unity, then it only makes sense that you take action in the present from that perspective as you become conscious of your own meaning and purpose.


\end{sffamily}\end{footnotesize}
