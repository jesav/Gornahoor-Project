\section{Prerequisites for the Training of the Mind}

\begin{quotex}
It is the hate of, the distaste for, life that sends one to the ball when one is old; when one is young one is on springs until the hour falls; but the love of God, which is the only true love, diminishes not with age; it grows deeper and intenser with every satisfaction. It seems as if in the noblest men this secretion constantly increases—which certainly suggests an external reservoir—so that age loses all its bitterness. \flright{\textsc{Aleister Crowley}, Liber DCCCXI, Energized Enthusiasm}

\end{quotex}
Young men, at least those of a certain character, easily get intoxicated with the Thelemic philosophy of Aleister Crowley. We see that in \textbf{Julius Evola}'s appreciation of him in his youth in the Ur group. A “friend” of mine was drawn at one time to Thelema when the tales of Carlos Castenada and the novelty of Timothy Leary were at the peak of their notoriety. When it seemed that the whole outer world was known, explored, and catalogued, the idea that the true secrets of the universe lay in the inner exploration of consciousness seemed worthy of a life. Crowley described this impulse:

\begin{quotex}
Now I am certainly of opinion, that genius can be acquired, or, in the alternative, that it is an almost universal possession. Its rarity may be attributed to the crushing influence of a corrupted society. It is rare to meet a youth without high ideals, generous thoughts, a sense of holiness, of his own importance, which, being interpreted, is, of his own identity with God. Three years in the world, and he is a bank clerk or even a government official. Only those who intuitively understand from early boyhood that they must stand out, and who have the incredible courage and endurance to do so in face of all that tyranny, callousness, and the scorn of inferiors can do; only these arrive at manhood uncontaminated. 

\end{quotex}
My friend spent his three years in the world, forgetting for a time that inner impulse, until he re-encountered Hermetism in the \emph{Meditations on the Tarot}. He then went back to the Thelemic literature and found something new in them.

In the spirit of free inquiry and respect for Tradition that Tomberg calls for, I will attempt some commentary on the sections of \emph{Liber Aleph} that Evola considered most important. This does not imply agreement with all, or even anything, that Crowley stood for, but the willingness to explore all worldviews just as a biologist explores forms of life.

\paragraph{Miguel de Molinos}
Crowley curiously considered the Spanish mystic \textbf{Miguel de Molinos} as a fellow traveler. Molinos founded a spiritual movement now known as quietism which extended to France with \textbf{François Fenelon} and \textbf{Madame Guyon}. He was quite popular and his works were printed with the Imprimatur. Nevertheless, he was eventually branded a heretic, persecuted, and his movement squashed. Unfortunately, that was a moment and opportunity lost, and the loss is the continued ascendancy of ratiocination, or discursive thought, over the pure immediacy of spiritual intuition. He described his method:

\begin{quotex}
There are two ways of going to God, the one by Consideration and Mental Discourse, and the other by the Purity of Faith, an indistinct, general and confused knowledge. The first is called Meditation, the second Internal Recollection, or acquired Contemplation. The first is of Beginners, the second of Proficients. The first is sensible and material, the second more naked, pure and internal. \flright{\textsc{Miguel de Molinos}, \textit{The Spiritual Guide}\footnote{\url{https://www.gornahoor.net/library/MolinosSpiritualGuide.pdf}}}

\end{quotex}
That is, pure contemplation is superior to the use of sensible images and thought. Now perhaps there were some excesses in practice or ambiguities in theology, but the fundamental point still stands. Suppose I went to the jungle and discovered a new species of beetle. Perhaps I didn't preserve it properly or assigned it to the wrong genus. That is a matter for the biologists to debate, but the fundamental fact of the discovery remains. Likewise, the fact of Molinos' spiritual understanding remains after all the condemnations.

The lost opportunity, the question that seems to agitate some readers, is ironically described in the Catholic Encylopedia\footnote{\url{https://www.newadvent.org/cathen/12608c.htm}}:

\begin{quotex}
In its essential features Quietism is a characteristic of the religions of India. Both Pantheistic Brahmanism and Buddhism aim at a sort of self-annihilation, a state of indifference in which the soul enjoys an imperturbable tranquility. And the means of bringing this about is the recognition of one's identity with Brahma, the all-god, or, for the Buddhist, the quenching of desire and the consequent attainment of Nirvana, incompletely in the present life, but completely after death. Among the Greeks the Quietistic tendency is represented by the Stoics. Along with Pantheism, which characterizes their theory of the world, they present in their apatheia an ideal which recalls the indifference aimed at by the Oriental mystics. The wise man is he who has become independent and free from all desire. According to some of the Stoics, the sage may indulge in the lowest kind of sensuality, so far as the body is concerned, without incurring the least defilement of his soul. The Neoplatonists held that the One gives rise to the Nous or Intellect, this to the world-soul, and this again to individual souls. These, in consequence of their union with matter, have forgotten their Divine origin. Hence the fundamental principle of morality is the return of the soul to its source. The supreme destiny of man and his highest happiness consists in rising to the contemplation of the One, not by thought but by ecstasy (\emph{ekstasis}). 

\end{quotex}
In other words, this would have brought Western theology more explicitly in line with Tradition. Debate is pointless; either such states are achievable or not. We should point out that quietism has some commonality with Hesychasm.



\flrightit{Posted on 2013-08-27 by Cologero }

\begin{center}* * *\end{center}

\begin{footnotesize}\begin{sffamily}



\texttt{Logres on 2013-08-28 at 08:24 said: }

Caussade's Sacrament of the Present Moment barely escaped condemnation as “Quietist”. In fact, probably the letters being kept unpublished was what saved them. Perhaps we Christians need to take a closer look at his work. It's not coincidence that he appeared in France.


\hfill

\texttt{scardanelli on 2013-08-28 at 10:21 said: }

Speaking of the French tradition, I'm currently reading Tomberg's Lazarus, Come Forth!, and de Molinos' two paths sound strikingly similar to Tomberg's “day path” and “night path.” The goal of the former, he writes, being “elevation of the spirit to behold the world in divine light so that it would appear as a revelation from God,” while the goal of the latter is “descent into the depths of the human being, where it encounters the being of God directly…” In other words, one leads to an exterior symbolic knowledge of God, and the other a direct interior knowledge of God.


\hfill

\texttt{Logres on 2013-08-29 at 10:59 said: }

I see that Molinos barely escaped with his life, once the Inquisition got their dander up. And Appiani (a Jesuit) was arrested, then disappeared.


\end{sffamily}\end{footnotesize}
