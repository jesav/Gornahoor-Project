\section{Esoteric Initiatic Centers}

Every so often on social media, someone naively pleas for information on finding an ``esoteric center", as if they are like McDonald's franchises just waiting for customers. Another frequent complaint is that such and such ``tradition" does not have an initiatic, esoteric center. I don't know how they know that or even what it means.

Unfortunately, none of these fellows would recognize an esoteric center even if it had a storefront on Main Street with bright neon signs. So they read everything Guenon wrote, in no particular order, hoping for a revelation, but that don't impress me none. On the other hand, if they learned Arabic, moved to Cairo, and joined Guenon's initiatic organization, I would show some respect. But Americans don't expect to go to that much trouble.

Initiatic esoteric centers are not eagerly waiting for you to show up. And if you do, they will probably make you clean the latrines for three years before teaching you anything.

\paragraph{Cults and Centers}
One can certainly find high priced cults that promise instant enlightenment. You will make many like-minded friends. They will tell you that you are full of love. They will tell you to give up attachments but not how. In the end you might notice other people's attachments, but not your own. But remember this:

A cult is easy to find, but hard to leave. An esoteric center is hard to find but easy to leave.

\paragraph{Augustine and the Fig Tree}
Augustine had explored Manichaeism and then philosophy, yet remained dissatisfied. Even the influential Neoplatonist pagan Victorinus had converted. Distraught, Augustine sat under a fig tree in a Milan garden and pleaded with God for relief. Hearing a child's voice saying, ``Take and read," Augustine picked up his Bible and read:

\begin{quotex}
Not in reveling and drunkenness, not in lust and wantonness, not in quarrels and rivalries. Rather, arm yourselves with the Lord Jesus Christ; spend no more thought on nature and nature's appetites. \flright{\textsc{Romans 13:13-14}}

\end{quotex}
Lady Continence appears to him, and he finds himself willing to live the celibate life. The Lady is the final appearance of the Anima to him. Previous appearances include his unnamed concubine, Eve, Dido, and especially his mother Monica.

\paragraph{The Buddha and the Fig Tree}
Everyone knows the story of the young prince who was shocked when he realized the true nature of life. He went on a lifelong search. Unable to find enlightenment in any of the ``initiatic esoteric centers" in India, he sat under a fig tree, vowing not to leave until he achieved enlightenment.

Like Saint Augustine, the answer suddenly came to him and he became the Buddha. Neither one needed an initiatic esoteric center. Chew on that for a while. Perhaps your own approach has been totally misguided this whole time.

\paragraph{Lama Yeshe}
There are several aspects that made Tibetan Buddhism so appealing to me at one time, namely

\begin{enumerate}
\item It is patriarchal 
\item It is hierarchal 
\item It requires a life of prayer and meditation 
\item It has a strict moral code 
\item It has a comprehensive cosmology 
\item It has an understanding of postmortem states 
\item It has a complex metaphysical system 
\item There is a path to salvation and liberation 
\end{enumerate}
Lama Yeshe was the head of the lineage into which I was initiated. I recently found out that he was reincarnated in the body of a young Spanish man, Ösel Hita Torres. Here is a video of the fellow\footnote{\textit{Taste of Buddha: One Big Love}, \url{https://www.youtube.com/watch?v=FLLaW-kKu1g}}. He seems more interested in Bob Marley than in the Buddha. I've visited Marley's house in Kingston, where the bullet holes in the walls have never been patched up. That is the price of one love. With his swaying motions and lack of a command presence, Ösel doesn't appear enlightened to me; however, it is not my place to judge, so I won't question your opinion.

I did not listen long enough to know if any of the points mentioned above came up as a question. But they seldom do. However, I usually asked about them, once in particular about the purpose of meditating on the Buddhist cosmology painted on a Thangka. The answer was that one should visualize oneself as enlightened in the presence of the devas, gods, and buddhas.

\emph{That was my own ah-ha fig tree experience}. I didn't really need the Thangka anymore. I could find all eight points in my own Tradition, visualizing myself in the cosmology of the Divine Comedy, or alchemical diagrams, so elegantly described by C. S. Lewis in \textit{The Discarded Image}. But why discard it?

All 8 points were available to me in texts that are natural to me. There are usually bad psychological reasons to choose an alien tradition over one's own. In any case, I still practiced the 8 points, although I suspect few American Buddhists do. Otherwise, they could not fail to notice how close they are to the lifestyle of the tradition they thought they left behind.

\paragraph{Isolation}
Before you commit to an ``esoteric center", consider the consequences. Your life will never be the same again. Whatever you used to enjoy will become empty experiences. Your friends won't understand you. And it is not very pleasant, as the young Prince discovered, to see life without any illusions.

And if you quit halfway, you will be worse off than if you had never begun at all.

\paragraph{Conclusion}
If you are serious about Tradition, then start acting that way. Otherwise, the next book won't make any difference. Here are some suggestions, which you can ignore or accept.

\begin{itemize}
\item Don't assume you know what you don't know 
\item The best choice is usually to follow the path of your own nation 
\item Pray and meditate, rather than read 
\item Treat your superiors with deference and respect 
\end{itemize}

\hfill

\paragraph{Appendix}
The topic of how to read Rene Guenon often comes up. Here is a suggestion.

\textbf{Foundation}: These works are simple to understand and provide a general overview. The basic issues are explained as well as the reason that a return to Tradition is both necessary and desirable.

\begin{itemize}
\item Crisis of the Modern World 
\item East West 
\item Oriental Metaphysics 
\end{itemize}
\textbf{Metaphysical Trilogy}: An understanding of metaphysics is the necessary preliminary step. Moreover, once comprehended, it brings intellectual certitude, as apodictic as any mathematical proof. However, that is only the first step. Proper training in spiritual exercises is required in order to fully understand the texts. Ultimately, book learning alone is insufficient to achieve metaphysical realization.

\begin{itemize}
\item Man and his Becoming 
\item The Symbolism of the Cross 
\item Multiple States of the Being 
\end{itemize}
The other books go deeper into specific topics and show the practical application to real world problems.



\flrightit{Posted on 2020-05-31 by Cologero }

\begin{center}* * *\end{center}

\begin{footnotesize}\begin{sffamily}



\texttt{William Zeitler on 2020-06-01 at 17:53 said: }

Another superb post! May I, however, presume to nuance your suggestion: ``Pray and meditate, rather than read" and suggest: READ SCRIPTURE! Let at least 50\% of your discretionary reading (that is, not job related) be Scripture. And pray and meditate on Scripture too! Reading ABOUT Scripture or spirituality (including this blog) doesn't count towards that 50\%! Just my \$0.02.


\hfill

\texttt{A.M. on 2020-06-02 at 10:21 said: }

A great post and a great comment from William. Although, it's easier for me to praise such well-articulated orientations than to dig into Scripture which overwhelms me with meaning and power whenever I start to read through it.


\hfill

\texttt{Sylvan Savant on 2020-06-03 at 17:26 said: }

I'm not sure if this is the right place or time to bring this topic up, but I wanted to ask you for your advice. Since this post does seem to deal with a number of subjects one of them concerning the trajectory one should take in life, I think it would be pertinent to ask which ones would be best to avoid. 

Like many young people of my generation, I feel not only horrified but also incredibly outraged at the state which the Western world is rapidly degenerating into. I do care about what happens to my country and my society, yet people like me are treated as absolute outcasts, enemies of the people, if not downright imprisoned or even called to be lynched by the purveyors of public opinion. How can one react towards the madness of accusations and lies of the most pernicious sort that surrounds us without falling victim to indignation and anger directed at those who are responsible for bringing about this state of affairs?

It's one matter to talk about Kali Yuga in the abstract as if it were an imminently approaching but not yet present event, but to see it unfold under your very nose sometimes feels like it's too much to bear. 

Do you advocate ``quietism, resigned withdrawal" or ``active, heroic life in the world"? Is there a dichotomy between the two? And even if I do follow the oft quoted advice to just limit my activity to my close circle of friends and the local neighborhood, then how can one possibly deal with the friction which arises when these circles inevitably falter under the influence of the spirit of the age, one way or another?


\hfill

\texttt{William Zeitler on 2020-06-03 at 19:31 said: }

Sylvan Savant: IMHO, I doubt that there's one answer for everyone. I personally think about ``what good can I do where I'm planted with the resources (inner and external) that I've been given?" Meditate and pray (and read Scripture) and continually do your best to follow the Spirit's prompting. Both the `heroic' and `withdrawn' Paths have their place — the question is which is YOURs. Expect your Path become clear over time — God will almost certainly need to prepare you for His purposes (although for some He makes it clear in a `blinding flash'. That may or may not be your Path). Perseverance, faithfulness and attentiveness to the Spirit's promptings as best you can for the rest of your life is the Way! I might point out that the Greek word for Way in ``I am the Way and the Truth and the Life" is hODOS, which can also be translated `Journey’ — ``I am the Journey, the Truth and the Life." He commanded us to ``Be following Me" (the verb tense makes it clear that continual following is in mind, not just a one time effort.) You know at least some of what you need to do (e.g. pray), start by being faithful about what you know for sure that you need to do. Allow God to reveal the rest in His good time as you are ready.


\hfill

\texttt{Thomas Walker on 2020-07-13 at 10:45 said: }

Just a comment on JRs post : under the influence of reading Guenon , I joined a tariqa about 40 years ago but left it after some time when I heard one of the moqaddems saying that all Christians were in hell or going there.


\hfill

\texttt{Cologero on 2020-07-21 at 07:48 said: }

JR: There is no specifically ``Guenonian doctrine"; either he is restating Traditional doctrine correctly or he is not. The ``decay of civilisation" is inevitable … you added ``irreverible". Institutions become corrupted, duh, that is obvious. As for the current leaders, Jesus predicted there would be ``wolves in sheep's clothing", so no adult should be surprised. But Catholicism, as a tradition, cannot be judged by the limitations of its ``leaders". That should be obvious to any thinking man, but it is not always the case. Many gloat over the malfeasances of the institution because of their loathing for the Traditional form of Western civ, showing a lack of intellectual integrity.

Deviation and subversion are the tactics used to destroy Tradition. You don't need to be a very astute observer to notice them in action. Just observe who and what is under attack … it's amazing how people prefer to blind themselves rather than to see. Probably because of their secret hatred for our Tradition despite their public persona. You don't want to be one of them … or do you?


\end{sffamily}\end{footnotesize}
