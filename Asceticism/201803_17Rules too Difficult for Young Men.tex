\section{Rules too Difficult for Young Men}

The Internet has been able to create a new type of Star. The iSoapbox has given non-standard viewpoints a small space in which to be heard. There are the pretty right-wing girls who do well. However, men also gain larger audiences.

Recently a Star was born in the persona of a Canadian psychologist, and his influence has extended beyond the confines of the Internet. Part of his appeal, it has been said, is that he provides a positive message to young men. That may be true, even if the necessity for it is sad.

Once in a while, someone suggests to Cologero that he should be on TV. I'm sure they intend that as a sort of praise, although they have really not thought things through. First of all, the message here is a bit too difficult to summarize. The professor's worldview, on the other hand, can be summarized in a paragraph or perhaps in 12 rules for life. Then it is a matter of repetition.

Also — and I can say this — Cologero does not have a very dynamic personality. Not experiencing emotions like most of the world, he cannot imitate that pained, thoughtful look, nor does he weep on cue. Furthermore, he is quite low on “amiability”. He eschews current events and does not believe in “debates”.

Nevertheless, Gornahoor has touched on many of the same themes to a much smaller audience. These rules for life have been available online for several years with minimal impact. While Cologero is preoccupied, I've taken the liberty to propose Gornahoor's rules for life for young men. There must be some who can handle a deeper understanding of Life.

\paragraph{Stand up straight}
Those who have participated in the online seminars would have heard of the importance of posture since the very first meeting. Cologero has recommended the Alexander technique for those who still have trouble. Moreover, posture is just one aspect of a larger theme which includes voice, intonation, facial expression, and so on. Mastery of the body is part of self-development.

In a larger perspective, this includes physiognomy which includes more than just a face. Curiously, once out of favor, scientific interest in physiognomy has been growing.

\paragraph{Hierarchy}
\begin{quotex}
The authors of the Middle Ages could not imagine … the Universal Church without a pope. Because if the world is governed hierarchically, Christianity or the \emph{Sanctum Imperium} cannot be otherwise. Hierarchy is a pyramid which exists only when it is complete. \flright{\textsc{Valentin Tomberg}, \emph{Meditations on the Tarot}}

\end{quotex}
It is one thing to joke about the hierarchy of lobsters, but quite another to follow the hierarchy created by God. There is no need to repeat the details here, since the notion can be found throughout Gornahoor. There can be no compromise. It can be a difficult teaching because few want to accept voluntarily their rightful place in any hierarchy.

\begin{quotex}
A pyramid is not complete without its summit; hierarchy does not exist when it is incomplete. Without an Emperor, there will be, sooner or later, no more kings. When there are no kings, there will be, sooner or later, no more nobility. When there is no more nobility, there will be. Sooner or later, no more bourgeoisie or peasants. This is how one arrives at the dictatorship of the proletariat, the class hostile to the hierarchical principle, which latter, however, is the reflection of divine order. This is why the proletariat professes atheism. \flright{\textsc{Valentin Tomberg}, \emph{Meditations on the Tarot}}

\end{quotex}
\paragraph{Purification}
This is the metaphysical equivalent to “clean your room”. Granting that as a metaphor for “get you own life in order”, that is not possible without first getting your inner life in order. Here are some links to get you started.

\begin{itemize}
\item Purification of the Will\footnote{\url{https://www.gornahoor.net/?p=6950}}
\item Purity of the Mind\footnote{\url{https://www.gornahoor.net/?p=6961}}
\item Purification of the Word\footnote{\url{https://www.gornahoor.net/?p=6969}}
\item Purification of the Sex Act\footnote{\url{https://www.gornahoor.net/?p=6985}}
\item Purification of the Breath\footnote{\url{https://www.gornahoor.net/?p=6991}}
\item Moral purification of the Will\footnote{\url{https://www.meditationsonthetarot.com/moral-purification-of-the-will}}
\end{itemize}
\subsection{Self-Mastery}
Self-mastery or autarchy is:

\begin{quotex}
For those who have absolutely possessed themselves, the alogon indicates only the unconditionality of his will, autarchy. Therefore, beyond the abstract intellectual, there is still an abyss between the man who lives his own life as lord and master, and the man who feels himself only as a demonic force of a nature that is passive to himself and that has its raison d'être outside himself. 

\end{quotex}
First, the autarch must master his thoughts, then his emotions, and ultimately his body insofar as it is possible. This is actually the Hermetic path, since each of those stages of self-mastery leads to the development of higher states of being.

\paragraph{Learn a martial art}
If so inclined, learn a martial art or become skilled in weaponry. A martial art develops concentration and attentiveness. The competition simulates life and death situations. Cologero wrote:

\begin{quotex}
A man, for his preparation, can conquer his fear of the elements: fire, air, water, and earth. Overcoming such fears leads to self-knowledge and a concentrated mind. Those with a warrior cast of mind need to excel at a martial art or develop skill with weaponry. Time spent in this way is more beneficial than excessive, or obsessive, reading. If you don't regard these activities as spiritual, then make them so. Do them with the right attitude and full consciousness. 

\end{quotex}
\paragraph{Spiritual Combat}
Spiritual combat is the analog of martial arts on a higher plane. However, at this level one's eternal life is at stake.

\begin{quotex}
Even without being killed a man can experience death, he can conquer, he can realize the culmination characteristic of a “super-life”. From a higher point of view, Paradise, the Kingdom of Heaven, Valhalla, the Island of the Heroes, etc., are only symbolic figurations forged for the masses, figurations that in reality designate transcendent states of consciousness, beyond life and death. The ancient Aryan tradition used the term \emph{jivan-mukti} to indicate such a realization while still in the mortal body. \flright{\textsc{Julius Evola}}

\end{quotex}
\paragraph{The Modern World}
It is necessary to have an adversary relationship to the modern world, even if only private. That means “to be in this world but not of it”. In anticipation of ridiculous comments, “modern” does not simply mean “contemporary”. We are neither Luddites nor Amish. Here are two of three steps (the third is in preparation).

\begin{itemize}
\item Understand the crisis of the Modern World 
\item Revolt against the Modern World 
\end{itemize}
\paragraph{Honor your father}
Honoring your father means gratitude for your inheritance and respect for those ancestors who created it. For Western Europeans, these are the three main currents of that inheritance.

\begin{itemize}
\item Hellenic Culture 
\item Roman Tradition 
\item Neoplatonic philosophy 
\end{itemize}
\textbf{Hellenic Culture}: The Hellenes have given us art, philosophy, political theory, and folk religion. Therefore, either create, or learn to appreciate, poetry, sculpture, drama (both tragedies and comedies). Develop a foundation in the trivium of grammar, logic, rhetoric: learn to speak well, logically, and persuasively. Learn the various types of political systems and the causes of their degeneration.

Pure thought is insufficient. A good life will include the Greek virtues: temperance, prudence, courage, and justice.

The Greek folk religion included reverence of ancestors, respect for the founders, and engaged the whole being in every aspect of life. Hegel describes it this way:

\begin{quotex}
In a folk religion in particular it is of the utmost importance that the imagination and the heart not be left unsatisfied: the imagination must be filled with large and pure images, and the heart roused to feelings of benevolence. 

\end{quotex}
\textbf{Roman Tradition}: The Romans brought us martial valor, administration, and engineering.

The primary Roman virtues included Piety, Gravitas, Dignitas, and either manliness or modesty.

\textbf{Neoplatonic Philosophy}: Neoplatonic philosophy integrated the best parts of Plato, Aristotle, and Stoicism. It is a powerful counterweight to contemporary philosophies of naturalism or materialism. A thinking man should always be an idealist, since the understanding of ideas is the aim of thought.

Start with \textbf{Augustine of Hippo}, \textbf{Dionysius the Areopagite}, and \textbf{Boethius}.

\paragraph{Conclusion}
If you've come this far, then draw your own conclusions.



\flrightit{Posted on 2018-03-17 by Aeneas }

\begin{center}* * *\end{center}

\begin{footnotesize}\begin{sffamily}



\texttt{manfred arcane on 2018-03-18 at 05:53 said: }

I see Peterson as a gateway drug, of sorts, to traditionalism. He is not there of course, not even close, but he is pointing people in the right direction and his lectures – if they bother to follow them – are providing them with tools with which they can approach a number of relevant thinkers. And, given his reach, that is a good thing – even if the 1\% of his followers decide to do a deep dive into Jung or Heidegger thanks to him, and even if the small portion of that 1\% eventually reaches the strain of thought that is espoused by the blogs like this one.


\hfill


\hfill

\texttt{Matt on 2018-03-19 at 08:44 said: }

George Heart, in his book “Christianity: Dogmatic Faith \& Gnostic Vivifying Knowledge” seeks to address the crisis of the lack of faith in modern people, by understanding the root cause, which is revealed in the Law of 3: the active force being the Church, the passive force being man's spirit/Personality, and the equilibrating force (the root of the issue) being change.

What has changed is that modern man has developed a stronger intellect, whereas the Church still preaches a simplistic “milk of the spirit” theology.

Perhaps this Canadian psychologist is gaining a foothold with many of his followers, on topics that don't usually have mass appeal (biblical analysis and hierarchy), partially due to his ability to engage and excite the intellect.

It's also interesting that he has discovered the hierarchical principle in nature, but fails to see it as the reflection of something higher.

http://matthewnanderson.wordpress.com/


\hfill

\texttt{LHOOQ on 2018-03-21 at 07:54 said: }

Manual of Epictetus is a good read to begin an ethical and philosophical life.


\hfill

\texttt{Kenneth Lyon MacNeil on 2018-03-25 at 14:33 said: }

A deep thank you to all who are responsible for the information and insights on this website. 

I will say little else till I do a deep dive into the information on offer here. I do look forward to doing so. 

Much appreciation,

\~{}Lyon


\hfill

\texttt{Robert on 2018-04-07 at 12:13 said: }

There are many traditionalist resonances in Peterson's work, as far as I can see, and he is far from being dismissive of religious belief. Quite the contrary. See, for example:

jordanbpeterson.com/philosophy/on-the-ark-of-the-covenant-the-cathedral-and-the-cross-easter-message-i/

Although I've seen no evidence that he has read Guénon, Schuon, Coomaraswamy or Evola, Eliade seems to have been important for him.


\hfill

\texttt{Bloom on 2021-03-18 at 17:59 said: }

Nice post, a lot of young men were captivated by Jordan Peterson for at least a year or two before moving onto greener pastures. He's a nice tutorial level boss.

Anyways, I want to add a comment on your last bit about the three main currents of Western Tradition. According to Evola, when we look at the Western Warrior Tradition, the three main currents are: Roman, Christian, and Nordic tradition.


\hfill


\end{sffamily}\end{footnotesize}
