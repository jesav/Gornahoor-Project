\section{Rational Man and Real Man}

\begin{quotex}
Every Real Man has realized all the possibilities of the human condition, but each one has does so in a way which is typical of him alone, and which differentiates him from all other Real Men. \flright{\textsc{Rene Guenon}, letter to \textsc{Julius Evola}}

Even in the West, whenever any distinction has been made between these two terms, the \textbf{Personality} has always been regarded as superior to the \textbf{individuality} and that is why we say that this is their normal relationship, which there is every reason to retain. Scholastic philosophy has not overlooked this distinction, but it does not seem to have grasped its full metaphysical significance, nor to have extracted the most profound consequences which follow from it; this is moreover what often occurs, even on occasions where Scholasticism shows the most remarkable similarity with certain portions of the Oriental doctrines. \flright{\textsc{Rene Guenon}, \textit{Man and his Becoming}}

\end{quotex}
\textbf{Aristotle} famously defined man as the Rational Animal, which can only be distorted by the modern mind, interpreting it in a biological rather than metaphysical sense. Animal is an animated or ensouled being. So man is ensouled with a sensitive soul like other animals. Rational, then, is what distinguishes man from the animals. Furthermore, rational refers to the capacity to understand eternal forms, above and beyond the phenomenal world. Hence, it is a third dimension.

To the modern mind, an animal is a biological being, determined by its DNA and bio-electrochemical processes. A rational animal, then, is merely a type of animal, much as animals can be separated into vertebrates and invertebrates. Rational is restricted to instrumental reason, the ability to solve problems, plan ahead, and so on, all qualities to insure the creature's future survival.

\textbf{Longevity} is the Taoist term for future survival, although in its metaphysical sense, it refers to the perpetuity of individual post-mortem existence. Guenon calls this \textbf{Salvation}, the actualization of the individual states, which he distinguishes from \textbf{Deliverance}, which transcends all states of the Being. Since the former is the most likely goal for men today, it is necessary to understand what it means to “realize all the possibilities of the human condition.”

Everyone has an opinion of what is a Real Man: the Poet, the Athlete, the Hero, the Saint, the King, the Father, the Sage, and so on. But to choose one as the ideal would destroy the distinctiveness of the Individual. Nevertheless, we do know what the ideal man is, namely, a rational animal, insofar as he is individuated. This means, as we have often pointed out, that the intellectual soul dominates the sensitive soul, then the vegetative soul and the body. Insofar as that does not hold, a man is such only potentially and not actually. That potential comprises all his possibilities for the human state, which must be discovered and made real.

However, that cannot be accomplished within the limits of individual existence, since only the unmoved mover can initiate motion (or change, which is the same). Hence, a real man must know himself as Subject, Person, or Self, not as an object whether in his consciousness or in the world. Since to know is to be, the Real Man will also be a Person, not just another individual in the world.

There is no formula or program for this task, that is why the hero has a thousand faces\footnote{\url{https://www.gornahoor.net/?p=1492}}.

\flrightit{Posted on 2012-02-07 by Cologero}