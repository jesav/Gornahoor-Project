\section{Multiple States in Christianity}

Philippe Guiberteau, a Catholic writer, was preparing an annotated translation (into French) of Dante's \emph{Banquet}. Although M. Guiberteau had been reading Rene Guenon for 25 years, he was still unconvinced about the applicability of the notion of the multiple states of the being to Catholic theology. He was then put in touch with Michel Valsan. There we have the rather odd situation of a Muslim explaining Catholic theology to Dante's translator.

In three letters to Philippe Guiberteau, Michel Valsan demonstrates that Christianity also has a doctrine of the universal Self and multiple states. He shows that the metaphysical doctrine of multiple states, which considers man in his totality, that is to say as a being mainly “divine”, “angelic” and “human”, is present in Christianity. Its developments are illustrated in three parts:

\begin{enumerate}
\item Theology of Saint Thomas Aquinas 
\item The spirituality represented by John of Ruysbroeck 
\item Catholic esoterism, as expressed by Dante 
\end{enumerate}
But before completing the introduction to Valsan's letters, I shall summarize the states of being as explained by \textbf{Adi Shankara} in \emph{Tattva Bodhah}. It will be a refresher to some and to others it may be new. In either case, Valsan will include a chart the encompasses these states and their correspondence to Christian esoteric teaching.

\paragraph{Three Bodies and Five sheaths}
\begin{enumerate}
\item Gross body: Perceived through the senses. 

\begin{itemize}
\item Annamaya-kosha: Physical body. Nourished by food. 
\end{itemize}
\item Subtle body: Known directly, not through senses. E.g., hunger, anger 

\begin{itemize}
\item Pranamaya-kosha: vital or vegetative soul. Nourished by air 
\item Manomaya-kosha: animal soul. Seat of emotions 
\item Vijnanamaya-kosha: Intellectual soul. The doer. Nourished by impressions. 
\end{itemize}
\item Causal body: All inherent tendencies which are the cause of the other two bodies 

\begin{itemize}
\item Anandamaya-kosha: Person in unmanifest condition. 
\end{itemize}
\end{enumerate}
Plants have a vital soul only, animals, a vital and emotional soul, and human beings possess all three sheaths of the subtle body.

\paragraph{Three States of consciousness}
These correspond to the three bodies listed above.

\begin{enumerate}
\item Waking state: experience of the gross world through the senses 
\item Dream state: experience of our individual dream world 
\item Deep sleep state: the absence of objects, emotions, and thoughts 
\end{enumerate}
\paragraph{The Universal Self}
The Universal Self, or Atman, is the final stage of Union well known in Catholic spirituality. Valsan quotes Saint Bernard of Clairvaux in \emph{De Consideration}:

\begin{quotex}
What then is God? That without Whom there is nothing. It is as impossible that anything be without Him as He Himself is without Him. He is to Himself as He is to everything and, therefore, in a certain way, He alone is, who is the very Being both of Himself and of everything. 

\end{quotex}
This does not annihilate the person, because only \emph{in a certain way} is God all Being. This is identical to the Vedantic doctrine of the Supreme Self. The scholastic philosophers can be read on two levels:

\begin{itemize}
\item The ontological level, which stops at Being 
\item The trans-ontic level, which goes beyond Being (Gilson) 
\end{itemize}
\paragraph{Knowledge and Being}
Aristotle and the scholastics taught the identity of knowledge and being. In the former's words, “a being is all that he knows.” Guenon indicates that is precisely what is meant by metaphysical realization. Since scholastic doctrine was close to being a true metaphysics, and there had to have been more to it than just the religious and exoteric aspects.

Valsan turns to the Church Fathers to show that the Way of Knowledge (i.e., Gnosis) was always part of tradition. Then he demonstrates that Dante and the Fedeli d'Amore were perfectly orthodox in their teaching. So with two of the objections out of the way, Valsan continues.

\paragraph{The Multiple States}
Valsan then launches into the notion of the multiple states of the being. Since most readers will be familiar with Dante and Thomas Aquinas, it is better to let Valsan speak for himself. Although it is clear that Dante understands the multiple states, Valsan has to spend some time on Thomas' understanding of the Active Intellect.

Less known, perhaps, are Nicolas de Cusa and John of Ruysbroeck. Nicolas is mentioned in passing. We can point out, however, that he understood the distinction between the indefinite and the Infinite, not unlike Rene Guenon. No number of steps can reach the Infinite, which is beyond all manifestation.

John of Ruysbroeck has been a personal favorite as we have written about him often, as can be found in this link.

Valsan goes into detail to show that Ruysbroeck's understanding is entirely compatible with the Vedantic teaching of the multiple states.

\paragraph{Mythical Creatures}
An intriguing side trip involves Valsan's understanding of so-called “mythical” creatures and how they can be understood with the idea of multiple states.

Daimons, Genies, and Jinns are situated between the angels and humans. In other words, they have a causal body and a complete subtle body just as humans do. However, they lack a gross (or physical) body.

Centaurs, Sirens, and the like, are in a degree of existence between man and animals, so they share the subtle body with both men and animals. Therefore, they would have an intellectual soul similar to man's but lacking in some regard.

{The complete text of Valsan's letters in English can be found here: \url{https://www.gornahoor.net/library/Doctrine_Multiple_States_Being_Christianity.pdf}

\flrightit{Posted on 2021-02-14 by Cologero }

\begin{center}* * *\end{center}

\begin{footnotesize}\begin{sffamily}



\texttt{Duncan on 2021-02-14 at 09:46 said: }

Thank you for this timely post.


\hfill

\texttt{MICHAEL on 2021-02-15 at 06:31 said: }

How is this different or distinct from Steiner's classification of the physical body, etheric body and astral body and their sheaths… ?


\hfill

\texttt{Cologero on 2021-02-15 at 09:04 said: }

If it is just a theory, then the coincidence would be remarkable.

But if the different states of being are observable facts, then we should expect some similitude in the descriptions.


\hfill

\texttt{MICHAEL on 2021-02-15 at 09:51 said: }

Yes, agreed… I don't believe it's a coincidence. Same with Daimons, Jinns, Centaurs and Sirens which reflect the same intermediate stages of being like salamanders, gnomes, sylphs and undines? How unique is Steiner when he links up these intermediate stages in us and the creatures of the world to the interweaving creative actions of the various levels of the spiritual hierarchies?


\hfill

\texttt{Logres on 2021-02-15 at 19:14 said: }

It is a fantastically dense letter, with loads of important references, particularly drawing together undeservedly obscure quotes from non-obscure theological writers. As for the last bit, Iamblichus taught that if we are good stewards of our efforts, we release our daemon, and we obtain a demigod to guide us. There are three divisions of demigods, leading upwards. From history: “An equation for me has no meaning, unless it represents a thought of God.”-and he wasn't kidding. Like ancient Indian mathematicians, Ramanujan only noted the results and summaries of his works; no proof was worked out for the formulae he came up with. He straightaway credited his work to the divine providence of Mahalakshmi of Namakkal, a family goddess whom he looked to for inspiration. The mathematician said that he dreamed of the Goddess' male consort Narasimha, who is denoted by droplets of blood, after which, scrolls of complex mathematical work unfolded in front of his eyes.” \url{https://www.indiatoday.in/education-today/gk-current-affairs/story/srinivasa-ramanujan-life-story-973662-2017-04-26}

There is also the legend of the Little Red Man of the Tuileries \& his attachment to France:

\url{https://esoterx.com/2014/08/07/the-little-red-man-of-the-tuileries-a-prophetic-parisian-imp/}

As possible examples of the latter mentioned stuff.


\end{sffamily}\end{footnotesize}
