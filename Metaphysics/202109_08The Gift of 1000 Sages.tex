\section{The Gift of 1000 Sages}

We, as a human race, have been blessed with a great number of Prophets, starting with Adam, Seers, Sages, Saints, Philosophers, Doctors, Poets, Worthies, and other Great Men. Most have been forgotten or known by name only. The reason to prefer the antique thinkers is because they have stood the test of time. Centuries of scholars have analyzed and commented on their works, so those who “survive” are worth our time.

\begin{wrapfigure}{ht}{.35\textwidth}
 \includegraphics[scale=.5]{a20210908TheGiftof1000Sages-img001.jpg} 
\end{wrapfigure}

There may be contemporaries at their level, but that may be difficult to discern. They can be treated with caution, more especially if they claim any sort of novelty. Those who can make clear the ideas of the ancients are better guides.

I was struck at one time by Vladimir Solovyov's nonchalant comment that Christianity did not create a new metaphysic, but was instead based on historical events. That made sense of Rene Guenon's famous claim of a common metaphysics that underlie the major religious traditions. There will be incompatibilities in their exoteric teachings which are based on historical events but the metaphysical assumptions should align.

Although Guenon make progress in this area, there is still much to be said about the common metaphysics. Given the two texts I am working with: \textbf{Jacques Maritain}'s \emph{Introduction to Philosophy} and \textbf{Syed Naquid}'s \emph{Prolegomena to Islamic Metaphysics}, this thought jumped right up into consciousness:

\begin{quotex}
If I remove the Latin from Maritain's book and the Arabic from Naquid's book, the metaphysical principles are virtually indistinguishable. I could go on and on, which I shall do over time, but I'll start with this one example: both agree that regarding God, essence and existence are the same. All else follows.

\end{quotex}
This reminded me of what \textbf{C S Lewis} wrote in \emph{The Discarded Image}: There are ancient texts which are not obviously Christian or Pagan, given that their worldviews were so closely aligned with each other. In the same vein, the ideas that Lewis assumed were Christian turn out also to be Islamic. Essence, existence, substance, act, potentiality and so on have the same meanings. These will be explored in details, not as arguments but as expository.

Since the West has effectively abandoned metaphysics, it needs to be reinvigorated by a tradition that still accepts it. Here are some principles to ponder.

\paragraph{Three Worlds}
There are three worlds: the Corporeal, the Imaginal, the Spiritual, corresponding to body, soul, and spirit. They are not independent, so by the Law of Correspondence, what happens in the celestial realm is reflected in the imaginal realm, and ultimately in the corporeal world.

Therefore the world has an inside as well as an inside. Knowledge begins in the senses and we can advance to the spiritual through a process of abstraction. Things and events have a representation and meaning in the life of the soul. To reach the essence of the thing, it is necessary to abstract even from that image into pure thought.

It is one thing to understand the essence of a species like a lion, for example. It is more difficult to understand the essence of an individual, such as someone you may know and love. Religious dogmas, then, are even more difficult. If you try to visualize it, you will be mistaken. You can recite the Apostle's Creed daily, until it loses its meaning. If you are visualizing beings in Heaven as if you are some objective onlooker, you will miss the point. It is also immoral since it is an idol.

This project is an attempt to recall the collective wisdom of humankind, not as a new theory, but as an old truth.


\flrightit{Posted on 2021-09-08 by Cologero}

\begin{center}* * *\end{center}

\begin{footnotesize}\begin{sffamily}

\texttt{Tom on 2021-09-10 at 09:31 said: }

” If you are visualizing beings in Heaven as if you are some objective onlooker, you will miss the point. It is also immoral since it is an idol “.

Just wondered how to square that statement with Tomberg's statement that ” “There is not a shadow of doubt for anyone who takes the spiritual life of mankind seriously, even if he is short of authentic spiritual experience, that the Blessed Virgin is not an ideal only, nor a mental image only, nor an archetype of the unconscious (of depth-psychology), nor, lastly, an occultistic egregore, but rather a concrete and living individuality – like you or I – who loves, suffers, and rejoices. “?\newline
Tomberg's quote is from this site :

\url{https://corjesusacratissimum.org/2012/07/on-degeneration-disaster-and-war-valentin-tomberg-and-the-lady-of-all-nations/}

In general I think the question is how to reconcile ” metaphysical ” formlessness as ontologically superior to form with the statements like this or the fact that Jesus is described in the Gospels as “sitting at the right hand of the Father “. Or that personality is higher than individuality. In other word aren't there bodies in heaven “” Even Dante suggests as much .


\hfill

\texttt{Cologero on 2021-09-10 at 10:09 said: }

There is nothing to reconcile. E.g., she is not a “mental image”. I would change “individuality” to “personality”, but she is as real as “you or I”. It is necessary to understood what it means to be a person, an “I”.

The explanation of the “Person” is on the schedule for a near future post, but it needs a build up. I think it will clarify a lot.


\hfill

\texttt{Cologero on 2021-09-10 at 11:44 said: }

Let me add come further clarifications, although these topics have appeared over and over again as posts.

The soul after death does not exist solely in the Intellectual or Celestial Realm, but also in the Imaginal Realm. So there will be appearances of bodies, but glorified spiritual bodies with the properties of impassibility, agility, clarity, and subtlety.

The physical realm is pure potentiality and thus part of Non-Being, so it is an error to assume that every experience is somehow physical.

Also, with metaphysical realization, i.e. awareness of being a Person, the Person exists in all his states of being simultaneously. Of course, the extent of this depends on the ability and nobility of the person.


\hfill

\texttt{Tom on 2021-09-10 at 12:17 said: }

Thank you for the extra clarification.


\hfill

\end{sffamily}\end{footnotesize}
