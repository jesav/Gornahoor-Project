\section{Vedanta and Western Tradition}

\begin{quotex}
Lacking nothing, contemplative, immortal, self-originated, sufficed with a quintessence: he who knows that constant, ageless, and ever-youthful Spirit, knows himself and does not fear death.

\flright{\textsc{Shankara} }

\end{quotex}
In his important but little read essay, \emph{The Vedanta and Western Tradition}, \textbf{Ananda Coomaraswamy} points the way for Europeans to come to an understanding of the metaphysics of the Vedanta. He begins with this advice:

\begin{quotex}
The educated man of today is completely out of touch with those European modes of thought and those intellectual aspects of the Christian doctrine which are nearest those of the Vedic traditions. \emph{A knowledge of modern Christianity will be of little use because the fundamental sentimentality of our times has diminished what was once an intellectual doctrine to a mere morality that can hardly be distinguished from a pragmatic humanism}. A European can hardly be said to be adequately prepared for the study of the Vedanta unless he has acquired some knowledge and understanding of at least

\end{quotex}
\begin{itemize}[nosep]\small
\item Plato 
\item Philo 
\item Hermes Trismegistus 
\item Plotinus 
\item Gospel of John 
\item Dionysius the Areopagite 
\item Meister Eckhart 
\item Dante 
\end{itemize}
\begin{quotex}
Eckhart, with the possible exception of Dante, can be regarded from an Indian point of view as the greatest of all Europeans. 

\end{quotex}
Here are some of the major highlights:

\begin{itemize}
\item Metaphysics is not a system, but a consistent doctrine 
\item It is concerned not only with conditioned and quantitative experience, but also with universal possibility 
\item There are things which are beyond the reach of discursive thought and which cannot be understood except by denying things of them 
\item The immanent Spirit within you is the only knower, agent, and transmigrant 
\item Ultimate reality is a Supreme Identity in which the opposition of all contraries, even of being and not-being, is resolved 
\item Its “worlds” and “gods” are levels of reference and symbolic entities which are neither places nor individuals but states of being realizable within you 
\item For the metaphysician, it suffices to show that a false doctrine involves a contradiction of first principles 
\item The quest is achieved only when he himself has become the object of his search 
\item The Vedanta can be known only to the extent that it has been lived 
\item Man is unaware of this hidden treasure within himself because he has inherited an ignorance that inheres in the very nature of the psycho-physical vehicle which he mistakenly identifies with himself 
\item What is called “creation” in religion, is called “manifestation” in metaphysics 
\item Only when we are convinced that nothing happens by chance that the idea of Providence becomes intelligible 
\item Each human life has run its course when all its possibilities have been exhausted 
\item Whatever has been an eternal reason or idea or name of an individual manifestation can never cease to be such; the content of eternity cannot be changed 
\item All the states of being are within you, awaiting recognition 
\end{itemize}


\flrightit{Posted on 2010-06-24 by Cologero }
