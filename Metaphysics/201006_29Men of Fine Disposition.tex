\section{Men of Fine Disposition}

If the primordial tradition is necessarily monotheistic, why does polytheism makes its appearance? Again, polytheism, properly understood, is compatible with monotheism provided the understanding of Principle is kept in mind. For example, Pico della Mirandola refers to the angels as ``gods" in his commentary on Genesis, albeit in a quite different sense from the true God. Furthermore, these periods of early history that seem to be polytheistic are still quite far from the prehistory of the primordial tradition and hence reflect a certain decadence. 

Nevertheless, the elite have always known monotheism, and true polytheism can only be attributed to the vulgar. For example, we have the example of Akhnaton who tried unsuccessfully to re-introduce monotheism to the Egyptian people. 

In the case of ancient Greece, the philosophers were aware of the truth of monotheism. The most famous of them, Socrates, was tried and executed for his beliefs. The charge was impiety against the established religion. 

So we see that persecution for religious heresy starts with the pagans and predates popular monotheism and Christianity. Unfortunately, thought crimes have always been part of history and continue to our day, led by the non-Christian governments of Europe. And, today, just as in the case of Socrates, truth is no defense. 

Plato learned a hard lesson from this, and suggested in the Republic that a man should follow the religion of his country, as least publicly. However, privately Plato held a different view. He felt it was improper to divulge knowledge of the true God to the vulgar, for fear of exposing such a great truth to ridicule. What was true in Plato's era is just as true today. 

Plato writes in his second letter: 

\begin{quotex}
Beware, however, lest these doctrines be ever divulged to uneducated people. For there are hardly any doctrines, I believe, which sound more absurd than these to the vulgar, or, on the other hand, more admirable and inspired to men of fine disposition. 

\end{quotex}


\flrightit{Posted on 2010-06-29 by Cologero }
