\section{Multiplicity and Unity}

When we make a judgment about what something is, we are actually intuiting its essence. For example, the squirrel we see is the manifestation of the idea of a squirrel. We experience its existence through the senses, but know its essence through intuition. We have the conviction that it is a squirrel.

When it comes to the most fundamental ideas, we need to develop that same inner conviction. It is instructive to relate \textbf{Vladimir Solovyov}'s understanding of manifestation with \textbf{Rene Guenon}'s.

\paragraph{The Three Moments}
The two agree that God, as Absolute, is One, and as All-possibility, is Infinite. With this in mind, we can re-express Solovyov's discussion of the three moments, or positings, from Chapter 6 of Divine Humanity.

\begin{enumerate}
\item God is One and contains all possibilities in the Divine Mind. Everything is contained in it. Multiplicity is a possibility of manifestation, which cannot be excluded from the One. Furthermore, this multiplicity cannot disrupt the One. This primal unity is immediate and undifferentiated. 
\item The second moment is the manifestation or actualization of this content. This objectification of the possibilities is projected as a separation from the One. Tomberg relates this moment to the teaching of the tsimtsum in the Kabballah. This is no different from the idea of the Void, which is not a possibility of manifestation, for obvious reasons. Since it cannot be manifested, it is the space out of which this objectification occurs. 
\item In the third moment, the Divine Unity is reasserted, but not as Absolute, but as Infinite. Unlike the first moment, Unity is now experienced as mediated and differentiated. 
\end{enumerate}
Analogously, we see this in the microcosm of our own mind. Our spirit manifests itself in a multiplicity of feelings, thoughts, and desires, while remaining one. This does not mean, however, that there is always an awareness of the unity of the spirit. On the contrary, usually a course of spiritual development is necessary to bring about this awareness of the I as subject of the multiplicity of experiences. This I is prior to its manifestations.

\paragraph{Mystical Absorption}
This description seems different from the three moments as described in Oriental religions. As explained to me by some Hindus, it is more like this:

\begin{enumerate}
\item Brahman is the one, absolute reality. 
\item For some reason, perhaps divine play or Lila, the one divides itself into multiplicity. This multiplicity is not real, but is an illusion of the finite mind which has forgotten its true source. 
\item In the third moment, the multiplicity is reabsorbed into the One, like a drop of water returning to the ocean. 
\end{enumerate}
There is a definite difference in understanding between the two. In the first conception, the empirical ego is remains. Drawing on sources similar to Solovyov's, \textbf{Giovanni Gentile} in the \textit{Theory of Mind as Pure Act} makes this clear:

\begin{quotex}
The particular individual is not lost in the being of the I which is absolute and truly real. For this absolute I unifies but does not destroy. It is the one which unifies in itself every particular and empirical ego. The reality of the transcendental ego even implies the reality of the empirical ego. It is only when it is cut off from its immanent relation with the transcendental ego that the empirical ego is falsely conceived.

\end{quotex}
Here again, we see that the empirical ego is a necessary manifestation of the transcendental ego. Illusion, in this conception, arises from the false belief of the empirical ego that it is absolute. In that sense, the Eastern and Western teachings converge.

\paragraph{Karma and Personality}
Although not part of the primeval, originary teaching, karma and reincarnation seem to have become integral parts of Hindu teaching. Karma tries to account for the diversity of individual circumstances as the result of one's past actions, even in previous lives. When pressed, the goal is not actually achieved. If you assume that everyone was equally knowledgeable and competent in some golden age, then the subsequent divergences make no sense.

On the contrary, Solovyov says that the diversity of people is the essential aspect of each person and cannot be reduced to something external. The person is not the differentiated acts of its psychic life. This is proved by the deep sleep state; the differentiated consciousness disappears, yet the person is not destroyed. Solovyov writes:

\begin{quotex}
We first have our primordial, indivisible, and integral subject. In a sense, this subject already contains the whole proper content of our spirit, our essence or idea, and this in turn determines our individual character. … Of course, it is contained there only substantially, in immediate unity with that subject as its inner idea, an idea as yet unrevealed and unembodied.

\end{quotex}
Here we see yet again a Guenonian teaching. The task of the spirit is to manifest those possibilities, which are salvation and liberation. Salvation, then, cannot simply lie in the amelioration of the external circumstances of our life.

Otherwise, human diversity would be unjust and the karmic teaching would make some sense as the ultimate source of justice. Rather, salvation comes as the awareness that the empirical self is not the real self.

\paragraph{The Idea of the Good}
Solovyov discusses the Idea of the Good, another fundamental idea also found in Plato, in the Justification of the Good. The task of higher rationality is to develop a deeper understanding of the Idea of the Good. The result of such understanding will lead to the desire to conform one's own conduct to that idea. This conformity must be freely chosen and cannot be mechanical. That is, there can be no pill or medication to coerce someone to behave morally.

If creation, or manifestation, is good, then it is good to manifest one's possibilities.



\flrightit{Posted on 2014-05-09 by Cologero }
