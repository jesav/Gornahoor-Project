\section{The Person as Subject}

Regarding translations from Introduction to Magic:

\begin{enumerate}
\item \textbf{Guido de Giorgio} — \textit{Notes on Ascesis and anti-Europe}\footnote{\url{https://gornahoor.net/?p=6397}}
\item \textbf{Julius Evola} — \textit{Esoterism and Christian Mysticism}\footnote{\url{https://gornahoor.net/?p=6438}}
\end{enumerate}
These essays deal with the question of finding a guru and recognizing the marks of an esoteric tradition. Before we do that, I am providing some questions to ponder, in anticipation that some light will be cast on them. The first issue is to grasp the ultimate source of an esoteric, metaphysical, or religious text. In the recent translation of \textbf{Julius Evola}'s review of \textbf{Bhagavan Das} and R. Pavese\footnote{\url{https://gornahoor.net/?p=6328}}, he gave us two options: either deduction or transcendental experience.

Now \textbf{Giovanni Gentile}, in his brief talk to educators\footnote{\url{https://www.amazon.com/The-Reform-of-Education-ebook/dp/B005DI65Z8/ref=sr_1_1?s=digital-text\&ie=UTF8\&qid=1367988340\&sr=1-1\&keywords=Giovanni+Gentile}}, said that philosophers will write a long work attempting to logically prove its main point, which suddenly arises at the end. In reality, he explains, the main point is arrived at first and the logical scaffolding added afterward. That established, it is fair to ask if the source of that idea is a random thought, a genuine insight, or a real spiritual vision. In the last case, the deductive and logical procedure can hint at, but cannot lead, to that same vision. As Evola wrote:

\begin{quotex}
The Vedas are the expositions of what the rishis have seen; in the rishi, meaning those who have realized themselves up to the level of Brahman, the deductive procedure of their knowledge could have in itself a character of intrinsic evidence and a justification, that cannot appear to those who look at it from the level of finite existence. 

\end{quotex}
This is the task and the hurdle. When reading a text, can we be sure it is the fruit of the self-realization of a spiritual clairvoyant?

\paragraph{Knowing Oneself}
Since the command to ``know thyself" is at the heart of the Western tradition inherited from the Greeks, that is where we shall start. On the one hand, there is the teaching of the person in the Medieval tradition; for this, we will rely on \textit{Reality} by \textbf{Reginald Garrigou-Lagrange}, O.P., particularly Chapter 58, ``Ontological Personality". The method will not be a logical analysis, which in any case is best left to academic philosophers, but rather to determine if it reflects, or could possibly reflect, a personal realization.

This will be compared to the method of \textbf{Ramana Maharshi} to recognize the real ``I". For this, I will rely on In Days of Great Peace by \textbf{Mouni Sadhu} and on the Vivekacudamani by \textbf{Shankara}, by way of the commentary of \textbf{Swami Dayananda Saraswati}.

Regarding the \textbf{Person} in the Thomist sense, which term is used also by Rene Guenon and Julius Evola, Fr. Garrigou-Lagrange tells us:

\begin{itemize}
\item Person (human, angelic, or divine) means a subject, a suppositum which can say ``I", which exists apart, which is sui juris. 
\item Ontological personality is that which constitutes the person as universal subject of all its attributes: essence, existence, accidents, operations. 
\item The person, or subject, is distinct from its existence, distinct from its nature 
\item Personality is form. It is real, distinct from nature and from existence. Personality is that by which a person is immediately capable of independent and separate existence. 
\end{itemize}
Hence, in the Medieval understanding, the Person is the pure Subject, the ``I". It cannot be an object; it is transcendent to its existence, which can neither determine it nor create it. Not only that, it is prior even to its essence. It cannot be the results of genetics, biochemical processes, or even cultural influences.

The personality is the form. This is compatible with what Evola wrote in the Esoteric Origins of the Species. The matter of the person is the genetic configuration in which the personality incarnates itself. We would add here that there is also the psychic configuration to take into account. The personality, then, would include the spiritual race and sex of the person that manifests. The measure of his power is the degree to which he is capable of manifesting all his possibilities in a given time and place.

From this exposition, the two obvious questions arise. The first question is whether that is a legitimate deduction from the Thomist teaching of the Person. If so, it would certainly be consistent with other metaphysical teachings.

The second, and more important, question is whether the teaching on the Person comes from a ``seeing", i.e., a spiritual realization. The Person as subject, as pure ``I", is really the same as Atman in the Eastern teachings. It is the subject not just of its essence and existence, but of all its experiences (``accidents"). This is the ultimate ``knowing oneself" and is also the goal of the Vedic teachings. It is actually non-trivial, since the large majority of people understand themselves as objects, not subjects. They believe they are their thoughts, feelings, opinions, likes, dislikes, and so on. Of course, there is nothing constant in that, since thoughts, feelings, etc., vary from day to day, from moment to moment. The Person as subject, however, is transcendent to all that.

That is the certain and necessary deduction from the definition. Was the definition of the Person an intellectual exercise, a logical tour de force, or did Thomas truly understand himself as a Person? If so, that means the Thomist teaching has the marks of a Tradition. Now whether he had to undergo some initiation or special training to reach that point, we can't know. Nevertheless, it is a mistake to presume that people at different times and places have the same interior experience of the Self. For example, the highly educated today believe that the mind or spirit is an illusion, at best the epiphenomenon of electro-biochemical processes. It is difficult for me to imagine how that feels from the inside in order to take it seriously, but it is a widely held opinion in intellectual circles.

The bottom line is that it is reasonable to presume that Thomas knew and understood himself as the absolute Subject of his experiences and the True Will of his operations. In what follows, we will look at the technique taught by Ramana Maharshi to reach that same self-understanding.


\hfill

\paragraph{The Technique of Vichara}
\textbf{Mouni Sadhu} made the journey from Theosophy to the Hermetic tradition, although he seems to have found his ultimate resting place in the Vedanta as taught by \textbf{Ramana Maharshi}. Interestingly, he and \textbf{Valentin Tomberg} had the same Russian Tarot teacher, although not at the same time. This may have some significance later.

In his book describing his experiences in India, Mouni Sadhu devotes one chapter to the specific practice of vichara. Ramana Maharshi is best known for posing the question ``Who am I", which his disciples are to meditate on constantly. In attempting to answer it, the disciple sees that he is not his thoughts, not his emotions, not his bodily sensations. Sadhu explains:

\begin{quotex}
When we try to shut out all the whims and fancies of our restless mind, and to concentrate on the one chosen for a definite purpose, the mind fights desperately in order to resist control. It depends upon our will, who wins this fight. Find out who is the creator of thoughts and you have achieved the goal. Such is true realization. 

\end{quotex}
He would repeat the mantra ``who am I", eventually reaching several thousand times per day. This conscious control of thought little by little crowds out all those other thoughts that arrive mechanically and then periods of inner quietude become longer. In this way, he comes to realize his true ``I", which is obviously transcendent to any of the contents of consciousness. This sounds in some fundamental way like the Way of the Pilgrim; it would be interesting if it leads to the same understanding of the ``I", or even a deeper one. But I don't know of anyone who has tried each technique over two different periods in order to compare.

The vichara practice leads to the source of life and even eternal life, which is understood as uninterrupted consciousness. This is the immortality of the spirit. So the true ``I" or Self (or Person) is consciousness, since that is unchanging despite the never ending flow of thoughts, visions, emotions, likes, sensations, etc., that the consciousness experiences. This can be understood in the various cycles. First there is the breath. After each exhalation, there is no guarantee you will take another breath. Try to maintain consciousness through the entire inhalation and exhalation. This must be why the prayer of the heart is synced with the breath. The next such cycles is wakefulness and sleep. It is more difficult to maintain the continuity of consciousness through the period of sleep. The first step is to learn to be conscious while in the dreaming state, and obviously I mean a detached awareness that one is dreaming.

Analogously, consciousness can be maintained after life ends, which we have tried to document in the review of the \textit{Tibetan Book of the Dead} and \textbf{Fr. Seraphim Rose}'s \textit{The Soul after Death}. As was seen, that period immediately after death can be shocking, terrifying, and a severe test. Provided one has learned to recognize who he really is, this transition will be experienced in a better way.

Mouni Sadhu recommends meditating on this and ``the truth will be made clear even to the outer mind." The danger of misunderstanding is that the technique described seems too mechanical. The reality is that an inner transformation is always required; the vichara may create the space, i.e., a mind free from random thoughts, in which that realization can occur. However, it cannot by itself ``create" such a realization. In the west, such a realization, which is also immortality, is tied in to certain moral virtues. Mouni Sadhu, like \textbf{Guido de Giorgio}, is a ``vedantized" Catholic (or perhaps vice versa), so he ties this understanding to his own background:

\begin{quotex}
Such are the heavens promised to the righteous and the saint, as told to us by Christ. For them there is no death any more. How clear now are the words of the great teacher of humanity! 

\end{quotex}
So we see that Mouni Sadhu is relating salvation to self-realization. But I don't know that salvation in the western sense is the realization that one is a Person, i.e., a transcendental I. This may be part of it, but the post about St. Gregory and the Soul asks for more. We will end with a few thoughts about this.

First I want to point out a teaching by Ramana Maharshi on reincarnation that is quite similar to Valentin Tomberg. In Mouni Sadhu's words: [Ramana Maharshi] denies reincarnation in the realm of spirit, but otherwise speaks about it as an established fact. Tomberg wrote that reincarnation is a fact, but not a necessary principle. But this is a separate discussion.

There is another point to ponder although the two options proposed will not be convincing to everyone. From the western perspective, the effectiveness of the eastern techniques is not denied, but it is relegated to a lesser level as merely ``natural". Of course it is not natural in the contemporary understanding of the term, but is it natural in the sense that there is no need for ``grace" or ``sacraments", nor even a moral conversion. Anyone who is willing to follow the exercise of asking ``who am I", may reach the state of Samadhi.

The other perspective is expounded by \textbf{Paul Sedir}, the French Hermetist who was greatly admired by Mouni Sadhu and also by Valentin Tomberg. Like \textbf{Thomas Aquinas}, Sedir seems to have undergone an intellectual or spiritual conversion, after which he gave up all his interest in initiatic societies. In the meditation on the Fool, Tomberg provides an extensive quote from Sedir. Sedir clearly has a deep understanding of metaphysics and the supra-individual states. He describes metaphysical realization as identification through knowledge. He describes the successive stages:

\begin{enumerate}
\item The unlimited development of all possibilities contained virtually in the individual 
\item Going beyond the world of forms to a degree of universality which is that of pure Being 
\item The final aim is the absolutely unconditioned state, free from all limitation. The liberated being is then truly in possession of the fullness of his possibilities. This is union with the supreme Principle. 
\end{enumerate}
Clearly, this must have been quite influential on the young \textbf{Rene Guenon}, since it could have come from his own pen. Sadhu and Maharshi recognize step 3, specifically, union of Atman with Brahman. Guenon criticizes Christian understanding as not moving past step 2, the union with Being.

However, the fact that Tomberg knows this, and even includes it as the conclusion of his meditations, shows that Guenon is mistaken. The point is not that Christian Hermetic meditation stops as step 2, but rather that is aim is greater, or at least, different. Actually, it aims for a step 4. Sedir writes:

\begin{quotex}
True metaphysics cannot be determined in time: it is eternal. It is an order of knowledge reserved for an elite … and then, all existing manifestations of the Absolute are not there for the sake of being ignored; to abandon them because they encumber us, as the yogi or the arhat does, is neither noble nor Christian. 

\end{quotex}
If you recall the post on the soul, one of the aims of salvation is that ``It exteriorizes itself in a life of virtue and beauty," while the yogi is no longer interested in his own exteriorizations or possibilities of manifestation. The essential difference is in Ramana Maharshi's teaching on the \textbf{I-I}, as opposed to the \textbf{I-Thou} of the West. Hermetic enlightenment, hence, is ultimately the alchemical marriage of the duality, not the I-I relationship of the Self with Brahman as in the Vedanta.



\flrightit{Posted on 2013-05-08 by Cologero }

\begin{center}* * *\end{center}

\begin{footnotesize}\begin{sffamily}



\texttt{Logres on 2013-05-09 at 22:17 said: }

``The Self beyond the Self" in Tomberg's phrase: this clarifies that immensely. I wondered the other day if the Gnostics ``got" the other side of Being, arguing that the Creator was ``manifestation" and therefore, unreal next to source of Being. Of course, since their doctrine was schizophrenic, they couldn't do much more with it than the Christians who didn't understand God as ``ground of Being" (supra-personal)?


\hfill

\texttt{Avery Morrow on 2013-05-10 at 04:22 said: }

As always, thank you for the insight. I have recently been wrestling with an esoteric text which claims, in its final line, that ``crossing borders manifests God". I was shocked by this statement and tried to re-translate it many times, but these stages seem to acknowledge it.

I thought a key aspect of Guénon's criticism of Christianity was that initiations somehow degraded into exoteric ceremonies, and because initiations are given away so freely, Christianity gets confused about who the ``elite" are in the first place. I do like this I-Thou and ``exteriorization" idea, though.


\hfill

\texttt{Synodius on 2013-05-10 at 16:13 said: }

Do you think the resulting states are different or is it just their expression that differs because of previous formation?


\hfill

\texttt{Jason-Adam on 2013-05-10 at 16:32 said: }

Cologero has shown that Christian initiation and esoterism does exist – not only in the past but also today – and has pointed us to writers such as Tomberg and Mouni Sadhu who can help us in our searching. 

What I would like to ask is – did the Roman papacy ever possess the esoteric knowledge and if it did -which I think so – why and how did it lose it at some point ? I think there is a need to investigate the history of the Catholic Church to uncover why events have come to the point they have.


\hfill

\texttt{Mihai on 2013-05-10 at 17:56 said: }

There is a difference between Christian esoterism and esoterism within a Christian setting. Christian Hermetism partakes to the latter, so does chivalric initiation around the Grail legend- both are foreign doctrines applied to a Christian context, the first egyptian(or egypto-hellenic), the other Nordic- which is why they have always been considered, by the clergy, as something heterodox. 

Whether a genuine Christian esoterism (although there is no real demarcation line between exoterism and esoterism in Christianity) existed in the West- as does the Hesychast discipline in the East- is something very probable. It was probably lost when the West turned hopelessly rationalistic, reducing knowledge to dialectics. Certainly, around the time of the confrontation between St Gregory Palamas and Varlaam, such things were considered madness or laughable in the West, as Varlaam's (who was raised with the scolastic teachings) reactions show. 

PS: Regarding Tomberg, about whom there is much talk here, though I haven't read him, I did encounter some fragments and pages from his ``Meditations…". Looking also to the influences he cites, he strikes me as rather syncretistic and doubtful. The only thing that made me suspend my judgement on him was the fact that his book contained an afterword from Hans Urs von Balthasar- a `mainstream' Catholic theologian. I really don't know if he is so trustworthy.


\hfill

\texttt{Synodius on 2013-05-10 at 20:15 said: }

Mainstream Catholic theologians nowadays tolerate almost anything, the link below is just an illustration from a traditional catholic site:

http://www.traditioninaction.org/RevolutionPhotos/A202rcMassageNun.htm

Wondering if those nuns sitting in their zendo even heard about the Hesychast…

Mentioning Hesychast, It would be really nice to see more posts about it here. 

I've read a few chapters from Tomberg's Meditations and it seems to be written maybe more for theosophists than for catholics, but still very inspiring. Maybe the gap between Hermetism and Catholicism is still too wide and so the main importance of Tomberg is that he even tried to join these perspectives.


\hfill

\texttt{Avery Morrow on 2013-05-11 at 03:41 said: }

I expect that further insight can be gained into this topic from the book Guenonian Esoterism And Christian Mystery by J. Borella, but I haven't yet had the opportunity to read it.


\hfill

\texttt{Jason-Adam on 2013-05-11 at 13:02 said: }

With Tomberg the point to remember is that he wrote to an audience of occultists – thus he uses the writings of men respected by his audience to prove his point. 

According to an interview I've read with his publisher, Tomberg was initiated by Rudolf Steiner and given the task by the Master to bring Anthroposophy and Catholicism together. 

I have read Borella's book and it is essential – it is based mainly on the works of St Dionysius and shows how the Catholic life can be perceived in a higher and inner dimension. For him esoterism is not a secret knowledge but the exoteric knowledge perceived with a different sense. Schuon believed the same.

Dugin believes that the West cut its ties with the Tradition when the Papacy (according to his understanding of things) cut ties with the East in 1054 – and earlier under Charlemagne – but I do not want to think that in order to be a man of Tradition I have to abandon my Church and my Culture and become an oriental.


\hfill

\texttt{Chris on 2013-05-11 at 15:13 said: }

Is it possible that some of the gnostic sects of late antiquity were not actually heretical dualists, but were simply distinguishing between the qualified and the unqualified absolute in the Guenon-Schuonian sense.


\hfill

\texttt{Caleb Cooper on 2013-05-11 at 21:58 said: }

Mihai, some indicators of where Tomberg stands; in `Lazarus, Come Forth!' he has nothing nice to say about Vatican II and plenty to deplore; in his meditation on the Emperor card he says that Europe is suffering from the absence of the Emperor (putting him to the right of every ethnic nationalist); and he was driven out of the Theospohical society because of his adamant Christianity. They rightly saw that Tomberg's aim was to bring all under Christ; as he quotes with the highest reverence in MOT ``Heart of Jesus, King and center of all hearts" (Litany of the Sacred Heart).

The last 2 popes seem to have approved; Blessed John Paul II had a copy prominently on his personal desk http://theosophist.wordpress.com/2008/04/24/pope-john-paul-ii-meditating-on-the-tarot and Pope Benedict XVI approved the Russian edition when he was a cardinal. 

You could say that MOT was Tomberg's attempt to baptize Hermeticism. I don't think anyone could read it with an open heart and NOT come away with a deepened commitment and appreciation for Christ and the Church. 

The mages that he quotes tend to be those who practiced personal and ceremonial magic in their youth, but eventually through seeing the miracles done by simple genuine Christian like Jacob Boheme, renounced all personal and ceremonial magic and embraced the Church. For example, here Tomberg quotes in MOT from Eliphas Levi's last work:

``The ancient rites have lost their effectiveness since Christianity appeared in the world. The Christian and Catholic religion, in fact, is the legitimate daughter of Jesus, King of the Mages. A simple scapular worn by a truly Christian person is a more invincible talisman than the ring and pentacle of Solomon. The Mass is the most prodigious of evocations. Necromancers evoke the dead, the sorcerer evokes the devil and he shakes, but the Catholic priest does not tremble in evoking the living God. Catholics alone have priests because they alone have the altar and the offering, i.e the whole of religion. To practice high Magic is to compete with the Catholic priesthood; it is to be a dissident priest. Rome is the great Thebes of the new initiation…" 

A number of quotes are used to illustrate negative principles, such as in the Death, Devil, and Tower of Babel card. A repeated theme is that personal and ceremonial magic are real, but illegitimate and powered by the serpent, man's ancient enemy, whom Tomberg shows again and again at work in the forces of modernity. The only legitimate magic is theurgy, sacred magic which comes from above.

Many of the quotes are also used to bring the fullness of Christianity out by comparing it to other faiths, then showing the larger scope and nobility of Christianity. On Buddhism Tomberg says ``The Buddha saw the true nature of the world and that it was sick. Considering it incurable he instituted an effective path to escape it. Christ saw the same truth, but believed that the world could be cured and instituted the path to heal the fallen world from the inside." 

Because of the book I gave up a lot of dangerous spiritual practices that probably would have killed me, joined the Church, and committed myself to prudently growing in the Tradition of Christ. I've heard of others who've come from New Age backgrounds and ended up as Traditionalist Catholics. As I said above, it's hard to imagine Tomberg not planting the seeds that lead to a growing conservative and traditional temperament in his readers.


\hfill

\texttt{Caleb Cooper on 2013-05-11 at 22:34 said: }

Jason, there's was some pretty bad rot in Rome even before the split. For example, they let a Prince put his 18 year old son on St. Peter's Chair as Pople John XII in 955. He was about as big a disgrace as could be imagined, and appeared to meet his end at the hands of an outraged husband who caught him having an adulterous encounter with his wife. 

The Church should have a head. I'd argue though that it should have been the Patriarch of Constantinople, and I'm Catholic. Rome's Imperium was spiritually derived from Troy, and Constantine, at the time Christianity become the faith of the empire, re-transferred the Empire's capital to it's spiritual birth place. That the Papacy committed the fraud of the Donation of Constantine shows that they were aware of their shaky Imperial legitimacy. In the east they called their city New Rome, signifying that old Rome was no longer the center. Popes like John XII make me think that old Rome may very well have already been corrupt and that Roman legitimacy lay in Byzantium.


\hfill

\texttt{Cologero on 2013-05-12 at 11:44 said: }

It is necessary to clear up a question of fact. Tomberg never met Rudolf Steiner in person. Allegedly — and Tomberg himself never claimed this in writing — Steiner appeared to him from the spirit world with that request. If that is true, then Tomberg is a fraud since he is quite clear about his sources and goals, viz., to develop the Hermetic tradition. Anthroposophy regards Catholicism as a past stage then had to be overcome by its new spiritual teaching. Tomberg's conversion was an affront to the very foundation of anthroposophy; at best, it can only be considered a minor stream in the larger Hermetic current.


\hfill

\texttt{Logres on 2013-05-13 at 10:05 said: }

It is possible that some were trying to do this, but whether or not they succeeded, it would have been just as difficult for them to do this as it was for the ``new Western Church" to uphold a purely exoteric doctrine over against them. That is, the split affects both sides, and distorts them – Gnosticism is no longer ``Gnosticism" when it is separated from rite and exoteric dogma, which it had to be (almost) to survive. But you are right, that may have been the intent and the deed of some sects.


\hfill

\texttt{Jason-Adam on 2013-05-14 at 13:54 said: }

I am not well versed in Rudolf Steiner but having seen some good posts on him at Gornahoor I plan on reading up on his ideas. I keep an open mind about him for now, obviously I am with Guenon and Evola and oppose progress but if Evola liked Steiner I feel he is worthwhile enough to at least read his books. 

People like to diss the Papacy but there is no getting around that if you believe in Jesus, you must follow the Rock he ordained to lead his Church – the infallible seat of Peter in Rome. There is no Catholic church where there is no obedience to Peter. The schismatic easterners are just sectarian Photian national churches no different from the Anglicans or the Lutherans. Read De Maistre's The Pope and Soloviev's Russia and the Universal Church if you haven't already……..


\end{sffamily}\end{footnotesize}
