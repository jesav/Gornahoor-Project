\section{Sacred Science 101: Introduction}

Our online discussions for 2022 begins with an extended commentary on \emph{Man and his Becoming} by \textbf{Rene Guenon}. This is based on our new translation, which will be published by the end of the year. This translation will use contemporary English style, punctuation, and production layout. Sanskrit words will be rendered in today's common usage with diacritics removed.

However, the goal of this series is to reframe the metaphysics based on Adi Shankara's explanation of the Vedanta, in Western terms. Guenon provides several examples of how that might be done, based primarily on Hermetic and Scholastic metaphysics. This will require a new vocabulary; this poses its own problems because there is no consistent terminology that we can rely on.

Our approach is not merely intellectual, but also phenomenological, i.e., the inner experiences associated with the doctrines will be described insofar as it is possible. Hence, it will be tied into practical exercises to develop awareness and understanding. Without that, many of the concepts described will be difficult to understand. The ultimate goal is to maintain Tradition in the West, the best we can.

What follows are highlights from previously published posts. Each topic will be treated in more detail as time goes on.

\paragraph{The Descent of the Absolute}
In \emph{Man and his Becoming}, \textbf{Rene Guenon} describes the manifestation of the Self into the human state according to the Vedanta. The West has its own Traditional understanding. It is not different, but it is worth the trouble to redo the same project in more Western terms. This post can provide the barest outline.

By the Absolute, we mean generally what is called Brahman, that is, whatever is beyond being. The Self (Atman) is the principle of the individual. As such, the Self transcends all manifestation. The descent through the degrees of being begins with formless manifestation, then to formal manifestation in the subtle state and ultimately to the gross state, i.e., the corporeal body. This is not a process that occurs in time. Nor does the Self descend through these stages, since it is their principle.

The human state is said to be rare, in the sense that it is just one possible state out of many. The situation of the being in the human state allows the being to act in a way to acquire knowledge, unlike other states that are passive. Thus, it is special and should not be squandered.

These are the primary degrees although each one can be further divided, potentially indefinitely, into subdegrees. The Self becomes conscious of the world of sensible manifestation through:

\begin{itemize}
\item Thought 
\item Inward senses or wits 
\item Individual consciousness 
\item Five senses 
\item Five organs of action 
\item Life force 
\end{itemize}

\paragraph{Buddhi}
The buddhi, Universal Spirit, or Higher Intellect, belongs to the realm of formless manifestation. As such it is the first degree of the manifestation of the Self. The possibilities of manifestation become ideas or essences in the Higher Intellect.

It is the principle of the formal manifestation of the Self and thus is the unifying principle. Rene Guenon calls it the “Spiritual Sun which shines at the center of the entire being.” Put another way, it is the “spark of divinity,” even if only virtually.

Thought is the faculty which gives form to ideas and associates them to each other. Intuition is how we experience the Higher Intellect, which is beyond discursive thought.

However, in the Higher Intellect, it is not yet individualized consciousness.

\paragraph{Manas}
The characteristics of the Manas, or Universal Soul, relate to faculties of the formal order.

Mental faculty or inward sense. Individual thought, memory, imagination

\textbf{Five elements}: Ether, air, fire, water, earth

\textbf{Five Senses}: hearing, touch, sight, taste, smell

\textbf{Five Wits}: memory, estimation, fantasy, imagination, and common sense

\textbf{Five actions}: Ingestion, excretion, reproduction, locomotion, grasping

From the esoteric perspective, the senses are prior in the ontological sense. Moderns believe that somehow matter is creating sensations in the brain, although the mind is already prepared to have sense experiences. The same for the other features. The desire for locomotion and grasping, for example, direct evolution. They are not the serendipitous results of a random evolutionary process.

\flrightit{Posted on 2022-09-18 by Cologero}