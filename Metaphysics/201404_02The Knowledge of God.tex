\section{The Knowledge of God}

The transcendental unity of religions is true insofar as it pertains to metaphysics. In Lecture 6 of \emph{Lectures on Divine Humanity}, \textbf{Vladimir Solovyov} shows his agreement with that point even for Christianity. In particular, he writes this about Neoplatonism:

\begin{quotationx}
It is impossible to deny the connection between Philo's doctrine and Neoplatonism, on the one hand, and Christianity, that is, precisely the Christian doctrine of the Holy Trinity, of the triune God.

The first speculations concerning God and His inner life by such Christian teachers as Justin the Philosopher, Hippolytus, Clement of Alexandria, and especially Origen reproduced the essential truth of the doctrine of Philo and Neoplatonism. 

\end{quotationx}
Furthermore, Solovyov traces this connection all the way back to the Egyptian theosophy of \textbf{Hermes Trismegistus}. This is because metaphysics deals with essences, but it is fair to ask if that is exhaustive, when it comes to existing things. \textbf{Rene Guenon} skirts around this is his notion of possibilities of manifestation. In that sense, an existing thing, or manifested possibility, is metaphysically no different from a possibility of non-manifestation. This seems to be what \textbf{Aristotle} is saying when he claims that God does not know particular things.

\textbf{Julius Evola} disputes Guenon on this point. As we saw in the discussions on the Individual and the Becoming of the World, to become manifest, an idea requires an act of will. As such, the individual is not an object of contemplation. Evola thought he was opposing action to Guenon's contemplation. However, they are not in opposition, since they are united under the “Person”, who is both the center of contemplation and of will. As we shall see, Evola was not recovering pagan ideas as he believed, but was actually recuperating Christian ideas.

\paragraph{The Pagan and Contemplation}
The fundamental revelation made to the rishis in the Vedas was the existence of a higher self and that the manifest world was the result of ignorance, or avidya. The solution, then, was to dissolve the lower self, as knowledge of the higher self was itself liberation. That knowledge is the end of ignorance. This higher self then passively observes the passing phenomena of the world. However, this involves the dissolution of the person, because there is no longer a will. The self does not create the phenomena, but merely observes.

This solution is carried into Buddhism. It deepens the Vedic revelation when it sees suffering as inherent in the phenomenal world and it ties this suffering to desire. The Buddhist solution is therefore the elimination of desire, or the will. Again, the person is denied by following this path, since a person has will by definition.

For the Greeks, too, the highest life was contemplation. Like the rishis, ignorance was the cause of suffering and ignorance was removed by knowledge of the self. This knowledge would lead to “justice”, or the proper arrangement and relationship of the various parts of the soul to each other. The world, then, was contemplated as an object.

\paragraph{The Christian Revelation}
Among the world's traditions, Solovyov explains the distinctiveness of the Christian Tradition:

\begin{quotex}
The originality of Christianity lies not in its general [metaphysical] views, but in positive facts, not in the speculative content of its idea but in its personal incarnation. This originality cannot be taken away from Christianity, and to affirm it, one does not need to try to prove, against history and common sense, that all the ideas of Christian dogmatics appeared as something absolutely new, that they fell, so to speak, ready-made from heaven. 

\end{quotex}
Solovyov explains the relationship between a person and an idea. Without a subject to actualize it, an idea would be passive and impotent. It could not really exist. To achieve real, full being, the inner unity of person and idea is necessary.

By analogy, the absolute idea is also determined in its inner subjective existence as a particular and unique person. This insight requires a living God, God as a fact, not the passive Brahman. The living God reveals Himself as “I am”, not with respect to particular things but with respect to the All, or the Infinite.

Still, the initial revelation to the Hebrews was incomplete and partial. They saw God's will as arbitrary in Himself, but compulsory for humanity. But the understanding deepened, and the will of God could not be understood as despotism, but rather as the consciously chosen good. For the person, the compulsion is experienced as internal necessity, i.e., true freedom. Hence, the path of salvation, or liberation, is not the annihilation of the will as in the various forms of paganism, but rather the alignment, or better submission, of the personal will to the divine Will. Since God is Absolute and Infinite, there can only be one such Will.

So on the one hand, there is the revelation of the absolute personhood of God made to the Hebrews. On the other, the absolute idea of Divinity was grasped in Greece. The complete knowledge of God requires the synthesis of both these elements. That was the task of Christianity.


\hfill

\flrightit{Posted on 2014-04-02 by Cologero }

\begin{center}* * *\end{center}

\begin{footnotesize}\begin{sffamily}

\texttt{Michel on 2014-04-03 at 13:41 said: }

“Hence, the path of salvation, or liberation, is not the annihilation of the will as in the various forms of paganism, but rather the alignment, or better submission, of the personal will to the divine Will. Since God is Absolute and Infinite, there can only be one such Will.”

I have a question, when you say “personal will” do you mean individual? Or do you mean Personality, as in Atma?Assuming you mean individual: so whether or not one's individual will subsists doesn't really matter, because God's Will happens anyways. Whether the individual aligns himself to it and does not experience anything higher than his individual self, or whether he goes beyond it (annihilation as a term to describe this happening is rather misleading, because it implies a loss of something and hence a limitation on the higher station that the being would achieve;synthesis would be a better term, the principal of the individual doesn't deprive him of his individuality, because, being its principal, it has synthesized it in itself) doesn't affect this. 

“By analogy, the absolute idea is also determined in its inner subjective existence as a particular and unique person. This insight requires a living God, God as a fact, not the passive Brahman. The living God reveals Himself as “I am”, not with respect to particular things but with respect to the All, or the Infinite.”

As you have said in your article, there are Possibilities of non-manifestation that are non-manifest-able in accordance with their nature and there are those that are manifest-able being Possibilities of manifestation. They are “passive” insofar as they are Possibility and they are “active” insofar as they are Infinity. Possibility and Infinity refer to the “passive” and “active” “modes” of the Infinite. To use an analogy, Dave is man. Dave can jump, run, sing, shout, sleep and so on. Dave's existence therefore necessitates/affirms jumping, running, singing, shouting, sleeping and so on. Dave encompasses all these possibilities simply because he exists and those are the possibilities that are in concordance to his nature; in his passivity, Dave still “emanates” running, jumping, sleeping etc. If Dave jumps, runs, sings, shouts and sleeps at the same time, then Dave would have instantaneously actualized all the possibilities that his existence necessitates in accordance with his nature. So when you say, “This insight requires a living God, God as a fact, not the passive Brahman” are you referring to the Infinite in its “Passive Mode”, viz Possibility?

Consciousness is the sufficient reason for the perception of a state of being. I can perceive this state of being, because I am conscious of it. Destiny is the sufficient reason for the persistence of a being in individual mode. I am here because I have a destiny here. To go beyond one's destiny would be to realize it, by harmoniously synthesizing all the individual modes of existence that exhaust one's destiny. To perfectly and wholly do so, would be to align oneself with Providence and hence realize the Divine Will in relation to oneself. But would one have realized the Divine Will for that which lies beyond oneself and individuality? If so, wouldn't this be a virtual realization, still tied to the cosmological perspective? My question is, why stop there? Why not go beyond one's individuality wholly? Why not Realize the Divine Will that is truly Metaphysical, beyond both the formal and formless states of Being? Beyond the soul and the Spirit? Why not realize the Person, and hence perfect the realization of the Divine Will? If the Entity/Unraveled Archetype possess in itself something that entitles it to supra-individuality, nay to even supra-manifestation , supra-spirituality, true Metaphysics, why not actualize that something and realize this?

I really mean these questions; I am not the type of person to be insincere, so do not take these the wrong way. Forgive me beforehand if they are vague and want of clarity; and if they show a misconstrued understanding of this post and theme in general.


\hfill

\texttt{Cologero on 2014-04-03 at 21:40 said: }

I'm not sure what you are getting at, Michel. We understand by “person” a being with consciousness and will; the individual is more vague. Atma is consciousness, but how can the Atma be a person? That is the fundamental point of the essay. If God's will happens anyway, then why do we pray “Thy will be done, on earth as it is in heaven”?

I don't know why I need to repeat in a comment what was already written, but if a “person” loses his will, then he is no longer a person … that is an annihilation. Our point, however, is that the person cannot be annihilated, as he is immortal. The highest state is “liberation”, i.e., total freedom, i.e., absolute self-determination.

The pagans all regarded knowledge as the solution. Even Socrates said the the virtues would follow from knowledge. We know, however, that the moral problem arises in the will. A few weeks ago, a commenter claimed something similar. When I pushed her for the reasons why enlightenment, as she conceived it, could not be taught, she replied with a list of moral failings of the will. I pointed that out to her, and she chose to vanish.

You described Dave in his animal nature; that is not the actualization of his possibilities as a man. That actualization is theosis, becoming fully conscious and fully free. How you and I differ is that I am using precise language to describe the states you mention, while you are just singing in the rain.


\hfill

\texttt{Michel on 2014-04-04 at 02:07 said: }

This is what I understand by Personality:

”

The personality, let us insist once more, belongs essentially to the order of principles in the strictest sense of the word, that is to the universal order; it cannot therefore be considered from any point of view except from that of pure metaphysics, which has precisely the Universal for its domain.

”

“If a person loses his will, then he is no longer a person.”

So the defining aspect of a person is his will. Well, as I understood it, the Personality/Self/Atma is:

”

Is the Transcendent and Permanent Principle of which the manifested being, the human being for example is only a transient and contingent modification, a modification which, moreover, can in no way affect the principle…

The Self as such is never individualized and cannot become so, for since it must always be considered in the aspect of the eternity and immutability which are the necessary attributes of Pure Being, it is obviously not susceptible of any particularization, which would cause it to be `other than itself'. Immutable in its own nature, it merely develops the indefinite possibilities which it contains in itself from a relative potency to act through an indefinite series of degrees. Its essential permanence is not thereby affected, precisely because this process is only relative

and because this development is strictly speaking, not a development at all, except when looked at from the point of view of manifestation, outside of which there can be no question of succession, but only a perfect simultaneity, so that even what is virtual in one aspect is found to be realized in `the eternal present'

”

[Man and his becoming according to the Vedanta, Rene Guenon]

Thank you for clarifying that you were not referring to the individual, but to the Supreme Identity, when you said “personal will”. (That was all I was asking really, the other bulk of the comment was developed under the assumption that you meant one's individuality).

By will, I understand: 

“Immutable in its own nature, it merely develops the indefinite possibilities which it contains in itself from a relative potency to act through an indefinite series of degrees”.

Hence realizing the Self would imply realizing Will; and because the Self is

“the Transcendent and Permanent Principle of which the manifested being, the human being for example is only a transient and contingent modification”, then nothing is lost because all would be: “found to be realized in `the eternal present'” in concordance with the Will of the Self.


\hfill

\texttt{August on 2014-04-04 at 20:17 said: }

” If God's will happens anyway, then why do we pray “Thy will be done, on earth as it is in heaven”? ”

A better way to penetrate the issue: Do you propose that there is a single event anywhere in the Universe that occurs against God's Plan?

The quoted question invites confusion between the `personal God’ (demiurge) and the Absolute (sum of all disorders); the aspiring student of metaphysics should take pains to differentiate the two.

Further, even in a prayer, that statement can be read as an assertion rather than a plea.


\hfill

\texttt{Cologero on 2014-04-06 at 16:08 said: }

August, your response invites confusion between God's Will and God's Plan, which presumably you claim to be privy to.

Aspiring students of metaphysics should take pains to actually read a text, learn to think organically, and avoid abstract, mechanical thinking.

For example, to understand the points raised, he may want to read what Rene Guenon wrote about Providence, Will, and Destiny in the Great Triad. Providence is the reflection of God's Will … we have touched on this topic on many occasions.

The aspiring student may also want to read \textit{Freedom and Power}\footnote{\url{http://www.meditationsonthetarot.com/freedom-and-power}}. Then he could find the post where we quote Julius Evola that the way to know God is to be god. That is, theosis. In organic thinking, there are lot of pieces that need to be tied together, and I mean in one's interiority, not in some artificial metaphysical formulation.

So the best way to know God's will is to actually do God's will … then you will be a man who knows and you will have some authority.

Now, more advanced students may go back a week or two to the discussion of Shankara. If the world we experience is the result of ignorance, or avidya, then that cannot be God's will. Ignorance is not real, but is just a privation of knowledge.

Do you see how things ramify for the student of metaphysics? He needs to balance all the ideas in his head and make them a living realtiy, not an intellectual showcase.


\hfill

\texttt{Synodius on 2014-04-06 at 19:07 said: }

`If the world we experience is the result of ignorance, or avidya, then that cannot be God's will. Ignorance is not real, but is just a privation of knowledge.'

Our usual experience of the world is the result of ignorance (avidya) and so we cannot see it as God's will. With real knowledge we can see the world as it really is, which involves seeing it now comparatively less real (with different metaphysical status) than we did while in the state of ignorance. 

Can it be just different terminology and stress for different points of view or really a fundamental flaw in eastern views?


\hfill

\texttt{Cologero on 2014-04-06 at 23:04 said: }

Synodius, I assume you are making the point that their is a higher principle uniting Providence, Will, and Destiny. But, before getting to that, it does not change the fundamental point that Providence is the instrument of God's will. As above, so below: these three terms correspond in man to the intellectual, psychic, and instinctual forces, i.e., spirit, soul, body (see Guenon). So from below, the man who lives through his spirit, or intellect, corresponds to living under Providence (the subtle rules the dense).

So the uniting principle is knowledge of the Self. From below, the “I” unites the three forces. From above, God is the unifying principle. Nevertheless, Providence is the instrument of his Will, just as for a (true) man, the intellect is the instrument of his personal will. If you want to insist that whatever happens is God's will then you explain and understand nothing. It is the night where all cows are black.


\hfill

\texttt{August on 2014-04-07 at 07:31 said: }

Focussing on Providence, do we not again descend from the Absolute to the manifested? I challenge any participant to differentiate the Divine Will and the Divine Plan for us sub specie aeternitatis.

It is sad that the probing of metaphysical certainty seems to you mechanical.

And what alternative to believing that everything happens only by the Will of God? The Poor man confirms his faith, so be gracious and expose the flaw better if you see it.

James 4, 12-15, KJV:

“There is one lawgiver, who is able to save and to destroy: who art thou that judgest another?

Go to now, ye that say, To day or to morrow we will go into such a city, and continue there a year, and buy and sell, and get gain:

Whereas ye know not what shall be on the morrow. For what is your life? It is even a vapour, that appeareth for a little time, and then vanisheth away.

For that ye ought to say, If the Lord will, we shall live, and do this, or that.”


\hfill

\texttt{scardanelli on 2014-04-07 at 11:18 said: }

Sad also are missed opportunities for understanding. The light shines in the darkness, and the darkness comprehends it not. To truly probe metaphysics, to comprehend the light, keep thy mind in darkness. As long as we eat from the tree whose fruit is death, we will only play at knowing…we will only be pretenders chasing after imaginary forms and phantoms.


\hfill


\end{sffamily}\end{footnotesize}
