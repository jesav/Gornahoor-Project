\section{Does Death Exist}

Another man who followed the \textbf{road of excess} was \textbf{Miguel Serrano}. The same man, who believed in flying saucers and secret Nazi bases in the Antarctic, could also write with great depth and sensitivity about Tantric love and his meetings with sages of the East and the West. A modern day rishi, Serrano started having visions even before adolescence even if sometimes what he tried to make out in the darkness may seem dubious in the light of day. Humble in his interviews with the great yogis of India, he nevertheless often added a significant insight of his own, as we saw at the end of his interview with Krishnamurti\footnote{\url{https://gornahoor.net/?p=2457}}.

In \textit{The Serpent of Paradise}, Serrano paid a visit to \textbf{Krishna Menon} who had followers in South America and was described by Serrano as a ``pure Vedantist". Serrano asked him this question, which every man has asked, or should have asked, himself at some point in his life.

\begin{quotex}
Do you believe in existence after death? Do the dead continue to live in some form, and can we meet them?

\end{quotex}
The sage answered him thus:

\begin{quotex}
Vedanta tells us that life is an illusion and that the individual ego is also an illusion. How then can death exist, if life does not exist?

\end{quotex}
Serrano then mentioned his visions of the inner world and his experiences with astral travel. Menon attributed them to Serrano's having practiced yoga in a previous life. Serrano noticed a curious inconsistency of this response. In his own words:

\begin{quotex}
I have often found it curious that Vedantist philosophers and saints, who are otherwise absolute monists, always have a strong belief in reincarnation. For them, everything is illusion, but reincarnation and the individual karma persist, apparently apart from illusion. Thus I am told that I am suffering from the consequences of having practiced yoga in a previous life. Yet the only one who can activate jivanmukti and overcome illusion is a yogi. 

\end{quotex}
Serrano then pointed out that Swami Menon himself had to deal with karma, for he suffered from chronic leg pain and underwent ayurvedic medical treatments.

Krishna Menon is now dead.



\flrightit{Posted on 2011-08-24 by Cologero }

\begin{center}* * *\end{center}

\begin{footnotesize}\begin{sffamily}



\texttt{Logres on 2011-08-25 at 01:32 said: }

That's interesting. Part of the reason I haven't given up Christianity is because it seems willing to ``get its hands dirty" by talking about materiality (albeit in a spiritual sense or context). If Life is an illusion, can the illusion also be an illusion? Or even thinking one sees through it? And at some point, does it matter? If one's ``heart" is in the right place? I am not sure.


\hfill

\texttt{Caleb C on 2011-08-25 at 01:44 said: }

I don't agree with traditional Buddhism on this, but if I understand Evola's interpretation, reincarnation and karma both exist, but they aren't personal; like a candle from one flame lighting another flame; an essence is passed on, but it is still another flame. 

The soul becomes suffused with the karma of it's current incarnation, but at death diffuses back toward the source, like a raincloud emptying into the sea. The soul that is inherited by a new being is highly unlikely to be the same as a previous soul, but rather a new mixture, partially determined by what the blood attracts, and also the resonance of form and action. Saying someone did something in a past life in this case is a shorthand for saying that they inherited an influential piece with that attribute.

Whether one accepts reincarnation without a persistent individual depends we believe the soul maintains its coherence. However, if we don't accept the existence of time it's pretty hard to argue that the soul would decohere.


\hfill

\texttt{Charlotte Cowell on 2011-08-27 at 14:39 said: }

That's a very astute observation from Caleb with respect to Time – or `time'. as the case may be. My own view on Reincarnation / `reincarnation' is that as we attain consciousness it is possible to discern both the history of the cosmos and the history of one's own soul within the self during its present life, which is of course bound by time in this dimension. There are other dimensions where time-space are meaningless, where we are free of that continuum. Ultimately the soul on the path of return would aim to free itself from the weight of karma, to be released from this history, which may be akin to a shadow of great density that keeps it from rising. This is a task so difficult that many probably try to avoid it, albeit unconsciously, by directing their endeavours towards liberating the higher (pure) self by severing it from its shadow – like a Fool deciding to dump his sack. This inevitably leads to a dualistic impasse, whereas the greater challenge consists in transforming the dark leaden shadow (something which has power in an individual sense) into the gold that will allow the unified being to rise, as if along sunbeams, to the deathless source of all life and eternal bliss, where there is no death, just timeless beauty, endless love, unadulterated ecstasy and all other good things. Our `karmic relations'. are invaluable in helping us to understand how to accomplish this task. To master time travel here is another key to accomplishment. It should be visualised as a spiral rather than a line or a circle, with epochs stacked upon each other like layers of an onion, thus from a higher perspective the distance between them is removed; past lives are like vibratory echoes from the deeper layers and our consciousness of them comes from occupying a similar point in space. Thus Caleb's point about the `new mixture' is also accurate. There is always a way out of the circle that appears to be closed, it's only a matter of adjusting one's perception – or `waking up' or turning around. There is a very simply analogy I have for describing the state of unenlightened souls. It is as if their essential selves are lying face down in dark earth and therefore cannot help but believe there is nothing to see and nowhere to go. Thus can an individual be `face down' for eternities – millions of years – until somethin gives it the impetus to turn around in its shell (the physical body) and open its eyes onto the `new world' that suddenly appears when the lights `come on'. Of course the lights were always on but the soul had its back turned and did not realise. Life here is real (another very difficult concept to fully grasp for a soul occupied fully with the great work and path of return, where spiritual essences seem more real than earthly ones and there is a clearly perceivable distinction between those realms. Because of this confusion – caused by Maya, which in itself is an instrument of the Creator enabling the function of free will – many souls do not fully incarnate, but as soon as the become conscious attempt to return at once to their source rather than take the whole rocky road down and risk – so they fear – never getting back up again.) and so is death. Resurrection is a greater reality than both of these.I do not think it is necessary to remember past lives in order to be enlightened and, I should also point out, on the other hand there are those souls so great that their karma is equivalent to a pair of wings rather than a sack on the back….


\hfill

\texttt{Sumanta on 2012-02-11 at 13:25 said: }

And how does the individually manifested self come to realize the unity of existence? By undertaking a journey, which features it taking on a succession of material bodies as it progresses on its path of realization. The concept of reincarnation is thus compatible with the concept of absolute monism.


\hfill

\texttt{Charlotte Cowell on 2012-02-12 at 08:43 said: }

i see reincarnation in a less `usual' way but then I am not a monist, I distinguish between Creator and Created. In the last life (ie, this one) one can/will `reincarnate' a number of times via a process that is probably seen as synonymous with `initiation'. which as we have seen is about beginnings. This necessarily entails `going back' to the primordial origin (the Adam/Eve) and `returning' from this position, expanding the consciousness to the extent that one comprehends the soul as mirror of the cosmos, and in particular the course of human history. This is something we've spoken of before, the past is ever present, every man is every man and every woman is every woman, with all the opportunities and drawbacks this entails. We carry the sins of our brothers and sisters as well as our own, but equally we can share in their joys and loves. In this we are as one, but not absolute, the mystery remains, God is greater.


\hfill

\texttt{Francisco on 2015-03-28 at 00:36 said: }

Have been at a snails pace, looking back through what I have of Serrano in print, or pdf. When first encountering him, about 15 years ago, he helped to support some confirmation bias regarding zoological racism. Then, I drifted apart from his thought, as it seemed just downright kooky–with ``UFOs" and related matters. Both estimations were in error. Upon approaching him again, my tendency is to see so very much of his writing as a great, immensely clever, and over-the-top, deliberate Hermetic ``mask", intricately woven from two fabrics: some threads directly out of the intact traditional world, others from the side of the dissolution. He appears to daringly take, for instance, discarded and detested ``forms"–such as the politician-man Hitler, and use the ``form" in a way that ``throws off" both literal ``rightists", and at the same time, those who would object, or reject, on liberal grounds. 

Hermetists always seem to miss something in their fundamental symbolism. The ``Ouroborus" never ceases moving while consuming its own tail–thus, what ``appears" to be ``Hermetic" at some time and place, is already expressed differently by the time it reaches the observer. It is never made ``fixed" until fixed within the individual being. Then there is Hermes himself, ``trickster", always changing the game rules–or maybe. There can not be equivalent ``initiations" for any individuals, Geunon states, citing the rule of identity; and with Hermetism, the mask must change frequently, using even some of the most bizarre and outlandish means of expression–Serrano it seems, was one of a very different ``race" living in modern times, as a ``witness", to have worn a most unusual, and nearly impervious mask.


\hfill

\texttt{Michael on 2023-08-24 at 11:02 said: }

Not falling under causation: how could this make the monk a fox?

Not ignoring causation: how could this make the old man emancipated?

If you come to understand this, you will realize how old Hyakujõ would have

enjoyed five hundred rebirths as a fox.

Not falling, not ignoring:

Two faces of one die.

Not ignoring, not falling:

A thousand errors, a million mistakes


\end{sffamily}\end{footnotesize}
