\section{Thought, Action, and the Elite}

In \textit{Perspectives on Initiation}, \textbf{Rene Guenon} deals with the notion of the elite in our time. In the kali yuga, there are fewer initiates and, apparently, it requires a larger population to generate the number of men with the possibility of being initiates. Before reviewing that in order to determine who may be a candidate for initiation, it may be wise to review the fundamentals. In our fall from the primordial state, we enter the world of duality. In Thought, that shows up as a yes and a no. Every opinion expressed engenders its opposite which then leads to infinite discussion without resolution. The Will enters a debate between ``desire" and ``avoidance". With nothing transcendent to resolve these dualities, thoughts and actions are formed on the basis of sympathy and antipathy, what I like and what I dislike.

\paragraph{World of Thought}
First of all, it is necessary to understand the ``person". The person, or ``Self", is beyond all manifestation, a fortiori, beyond thoughts, desires, like, and dislikes. It can never be an object in the world, hence never an object of experience, either of my own or of someone else's. This is unnatural to most people since they regard their person precisely as the contents of their consciousness rather than always outside of consciousness. Hence, there is a widespread misconception anent the notion of a personal God. Projecting their own misunderstandings of personhood, they presume that refers to some super-being who also experiences thoughts, desires, emotions, and so on.

What people consider their person or self is actually its projection into manifestation, and Guenon calls this the ego. That becomes the false center around which people live out their lives, rather than their true transcendent center. That ego is unstable, forever shifting with each new idea, desire, or emotion.

So, although it seems that metaphysical principles are expressed in thoughts and words, that is not how they must be ultimately known. As every metaphysician has taught, there is a higher faculty, ``intellectual intuition", that transcends thought. As such, metaphysics can never be just another theory about the world, something to latch onto or not, a school of thought that one can say either yes or no to.

Words like ``intuition" and ``transcend" refer to very real things, they are not meant to obfuscate, and they are not reserved to some erudites who have no familiarity with the ``real" world. Quite the opposite. Hence, the metaphysician doesn't bother to debate these principles although he may try to explain them. It is pointless to enter a dispute, because all that amounts to is a no to his yes, a project without ultimate resolution. The most you can do is regard him as deluded.

Through certain practices one learns to ``transcend". For the ordinary consciousness, thinking and the ego are on the same level. Men assume they ``think", i.e., there is a thinker and a thought and together they define the man. That is the dualist view. The higher view is different. The person, not the ego, is at the center, and the realm of thoughts is just one of the layers of his being. From the vantage point of the Self, the thoughts are ``below", i.e., they are transcended. Hence one's consciousness is superior to the yes and no of the stream of thoughts. That is non-dual awareness, since there is no longer a thinker and his thoughts, instead just the observer.

\paragraph{Field of Action}
Action takes place in manifestation. Whereas metaphysical principles are known with absolute certainty, that is not the case when acting. Here one's actions are fraught with risk and uncertainty; instead of known principles, what is required is prudential judgment. Curiously, the bulk of men see things quite differently from that. For them, thought can be no more than opinion, or a likely story; a man is free to adopt whatever philosophical, scientific, or religious system he likes based on his ``personal equation".

Yet when it comes to making concrete decisions, he is suddenly filled with total certainty, that he knows what to do and everyone else is going down the wrong path. That is the rash man. His counterpart is the scrupulous man how will weigh his options interminably and may fail to make a decision at all.

Obviously, to the extent he doesn't understand his own nature and the workings of the world, his actions will be ineffective. Here his guide must be the virtues: prudence, to be able to evaluate the possibilities of manifestation in a concrete situation; courage, or fortitude, to have the strength of will to act effectively; justice, to be able to act objectively and fairly. Now, as we may suspect from some of Plato's dialogues, these virtues cannot be understood dualistically; the attempts to define them by rational thought ends in aporia. That does not mean they don't exist. Rather it means they cannot be reduced to thought, but must be the expression of one's Self. The courageous man knows directly, intuitively, and non-dually what courage is, even if he can never express it fully in words.

Man's dual nature is expressed in the gnomic will, i.e., he deliberates the various courses of action open to him. His true will, on the contrary, is the expression of his being; this is non-dual action, arising from a man's primordial state. Purity of heart is to will one thing.

It should be clear that watchfulness is the initial step, the necessary skill to master. All spiritual schools recognize this. It is hardly unknown in the West. For example, it is present in the Philokalia, Scuopli's \textit{Spiritual Warfare}, and even the Gospels. Through constant practice in watchfulness, and unceasing efforts to remove the myriad obstacles that would prevent such, a man can gradually reach non-dual awareness to a greater of lesser extent. There is no point debating non-duality, except in the context of mutual support a spiritual group. Through such practice, one learns to see there are not-two, yet neither term annihilates the other. The ``Supreme Identity" will explain itself.

This sounds simple enough, but in practice I've found that some people have great difficulty grasping this. For them, presumably, the esoteric spiritual life is not an option. They are content with the illusion of ordinary life and there is a necessity for that.

\paragraph{The Notion of an Elite}
Guenon claims that the notion of an elite no longer exists in the West, so that its ``reconstitution is the first and essential condition for an intellectual rectification and a traditional restoration." It is the men who accept that who are the objects of our interest, not those who reject the spiritual in favor of some at-best ephemeral political victories. Obviously, this word has a specific meaning, which is quite different from its common usage. It has nothing to do with a small group exploiting the masses, since its goal is the restoration for all. Guenon makes this definition:

\begin{quotex}
[the elite] represents the totality of those who possess the qualifications required for initiation, and who naturally are always a minority among men. 

\end{quotex}
The dilemma is this: is there no elite in the West because there are no initiatory organizations or are there no initiatory ogranizations because there is no elite? This chicken or egg problem is not resolvable on the rational plane. But it does lead logically to this: does the West have to wait for initiatory organizations in order to form an elite or will the presence of an elite lead to the formation of initiatory organizations?

These are dualistic questions that can only be resolved from a higher perspective. Of course, as things stand, the elite in the West is only virtual and those who believe they are chosen will need to become conscious of their own qualification to be initiates. Consider: if you understand anything of what was written tonight, if you are obsessed with spiritual and metaphysical topics, if the glamor of the world is losing its grip on you, if you have the energy and perseverance to follow an ascetic path, then perhaps you are becoming conscious of your calling. If you know who you are, you will know this to be true.



\flrightit{Posted on 2013-06-19 by Cologero }

\begin{center}* * *\end{center}

\begin{footnotesize}\begin{sffamily}



\texttt{Ash on 2013-06-19 at 23:59 said: }

``The most you can do is regard him as deluded."

This is exactly the problem I struggle with when thinking about Gnosis (perhaps one day I will be able to receive it). True gnosis is indeed beyond questioning. But in terms of finding the authentic guide and teacher, particularly in real life and beyond books and the internet, it is difficult to see between those who really have it and those who don't sometimes. Like asking a dog to explain the difference between red and green. A key, I suppose, would be that the one with Gnosis isn't going around making claims to it. Christ taught in dark sayings, after all.


\hfill

\texttt{Cologero on 2013-06-20 at 00:07 said: }

Dogs are color blind. I don't think you can explain the difference between red and green.


\hfill

\texttt{Scardanelli on 2013-06-20 at 10:02 said: }

Thank you for this post Cologero. I am grateful for your advice catered to those wishing for practical advice along the way. The petty squabbling in the comments section is growing tiresome and it's nice to get back to the work at hand.


\hfill

\texttt{Jason-Adam on 2013-06-20 at 11:55 said: }

I am one who feels called, I despise the world and am obsessed with the idea of being able to see and know God, I am ready to do whatever is required of me to get there, which for me is on faith as I do not know what I must do, but I dont even know where to begin, I read a lot of anecdotes from people who have crossed the threshold but what I seek is a map, a guide, someone or something that can tell me what to do to reach the true world. In the abscence of a master, I've been looking at some practical books and next week I'm going to start perform the excersises from Mouni Sadhu's book Concentration and from there maybe go on to his Meditation book. I've also been thinking about using Franz Bardon's Initiation into Hermetics. If anyone here can advise m I'd be most grateful indeed.


\hfill

\texttt{scardanelli on 2013-06-20 at 12:19 said: }

@Jason-Adam

I have not found a clearer exposition or ``roadmap" than that found in Boris Mouravieff's Gnosis series. All of Cologero's above summation can be found therein plus much more. I have read Mouni Sadhu's Concentration and performed the exercises, but stopped them after reading Mouravieff. I am no authority, but it seems that MS gives you piecemeal information whereas BM gives you the whole, step by step. If you are a man that likes to see the bigger picture, that likes to see where he is going though he may not be able to reach it yet, I would recommend BM.

Also, have you ever looked into any lay organizations? I came across the Militia Templi a few days ago, and from your comments, it seems like it might be a good fit for you. I am not advocating them as I know very little about them, but perhaps something you can look into.


\hfill

\texttt{h.ontologia on 2013-06-20 at 12:20 said: }

Jason-A., have you had a look at a meisterwerk, featured on Gornahoor, titled ``Meditations on the Tarot"?


\hfill

\texttt{Constantine Aetos on 2013-06-20 at 13:04 said: }

Very good article Cologero 

Jason I recommend you read Christ the Eternal Tao by Hieromonk Damascene


\hfill

\texttt{Jacob on 2013-06-20 at 13:18 said: }

This is the diamond in the rough of the Internet. Thanks for another piece. 

I have wanted to buy the Gnosis books, but I have always assumed Gurdjieff was anti-traditional like Steiner and Bessant (mainly because of Guenon's plague comment). This place seems to have a better opinion of him though. Do I need to have an understanding of Gurdjieff before reading Gnosis? Or can he be read without it?


\hfill

\texttt{Jason-Adam on 2013-06-20 at 17:27 said: }

I've read Meditations on the Tarot and admire that work truly but it doesn't offer, at least to me, an actual system of works that can be done to raise spiritual awareness. Mouravieff troubles me because of the anti-western statements he made in Gnosis III reminded me a bit too much of Dugin but I will reconsider that. I've thought about looking into the Gurdjieff work as well.


\hfill

\texttt{Ash on 2013-06-20 at 20:03 said: }

@Cologero

Yes, hence the metaphor. I can judge doctrines, perhaps, but if a doctrine appears legitimate then judging whether or not the teacher is truly speaking from gnosis would be asking someone to explain something he has never experienced himself.

@Constantine

I was able to acquire this book some time ago but haven't yet had the chance to read it. Did you find it helpful? Also, do you know what Hieromonk Damascene's relation to the perennial philosophy is?


\hfill

\texttt{Constantine Aetos on 2013-06-21 at 01:34 said: }

@Ash

I actually just finished reading it last week and I can tell you it definitely has given me not only a better understanding of God, but also how to reach to God (e.g.: Watchfulness \& Jesus Prayer). 

I don't know much about Fr. Damascene other than his spiritual father was Fr. Seraphim (Rose). My guess is he probably had some knowledge of perennial philosophy. 

\url{http://orthodoxwiki.org/Damascene\_\%28Christensen\%29}


\hfill

\texttt{Logres on 2013-06-21 at 21:27 said: }

Jacob, Mouravieff is probably more accessible than Gurdjieff – G wrote for a very specific time, and was unorthodox in many ways, even by esoteric standards. Ousepensky's The Fourth Way is helpful, but much more so if you read Gnosis first.


\hfill

\texttt{Cologero on 2013-06-22 at 10:01 said: }

The point about Gnosis is its claim to be Orthodox, that is, it is actually the teaching from the monks of Mt Athos. The publisher, Robin Amis, backs that up as he claims to have spent time on Mt Athos. You can believe him or not; the web site is Praxis Institute\footnote{\url{http://www.praxisresearch.net/home.cfm}}. I haven't found any independent confirmation.

The Gurdjieff groups are opposed to Mouravieff, regarding him as a plagiarist. However, since Gurdjieff himself confesses to having ``stolen" his system, it is legitimate, and even advisable, to locate the actual ``orthodox" (in its traditional sense) tradition rather than follow G's fragment of a tradition. In point of fact, after Ouspensky's death, some of his followers went in search of the real tradition behind the teaching. Some became Orthodox Christians and were instrumental in getting the Philokalia translated into English.


\hfill

\texttt{Cologero on 2013-06-22 at 10:04 said: }

I need to add that Gnosis is incomprehensible without direct practice of the spiritual techniques that would make it clear. Also, keep in mind that Mouravieff uses the word ``personality" in a way opposite to its definition here.


\end{sffamily}\end{footnotesize}
