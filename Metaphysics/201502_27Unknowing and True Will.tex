\section{Unknowing and True Will}

\begin{quotex}
First and foremost, I will tell you who should work in this work, and when, and by what means: and what discretion you shall have in it. If you ask me who shall work thus, I answer you—all who have forsaken the world in a \textbf{true will}. \flright{\emph{The Cloud of Unknowing}}

\end{quotex}
\paragraph{Going through the Ruins}
To reconstruct Tradition in the West, we need build on the ruins left behind. Despite a mini revival of Thomism, as useful as that it, it does not go much further than skirmishes in academic debates. Rather, we must take it seriously, so seriously in fact, that it leads to an intellectual conversion. That conversion is a ``forsaking" of the modern world for the real world. In other words, it begins with an ``unknowing" and ends in the contemplative life which is true will.

Here are some basic notions that cry out for further development:

\begin{itemize}
\item the distinction between essence and existence 
\item the soul life of man 
\item the intellect as the form of the body 
\item human nature 
\end{itemize}
Those are recited and then neglected. Now the Summa was intended as a beginner's book so it cannot, as such, contain the whole of the teaching. The logical consequences of each notion must be teased out.

\paragraph{Essence, Existence, and Freedom}
Every substance is the result of the union between an essence and existence, or, form and matter. What is unclear in that is how an essence becomes a substance. Science, the dominant ideology of today, only needs to deal with substances and their interrelations. The capability to ``see" the essence in the thing has been lost. The intellectual conversion will therefore reveal that the world of things is but the reflection of a transcendent, perfect order.

But back to the problem. Kant determined that we cannot know the real world through pure reason since our experience of it is filtered through the categories of the mind. Only practical reason uncovers the truth about God, freedom, and immortality. This led those who followed, especially Schopenhauer and Fichte, to conclude that it is the Will that is fundamental, since that is what we know directly, ``from the inside" as it were. \textbf{Julius Evola}, drawing on that line of thought as well as the energy (``Shakti") of the Tantras to make his own contribution: The world, then, is my Will. (from the \emph{Individual and the Becoming of the World}). This is justified on the grounds that

\begin{quotex}
the concept can account for what constitutes the ``essence" of a thing, or the totality of the characteristics that logically define it, but it is impotent to deduce – and even less to produce – its ``existence", the naked fact of its ``being there" (\emph{dasein}) as a real thing. 

\end{quotex}
In Guenon's terms, the possibilities of manifestation or of non-manifestation are logically the same. Hence, only a Will can bring an essence into manifestation. The extent to which the manifestation does not match the essence is called ``privation", the failure to fully manifest. In other words, the amount of ``being" a thing has is related to its power and esoterically, power is freedom. (from Gnosis 1, IX, (7))

\paragraph{Essence, Accident, and Human Nature}
The popular view is that humans share a common nature. That nature, then, is the essence of man and all other qualities are ``accidents", i.e., they do not change one's essential nature. In Aristotelean terms, that nature is rational. But is that sufficient? Does it represent the ``totality of characteristics" that define a man or woman?

Take, for example, one's sex. In today's belief system, sex is an accidental quality. Given that, same sex marriage and transgenderism make perfectly good sense. The Church, however, tries to hold out against the modern world. For example, a transgendered person can go to the political authorities and have documents such as birth certificates and drivers licenses changed to reflect the new sex. However, baptismal records cannot be changed. Another example is the commitment to the male priesthood, something that is unjustifiable is sex is an accidental quality.

Hence, one's sex must be part of one's essence, but Thomism as currently understood has no place for that. The same goes for one's race, ethnicity, and so on: are these accidental or essential? Of course, a man is more that an abstract essence, he is also a person, an ``I".

\paragraph{Esoteric Evolution}
The world process takes into account both involution and evolution. Involution is ontologically prior. Beings fall into different states of existence along a chain of being from the higher to the lower. The inverse of this is the apparent evolution from the lower to the higher, which is really the manifestation in time of involution.

In esoterism, we seek regeneration through the reversal of involution through evolution. In our human situation, we are born into a ``matrix" that precedes us while keeping us within its bounds. If you recall, \textbf{Joseph Ratzinger} put it this way:

\begin{quotex}
The seat of original sin is to be sought precisely in this collective net [or matrix] that precedes the individual existence as a sort of spiritual datum, not in any biological legacy passed on between otherwise utterly separated individuals. 

\end{quotex}
That is the human-created reality, the justification for Seth's claim that we ``create our own reality". Battered by the vast number of conflicting opinions, the first step is an ``unknowing" of that false reality. Gradually one develops a ``Real I" that is transcendent to the matrix. To the extent we are freed from false opinions, we become a unified and integral being. As we become more conscious of the forces that operate through us, we develop a higher intellect that is open to revelation free from the distortions of the collective net.

The next Thomist principle is that the Will follows the Intellect. Hence, the next stage in esoteric evolution is the True Will. Note that the Will is beyond even consciousness. From the Intellect's perspective, this is ``dark and irrational", but freedom, to be truly free, cannot be determined by anything else, not even by God. For the time being, I'll refer you to \textit{The Teaching about the Ungrund and Freedom}\footnote{\url{http://www.berdyaev.com/berdiaev/berd_lib/1930_349.html}}, an essay by \textbf{Nicolai Berdyaev} on \textbf{Jacob Boehme}.

\paragraph{The Subtle Rules the Dense}
Involution then is the reversal of evolution. First there is the Will, then Consciousness, then the Person or I, and finally manifestation. Now it is obvious that we are not manifesting the True Will exactly since we are born into the collective net. That is original sin and represents a privation of what we truly are or could be.

This is the true understanding of ``creating our own reality", as the Will seeks to manifest itself in the various stages. We don't remember doing this, and on the occasions that we get a flash of remembrance, many people assume it to be a proof or ``reincarnation". A good exercise is to try to recall how much of our own lives we have chosen or created. At night, go through the events of the day and see how much you may have contributed, even unconsciously and unwittingly, to what happened. Plato describes this in the \textit{Myth of Er}\footnote{\url{http://en.wikipedia.org/wiki/Myth_of_Er}}.

\paragraph{Amor Fati}
This is not incompatible with exoteric teaching insofar as it claims that God creates life. Of course, but God created man with a free will which cannot be interfered with. As a man aligns his own will with God's will, the distinction becomes moot.



\flrightit{Posted on 2015-02-27 by Cologero }

\begin{center}* * *\end{center}

\begin{footnotesize}\begin{sffamily}



\texttt{Sparrow on 2015-02-27 at 23:31 said: }

While I'm not sure I completely understand the doctrine of essences, it does seem to make more sense than the current Cartesian worldview that seems to predominate in spiritualist circles. 

Though I'm not Catholic, I worry that Pope Francis may compromise more and more with the modern world to try and find acceptance for the Church. We may end up seeing transgender ``priests" sometime in the future. He is the metaphorical gatekeeper with the keys to the kingdom after all. We'll have to see whether or not he lets the modern world into the Catholic Church.


\hfill

\texttt{rhondda9 on 2015-02-28 at 09:50 said: }

You have inspired me to dig out my dog-eared and coffee stained copy of Berdyaev's Slavery and Freedom. It was my Bible at university; a place I really did not want to be, but was told I was not a person without a degree by some status seeking relatives. He was my secret guardian against the assaults upon my psyche, although I did not know that at the time.


\hfill

\texttt{Scardanelli on 2015-02-28 at 11:35 said: }

Sparrow,

Many would reply that the spirit of the modern world has already entered the Catholic church. But again, as Cologero has mentioned a few times, we must not confuse the tradition with it's current representatives.

Speaking of the ``transgendered" though… How can we understand the transgendered in light of the above? Would a transgendered person born a man be considered masculine in essence but seeking to be a woman due to an extreme privation or confusion? Or could the opposite be said… This person was born a man due to extreme privation though they are in actuality feminine in essence and thus seek to be a woman? I'm thinking the former is the case, but I'm not sure why necessarily…

I'd be interested to know how Mouravieff's pre-adamic man fits into this concept of spiritual race.


\hfill

\texttt{Matt on 2015-02-28 at 17:06 said: }

Scardanelli,

I suppose it depends on ``where",if that's even a sufficient term in this case, gender resides within the human being. Does it reside within one's bodily corporeality, and therefore fundamentally tied to it, or does it reside within one's soul – one's substantial form (to use Scholastic terminology)? If one accepts the first proposition, then the scenario you mention would be an example of being psychologically broken (that these individuals almost always can't function in their day-to-day life would support that conclusion), and so an interior privation is at issue. But if one accepts the second proposition, then what one would have is maybe something along the lines of a natural/material privation; there are ontological and metaphysical implications from that that I think are relevant to Tradition – and could be problematic. Would a disembodied soul then also have a gender? What about the pure intelligences (usually referred to as the angelic and the demonic)?

If the first propostion is correct, gender/sex wouldn't necessarily be an accidental quality. If the essence of being human is a rational/intelligent animal, or if one prefers, a rational/intelligent embodied being, one's bodily corporeality could make up part of ``what" one is (I didn't put ``who" for a reason which I will get to); so if one's bodily corporeality is, say, male, then part of what you are is male. Now there are biological conditions like AIS where one's genetic chromosome structure is male, but it is not manifested phenotypically, and so one would appear female, but that would seem to be a case of natural/material privation.

All of that deals with the issue of the ``whatness" (essence) of the human being, but there is obviously the question of the ``who"; this is the domain of the person – the ``I", or self. In Thomistic ontology, the person is not synomous with the essence, the whatness of the human being; and so questions of gender/sex I guess would not fall under that domain of the person as such, though these questions are obviously relevant to the human state (which we experience) as previously noted.

I don't if that adequately solves the issues mentioned in your comment and the original above post. I could be missing something (I probably am); but maybe it's at least a start.

Feel free to shoot a reply back.


\hfill

\texttt{obscure on 2015-03-01 at 00:27 said: }

I've always understood the essential equality among men to be the very basis for existential inequality; manifested particular essences being individually distinct from one another and yet all measured against the universal essenc necessitates hierarchy. hierarchy, both in terms of the individual existential subjects and their general orders. So, all is accidental from the highest perspective regarding man and his diversity, but this does not negate the obvious subsidiarity of existence.


\hfill

\texttt{obscure on 2015-03-01 at 00:43 said: }

Another note of interest: From the Darwinian/Malthusian perspective (after all, is not Darwinism merely Malthusianism for the organism?) it is the organism which is more particularized and hyper-specialized which is `well-adapted' whereas from the traditonal perspective it is the man who is most essential and indeed `generally competent' who is superior; hence the etymological origins of the military rank `general' or the term `genius'. to be sure, the founder of a state or a law-maker/jurist is always first and foremost held to be `generally competent' which is what is most necessary for determining universal laws and ordering species to their ends. 

what happens when general competency is utterly eliminated? does classical leadership disappear? to paraphrase some passage of Nietzsche, `the race of the future is a race of bridge-builders and mechanics.'

Of course, it is necessary that no such future shall ever reach a pure fruition of its pseudo-principles.


\hfill

\texttt{Sparrow on 2015-03-01 at 07:17 said: }

Scardanelli;

I have wondered the same, but the impression I got while reading the words of those who are transgender themselves, was that these were very insecure people. I suppose one could argue that this is the result of years of ``persecution," but I've always found people who were confident of what they were to not care about the words of others.

So if these people are truly masculine/feminine, why do they become insecure and outraged when not referred to their preferred pronouns?

I honestly don't think the spirit and soul would somehow make a ``mistake." you also have people claiming to be ``genderfluid," or neither. So working under the assumption that a transgender person has a different essence than their sex, you would have to say that there are people who have changing essences or no essence whatsoever. 

It seems more likely these people have somehow become confused or not matured fully, and are thus stuck in a fantasy about being feminine, masculine, ``fluid," or none of the above.


\hfill

\texttt{ja on 2015-03-01 at 12:31 said: }

Boris Mouravieff mentioned transgender people in Gnosis I (chapter 17 or 18, I don't have my hard copy with me but someone can confirm) , he wrote they couldn't obtain the kingdom of heaven until the body sex and mental sex were brought back to alignment.


\hfill

\texttt{Cologero on 2015-03-01 at 12:58 said: }

Something like that, JA, in Chapter XVIII, section (9). However, the solution, if there be one, is to change one's mentality.


\hfill

\texttt{ja on 2015-03-01 at 14:24 said: }

On the other hand, Evola following Weinenger considered transsexuals to be truly women in men's bodies (see Metaphysics of Sex ), which puts him in accord with modern thinking. Il Barone however saw the trangender phenomenon as proof of the decline of ``real masculinity ``in the kali yuga. I think he'd have argued that trannies should be left to themselves as a group of outcasts like the Indian hijiras (spelling may be off )


\hfill

\texttt{ja on 2015-03-01 at 14:27 said: }

Interesting Cologero, because a big issue in modern debate vis à vis glbt is the issue of ``conversion therapy ``, whether it works or if it's ``abuse ``, I assume you think it can ? I personally have no idea because I never knew anyone with gay or trans inclination who sought to change themselves.


\hfill

\texttt{obscure on 2015-03-01 at 23:58 said: }

I've noticed that transgenderism in men is intimately connected to despair. I mean despair in the rigorous sense. For example, there was the Unabomber who was clearly a man without an objective; until later of course. Now, some people believe Ted was motivated by `sexual frustration'. but the answers were much more blatantly expressed in the man's own words. Consequently, I knew a man who had spent a prolonged period of time steeped in apocalyptic thought-processes of a hopeless sort. This man also turned to sex-reassignment, but realized that he had lost his sense of self and snapped out of it. When men lose the very notion of an objective they emasculate themselves; spiritually and physically.

Most late moderns don't understand this though.


\hfill

\texttt{Mark Citadel on 2015-03-04 at 21:08 said: }

The nature of this malformation known as `transgenderism' is largely an irrelevant curiosity. Such individuals, whether victim of genetic circumstance or their own pursuit of self-destructive ends, do not have a place in Traditional society, at least if they are to affirm these desires which are well and truly disordered and especially harmful to immature members of the society and their own actualization of true virility. Those of aberrant psychology who lack the will to bury it under a mound of earth in order to pursue a good life are destined to live outside the city walls, as exiles.

The point being, if we are to posit an `essence' as such then regardless of whether it fits or does not fit with the body one is gifted with at birth, it remains a duty to act in accordance with Traditional expectations. Hiding this inner aberration is itself may be considered measure of virility.


\hfill

\texttt{obscure on 2015-03-04 at 22:28 said: }

Mark,

Transgenderism has no essence. There is certainly an active and a passive aspect to all things insofar as they are finite, but there are no metaphysical arguments for sexual perversions and gross mutilations. A man shows his feminine or passive aspect in a virtuous and valid manner through simple compassion. Do not accept arguments for anything more than that regarding what is passive in a man. Even those men who are predominately `passive' in comparison with other men have appropriate roles to play in a community which do not entail vulgar and strange behaviors. 

Always disabuse men of their abstruse and incorrect notions if they alter subtle discourse in such a way as to make it gross. Otherwise, men will prevaricate in such a way as to excuse their vices with gibberish. I would not even attribute any behavior (insofar as behavior is irreducibly dynamic) to that certain micro-structural intension of the body which we signify by the term `genetic'. If one goes anywhere near that route then the immoral ones will take up all sorts of mechanical notions in order to justify their behaviors as being pre-determined, unilateral motions.

These diverse and serious ways in which men err are illnesses of the soul. Sober discourse has a certain power to cure them though. No one ought to doubt that lest they begin to err as well.


\hfill

\texttt{Mark Citadel on 2015-03-07 at 00:57 said: }

Yes, this analysis seems apt. It is likely possible to cure all aberrations of self-concept and set somebody straight. Something tells me there may be some theological aid to be found here, and perhaps a man may be cleansed through ritual. Men who express sexual perversity are in fact channeling a part of themselves incorrectly, and even the strongest predispositions to such abhorrances are minute compared to the strength of will.

An element of choice always exists. Even the strongest chemical addiction can be overthrown with sheer will. Those who wallow in depravity will spin sob-stories of their lacking control, but rest-assured they are EXACTLY where they want to be. To tackle the problem, they first have to want the problem solved.


\end{sffamily}\end{footnotesize}
