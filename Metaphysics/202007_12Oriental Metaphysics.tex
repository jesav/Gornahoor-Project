\section{Oriental Metaphysics}

\begin{quotex}
The greatest of all lessons is to know one's self. For if one knows himself, he will know God; and knowing God, he will be made like God, not by wearing gold or long robes, but by well-doing, and by requiring as few things as possible.\footnote{\url{https://www.newadvent.org/fathers/02093.htm}} \flright{\textsc{Clement of Alexandria}, \emph{The Paedagogus}}

The soul is all that it knows. \flright{\textsc{Aristotle}, \emph{On the Soul}}

The conceptions of Aristotle are in complete agreement with those of the East. \flright{\textsc{Rene Guenon}, \emph{Man and his Becoming}}

Dear Unknown Friend, do not scorn mediaeval scholasticism. It is, in truth, as beautiful, as venerable and as inspiring as the great cathedrals that we have inherited from the Middle Ages. To it we owe a number of masterpieces of thought—thought in the light of faith. And, like all true masterpieces, those of mediaeval scholasticism are beneficial. They heal the disorientated, feverous and confused soul. What is at stake with scholasticism is God, the soul, freedom, immortality, salvation, good and evil. Therefore, do not despise mediaeval scholasticism, dear Unknown Friend; it is still of value. \flright{\textsc{Valentin Tomberg}, \emph{Meditations on the Tarot} }

\end{quotex}
In a letter to Guido De Giorgio\footnote{\url{https://www.gornahoor.net/?p=4460}}, Rene Guenon mentioned that he had given a lecture to a study group at the Sorbonne in 1925\footnote{The text of that lecture can be found at Oriental Metaphysics: \url{https://www.gornahoor.net/library/OrientalMetaphysics.pdf .}}. This post is a summary of that lecture. In it, most of Guenon's main themes are covered. The text seems straight forward after a cursory reading, yet many people seem to struggle with the ideas. On the one hand, this may lead to confusion where there is none, and on the other, to widely speculative notions that have no basis in fact, and are, mere diversions.

Keep in mind that Guenon is not an original thinker, and, in his case, that would be a compliment. However, he has provided a valuable task of synthesis, and more importantly, he has demonstrated how metaphysical knowledge, which used to be known, has all but disappeared in the West. That does not mean, however, what many seem to think it means. The teachings are available in plain sight; what is missing, apparently, is how to transform rational knowledge into spiritual realization.

\paragraph{Metaphysics}
In itself, metaphysics is neither Eastern nor Western. Guenon deals with “Oriental” metaphysics because in the West such knowledge

\begin{quotex}
is a thing forgotten, generally ignored, and almost entirely lost, while in the East it still remains the object of effective knowledge. 

\end{quotex}
Guenon mentions specifically Hinduism, Taoism, and Islam where the true metaphysics can be found. In 1925, Guenon concedes that he was most familiar with Hindu metaphysics.

He is more sanguine about the East in 1925 than I am in 2020. China has been irreligious for four generations, and I have yet to meet a Chinese national familiar with Taoism. I don't know about the state of Islamic metaphysics; there are two mosques in town, but they don't advertise esoteric schools to the general public. At that time, Guenon did not consider Buddhism to be a valid tradition, although today it is more missionary than either Taoist or Islamic esoterism. Hinduism, on the other hand, is widely available in the West in the intervening years since Guenon's lecture.

I used to give talks and lead classes at a metaphysical bookstore in Deerfield Beach. There I met many Western students of Hindu doctrine, even gurus from India and their America acolytes. When I was leading a weekly class on Guenon, none of the Western Hindus recognized anything corresponding to their understanding. Usually, they were more interested in reincarnation, and even adopting Indian styles of dress and cuisine. Indian tech workers in the USA are typically more interested in technical subjects and practice their exoteric rites. I asked a group how many had read the Bhagavad Gita; the response was none. Guenon claims that is normative in Hinduism, since individuals absorb as much of the esoteric doctrine that suits their abilities.

So I decided to focus on the recovery of the Western Tradition. This decision is justified because Taoism and Islam are restricted, and there are too many fraudulent Hindu gurus in the West:

\begin{quotex}
these doctrines are reserved for a relatively restricted and closed elite [in Taoism and Islam]. This was also the case in the West in the Middle Ages, in an esotericism comparable in many respects to that of Islam and as purely metaphysical as the Islamic one; of this the moderns, for the most part, do not even suspect the existence. 

\end{quotex}
\paragraph{Western Metaphysics}
The notion of a common metaphysics across traditions was not unknown in the West, for example Medieval Scholasticism drew on:

\begin{itemize}
\item Pagan metaphysics, particularly Aristotle and Plato 
\item Influence of Arabic and Islamic Philosophy on the Latin West\footnote{\url{https://plato.stanford.edu/entries/arabic-islamic-influence/}}, including al-Fārābī, Avicenna, and Averroes 
\item Hebrew metaphysics, at least via Maimonides 
\end{itemize}
Had they known about it, they probably would have included Hindu metaphysics. So Guenon is preaching to the choir here.

\paragraph{Intuition}
Reason is a human feature and thus is fully natural. If physics is the study or the natural world, then metaphysics is the study of the supernatural, i.e., what is beyond the human state. The method of knowing the supernatural is “intuition”, that is, a direct understanding of metaphysical ideas. It is analogous to sensory intuition, but differs since it does not involve the senses. Intuition is beyond Reason, so that bookish knowledge is not at all what Guenon means. Nevertheless, such knowledge use usually necessary as preparation.

\paragraph{Realization}
Therefore, metaphysical doctrines must be “realized”, not just “known”. Specifically, they must be made “real” and to do so requires the transcendence of the merely human state. The indispensable means to such realization is \textbf{Concentration}. Guenon explains:

\begin{quotex}
There is nothing in common between metaphysical realization and the means leading to it, or, if preferred, which prepare for it. This is why, moreover, no means are strictly or absolutely necessary; or at least there is only one indispensable preparation, and that is theoretical knowledge. This, on the other hand, cannot go far without a means which will play the most important and constant part: This means is \emph{concentration}. 

\end{quotex}
Many readers get hung up at this point. Instead of learning concentration, they believe that they just need to read yet another book to explain it. However, what they really need is to learn concentration, which is the very first thing an esoteric school will teach you. Unfortunately, there is no “method” or “recipe” for this, since the means “have to be adapted to the temperament of each individual and to his particular aptitudes and disposition.” Perhaps it will be helpful to describe some particular exercises.

\textbf{Soul Awareness}. Aristotle and the Scholastics taught that plants, animals, and humans each add to the layers of the soul. These correspond to the sheaths or koshas of the Vedanta\footnote{\url{https://www.gornahoor.net/?p=9647}}. One can learn about them simply as a doctrine, using the faculty of Reason, or one can realize it through intellectual intuition.

To do that, one needs to learn to concentrate on the observation of one's inner states: i.e., all the sensations, emotions, fantasies, thoughts, etc. Through these observations, he will see directly the operations and interactions of the vegetative, sensitive, and intellectual souls. Obviously, that is a huge task and will take some time. We can't provide a guidebook here, but ultimately one wants to be able to become the master of one's inner states.

\textbf{The Unmoved Mover}. Guenon refers to Aristotle's notion of the unmoved mover several times in his writings. Specifically, this means the principle of manifestation, which is motion and action, must be actionless. As a concentration exercise, sit still and silently and passively observe whatever appears in consciousness. You should observe random changes. In the midst of them, try to find the “unmoved mover”, that is, what does not change while everything else is changing.

\paragraph{Primordial State}
This realization of the integral individuality is described by all traditions as the restoration of what is called the “primordial state” which is regarded as man's true estate and which moreover escapes some of the limitations characteristic of the ordinary state, notably that of the temporal condition. The person who attains this “primordial state” is still only a human individual and is without effective possession of any supra-individual states; he is nevertheless freed from time and the apparent succession of things is transformed for him into simultaneity; he consciously possesses a faculty which is unknown to the ordinary man and which one might call the “sense of eternity.”

This is of extreme importance, for he who is unable to leave the viewpoint of temporal succession and see everything in simultaneity is incapable of the least conception of the metaphysical order.

\begin{quotex}
Why this appellation of “primordial state”? It is because all traditions, including that of the West (for the Bible says nothing different) are in agreement in teaching that this state was originally normal for humanity, whereas the present state is merely the result of a fall, the effect of a progressive materialization which has occurred in the course of the ages, and throughout the duration of a particular cycle. 

\end{quotex}
There is not much to add to this except for this point. In a lecture by a hand-picked student of a guru, she mentioned the Fall, but also said she did not know the reason for it. On the other hand, the dominant Western Tradition of the past two millennia does provide a deeper understanding, which can even be verified through its effects on consciousness.

\paragraph{Higher States}
The Primordial State is still a human state, so Guenon describes higher states, as summarized:

\begin{itemize}
\item \textbf{Ordinary Human state} = result of a fall 
\item \textbf{Primordial state} = realization of the integral individuality. Freed from time. Sense of eternity. Originally normative for the human 
\item \textbf{Supra-individual but still conditioned}. Includes substates in these categories: 

\begin{itemize}
\item Informal but still pertaining to manifested existence 
\item Universality which is pure being 
\end{itemize}
\item Principle of all manifestation or Deliverance. “In this unconditioned state all other states of being find their place, but they are transformed and released from the special conditions which determined them as particular states.” 
\end{itemize}
Deliverance means being in possession of the fullness of one's own potentialities. It is the state of absolute plenitude. Limiting conditions, or privations, disappear. The human state is a mix of Act and Potency, or Essence and Existence. As the potential becomes actualised, one gets closer to God, for Whom there is not privation so Essence and Existence coincide.

\paragraph{Phenomena}
These states have nothing to do with woo-woo. That is, these states have nothing to do with preternatural phenomena, visions, healing, powers, out of body experiences, channeling, or other unusual happenings. These are still part of the physical, not metaphysical, order. Such phenomena do not in themselves indicate the states of metaphysical realisation, even should they appear as side-effects.

\paragraph{Source of Teachings}
Tradition has no human origin in time. It was known in the Primordial State but forgotten in the Fall. The Tradition has taken various exterior forms. Augustine recognized this in his claim that there is just one Religion, which is now called the Catholic religion, at least in the West.

Metaphysical knowledge, as well as the realization that will turn it into all that it truly ought to be, is thus possible everywhere and always. This means that initiation by an esoteric group is not strictly speaking necessary. We have already shown that it is not even sufficient.

Metaphysical realization cannot be achieved by human means alone. Rather, it is a gift and one can only prepare to receive and recognize it.



\flrightit{Posted on 2020-07-12 by Cologero }

\begin{center}* * *\end{center}

\begin{footnotesize}\begin{sffamily}



\texttt{Adil on 2020-07-18 at 17:48 said: }

“The Tradition has taken various exterior forms. Augustine recognized this in his claim that there is just one Religion, which is now called the Catholic religion.”

This makes me think of the recent conversion of the Agia Sophia back into a mosque. The West (including Eastern Orthodox churches) have not been late to condemn this move. But in reality, what has changed? Agia Sophia was a mosque for 500 years, until it was turned into a secular museum. Inside it, you can see a blend of Islamic and Christian art. Now, it has once again been turned into a sacred space, instead of a poorly maintained museum piece with an entrance fee.

Don't get me wrong. Ideally, the cathedral would be used as a church, but history is what it is. This wasn't exactly unexpected. At least, somone will now feel responsible for the building, and perhaps not let it fall apart. If the west was serious with its critique, perhaps it could respond by forbidding any mosques in Europe, and turn the existing ones into museums?


\end{sffamily}\end{footnotesize}
