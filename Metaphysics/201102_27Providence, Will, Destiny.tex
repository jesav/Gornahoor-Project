\section{Providence, Will, Destiny}

In \emph{The Great Triad}, Rene Guenon deals with the relationship between Providence, Will, and Destiny. Here he relies on the work of Fabre d'Olivet, which is based on Pythagoreanism and Chinese metaphysics. This triad is related to the “Great Triad” of Heaven, Man, Earth. D'Olivet writes:

\begin{quotex}
That universal Man is a power in his own right is a fact acknowledged by the sacred codes of every nation, realised by every man of wisdom and admitted by every true man of learning … The two other powers that he stands between are Destiny and Providence. Beneath him is \textbf{Destiny}: `natured nature'. nature bound by necessity. Above him is \textbf{Providence}: `naturing nature'. nature in its freedom.

He himself is the mediating, efficient \textbf{Will}, situated between the two natures so as to sere as the link and means of communication between them and combine two actions, two motions, which otherwise would be incompatible.

\end{quotex}
The three powers of Providence, Will, and Destiny account for all action. (Since we are focusing on activity on this plane, in the human state, we will bracket out Guenon's comments regarding transcendent states.) The human Will represents the inward and central element which unites the intellectual, psychic, and instinctive spheres. Since these correspond to spirit, soul and body, Man is the microcosm of the macrocosmic ternary itself (i.e., Providence, Will, Destiny).

In human terms, Man is the mediator between Heaven and Earth. Providence, that is, the Principle or Ideas, do not act directly on Nature, but rather through the intermediary of the Will of Man. Nature, in itself, follows its own laws. So this brings up the question of why, although Earth is the reflection of Heaven, it doesn't appear to be such.

Guenon writes that the Transcendent Man, who is unrecognizable to ordinary man, is the intermediary between Heaven and Earth, Providence and Destiny; that is his action, or influence, incarnates the activity of Heaven. Of course, this influence is actionless (\emph{wei-wu-wei}); the `one and only man' exercises his role as unmoved mover. He controls everything without intervening in anything. It is difficult not to notice the correspondence between Guenon here and Evola\footnote{\url{https://www.gornahoor.net/?p=1783}}. The one-and-only man, or Transcendent Man is the singular I who controls everything, not through force or compulsion, but rather through the activity of his Will; note that this necessarily includes other “subjectivies” which, in any event, are illusory. Guenon's book would have had the title: “The Transcendent Man and the Becoming of the World.

Although Guenon concedes the existence of the lesser mysteries\footnote{\url{https://www.gornahoor.net/?p=1721}}, he writes from the perspective on the greater mysteries which is not always helpful. Hinduism has six orthodox schools, only one of which is the Vedanta. Hence, we are justified in conveying Tradition in terms other than the Vedanta. The great majority of Westerners attracted to Tradition would be of the Kshatriya type, or those on the path of the lesser mysteries. They are not called to the life of an Egyptian peasant; although not immoral, it is certainly not a universal call in the Kantian sense. For those involved in the world, Confucius, and his Western counterparts, are more appropriate than Lao Tzu. Thus we consider it legitimate to refer to “secular philosophers” (as Guenon calls them), Hermetists, and so on.

So Evola did not write Guenon's book, he wrote the \emph{Individual and the Becoming of the World}. As an individual in my current state, I don't experience the World as fully under my control; I experience resistance, or privation. That is what we need to understand. Part of that understanding will be the rejection of false explanations derived from religions, philosophy or science. Instead we see that privation as the area where we need to develop more strength. When we understand that, we will also understand the purpose of those spiritual exercises meant to develop the will. For example, there is the book “Opening the Dragon Gate” about Taoist initiation, or the Fourth Way work.

So to summarize: Man is the intermediary between Heaven and Earth, Providence and Nature; or better, “I” am that intermediary. If this is a task to take seriously — and it can hardly be avoided — it is imperative that I come to understand the Principles (or Ideas, Heaven, Providence) while at the same time developing the power to bring them into manifestation in Nature. We have often emphasized that Tradition is not for the bookish; it requires a change in Being that can only come through efforts. In the Microcosm, “I” am to bring my soul and my body under the dominion of the Will. Then the Macrocosm …



\flrightit{Posted on 2011-02-27 by Cologero }
