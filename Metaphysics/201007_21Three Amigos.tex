\section{Three Amigos}

We recently mentioned Guenon's role in bring awareness of Tradition to the West, but there are two other very important figures: Ananda Kentish Coomaraswamy (AKC) and Julius Evola. Operating on three different continents, they kept in contact with Guenon via post, exchanging letters and even books. That was the precursor of the Internet, in bring together the scattered remnant of Tradition, spiritually and intellectually, if not physically. Both men were accomplished thinkers in their own right, but their respective encounters with Guenon were transformative and heavily influenced their own writings. 

There are some who would exclude Evola as a Traditional writer, but the other two did not see it that way. AKC wrote a favourable review of Julius Evola's \emph{Revolt Against the Modern World}, which was published in the Feb-Apr 1940 edition of \emph{The Visva-Bharati Quarterly} and included an English translation of the chapter ``Man and Woman". Guenon himself reviewed several of Evola's works. 

There are clearly differences in interests and interpretations among the three authors, which Evola would attribute to differences in the ``personal equation". However, these differences could never be over metaphysical principles, but rather differences in the interpretation of contingent historical events. For example, there were disagreements over the Traditional characteristics of Hermeticism or Buddhism, but in the latter case, Guenon was later to alter his opinion. 

AKC was an enormously erudite man and his contributions not just in metaphysics, but also in traditional art, crafts, folklore and mythology are important to anyone not just seeking to understand Tradition abstractly, but also in its potential applications. Similarly, Evola's interests were in the forgotten Traditions of the West, whose characteristics he sought to reconstruct from the analysis of myths, sagas, and legends. He also applied his understanding of Tradition to current political and cultural events. 

Another important difference is that whereas Guenon eventually opted to live within a specific tradition (Sufism), AKC and Evola were those rare men who could live outside tradition and caste.



\flrightit{Posted on 2010-07-21 by Cologero }

\begin{center}* * *\end{center}

\begin{footnotesize}\begin{sffamily}



\texttt{EXIT on 2010-07-21 at 20:47 said: }

How come you never mention Frithjof Schuon?


\hfill

\texttt{Cologero on 2010-07-21 at 21:21 said: }

Perhaps he plays the role of d'Artagnan? Although in correspondence for many years with Guenon, Schuon was younger then the others and his most influential works appeared after the ground had been prepared by the others.

At the moment, Schuon is outside the direct interest of Gornahoor, although his works merit close study. Perhaps at a future date.


\hfill

\texttt{James O'Meara on 2010-10-06 at 15:56 said: }

Speaking of Traditionalists who lived ``outside tradition and caste" reminds me of a strangely neglected Traditionalist, Alain Danielou. By `neglected' I mean discussed, since Inner Traditions keeps most of his books in print, but you seldom see anyone refer to him [outside of his Kama Sutra translation]. 

And yet Danielou is the only Traditionalist outside Guenon who really lived in a Traditional culture, traveling in India with his lover in a silver house trailer for many years, long before he knew of Guenon. Ironically, he points out that as a Westerner he was an out-caste, and therefore no one cared about his `out' lifestyle! Despite this, he immersed himself in Traditional culture, leaning Sanskrit and even becoming a Shaivite initiate. 

His importance: due to this background, when he did discover Guenon's work, he was able to independently verify the truth of Guenon's assertions about Indian thought. 

Another irony: Guenon actually wanted to live in India, and only remained in Cairo because the British, on the eve of the war, wouldn't provide a visa. And Danielou, whose family had ties to the diplomatic corps, tried to get him into French India [whre he could have met Aurobindo!] but to no avail. How different would the whole ``Traditionalist" movement be, if the over-emphasis on Islam as ``the last revelation" had been short-circuited?


\hfill

\texttt{Cologero on 2010-10-07 at 01:02 said: }

\texttt{I'm not sure about that Guenon story. He gives as a reason for selecting Islam over Hinduism what you said: in India he would always be an out-caste, and not part of that Tradition. When I get some time, I'll look for the reference.}

I have posted on the blog how Guenon described his ``conversion", which was not such in any common use of the word. He may have accepted Islam as the last revelation, but not as any ``truer" than other revelations.

BTW, he regarded Aurobindo as not Traditional.


\hfill

\texttt{James O'Meara on 2010-10-07 at 18:25 said: }

Looking at the passage again

\url{http://books.google.com/books?id=GCOewSVrf2sC\&printsec=frontcover\&dq=danielou+labyrinth\&source=bl\&ots=ZzusC8FF9S\&sig=vox1ybEpTW-RAURzsaiZl0fJ4O4\&hl=en\&ei=DUWuTLeIMIT6lwev0anjBA\&sa=X\&oi=book\_result\&ct=result\&resnum=1\&ved=0CBUQ6AEwAA\#v=onepage\&q=guenon\&f=false}

I may have given a misleading impression of it. D. talks about Guenon ``settling" in Egypt, not of his conversion to Islam. Obviously, he could have lived as a Moslem in India as well! Given his rather Indian POV, due to his teachers, and ``transcendent" POV, he may have been initially more attracted to India as a place of residence. D.'s point though is that living in Egypt prevented him from contacting living centers of pre-Aryan tradition, such as his own Shaivism. And my speculation is I think still related to that; how different the ``atmosphere" of Traditionalism [to say nothing of the ``Guenonian Scholasticism" Evola mocked] if he had been in a more ethnically diverse culture. The example of D. himself makes questionable the strong implication from the later Traditionalists that modern, Western man is ``uniquely" suited for Islam as opposed to Hinduism, for example, although even D. admits the traditional world he entered is no more.


\hfill

\texttt{kadambari on 2010-10-07 at 23:12 said: }

``in India he would always be an out-caste, and not part of that Tradition."

Ha ha that's funny and quite ridiculous he felt like that. There was a famous missionary in Himanchal, India, who introduced the famous apples found there, who initially sought to convert the natives, but then just married locally and became a Hindu instead! I think Hinduism is large enough for Guenon not to feel an outcaste, and it seems a personal choice. Also I find it odd intellectually that anyone can embrace Isalm reading that book cover to cover by choice! Guenon would have been better off as a Christian considering all the Sufi mysticism (perhaps the only intellectual component of the religion) is just Aryan Persian mysticism (pre-Islamic i. e. Zoroastrian) creeping into Isalm (which was forcibly imposed upon a mystical peoples like the Persians after Arab conquest) combined with Buddhist/HIndu/ influence. I find not a single original neuron in it. Sufism is every bit Persian and not Islamic. Still I have yet to read Guenon!


\hfill

\texttt{kadambari on 2010-10-07 at 23:29 said: }

The missionary was Graham Stokes, they made him a Brahmin out of respect for him, as he was intelligent. His daughter is active in Himanchal parliament otday, the hill state being almost completely Hindu is one of the most pleasant states left in India. He introduced the golden delicious apples to Himanchal.


\hfill

\texttt{kadambari on 2010-10-08 at 00:09 said: }

Sorry, his first name was Samuel Evans Stoles not Graham Stokes. He is somewhat of a legend in Himanchal, and to be precise the apples were red delicious. Anyway, he was later known as Stayanand Stokes.

Guenon must have had some other motive for converting to Islam, perhaps he was drawn to it as it is a semitic faith much closer to Christianity its brother and is completely at odds with Hinduism. Anyway, after his conversion, I guess he must stuck to it if his conversion was genuine, as apostasy in Islam is a crime in that religion.


\hfill

\texttt{James O'Meara on 2010-10-08 at 09:05 said: }

Kadambari,

It's not a question of ``becoming Hindu" but what role a Westerner could play in traditional society. According to D.

they would be of the lowest caste. D. could learn the Vedas but not recite them [presumably in public]; contrarywise, his Brahmin teachers could not even look at him, and he had to listen from a concealed place. Hard to imagine RG putting up with that, although just the sort of thing he would promote as ``traditional social structure'. perhaps!

D. eventually sank to the level of outcast, not because of his relations with his lover, Raymond, but because Raymond married a Brahmin wife, a theosophist to boot! In the decadent West, of course, it would be the other way around, and Raymond would be socially redeemed by marriage to anyone.

Your remarks on Sufism are very interesting. Despite the happy talk promoted by H. Smith, for example, RG never taught that there was some esoteric ``core" to every ``religion." In China, C. and T. are quite separate traditions, while the Semitic religions [`religion' as such being only in the West] seem to have ``hidden" cores; Islam was sort of mid-way, and he spoke of ``Islamic esotericism" rather than ``Esoteric Islam." 

It's hard to see anything `intellectual' in Islam, or the other Semetic faiths; think of all the nonsense about ``koshering" this or that. Your reference to apostasy as a `crime' for example; social regulation posing as `wisdom.' F. Schuon and his school have been taken to task for misleading Westerners by presenting as ``Islam" what only pertains to Sufism, and a peculiar brand at that. 

I've always thought Sufism was derived from Neoplatonism, via the other Greek wisdom Islam assimilated, but I'm just an amateur. Your Aryan connection through Persia is intriguing; can you recommend anything on that?


\hfill

\texttt{Matt on 2010-10-18 at 14:17 said: }

I won't get too much into this because I realize that the intent of this site is to speak of metaphysics and pure principles, but I do share the sentiments of James and kadambari. I think it can be too easy to just accept every major religion on the basis of a unity at the metaphysical level and not point out the serious problems of a supposed tradition at the contingent level. This can also lead to many in the self described traditionalist movement to not pose the legitimate question of whether features of a religion conforming to Tradition came from within, or were placed on it from the outside (a question that seemed to only be taken into account with seriousness by Evola). And yes, Schuon and his followers are particularly guilty of all this.

I've always found it funny that Evola became such a black sheep to the perennial movement because of certain political associations he had for a period of his life, and yet Schuon is held up as some type of avatar for the movement when he indulged in what seems to be the cardinal sin in the movement, mixing traditional forms together. At least Evola realized the mistake he made in getting into politics, and never thought of himself as some figure destined to create a universal multicultural esoteric church.


\hfill


\end{sffamily}\end{footnotesize}
