\section{Creative and Heroic Action}

I have recently been asked why we are so ``one-sided". Such a thought can arise only in a mind conditioned to think on a single plane, thus interpreting everything polemically, in the manner of a teeter-totter. The Traditional mind, on the other hand, thinks in terms of ``transcendence", which Evola defines as ``surpassing while ascending". The idea of cycles does not mean an impossible return to the ways of antiquity, but rather an absorption of their highest spiritual values in a new synthesis.

In the essay, Rene Guenon, a Teacher for our Times, Evola calls for a ``\emph{Creative and Heroic action}." Evola rejects any turn to the East, which, in any case, is in its own process of degeneration. It is actually unnecessary, since the West has its own Tradition, discernible to those with the ability to see. The result will be a work ``endowed with a metaphysical, transcendent, ethical, and social character." This is the vision and the task taken up by Gornahoor, however unclearly or poorly expressed.

\begin{itemize}
\item \textbf{Metaphysical}, by emphasizing the need for a genuine gnosis as the foundation. 
\item \textbf{Transcendent}, by integrating and surpassing the best of antique pagandom and Christendom. 
\item \textbf{Ethical}, by our emphasis on freedom or the will and understanding the creation of a world as a moral act. 
\item \textbf{Social}, by trying to create a \textbf{Solidarity} of traditional thinkers in Continuity with the ways of our ancestors. 
\end{itemize}
This is a positive task and should be a thing of beauty, suitable for men with a sense of noblesse oblige. If we seem negative, that is because of the necessary task of outing those who claim to be on the side of Tradition, but who, in reality, form a counter-tradition, which is revealed through the ugliness of their discourse, their unaesthetic music, and their rejection of a spiritual Solidarity in the present and Continuity with the past. To be considered a man, they misunderstand the true Pagan as a creature given over to his liberated elan vital, and ``might is right" philosophy. Thus, they demonstrate their lack of understanding of the counter balancing frein vital, the Buddhist ``battle in the cave of dragons", or the man able to dominate and control his unbridled passions.


\hfill

The cave of the blue dragon is ominous.

Only the fearless dare to enter.

It is here that the forest of patterns is clearly revealed.

It is here that the one ripe pearl is hidden.



\flrightit{Posted on 2011-07-25 by Cologero }
