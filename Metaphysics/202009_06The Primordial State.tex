\section{The Primordial State}

\begin{quotex}
The present age must come to terms drastically with the facts as they are, with the absolute opposition that is not only tearing the world asunder politically but has planted a schism in the human heart. We need to find our way back to the original, living spirit which, because of its ambivalence, is also a mediator and uniter of opposites, an idea that preoccupied the alchemists for many centuries. \flright{\textsc{C G Jung}, \emph{Aion}}

\emph{O felix culpa quae talem et tantum meruit habere redemptorem}

“O happy fault that earned for us so great, so glorious a Redeemer.” 

\end{quotex}
Memory of the Primordial State was retained in several traditions. In \emph{Man and his Becoming}, Rene Guenon describes three stages leading to Union: in ascending order, \emph{balya}, \emph{panditya}, and \emph{mauna}.

In Balya, or the Primordial State,

\begin{quotex}
all the powers of the being are concentrated in one point, realizing by their unification an undifferentiated simplicity, comparable to embryonic potentiality. It means the return to the `primordial state'. 

\end{quotex}
In that state, the powers of the being are concentrated in the Self, which is the unifying factor. In the natural state, man is not unified under a unified will. He is uncertain, opinions and desires change from minute to minute, unconscious drives cloud the intellect, knowledge is hard to come by.

Guenon then compares it to the Western analogue:

\begin{quotex}
This is the `Edenic state' of the Judeo-Christian tradition; it explains why Dante placed the Terrestrial Paradise on the summit of the mountain of purgatory, that is to say at the exact point where the being quits the Earth, or the human state, in order to rise to the Heavens, described as the `Kingdom of God'. 

\end{quotex}
The balya stage is childlike because of the analogy to embryonic potentiality. Jesus implies the same idea:

\begin{quotex}
Whoever does not receive the kingdom of God like a child shall not enter it. \flright{\textsc{Luke 18:17}}

\end{quotex}
This is the state enjoyed by Adam and Eve. Nevertheless, it is only preparatory and is far from the state of Union (\emph{mauna}) or the Beatific Vision.

\paragraph{Life in the Edenic Bower}
\begin{quotex}
That spot to which I point is Paradise,

Adam's abode; those lofty shades his bower. \flright{\textsc{John Milton}, \emph{Paradise Lost}}

\end{quotex}
Origins are always a mystery.

\begin{quotex}
In paradise man would have been like an angel in his spirituality of mind, yet with an animal life in his body. \flright{\textsc{Thomas Aquinas}}

\end{quotex}
Adam awoke from his slumber with an enhanced consciousness. He was able to name the plants and animals. That is, he knew them by their natures or essences, not just as material objects. Pythagoras claimed that it was the wisest man who gave names to things.

Adam was formed, not made. To be formed means to be informed by one's essence. The form acts on matter to make it what it is. At that point, Adam was no longer an animal. He had an intellectual soul. He was a person, since he had an individual name, not a species name like the animals he named.

To make is to construct from existing material parts. I can make a mousetrap. I can even make the cheese (although it is much easier to just buy it). But I cannot make a mouse out of existing parts. A mouse can only be conceived, grow, and be informed by mousiness. There is no divine essence for a mousetrap. That is the difference between a natural kind and an artifact.

As he looked about, he saw the animals. There were vegetarians, and the predators of those vegetarians. He saw them give birth, live, hunt, and die. But as a fully conscious and aware being, he was able to avoid danger. He knew the healthiest foods for himself. He knew the healing powers of the various herbs if any problems came up.

He looked for others like him. There were beings who were similar to him in outward appearance, but quite different in their interiority. There was one who grabbed his attention. She was a woman, just like him, “bone of my bones, and flesh of my flesh”. He named her Eve and took her as his wife. They belonged to each other, and no longer with the beings they had derived from. In particular, they had to separate from their parents.

\begin{quotex}
a man shall leave father and mother, and shall cleave to his wife: and they shall be two in one flesh. 

\end{quotex}
Adam and Eve were innocent and unashamed.

\paragraph{Innocence}
\begin{quotex}
Freedom from evil desire is known as innocence. In the state of innocence, man's reason keeps the lower part of his nature, namely his body and its senses, so much in subjection that the latter can never interfere with the free deliberation of the mind, but continues to be wholly subservient to it. Reason can then activate the powers of the will, and, when they are excited, curb and suppress them. The first human beings had a nature that was pure and strong, and they had powerful and healthy bodies, nor were they denied the delights of sense, though these were always kept under control and subjected to the reason. \flright{\textsc{St Thomas Aquinas}, \emph{Summa Theologica} I, q. 98, a. 2}

\end{quotex}
In the primordial state of innocence, there are four transcendental, that is, preternatural, attributes that are missing or depraved in the natural state:

\begin{itemize}
\item Knowledge 
\item Integrity 
\item Impassibility 
\item Immortality 
\end{itemize}
\paragraph{Knowledge}
Theologians commonly refer to three areas of special knowledge possessed by Adam: regarding God and His attributes, the moral law or man's relations to God, and the physical universe both material and spiritual.

In Adam, the knowing functions appropriate to men were perfect. These are

\begin{itemize}
\item \textbf{Sense awareness}: His impressions of the world revealed to the senses were without distortion. 
\item \textbf{Rationality}: The intellectual soul separates Adam from the animals. He is rational and can follow a logical argument. That is quite rare today. 
\item \textbf{Intellection}: He had the higher functions of intuitive knowledge. Unlike rationality, this is a direct, unmediated grasp of an idea. This is the way angels know. 
\end{itemize}
In addition, he had special infused knowledge which follows from those functions.

\begin{itemize}
\item \textbf{Reality}: He had accurate knowledge of the world and of other people. 
\item \textbf{The Moral Law:} He knew the moral law and man's relationship to God. This means he also had complete self-knowledge and internal psychological processes. 
\item \textbf{Transcendence}: He had knowledge of God and his attributes. 
\end{itemize}
Adam knew Eve perfectly,

\begin{enumerate}
\item First as the image he had of her in his soul 
\item Then as a person in the flesh 
\end{enumerate}
There was no conflict between them.

\paragraph{Integrity}
Integrity means wholeness, so that Adam was unified in body, soul, and spirit. His Self was master and his desires and emotions were subjected to the dictates of reason and free will.

Adam did not experience any division or discord within himself. This was reflected, also, to his outer life. Integrity is the conformance of the interior man to the exterior man in his life, speech, and actions.

\paragraph{Impassibility}
Impassibility is freedom from suffering. In particular, one is free from negative emotions, such as hate, fear, rage, jealousy, depression, anxiety, worry, uncertainty, melancholy, annoyance, boredom, lack of self-confidence, and the like. That is anything that adversely affects one's quality of life.

Yet this is not a stoical state. In the Primordial state there are positive emotions including love, joy, satisfaction, contentment, serenity, awe. Moreover, even the negative emotions may have their place as long as they are intelligent. There is a healthy fear that protects us from danger. Anger can motivate positive actions to fix problems.

The preceding emotions are experienced in the lower emotional centre, or thumos. Above that there is a higher emotional centre, often called the Heart. Whereas the lower centre is oriented to the world, the higher centre is oriented upwards. Emotions are of a different class and more intense, such as bliss, gratitude, and other emotions that have an analogue in the lower centre.

In the lower centre, suffering is not willed. However, in the higher centre, one can take on suffering voluntarily. Fear of God, or regret for one's sins, are emotions that sound negative, but are not, because they are willed.

\paragraph{Immortality}
The Body was not a burden in the primordial state, since it was responsive to Adam's will. It was not affected by disordered emotions like anxiety, fear, boredom, excessive desire, etc., which adversely affect the systemic functions of the organism.

Adam able to avoid all dangers. Since he was fully aware of himself and his surroundings, he was able to avoid external dangers like accidents, wild animals, fire, and so on. Accidents result from

\begin{itemize}
\item momentary lapses of concentration 
\item ignorance of cause and effect 
\end{itemize}
Internal dangers like sickness, poison, ageing, and the like were also thwarted. The I not only had mastery over the emotions and desires, he also had mastery over the body. As we can see from the examples of yogis, fakirs, saints, and even contemporary magicians and endurance performers, there is more mastery over the body than we admit. Saints would be singing even while being tortured to death.

In the Primordial State, there is concentration and meditation. \textbf{Valentin Tomberg} describes the effects of deep meditation in \emph{Inner Development}.

\begin{quotex}
What is striven for, and achieved, by means of meditation is a new relationship between spirit, soul, and body. The spirit approaches the human being, it inclines downward; the soul becomes larger; and the body becomes ensouled. This is the body's inner purification. It becomes pure when everything bound up with its life is permeated by the heart. Nothing in life is ugly if the heart is present. Everything cynical that is said about the life of the body occurs through the fact that those who say such things lack the experience of the soul permeating the body. When the soul permeates the body, the body is raised to the dignity of the soul. Through meditation, a harmony arises between body, soul, and spirit—a harmony that is attained when the spirit inclines downward, the soul expands, and the body is raised to the dignity of soul. 

\end{quotex}
\paragraph{The End of the Beginning}
As pleasant as life in the Edenic Bower must have been, without suffering, anxiety, or unfulfilled desire, it had a terminus. Like Limbo or the Buddhist heaven, there was no possibility for ultimate deliverance of the Beatific Vision.

Although he was virtually immortal, Adam was obviously not actually immortal. Evil tempted him. He tried to extend mastery over himself to mastery over the entire world, which in effect is to take over the role of his creator. The delicate balance between his Ego and his Self descended into the Ego.

Only a saviour who combined both natures could, not only restore the balance, but lead to the Beatific Vision.



\flrightit{Posted on 2020-09-06 by Cologero }
