\section{Outline of Perennial Philosophy: Being}

After Why, there is the How. First, it is necessary to develop a Traditional mind. In Traditional societies, that perspective was taken as a given. In our time, however, it is not a given. We are exposed to and taught a much different point of view due to the influence of positivism, rationalism, empiricism, and materialism. It is nearly impossible for someone trapped in those worldviews to regard the spiritual as primary and to understand organic life as arising from an inner necessity rather than from external causality.

Therefore, an intellectual conversion is required; this involves a radical change in the way we understand the world. In particular, we need to consciously reject naive realism. This requires the first trial\footnote{\url{https://www.gornahoor.net/?p=1902}}, which is to become aware of those “belief systems” described by \textbf{John Lilly}\footnote{\url{https://www.gornahoor.net/?p=7512}} and how they affect our worldview. This is the method to being that process.

\begin{itemize}[nosep]
\item Start with 24 Thomistic theses\footnote{\url{http://www.catholicapologetics.info/catholicteaching/philosophy/thomast.htm}}, which represent the Perennial Philosophy 
\item Bring out logical consequences of these theses 
\item Incorporate into a larger whole of Tradition. In particular, we will rely on the following works to supplement the theses 

\begin{itemize}[nosep]
\item \emph{Multiple States of Being}, \textbf{Rene Guenon} 
\item \emph{The Individual and the Becoming of the World}, \textbf{Julius Evola} 
\item \emph{Reality}, \textbf{Reginald Garrigou-Lagrange, OP} 
\end{itemize}
\item Read them from the esoteric perspective rather than exoteric, i.e., as states of consciousness rather than things experienced. (“Truth lies in interiority,” \textit{St. Augustine}) 
\item Understand them as metaphysical doctrines that must be lived and intuited, rather than as debating points for philosophers. 
\end{itemize}

\paragraph{Essence and Existence}
Being has two poles: essence and existence. The Existence, or manifestation, of something expresses “that it is”. Its Essence is “what it is.”

Essences, or ideas, have their being in the Divine Mind. Hence, they are real. We'll use the word “subsist” to describe the Being as idea and “exist” to refer to its manifestation.

Ideas are hierarchical. The higher ideas encompass more and more of reality, so a grasp of them gives a greater understanding of the whole.

Essences may or may not be possibilities of manifestation. Meinong's golden mountain\footnote{\url{http://plato.stanford.edu/entries/meinong/}} and other nonexistent objects\footnote{\url{http://plato.stanford.edu/entries/nonexistent-objects/}} are not possibilities of manifestation. Possibilities of manifestation may exist or not at any moment. Possibilities existing simultaneously are said to be compossible.

\paragraph{Form and Matter}
Matter is pure potentiality and has no existence in itself. Essence informs matter, giving it existence. Matter can be experienced through the senses, but it is unintelligible in itself. Only ideas or essences are intelligible.

Ontologically, form is prior to existence.

Another way to put it is that the “subtle rules the dense.”

Or, “ideas create facts, not vice versa.”

\begin{quotex}
The higher a form is, the less it is immersed in matter, the more likewise does it dominate matter, and the higher does its operation rise above materiality. \flright{\textsc{St Thomas Aquinas}}

\end{quotex}
\paragraph{Act and Potency}
\begin{quotex}
Every organism is a set of possibilities within a certain framework, and its life is the process of actualization of these possibilities. \flright{\textsc{Ulick Varange}} 

\end{quotex}
A Being is a mix of Act and Potency. Act is what has been actualized, or manifests, i.e., it exists. Potency refers to those qualities of the idea that have not been actualized, yet have the possibility to exist.

Existing things are subject to external forces that do not emanate from its essence; these are called accidents.

Qualities of a thing that arise from its essence are called essential.

\paragraph{Privation}
Since a Being does not actualize all its potency, there is a gap between its essence and its existence. This is called privation.

Existence is a good, so the will is attracted to it. Privation, then, is bad. Privation is indicative of the will's lack of power.

\begin{quotex}
Everything I cannot act on, everything that resists my will, is only a privation of this very will, something negative, not a being, but a non-being. \flright{\textsc{Julius Evola}} 

\end{quotex}
\paragraph{Human Being}
\begin{quotex}
A person is an individual substance of a rational nature. \textsc{Boethius}

Man is a rational animal. \textsc{Aristotle} 

\end{quotex}
As a substance, he subsists in the Divine Mind. As individual, he exists in the material world.

Since a being is a unity, the unifying force is the I or Self.

As rational, he has intelligence and will, or an rational soul. That is his essence. As animal, he has vegetative and sensitive souls and a physical body. The rational soul is therefore the form of the lower souls and the body, which together form a unity.

Hence, the human body is formed from the rational soul. Therefore, characteristics like sex or race are essential and spiritual qualities, and not accidental or biological. The soul incarnates when the material circumstances are disposed to receive it; i.e., its manifestation is compossible with the environment and historical situation in which it manifests.

\paragraph{Human Perfection and Privation}
The human person is his essence, or primordial state, is “good”. That is, he possesses a Real I, Consciousness, and a True Will.

The Real I is the unity of the human being. The Real I is intelligent and grasps knowledge intuitively and directly. The Real I has two transcendent faculties: Intellect, or Consciousness, and Will

The Intellect can grasp the inward raison d’être of a thing as well as necessary and universal principles. The more it knows, the more conscious it is.

The True Will follows the Intellect. It will manifest the Real I according to intelligence.

Obviously, in our present circumstances, man experiences privation rather than the primordial state.

Unity is lacking, since the Real I is forgotten and the human being as existing is subjected to the whims of various sub-personalities.

Consciousness, or knowledge, is unreliable, since the human being is not conscious of his own self or what he is doing. His essential characteristics are opaque to him and considered as the not-I. He attributes them to external forces.

His Will lacks the power to manifest the primordial state. It no longer follows the Intellect, but becomes enthralled by various external thoughts and opinions. Since the I is not unified, neither is the will. It will be driven by the emotions rather than the intellect.

Why this privation occurred and the way back to the primordial state are not considered here.

\paragraph{Active Creation}
Exoteric religion treats man from a passive point of view. Creation

Esoteric: active, emanation. Not in conflict. God is outside time, so He doesn't poke around in the world every now and then. The person subsists in the divine mind, since God is absolute, simple and without parts.

Hence it is just as correct to say that the human being chooses his incarnation.

By analogy, the Real I is the Christ, hence there is no distinction.

The True Will is aligned with the will of God, so again there is not distinction between what I will and what God wills.

Man creates his life and the more being he has, the more of his life's potentialities are actualized.

\paragraph{Summary}
Although this outline can be clarified, corrected, and expanded, its points are not intended for a philosophical style debate. If you think about it, you will see that it is true. Much becomes clear when seen in this light. Absorbing these principles interiorly will lead to that “aha” moment of intellectual conversion.

The next outline will concern Knowledge.

\flrightit{Posted on 2014-09-18 by Cologero}
