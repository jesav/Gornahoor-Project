\section{Logos and Chaos}

The idea of Chaos was known to the early Greeks. The poet \textbf{Hesiod} wrote:

\begin{quotex}
First of all, Chaos came into being; then Earth with her broad breast, for all things a seat secure forever. 

\end{quotex}
\textbf{F. M. Cornford} explains that Chaos is not `formless disorder'. as we conceive it today, but rather the `yawning gap' between earth and heaven. \textbf{Euripides} wrote: `Heaven and Earth were once one form.' In the Rig Veda, Varuna `held asunder spacious Earth and Heaven.' For the ancient Egyptians, \emph{Shu} separates \emph{Nut} (sky) from \emph{Seb} (earth). Of course, in China it was taught that from the Tao came the duality of Yin and Yang, Earth and Heaven.

In the ancient sense, Chaos does not represent the destruction of the ordered physical world, but rather the unmanifested Void that is the principle of heaven and earth, male and female. So behind the physical world, there is a principle that represents the unity of Being. For \textbf{Heraclitus}, there is one Logos, one reason for everything, throughout “the one cosmos which is the same for all.”

\textbf{Rene Guenon}'s criticism of Western theology is that it stops at the consideration of Being alone, and does not penetrate beyond Being. This is true to the extent that God is understood as Being, which is true from the point of view of manifestation. Nevertheless, some figures such as \textbf{Meister Eckhart} and \textbf{Jacob Boehme} have recognized the “God beyond God” or the Urgrund (Void) ontologically prior to Being. In the East, there is the understanding of the difference between the essence and energies of God. In a metaphysical sense, the West can perhaps be accused of a forgetfulness of Non-being rather than the forgetfulness of Being.

Guenon identifies the Divine Intellect as the `location' of the eternal ideas. In his conception, the theologian only recognizes the possibilities of manifestation, whereas the metaphysician also recognizes the possibilities of non-manifestation. Guenon identifies the Divine Intellect as the Logos (or Word), which is the principle of manifestation since all things are created from it. Hence, at the metaphysical level Logos is the same as Chaos, as the unifying principle of manifestation. The Logos can also be identified with the Tao, as Tao is used in Chinese translations of the Bible for the Logos.

In Mathematics, Chaoas Theory\footnote{\url{https://en.wikipedia.org/wiki/Chaos_theory}} demonstrates that there can be a hidden order behind seemingly chaotic systems.

\begin{quotex}
Chaos theory states that within the apparent randomness of chaotic complex systems, there are underlying patterns, interconnectedness, constant feedback loops, repetition, self-similarity, fractals, and self-organization. 

\end{quotex}
Thus, there is no opposition — and there cannot be — between Logos and Chaos at the metaphysical or transcendental level. There the battle has already been won. Hence, the battle is fought on the \emph{psychic} and \emph{hylic} levels, that is, between \emph{cosmos} and \emph{chaos}, which can be understood as `formless disorder' only at this level\footnote{References:

[1] From Religion to Philosophy, F. M. Cornford

[2] Miscellanea, Rene Guenon}.

This will be explored in the “Conditions of Corporeal Manifestation”.


\flrightit{Posted on 2012-09-19 by Cologero }

\begin{center}* * *\end{center}

\begin{footnotesize}\begin{sffamily}


\texttt{Cassiodorus on 2012-09-20 at 01:19 said: }

Cologero,

I recently discovered this site and I've been devouring the many interesting and insightful posts . There is something that I am somewhat puzzled about. Being that Rene Guenon has such prominent role here, why is there such a dearth of treatment of Frithjof Schuon and others of the Traditionalist school?


\hfill

\texttt{Andrew on 2012-09-20 at 15:59 said: }

This actually answers (at least in part) a question I had asked about Buddhism and Tradition. In Buddhism the condition for a thing is its principle.

If the Void is the Principle of Manifestation, then Form is Emptiness, Emptiness is Form. Another name for Shunyata (Voidness) is Dharmadhatu (which I translate as “World of the Word”). Also, the Buddha said, “He who sees Dharma sees Me; He who sees Me sees Dharma”.


\hfill

\texttt{Logres on 2012-09-20 at 21:57 said: }

Cologero has stated that he doesn't post on Schuon because there are (in his judgement) many well done sites on such already. At least this was his answer if my memory serves, last time it was brought up.


\hfill

\texttt{Cassiodorus on 2012-09-20 at 22:49 said: }

Cologero or Logres,

I'm an ex- postmodern agnostic that discovered the teachings of Tradition and the Perennial Philosophy several years ago. Since then, the focus of my studies has centered on Catholic Christianity, resulting largely from my increasing acquaintance with the Church fathers and, most importantly, with the Angelic Doctor, Thomas Aquinas. Interestingly, I've noticed that most traditional Christians are hostile to the Primordial Tradition, quick to make the charge of pantheism, monism, or even gnosticism. For most Thomists that I've spoken to, the Perennial Philosophy is often rejected on the grounds of being a modern variant of “eastern mysticism” and that Tradition is antithetical to monotheism by blurring the line that separates God and man. Plainly stated, do you think there is a problem in reconciling the classical theistic traditions of the West with nondualism and the doctrine of the Self?


\hfill

\texttt{Cologero on 2012-09-20 at 23:42 said: }

Cassiodorus, rather than ask why we don't do something–a negative, the better question is to understand our purpose. Primarily, it is to determine how Tradition can be recovered in the West. Coincidentally–if you believe in coincidences–your namesake serves as a suitable model. The real Cassiodorus was able–in his own mind–to integrate the Ancient with the Medieval, the Latin West with the Greek East, the Roman with the Germanic.

We agree with Guenon that Medieval Europe had more in common with the civilizations of the East than with the Modern West. Even “traditional” Catholics go back only to the Council of Trent and are thus still far from the worldview of the Medievals. Following various suggestions by Guenon, we have pointed out the commonality of the doctrines of Thomas with other Traditions. Thomism today is reduced to a “philosophy”, one among other competing systems. However, it must be understood as a metaphysics, which is known experientially, not just rationally; it is totalitarian, demanding one's total adherence.

Thomism will be deepened by the encounter with Tradition. Some can't see it; others do. The charges of pantheism or gnosticism arise because of a misunderstanding of Being or an ignorance of Christian gnosis.

When you can make a certain shift in consciousness, many seemingly insoluble puzzles are resolved.


\hfill

\texttt{Cassiodorus on 2012-09-21 at 01:15 said: }

Cologero,

Many thanks for your thoughtful response. “Thomism will be deepened by the encounter with Tradition. Some can't see it; others do.” This comment reminded of something that Ananda Coomaraswamy said in “Vedanta and the Western Tradition”,

To say that “I am a pantheist” is merely to confess that “I am not a metaphysician,” just as to say that “two and two make five” would be to confess that “I am not a mathematician.”

Nevertheless, on the subject of Christian gnosis, I've been told by both Catholics and Orthodox that there is a fundamental difference between Christian divinization, theosis, and the Indian Supreme Identity, moksha. Furthermore, it is said, the Perennial Philosophy that Agostino Steuco and Leibniz refer to cannot be expanded to encompass the traditions of the East because participation in the Divine and identification with the Divine represent two incompatible ontologies. 

Would you say that apophatic theology and pure metaphysics are essentially the same thing?

Best Regards,

Cassiodorus


\hfill

\texttt{Cologero on 2012-09-21 at 08:05 said: }

Cassiodorus, find the man who has experienced both “participation” in the divine and moksha; I'm sure he could explain the differences to us. Otherwise, it is idle chit chat. This is a journey every man must make for himself and he can't rest content just reading someone else's postcards.


\hfill

\texttt{Cassiodorus on 2012-09-21 at 10:13 said: }

I believe I see your point.


\hfill

\texttt{Michael on 2012-09-21 at 20:32 said: }

Cologero, so how does one go about experiencing Aquinas and not just absorbing his writings rationally? Sorry for the ignorant questions.


\hfill

\texttt{Cologero on 2012-09-22 at 10:08 said: }

It is not an ignorant question, but the only question worth asking. Check out Letters of AKC\footnote{\url{http://akc.satishankar.com/2012/09/selected-letters-of-ak-coomaraswamy.html}} and read the intro to the first volume. Then bring up the topic again.


\hfill

\texttt{Cologero on 2022-02-12 at 08:04 said: }

Although too much Wittgenstein is risky, a small dose may be helpful:

“Whereof one cannot speak, thereof one must be silent.”

Otherwise, one is just engaging in a language game, trying to make various propositions compatible with each other. Well, the world is not a proposition nor a riddle to be solved.

“Thinking”, as important and as pleasant as it is, is nevertheless a low stage of development, actually only at stage three of Augustine's seven stages of the ascent to God\footnote{\url{https://www.gornahoor.net/?p=10801}}.


\end{sffamily}\end{footnotesize}
