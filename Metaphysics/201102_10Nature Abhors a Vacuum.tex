\section{Nature Abhors a Vacuum}

In \emph{The Multiple States of Being}, \textbf{Rene Guenon} describes the concept of the Infinite, Being, the World, among other ideas. The following definitions are taken from this work.

\begin{description}
\item[Infinite ]

All possibilities 

\item[Non-Being ]

Possibilities of non-manifestation and possibilities of non-manifestation not manifested 

\item[Being ]

Manifested possibilities 

\item[World ]

A domain formed by a certain ensemble of compossibilities which are realized in manifestation. 

\item[Void ]

Absence of any formal qualities or anything to do with any mode of manifestation 

\end{description}
He points out that the \textbf{Void} has no possibility to be manifested in any world, by its very definition. For those who think that metaphysics has no practical application, this idea demonstrates its fruitfulness for our understanding.

First of all, in the purely physical sense, a Vacuum cannot actually exist. Even is you deny the existence of the Ether, which physics has not really completely ruled out, there is more than nothingness going on in what we call a vacuum.

But more important for our purposes, the non-manifestation of the Void tells us something about the cosmic environment. There can be no gaps in the psychic and physical processes; that it, there is necessarily a continuity synchronically and diachronically. Another way to say it is that everything is in relationship to everything else. Guenon writes:

\begin{quotex}
The cosmic environment can only be conceived of as a whole of which each part is linked to every other part without any break in continuity. To try to conceive of it otherwise would be to assume the existence of a void, but this is not a possibility of manifestation and can have no place in the Cosmos. 

\end{quotex}
This, we have been at pains to point out the relationships between civilizations, traditions, and the transitions from age to age of the cosmic cycle. So what we experience now is in in continuity both with the past and also with everything else contemporaneous. The whole weight of the past is with us now. Hence, something, for example a particular tradition, cannot have been lost and then suddenly reappear.



\flrightit{Posted on 2011-02-10 by Cologero }
