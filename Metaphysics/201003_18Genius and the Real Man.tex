\section{Genius and the Real Man}

\begin{quotex}
A human being should be able to change a diaper, plan an invasion, butcher a hog, con a ship, design a building, write a sonnet, balance accounts, build a wall, set a bone, comfort the dying, take orders, give orders, cooperate, act alone, solve equations, analyze a new problem, pitch manure, program a computer, cook a tasty meal, fight efficiently, die gallantly. Specialization is for insects. \flright{\textsc{Robert Heinlein}}

Do I contradict myself?

Very well then I contradict myself,

I am large, I contain multitudes.

\flright{\textsc{Walt Whitman}, \emph{Song of Myself} }

All great men, then, have a conviction, really independent of external proof, that they have a soul.

\flright{\textsc{Otto Wieninger}}

\end{quotex}
Evola wrote to Guenon to clarify what is meant by “Real Man” and this is a summary — with comments — of the response he received.

\begin{quotex}
Every Real Man has realized all the possibilities of the human condition, but each one has done so in a way which is typical of him alone, and which differentiates him from all other Real Men. If that were not the case, how could there be room, in our world, even for beings who have not achieved that level?

\end{quotex}
In other places, Guenon identifies this as the characteristic of a man of the Kshatriya caste. This differentiates him from the Brahmin caste, whose goal is the jivan-mukta — liberated while still in this life — who has realized the totality of the possibilities of all the states of being.

In order to realize his possibilities, the Real Man needs to engage the world. Since Guenon's characterisation is general and therefore vague, it is instructive to compare the concept of the Real Man with the concept of the Genius as described in great detail by Otto Weininger in has masterwork, \textbf{Sex and Character}. He writes: “Genius is, in its essence, nothing but the full completion of the idea of a man.” It is remarkable how close these two conceptions are, despite coming from radically different sources.

The Genius, then, has a complete understanding of men, since he embraces all their personalities in his own consciousness. This is how the great Leader is able to deal appropriately with many sorts of men, or how the great Writer can describe so many varied characters, their actions, and their motives. Weininger writes:

\begin{quotex}
The man of genius takes his place as someone who understands other beings incomparably more than the average man does. Goethe is said to have said of himself that there was no vice or crime of which he could not trace the tendency in himself, and that at some period of his life he could not have understood fully. The genius, therefore, is a more complicated, more richly endowed, more varied man; and a man is the closer to being a genius the more men he has in his personality, and the more really and strongly he has these others within him. 

\end{quotex}
This is what makes the Genius free. Since he has more possibilities, he has more choices in any situation. The lesser man is inhibited and restricted by his own personality.

\begin{quotex}
And so the ideal genius, who has all men within him, has also all their preferences and all their dislikes. There is in him not only the universality of men, but of all nature. He is the man to whom all things tell their secrets, to whom most happens, and whom least escapes. He understands most things, and those most deeply, because he has the greatest number of things to contrast and compare them with. The genius is he who is conscious of most, and of that most acutely. 

\end{quotex}
The Genius is distinguished from the merely talented man by his universality. The latter may do one thing well, even very well, but he is still not a Genius. The talented man may develop one idea, perhaps two, but then spends the rest of his life as a series of footnotes to that original impulse. A new idea has little chance of penetrating his mind.

\begin{quotex}
Universality is the distinguishing mark of genius. There is no such thing as a special genius, a genius for mathematics, or for music, or even for chess, but only a universal genius. The genius is a man who knows everything without having learned it. 

\end{quotex}

\hfill

\flrightit{Posted on 2010-03-18 by Cologero }

\begin{center}* * *\end{center}

\begin{footnotesize}\begin{sffamily}


\hfill

\texttt{Ryan Haecker on 2010-03-21 at 01:30 said: }

This is an enjoyable summary of Weininger's description of a genius.


\hfill

\texttt{Simon Friedrich on 2010-04-26 at 13:47 said: }

You might like to compare these ideas of the Real Man/Genius with Gurdjieff's explanation of the (potential) “Real I” and the multiple personalities with a man. 

Everyone, including Real Men, possess multiple personalities. Most individuals are hypnotized into believing these personalities – at one moment one, another moment another – to be who they are. A Real Man still possesses these personalities, but he literally “possesses” them, and is not them. He is one “I”, and a unique one at that, in possession and in control of his various personalities. (See also Max Stirner – “The Unique and his Property” – “Der Einziger und sein Eigentum”)

The richer his inheritance and upbringing, the broader his property of personalities, to the point that he approaches being a universal manifestation of Man.

He also possesses dark and evil personalities, since these too form part of Man's possibilities. But he controls them, deactivates them.

All the above may also be seen in Ernst Juenger's idea of the Anarch, which he developed in part from Stirner's Unique.


\hfill

\texttt{Cologero on 2010-04-26 at 22:18 said: }

Genau. But to bring in the “ideas” of Gurdjieff would be lock them in at the level of dianioa. As you assuredly know, along with the ideas there must be “work on onself” to raise these ideas to the level of episteme.


\hfill

\texttt{Kelly Jones on 2014-02-17 at 01:45 said: }

Everyone can become a genius. It just requires infinite-mindedness. It doesn't mean developing a richer personality, or multiple skillsets. It's about seeing the whole multiplex of Universes as your own self, which is the case given there are no inherent boundaries.


\hfill

\texttt{Cologero on 2014-02-18 at 00:35 said: }

Kelly: Kevin Solway, David Quinn, and I go back a long way. Presumably you have seen a lot of geniuses arise during your association with those gentlemen?


\hfill

\texttt{Kelly Jones on 2014-02-18 at 22:48 said: }

Hi Cologero. You asked, “Presumably you have seen a lot of geniuses arise during your association with those gentlemen?” 

Not at all. Unfortunately, even though everyone can be a genius, it is very rare for anyone to persevere past the early bursts of intensity. Most of the budding geniuses I've seen over the last decade want happiness, and social recognition as the leaders in a kind of philosophical revolution, crowned in fanfare and glory. They want recognition and endorsement by the leading authorities of the land (as all the children's fairy tales would have it). Ah, as if such glory is real!

When these young wisdom-seekers (they can only be men) realise most people disapprove and ridicule their ideas, and when they realise this puts them at high risk of losing their human dreams like an adoring family, worldly security, and material comforts, then all the stuffing goes out of them. They want happiness far more than truth. Glory, yes! Happiness, yes! Truth…. hang on, who said anything about truth?!

It is a hard path, but it is a noble path. It requires a strong character, and strong purpose. So, everyone can become a genius, but few want it.

So the biggest stumbling block for budding geniuses is their need for social approval. So, instead of recognising this deep need for happiness, and fighting to submit that desire to their higher goals, they make excuses like: “other people aren't ready”, or “no one will listen to these ideas”. Then they turn around and fall into easy but mediocre, short-sighted pathways, like becoming a good scholar, an environmental activist, an eccentric artist, a vegan socialist, and so forth. They excuse it as saying, this temporary step is what is needed more — but if there is no one standing at the final destination of all those temporary steps, then how can people's gaze be drawn to it? Who can know about it?

I don't think I have heard of your pseudonym before. What is your real name, if I might ask?


\hfill

\texttt{Cologero on 2014-02-20 at 00:50 said: }

So, Kelly Jones, becoming a genius is really a moral issue, not simply an intellectual one. These are some of the qualities you mention:

perseverance, strong character, strong purpose

There are the obstacles you mention such as the desire for pleasure and worldly success that prevent the attainment of genius.

But if there are no boundaries, then you yourself are reflected in those that fall short. Or am I misunderstanding you?

I don't believe I am using a pseudonym; perhaps someone will vouch for me.


\hfill


\end{sffamily}\end{footnotesize}
