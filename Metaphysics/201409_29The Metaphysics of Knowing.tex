\section{The Metaphysics of Knowing}

\paragraph{Sense Knowledge}
Sense knowledge is known through the senses and mediated by thought. There are three inner aspects involved in knowing.

\begin{itemize}
\item \textbf{Sensation}. The senses apprehend an object 
\item \textbf{Imagination}. The mind forms and image of the object 
\item \textbf{Intellect}. The intellect knows what the object is 
\end{itemize}
What the intellect knows is the form or essence of the object. In that case, the knower \emph{is} the thing, since its essence exists in his mind. However, he is \emph{not} the thing in terms of its existence in manifestation. This is the principle that knowledge is being.

A consequence is that deeper understandings are possible. For example, some people are attuned to animals. They are grasping more of the essence than most people. Others, like Padre Pio, can read men's souls.

\paragraph{Abstract Knowledge}
Beyond sense knowledge, there is knowledge that derives from abstracting from the world of sense. These are necessary and universal principles, which are independent of all particular facts in space and time. In other words, they are forms of thought. There are three levels of abstraction:

\begin{itemize}
\item \textbf{Natural Science}. This is the science of the inner universal nature of mineral, plans, and animals, apart from any particular thing. 
\item \textbf{Mathematics}. This includes arithmetic, the science of number, and geometry, the science of shapes. Again, these are abstracted from any sense qualities. 
\item \textbf{Metaphysics}. This is the science of Being in which the intellect recognizes the characteristics of being as such. We previously outlined the main principles of this science. 
\end{itemize}
These abstract sciences demonstrate the immateriality of the soul. Since its knowledge is beyond time, and knowledge is being, the intellectual soul is likewise beyond time.

\paragraph{Analogical Knowledge}
Since the natural man knows through sense, imagination, and thought, his understanding of spiritual things also begins in the senses, or imagination. That is why spiritual things are explained in terms of images, art, or symbols. That is a way to get to the thought about such things.

The principle of analogy then is that the intellect knows God and spiritual beings by analogy with the sense world. That is, the knowledge of spiritual things is mediated by the imagination and thought.

However, those who do not understand this principle, are able to conceive the spiritual world only in sensual imagery. This leads to misconceptions and distortions.

Therefore, it is necessary to grasp this principle. In Hermetism, this is expressed in the formula, “As above, so below.”\textbf{ Valentin Tomberg} explains this principle:

\begin{quotex}
Analogy is the first and principal method whose use facilitates the advance of knowledge. It is the first conclusion drawn from the tenet of universal unity. Since at the root of the diversity of phenomena their unity is found in such a way that they are at one and the same time different and one, they are neither identical nor heterogeneous but are analogous insofar as they manifest their essential kinship. 

\end{quotex}
Both Valentin Tomberg and Rene Guenon offer extensive discussions of how to understand metaphysical and religious symbolism. They are worth the study.

\paragraph{Intuitive Knowledge}
A direct, intuitive knowledge (not mediated by symbols or thought) is possible beyond the natural state. This is the way angelic intelligences know. \textbf{Fr Reginald Garrigou-Lagrange} explains knowledge in the postmortem state:

\begin{quotex}
The separated [i.e., postmortem] soul knows itself directly, without medium. … By this immediate self-knowledge, it sees with perfect evidence its own native spirituality, its immortality, its freedom. … It knows God, no longer in the sense world as mirror, but as mirrored in its own spiritual essence. It sees with transcendent evidence the solution of the great philosophic problems, and the absurdity of materialism, determinism, and pantheism. [postmortem] souls have knowledge of one another and the angels. 

\end{quotex}
Clearly, that explanation leaves out of consideration the possibility of achieving such states before physical death. Of course, the whole point of the spiritual path of theosis is indeed to achieve such states. We have the testimony of those who have achieved them and we have the metaphysical understanding that shows such states are possible while alive.

Knowing in this way is the way that angels know\footnote{\url{https://www.gornahoor.net/?p=2995}}. Angelic intelligences can also be understood as higher states of being, i.e., beyond the natural man.



\flrightit{Posted on 2014-09-29 by Cologero }
