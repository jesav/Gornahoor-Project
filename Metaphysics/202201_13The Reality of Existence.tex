\section{The Concept of Existence}

\begin{quotex}
In the eyes of an ordinary man representing the common-sense view of things, the phenomena are the visible and manifest while the Absolute is the hidden. But in the unconditioned consciousness of a real mystic-philosopher, it is always and everywhere the Absolute that is manifest while the phenomena remain in the background. \flright{\textsc{Toshihiko Izutsu}, \emph{The Concept and Reality of Existence}\footnote{\url{https://www.gornahoor.net/library/Concept_and_Reality_of_Existence.pdf}}}

\end{quotex}
The foundation of esoteric teaching is the “unity of existence”. This is presumed in \textbf{Rene Guenon}'s \emph{Symbolism of the Cross}.

\begin{quotex}
Existence is one and indivisible in its inner nature, just as Being is one in itself; indeed this unity of Existence derives directly from the oneness of Being, since universal Existence is nothing but the total manifestation of Being, or, to be more exact, the realization, in manifested mode, of all the possibilities that Being implies and contains principially in its very oneness.

\end{quotex}
What follows is a simplification, but not a misleading, explanation, so it may clarify things for those just starting out and orient them toward the full teaching.

The common opinion is that existence consists of multiplicity rather than unity. That is, for the person at this stage, all the things in the phenomenal world, taken together, form existence; this approach fails to perceive the unity of existence. This common sense view is that objects have properties. For example, the table is brown, the grass is green, the crow is black. This assumes that the table, grass, and the crow exist before they can have properties. A conundrum then arises when you say, “the table exists.” This assumes that existing is a property of the table, but how can it exist before it exists?

As a thought experiment, try reversing things. Then assume that the table is a property of existence, as is the grass and the crow. Then existence is one, but it manifests, to all appearance, as a table, grass, or a crow; phenomena, thus, are just modes or attributes of existence. Otherwise, the experience of the multiplicity of things forms a veil which obstructs the sight of the unity of existence.

This applies, a fortiori, to the understanding of God (Absolute). 

\begin{quotex}
When the Pharoah asks Moses the trick question, “What is the Lord of the Worlds” (Quran 26:23), he is expecting an “essence” that exists. For example, the simple-minded answer is to answer something like, “God is the most perfect being” or “God is an omnipotent Spirit”, etc. These responses presume that the essence precedes existence. So Moses avoids the trap and answers, “The Lord of the heavens and the earth, and everything between them, if you are aware.” \flright{\textsc{Ibn Arabi}, \emph{The Wisdom of Sublimity in the Word of Moses}}
\end{quotex}

That means metaphysically that God's essence is His existence. That is Sufi teaching and it is also Catholic teaching. In the medieval period, while war was going on among the exoterists, esoteric teachings were being shared.

When this is understood, then the things in their multiplicity are seen as revelations of the Absolute, or as a theophany. The first stage, then, is to understand this as the concept of reality, something that can be reached through human reason.

That is Knowledge of the Head.

The next stage of metaphysical realization is to transform the concept into the reality of existence in one's own consciousness.


\paragraph{The Reality of Existence}
\begin{quotex}
In [the] major philosophies of the East, metaphysics or ontology is inseparably connected with the subjective state of man, so that the selfsame Reality is said to be perceived differently in accordance with the different degrees of consciousness. \flright{\textsc{Toshihiko Izutsu} }

\end{quotex}
When the being understands the concept of existence as concept, he still doesn't know the reality of existence. That is because he remains in the human state. He still experiences the multiplicity of beings and the unity of existence remains hidden. To attain the realization of the unity of existence, he needs to transcend the human state. In \emph{Symbolism of the Cross}, \textbf{Rene Guenon} hints at the process:

\begin{quotex}
This station, or degree of the being's effective realization, is attained by al-fana, i.e., by the annihilation of the ego in the return to the primordial state; such annihilation is not without analogy to the Nirvana of the Buddhist doctrine. 

\end{quotex}
As he emphasizes in \emph{Man and his Becoming}, the fundamental distinction between the Self and the ego must be grasped. As long as the being identifies with the ego, the higher states are obscured, much like the way a constant cloud cover hides the sun, the moon, and the stars. Such as we are, the ego is drawn to the world. Ego consciousness is dominated by desires arising from lower states; it is moved by the negative emotions of fear, anxiety, worry, depression, and so on. The mind entertains extended fantasies of base desires. Thoughts are feeble, confused, and illogical.

The annihilation of the ego requires the undoing of all those tendencies. It is better to have a guide who can explain the process and give feedback. Through concentration exercises, one develops self-knowledge, including both the good and bad qualities of the soul. Through a process of purification, ego consciousness diminishes and the Self reveals higher states of being.

Knowledge of the higher states cannot be attained by rational reasoning, but only by a higher type of intellect called “intuition”, which is a direct experience or reality unmediated by verbal descriptions. At this stage, there is no longer a distinction between the knower and the known. Rational thinking always assumes a subject and an object, so it never knows existence which is never an object. But at the intuitive stage, existence is not known as an object from the outside, but rather by the knower being inseparably existence itself. At this stage, existence is made manifest and the multiplicity of things are hidden in the bright light of Existence. They have become indiscernible from each other.

This is Knowledge of the Heart.

\paragraph{The Unity of Existence}
\begin{quotex}
Beyond \emph{fana}, there is still \emph{fana al-fana}, the annihilation of the annihilation, which similarly corresponds to Parinirvana. In a certain sense, the passage from one of these degrees to the other is related to the identification of the centre of a state of the being with that of the total being. \flright{\textsc{Rene Guenon}, \emph{Symbolism of the Cross}}

\end{quotex}
The intuition of the unity of existence is often accompanied by blissful experiences. In many cases, there is a strong desire to re-experience that bliss. Oftentimes, people will stumble accidentally onto the intuition of the unity of existence. They may assume their experience is unique and they will talk about the “being one with everyone” and the like.

Like cargo cultists, they will repeat rituals like chanting, dancing, drumming, breathing exercises, extreme ascetical practices, psychedelic drugs, etc., in order to recreate the experience. What all these practices have in common is the attempt to annihilate the ego artificially by tiring it out or bypassing it, rather than through self-purification. In the worst cases, they might resort to sexual practices because the bliss of physical pleasure is thought to imitate the bliss of the higher states.

Megalomania, immorality, and heresy are the fruits of such practices. An esoteric school will guide seekers to avoid such excesses and the exoteric teaching will prevent immorality and heresy. So the next stage requires the annihilation of the annihilation, the disappearance of the consciousness of the ego annihilation. The consciousness of the experience is still not consciousness of absolute Reality.

Although the annihilation is a human experience, the subject is no longer the human state, but the Self revelation of absolute Reality. At the early stage, Reality was concealed and revealed itself only in the things and events of the phenomenal world. At this stage, however, Reality reveals itself as Absolute, i.e., it is unveiled and the manifest world is empty. Guenon describes it this way:

\begin{quotex}
The emptiness is complete detachment from all manifested, transitory and contingent things

\end{quotex}
Nevertheless, the Being regains his experience of the world, now understood both in its Unity and Multiplicity. Toshihiko Izutsu explains it like this:

\begin{quotex}
He regains his normal, daily consciousness and accordingly the normal, daily, phenomenal world of multiplicity again begins to spread itself out before his eyes. The world of multiplicity appears again with all its infinitely rich colors. Since, however, he has already cast off his own determination, the world of multiplicity he perceives is also beyond all determinations.

\end{quotex}

\hfill

\paragraph{The Experience of Existence}
Tradition is not an intellectual system, because it cannot be fully understood without having undergone a change in one's being. This is \textbf{Rene Guenon}'s description of the process of bringing that about, from \emph{Symbolism of the Cross}.

\paragraph{The Greater Holy War}
\begin{quotex}
The “greater holy war” is man's struggle against the enemies he carries within himself, that is, against the elements in him that are opposed to order and unity. There is however no question of annihilating these elements, which, like everything that exists, have their reason for existence and their place in the whole; what is aimed at is to “transform” them, by bringing them back and as it were reabsorbing them into unity. Above all else, man must constantly strive to realize unity in himself, in all that constitutes him, through all the modalities of his human manifestation:

\end{quotex}
\begin{itemize}
\item unity of thought 
\item unity of action 
\item and also, which is perhaps hardest, unity between thought and action 
\end{itemize}
\paragraph{The Will of God}
He then addresses what life is like after the realization of the unity of Existence.

\begin{quotex}
For whoever has achieved the perfect realization of unity in himself, all opposition has ceased and with it the state of war, for from the standpoint of totality, which lies beyond all particular standpoints, nothing remains but absolute order. Nothing can thereafter harm such a being, since for him there are no longer any enemies, either within him or without.

\end{quotex}
His actions become aligned with the Will of God. Guenon continues:

\begin{quotex}
Permanently established at the centre of all things, he “is unto himself his own law” because his will is one with the universal Will. He has obtained the Great Peace, which is none other than the \textbf{Divine Presence}; being identified, by his own unification, with the principial unity itself, he sees unity in all things and all things in unity, in the absolute simultaneity of the \textbf{Eternal Now}.

To return to one's root, is to enter into the state of repose. By it the being escapes from the vicissitudes of the “stream of forms”, from the alternation of the states of life and death, or of condensation and dissipation, and passes from the circumference of the cosmic wheel to its centre, itself described as “the void (the unmanifest) which unites the spokes and makes them into a wheel”.

\end{quotex}
\paragraph{Detachment and Impassibility}
\begin{quotex}
The ideal is the indifference (\textbf{detachment}) of the transcendent man, who lets the cosmic wheel tum. This absolute detachment renders him the master of all things, because, having passed beyond all oppositions inherent in multiplicity, he can no longer be affected by anything: He has attained perfect \textbf{impassibility}; life and death are equally indifferent to him, the collapse of the (manifested) universe would cause him no emotion. By dint of search, he has reached the immutable truth, the unique universal Principle. He lets all beings evolve according to their destinies, and he stands at the motionless centre of all destinies.

\end{quotex}
\paragraph{Imperturbability}
\begin{quotex}
The outward sign of this inner state is \textbf{imperturbability}: not that of the hero who hurls himself alone, for love of glory, against an army in line of battle, but that of the spirit which, higher than heaven, Earth, and all beings, dwells in a body to which it is indifferent, taking no account of what its senses convey to it, and knowing all by global knowledge in its motionless unity.

\end{quotex}


\flrightit{Posted on 2022-01-11 by Cologero }

\begin{center}* * *\end{center}

\begin{footnotesize}\begin{sffamily}

\texttt{Mike M on 2022-01-17 at 23:10 said: }

The emphasis on not becoming attached to anything is one of the positives of the Zen tradition, it is no coincidence it came out of the East in a time after many “esoteric” traditions came into the world. I think there is potential for taking the practice's essence into other exoteric forms even if it is just a starting point, as one may suggest from Toshihiko Izutsu's life and some of my own experiences.

For those familiar, the Guenon quote is also in exact correspondence with Dogen's journey where Mind and Body dropped away as well as dropping Mind and Body: To study the Way is to study the Self. To study the Self is to forget the self. To forget the self is to be enlightened by all things of the universe. To be enlightened by all things of the universe is to cast off the body and mind of the self as well as those of others. Even the traces of enlightenment are wiped out, and life with traceless enlightenment goes on forever and ever.

\end{sffamily}\end{footnotesize}