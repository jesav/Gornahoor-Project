\section{Principial Origin and Final Destiny}

\begin{quotex}
While relative freedom belongs to every being under any condition, absolute freedom can belong only to the being freed from the conditions of manifested existence, whether individual or even supra-individual, and has become absolutely “one”, to the degree of the pure Being, or “without duality” if its realization surpasses Being. It is then, but only then, that one can speak of the being “who is in himself his own law”, because that being is fully identical with his sufficient reason, \textbf{which is both his principial origin and final destiny}. \flright{\textsc{Rene Guenon}, \emph{Multiple States of the Being}}

\end{quotex}
In the final chapter of the third book of his metaphysical trilogy — \emph{Multiple States of the Being}, Chapter 18: The Metaphysical Notion of Freedom — Rene Guenon provides a metaphysical proof for free will, properly understood. The proof has two stages:

\begin{enumerate}[nosep]
\item Demonstrate the possibility of Freedom from Nonbeing 
\item Demonstrate the actuality of Freedom from Being 
\end{enumerate}

\paragraph{Freedom}
First of all, let's be clear about what freedom is. Guenon provides this definition:

\begin{quotex}
Freedom is the absence of constraint, a definition negative in form but fundamentally positive, for it is constraint that is a limitation, that is to say a veritable negation.

\end{quotex}
The absence of constraint means that the being acts in accordance with its own nature.

\paragraph{Nonduality}
First of all, Guenon points out that it is sufficient to establish that free will is a \emph{possibility}, since the possible and the real are metaphysically identical. Universal Possibility is beyond Being, i.e., it is part of Nonbeing. Since Nonbeing is metaphysical Zero, the idea of unity is not applicable, so the term nonduality is more appropriate. Because there is no duality, there are necessarily no constraints. This proves that freedom is a possibility insofar as it results immediately from `nonduality’, which is obviously exempt from every contradiction.

Since Nonbeing can neither be determined nor determine itself, the absence of constraint can only result in “non-action”, called wu wei in the Tao Te Ching. This is the “freedom of indifference”, i.e., the sense of detachment, paradoxically, acting without acting. There is no “doer” to do something.

\paragraph{Unity}
Now that metaphysical freedom has been established as an abstract possibility, the next step is to establish it as a living reality. Any alleged scientific or philosophical proofs against free will are therefore necessarily flawed and are tantamount to squaring the circle or building a perpetual motion machine.

Being is One, i.e., metaphysical Unity, the first determination from the Zero of Nonbeing. As One, it is obviously not subject to constraint by anything else. The absence of constraint is what is meant by “freedom”. Hence, freedom exists in the domain of Being or manifestation. To say it another way:

\begin{quotex}
unity presents itself in a way as a specification of the principial ‘nonduality' of Nonbeing.

\end{quotex}
Freedom is therefore a possibility of being, that is, a possibility of manifestation. Since manifestation is the principle of multiplicity, each being is limited by the others; this limitation is a restriction on the freedom of each being. But to the extent that a being is unfree in any degree, it is not fully itself. A totally unfree being does not even have its raison d’être within itself, which is equivalent to saying that it is not even a real being.

On the other hand, the Unity of Being is the principle of freedom, so a particular being is free to the extent that it participates in this unity. In other words, it is freer the more unified it is in itself. Thus, even Buridan's donkey has some degree of freedom so it does not starve itself.

Universal Being cannot be determined, but determines itself. Hence, in manifestation, freedom necessarily operates in differentiated activity, unlike the wu wei of nonduality. In the human state, particularly, this is “action” in the conventional use of the word. The human being is conscious and creative, manifesting his possibilities. He is One, a Whole, insofar as he has a stable Self which is his principal of unity.

In our world, the human being is encumbered with constraints, both physical and subtle. As a body, he is subject to physical laws, and also to the needs of the vegetable and animal souls. As these constraints demand psychic energy, his inner life becomes multiple. It takes self-knowledge and efforts of self-purification, to maintain these inner demands in balance.

The final, and most difficult constraint to overcome, is ignorance. As a rational animal, there is ultimately one choice, the most important choice: the choice between good and evil. Since freedom is to act in accordance to one's true nature as rational, only choosing the “good” is a free choice. Choosing evil is unfree, since its foundation is ignorance.

\paragraph{Summary}
Once metaphysics is understood, all the popular arguments for and against “free will” turn out to be pseudo-problems. The naïve view is that free will means the ability to choose between two options. As we just demonstrated, freedom means the power to act from one's own essential nature, unconstrained. As such, some choices are free, others are not. Since the Self is transcendent to the physical world, the so-called laws of physics do not apply.

To become One, to be made whole, is to be totally free. We are born potentially free, but to be actually free is not so simple. It requires training. You may stumble upon a solution yourself, but usually you will need help along the way. This is the goal of initiation.

\flrightit{Posted on 2021-12-04 by Cologero}