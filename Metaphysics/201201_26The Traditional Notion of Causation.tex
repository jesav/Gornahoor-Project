\section{The Traditional Notion of Causation}

As a reminder to old readers, and a notice to the new, let us repeat Gornahoor's purpose. We claim that there was a \textbf{Primordial Tradition} of the Indo-European peoples, which manifested in the Vedic civilization, Ancient Greece, and Medieval Europe, \emph{inter alia}. We are not here to teach \textbf{Aristotle}'s four causes, which anyone can find on the Internet, but rather to demonstrate the intrinsic commonality of worldviews between the Middle Age and Antiquity, and ultimately ancient India. Specifically, the Medieval Tradition is closer to Pagan Antiquity than is any current form of neo-paganism.

First of all, all three worldviews held to some version of hylomorphism, as shown is these two sets of related terms:

\begin{enumerate}
\item Form, essence, idea, Purusha, potency 
\item Matter, substance, hyle, Prakriti, act 
\end{enumerate}

A cause is a principle on which something else depends for its being. Thus the cause has priority over the effect, not just temporally, but in the wider sense that we give it. Aquinas and Aristotle name four types of causes in two classifications, as shown in Table~\ref{tab:causation}:

\begin{table}[ht]
\centering
\begin{tabular}{cc}
\toprule
Intrinsic Causes &
Extrinsic Causes\\\midrule
Material &
Efficient\\\midrule
Formal &
Final\\\bottomrule
\end{tabular}
\label{tab:causation}
\caption{Classes of causation}
\end{table}

The extrinsic causes are exterior to the being, while the intrinsic causes are what make it what it is.

\begin{itemize}
\item\textsc{Efficient cause:} That by which the effect is produced. 

\item\textsc{Final cause:} That for which the effect is produced. 

\item\textsc{Material cause:} That out of which the effect is produced. 

\item\textsc{Formal cause:} That which makes the effect to be of a particular kind. 

\end{itemize}

In Man and his Becoming, Guenon associates \textbf{Purusha} with essence of form and \textbf{Prakriti} with matter or substance. He writes:

\begin{quotex}
The meaning of the word \emph{hyle}, in Aristotle, is exactly that of substance in all its universality, and \emph{eidos} corresponds no less precisely to essence regarded as the correlative of substance. Indeed, these terms, Essence and Substance, taken in their widest sense, are perhaps those which give the most exact idea in Western languages of the conceptions we are discussing [Purusha and Prakriti]. 

\end{quotex}
Again, we see the perfect correspondence between the Sankhya Hindu school and the metaphysics of Aristotle and Aquinas. In discussing the notion of cause, Guenon continues:

\begin{quotex}
The unmanifested state … is identified with \emph{Mula-Prakriti}, Primordial Nature: but in reality it is Purusha as well as Prakriti containing them both in its own undifferentiation, for is it cause in the complete sense of the word, that is to say both at one and the same time efficient cause and material, to use the ordinary terminology, to which however, we much prefer the expressions essential and substantial cause, since these two complementary aspects of causality do in fact relate respectively to essence and to substance in the sense we have previously given to those words. 

\end{quotex}
This is clearly a slip of the pen by Guenon and it reads the same in French. I'm sure he means “formal cause” where he wrote “efficient cause”, otherwise, the whole conception makes no sense. So, let us recap his renaming of the causes:

\begin{itemize}
\item Material Cause $\Rightarrow $ Substantial Cause
\item Formal Cause $\Rightarrow $ Essential Cause
\end{itemize}

Now we can recap the notion of causation, in the hope that readers will learn to experience and understand the world in this way.

The unmanifested is polarized (the Tao becomes two), the Purusha (essence) and Prakriti (primordial matter). These are the intrinsic causes of a thing: the formal or essential cause from Purusha and the material or substantial cause from Prakriti. Since no thing can create itself, there must be a connecting link, or extrinsic cause, between the idea and the thing, or potency and act. The Idea is, as Guenon points out, a possibility of being. There must be a Will, or efficient cause, to actualize the potential of the Idea. Since Will follows intelligence, there must be a reason, of final cause, for it to act.

\flrightit{Posted on 2012-01-26 by Cologero}

\begin{center}* * *\end{center}

\begin{footnotesize}\begin{sffamily}

\texttt{Charlotte Cowell on 2012-01-27 at 10:47 said: }

This may be tangential to your main point here, but it's in my head as I had a yoga class this afternoon. I often read comments by western esotericists warning of the dangers of the eastern physical training. I understand the grave dangers of trying to forge an indestructible `godlike' diamond body, but beyond that I have to say that I find yoga to be one of the most beneficial mind, body, soul practices it's possible to do, it is very balancing for the energy. I do not, however, tend to focus overly on the breathing, which is where I believe a lot of the contention comes from – at one stage I got to the point where I was regularly forgetting to breath during meditation, which got quite frightening at times – is this what people mean by the dangers? Also I heard of the unregulated kundalini awakening but I'm not sure it's possible to do this unless one has undergone extensive training/practice and is therefore `ready', what are you thoughts on this? I am not disagreeing, by the way, the most natural `tradition' for me has always incoporated the respective wisdoms of India and the western mysteries


\hfill

\texttt{Caleb Cooper on 2012-01-27 at 11:29 said: }

The cases you've heard of typically occur when someone takes up the energetic practices like Kudalinia Yoga or QiJong. They're often practicing recklessly on their own rather than under the supervision of an experienced practitioner, and through too much forceful will end up opening their third eye before they're ready. As long as you stay in flow rather than forcing things with the ego's will you shouldn't unleash anything it's not time to handle.


\hfill

\texttt{Charlotte Cowell on 2012-01-27 at 13:51 said: }

I see, thank you for drilling it down a bit! Kundalini yoga practice is quite specific, I have done it once, it's intensive, but superb from a meditative point of view and also agony on the stomach muscles! I went with a friend who had a classic yoga moment, asking me afterwards `I saw the light, did you see the light?!'. I had not seen it in the way she described (third eye), but funnily enough, despite being in a huge but jam-packed studio, with scores of people in there like sardines, literally just an inch apart, for the whole two hours I felt myself to be a personal circle of light several metres across, wtih loads of space inside. I didn't feel the need to keep going back or keep trying the serpent movements but I can see why and how it `works'.

\hfill

\end{sffamily}\end{footnotesize}
