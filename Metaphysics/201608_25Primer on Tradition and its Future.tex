\section{Primer on Tradition and its Future}

\begin{quotex}
The ignorant are humble since they know that they do not know. The intelligent are humble because they understand how difficult it is to know something. But the half-smart are so absolutely certain about everything. \flright{cjs}

\end{quotex}
\paragraph{Executive Summary}
In the \emph{Reign of Quantity}, \textbf{Rene Guenon} describes two opponents of Tradition. The anti-Tradition, since it is openly opposed to all traditional ideas, is easily recognized. However, the counter-tradition is a caricature, counterfeit, and deformation of Tradition, so that ``it can fool even the elect."

The purpose, therefore, of this essay is to point out the signposts of Tradition so that its counterfeits can be more easily identified. Julius Evola identified Traditional civilization as a civilization of space\footnote{\url{https://gornahoor.net/?p=12696}}. Thus it is considered, but modern standards, to be ``stable, static and ultraconservative". In contrast, modernity is a civilization of time, seeking ``progress", novelty, change. The counter-tradition adopts the values and vision of the modern world while using the language of Tradition.

Ergo, historical studies of a Tradition are ultimately irrelevant. History is Brownian motion, not leading anywhere. The goal is to find the escape: out and up.

Our purpose is certainly not quantity, but rather to identify the one, the two, or the three individuals who are really interested. If you are here, try to remember exactly why you were born and what might be your life's goal. Certainly it is not to become the best ping pong player at the Olympics.

The counter-tradition will reject a known Tradition, all the while fantasizing about some pseudo-tradition that never has, nor never will, exist. Yet, even if the exterior form of Tradition changes, our task remains the same, although, perhaps the language would differ somewhat. The task is still to be freed from our automatic responses, to truly be free. Few are those who are willing to make serious efforts in that direction. And even among those few, not many are serious.

\paragraph{Category Theory}
In order to remove emotional responses, which are a sure indicator of the counter-tradition, I will describe Tradition in terms of the mathematics of Category Theory – knowing the category tells you much about its individual members. As a category is an abstraction, Tradition never manifests as such, only traditions.

Mathematical theorems take this form: If a set of conditions hold, then some conclusion is true. For our purposes, we won't need to assume that the conditions hold, only that the syllogism holds. To eliminate or modify one or more of the conditions will most likely produce an incoherent or incompatible system. Nevertheless, it is common to pick and choose among the conditions, or else to accept conclusions without accepting the conditions that justify them.

The conditions, or axioms, cannot be proven empirically or logically. Rather they must be ``seen". A life of prayer, meditation, contemplation, and the development of detached self-awareness are necessary. Hence, we don't try to ``prove" them.

A Tradition has two aspects: the esoteric \& the exoteric. There is a horizontal isomorphism between the esoteric aspects of two Traditions. This represents the metaphysical teaching. Different traditions will have similar notions of Non-being, Being, the Absolute, the Infinite, Transcendence and the transcendentals such as the One, Truth, Goodness, and Beauty. There will be similar teachings on degrees of existence. There is a recognition that man, as he is born, falls far short of his potential and thus is in need of salvation or liberation. There will be a teaching on how to achieve that. Man's interior life is understood as multilayered, with various sheaths or souls. The macrocosm is reflected in his consciousness.

Thus esoteric practitioners of different traditions will have a common understanding, making dialogue between them possible and even fruitful.

Within a Tradition, there is a vertical mapping between its esoteric and exoteric aspects. This is a mapping between the metaphysical teachings of the esoteric aspect and the religious-theological teachings of the exoteric aspect. Religious language is in the form of imagery in space-time unlike metaphysical teachings. Exoteric teachings recognize this, since they admit that its sacred teachings can be understood on multiple levels, the literal being the lowest. For example, Valentin Tomberg describes the second birth of the Logos in the soul, or the esoteric meaning of the Garden of Eden story.

\paragraph{Fundamental Conditions}
The first condition is that Tradition comes from above, from a super-human, transcendent source, as a revelation. Thus, it doesn't arise ``from a people" like some folk dance, but rather it descends into a people, providing its spiritual formation.

Hence, no one ``creates" a Tradition, least of all neopagan re-enactors.

The esoteric and exoteric aspects point to the doctrine of the two worlds. In the very first chapter of Revolt, Evola describes this as the ``\emph{fundamental doctrine}". Moreover, he says, in order to understand both Tradition and the modern world, it is necessary to begin with this doctrine. Apparently many readers of Evola treat the book like a mystery novel so they read the end first to find out ``who dun it" while ignoring the clues at the beginning that would make it plausible.

Knowledge of these two worlds has always been present in every Tradition: viz., the World of Being and the World of Becoming. The inability to understand this distinction leads to many misunderstandings. The counter-tradition is seldom aware of this doctrine.

\paragraph{Manifestation}
The mapping of Tradition to its exoteric aspect is best understood as vertical causation, i.e., the descent of the Tradition from the World of Being to that of Becoming. Three things will result:

\begin{itemize}
\item The pure tradition will fall onto the two horizontal currents of material-physical and historical-psychological-cultural causes. What are called borrowings, mimicry, etc. in outward forms, myths, rites, and so on, are therefore perfectly normal. Pointing them out in an attempt to discredit a tradition makes no sense. 
\item Manifestation involves privation. That is, the features of the Tradition fail to manifest perfectly, leaving shortcomings in the exoteric forms. Hence, it is beside the point to point out historical or other shortcomings. In a decadent age, there will be many such shortcomings. 
\item The individual adherents of the Tradition form a statistical distribution, like, for example, a normal distribution or bell curve. Thus it is impossible to judge a Tradition fairly or adequately from its adherents because their own understanding and practices will vary widely. The Tradition is not judged by the individual, rather the Tradition judges the individual. 
\end{itemize}
In other words, when discussing Tradition, it is necessary to be able to distinguish between the Essential and accidental qualities. The essential qualities pertain to the world of being and the accidental, to the world of becoming.

In particular, the Traditional doctrine itself is independent of any organization or of those who claim, within that organization, to be upholders of the Tradition. Hence, the persistent criticisms of certain accidental manifestations are characteristic of the counter-tradition.

\paragraph{Tradition and the Political System}
The purpose of the Tradition is salvation (exoteric) or liberation (esoteric), without going into the specifics of those terms right now. Its purpose is \emph{definitely} not to support some political, racial, or secular goal as many seem to believe these days.

Rather, the first duty of the political system or power is to protect the spiritual tradition, allowing it to function in its roles as a means to salvation and as educator. The political power needs to maintain public order, protect the group against internal and external enemies, and support an economic order that ensures widespread prosperity. To accomplish those ends, the political power has wide latitude, as long as positive law is in conformance with moral and natural law.

The political power is responsible for matters of citizenship or immigration. If someone does not like some policy, for example, the blame goes to the temporal powers, not the spiritual authority.

Theocracy, patriarchy, hierarchy, transcendence, and sexual restraint characterize traditional societies.

\begin{itemize}
\item \textbf{Theocracy}. A traditional society is based first of all on spiritual unity. Secondary factors follow that. The attempts today to base a society on race or nation will fail, as long as the members believe any old thing. Spiritual Authority not based on force, which is the prerogative of the political power. 
\item \textbf{Patriarchy}. Societies used to be based on rule by fathers. A so-called ``mannerbund" of unaffiliated men is a mirage. 
\item \textbf{Hierarchy}. Due to the statistical distribution of individuals in the process of manifesting, there will always be variations in knowledge, skills, etc. The attempt to reject hierarchy in the name of equality is misguided, since hierarchies of one sort of another will always arise. The issue, then, is to determine the best hierarchy. 
\item \textbf{Transcendence}. A traditional society recognizes the transcendent source of its existence. Attempts to base a society on science or atheism will be ineffective, because they can't define any rational purpose. 
\item \textbf{Sexual Restraint}. By contemporary standards, traditional societies are considered ``puritanical" or ``up tight". This is a large topic which we've addressed several times. What the modern mind cannot discern is the extant that artificially induced sexual excitement serves as a form of political and economic control. 
\end{itemize}
\paragraph{The Revival of Neo-Paganism}
The counter-tradition denies the obvious examples of Tradition. In particular, these days, the Middle Ages are denied to be a Traditional civilization, e.g., by an Alain de Benoist\footnote{\url{https://gornahoor.net/?p=8547}} who, on the contrary, considers that civilization to be a disaster.

Au contraire, according to Guenon and Evola, it is normative for Europe and the modern world is the disaster. Nevertheless, Benoist's position has become an idée fixe\footnote{\url{https://en.wikipedia.org/wiki/Id\%C3\%A9e_fixe_(psychology)}} among otherwise intelligent men. They need an exorcism more than an argument.

The related part of that idea is the anachronistic tendency to criticize contemporary manifestations of the religion of the Middle Ages, as though they applied to that truly Traditional civilization.

Another ignorant criticism is to accuse the Medieval Tradition of some sort of cosmopolitanism. This disregards the pagan Stoics who first came up with the idea of the ``Brotherhood of Man", and especially Alexander who himself promoted a brand of cosmopolitanism.

The idea of a ``European" paganism is an anachronism. A universal doctrine or practice never existed. Europe never existed, except as a geographical designation. The pseudo-sophisticates of neo-paganism deny that the gods even exist, believing in that way that they can tame the world. True paganism was much more primitive. If you can find neo-pagans like those of Romans and Spartans\footnote{\url{https://gornahoor.net/?p=2024}} at war, let me know.

\paragraph{The Decline of Tradition}
This is the hard one to swallow. A civilization declines — and many today believe that the nations of Europe and North America are in decline — not because of a ``bad" tradition, but rather because the people themselves are in decline. The Tradition declines because its leaders decline.

Those seeking for some racial unity for Europe fail to realize that race is not a constant. As a part of the ``world of becoming", it is subject to the vagaries of time and lower forces. Evola tried to point this out in his theory of the spiritual races: a Swede of today has a quite different inner spiritual quality from that of a Viking a millennium ago.

Hence, any sort of mass movement will fail to achieve its stated goal. The future belongs to the few, not the many.

\paragraph{The Future of Tradition}
\textbf{Joachim of Fiore}, recognized by \textbf{Dante} as being in Heaven, intuited three ages: the Age of the Father, the Age of the Son, and the Age of the Holy Spirit. Without going into all the details right here, we can point out two of his errors which, when corrected, give his vision more cogency. First of all, his attempt to give a precise date to the Third Age was misguided. Such things do not follow quantitative or chronological time, but rather sacred time. One must be attuned to the signs of the time to determine when ages begin and end.

The other misunderstanding is to assume that Joachim was describing some automatic process of human spiritual evolution into higher states. This leaves out his prophecy that there will be a period of tribulation before the Third Age. In other words, the Age of the Son brings the current age to an end, marked by a time of desolation.

Hence, the Age of the Holy Spirit is actually the beginning of a new cycle, not the final stage of some evolutionary process. If the first age was that of the fathers, then of priests, Joachim thought the Third Age would be that of an elite corps of contemplative monks. Actually, it is more likely that that corps will consist of men and women active in the world, while transcending it.

\paragraph{Further Reading}
\textbf{Rene Guenon}: \emph{The Crisis of the Modern World}. An early book that explains Tradition in the West.

\textbf{Rene Guenon}: \emph{The Reign of Quantity and the Sign of the Times}. One of his last works, summarizing the obstacles to Tradition in the modern world.

\textbf{Julius Evola}: \emph{The Revolt against the Modern World}. Builds on Guenon's works, adding more specific details.

\textbf{Ananda Coomaraswamy}: \emph{Metaphysics}. This is a collection of his essays. \textit{Vedanta and Western Tradition}\footnote{See Section \ref{sec:VedantaWesternTradition} in this book.} provides a reading list for Westerners to master before they will be ready to study the Vedanta. In particular, \textbf{Dante} and \textbf{Meister Eckhart} are held out as the exemplars of what the European mind is capable of.

Of course, the primary source should be the sacred writings, and those of the saints, mystics, and esoterists of your own Tradition.

\flrightit{Posted on 2016-08-25 by Cologero }
