\section{Guenon and Islam}

Regarding \textbf{Rene Guenon}'s ultimate decision to follow a Sufi path\footnote{\url{https://gornahoor.net/?p=262}}, he wrote:

\begin{quotex}
Contrary to what takes place in `conversion', nothing here implies the attribution of the superiority of one traditional form over another. It is merely a question of what one might call reasons of spiritual expediency, which is altogether different from simple individual `preference'. 

\end{quotex}
It may be instructive to explore the reasons behind that ``spiritual expediency", which is not the same as psychoanalyzing Guenon from afar. No, it is a question only of intellectual reasons, not personal or psychological.

For reasons of spiritual expediency, Guenon followed a path beginning with the French Hermetic revival. He came to reject this, but not in toto, since he continued to refer to the writings of certain Hermetists. He then tried to influence Catholic circles. For example, Fundamental Symbols is replete with Christian symbolism. In Crisis of the Modern World, Guenon asserted that only a return to Catholicism could restore Tradition to the West. Eventually, however, Guenon was rejected by the Thomist philosopher \textbf{Jacques Maritain}, precipitating Guenon's break with this approach. Personal factors — the death of his wife followed by a trip to Egypt — provided the opportunity to adopt Islam totally. As we will see, certain characteristics of Islam fit perfectly into Guenon's idea of Tradition; we therefore suspect that he may even have generalized Islamic theology to provide his understanding of Tradition. Therefore, there is no surprise that most of his closest followers themselves adopted Islam.

\paragraph{Tradition}
We owe it to Guenon for delineating the features of religious traditions intellectually. This accounts for the variety of religious forms in a manner totally different from the historical, empirical, archaeological, empirical, and exegetical approaches appropriate to academic research. He identified several specific traditions, specifically, Taoism, Hinduism, certain pagan mystery schools, Medieval Christianity, and Islam.

Nevertheless, some traditions are more comfortable within the parameters specified by Guenon than others. In particular, Christianity, as conventionally understood, rejects certain aspects of it. The most prominent theologians favorable to Guenon — Jean Borella, Philip Sherrard, Seraphim Rose — eventually rejected the Traditional approach. Neo-pagan revivals, as long as they are bound to certain of Evola's heterodox notions, also reject some primary elements of Tradition, so are doomed to insignificance. Let's look at the lineaments of Tradition as Guenon outlined.

\paragraph{Categorical}
The fundamental point to grasp is that for Guenon, Tradition is categorical, not essential. That is, it categorizes traditions according to certain criteria, but is not a Tradition itself. Therefore, to call oneself a Traditionalist, ``at-large" as it were\footnote{\url{https://gornahoor.net/?p=728}}, represents a misunderstanding. Rather, one must follow a specific tradition, albeit understood within these categories. Under normal circumstances, that means a man follows the Tradition of his own place and time. Whether or not a specific religious form is a Tradition is a question decided by these categories.

\begin{itemize}
\item \textbf{Authentic}

A religious form is an authentic tradition if it is ``orthodox" in Guenon's sense. I.e., is it doctrinally consistent with the ultimate metaphysical principles? 
\item \textbf{Complete}

Does it have a complete understanding, or are some things lacking? This lack does not necessarily make it heterodox. 
\item \textbf{Living}

Since principles cannot be fully understood merely from reading and studying, continuously existing initiatory organizations are required to pass on that knowledge. A living tradition will have such groups; a religious form may have lost them, remaining just a shell of what it once was. 
\end{itemize}
Note that the Evolian categories have absolutely no relevance. Hence, his several attempts to label traditions as solar vs lunar, Nordic vs Mediterranean, Aryan vs Semitic, etc., add nothing to our understanding. They are not based on any metaphysical principles but derive from profane sources, primarily Bachofen and Weininger.

\paragraph{Esoteric vs Exoteric}
Traditional doctrines have two complementary aspects. Guenon writes (incidentally illustrating our point):

\begin{quotex}
Of all traditional doctrines, perhaps Islamic doctrine most clearly distinguishes the two complementary parts, which can be labeled exoterism (\emph{sharia}) and esoterism (\emph{haqiqa}). 

\end{quotex}
He defines them this way:

\begin{itemize}
\item The exoteric aspect is the ``great way", common to all. 
\item The esoteric is the inward truth, reserved to an elite. This is from the very nature of things, since not all men possess the qualifications required to reach knowledge of the truth. 
\end{itemize}
The exoteric part is a rule of action (laws, rites, etc.) and the esoteric represents pure knowledge. The esoteric must also include a way, or path, to achieve that knowledge. There must be no conflict between the exoteric and the esoteric; they look at the same thing from a different perspective. (The exoteric from the circumference, the esoteric from the center).

Unfortunately, what often happens is that the vulgar misinterpret the verbal formulations of the esoteric as a new doctrine, opposed to the exoteric. This causes strife and results in religious forms detached from any tradition; New Age movements are obvious examples.

\paragraph{Multiplicity of Forms}
This is a fundamental doctrine of Islam, which teaches that there have been several revelations made to mankind at various times. The Koran names 25 prophets (including Jesus), and admits many more. Hence the existence of multiple religious forms poses no threat to exoteric dogma. Exoteric Christianity regards those religious forms as false and Jesus as unique. Although we have provided several examples of a broader outlook, this is a hurdle many will find difficult to overcome.

\paragraph{Primordial Tradition}
Guenon claims that there was a Primordial Tradition, and subsequent religious forms derive from it. Guenon uses the doctrines of the Vedanta as the prime examples for Tradition; this is consistent with the belief of some Traditionalists that the Vedanta was the original revelation made to Adam. The idea of a Primordial Tradition is consistent with Islamic dogma.

Exoteric Christianity believes in a series of covenants made between God and Adam, Abraham, Moses, etc., but specifically excludes all peoples outside them. Furthermore, outside from a few writers, salvation is not the attempt to return to the Primordial state of Adam.

\paragraph{Ahistorical}
Tradition is ahistorical, that is, it depends on a revelation from above, not on any specifically historical factors. Once again, this is consistent with Islam, but not exoteric Christianity, which sees tradition as the outcome of specific historical events.

\paragraph{Final Revelation}
Islam claims to be the final revelation made to man, an idea also asserted by Guenon, although he would add ``in this cycle". This doctrine is difficult to justify solely from metaphysical principles, so we can explore the reasons. Guenon's justification is that it is the only tradition to arise within history rather than prehistory. Another factor is that previous revelations were given to specific peoples at specific times for their own benefit. Islam, on the other hand, claims to be universal, the final revelation for which every prior one was a preparation. This universality is affirmed by Guenon:

\begin{quotex}
If there is a tradition where questions of race and origin do not in any way arise, it is certainly Islam, which in fact counts among its adherents men belonging to the most diverse races. 

\end{quotex}


\flrightit{Posted on 2012-06-24 by Cologero }

\begin{center}* * *\end{center}

\begin{footnotesize}\begin{sffamily}



\texttt{Michael on 2012-06-24 at 16:20 said: }

Interesting post, but I (gently) disagree with some of its points. First, with regards to being ahistorical, it is true that Christianity (and Judaism) look back to specific historical events, but then so does Islam. It completely relies on the historic Mohammed as much of the teaching in Islam comes from the Hadith.

Catholicism sees the covenants as growing in inclusiveness. The covenants of Noah, Abraham, Moses, and David correspond to a covenant with a family, tribe, nation, and kingdom. The final covenant with Christ is a universal covenant that encompasses all people. Of course, people are free to reject the offer.

I can understand how Guenon was attracted to Sufism, but I have to agree with Tomberg that he was impoverished by his decision to reject Christianity.


\hfill

\texttt{satishapte05tish on 2014-03-26 at 06:36 said: }

Can you give reference of Rene's assement of Islam as Final revelation ?


\hfill

\texttt{August on 2014-03-29 at 05:46 said: }

@ satishapte05tish

`Studies in Hinduism'. Sophia Perennis,

Extract from Chapter 11 – Sanatana Dharma:

``It is also interesting to note that the Hindu and the Islamic traditions explicitly affirm the validity of all the other orthodox traditions, and if this is so it is because as the temporally first and the last in the course of the Manvantara they must to the same extent integrate – although in different modes – all the diverse forms that have arisen in the interval, so as to render possible the `return to origins' by which the end of the cycle will rejoin its beginning, whence, at the starting-point of another Manvantara, the true Sanatana Dharma will again be outwardly manifest."


\hfill

\texttt{Michel on 2014-03-30 at 14:02 said: }

Just to add a source that further shows Guenon's position on Islam being the final revelation, see here:

\url{http://www.studiesincomparativereligion.com/uploads/ArticlePDFs/353.pdf}

The article is a short excursion on the symbolism of the letter Nun. The letter resembles the lower half of a circle with a dot at it's center. It represents, in regard to macro-cosmic cyclic manifestation, the complementary Tradition that completes the Heritage of an entire Humanity; the first being the Primordial Tradition, symbolized by the rainbow, which resembles the upper half of a circle. The letter symbolizes the Tradition that is to act as the ark which would ferry one Humanity's cycle to another, after all the former's possibilities have been exhausted.

Combining the two halves of the circle, the former preceding the latter, the symbol of a full circle is formed, with a dot at its center. Islam(Ark), being the Seal of Prophecy represents the ultimate form of orthodoxy for this present cycle, just as the Hindu Tradition represents the most direct heritage of the Primordial Tradition (Rainbow).


\hfill

\texttt{Frater Iaomai on 2014-03-31 at 02:26 said: }

Many of us perpetually seek an ``esoterism" somewhere, someplace, ``outside" ourselves, outside our ``times"–perhaps in the distant past, or maybe in a far flung future. Seldom are we willing to witness, meet, or greet the ``esoteric", the ``transcendent", right here, right now, in our own very tradition, let alone our own very lives–be they what that are, or in whatever state they are. Something in us refuses to take what presently ``is" as an acceptable ``departure point", and we keep ``looking" for ``secrets", or ``keys". To attain ``initiation" we think thoughts such as, ``we must have the exact, and most timely `revelation'", ``be in accord with `logical' thinking", ``be `good'. or `ritually pure'", ``be in contact with a `real' teacher/guru", and a thousand other things….

No doubt, many of these elements are helpful, important, valid in a way, and useful….but, how often does our entry into ``tradition", our respect for the ``rules", become mired in the sands the ``letter of the law"? 

While we look at the Islamic theological doctrine that Islam is the ``final revelation", and anticipate certain temporal events in accord with the notion, do we likewise integrate into our world view, the Islamic esoteric teaching, that with EVERY human breath, the cosmos is created anew? This idea exists pretty much in all traditions–and so, the onus isn't Deity's alone, but part and parcel of mans' activity. 

How can anybody knowing esoterism ``wait" for something to happen?

How can anybody knowing esoterism ``wait" for ``God" to ``act in the world"?

One hardly even need consider ``esoterism" to discern from the Abrhamic revelations, even in their exoteric aspects, that man is to act as viceroy of Deity on earth, in the corporeal worlds. If things seem ugly, disgusting, and ``effed-up" here and now, no matter how the causal chain began, we have a DUTY by intrinsic nature to present the opposite–and that, no matter if our efforts are immediately effective or not. The martyrs of yesteryear were prominent because of their position in time, and have become known as ``saints"–today, the opportunity to become a true Saint increases, because the voice of the martyr no longer stands out–who today knows the names of any of those of the ``slaughter of the innocents"?–yet, these are those nearest the Most High–the most secret…most ``esoteric". Do you wish to be like those known well for ``fasting" while appearing in the synagogues? Or, rather those who ``pray in secret"? Whose ``reward" is in secret?

Now, those who investigate ``Tarot", ought see some commonality of symbols. Where does Tarot begin and end?

22 ``Keys", one of which is ``zero"….as Wirth suggests, put them in a circle, with ``The Fool" between 1 and 21, ``Magician" and ``The World". What does that mean? Infinite possibility, the ``ein soph", ``linking" some individual creative consciousness (Magician) with some ``omega point" (``World"/"New Jerusalem"), in an endless, non-repetitive Self-disclosure–an ``Ouroborus"–a beginning of a ``Manvantara" meeting the ``end" of its own ``Manvantara"–so to speak. 

In any event, ``desperation" is in Christian Orthodoxy a sin….perhaps the most mortal sin. I know this, not just by formal doctrinal or canonical familiarity, but because it has been my own most terrible sin, I know it most well, know its effects, and know its deadly poison, having been sufficiently stung, again and again. 

It is ``sinful" because it is juxtaposed against faith and hope, and even more so, joy.

It is sinful NOT because it offends any ``rules" or any Diety, who in no way needs our ``faith"; but rather because it crushes our virtue, heroism, and instinct to turn chaos into order, rough into hewn, ugly into beauty, violence into justice–it undermines the Deity in us, causing us to ``second guess" our roles, our functions, in consciously participating as ``here and now" agents in the ``Self-disclosure" of ``I Am". 

Christians are not subscribers to a ``belief"–but an eternal priesthood, a ``race"–all Christians are priests technically–all have a ``called' function, only some are specifically involved with sacramental function–but that doesn't let ANYBODY else off the hook! 

None of this contradicts the doctrine of Islam as ``final revelation" either–except to those engrossed in some fundamentalism (either Islamic or Christian). Schuon and others took the question up, from many angles, and it has been beaten like a dead horse. Yet, just because it is the ``final revelation", does not imply, as some seem to imagine, that this signifies it is the ``better", or ``most appropriate" revelation. Yes, it is the ``final revelation"–as Guenon said again and again, a ``recapitulation"–yet, if you ``got it" well before, or in another time or place, there is no need to ``convert" to the ``latest thing". That in no way denies the validity of Islam–or especially esteems it–but, only locates it in history, and civilization. 

And, beside all of that, we ought also never forget, that from an exoteric Western, or Catholic exegesis, Islam did not just ``suddenly" appear–but began in Genesis. Abram had two sons, Issac and Ishmael. Issac from Sarah, Issac leading to ``Israel", and Ishmael from Hagar, with Ishmael Yahweh promised to raise another ``nation"–traditionally treated as Islam, inevitably through the Quraish tribe–another ``revelation". Indeed, something similar happens in the gospel of John, with the Christ explaining that ``other" sorts of sheep exists ``out of this pen" (10:16).


\hfill


\end{sffamily}\end{footnotesize}
