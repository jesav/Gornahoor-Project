\section{Divine Journey Summary}

In \emph{Man and his Becoming}, \textbf{Rene Guenon} outlines a path to liberation. If you start at the beginning and work through it diligently with understanding, you might attain the Supreme Identity. Just as a master mechanic understands the workings of an automobile, the student must have a complete understanding of the inner workings of the human being. These are the main topics in the process, as illustrated in \emph{Man and his Becoming}.

\begin{enumerate}
\item The distinction between the true Self (Atman) and the Ego 
\item God (Brahman) resides in the human heart 
\item The process of manifestation and the play of the three fundamental forces 
\item The origin of thought, mind, the senses, and external organs 
\item The different states of the human being 
\item The five sheaths of the human being 
\item The celestial journey through the various spheres 
\item Union, which is the realization of the identity of the Atman and Brahman 
\end{enumerate}
\paragraph{Postmortem States}
These are the possible outcomes of man's postmortem states.

\begin{description}
\item{\bfseries Longevity} Indefinite prolongation of human individuality in the subtle state.
\item{\bfseries Path of the Ancestors (pitriyana)} After the resorption of human individuality, the being will pass into other states of individual manifestation.
\item{\bfseries Passive Liberation of path of the mystics} Jivatman unites with subtle body until the end of the world. It then unites with Brahman passively, without full realization.
\item{\bfseries Delayed liberation} Jivatman unites with subtle body. Liberation will be obtained, at the end of the cycle, by means of intermediate stages [conditioned postmortem states], and not in a direct and immediate way.
\item{\bfseries Liberated at death} Jivatman does not unite with subtle body, nor pass through other states, but is liberated at death.
\item{\bfseries Jivanmukta} Liberated in life.
\end{description}

\paragraph{Transhuman states}
Guenon describes multiple transhuman states on the path of the Divine Journey. This journey can be made during human life or possibly after. In the Vedic version described by Guenon, each stage may be associated with a god or deva. Unless you were born and raised in India, gods like Agni, Varuna, or Indra elicit no visceral reaction. At best, you can gain an intellectual understanding of what the various gods represent.

However, as Guenon points out, Dante's description is virtually the same. Do not be deceived by the lack of exact correspondences, which should not be expected in symbolic representations. For example, Dante encounters various personages instead of gods at the various spheres. The point is to understand the state of consciousness described. Guenon also points out that the angels represent higher stages of consciousness. For many living in the West, this may be a better path.

\paragraph{Ideal type}
In the Vedanta, the Yogi represents the ideal type. Since our goal is the restoration of Tradition in the West, a different model may be appropriate.

Our path is that of the \textbf{spiritual knight}, who is active in the world. This provides the opportunity to manifest all the possibilities of the Self. King David is one such model: he was a warrior and a poet (Psalms). He lived and loved. Work and love are what matters. What tasks are you called to do? Do you have a love, whether real or aspirational?

\paragraph{Individual Path}
Ultimately, we make the journey alone just as did the Knights of the Round Table, not to mention Don Quixote. There is no operating manual for the Divine Journey, since the specific path depends so much on the capabilities and possibilities of the individual. Oftentimes you don't even know if and when you are on the path. After some consideration, I decided that it is appropriate to use personal experiences to demonstrate this.

So I am doing an experiment using Bergson's and Tomberg's understanding of memory. The goal is not merely reminiscence, but the bringing of the past into the present. That will show the memory in a new light, with the highest being moral memory. The exercise of bringing the past into the present can be fruitful. Only then can you recognize the path you have been on.


\textit{From meeting notes of 12 XII 2022}

\flrightit{Posted on 2022-12-14 by Cologero}
