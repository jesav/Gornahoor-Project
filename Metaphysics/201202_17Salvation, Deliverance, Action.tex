\section{Salvation, Deliverance, Action}
\begin{quotex}
Deliverance, together with the faculties and powers which it implies by superaddition (since it encompasses all states), but which must only be considered as accessory and even accidental results and in no wise constituting a final goal in themselves. Deliverance can be obtained by the Yogi with the help of the observances indicated in \textbf{Patanjali}'s Yoga Sutras. It can also be favoured by the practice of certain rites, as well as of various particular styles of meditation. It must be understood that all such means are only preparatory and have nothing essential about them.

\end{quotex}
Guenon points out that a man can acquire true Divine Knowledge even without observing the prescribed rites. The Vedas compare the functions of these rites to a saddle-horse, which helps a man reach his destination more easily and more rapidly, but without which he is able to reach it all the same. Obviously, a sane man gets off his horse when he has arrived at the destination.

The acquisition of higher states still does not constitute realization, because even though they may be higher than human individuality, they are still finite in relation to the Infinite. Nevertheless, they may be helpful, as we see them pursued in Buddhism.

\begin{quotex}
In Highest Yoga Tantra there are special subtle minds, normally of no help to an individual, which become aids on the path to Buddhahood when they are generated in meditation for the purpose of realizing emptiness. \flright{\textsc{Jeffrey Hopkins}, \textit{Meditations on Emptiness}}

\end{quotex}
Deliverance implies a perfect knowledge of Brahma, beyond Being and beyond all distinction. Deliverance and total and absolute Knowledge are truly but one and the same thing. Moreover, Knowledge, unlike action, carries its own fruit within itself. That is because Knowledge is both the means and the end for deliverance. Action, including rites, is not opposed to ignorance and thus cannot remove it. It may serve as a preparation, usually by quieting the flow of discursive thought either through meditation, or by the attempt to exhaust the mind, as in more extreme practices. Guenon explains:

\begin{quotex}
Action, no matter of what sort, cannot under any circumstances liberate from action; in other words it can only bear fruit within its own domain, which is that of human individuality. Thus it is not through action that it is possible to transcend individuality.

\end{quotex}
This is not to say that the consequences of action are restricted to the corporeal, or gross, realm, but also to the subtle realm. Even so, the subtle realm still is part of human individuality. Nevertheless, Lama Anagarika Govinda can still speak of the path of action. He writes:

\begin{quotex}
Every individual represents a certain position in space and time. Even if his consciousness has become all-embracing, by breaking down all limitations, it retains the character of its position as a particular centre of experience. This explains why each Buddha preserves his particular character. In this sense, individual character is not a karmic bondage. In an Enlightened One, the conflict between law and free will does not exist any more, because in the light of full knowledge, his own will and the laws governing the universe coincide. One's own nature proves to be a modification and conscious embodiment of universal law.

He is no more a slave to law, but its master, because he has understood and realized it so profoundly. Through knowledge we master the law, and by mastering it, it ceases to be necessity, but becomes an instrument of real self-expression and spiritual freedom. 

\end{quotex}

\flrightit{Posted on 2012-02-17 by Cologero}

\begin{center}* * *\end{center}

\begin{footnotesize}\begin{sffamily}

\texttt{Charlotte Cowell on 2012-02-17 at 06:11 said: }

I like the `burning house' analogy – I experienced this for many, many years, but I'm not quite sure yet if I escaped or died lol as the effort made me paranoid!!! I guess there has to be SOME time for relaxation….I'm put in mind of VT's comment in (I think) his letter on The Devil, where he says that there must be moments when even `front line' workers have respite from demonic forces. Either way I think I need a holiday :-)


\hfill

\texttt{escapee on 2012-02-18 at 21:45 said: }

“But all these states are eternal with respect to the state of manifestation, but not forever. At the end of the cycle the being will pass onto a new state.”

Without doubting your understanding of the ideas, I humbly suggest that the above expression be rectified by substitution of `eternal' with `perpetual'. 

`Eternal' should not be used, even in a relative way, to denote any state that is subject to duration or succession.


\hfill

\texttt{Cologero on 2012-02-19 at 01:53 said: }

Yes, eternal is the wrong word, but perhaps so is perpetual. Guenon himself writes (in Man and his Becoming, as well as in Initiation and Spiritual Realization):

“Salvation is properly speaking the attainment of the Brahmaloka … This accords perfectly with the Western conception of immortality which is simply an indefinite prolongation of individual life, transposed into the subtle order and extending to the pralaya.”

Brahmaloka refers to translunar states, but not the Supreme Identity (with Brahman). These are subtle states, hence not subject to space, but subject to time.

So, it would seem that the differences between eternal life, perpetual life, and an indefinite prolongation of individual life are not so obvious.

\hfill
\end{sffamily}\end{footnotesize}
