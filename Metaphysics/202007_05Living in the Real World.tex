\section{Living in the Real World}

\begin{quotex}
I turned myself to other things, and I saw the oppressions that are done under the sun, and the tears of the innocent, and they had no comforter; and they were not able to resist their violence, being destitute of help from any. And I praised the dead rather than the living: And I judged him happier than them both, that is not yet born, nor hath seen the evils that are done under the sun. \flright{\textsc{Ecclesiastes 4:1-3}}

\end{quotex}
\paragraph{The Three Worlds}
Tradition understands three worlds:

\begin{itemize}
\item The angelic or intelligible world, called the causal or Sivaloka in Hinduism 
\item The celestial, heavenly or mixed world, called subtle or Antarloka 
\item The sublunary or sensible world, called Bhuloka 
\end{itemize}
The sensible world is a world of darkness. It is the least real, i.e., it has less being, than the other worlds. It is also the source of multiplicity. As Thomas Aquinas stated, \emph{numerus stat ex parte materiae}, number comes out from the material part. Rene Guenon mentioned the consequences:

\begin{quotex}
Matter is essentially multiplicity and division; and this is why all that proceeds from matter can beget only strife and all manner of conflicts between peoples and between individuals. The deeper one sinks into matter, the more the elements of division and opposition gain force and scope; and, on the other hand, the more one rises towards pure spirituality, the nearer one approaches to that unity which can only be fully realized by consciousness of the universal principles. 

\end{quotex}
\paragraph{Corporeal World}
The corporeal or sensible world is that which is perceived by the senses. It integrates quantity from matter, or substance, with qualia from essence. It is considered to be the lowest of the three worlds. This is described in Hinduism as:

\begin{quotex}
The physical plane, or Bhuloka, is the world of gross or material substance in which phenomena are perceived by the five senses. It is the most limited of worlds, the least permanent and the most subject to change. 

\end{quotex}
Since it is impermanent and subject to change, our knowledge of it is merely opinion, or the most likely story. Therefore, attempts to found our knowledge on science will always be tentative, incomplete, and subject to revision. That is why debates about alternative scientific views, such as heliocentrism vs geocentrism, are ultimately irresolvable, assuming of course that the presentations are coherent and free of contradictions. Hence, it is unwise to base a worldview solely on science.

Some physicalists will dispute that the senses provide an accurate representation of reality. How would they know? Can they compare a subjective representation against ``external reality" and give it a score? Of course not. The material part of the sensible world provides the context for sense experience, although the qualities themselves don't emanate from matter. Rather, they arise from essence of the thing.

If the various experiments of physicalists about consciousness show some anomalies, that just proves the point: the sensible world is unreliable and no more than a matter of opinion.

\paragraph{Size of the Universe}
All the whining about the age and size of the universe are nothing but sentimental intrusions into the facts. The existentialists will opine that the world is meaningless since the earth is ``insignificant" among so many stars and period of time.

Adam in \textit{Paradise Lost} wondered about the same thing, so it is hardly a new observation.

\begin{verse}
When I behold this goodly Frame, this World\\
Of Heaven and Earth consisting, and compute,\\
Their magnitudes, this Earth a spot, a grain,\\
An Atom, with the Firmament compared\\
And all her numbered Stars, that seem to roll\\
Spaces incomprehensible merely to officiate light\\
How Nature wise and frugal could commit\\
Such disproportions
\end{verse}
The \textbf{Weak Anthropic Principle} relates the mass and the age of the universe. Its definition is:

\begin{quotex}
The observed values of all physical and cosmological quantities are not equally probable but they take on values restricted by the requirement that there exist sites where carbon-based life can evolve and by the requirement that the Universe be old enough for it to have already done so. 

\end{quotex}
The size and age of the universe are related by the velocity of light, according to the theory of relativity. Just multiply the time by the velocity of light in order to get the distance. If an extended time period is necessary for the appearance of intelligent life, then the size of the universe would need to be correspondingly large (assuming of course that the Universe had a beginning).

The total mass of the universe can also be estimated. For example, if the universe contained just the Milky Way galaxy with $10^{11}$ stars, then the universe would stop expanding after one month, not long enough for planets and life to appear. Since God has determined the mass of the universe, we are assured that the size of the universe was chosen so that human beings can appear.

From the metaphysical perspective, everything necessarily manifests at the time and place appropriate for it; it is not a random event.

\paragraph{Animal Life}
\begin{quotex}
The nature of animals was not changed by man's sin, as if those whose nature now it is to devour the flesh of others, would then have lived on herbs, as the lion and falcon. … Thus there would have been a natural antipathy between some animals. \flright{\textsc{Thomas Aquinas}}

\end{quotex}
Since Life is one of the conditions of manifestation, Plant and Animal life are also part of the sensible world. That means they are wholly natural beings. Plants have a vegetative soul, or etheric body, and animals have that as well as a sensual soul. Besides the powers of plant life, animals add emotional life to being.

\paragraph{Lemurians}
\begin{quotex}
The subtle plane, or Antarloka, is the mental-emotional sphere that we function in through thought and feeling and reside in fully during sleep and after death. It is the astral world that exists within the physical plane. The astral plane is for the most part exactly duplicated in the physical plane. 

\end{quotex}
A human being is a ``rational animal". That means he is both an animal, yet different from, and more than, an animal because of the rational or intellectual soul. The rational animal is not a type of animal, as, for example, a vertebrate is one type of animal.

Rational does not simply mean ``intelligent" because even the animals display some intelligence, or the ability to solve sudoku puzzles. Rather it means the ability to choose rational ends, or the good. This entails having an ego or sense of ``I". From the natural Darwinian point of view, that makes no sense. There is no way to conceive how abilities like the understanding of mathematics or the choice of good and evil could have evolved. At most, all such behaviours should be explained by evolutionary psychology. But even that won't do, since as the physicist \textbf{Roger Penrose} demonstrated, the mind is not a computable process.

Nevertheless, physicalists and naturalists insist that the human being must be understood by science. The powers and abilities listed above are therefore considered to be just complex animal behaviours. So, on those premises a human being is metaphysically just an animal. For various reasons, let us use the term \textbf{Lemurian} to designate a being that in appearance looks human, but interiorly is just an animal. That is, a Lemurian has a physical body, etheric soul, and sensual soul, but not an intellectual soul nor a consistent sense of I.

This is compatible with some traditional and esoteric ideas that see the Lemurian as a stage in the development of a human being, properly so called. Even some contemporary philosophers assert the existence of such beings, at least in the past. But this is certainly a very constrained understanding of the human being. Nevertheless, the emotional soul today is almost universally considered to be what defines the human being. You may often hear talk about how animals behave almost ``human", when the proper interpretation is to note how ``animalistic" — not always meant in a bad way — humans behave. Emotional life is considered quintessentially human.

\paragraph{Human Beings}
\begin{quotex}
But you are come to mount Sion, and to the city of the living God, the heavenly Jerusalem, and to the company of many thousands of angels \flright{\textsc{Hebrews 12:22}}

\end{quotex}
In the Heptaplus, \textbf{Pico della Mirandola} offers an interpretation of Genesis 1 as it applies to the human being. Heaven refers to the rational soul and Earth to the physical body. But there is more. Below the rational soul, there is the sensual soul which is shared with the brutes, and above reason, there is an intelligence which is shared with the angels. These are separated by the waters. So the human lives fully neither in the corporeal world like the animals, nor in the intelligible world like the angels, but in the celestial world which is a mix of the darkness of the corporeal world and the light of the angelic world.

The Sun, moon, and stars refer to different light sources. The Sun represents dianoia, the knowledge of the rational soul, and the Moon is the light of doxa, that is, mere opinion. Pico explains:

\begin{quotex}
While we wander far from our fatherland and live in the night and darkness of this present life, we make most use of the part of us which is turned toward the senses, and hence \emph{we believe more than we know}. 

\end{quotex}
The Sun is the light of day and the moon of night. The stars are auxiliaries such as ``the powers of combining and dividing, of reasoning and defining, etc." This is Moses' teaching on the nous. He next turns to the seats of anger (\emph{thumos}) and wantonness (\emph{epithumia}).

The brutes in Genesis then represent these inner aspects. Some desires pertain to the body and others to the imagination. When the desires for food and sex exceed what is required, they are used solely for pleasure. Paul said ``make not provision for the flesh in its concupiscences" (Romans 13:14). These are represented by cattle and wild beasts which dwell on the earth.

The waters then pertain to the imagination and its impulses toward fame, anger, revenge and related feelings. When revenge is the urge for justice or fame for justified honours, these impulses are necessary and useful. However, when they become excessive, we turn something good into evil.

We then understand that dominion over the beasts means that Reason needs to dominate the senses and the imagination. ``And man when he was in honour did not understand; he is compared to senseless beasts, and is become like to them." (Psalms 48:13)

\paragraph{Act and Potency}
Essence in Aristotelean terms is Act, what we are meant to be, and Potency is our material and corporeal condition. This means that we are meant to actualise our potentials in this earthly life. These are the stages:

\begin{itemize}
\item Our sensible life is multiplicity, so we have multiple I's each claiming to be real and each sapping our psychic energy. 
\item Our rational soul needs to dominate all those pretenders and develop as stable self. 
\item Then it may be possible to develop a higher self that knows the intelligible world of ideas in the Mind of God. 
\end{itemize}
As much as we are attached to the glamour of the sensible world, it is not our home. Our home is not to live amongst the brutes but rather amongst the angels.

\paragraph{Real Life}
\begin{quotex}
God is light, and in him there is no darkness. If we say that we have fellowship with him, and walk in darkness, we lie, and do not the truth. But if we walk in the light, as he also is in the light, we have fellowship one with another, and the blood of Jesus Christ his Son cleanseth us from all sin. If we say that we have no sin, we deceive ourselves, and the truth is not in us. If we confess our sins, he is faithful and just, to forgive us our sins, and to cleanse us from all iniquity. If we say that we have not sinned, we make him a liar, and his word is not in us. \flright{\textsc{1 John 1:5-10}}

\end{quotex}

\hfill

\paragraph{Acknowledgements}
\emph{The Anthropic Cosmological Principle} by \textbf{John Barrow} and \textbf{Frank Tipler}

Triloka: The Three Worlds\footnote{\url{https://www.hinduismtoday.com/modules/smartsection/item.php?itemid=3432}}

\emph{Heptaplus }by \textbf{Pico Della Mirandola}

\begin{quotex}
In Lemurian times men walked the earth who had only a physical, and etheric and an astral body. But these were not men who could think, in the sense of today, or who could develop humanly — in the sense of today. \flright{\textsc{Rudolf Steiner}, \emph{Spiritual Hierarchies}, Lecture VII}
\end{quotex}

\flrightit{Posted on 2020-07-05 by Cologero }
