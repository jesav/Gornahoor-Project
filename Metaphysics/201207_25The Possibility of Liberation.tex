\section{The Possibility of Liberation}

\begin{quotex}
Past performance is not necessarily indicative of future results. Nevertheless, it is much more reliable than wishful thinking. 

\end{quotex}
In The Multiple States of Being, \textbf{Rene Guenon} explains Traditional doctrine in Western terms, free of religious symbolism and Sanskrit terms. The universal Whole, or Absolute, can also be considered from the perspective of Possibility; thus it is both One, from the perspective as the Whole, and Infinite, from the perspective of Possibility. This is because no boundary can be set to what is possible (all things are possible with God); the impossible simply does not exist, hence it cannot limit anything.

\paragraph{Possibilities in Experience}
It is imperative to understand that Guenon is not proposing another philosophical system; rather, he is describing, insofar as it is possible in words, the fundamental nature of ultimate reality. This is the necessary foundation of any such system, or proper view of the world. As such, a man is not convinced of it by some external evidence, as in science, or by a proof, as in mathematics. Rather, it must be grasped in its immediacy, in direct intuition, as a whole.

If we judge by Guenon's letter to Evola\footnote{\url{https://gornahoor.net/?p=4519}}, it appears that \textbf{Julius Evola} was unclear about what Guenon called ``intellectuality", and could only see in Guenon a form of ``rationalism". I suspect quite a few readers of this blog are equally unclear. As a pedagogic aid, rather than focusing on the words, try to observe what is described directly. Ideas, or possibilities, arise in the mind. Observe how they arise, such as certain thoughts in similar circumstances. See how one leads to another similar to a meshed chain. Observe if they provoke emotional reactions; if so, that is an indication of bondage. This applies also to ideas that ``pump you up"; it is still an addiction.

Observe which thoughts are real possibilities of manifestation, and which are idle fantasies or wishful thinking, that could never manifest in any possible world. Of those few good thoughts, determine which of them will lead directly to action which, after all, means to bring the Potential into Act. With some serious self-reflection, honestly evaluate if you are living up to your true potential or selling yourself short. This is a painstaking process, requiring the courage to face up to the hellish aspects of one's being. When conscious attention is lacking, the mind is taken over by telluric and lower forces of disorder. Through many efforts, a man can bring his mind closer to a state of order. But first he must be perfectly clear, through study and instruction, about what constitutes order and disorder. Guenon refers to the ``Christ principle", or Logos, as the force of order.

\paragraph{Essence and Existence}
Possibilities can be those of non-manifestation or manifestation, and the latter can be Potential or Actual, as shown in this diagram:

\begin{itemize}
\item Possibilities of non-manifestation 
\item Possibilities of manifestation 

\begin{itemize}
\item Potential 
\item Actual 
\end{itemize}
\end{itemize}
Guenon relates this to Essence and Existence\footnote{\url{https://gornahoor.net/?p=1917}}, a topic we have covered many times, including discussions related to Evola's \textit{The Individual and the Becoming of the World}.

Actualized possibilities are made manifest when time and conditions are proper for them, and they are compatible with all other manifested possibilities, since existence is an interconnected system. Potential possibilities are not manifest either because they are incompatible with existence or there is a lack of power to bring them into manifestation. This latter situation is called a privation, since it belongs to the essence of the Being, but not to its existence.

\paragraph{Salvation}
Now a Being is manifested in an indefinite number of possible states; anyone reading this is most likely a Being in the human state. The totality of a Being will include both potential and actual possibilities. The human state is an individual state and, as such, a man has many possibilities open to him. I am considering here just those possibilities available as a man in the human state. Man actualizes himself in the world, taking into account his time and circumstances. However, as long as he is unable or unwilling to transcend the human state, he will be bound to the ways of the world. Nevertheless, by transmuting, transcending, and overcoming the negative forces in the world that seek limit his possibilities, he can achieve ``salvation", freedom within the world and the indefinite prolongation of the possibilities of the human state. This is no small matter.

\paragraph{Deliverance or Liberation}
Unsatisfied with ``salvation", a small number of men will seek instead to deliver themselves from all attachments while in the human state. This involves not just actualizing the possibilities of the human state, but also the possibilities of all states of the Being. Such a being is liberated, or delivered, from all restrictions and privations. In other words, for that being, essence and existence are identical. There can be only one such being, hence the formula Atman is Brahman. In the Western Tradition, that is what is known as ``God".



\flrightit{Posted on 2012-07-25 by Cologero }

\begin{center}* * *\end{center}

\begin{footnotesize}\begin{sffamily}



\texttt{Andrew on 2012-07-26 at 23:14 said: }

This should answer the question from a previous post about the difference between salvation and liberation.


\hfill

\texttt{Andrew on 2012-07-26 at 23:15 said: }

I usually put it that salvation is salvation from death. Liberation is liberation from existence.


\hfill

\texttt{Caeliger on 2017-05-06 at 17:01 said: }

May I ask exactly what is intended by ``indefinite prolongations of the human state"? If I understand the notion correctly, it would imply that the saved man can, say, enter a world for any arbitrarily indefinite amount of time, become a thousand feet tall, turn his hair pink, and drink from oceans of chocolate milk, because such a thing, indeed any non-absurd state, certainly ought to be possible, even if this particular world with its conditions allows no such thing.


\hfill

\texttt{Cologero on 2017-05-06 at 20:33 said: }

You got that right, Caeliger.


\end{sffamily}\end{footnotesize}
