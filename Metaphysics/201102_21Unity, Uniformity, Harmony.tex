\section{Unity, Uniformity, Harmony}

In lieu of an essay today, we shall instead list some definitions and ideas adapted from various works of Rene Guenon. Once understood, the point of the inexistent essay will become immediately apparent. Note that some of the terms are not used by Guenon (although the concepts themselves are); this is to ensure consistency moving forward.

\begin{quotex}
As long as the world endures, there will always be irreducible differences. \flright{\textsc{R. Guenon}}

\end{quotex}
\begin{description}
\item[Whole ]

The Whole is logically anterior and independent of its parts. The Whole is a real principle of unity superior to the multiplicity of its parts. 

\item[Heap ]

A whole conceived as logically posterior to its parts is a whole whose existence depends on the condition of being thought of as such. 

\item[Convention ]

Any thought that does not express a reality, but is instead the result of mutual agreement. A Whole is real. A heap is conventional. Stopping at a red light rather than green is a matter of convention, although the concept of controlling traffic flow is real. The inability to distinguish the conventional from the real is the cause of most disputes today. 

\item[Multiplicity ]

Multiplicity is situated beneath the level of manifested existence, and represents the extreme opposite of principial unity. Manifested existences become progressively less qualitative and more quantitative as they gradually move away from principial unity. 

\item[Synchronic ]

Simultaneous manifested existences are related to each other hierarchically as they move away from principial unity. 

\item[Diachronic ]

Manifested existences are related to each other over the course of time as they move away from principial unity. 

\item[Uniformity ]

The attempt to create an artificial, or conventional, unity in manifestation. 

\item[Harmony ]

The reflection of principial unity in manifestation. 

\end{description}
\textbf{Some reflections.} Manifestation is on two levels: gross and subtle. Gross manifestation is further from principial unity than is subtle. Thus, for the individual, the Spirit (or Atman) is his principial unity. The proper hierarchical relationship, then, is spirit → subtle (soul) → gross (body). Then the being is in harmony.

Attempts to create a unity based on physical characteristics alone will necessarily fail. Examples are attempts to create a unity from genetic racial types. The next step is to create unity based on thought alone, such as liberal states based on mutual contracts or mutually agreed “propositions.” But these are necessarily conventional, not real. Only spiritual unity is a real unity. Thus we try to see how the Spirit manifests synchronically and diachronically.



\flrightit{Posted on 2011-02-21 by Cologero }
