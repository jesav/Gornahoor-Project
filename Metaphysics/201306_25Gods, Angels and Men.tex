\section{Gods, Angels, and Men}

Every now and then, it is necessary to go over some basic concepts, particularly because the way we use them is not necessarily how they are commonly used. The claim of metaphysics is that it leads to perfect gnosis. As such, it is not a theory, a set of beliefs, or mere opinion; rather, it sees things as necessarily true. Moreover, besides the formal knowledge of the ideas, principles, and concepts, a change in being is also necessary to turn that knowledge into gnosis. One “sees” that it is so; at some point, as that gnosis sinks in, it becomes impossible to see the world in any other way.

Few men concern themselves with issues of epistemology; that is, they seldom concern themselves with the reasons for their beliefs. A thought crosses their mind, they are pleased with it, probably it brings them esteem in their social circle, it provides a modicum of order to the unstoppable stream of consciousness, and it is imprinted with a strong emotional attachment. Hence, it must be true and they defend that opinion fervently. We metaphysicians, on the other hand, are not so sanguine. We realize how difficult it is to achieve even the beginnings of knowledge. We endure the hardships of the spiritual combat, clearing out mind of the debris of unsupportable opinions. The Muslim sage, \textbf{Suhrawardi}, says there are four categories of men who seek after knowledge. The task at hand is to move from (1) to (4).

\begin{enumerate}
\item Those who feel the thirst for knowledge and embark upon the path of seeking it. 
\item Those who have attained formal, or discursive, knowledge, but not yet gnosis. 
\item Those who have achieved gnosis or intellectual intuition, but not discursive knowledge. Many saints fall into this category, since they may not necessarily have the proper metaphysical instruction 
\item Those who have perfected discursive knowledge as well as gnosis and intellectual intuition. 
\end{enumerate}
\paragraph{Being and God}
This issue comes up often, so this would be a good time to go over it again. Guenon makes the claim that the primordial unity is universal being. As Being is the principle of any existing thing, this is understood as God in the Western classical tradition. Since it is a unity, it must be monotheism. This is precisely what is meant, and we reject any anthropomorphic god, or even a “personal” god. Unfortunately, to those ignorant of metaphysics, they presume these false and idolatrous notions of God.

To deny this principle, therefore, is to reject the unity of being. One alternative would be complete chaos, or an indeterminate universe like some enormous Schroedinger cat. That would be like the matrix, with the mind confabulating a coherent worldview out of the random universe, not unlike the mind during sleep fabricating dreams out of random neural impulses.

Or else, there are two or more irreconcilable principles with no unifying higher principle. An example is Manichaeism, with an infinite and pointless battle between good and evil. In ancient paganism, the gods and goddesses were independent, some giving their blessings to a project and others, curses. They themselves are not ultimate, but are subject to the Fates.

Now this understanding of God is common to all Tradition. For example, in the Aristotelian-Thomist tradition mentioned in the Round Table, God is understood as Being. The Muslim metaphysician, Avicenna, says substantially the same thing. God is the necessary Being, the only such Being, and all other beings are contingent. Now these days, you can easily run across web sites that try to demonstrate that Christians and Muslims do not “believe” in the same God. To the contrary, as was just shown, the understanding of God is basically the same, and could not be otherwise. The real question is whether Traditional Christianity is compatible with Protestantism. Even more so, is Mormonism incompatible with this view; they really believe in some sort of demiurge that they name god. Esoterically, one's allies are other than what seems obvious exoterically.

\paragraph{Angelic Intelligences}
As the gnosis of God was lost in the West and He became regarded as just another being or person, albeit of great powers, so too did knowledge of the angels. In large part this was due to the hyper-rationalism that discredited \textbf{Dionysius the Areopagite} and \textbf{Hermes Trismegistus} because of the allegedly later dates attributed to their writings; this was also related to the false doctrine of Sola Scriptura. The Hermetic tradition did keep it alive, and we see that also in \textbf{Valentin Tomberg}. Tomberg pointed to Rudolf Steiner primarily for his writings on spiritual hierarchies, but we will see what it would mean to restore knowledge of this realm.

The Muslim tradition, or at least one branch of it, did however keep this knowledge alive. They regarded Pythagoras, Hermes, and Plato, among others, as the bearers of an ancient, but still valid, tradition. \textbf{Avicenna}, for example, has a cosmology not unlike the one we indicated in Spiritual Beings\footnote{\url{https://www.gornahoor.net/?p=1941}}. For him, the process of creation is the same as intellection. Keep in mind that angels are pure intellects, hence, non-formal beings. As each level in the angelic hierarchy contemplates the ideas, the world is created until the physical world, which is the final stage in the actualization of possibilities.

Our experience of the ideas is obviously not through the senses, but rather as thoughts. Thus, when we observe the thoughts that appear in consciousness, we are contemplating along with the angels. We recognize how ideas in the fullness appear as constellations of related thoughts. The more detached we are, the higher we can rise to experience more and more subtle realms. Unfortunately, the way is usually blocked by the “spirits of the air”, that is, demonic intelligences that fill our minds with counter-creative ideas. This is why “watchfulness” is so necessary to achieve gnosis, so we can “see” the source of all those thought forms that influence and control our actions.

At a still lower level, the erotic and thymotic impulses work in consciousness. They each have an intellectual component, and thoughts related to these impulses form a very large part of our daily consciousness. Spiritual training is therefore focused on the observation of thoughts.

\paragraph{Man and Free Will}
\begin{quotex}
Omnipotence does not mean power to do absurdities. The compulsion of another's will is such an absurdity, and therefore no real omnipotence could force such a compulsion. Omnipotence is spiritual, and spirit acts not by brute compulsion but by knowledge and inspiration. \flright{\textsc{R. G. Collingwood}}

\end{quotex}
Man, as spirit, has free will. Higher intelligences, therefore, cannot “take over” a mind; however, a mind can latch onto a thought, harbor it, and freely make it a part of his own identity. Often after a tragic event, people will plaintively ask, “Why did God allow this to happen?” However, to forcibly prevent a free man from acting, is an absurdity, such as making a square triangle. He simply would not be what he is. The real task, then, is to increase knowledge and to be inspired by the higher intelligences.

Yet men are not equal in their intelligences. A hierarchical society will provide the visual guidance to those unable, or not yet able. For example, the Brahmin type is idealist and objective, while the Kshatriya is idealist and subjective. The former contemplates things objectively; the latter is oriented to action. Hence, his subjectivity is crucial when administering a state or conducting a war, since he needs to take sides. The vaisya is objective but materialist; hence he can work in science and industry, the business of the world. The shudra is subjective and materialist, hence, incapable of true self-government.

In an organic society, the various castes work in harmony until, through degeneration the lower castes will rebel against the higher. At our time, the highest caste is disorganized, and the levers of power are in the hands of the shudra. On their own, they cannot have a direct conception of God and higher intelligences, which is why societies ruled by shudra tend to be atheistic and oriented to the satisfaction of material desires, including base and perverse desires.

These are the clues to understand events from a transcendent perspective. The first task is to understand man and his types, caste being fundamental. When the characteristics of each caste are understood, events and the reasons for decisions and actions will become clear. Beyond that, one learns to see intuitively the various higher forces that strive to influence the world; these are generally of three types: constructive, preservative, or destructive. One will see how various thought forms take hold in men; it is still amazing how groups of men will suddenly latch onto a new idea at the same time. Their makeup predisposes them to accept it.


\hfill

References:

Nasr, Seyyed Hossein, Three Muslim Sages

Schuon, Frithjof, Castes and Races



\flrightit{Posted on 2013-06-25 by Cologero }

\ \begin{center}* * *\end{center}

\begin{footnotesize}\begin{sffamily}



\texttt{Ash on 2013-06-26 at 01:53 said: }

The relationship between knowledge and gnosis is interesting. The third type described here would likely be difficult for many of us to encounter precisely because we could not waste time on sophistry and intellectual pride on them. The holy fool is the most clear example – good luck trying to talk about “the essence of Being”, political ideologies, or Heidegger with them. While we try to think our way to Heaven, Heaven finds itself on earth in them and shakes its head in bewilderment.

The idea of the angelic intelligences is one I will have to ponder further. It seems to be a short step from “acting” spiritual entities to the “acting” God to the anthropomorphic God. Father Seraphim Rose's doctrine of the toll houses troubles some of his followers in a similar way…it seems to be too corporeal at least at first glance. I find it interesting that the personalization of God was referred to as idolatry. This leads to a topic I often wonder about: how exactly aspects of exoteric relation are brought into an esoterically minded person's life. Tradition, of course; but how do we approach the Bible, with its very personal God and tribal histories? Christ the Logos, yes; but what of Jesus the Teacher and Carpenter? As a Catholic, Cologero, I'm sure you have encountered this as well in some form. 

The way the demonic intelligences are discussed here remind me very much of how they are presented in Lewis' The Screwtape Letters, a book a highly recommend. Not a word wasted.

“When two humans have lived together for many years it usually happens that each has tones of voice and expressions of face which are almost unendurably irritating to the other. Work on that. Bring fully into the consciousness of your patient that particular lift of his mother's eyebrows which he learned to dislike in the nursery, and let him think how much he dislikes it. Let him assume that she knows how annoying it is and does it to annoy – if you know your job he will not notice the immense improbability of the assumption. And, of course, never let him suspect that he has tones and looks which similarly annoy her. As he cannot see or hear himself, this easily managed.” 

“Indeed the safest road to Hell is the gradual one–the gentle slope, soft underfoot, without sudden turnings, without milestones, without signposts,…Your affectionate uncle, Screwtape.”


\hfill

\texttt{Jacob on 2013-06-26 at 11:47 said: }

I have some trouble with the personal God concept as well. Hopefully, Cologero can help clarify for me. I know Charles Upton once pointed out that when we say something is non personal we equate it with a non intelligent force like Gravity. So he used the word Transpersonal, something transcend to personhood. That helps point towards an answer, but it's still very difficult for me to grasp, especially with the problems you point out, Ash. 

@Cologero and Logres: BM's Gnosis I should arrive today. I'm looking forward to reading it. I'll probably have to return with questions though.


\hfill

\texttt{Avery on 2013-06-26 at 12:29 said: }

At least in my mind's eye, the existence of angels makes perfect sense to me. Devils clearly exist (I agree that The Screwtape Letters, one of the best books of the 20th century, comes in handy here, and Evola makes a compelling but somewhat less poetic case in Ruins as well); devils are intermediaries, and it is odd to suggest the evil one can make intermediaries but God cannot; therefore angels should exist too.

G.K. Chesterton wrote, “The human race, according to religion, fell once, and in falling gained the knowledge of good and of evil. Now we have fallen a second time, and only the knowledge of evil remains to us.” It would be irresponsible to rely on one's sense of evil without enriching the much more important sense of good.


\hfill

\texttt{Avery on 2013-06-26 at 12:29 said: }

At least in my mind's eye, the existence of angels makes perfect sense to me. Devils clearly exist (I agree that The Screwtape Letters, one of the best books of the 20th century, comes in handy here, and Evola makes a compelling but somewhat less poetic case in Ruins as well); devils are intermediaries, and it is odd to suggest the evil one can make intermediaries but God cannot; therefore angels should exist too.

G.K. Chesterton wrote, “The human race, according to religion, fell once, and in falling gained the knowledge of good and of evil. Now we have fallen a second time, and only the knowledge of evil remains to us.” It would be irresponsible to rely on one's sense of evil without enriching the much more important sense of good.


\hfill

\texttt{Matt on 2013-06-26 at 17:28 said: }

Jacob,

I think what Cologero is getting at with the idea of God not being a person is a fundamental distinction between the classical theist tradition – part of Tradition itself – and what is known as “theistic personalism” which is advocated by thinkers like William Lane Craig, Plantiga, and Swinburne. Theistic personalism is also what most protestants believe and probably even what the average Catholic in the laity believes in – sadly.

The classical theist tradition affirms that while God is person-like, that does not mean God is tantamount to being a person jut like us, except with qualities to the maximal degree. Its the opposite with theistic personalists – when they say God is a person, they believe that God is a person just like us. It gets down to how they view the divine attributes. Classical theism views the divine attributes in an analogous way – there is something of the reality of God that is analogous to the will and intellect within us – whereas theistic personalists view God's attributes in a univocal way, so his personal attribute is just like our personhood, but to the maximal degree like all of His other attributes such as power (same as as ours but to the “max”), and goodnes (he is a moral agent that is maximally good instead of the Sovereign Good). The A-T philosopher Edward Feser, whom Cologero has mentioned a few times here, goes into detail about this fundamental difference and why theisitc personalism is so problematic. He makes the apt comparison of the theisitc personalists' god to how much of the historical greek religion viewed Zeus – just a being among a class of beings, though a very exceptional one (instead of the Being of beings, Being Itself). Feser also makes the similar comparison to Superman, which I think is also an appropriate one.


\hfill

\texttt{Jacob on 2013-06-26 at 19:38 said: }

Ok, thanks Matt. That helps a lot. Also explains why the Kalam argument theistic philosophers use so often grew out of a discipline most Muslims scholars looked down upon.


\hfill

\texttt{Matt on 2013-06-26 at 19:59 said: }

No problem Jacob.

How theistic personalists view God's attributes is the main reason why they don't accept divine simplicity, the convertibility of the transcendentals – the Oneness of God essentially. Since our attributes are metaphysically separate, and if God's attributes are just like ours, then God's attributes are therefore also metaphysically separate. They also think that God – classically/traditionally conceived – can not have accidental properties, but it doesn't seem like they've taken into consideration that God classically concieved can have what is called `Cambridge properties”. 

This has a number of logical implications – most of which I don't think have been considered by theistic personalists – in relation to the fundamental tenets of Christianity (and probably Tradition as a whole). If God really is indeed composed of parts, then one can't really say God is infinite (The Infinite is a better term) since God would now have intrinsic boundaries. And, at least to me, I don't see how the fundamental Christian tenet of the Logos as the common principle that unities all is compatible with theistic personalism.


\end{sffamily}\end{footnotesize}
