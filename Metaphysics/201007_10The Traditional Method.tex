\section{The Traditional Method}

In the Forward to \textbf{Revolt Against the Modern World}, Evola characterizes the ``Traditional Method" with two principles, which I schematize in Table~\ref{tab:TraditionalMethod}:

\begin{table}[h]\centering\small
\begin{tabular}{ccc}\toprule
&
\textbf{Perspective} &
\textbf{Principle}\\\midrule
Ontology &
Objective &
Correspondence\\\midrule
Epistemology &
Subjective &
Induction\\\bottomrule
\end{tabular}
\caption{The Traditional Method}
\label{tab:TraditionalMethod}
\end{table}
\paragraph{Ontology}
The Principle of Correspondence ``ensures an essential and functional correlation between analogous elements, presenting them as simple homologous forms of the appearance of a central and unitary meaning." 

This makes two claims:

\begin{enumerate}
\item There is a central and unitary meaning 
\item This meaning may appear differently in different circumstances and different times 
\end{enumerate}
That this is true can be seen in the use of languages. For example, ``dog" and ``perro" look different and sound different. Nevertheless, they both ``mean" the same thing. However, for someone who doesn't know English and Spanish, the two words (``appearances") appear unrelated. So the denial of this principle implies the absurd conclusion that languages are mutually unintelligible. 

In practice, this means that the source of events in the world must be traced back to its corresponding idea. Thus different myths, legends, and symbolic forms from disparate cultures can be shown to be expressions of the same idea. But just as I cannot ``prove" that dog=perro to someone ignorant of those languages, neither can Tradition be proven to someone ignorant of spiritual reality. 

\paragraph{Epistemology}
Principle of Induction is a ``discursive approximation of a spiritual intuition, in which what is realized is the integration and the homologous unification of the diverse elements encountered in the same one meaning and in the same one principle." 

Again, this contains two elements:

\begin{enumerate}
\item Discursive approximation 
\item Spiritual intuition 
\end{enumerate}
Although a metaphysician may express himself in speech or writing, this can never be more than a discursive approximation, since spiritual reality is ultimately ineffable and irreducible to objective language. However, what the metaphysician knows, he knows by direct intuition. So, in the case of myths, legends and other symbolic forms, one ``realizes" that they encompass a unitary meaning by a sort of clairvoyance or spiritual ``seeing". 

\paragraph{Revolt of the Spirit}
Via this ``traditional method", Evola will portray the world of Tradition as a unity and a universal type. Specifically, this means that the world of Tradition has manifested in various forms and can still manifest in a different form. The man of Tradition does not blindly desire to repeat the past — which ``past", since the universal type has manifested in different forms in the past. 

Furthermore, Evola will show that the idea or type of Tradition is ``capable of creating points of reference and of evaluation different from the ones to which the majority of the people in the West have passively and semiconsciously become accustomed". Specifically, this the idea that everything is ``text", can be written down and debated endlessly — a process that does not result in understanding but only aporia. This is the viewpoint of the modernists and those who think like them. 

Only by understanding the sense of the world of Tradition, only by intuiting the spiritual reality behind the world of appearances, then

\begin{quotex}
this sense can also lead to the establishment of the foundations for an eventual revolt (not a polemical, but real and positive one) of the spirit against the modern world. 

\end{quotex}

\paragraph{Tentacles, Paws, Arms, Wings}

Since Evola wrote a thorough book on the Hermetic Tradition, I will use an example from the Hermeticist Valentin Tomberg to illustrate the Principle of Correspondence. In Letter XIV on Temperance in \textbf{Meditations on the Tarot}, Tomberg writes: 

\begin{quotex}
Tentacles, paws, arms, wings — are they not simply diverse forms manifesting a common prototype or principle? 

\end{quotex}
In other words, Tomberg will show that they are homologous forms of a central and unitary meaning, which is precisely the Principle of Correspondence. He explains: 

\begin{quotex}
They are insofar as they express the desire to bear the sense of touch further, to be able to touch things more removed than those in the immediate neighbourhood of the surface of the body. They are \emph{active extensions} of the passive and receptive sense of touch which is spread out over the surface of the organism. In making use of them, the sense of touch makes ``excursions" from its usual orbit circumscribed by the skin which covers the body. 

\end{quotex}
Clearly this is absurd and incomprehensible to one-dimensional thinkers who live and move on a line. So how do we justify this epistemologically? As we have repeatedly mentioned, the sufficient reason of the world of appearance is the Will. Since we have direct experience of our own Will, we should understand how the Will strives to bring our ideas into manifestation. So it is actually the rationalist who is absurd, since he denies the existence and efficacy of his own will, attributing it to some external force, fashioned by electro-chemical activity in the brain. Thus he denies his own existence while pretending to exist. Let's allow Tomberg to explain:

\begin{quotex}
The organs of action are simply crystallised will. I walk not because I have legs but rather on the contrary. I have legs because I have the will to move about. I touch, I take, and I give not because I have arms, but I have arms because I have the will to touch, to take, and to give. 

\end{quotex}
The Will is creative. It takes the idea and brings it into the appropriate form. Tomberg makes this clear: 

\begin{quotex}
The ``what" [the idea] of the Will engenders the ``how" of the action (the organ) and not inversely. The arms are therefore the expression of the will to bear touch further than the surface of one's own body. They are the manifestation of extended touch due to the will to touch things at a distance. 

\end{quotex}

\flrightit{Posted on 2010-07-10 by Cologero }

\begin{center}* * *\end{center}

\begin{footnotesize}\begin{sffamily}



\texttt{James O'Meara on 2010-07-12 at 15:38 said: }

This is another one of your excellent and helpful explications. If I may make a contribution: a couple years ago I found a passage in one of the few books on Spengler in English, by H. Stuart Hughes, where it seemed like he was actually giving a good explication of Guenon's metaphysical [vs. systematic philosophy] method. I think it could apply to Evola's method as well:

QUOTE:

Spengler rejected the whole idea of logical analysis. Such ``systematic" practices apply only in the natural sciences. To penetrate below the surface of history, to understand at least partially the mysterious substructure of the past, a new method — that of ``physiognomic tact" — is required.

This new method, ``which few people can really master," means ``instinctively to see through the movement of events. It is what unites the born statesman and the true historian, despite all opposition between theory and practice." [It takes from Goethe and Nietzsche] the injunction to ``sense" the reality of human events rather than dissect them. In this new orientation, the historian ceases to be a scientist and becomes a poet. He gives up the fruitless quest for systematic understanding. … ``The more historically men tried to think, the more they forgot that in this domain they ought not to think." They failed to observe the most elementary rule of historical investigation : respect for the mystery of human destiny UNQUOTE

So causality/science, destiny/history. Rather than chains of reasoning and ``facts" the historian employs his ``tact" [really, a kind of Paterian ``taste"] to ``see" the big picture: how facts are composed into a destiny. Rather than compelling assent, the historian's words are used to bring about a shared intuition.

I suppose Guenon and Co. would bristle at being lumped in with ``poets" but I think the general point is helpful in understanding the ``epistemology" of what Guenon is doing: not objective [but empty] fact-gathering but not merely aesthetic and ``subjective" either, since metaphysically ``seeing" the deeper connection can be ``induced" by words and thus ``shared".


\end{sffamily}\end{footnotesize}
