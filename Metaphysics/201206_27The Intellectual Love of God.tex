\section{The Intellectual Love of God}

\begin{quotex}
A common man marvels at uncommon things; a wise man marvels at the commonplace. \flright{\textsc{Confucius}}

Men employ their reason to defend conclusions arrived at by reason, but conclusions arrived at by the passions are defended by the passions. \flright{\textsc{Benedict de Spinoza}}

\end{quotex}
Realizing that the drug culture was no path worth following, I turned to Hindu philosophy and learning meditation and yoga practices. \textbf{Alan Watts} was the bridge to this; the older brother of a friend of mine used to hand out Watts' \textit{The Book: On the Taboo against Knowing Who You Are}, which I eventually read. I read all his books, finding I preferred his early books, promoting a sort of Vedantized Christianity. That is where I first encountered the name of ``Rene Guenon". However, his books were unavailable and he was otherwise quite obscure. The name was filed away for quite some time.

Hence, I studied Vivekananda, Ramakrishna, Nisargadatta, Aurobindo, Maharishi Mahesh Yogi, Paramahansa Yogananda, as well as westerners such as Bubba Free John. I noticed that those around me would adopt Indian ways, such as manner of dress and diet. These superficial aspects were much easier than penetrating into the heart of the Vedanta. This was opposed to my goal of incorporating the metaphysics of the Vedanta into the West. I also reasoned along these lines, since it coincided with the beginning of the Neo-Christian revival. If I were an Indian, I would most likely be unable to distinguish these TV evangelists from the deepest of Western thinkers. So, of all the visiting Gurus, how could I tell the most authentic? I couldn't go to India to visit these men.

That is when I turned to \textbf{Benedict de Spinoza}. As someone who had written his 10th grade term paper on Russell's and Whitehead's \textit{Principia Mathematica}, the claim that the foundations of man, God, and the universe could unfold logically from first principles was quite appealing. I spent some time going through the Ethics line by line. This, in itself, was a transformative process. Obviously, no one today can agree on what Spinoza actually meant. As I am writing from memory, I won't try to defend my own rather mystical interpretation, although it does seem to have its supporters. Rather, the subjective impact is what matters.

First of all, Spinoza was admirable as a man, so I modeled some of his character. There was his commitment to logic and reasoning, but he didn't preclude a higher form of knowledge. He worked independently as a lens grinder. So I, too, valued my economic independence and sought a technical career. When a man is forced to deal with the material world, affect it, and transform it, he is usually protected from the airy impractical ideas of the so-called intelligentsia. It appealed to me, too, that Spinoza was described by some as an ``atheist", yet by others as a ``God-intoxicated man". That perhaps is as it should be.

If I was reading the Vedanta into Spinoza at that time, I probably now read Guenon into him. This is not without justification. Spinoza can be read as one of the early moderns, or else as one of the last medievals, incorporating the rich philosophy of Moorish Spain into his own system, or even a Cabbalist as some claim. The Italian writer, \textbf{Piero di Vona}, whose book Evola Guenon De Giorgio is an invaluable resource, claims that Guenon was influenced by Spinoza in significant ways. That Guenon was familiar with Spinoza is true enough, since he is frequently mentioned, but not in a positive way. Nevertheless, there are points of contact. Unfortunately, I haven't been able to locate di Vona's monograph on the topic for a full treatment of the question.

\textbf{Stanislaus de Guaita} was one of the leaders of the French Hermetic revival and was well known to Guenon. In The Temple of Satan, he wrote:

\begin{quotex}
It is the \textbf{High Science}, what Spinoza magnificently defined as regarding objects \emph{sub specie aeternitatis} [under the aspect of eternity]. 

\end{quotex}
So we see that that circle regarded Spinoza highly, and ``regarding objects \emph{sub specie aeternitatis}" is a fundamental point for Guenon.

In Spinoza's teaching, God is the fundamental reality, in which essence and existence coincide; this does not really differ from the Western understanding. God has an indefinite number of attributes, but only two of which — Thought and Extension — are known to us. But what we experience are not the attributes directly, but their modes, or modifications. Oddly, Thought and Extension have no interaction between them. Hence, we can know the world entirely through its material processes, or else through the interplay of ideas. Either would be correct.

In conformity with Western metaphysics and Tradition, Spinoza recognized three forms of knowledge; \emph{opinion}, or knowledge of sensual things; \emph{reason}; and \emph{intuition}, the direct grasp of things understood from God's perspective. There are three types of men corresponding to the three ways of knowing:

\begin{itemize}
\item the \textbf{ignorant }[ignarus], who only have opinions 
\item the \textbf{free man} [homo liber], guided by reason 
\item the \textbf{sage} [sapiens], who understands through intuition 
\end{itemize}
In this schema, ``reason", for most men, is a higher state of consciousness, well worth pursuing. Intuition does not contradict reason, but moves beyond it. This differs from New Age thinking which wants to dispense with Reason, hence intuition can mean any old thing. A woman recently told me she has ``hunches", which she regards as access to higher knowledge.

For Spinoza, God is known through the laws of Nature. Unlike common people, who can only understand God through the miraculous and the unusual, the sage knows him through regularity. The same applies to man, as Spinoza devotes much of the book to the understanding of the emotions or passions. Now a ``passion" is passive, something undergone. When a passion is understood intuitively, i.e., from the aspect of eternity, one is conscious of and above that passion. This is a skill to be learned: when faced with a strong emotional experience, say of Anger or Anxiety, try to remain conscious of that experience, detaching from it, watching it as one would the clouds passing overhead. The bad experience will often dissipate.

In the third state of knowledge, there is the emotional experience of the Intellectual Love of God. Unlike the passions, this is an active state of Joy. Such a being

\begin{quotex}
is hardly troubled in spirit, but being, by a certain eternal necessity, conscious of himself, and of God, and of things, he never ceases to be, but always possess true peace of mind.

\end{quotex}
This is the same as the experience of Brahman, or Sat-Chit-Ananda, or Being (God), Consciousness, joy or peace of mind.

Like Guenon, Spinoza says there are possibilities of manifestation and non-manifestation. What is possible must also be actual. Hence that which can be rationally conceived must also happen; this is consistent with the independence of the two attributes. But, on the other hand, not every passing thought represents a possibility of manifestation. This is more difficult to grasp than it seems at a quick look.

A very intelligent friend recently told me it is conceivable that a wealthy man, tired of life, would commit suicide; but prior to that, he would have put my friend in his will, thus making him rich. I challenged him, so he tried to put together a coherent narrative. Yet the devil is in the details, the holes in the narrative. A real possibility necessarily will manifest.

How can Spinoza be improved? Perhaps, we can regard the Attributes as gross and subtle manifestation. Then we could understand the other unnamed attributes as higher states of non-formal manifestation as Guenon does.


\hfill

Update:

In The Sacred Heart \& the Legend of the Holy Grail, Guenon writes:

\begin{quotex}
It is then said that the Grail was entrusted to Adam in the Terrestrial Paradise, but that at the time of his fall Adam lost it in his turn for he could not take it with him when he was cast out of Eden; and this also becomes very clear in light of what we have just indicated: man, separated from his original center through his own fault, found himself henceforth confined to the temporal sphere; he could no longer regain the unique point from which \emph{all things are contemplated under the aspect of eternity}. 

\end{quotex}
It seems that Guenon and Spinoza mean the same thing by that phrase.



\flrightit{Posted on 2012-06-27 by Cologero }

\begin{center}* * *\end{center}

\begin{footnotesize}\begin{sffamily}



\texttt{Aghorable on 2012-06-28 at 01:02 said: }

``Realizing that the drug culture was no path worth following, I turned to Hindu philosophy and learning meditation and yoga practices…"

Looks like someone didn't get the memo:



\url{http://en.wikipedia.org/wiki/Religious\_and\_spiritual\_use\_of\_cannabis}


\hfill

\texttt{Cologero on 2012-06-28 at 08:57 said: }

Au contraire, Aghorable, I have made the pilgrimage to Bob Marley's house in Kingston. The bullet holes are still visible in the wall. Jamaica is a very spiritual country.


\hfill

\texttt{Aghorable on 2012-06-28 at 18:57 said: }

The above are actually Hindu Sadhus making offerings to Shiva.

But it is quite interesting that you bring up the Rastafari movement, the pièce de résistance of the black man's charismatic cults. It's almost believable that you've made the pilgrimage, given the historical relations between the Ethiops and Italians; bickering like forlorn lovers last century! This area, with its actions and extensive reactions, could be fertile ground for a study of the occult war.


\hfill

\texttt{Izak on 2012-06-28 at 19:12 said: }

Actually — without wanting to derail this discussion altogether — are there any available discussions on the charismatic black movements, from a traditional perspective? You have the Rastafari, the 5\% Nation, the Nation of Islam, the Black Hebrew Israelites… all of these seem like various aspects of the greater counter-traditional distortion, but they're utterly fascinating nonetheless.


\hfill

\texttt{Cologero on 2012-06-28 at 23:12 said: }

Aghorable, you are struggling to make a point, but it is alluding you. Let's assume the goal is an ``intellectual conversion" or ``liberation" as understood by the Traditional writers. What is the message of the memo? Obviously, it is not sufficient, otherwise enlightenment would be quite common. But is it necessary? That would be a hard case to make. So, at best, it is somewhat of an optional distraction.

As for Ethiopia, it has an ancient culture, not well known outside its boundaries. Mussolini was so desperate to create an Italian Empire, he settled on Ethiopia. It was more difficult than expected, given the intrigues of the French and British. Since boys will be boys, he had to pass an anti-miscegenation law back home, in order not to upset his new alliance with the Third Reich.

I absolutely made the pilgrimage. I found myself in Kingston, for reasons I cannot make public. What other attraction do you think is there?


\hfill

\texttt{Cologero on 2012-06-28 at 23:25 said: }

Izak, I'm not aware that those groups have provided sufficient written documents, or, if they have, their categories of thought may not correspond to what is ``expected" in Guenonian terms. You neglected to mention Santeria, which is more common in my area. I have visited Botanicas, but not yet witnessed a ritual despite some inquiries. Would you compare these movements to white neo-pagan movements? Or are they of a totally different order?

P.S. I just looked up the 5\% group. Curiously, like Spinoza, they assign people into three similar categories, with their own notion of an ``elite".


\hfill

\texttt{Aghorable on 2012-06-29 at 00:54 said: }

Dear Cologero, read not too much into my `memo' comment, it was a humorous play following on the dissenting opinions of some readers concerning your firm stance on intoxicants, saying tongue-in-cheek that the pictured pursuant of Mok?a didn't receive the memo, not that you were failing to understand something. If you ask whether drug use is necessary in the pursuit of liberation, it most certainly is not; we know that the one indispensable criterion is the establishment of an effective initiatory connection to a living spiritual center, and even this necessity is not so rigidly specified that it has to happen in a comparable way for all aspirants.

But we can leave drug use to one side for now, it is very much a secondary matter that has been dealt with sufficiently here and in an excellent chapter of Cavalcare la Tigre. If anyone wants to develop dialogue on the matter, the forum is still available.

You will forgive our interpretation of sarcasm where there was none, in your comment on Bob Marley and Jamaica. I'm sure it wouldn't take us long to think of a few other attractions Kingston has to offer! To name just one; savouring a nice cup of Blue Mountain coffee. Were you received hospitably by the general population?


\hfill

\texttt{Cologero on 2012-06-29 at 14:16 said: }

Kingston is divided into two parts. Nestled in the hills are expensive houses, while down below there is the core of the city. We made a motor trip up the mountains, but had to turn back when the road was blocked. There were no Starbucks along the way. After that, I was not in the mood for coffee.

My hostess lived in a large home high up the hill with an armed security guard watching the driveway at all times. Whenever we got near the ``general population", we were asked to raise up the windows of the limo and lock the doors. Some of the ``general population" were kind enough to wash the windshield at red lights, but they got no tips because no one dared open up a window.

Probably not the best way to experience Jamaica, but I got an inkling as to how our government thinks.


\hfill

\texttt{Izak on 2012-06-30 at 15:44 said: }

Cologero – I'd only compare them to the quasi-nationalist white pagan movements, although they must be fairly benign, since modern society treats them with plenty of kindness. My only interest is really in the more militant black spiritual movements, and they all appear to be personalists who believe in a single God (and therefore absorb the Semitic tradition), whereas the more militant white pagan movements appear to be impersonalists who believe in some sort of natural force whose various aspects are represented by different Gods in old myths, and they reject the Semitic tradition. They all appear to see Christianity as a grotesque aberration from true traditional practice. But when I say all of this, I'm speaking as a complete outsider. My only knowledge of the Black Hebrew Israelites is from watching their presentations on public cable access TV, wherein I learned that they believe that the King James version of the Bible is the most authoritative and trustworthy translation (because it contains more references to the Israelites being black), and they really take Song of Solomon 1:5 to heart. The 5\% Nation I'm only really familiar with through their wikipedia page, and the lyrics and imagery of their musicians, most of which are amusing but seemingly incoherent.

Some more research into the matter could be fruitful… if not just because all of these groups seem to be dissenting from the more peaceful globalist sentimentalism of the New Age groups and similar movements, and therefore might resemble some undesirable outgrowths for the greater counter-traditional process.


\hfill

\texttt{Cologero on 2012-07-01 at 22:50 said: }

Izak, it always struck me as curious that Guenon would devote an entire essay on Mormonism; I can't imagine Mormonism had any influence at all in France of the 1920s. Clearly, he read through their literature and he had his reasons. So I am sure your idea has merit, if only for the reason you propose: opposition to globalism will arise where least expected. Unless those groups have published some sort of catechism, it would be a time consuming task.


\hfill


\end{sffamily}\end{footnotesize}
