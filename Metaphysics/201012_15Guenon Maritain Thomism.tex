\section{Guenon, Maritain, Thomism}

\emph{Prolegomena to any future Western metaphysics.}

\paragraph{Meeting of the Minds}
In 1921, the great Thomist philosopher Jacques Maritain criticized Rene Guenon for participating in the rebirth of gnosis, the “mother of heresies”. Guenon responded, “It would make as much sense to speak of Catholicism as the father of Protestantism. In fact, you are simply confusing gnosis with Gnosticism.”

“If you take the word `gnosis' in its true sense, that of pure knowledge, as I always do when I happen to use it … Gnosis so understood — and I refuse to understand it otherwise — cannot be called the mother of heresies. That would be the same as saying that the truth is the mother of errors.”

To be clear, by `gnosis' or `wisdom' as Evola usually called it, Guenon is referring to a state of being, not to a science or a set of doctrines to learn. At that point in his career, Guenon was involved in studying eastern doctrines on the one hand, and Christian symbolism on the other.

On May 25, 1925, Guenon participated in a round table discussion that included Maritain, where Guenon defended Hindu metaphysics. Guenon denied that it was either pantheist or idealist, contrary to academic consensus. Rather, it is connected more closely to the Aristotelian tradition, including the scholastic philosophy of the Middle Ages as exemplified by Thomas Aquinas.

Maritain objected because, from his point of view, this alliance between eastern and Catholic metaphysics is “an inadmissible subordination and the ruin of the distinction between the natural and supernatural order, between nature and grace.” For Maritain, metaphysics is not beyond theology, the “supreme science”. Although Maritain had an authentic intellectual respect for Guenon, he eventually forced Guenon out of contributing to Catholic journals, and an opportunity was lost to further develop Thomism more fully and completely.

\paragraph{Maritain's Misunderstanding}
First of all Maritain fails to grasp Guenon's distinction between philosophy and metaphysics, so he thinks from the perspective of the former. Since metaphysics is by definition the study of supernature (“beyond physics”), the distinction between the natural and the supernatural order is preserved. Maritain simply asserts that the supernatural order can be grasped only by faith and not by any sort of gnosis; this is remarkable, since as Guenon reminds us, for “Aristotle and his Scholastic successors … the intellect was in fact that faculty which possessed a direct knowledge of principles.” In other words, Thomism does indeed admit a gnosis, though its full consequences have not been incorporated into theological thinking insofar as it may present a threat to the primacy of faith.

\paragraph{Further background}
Evola also accepts Guenon's judgment about Thomism, which he sees as part of the process of “rectification”. Interestingly enough, John Woodroffe, although he does not explicitly refer to the Aristotelian tradition (as far as I can recall), similarly denies that Tantrism is “idealist”, but is likewise “realist”. In his \emph{Introduction to the Study of Hindu Doctrines}, Guenon develops this topic more fully. Although at one point he claims that the only Western metaphysics is that of Aristotle and the Scholastics, he clarifies: 

\begin{quotex}
We do not include the Alexandrians, however, upon whom Oriental influences came to be exercised in a direct manner. 

\end{quotex}
The Alexandrians would include St Anthony the Great\footnote{\url{https://www.gornahoor.net/?p=1224}} and what we call the Hermetists\footnote{\url{https://www.gornahoor.net/?p=1256}}.

\paragraph{Future Directions}
Where Thomism falls short is that it is a metaphysic of Being. It needs to be enhanced with an understanding of non-being as described in the \textbf{Multiple States of Being}. Catholic theology is hampered somewhat by the Eighth Ecumenical Council which denied the tripartite nature of man as spirit, soul, and body. (The Eastern churches don't accept this council.) This needs to be overcome, so elements from the \textbf{Great Triad} and \textbf{Man and his Becoming according to the Vendanta} can be incorporated.

In summary: to recreate a Western metaphysic for our time one would:

\begin{enumerate}
\item Begin with Thomism 
\item Incorporate an understanding of non-being, infinity, and non-duality from the Vedanta and Taoism 
\item Develop more fully the understanding of tripartite nature of man 
\item Integrate it with the ancient Hermetic tradition 
\item Integrate this with a spiritual practice so it arises from a true gnosis and does not devolve into yet another intellectualizing philosophy or theology 
\end{enumerate}
The young man who will take up this task may already have been born.


\hfill

Reference: \emph{Rene Guenon: Le philosophe invisible} by Jean-Luc Maxence. All translations from the French are mine.

\flrightit{Posted on 2010-12-15 by Cologero}

\begin{center}* * *\end{center}

\begin{footnotesize}\begin{sffamily}
\texttt{James O'Meara on 2010-12-15 at 23:27 said: }

Oddly enough, that 5 step program reminds me of Alan Watts' intellectual journey [see his Beyond Theology or In My Own Way] although he may have gone a bit astray on the last step…


\hfill

\texttt{Cologero on 2010-12-16 at 00:31 said: }

Perhaps I was just recalling ideas hidden in the unconscious from when I read those books many years ago.

I know someone who, as a young man, managed to get to San Francisco from Chicago in order to meet Mr. Watts. On his arrival, he looked up Watts in the phone book and called. His wife answered. There were noisy children in the background and the wife made some disparaging remarks about why Watts wasn't home … in particular referring to his drinking. Disillusioned, the young man left San Francisco, not having met Watts.


\hfill

\texttt{GF on 2010-12-17 at 12:11 said: }

Why not begin with Dionysius the Areopagite? He already has fully developed arguments on Non-Being and gnosis, and at least an implied understanding of tripartite nature. Do you include him among the Alexandrians? 

Which makes me think: If the finished doctrine is supposed to combine (Alexandrian) Hermetism with Thomism, as an admittedly superior leaven, isn't it truer to say that one should begin with the Alexandrians? As in fact Gornahoor has been doing?

But I see you say “for our time”: and I understand why *appearing* to begin with the Alexandrians would be impolitic in our time, at least among Catholics.


\hfill

\texttt{James O'Meara on 2010-12-17 at 12:14 said: }

How Watts went astray: searching and `finding' analogies btw esotericism and what would become the New Age, he confused the Taoist Sage with the stoned hippie or hot-tub psychologist:

“…SERIOUS PLAY [a very Wattsian idea] contains at the same time a serious warning: there is Play and play, there is the Magician and the magician; this is why anyone who confuses lack of concentration with concentration without effort…will necessarily become a CHARLATAN.” — Meditations on the Tarot


\hfill

\texttt{GF on 2010-12-17 at 12:17 said: }

Dionysius also has a conception of esoterism. So my `counter-proposal' is this:

\begin{enumerate}
\item Begin with Dionysius

\item Integrate with Thomism

\item Integrate with spiritual practise
\end{enumerate}

Your steps 2 and 3 are basically eliminated, and the represented order is more accurate (Thomism, being lesser, comes later).


\hfill

\texttt{James O'Meara on 2010-12-17 at 12:29 said: }

Again, oddly like Watts, who indeed began with Dionysius, published a translation of the Theologica Mystica whilst in the seminary in 1944 [!] and later republished it in his full hippie period, saying something like “Here had been the opportunity for Western theologians to close up shop and become schools of contemplation of the nameless, but instead they just kept chattering.” From 1 to 3, skipping 2?


\hfill

\texttt{GF on 2010-12-17 at 13:04 said: }

Yes, I suppose it comes down to that. St. Bernard himself, whose praises Guenon sung, had only contempt for the verbiage of philosophers and theologians. Though if a theological `ornamention' must be added, St. Thomas would be valuable.


\hfill

\texttt{Cologero on 2010-12-17 at 14:11 said: }

I might never have heard of Guenon were it not for Watts. He follows a familiar path … try to reform one's own “tradition”, get frustrated and turn East. And then fall into the confusion that Tomberg describes.


\hfill

\texttt{GF on 2010-12-17 at 14:12 said: }

One of the very interesting things about Dionysius is that he claims his written work is informed only by the `Oracles’ (Old and New Testaments) – and yet how different it is than sola scriptura Protestantism. So much depends on the atmosphere in which one lives. – In case you didn't know, James, Dionysius was Bishop of *Athens*.


\hfill

\texttt{Cologero on 2010-12-17 at 14:22 said: }

Yes, Dionysus, whom Aquinas quotes some 1200 times, is among the Alexandrians. I would regard the Aristotelean and Platonic perspectives as two \emph{darshanas}, or perspectives. We are well past the time for worrying about being “impolitic” about issues related to these topics since the era of persecution, at least in this area, is over. There is no need to convince anyone. This writer was driven from a self-described traditional Catholic forum for promoting Dionysius, so I wouldn't waste time in a futile debate.

What is required is a group effort to define and implement the task.


\hfill

\texttt{Cologero on 2010-12-17 at 14:38 said: }

In his commentary on Vatican II — an event which was pretty much pointless — Evola proposes his own program to reform the Catholic religion. I'll have to make time to translate it soon. Suffice it to say that I read Tomberg in that light, in particular the way he understands Bible stories. Evola concedes that St Clement of Alexandria's distinction between the gnostics (knowers) and believers was behind his own conception of “those who know and those who believe”.

Guido de Giorgio is a figure, close to both Guenon and Evola, who unfortunately remains unknown. While Guenon was turning East and Evola remained staunchly in the West, di Giorgio developed a synthesis of both. Like Gornahoor, he sees a continuity, not a break, from the ancient world through the Medieval period.


\hfill

\texttt{GF on 2010-12-17 at 15:25 said: }

“What is required is a group effort to define and implement the task.”

This is exactly why I ask you about Catholicism. If I'm reading you right, the outline of your plan would be the development of an order, which, like the Fransiscans and the Jesuits Tomberg cites, would replenish the Church from the outside. I gather that ultimately you are indifferent to the Church: if it can be bent to Traditional ends, use it; if not, forget it. So much for the Church: what of Christ Himself? 

I'm looking forward to the new translations.


\hfill

\texttt{Cologero on 2010-12-24 at 16:04 said: }

For those interested in the current state of Hinduism in India, this article may be of some interest: Rush Hour for the Gods.

\begin{quotex}
While the West often likes to imagine the religions of the East as deep wells of ancient, unchanging wisdom, in reality, much of India's religious identity is closely tied to specific social groups, caste practices and father-to-son lineages, all of which are coming under threat as Indian society transforms beyond recognition.

\end{quotex}

\hfill

\texttt{Cologero on 2010-12-24 at 16:19 said: }

Some points:

\begin{itemize}
\item Thomism takes non-being into account in the distinction between \textbf{potency} (non-being) and \textbf{act} (being). 
\item Hinduism in itself is no gold standard. The Vedanta is just one of six orthodox schools and affects very few. 
\item Westerners, in their arrogance, assume they can just jump into Advaita Vedanta without undertaking all the ritual requirements of a Brahmin. 
\item Westerners simply assume they would be Brahmins in a “traditional” culture. 
\item If Westerners cannot even recognize their own traditional elements, it would strain credulity that they could recognize them in another culture. I recommend these comments by Ananda Coomaraswamy: Vedanta and Western Tradition. 
\end{itemize}

\hfill

\texttt{Brother Otto on 2020-12-08 at 22:23 said: }

To my knowledge, Guenon and his followers have never presented a real argument for why Thomism is inadequate. They simply privilege a negation (non-being, which Guenon reminds us, is not nothing!) and beg the question. Multiple States of Being is one of Guenon's worst books, where he repeats points he made better (albeit still part charlatan) elsewhere.

Thomism is not incomplete, and it certainly does not need crypto-Buddhist advaita vedanta. Catholics don't need moksha.

\hfill

\texttt{Cologero on 2020-12-19 at 21:58 said: }

Catholics don’t need Saint John of the Cross, either.

\end{sffamily}\end{footnotesize}
