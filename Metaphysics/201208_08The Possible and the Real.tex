\section{The Possible and the Real}

In an exchange of letters, \textbf{Rene Guenon} points out that \textbf{Julius Evola} misunderstood the metaphysical meaning of the “possible” and the “real”. It would be one thing had Evola understood Guenon and then attempted to refute him, but that was not the case. Instead, Evola resorted to accusations of “Guenonian scholasticism” or even “rationalism”. Unfortunately, this lack of understanding of a fundament thesis of metaphysics colors—or better said—discolors Evola's works. First, we will review what Guenon says about the topic in Chapter 2 of \textit{The Multiple States of Being}; in subsequent posts, we will show how this affects Evola's viewpoint.

To start, Guenon shows that the Infinite is identical to the Absolute, and by Infinite he means universal Possibility. Without being too misleading, it may be easier to regards possibilities as ideas, or even essences. By Infinite, in the metaphysical sense, Guenon means that Possibility is unlimited. Possibles are of two types:

\begin{itemize}[nosep]
\item Possibilities of non-manifestation 
\item Possibilities of manifestation 
\end{itemize}
Here are some preliminary definitions:

\begin{itemize}[nosep]
\item\textbf{Existence:} possibilities that are manifested 

\item\textbf{Compossibles:} Possibilities that are mutually consistent 

\item\textbf{World:} the entire domain formed by a certain ensemble of compossibles realized in manifestation 

\end{itemize}
Thus a world is characterized by the totality of possibles that satisfy certain conditions, and a world is just one degree, or level, of Existence. Hence, the other possibles which are incompatible with a given (our) world are nevertheless realizable in a different world. This means that

\begin{quotex}
every possibility that is a possibility of manifestation must necessarily be manifested by that very fact, and that, inversely, any possibility that is not to be manifested is a possibility of non-manifestation.

\end{quotex}
Simply put, everything that can happen, will happen. Grasping this with full understanding will alter how you look at the world, not that it is necessarily trivial to determine which possibilities are compatible with each other at any moment in a given world.

The domain of manifestation is limited in that it is a totality of worlds, i.e., conditioned states, and there are an indefinite number of such worlds. By \emph{indefinite}, Guenon means what mathematicians call countably infinite. Hence, the possibilities of manifestations, as thus conditioned, are countably infinite.

However, remember that all Possibility is unlimited and not conditioned by anything, specifically in this case, it is not conditioned by manifestation. Therefore, there are necessarily possibilities of non-manifestation, and, obviously from what has been said, their number is uncountably infinite.

Guenon points out that the possibilities of manifestation are not “superior” in any sense to those of non-manifestation. That is, it is not the result of a “moral” choice, or the “best of all possible worlds”. In this regard, Evola agrees to some extent, but it is equally clear not this does not equate to being amoral. From all this, Guenon concludes:

\begin{quotex}
The distinction between the possible and the real, upon which many philosophers have placed so much emphasis, thus has no metaphysical validity, for every possible is real in its way, according to the mode befitting its own nature; if it were otherwise, there would be possibles that were nothing, and to say that a possible is nothing is a contradiction. … the impossible alone is a pure nothing. 

\end{quotex}
The application of this notion to the conditions of material existence or to post-mortem states will be discussed in subsequent posts. However, in the meantime, I will point out and example from the chapters “Spirit and Intellect” and “The Eternal Ideas” from Guenon's Miscellanea.

Guenon brings up the common misunderstanding that there is a distinction between the possible and the real, by regarding the unmanifested possibilities as merely virtual. Guenon makes clear,

\begin{quotex}
There can be nothing virtual within the Principle but, on the contrary, only the permanent actuality of all things in an ‘Eternal Now', and it is this very actuality that constitutes the sole foundation of all existence.

\end{quotex}
This is obviously equivalent to the Western Tradition's understanding of God, who is all actual and nothing virtual, and is the one and only timeless foundation of existence.

As another example, Guenon says that \emph{Atman}, that is, the Self or the “I”,

\begin{quotex}
always remains unmanifest, not being affected or modified by any contingency. 

\end{quotex}
Clearly, Guenon intends by \emph{Atman} the most real, although not manifested, certainly more real that the conditioned manifested states of the being. We conclude by emphasizing once again, that the understanding of these metaphysical principles does not come from the sort of logical and rational argument we have been making. Ultimately, it can only be truly known through a change in consciousness. Guenon writes:

\begin{quotex}
From the moment one recognizes that the existence of manifested beings in all their positive reality can only be a participation in principial Being, there cannot be the slightest doubt about this matter. … What is virtual is not our reality within the Principle, but only the awareness we may have of it as manifested beings, which is obviously something quite different; and \emph{it is only through metaphysical realization that this awareness of our true being, which is beyond and above all becoming can become effective}, that is actualized in the awareness … an awareness of that which we really are principially and eternally, and \emph{this is in the most absolutely real sense possible}. 

\end{quotex}

This important topic is well worth discussing, perhaps clarifying Guenon's ideas, pointing out flaws in my exposition, drawing out consequences, whether it really is important, etc. However, specificity is mandatory.

\flrightit{Posted on 2012-08-08 by Cologero}