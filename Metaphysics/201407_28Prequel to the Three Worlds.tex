\section{Prequel to the Three Worlds}

\textbf{Cornelius Agrippa}, in \emph{Occult Philosophy}, posited the existence of three worlds: the elementary, celestial, and intellectual worlds, hierarchically arranged, as illustrated, for example, in \textbf{Robert Fludd}’s Diagram\footnote{\url{https://www.gornahoor.net/wp-content/uploads/2011/03/hierarchy.jpg}}.

\begin{quotex}
Since each inferior world is governed by its superior and receives its influence, so the mages believe that one can penetrate naturally by means of the same levels and for each one of these worlds up to the “archetypical world”, constructor and ruler of all things and from there to act not only on natural powers, but to also create new ones.

\end{quotex}
In the essay on \textbf{Hermann Keyserling}, \textbf{Julius Evola} added that as a footnote to some of Keyserling’s ideas, in particular, “The representation creates reality, not vice versa.” Clearly, both Keyserling and Agrippa recognize that the subtle rules the dense. At this point, it is worth the time to consolidate the ideas about the three worlds.

Other terms for the three worlds are “hylic, psychic, and pneumatic” or “exoteric, mesoteric, and esoteric”. I was going to summarize these worlds in a post, but it became too long and involved. So, first there is this prequel which serves as the introduction.

\paragraph{What Is Spiritualism}
I recently saw part of a movie, which I don’t recommend: you will meet a tall dark stranger. In it, the ditzy mother, a divorcee, met a man she described as “spiritual”, because he dabbled in séances, reincarnation, psychics, etc. Obviously, that is quite far from what we mean by spirit. Rather we mean interiority.

That is the heart of the Western tradition, also called “spiritualism” in other languages, but, unfortunately, that has much different connotation in the Anglosphere. So we are forced to use the term “idealism”. As we have pointed out many times, Plato is but one link in a long chain that goes back much further than him. It took different forms in 19th century Germany, moved on to France and Italy, where the trail goes cold in the second half of the 20th century.

In the mission to contribute to the living stream of Hermetism, it is imperative to revive this way of thinking. At a time when materialism and positivism are seen as the touchstones of rational discourse, the spiritual worldview may seem incredible. Nevertheless, an intellectual conversion is possible, in which case it will be understood as the most sane and intelligent view. Since Julius Evola consciously attempted to tie European idealism to Hermetism, his system of magical idealism is worthy of study.

For more background, I suggest reading through The Science of Peace by \textbf{Bhagavan Das}\footnote{\url{https://www.gornahoor.net/library/ScienceOfPeace.pdf}}, reviewed by Evola\footnote{\url{https://www.gornahoor.net/?p=6328}}. He connects Advaita Vedanta to German idealism, particularly Fichte.

\paragraph{Macrocosm}
Whenever some people hear of “occult philosophy”, “magic”, or “mages”, they tend to shut down, visualizing something heretical, dangerous, or even satanic. However, properly understood, it is the consequence of orthodox teachings. The purpose of exoteric practices is to create certain states of consciousness, although in a spontaneous and passive way. So a particular devotion may lead to one such state or another. As spontaneous, it is experienced by the practitioner as originating from the outside.

The esoterist, on the other hand, achieves the same states in an active way, as the free act of the will. He does not oppose the exoterist, but rather understands the teachings in a very different way. This can be revisited after the commentary on Keyserling on the topics of understanding and meaning. Since the exoterist cannot understand the esoterist, he assumes there is something false about it.

Another difference is the relationship to ideas. The exoterist uses ideas more as weapons to buttress an argument or to win debates. For the esoterist, an idea is a real, active, and living force. So let’s take the idea of the three worlds into consideration. Basically, it is saying that the Divine creates the order of the universe, the angelic hierarchies are charged with the maintenance of that order, and the elementary world, i.e., the phenomenal or natural world, is the result. This is the principle that the subtle rules the dense on the macrocosmic scale.

The exoterist agrees with that, but sees it as something happening from the outside, with no further relevance. The Mage, as Agrippa points out, instead understands that teaching interiorly and then strives to “penetrate” into those levels. There is certainly justification for this. Since God and the angels are spiritual beings, there is no “outside”. Either they are known interiorly or else the possibility of gnosis is denied a priori. Those who have experienced gnosis know the answer.

\paragraph{Microcosm}
The analogy of being —as above, so below— is also orthodox. So corresponding to the three worlds in the microcosm is the spirit, the soul, and the body. These are definitely experienced interiorly. And they functionally correspond to Agrippa’s three worlds. The spirit rules the soul and the body. The body is not independent; the spirit and soul are the form of the body. Hence, any mage who can penetrate into the knowledge of the inner workings of the spirit and soul can then have control over the natural world as represented by the body. This can even happen spontaneously in various psycho-somatic manifestations.

The point here is not a scholarly discussion of Agrippa, but rather to come to experience these teachings from the inside.

\paragraph{The Thirst for Action}
\textbf{Theophan the Recluse} pointed out that there are three stimuli that motivate the thirst for action:

Stimulus $\rightarrow$ Object of desire

The Necessary $\rightarrow$ Duty

The Useful $\rightarrow$ Service

The Pleasant $\rightarrow$ Pleasure

He writes that there are natural and legitimate activities of the will,

\begin{quotex}
which is the master of our powers and our whole life. Its work is to determine the form, the means and the degree of satisfaction to be given to the different desires that arise from our needs, and to decide on substitutions so that life may flow smoothly and bring peace and joy to the individual.

\end{quotex}
The high-functioning Will, then, will balance the natural needs that relate to the sour, the body, everyday life, and to society. Unfortunately, the Will can be dissipated and “alien things stimulate desires that are foreign [unnatural] to us: anger, hatred, envy, miserliness, vanity, pride, etc.”

\paragraph{Active Spirituality}
The last time, I contrasted two spiritual perspectives:

\begin{itemize}
\item The desire for worldly things 
\item Detachment from the world 
\end{itemize}
As such, they are both passive approaches, so what happens if we flip them around as active forces. Two things result from that. Rather than being two distinct incompatible approaches, they instead become complementary.

\begin{enumerate}
\item The prayers for worldly successes turn into the work of the mage to penetrate the archetypal world of ideas in order to bring them into manifestation. However, the desired manifestations should conform to our natural needs arising from our sense of duty, the utility of the manifestation, and legitimate pleasures. 
\item Yet, there is detachment from the world and the results of our willing as described in the fruits of mental prayer. 
\end{enumerate}
As Theophan concludes, all is done for the greater glory of God. The Warrior-Monk will have complete confidence that his Will is in conformity with the Divine Will and his prayers will be answered. Yet he is also detached from the world.

\flrightit{Posted on 2014-07-28 by Cologero}
