\section{The Traditional Approach}

In \textit{Traditionalism, this is the Enemy}\footnote{\url{http://www.counter-currents.com/2010/06/traditionalism-this-is-the-enemy/}} the ``philosopher" of the European New Right, Guillaume Faye, critiques ``Traditionalism", allegedly from the inside. He calls it ``metaphysical traditionalism", which no one else has ever called it, whatever ``it" may be, but, since he attributes ``it" to Julius Evola, we prefer to critique the critique with the words of Evola himself. Since this comes from the forward to \emph{Revolt Against the Modern World}, one would not have to read very much to realize Faye's misunderstanding. First of all, he claims to have exhumed three axioms of ``metaphysical traditionalism", axioms which are nowhere to be found in either Evola or Guenon. Since the rest of the article depends on those axioms, it is not worth reading past that point, unless you are one of those people who stops and stares at \textbf{road kill}.

First of all, Tradition does not refer to a nostalgic looking back in time, though it does involve a remembering. Furthermore, it is not so much concerned with the affairs of men, but rather of the revelations of the gods that give meaning to the past. This is how Evola describes his project:

\begin{quotex}
The order of things that I will mainly deal with in this present work, generally speaking, is that in which all materials have a ``historical" and ``scientific" value are the ones that matter the least; conversely, all the mythical, legendary, and epic elements denied historical truth and demonstrative value acquire here a superior validity and become the source for a more real and certain knowledge…

The scientific ``anathemas" in regard to this approach are well known: ``Arbitrary!" ``Subjective!" ``Preposterous!" In my perspective there is no arbitrariness, subjectivity, or fantasy, just as there is no objectivity and scientific causality the way modern men understand them. All these notions are unreal; all these notions are outside Tradition. Tradition begins wherever it is possible to rise above these notions by achieving a superindividual and nonhuman perspecive; thus, I will have a minimal concern for debating and ``demonstrating." The truths that may reveal the world of Tradition are not those that can be ``learned" or ``discussed"; either they are or they are not. It is only possible to \textbf{remember} them, and \emph{this happens when one becomes free of the obstacles represented by various human constructions}, first among which are all the results and the methods of specialized researchers; in other words, one becomes free of these encumbrances when the capacity for \emph{seeing} from that nonhuman perspective, which is the same as the traditional perspective has been attained. \emph{This is one of the essential ``protests" that should be made by those who really oppose the modern world."}

\flright{\textbf{Julius Evola}, Forward to \emph{Revolt Against the Modern World}}
\end{quotex}

\flrightit{Posted on 2010-07-08 by Cologero }
