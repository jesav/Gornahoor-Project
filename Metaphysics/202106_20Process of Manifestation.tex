\section{Process of Manifestation}

The Universal includes the unmanifested as well as formless manifestation. They are the principles of formal manifestation. Formless manifestation has Being but not individual Existence. The Individual states are all part of formal manifestation.

The \textbf{Unmanifested state} comprises all the possibilities which are not susceptible of any manifestation, as well as the possibilities of manifestation themselves in principial mode Formless manifestation is not part of multiplicity. The following diagram is from \emph{Man and his Becoming}.

\begin{itemize}
\item Universal 

\begin{itemize}
\item Unmanifested 
\item Formless manifestation 
\end{itemize}
\item Individual 

\begin{itemize}
\item General 
\item Specific 

\begin{itemize}
\item Collective 
\item Singular 
\end{itemize}
\end{itemize}
\end{itemize}
The being that manifests is necessarily what it is, through its principal manifestation. It is not a matter of chance or fortune

\paragraph{General and Specific}
The general is related to the Specific as genera to the species. The being has to pass through an indeterminate number of states. Guenon cites Aristotle's categories as the most general of all genera, hence they belong to formal manifestation. Some simple examples might be:

\textbf{Place} $\Rightarrow $ galaxy $\Rightarrow $ solar system $\Rightarrow $ planet $\Rightarrow $ continent … city $\Rightarrow $ house

\textbf{Time} $\Rightarrow $ Manvantara $\Rightarrow $ yuga $\Rightarrow $ year

These genera can be further subdivided, but these examples make the point

\paragraph{Collective Entities}
Collective entities are just as real as singular entities. Guenon describes one such collective entity

\begin{quotex}
The names which are attached to the formulation of the different darshanas cannot be related in any way to particular individuals. They are used symbolically to describe what are really `intellectual groupings'. composed of all those who have devoted themselves to one and the same study over the course of a period the duration of which is no less indeterminable than the date of its beginning. 

\end{quotex}
There are more obvious examples such as religions, nations, sacred orders, and so on. As in the case of intellectual groupings, such collective entities cannot be judged by the individuals who are representatives of members of the collective. Rather, they must be understood in terms of the guiding principles of the collective. That is why, for example, the ancient Greek city-states were founded by a divine being and why nations are today guided by archangels. But just as material beings cast a shadow, so do collective entities. In the latter case, the shadow is called an egregrore. Unfortunately, people often can only see the egregore rather than the purity of the collective.

\paragraph{Singular Entities}
Singular entities are identified by perceptible bodies, whether gross or subtle. This topic will be developed in an upcoming post on human manifestation.



\flrightit{Posted on 2021-06-20 by Cologero }
