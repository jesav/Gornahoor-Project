\section{Spirituality for the XXIst Century}

\begin{quotex}
As the moon is the intelligible unity in the four-dimensional manifold of positions and phases, so human subjectivity is the intelligible unity in the multi-dimensional manifold of the conscious events of a lifetime. \flright{\textsc{Bernard Lonergan}}

Matter is essentially multiplicity and division; and this, be it said in passing, is why all that proceeds from matter can beget only strife and all manner of conflicts between peoples and between individuals. The deeper one sinks into matter, the more the elements of division and opposition gain force and scope; and, on the other hand, the more one rises towards pure spirituality, the nearer one approaches to that unity which can only be fully realized by consciousness of the universal principles. \flright{\textsc{Rene Guenon}}

Come Timothy Leary, crede che essere religiosi nel XX secolo senza usare l'LSD sia come pensare di studiare i pianeti a occhio nudo. [Like Timothy Leary, Philip K Dick believed that to be religious in the 20th century without using LSD is like believing you can study the planets with the naked eye.] \flright{\textsc{Andrea Scarabelli}, \textit{Science Fiction and Metaphysics}\footnote{\url{http://blog.ilgiornale.it/scarabelli/2016/06/03/philip-k-dick-emmanuel-carrere-fantascienza-e-metafisica/}}}

If LSD was necessary to be religious in the 20th century, then Rene Guenon is necessary to be religious in the 21st century. \flright{\textsc{Cologero Salvo}}

\end{quotex}
\paragraph{Three Modes of Spiritual Recovery}
With the decline of religious influence in the modern world, there have been successive attempts among intellectuals to justify an alternative spiritual worldview. These are summarized as:

\begin{itemize}
\item \textbf{19th Century}: Philosophical idealism 
\item \textbf{20th Century}: Direct experience 
\item \textbf{21st Century}: Tradition 
\end{itemize}
First in Germany, then in England and Italy, philosophical idealism served as a replacement to religious belief. This has been previously dealt with in other posts\footnote{\url{https://gornahoor.net/?p=8366}}.

The 20th century tried to use direct experience in the manner of scientific positivism. Thus, there was an Aldous Huxley, a Timothy Leary, an Alan Watts, a Philip K Dick, or a Carlos Castanada, who relied on psychedelic experiences to break the spell of consensus reality. Channeling became popular as an alternative to ancient texts. New Thought techniques sought to manipulate the physical world in conformance with one's will. Sex Magick took the sublimation of sexual energy all too literally as a means to transcendence. Of course, an Aleister Crowley could combine all these elements into a larger system.

Both these approaches had to overturn, deny, or reject archaic traditions, yet there lasting effects have been minimal. Now that the works on Tradition by Rene Guenon have become generally available, we finally have a more fruitful and satisfying approach. While idealism appealed to the reasoning mind, and experience to the senses and emotions, Tradition goes straight to the intellectual center of man. Guenon gives us the key to understand religious symbols and dogmas, explains the relationship between the exoteric and esoteric, and outlines a satisfying metaphysical conception of man, God, and the cosmos.

Personally speaking, I have traveled through all these options. Living in tune with a Tradition may seem difficult today, but, judging by drug usage, suicide, and general unhappiness, life in the modern world has its own difficulties. Ultimately, Tradition is a rational, satisfying, and a fully human way to live in the world.

\paragraph{Thinking and Acting}
\begin{quotex}
Actions speak louder than words; let your words teach and your actions speak. We are full of words but empty of action, and therefore are cursed by the Lord since He Himself cursed the fig tree when He found no fruit but only leaves. \flright{\textsc{St. Anthony of Padua}}

What you do speaks so loudly that I cannot hear what you say. \flright{\textsc{Ralph Waldo Emerson}}

\end{quotex}
The American logician \textbf{Charles Sanders Peirce} in the essay \textit{How to Make our Ideas Clear}\footnote{\url{https://gornahoor.net/library/HowToMakeOurIdeasClear.pdf}}, asserts that the ``final upshot" of thinking is volition. That is, a clear idea is manifested in action. For example, if I really believe it will rain on Saturday, I will cancel my picnic plans. However, few people are at all self-aware about the actual content of their thoughts. A believer can justify a course of action as a manifestation of her Christian beliefs while an atheist can advocate the same action on totally different principles. Hence, we can suspect that there is something deeper involved, even if unknown and unacknowledged to the Christian and the atheist.

That is because most spiritual beliefs are limited to exoteric forms, and conversions are from one exterior form to another, i.e., a horizontal move. An intellectual conversion, on the other hand, is in the vertical direction, and involves one's entire being. Hence, the interminable debates among atheists, Christians, pagans, and so on, are pointless as long as they are restricted to external forms only. Instead, discussions should be based on certain metaphysical principles and the consequences of their acceptance or denial. We will address these questions on three levels.

\begin{itemize}
\item \textbf{Cosmic}. These are the universal metaphysical principles that need to be considered. 
\item \textbf{Historical}. The cosmic does not efface the historical, so metaphysical principles do have consequences. Now Guenon accepts the unity of all Traditions, so that a particular tradition is somewhat arbitrary, at least for certain types of men.

We are inclined, however, to Augustine's understanding that although there is but one Tradition, it is now known as the ``Catholic" tradition. This we will explain and justify. 
\item \textbf{Ethical}. How does the intellectual conversion affect the way that life is lived? 
\end{itemize}
In this post, we will deal with the cosmic aspect, with the others to follow subsequently.

\paragraph{Cosmic}
Interiority is a great mystery to most people. Seldom do they bother to examine, in an objective and detached way, their thought processes nor are they interested in the unspoken assumptions supporting their worldviews. Typically, people are interested in the answer to the ``what is" question: e.g., he is a democrat, a pagan, a Zoroastrian, and so on. Those turn out to be merely nominal associations and denote nothing consistent or transcendent.

Therefore, what is of more interest is a person's viewpoint on various metaphysical principles. A nominal affiliation, except for those who consciously and deliberately choose it, seldom serves to answer such questions. Here we will consider some of these principles, detailing the traditional view, and the implications of their denial. Of course, what follows are not arguments, but rather first principles. Either they are assumed or intuited or they are not. What is different however is this: the person who bothers to think about such principles can see clearly the effects of their denial, whereas denialists are seldom even aware of them. Hence, they often act as though the principle were true, while denying it at the same time.

\subparagraph{Being}
The Universe really is One and all things are interrelated. God is the Principle of Being, so one strives to keep the awareness of Being in consciousness and to live in that presence.

The denial of Being is to assume that only beings exist. Unfortunately, neotheism, which is the belief that God is one of those beings, is quite prevalent. That assumes that God is not the ultimate principle, so neotheism does not respond to the question of Being.

\subparagraph{Order}
The universe is an ordered whole. Moreover, Order, or Logos, must be one with Being, otherwise there will be a dualism between Order and Chaos.

The denial of Order means that there is just the random movement of things.

\subparagraph{Intelligibility}
Since Order is inherent to the universe, the real is intelligible. Beings with an intellect, then, are able to understand the real. Whatever is not intelligible is not real. To know a thing is to know what it is.

The opposite is to regard the world of the senses as real. We can never know that world, but only discover better theories. Words are used to group things with related properties and do not point to a higher intelligible reality. Hence, entirely new classes of people are created simply by definition, with no relation to real distinctions.

\subparagraph{Providence}
There is a purpose and a direction to the cosmos. We don't mean, by this, a simplistic notion of a god interfering with events from time to time, but rather a permanent, omnipresent principle. It is a responsibility to be aware of signs of Providence and to act on them.

The denial means that there is no real purpose to life. Events are the result of chance and determined physical processes. Many would include human willfulness as a factor in events. Yet this willfulness itself can have no ultimate purpose as much as it may try to live a purposeful life.

\subparagraph{Fall of Man}
Without going into specific details, which will be considered in the follow up, the various traditions regard man as existing in a ``fallen" state, that is, from a higher state of being. His task in life is to overcome the forces that maintain him in that lower state in order to return to his rightful place in the cosmic order.

The denial means that man is born free or perfect, and that it is society that condemns him to a lower state. Hence, if a man has certain desires or habits, these are ipso facto considered to be ``good". There is no necessity for him to ``overcome" anything. Rather, all efforts must be directed to changing the environment, both social and physical.

\subparagraph{Summary}
What is considered to be ``progress" is actually, in most cases, a regression to a more primitive worldview. Materialism, atomism, and the belief that events result from unchangeable fate, random processes, or pure will, are opposed to the process of regeneration.



\flrightit{Posted on 2016-06-30 by Cologero }

\begin{center}* * *\end{center}

\begin{footnotesize}\begin{sffamily}



\texttt{muaddibbr on 2016-07-01 at 09:15 said: }

When focused in the first principles one can see more easily if there is such Unity behind the exoteric veils. But Im not sure about such Unity, there are some issues to be resolved such as: 

1) the question of grace 

(in Islam and Christianity is clear, but how is that in Buddism? the idea of rendemption seems incompatible with their karma-chain: 

``It results in a mechanistic and deistic universe where God can have no merciful, enlightening, forgiving and redeeming relationship with the worlds and the souls He has created—a universe where, because there can be no dharma, no saving Divine intervention, no religious dispensations, karma is absolute. I can neither repent, in such a universe, nor can God forgive. It was this absolutization of karma which led Mme Blavatsky (who, as we shall see, did accept reincarnation in her final work, The Secret Doctrine, despite denials by some of her followers) to hate and reject the Christian doctrine of the forgiveness of sins as a violation of the law of karma, and even to define prayer and sacrifice, conceived of as attempts to alter or circumvent karma, as acts of black magic. But to take karma as an absolute is absurd and self-contradictory. Karma, as the chain of causal actions and reactions in the relative world of samsara, is relative in essence; it can never be absolute. Every condition of causal inevitability on the horizontal plane can be compensated for by the operation of human freedom, and Divine Mercy, on the vertical one." Charles Upton)

2) the question of creation: creationism (ex nihilio) vs manifestionism (also, question of Freedom of God and His necessity to create the world).

3) the question of a `left-hand' path: how can there be a left-hand path in religions like Christianity and Islam? The acceptance of the left-hand path is derived from metaphysiical assumptions (that one can actively destroy his ego by serving a Descrutive Principle)

Those are some I can remember now, but Im sure there is more issues on the level of first principles. And as a Christian, of course, I hold that Jesus is the Logos in flesh, there is no ``exoterism" in this.


\hfill

\texttt{Boreas on 2016-07-01 at 12:00 said: }

muadddibr, the way I personally have interpreted the law of karma and the forgiving light of grace is that karma is in the left hand of God, that is, Satan (as the third emanation of the tree of life, Binah), and grace resides in the masculine Chokmah, which I personally interpret to be the ``Christ principle". One can also see that the left hand of God in this kabbalistic scheme is the ``Pillar of Jugdment", which I see to be the esoteric essence of Saturn / Binah, and the Right Hand of God is the ``Pillar of Mercy" that resides in the hands of Christ. By defeating Satan in the desert Christ made a perennial change in the balance of the Tree by taking the sceptre of Justice from the hands of Satan and establishing the law of grace and mercy as the highest law.

There is no question that many passages of the Bible can also be seen supporting a left hand path interpretation of things, and I don't mean this as ``the devil interpreting the bible" way. There are especially in the old testament many passages of scripture that speak in the mouth of God with fire and brimstone, therefore judgement and destruction.

It is only as giving way to the entropic ``anti-principle" that the destructive impulse emanating in the end from the absolute divinity with all its aspects is transformed into a downward path; but the destructive hand of Shiva is always benevolent in the end, that is, the destructive principle can also be transformed into a force in the service of the higher good.


\hfill

\texttt{Matt on 2016-07-01 at 18:30 said: }

Muaddibbr,

Your questions and concerns have been addressed to a certain extent here on Gornahoor in past comment sections, past posts by both Cologero and Logres, and even in this very post that we're commenting on (see the reference to Augustine's idea in the bolded Historical part).

It seems (correct me if I'm wrong) that what you're getting at are the questions: is the incarnation of the Logos in the person of Jesus something unique (distinct from all the other conceptions and teachings of avatars in the other traditions) and if so, does the Christian revelation and tradition stand-out from all the others?

These are questions that I am continuing to ponder and wrestle with, and there is one answer that I find myself slowly starting to move towards. I don't doubt that the contributors and other commenters to this blog have been seriously considering the questions as well; so don't feel like you're alone with this.

We might have to look at that Unity from a different angle. Guenon's conception of this Unity was that all the great traditions contain the fullness of the Truth at their respective cores. Maybe, dare I say, that he wasn't quite correct in his conception. This again goes to the reference to Augustine contained in this post.

More light could be shed on this with the next couple of posts in this series by Cologero.


\hfill

\texttt{muaddibbr on 2016-07-02 at 07:55 said: }

Matt,

One particularity of Christianism is that if other traditions are pure mythical, that is, a-historical (and thus they despise the historic dimension), Christianism, for its part, is actually a Myth taking place in History. So, Christianism is a fullfilment in Time of some aspects of earlier mysteries and myths: Christianism is both Mythical and Historical (is there other tradition like this?). 

Jesus embodied a Myth in the Historic time (and if one will complain that maybe in other religions, like Hinduism, this also happened, I can say that no one can prove this, for they go back to ancient times where there `records' like Christians have). Christianity made the historical time begin and thus its hard to think in mythical terms. This will be something negative for older traditions which are linked in mythical time, but I can say that without the historical time humanity would never develop some possibilites that is inherent to such thinking. Without a historical time we would never fulfill some possibilites, and the fulfillment of all possibilites is a `law' in the cosmic process. Christianism make the mythical time disapear and to go back to mythical time is very difficult, if not impossible, for there should be a certain `naivety' which we lost. The humanity gave one step forward the fulfillment of the cosmic process possibilites. 

``The development of the old world conception thus is split. In Neoplatonism and similar conceptions of the world it leads to a concept of Christ related only to the spiritual realm, and on the other hand it leads to a fusion of this concept of Christ with a historical manifestation, the personality of Jesus. The writer of the Gospel of John may be said to unite these two world conceptions. ``In the beginning was the Word." He shares this conviction with the Neoplatonists. The Neoplatonists conclude that the Word becomes spirit in the innermost soul. The writer of John's Gospel, and with him the community of Christians, conclude that the Word became flesh in Jesus. The more intimate sense, in which alone the Word could become flesh was provided by the whole development of the old world conceptions. Plato says of the Macrocosm: God has stretched the soul of the world on the body of the world in the form of a cross. (see Note 77a) This soul of the world is the Logos. If the Logos is to become flesh He must repeat the cosmic process in physical existence. He must be nailed to the Cross and rise again. This most significant thought of Christianity had long before been outlined as a spiritual representation in the old world conceptions. This became a personal experience of the mystic during ``initiation." The Logos become Man had to experience this deed as a fact, valid for the whole of humanity. Something which was a Mystery process in the development of the old wisdom becomes historical fact through Christianity. Thus Christianity became the fulfillment not only of what the Jewish prophets had predicted, but also of what had been pre-formed in the Mysteries. — The Cross of Golgotha is the Mystery cult of antiquity condensed into a fact. We find the Cross first in the ancient world conceptions; at the starting-point of Christianity it meets us within a unique event which is to be valid for the whole of humanity."

— Rudolf Steiner


\hfill

\texttt{Max on 2016-07-02 at 19:02 said: }

What is often lacking is strength of conviction to follow our hearts. If beliefs are just empty shells making no discernable difference in practice they are not worth much. This even applies to secular materialists since they do not live as they teach. The problem is an inner split of the soul that makes us divided, a defining charactersitic of the age. We often complain about how evil, wrong, boring, meaningless – whatever it is – things are, but then goes on to do the exact same things the next day as we do every day, which is a kind of sleepwalking. As long as that continues the world will eventually disintegrate by entropy since maintainance requires conscious action. This is an insight that is easily missed by the modern ``intellectual" class of bureacrats, journalists and so on, since they have never had to maintain anything, giving them the personal maturity of an overgrown child.

``He who desires but acts not, breeds pestilence." -A Proverb of Hell, through William Blake

Perhaps it comes down to fear; we are afraid to live integrally. It does however not mean that we have to blindly follow every whim and desire, although leaving them buried and unexamined also becomes poisonous. What cannot be denied is not the source of desire – because it can, but that the desire is there, which is a precondition to rise above it, or to act on it if it is deemed good.

Guenon's ``The Great Triad" is about the triad of Heaven and Earth with Man having an intermediate position. However that intermediariness risks making man dividedly drawn in different directions, so that his unique place and destiny could both be a potential source of disintegration or of overcoming the condition.

I have come across several fourfold visions, most recently Heidegger who suggests this: Mortals, Gods, Earth, Sky. That could be read as having to recognize and live in light of human mortality, potential divinity, transcendence and immanence.

In the ``Tao", chapter 25, there are also four called ``greats":

``Man is ruled by Earth.

Earth is ruled by Heaven.

Heaven is ruled by The Way.

The Way is ruled by itself."

What is described here has nothing deterministic about it since the way is perfect freedom to rule ``itself" according to its nature. However, humans in their common state are ruled by earth, an affinity to humus, or the humid principle (etymologically, likely from the combination of ``earth" and ``wet"). Drugs as an embodiment of this humid, earthly and corrosive thing, can never go beyond that passive state. It makes explicit and reinforces the human as a slave to earth, such as in karmic relationships of cause and effect, although many apparently remain happy with that. 

The Idealists as well as Phenomenology develops mortal man's faculties of reason in the context of internally evaluated subjectivity. Substance-induced experience is like Earth, a subjective experience immanent to the substantial world. Tradition on the other hand envelopes us like a distant -and unreachably high clear-blue sky. That leaves an expectancy of ``gods" revealing the way, since in this view, Tradition is ruled over by Christ. 

A problem with some metaphysical systems is that they seem to put the impersonal above the personal but neglects how the personal could proceed from some purely abstract architectural structure, which makes them fall near the modern rational-scientific framework, and come across a little life-less. In terms of reasoning it belongs to the level of thinking that mixing together a ``primordial soup" makes life emerge on its own. Hence, either personal Being belong to the ``absolute" or one would have to deny personhood in man.

In reality, when we compose a compost in our garden, life is already present to help transform the elements into humus, black earth, so it is not precisely right to say that earth causes the compost. Many humans settle into making dirt with the progression of bodily decompososition, but this is a task that worms can perform just as well, so objectively there is not that much of a difference. This is the much glorified ``left-hand path": basically to produce fertilizer. Not so terribly ``secret" or ``forbidden" that its adherents tries to convince us of, it is rather mundane and boring to tell the truth, but I guess any crap can be sold by flashy marketing and an air of exclusivity nowadays.


\hfill

\texttt{Boreas on 2016-07-04 at 12:14 said: }

My take is that at the moment of conception and birth of Jesus ``myth entered history", and I also do not believe it was the first time in the history of the world, but maybe the grandest happening thus far, the birth of the last Avatár at moment of Winter Solstice, which again moves the birth of Christ back into the ancient seasonal mode; like we all know, Jesus was probably born sometime in the autumn or so historically, but here tmyth is more powerful and real than the temporal order of things.

This propelled the world into the darkest 2000 years of Kali Yuga, but at the same time it was the highest happening of things at planet earth thus far. The awakening of the Buddha and the life of Krishna were before this almost the same mythological structure enetering the world, but only Jesus brought the Christ into his conscious personality by fusing his own personal impulse to the Cosmic Christ, which enlightened and awakened the whole globe.

Just my two cents to your noble conversation, gentlemen.


\end{sffamily}\end{footnotesize}
