\section{The Concept of Existence}

This is the first of three parts describing the three stages of the realization of the Unity of Existence.

\begin{quotex}
In the eyes of an ordinary man representing the common-sense view of things, the phenomena are the visible and manifest while the Absolute is the hidden. But in the unconditioned consciousness of a real mystic-philosopher, it is always and everywhere the Absolute that is manifest while the phenomena remain in the background. \flright{\textsc{Toshihiko Izutsu}, \emph{The Concept and Reality of Existence}\footnote{\url{https://www.gornahoor.net/library/Concept_and_Reality_of_Existence.pdf}}}

\end{quotex}
The foundation of esoteric teaching is the ``unity of existence". This is presumed in \textbf{Rene Guenon}'s \emph{Symbolism of the Cross}.

\begin{quotex}
Existence is one and indivisible in its inner nature, just as Being is one in itself; indeed this unity of Existence derives directly from the oneness of Being, since universal Existence is nothing but the total manifestation of Being, or, to be more exact, the realization, in manifested mode, of all the possibilities that Being implies and contains principially in its very oneness.

\end{quotex}
What follows is a simplification, but not a misleading, explanation, so it may clarify things for those just starting out and orient them toward the full teaching.

The common opinion is that existence consists of multiplicity rather than unity. That is, for the person at this stage, all the things in the phenomenal world, taken together, form existence; this approach fails to perceive the unity of existence. This common sense view is that objects have properties. For example, the table is brown, the grass is green, the crow is black. This assumes that the table, grass, and the crow exist before they can have properties. A conundrum then arises when you say, ``the table exists." This assumes that existing is a property of the table, but how can it exist before it exists?

As a thought experiment, try reversing things. Then assume that the table is a property of existence, as is the grass and the crow. Then existence is one, but it manifests, to all appearance, as a table, grass, or a crow; phenomena, thus, are just modes or attributes of existence. Otherwise, the experience of the multiplicity of things forms a veil which obstructs the sight of the unity of existence.

This applies, a fortiori, to the understanding of God (Absolute). 
\begin{quotex}
When the Pharoah asks Moses the trick question, ``What is the Lord of the Worlds" (Quran 26:23), he is expecting an ``essence" that exists. For example, the simple-minded answer is to answer something like, ``God is the most perfect being" or ``God is an omnipotent Spirit", etc. These responses presume that the essence precedes existence. So Moses avoids the trap and answers, ``The Lord of the heavens and the earth, and everything between them, if you are aware." \flright{\textsc{Ibn Arabi}, \emph{The Wisdom of Sublimity in the Word of Moses}}
\end{quotex}

That means metaphysically that God's essence is His existence. That is Sufi teaching and it is also Catholic teaching. In the medieval period, while war was going on among the exoterists, esoteric teachings were being shared.

When this is understood, then the things in their multiplicity are seen as revelations of the Absolute, or as a theophany. The first stage, then, is to understand this as the concept of reality, something that can be reached through human reason.

That is Knowledge of the Head.

The next stage of metaphysical realization is to transform the concept into the reality of existence in one's own consciousness.



\flrightit{Posted on 2022-01-11 by Cologero }
