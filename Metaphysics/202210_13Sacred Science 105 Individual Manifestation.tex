\section{Sacred Science 105: Individual Manifestation}

The \emph{Bhagavad Gita} mentions three Purushas.

\begin{enumerate}
\item \textbf{Jivatman}. This is human Individuality and is destructible. 
\item \textbf{Atman}. The indestructible Self or the Personality. 
\item \textbf{Paramatman}. Spiritually identical to the absolute and ultimate reality 
\end{enumerate}
\paragraph{Vertical Causation}
Purusha is the essential (or efficient) cause and Prakriti, the substantial (material) cause of all manifestation. Prakriti contains in potential all the possibilities of manifestation while Purusha determines the development of the possibilities of Purusha thought their passage from potency to act.

In the West, Prakriti is analogous to Prime Matter, at least insofar as it concerns the human state. Hylomorphism is the process by which prime matter (potency) is individualized by an immaterial essence (act).

Of course, matter, as physicists think they understand it, is not what is intended by Prime Matter. Nevertheless, when physicists reach the foundation of matter, they don’t discover things but rather probability distributions described by a wave function, i.e., potency without act. How things come into existence is an inexplicable mystery; that is the so-called “measurement problem”. But physicists have no understanding of Purusha or Prakriti because, as unmanifested, they cannot be measured no observed. Hence, that mystery will never be solved.

\paragraph{Jivatman}
Ahankara is the ego or individual consciousness brought about by thought. Hence, Descartes could explain, “I think, therefore I am!”

The ego is the reflection of Buddhi — part of formless manifestation — into subtle manifestation. The ego is the reflection of the Atman on the soul, like the way sunlight is reflected onto water. Hence, it appears identical to the Atman which gives it birth. The quality of the image depends on the stillness of the water; hence, the freer the soul is from perturbations and undulations, the better will be the reflection of the Atman.

\paragraph{Microcosm and Macrocosm}
If Paramatman represents the Macrocosm, then jivatman is the microcosm. The ego may seem to be one, but careful self-observation reveals it to be multiple. One day, you make a resolution and break it the next day. One day you are happy and then anxious. These represent little I’s that contend for a small part of conscious life.

This is seen more clearly in the dream state. In a dream, or even lively fantasies, jivatman is creating all the characters while believing that they each have a life of their own.

Just as Atman develops all the possibilities of manifestation, the jivatman develops all the possibilities of the individual human state.


\textit{This is tied to Chapters 5 \emph{Purusha Unaffected by Individual Modifications} and 6 \emph{Degrees of Individual Manifestation} from \emph{Man and his Becoming}.}

\flrightit{Posted on 2022-10-13 by Cologero}