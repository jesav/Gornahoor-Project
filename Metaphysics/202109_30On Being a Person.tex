\section{On Being a Person}

\begin{quotex}
The personality, let us insist once more, belongs essentially to the order of principles in the strictest sense of the word, that is, to the universal order; it cannot therefore be considered from any point of view except that of pure metaphysics, which has precisely the Universal for its domain. \flright{\textsc{Rene Guenon}, \emph{Man and his Becoming}}

\end{quotex}
Guenon noted:

\begin{quotex}
the conceptions of Aristotle are in complete agreement with those of the East.

\end{quotex}
With that in mind, we will use that as a guide to interpret not just Aristotle, but also Thomism as its extension. The goal is not to engage in an academic philosophical debate, but rather to explore how consciousness has altered from the Medieval period until now. We accept Owen Barfield's justification for focusing on Thomism:

\begin{quotex}
In the mind of Aquinas, with his enormous erudition, the whole corpus of medieval thought is in a manner recapitulated; and he is as sober as he is profound.

\end{quotex}
\paragraph{Person and Ego}
In this system, a human being is the individuation of a human essence into the corporeal world. That answers the question of “what” it is, but not who. Where then is the specific person, say Peter? Every substance has a \emph{suppositum} or center which, in the case of rational beings, is called “Person” (or Self or Spirit). Hence, Peter as person is also an essence, i.e., an idea in the Divine Intellect (Buddhi). Guenon relates the Person to its manifestation:

\begin{quotex}
The personality is unmanifested, even insofar as it is regarded more especially as the principle of the manifested state

\end{quotex}
In practical terms, this means

\begin{itemize}
\item The Person always exists as the principle of manifestation 
\item The Person cannot be detected by any empirical, physical, or psychological test. 
\end{itemize}
The Person transcends psychical and corporeal manifestation. You are not a “spirit in the world”, rather, you are in the world but not of it. You cannot “contact” the higher self. Conclusion: every human is a Person, even if only virtually (i.e., unaware of his real nature).

Contemporary thinking confuses the Person with some psychological state, ignoring

\begin{quotex}
the fundamental distinction between the `Self'. which is the very principle of the being, and the individual `ego'.

\end{quotex}
The Self, or Person, as intellect, is in the spiritual realm whereas the Ego is part of manifestation. Although claiming a pedigree from Aristotle or Thomism, they are not comprehending its source. This is not a failure of a convincing argument, but rather of the inability of consciousness in a give state to even grasp it. Guenon here points out that Aristotle did grasp it:

\begin{quotex}
For Aristotle, pure intellect is of a transcendent order and can claim knowledge of universal principles as its proper object; this knowledge, which is not discursive in any respect, is acquired directly and immediately by intellectual intuition.

\end{quotex}
This directly acquired, not-discursive, knowledge is called “knowledge of the heart”. This is opposed to a lower form of knowledge that is mediated by sensual images, or even thoughts:

\begin{quotex}
[Aristotle] said that man [as an individual] never thinks without image.

\end{quotex}
\paragraph{States of Being}
Contemporary philosophy restricts the Person to the human state, for the reason stated above: the Ego is assumed to be the Self. This is a failure to understand the medieval texts.

Dante, who claimed to be following Thomas, certainly knew other states of being. There are the subhuman states of hell, states in purgatory, and more famously, the transhuman states of the planetary, celestial, and angelic spheres. Ultimately, he was aware of heavenly states.

\paragraph{Metaphysical Realization}
In “The Eternal Ideas” from \emph{Miscellanea}, Guenon brings up the common misunderstanding that there is a distinction between the possible and the real, by regarding the unmanifested possibilities as merely \textbf{virtual}. Guenon makes clear:

\begin{quotex}
There can be nothing virtual within the Principle but, on the contrary, only the permanent actuality of all things in an `Eternal Now'. and it is this very actuality that constitutes the sole foundation of all existence.

\end{quotex}
The Self “always remains unmanifest, not being affected or modified by any contingency.”

This clearly indicates that there is no psychological explanation of the Self because it transcends all phenomenal reality. The Self remains virtual, that is, a mere intellectual concept, until there is a consciousness of oneself as that Self. G concludes:

\begin{quotex}
From the moment one recognizes that the existence of manifested beings in all their positive reality can only be a participation in principial Being, there cannot be the slightest doubt about this matter. … \textbf{What is virtual is not our reality within the Principle, but only the awareness we may have of it as manifested beings}, which is obviously something quite different; and \emph{it is only through metaphysical realization that this awareness of our true being, which is beyond and above all becoming can become effective}, that is actualized in the awareness … an awareness of that which we really are principially and eternally, and \emph{this is in the most absolutely real sense possible}.

\end{quotex}
Barfield explains it this way:

\begin{quotex}
Participation ceases to be conscious precisely because we cease to attend to it. … participation does not cease to be a fact because it ceases to be conscious. It merely ceases to be what I have called `original' participation.

\end{quotex}
The corollary is that ignorance is just “forgetting”, we cease to be conscious of what we know. As the following discussion shows, we can even be ignorant of our true Self. So \textbf{virtual} does not mean \textbf{unreal}, rather it is tantamount to unawareness or, in Barfield's term, unrepresented. Yet what is virtual still exists, one is simply unaware of it.

\emph{So the more you can become conscious of, the closer you are to, the true Self. Don't forget}.



\flrightit{Posted on 2021-09-30 by Cologero }
