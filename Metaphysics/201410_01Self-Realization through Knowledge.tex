\section{Self-Realization through Knowledge}

According to \textbf{Fr Reginald Garrigou-Lagrange}, there are three orders of knowledge: human, angelic, and divine. They each have their proper object of knowledge:

\begin{itemize}
\item \textbf{Human}: Sense objects. 
\item \textbf{Angelic}: His own angelic essence 
\item \textbf{Divine}: His own divine essence 
\end{itemize}
Note the lack of symmetry in the object of human knowledge. That is because the natural man is a determinate, material being, unlike angels and God. The type of knowledge proper to man is subject-object and discursive, whereas spiritual knowledge is intuitive and direct.

To restore the symmetry, we can consider the possibility that the proper object of man's knowledge is analogously his own essence. That necessitates the transcendence of the natural human state. Actually, this is the only true knowledge, since our knowledge of sensual things is more properly classified as opinion. \textbf{Rene Guenon} asserts:

\begin{quotex}
The being assimilates more or less completely everything of which it is conscious; indeed, there is no true knowledge in any domain whatsoever, other than that which enables us to penetrate into the intimate nature of things, and the degrees of knowledge consist precisely in the measure to which this penetration is more or less profound and results in a more or less complete assimilation. In other words, the only genuine knowledge is that which implies an identification of the subject with the object.

\end{quotex}
Thus, to understand one's own essence, one must become more and more conscious of himself. This knowledge is not discursive, but rather intuitive and non-dual. Keeping in mind the principle that “knowing is being”, this means that the more we know (in this sense), the more being we have. This knowing is identical to achieving higher states.

It is helpful, therefore, to understand the sense in which angels know in order to achieve such states. Furthermore, we are commanded to “know God”. Clearly, then, this means to achieve a certain state of being, which is the idea of \emph{theosis}.

\paragraph{Angelic Knowledge}
Angels do not acquire knowledge through sense objects. They know ideas, which are in the divine mind, directly. This knowledge is both universal and concrete; that is, it is knowledge both of essence and existence. In Guenonian terms, the former is the knowledge of possibilities and the latter of those possibilities as manifested. However, their knowledge is not co-extensive with the infinity of all possibilities.

The angelic ideas represent a region of intelligible reality and each angel has his own domain. The angels themselves form a hierarchy; since this represents a difference in level of being, there must likewise be a difference in level of knowledge. Fr Garrigou-Lagrange explains:

\begin{quotex}
The higher the angel, the stronger is his intelligence and the fewer are his ideas, since they are more rich and universal. Thus, with ever fewer ideas, the higher angels command immense regions of reality, which the lower angels cannot attain with such eminent simplicity. A human parallel is the sage who, in a few simple principles, grasps an entire branch of knowledge.

\end{quotex}
We noted something similar to this previously when discussing abstract knowledge. By knowing certain mathematical or metaphysical principles, the expanse of knowledge just opens up. For the angels, their domains over human existence become increasingly more universal. There are guardian angels or daemons for the individual; then there are higher angles that represent nations, and so on.

\paragraph{Classical Theism}
Classical theism\footnote{\url{http://edwardfeser.blogspot.com/2010/09/classical-theism.html}} is the term used by philosophers to represent our conception of God. The downside of this is that it may seem that it is just one possible conception among others, which are then debatable. However, the more correct understanding is that classical theism is the correct conception and all others are simply misconceptions. (For more details from a philosopher's viewpoint, please visit the link provided.)

While it can be difficult to grasp, it is worth the effort since it reveals many problems in regard to God to be pseudo-problems. The common mind still tends to conceive of God as a Zeus-like figure, that is, a very powerful being, involved in time, interfering in the affairs of men, and so on. Even many philosophers and theologians have but a more sophisticated view of Zeus.

Fundamentally, God is Being itself. As Guenon points out, Being is a possibility of non-manifestation, hence God cannot exist as a being among other beings; furthermore, the corollary is that God is non-material. Being is the cause of Existence, by making the potential actual.

God's knowledge is not different from his being (knowing=being). He knows all possibilities, directly, as they subsist in the divine mind. In particular, God's knowledge cannot take the form, “God knows all true propositions”, as I read on a blog a short while ago.

There is a logical consequence of this, that is seldom brought out. As we pointed out last time, when I know a squirrel, for example, I am that squirrel insofar as the idea subsists in my mind. Since the idea of the squirrel, both as an abstract and universal idea, as well as each particular squirrel, we can assert that God is likewise a squirrel, a tree, and so on for everything, insofar as he knows himself. This is not quite pantheism, since we can't go in the other direction and say, for example, that the squirrel is God.

Higher than all the angels, God knows the entire world, as Guenon defines the world as:

\begin{quotex}
the entire domain formed by a certain ensemble of compossibles realized in manifestation; these compossibles must be the totality of possibles that satisfy certain conditions characterizing and precisely defining that world

\end{quotex}
He also knows the world and its future because He wills it.

\paragraph{Consciousness and Knowledge}
God is conscious only analogously to the way man is conscious. Guenon explains that total truth must be coextensive with Being and universal Possibility, while consciousness is a human state. This precludes the idea of God having intentional consciousness, that is, the duality of a subject knowing an object.

In his best book, \emph{Answer to Job}, \textbf{Carl Jung} has an interesting speculation that God is unconscious, which is true from Jung's psychological point of view, and that man is God's consciousness. While Jung tries to describe this psychologically and phenomenologically, there is a more metaphysical interpretation. God does not look out on the world, Zeus-like, from on high; rather, he knows the world, as defined above, in his own mind.

Now, man knows sense objects. Specifically, he not only knows the idea of a tree, he also senses the tree: its color, scent, the rustling of the branches in the wind, etc. God has no physical senses. However, he has a perfect knowledge of man as manifested, hence he knows the tree in the same way the man does. God is psychically unconscious in man. Jung writes:

\begin{quotex}
It is only through the psyche that we can establish that God acts upon us, but we are unable to distinguish whether these actions emanate from God or from the unconscious.

\end{quotex}
This means that we need to differentiate between the influences that impinge on our consciousness: is their source worldly or transcendent? Jung says that through the Holy Spirit, the divine child, or one's real Self, is born in consciousness. God then becomes conscious in man as man becomes conscious of God.

This is self-realization through knowledge.



\flrightit{Posted on 2014-10-01 by Cologero}

\begin{center}* * *\end{center}

\begin{footnotesize}\begin{sffamily}

\texttt{Matt on 2014-10-04 at 12:06 said: }

There was a lot to unpack from this post (which was very good by the way), and so I thought it was best to do a number of re-readings of it before making my comment.

My comment is really just about a question of clarification with respect to the last part of the post on the Answer to Job and God's relation with man. Would God becoming conscious in man be compatible with the principle of God as pure act? It seems that God becoming conscious implies a potential that is yet actualized. I wish I had a copy of Jung's book at hand so I can read it in light of your post.

\hfill

\texttt{Cologero on 2014-10-05 at 17:39 said: }

Nice catch, Matt, but the problem is just that I may have expressed it poorly. Jung does use that type of language, presumably figuratively. Language is simply insufficient. Jung's point is that the consciousness of God is lacking, but man can become more conscious of God. If God knows every person, then his becoming conscious of God is God “becoming” conscious through the person. The “becoming conscious” is a possibility of manifestation, which God knows eternally, so it is not a question of becoming from His point of view, but it is from the person's. This is not a dogma, but it is worth pondering.

\hfill

\texttt{Cologero on 2014-10-07 at 23:40 said: }

“We are not to call any man a real philosopher who is a friend of wisdom for profit, as are lawyers, physicians, and almost all the clergy, who do not study in order to know, but in order to get money or office; and if anyone would give them that which it is their purpose to acquire, they would linger over their study no longer.”

“Beyond the delightful and the useful, there is a love for wisdom for its own sake; wisdom alone is worthy of man. And just as love between men is genuine when each loves the other without reserve, so the true philosopher embraces wisdom in her totality, and wisdom in her totality opens herself to the whole man.”

\flright{\textsc{Dante}}

\hfill

\end{sffamily}\end{footnotesize}
