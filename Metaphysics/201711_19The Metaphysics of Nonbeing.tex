\section{The Metaphysics of Nonbeing}

\paragraph{Metaphysics of Being}
\textbf{Thomas Aquinas} developed a comprehensive metaphysics of Being by integrating Philosophy, predominantly via Aristotle. Plato and Plotinus entered into his system indirectly as mediated by Dionysius the Areopagite and Augustine. As such, Scholasticism is consistent with Traditional metaphysics. Of course, by Being, the Scholastics meant much more than what today's naturalists and rationalists accept as existing.

The Medieval conception of being included nonformal as well as formal states of being. From the metaphysical point of view, the nonformal states as understood theologically, may refer to superior states of the being. Even the Russian Orthodox theologian \textbf{Sergius Bulgakov} reached that understanding. In particular, he notes that the guardian angel and the person both share the same essence. In effect, that means that they are different states of the same being.

From the esoteric perspective, then, these states represent the initiatic degrees that correspond to their respective realizations. This was brought out most forcibly in Dante's \emph{Divine Comedy} which used the astrological symbolism of the planetary and stellar spheres to indicate higher and lower states of being that the initiate needs to traverse.

\paragraph{Nonduality}
\textbf{Meister Eckhart} deepened the understanding of the metaphysics of being. Thomism has a limitation that is sometimes misunderstood as a reversion to duality, with a too strong distinction between the natural and the supernatural, between what could be known by reason and what can only be known by faith. Although Scholasticism recognizes a form of knowing — intuition — that is higher than rational or conceptual knowledge, it was Meister Eckhart who developed that notion more fully.

Intuition overcomes the duality of the sensual image and conceptual thought, which are united in spiritual seeing that unites the two. Moreover, some concepts that Thomas had left to faith, can be understood on a deeper level. Specifically, these are the Incarnation and the Trinity. For Eckhart, the Incarnation is much more than an historical event, since it is repeated whenever the Logos is born again in our interiority. By intuition, Eckhart also sees into the Trinity. Since we have recently discussed these ideas, they need not be repeated here.

\paragraph{Nonbeing}
In \emph{The Multiple States of Being}, \textbf{Rene Guenon} points out that a complete metaphysic needs to include Nonbeing as well as Being, as the latter arises from the former. He points out that the West has failed to develop that notion very well. Guenon does mention the notion of the Abyss in Alexandrian Gnosticism as referring to an aspect of Nonbeing.

Independently, \textbf{Nicolas Berdyaev} in Boehme: Unground and Freedom notes that:

\begin{quotex}
Boehme was perhaps the first in the history of human thought to have seen, that at the basis of being and prior to being lies a groundless freedom. 

\end{quotex}
Perhaps, assuming Guenon is correct, \textbf{Jacob Boehme} is not the first in the history of human thought, but is at least the first in the history of Western thought to develop a metaphysic of nonbeing. Guenon's presentation of Nonbeing is rather dry and rationalistic while Boehme's comes from a deeper source, a direct intuition of Nonbeing which he sometimes calls the Unground. In the \emph{Mysterium Pansophicum}, he attempts to describe it. (It is impossible to define, since a definition is a limitation.)

\begin{quotex}
The unground is an eternal nothing, but makes an eternal beginning as a craving. For the nothing is a craving after something. But as there is nothing that can give anything, accordingly the craving itself is the giving of it, which yet also is a nothing or a desirous seeking. And that is the eternal origin of Magic, which makes within itself where there is nothing; which makes something out of nothing, and that in itself only, thought this craving is also a nothing, that is, merely a will. It has nothing, and there is nothing that can give it anything; neither has it any place where it can find or repose itself. 

\end{quotex}
Obviously, the unground is nothing, since it is not itself manifested. There are two important points that follow from this.

\begin{itemize}
\item The Unground contains all possibilities, light as well as darkness, good as well as evil, God's wrath as well as God's Love. 
\item Being arises from Nonbeing. A fortiori, there is a “craving” for Being, and thus this can be described as a primal Will. Moreover, since Nonbeing is undetermined, it must be perfectly free. 
\end{itemize}
This is just the surface, as Boehme brings an understanding to many other notions, although this is not the time to discuss them.

\paragraph{Letting Go}
In his commentary on the \emph{Secret of the Golden Flower}, a Chinese alchemical text, \textbf{Carl Jung} writes:

\begin{quotex}
The art of letting things happen, action in nonaction, letting go of oneself, as taught by Master Eckhart, became a key to me with which I was able to open the door to the “Way”. The key is this: we must be able to let things happen in the psyche. For us, this becomes a real art of which few people know anything. Consciousness is forever interfering, helping, correcting, and negating, and never leaving the simple growth of the psychic processes in peace. It would be a simple enough thing to do, if only simplicity were not the most difficult of all things. It consists solely in watching objectively the development of any fragment of fantasy. 

\end{quotex}
Jung is referring the Eckhart's idea of “letting go” or \emph{gelassenheit}. Obviously, this is more like the Hermetic teaching of concentration without effort. Simplicity is the detached awareness of whatever psychic processes may be occurring. This watching often dissipates any interference in the psychic process. This letting go then allows something else, something much deeper, to arise. There is a primal will, different from purposeful and conscious willing, that is in touch with more possibilities. In a phrase reminiscent of Jacob Boehme, Jung explains this idea:

\begin{quotex}
Whether arising from without or within, the new thing came to all those persons from a dark field of possibilities; they accepted it and developed further by means of it. It seemed to me typical that, in some cases, the new thing was found outside themselves, and in others within; or rather, that it grew into some persons from without, and into others from within. But it was never something that came exclusively either from within or from without. If it came from outside the individual, it became an inner experience; if it came from within, it was changed into an outer event. But in no case was it conjured into existence through purpose and conscious willing, but rather seemed to flow out of the stream of time. 

\end{quotex}
The important notion is that a change from within brings the outer event into existence. The art of “letting things happen” in the psyche, or action in nonaction, needs to be developed. Then we may experience synchronicities in our lives, or perhaps better said, “grace”. In other words, the pursuit of the outer needs to yield to an interior transformation. This ties in nicely with what Eckhart writes in Sermon 77 about asking God for gifts:

\begin{quotex}
I say, `God is Love,' because He must love all creatures with His love, whether they know it or not. … I will never pray to God for His gifts, nor will I ever thank Him for His gifts, for if I were worthy to receive His gifts He would have to give them to me whether He would or not. Therefore I will not pray to Him for His gifts, since He must give: but I will surely pray to Him to make me worthy to receive His gifts, and I will thank Him for being such that He has to give. Therefore I say, “God is Love,” for He loves me with the love with which He loves Himself: and if anyone deprived Him of that, they would deprive Him of His entire Godhead. Though it is true that He loves me with His love, yet I cannot become blessed through that: but I would be blessed by loving Him and be blessed in His love. 

\end{quotex}
\paragraph{The Purification of the Will}
That is a very important lesson in how to pray. There is no point to pray to God for his gifts, because he is by nature willing to disperse his gifts freely. Rather, the prayer should be to be made worthy to receive the gifts. (Seek ye first the kingdom of God, and his righteousness; and all these. things shall be added unto you.) Thus, the inner transformation comes first, and only then perhaps the external things and events.

To become worthy means to please God. That is the purification of the will. Failure to achieve that will bring obstacles to life. The introduction to the Mass of the 20th Sunday after Pentecost in the Roman Missal makes this clear:

\begin{quotex}
The Liturgy shows us that our misfortunes are caused by our unfaithfulness in conforming to the will of God. Let us beseech the Lord, through the prayers of Holy Church, to pardon our sins, so that we may serve Him with a quiet and trustful heart, always obeying His precepts. 

\end{quotex}
And from the Introit:

\begin{quotex}
Blessed are the undefiled in the way; who walk in the law of the Lord. 

\end{quotex}
\paragraph{Freedom and License}
To be “undefiled” is to be made worthy. Undefiled = pure, so our previous tasks of the purity of the mind and the purity the will must not be forgotten. The result is to walk in the law of the Lord. For Eckhart, that is not a restriction on freedom as the world wants to believe, but is rather freedom itself. He writes:

\begin{quotex}
For the man who stands in God's will and in God's love it is a joy to do all the good things God wills, and to leave undone all the evil things which are against God. And it is impossible for him to leave a thing undone which God wants to have accomplished. As it would be impossible for one to walk whose legs are bound, so it would be impossible for one to do ill, who is in God's will. 

\end{quotex}
Eckhart is emphatic that this does not mean the license to do anything at all:

\begin{quotex}
Some men say: If I have God and God's freedom, then I can do everything I want. They understand these words amiss. As long as you can do anything which is against God and His commandment, you do not have God's love; you can only deceive the world into the belief that you have it. 

\end{quotex}
This is opposite of the contemporary view that if any action motivated by “love” is thereby legitimate. However, that idea is still focused on the “self”, but the self is precisely that which occludes the Will of God. The attachment to sensory images and intellectual ideas are obstacles to spiritual vision. If redemption is the recovery of the primordial state prior to the Fall, this state can be understood only by a pure mind. \textbf{Wolfgang Smith} elucidates this point:

\begin{quotex}
The Patristic understanding of Paradise is mystical in two respects: first, because it insists that the nature of Paradise exceeds categorically what the “carnal man” — St. Pauls \emph{psychikos anthropos} — is able to comprehend; and secondly, because it claims that the things of Paradise can in fact be “seen” when certain degrees of contemplation have been attained. St. Gregory the Sinaite speaks of this explicitly when he explains “the eight primary visions accompanying the state of perfect prayer when things previously happen are clearly beheld and known by those who have attained by grace complete purity of mind.” 

\end{quotex}
Eckhart knows that teaching also:

\begin{quotex}
Not an already existing life — “being” — is to be understood in the logical sense; but the higher understanding — “seeing” — is itself to become life; the spiritual, that which belongs to the idea, is to be experienced by the seeing man in the same way as the individual human nature experiences ordinary, everyday life. 

\end{quotex}
For that to occur, depth is required, where there is a perfect “morality”, i.e., the Will of God is known. The world lives on the surface: it looks for understanding (i.e., “explanations”) and the satisfaction of desires. There is a fear of “letting go”, for the end of understanding is experienced as darkness. However, God's light shines precisely in that darkness. Eckhart writes:

\begin{quotex}
For everything the understanding can grasp, and everything desire demands, is not God. Where understanding and desire have an end, there it is dark, there does God shine. There that power unfolds in the soul which is wider than the wide heavens. The bliss of the righteous and God's bliss is one bliss; for when God is blissful, the righteous are blissful. 

\end{quotex}


\flrightit{Posted on 2017-11-19 by Cologero }

\begin{center}* * *\end{center}

\begin{footnotesize}\begin{sffamily}



\texttt{jonathansolvie on 2017-11-20 at 10:22 said: }

Great work!

“For Eckhart, the Incarnation is much more than an historical event, since it is repeated whenever the Logos is born again in our interiority.”

This is true with the qualification that the esoteric understanding of the incarnation of the Logos within the initiate by no means dispenses with the objective and macrocosmic significance of the Incanation that took place in the historical Jesus Christ, non-dually uniting time and Eternity, concrete event and universal metaphysical truth and symbolism. While the Logos that incarnates in any initiate of the Greater Mysteries may be principally non-dual with Christ (“I am crucified with Christ, but I live; yet not I, but Christ liveth in me”), the incarnations are are from equal in the persons of Jesus of Nazareth and Meister Eckhart in terms of total manifestation and macrocosmic and eschatological function. Yes, the incarnation of the Logos reflects in the mirrors of many Hearts, but there nevertheless needs be one Incarnation in the unfolding within time of a human world-cycle that is unique, preeminent and unsurpassable, as the highest peak of personal divine manifestation within creation. Therefore the above esoteric interpretation of Incarnation within our own Heart depeens our understanding of the principle of Incarnation, but does not exhaust the esoteric depth of the historical Incarnation, which has a macrocosmic dimension as well, the delving into which would allow us to understand also the literal truth of exoteric dogma regarding the Incarnation as an objective and historical macrocosmic and eschatological event, and yet in more profound hermetic depth.


\hfill

\texttt{jonathansolvie on 2017-11-20 at 15:43 said: }

Some might assume from reading this article that the `Unground' is the Absolute, since Being is said to proceed from it. However, the Absolute Reality, the Gottheit, is not nothingness or non-being as opposed to Being; both of these are `dependently originated' and mutually conditioned. Since non-being stands in relation to Being, it is not undetermined. The very `craving' explained as arising from the Unground testifies to its limitation and its need for Being which stands as its opposite pole. If from one perspective Being arises from the former, both derive their reality from the unconditioned freedom of the Absolute. The undetermined and infinite Reality is beyond both Being and non-being, the manifest and the unmanifest, and the latter is not superior to the former in face of That. 

I just thought this might be worth pointing out for certain readers; I do not mean to correct the author, who undoubtedly knows this better than I.


\hfill

\texttt{Fred on 2017-11-21 at 16:05 said: }

I agree with jonathansolvie.

Possibly the Absolute is even beyond Being and Non-Being, not only beyond all affirmations but also beyond every negation.

There was a time where I tried to study the highest metaphysics, then a monk told me studying the mystics is of no benefit if you are struggling with the purification of beginners. I'm much more comfortable with the daily fights now.

I will keep following this advice until it's time.


\hfill

\texttt{Cologero on 2017-11-21 at 21:18 said: }

Thanks, Fred and jonathansolvie, for making that clarification. Yes, I was unclear about Non-being and the Absolute. Moreover, I had in mind the Absolute as universal Possibility, without explicitly stating so. There is certainly work to do in adequately relating Boehme's insights to traditional metaphysics. This will be an ongoing project.


\hfill

\texttt{Cologero on 2017-11-21 at 21:22 said: }

Yes, jonathansolvie, those are good points. The esoteric is not intended to replace or supersede the exoteric. That is what you may find in certain Gnostic or Theosophical circles which claim to reveal the “true” meaning of the exoteric.


\hfill

\texttt{Caeliger on 2017-11-22 at 03:42 said: }

Non-Being is not merely the `void'. which follows the primordial `contraction' or expiration of God (needless to say, this must happen `before the foundation of the world’), but we could say that the void is one of its `aspects'. one which will explore the possibilities of the One and the many. 

It could certainly be taken in a limiting sense as `that which is opposed to Being'. in which case it is indeed necessary to conceive a yet higher priciple which is truly infinite.


\hfill

\texttt{Arthur Konrad on 2017-11-22 at 11:50 said: }

“Some men say: If I have God and God's freedom, then I can do everything I want. They understand these words amiss. As long as you can do anything which is against God and His commandment, you do not have God's love; you can only deceive the world into the belief that you have it. ”

This is valid only if interpreted on non-moralistic grounds. Otherwise it makes no sense. There is either inner freedom and hence inner commandment, or you do not posses inner will at all. Discovering inner will absolutely does not equate discovering any commandments not one's own, since one's fate is then already perfectly clear and resolved (hence utter futility in attempting to draw parallel between inner will and religious law). Ekchart has deployed a rather worn out linguistic device here to draw a parallel between apparently religious commandment and an inner will, which demands an esoteric interpretation to overcome it's obstacles.


\hfill


\hfill

\texttt{Numen on 2017-11-23 at 01:24 said: }

jonathansolvie's clarification is excellent and succinct.

S.H. Nasr once explained in a presentation that the basis of the theme of the void in Islamic art stems from the understanding that “if the world is everything, then God is nothing; If God is everything, then the world is nothing.” In a sense, our understanding of God has to oscillates between the two (Being and Non-being); the Non-being of God has to be a part of the transcendent principle, otherwise God is seen just merely as Being, part of the manifestation itself. Or, as Eckhart put it, God is “Being above Being and superessential Negation.”

Dante's intuition is nicely simplified when he wrote: “as certain things are affirmed to exist which our intellect cannot perceive (namely God, eternity, and primal matter), things which most certainly are known to exist and are with full faith believed to exist. But given the nature of their essence we cannot understand them: only by negative reasoning can we approach an understanding of these things, and not otherwise.”


\end{sffamily}\end{footnotesize}
