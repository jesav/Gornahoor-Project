\section{Creating a World: Day One}

It's curious that even atheists and neo-Darwinists are sometimes willing to believe that the human race was seeded by aliens or even that we ``inhabit" a computer simulation created by more advanced beings; in other words, the implication is that the world was designed, albeit not by God. It would be an interesting thought experiment to design such a world. After all, games like Roblox and Minecraft are so popular in part because they allow the creation of beings with various characteristics and destinies in life. A being that was a genuine ``blank slate" with no innate characteristics and no life purpose would certainly not be very popular, even though people claim to deny any such innate characteristics.

Moreover, in fulfilling their destinies, the characters will experience conflict and obstacles. Software such as Dramatica will help you create conflicting characters and plots if you need help.

But first, before moving onto human and other living beings, we need to start with physics.

\paragraph{Possibilities of Manifestation}
\textbf{Rene Guenon} in The Multiple States of Being describes the properties of Existence, that is, manifested being. The Infinite contains all possibilities which can be in one of two categories. These will set the standards for our world.

\begin{itemize}
\item Possibilities of non-manifestation. These possibilities can never be manifested. Hence, they belong to Non-being 
\item Possibilities of manifestation. These are part of Being. 
\end{itemize}
The following two sections illustrate some of these points.

\paragraph{Non-manifestation}
\begin{quotex}
The absurd, in the logical and mathematical sense, is that which implies contradiction; it is therefore identical with the impossible, for it is the absence of internal contradiction that defines possibility, logically as well as ontologically. \flright{\textsc{Rene Guenon}, \textit{The Multiple States of Being}}

\end{quotex}
I've read objections to this idea, since it seems to put a limit on God's power. Of course, that is not the case, since all possibilities — both of non-manifestation and of manifestation — exist in God's mind. That totality of possibilities comprises the Wisdom of God. Hence, it is not really a limitation of God, just a limitation of manifestation. Here are some examples of the possibilities of non-manifestation.

\begin{itemize}
\item Obvious self-contradictory statements like a ``round square" or ``married bachelor". Unfortunately, self-contradictory statements are not always so obvious as many philosophical systems contain their own refutation. In everyday life, people will make high sounding claims without realizing they are often oxymorons. 
\item The past cannot be changed 
\item There can be no actual infinity. Only God is Infinite. 
\item e.g., infinite space would be as Parmenides described, static being with no change because time could not exist were space infinite. 
\item The same for time, there can be no infinite past. Thomas Aquinas seems to have gotten this wrong but Bonaventura got it correct. 
\item Void, or said another way, Nature abhors a vacuum. 
\item A Being cannot exist in the same state twice, since that would make it the same being. 
\item There can be no system of arithmetic that is both consistent and complete. 
\item The Traveling Salesman problem is non-computable. 
\item A Free being cannot be made unfree. 
\end{itemize}
\paragraph{Manifestation}
\begin{quotex}
Whatsoever does not involve a contradiction, God can do in his creatures. \flright{\textsc{Thomas Aquinas}, \textit{Commentary on the Sentences}}

We must distinguish this logical necessity, which is the impossibility of a thing's not being or being other than it is — and this independently of any particular condition — from what is called `physical' necessity or necessity of fact, which is simply the impossibility that beings and things could fail to conform to the laws of the world to which they belong, this latter kind of necessity consequently being subordinate to the conditions by which that world is defined, and which are valid only within the special domain concerned. \flright{\textsc{Rene Guenon}, \textit{The Multiple States of Being}}

\end{quotex}
Possibilities of Manifestation have two subcategories:

\begin{itemize}
\item Those which have been, or are, manifested 
\item Those which have not been manifested 
\end{itemize}
The first principle of manifestation is that creatures are composed of matter and form; the former indicates ``that it is" and the latter ``what it is". That is the view of the Scholastics as well and the Nyaya school of philosophy:

\begin{quotex}
Compare the Nyaya view with that of St Thomas that the primary object of man's knowledge is a synthetic unity in which both the senses and the understanding play their indispensable part. The individuation or the quantitative specification is derived from the sense, while the qualitative unity is from the understanding. The object of knowledge contains within itself the intuition of essence and the sense knowledge of particulars. It is neither the essence alone, as Descartes thought, nor the sense datum alone, as the empiricists believed. We know things and things are neither disembodied essences not subjective images. To separate universals from individuals is to miss the unity of the two in things. \flright{\textsc{Sarvepalli Radhakrishnan}, Indian Philosophy Volume II}

\end{quotex}
The consequence is that knowledge comes through the senses first. Hence, the senses cannot be deceived by God. I don't know which came first – the chicken or the egg, but I've never encountered a chicken that didn't start life as an egg. When Adam ``named" the animals, that was an acknowledgement that he understood their essences. If the animals appeared fully grown, that would not have been possible, since the essence of a chicken includes its full development. Thomas Aquinas writes about this in the Summa:

\begin{quotex}
Our intellect, which takes cognizance of the essence of a thing as its proper object, gains knowledge from sense, of which the proper objects are external accidents. Hence from external appearances we come to the knowledge of the essence of things. 

\end{quotex}
As an example of what Guenon calls a ``necessity of fact", we can point to the digestive system of a lion which would make it impossible for the lion to survive on a vegan diet. This is implied by Thomas, again from the Summa:

\begin{quotex}
In the opinion of some, those animals which now are fierce and kill others, would, in that [Edenic] state, have been tame, not only in regard to man, but also in regard to other animals. But this is quite unreasonable. For the nature of animals was not changed by man's sin, as if those whose nature now it is to devour the flesh of others, would then have lived on herbs, as the lion and falcon. … Thus there would have been a natural antipathy between some animals. 

\end{quotex}
\paragraph{Conditions of Manifestation}
There are five conditions of manifestation that are necessary for there to be any experience at all. These are:

\begin{itemize}
\item \textbf{Space and Time}. Space is geometry, coexistence, visible objects. Time is arithmetic, succession, things produced and destroyed. 
\item \textbf{Matter and Form}. Prime matter is malleable. Form is the idea or essence of a thing. 
\item \textbf{Life}. This is the qualitative aspect of the world, its ``innerness". Everything is conscious to some degree. Teilhard de Chardin asserted that there is an inside as well as an outside to the world. 
\end{itemize}
The physical world requires five elements: earth, water, light (or fire), air, and akasha (or ether); matter is a mixture of those elements. The following descriptions are taken from the Vaisesika school of Hindu philosophy.

\begin{itemize}
\item \textbf{Earth}. Solid. Qualities: smell, taste, colour, tangibility. 
\item \textbf{Water}. Liquid. Qualities: taste, colour, tangibility. 
\item \textbf{Light}. Luminous. Qualities: colour, tangibility. 
\item \textbf{Air}. Gaseous. Quality: tangibility. 
\item \textbf{Akasha}. Etheric. Quality: sound. 
\end{itemize}
Moreover, the Vaisesika school has a well-developed atomic theory (not in the chemical sense of the term). It is more advanced that the Greek atomic philosophers, for whom the atoms are quantitative and without qualities. The Greeks furthermore claimed that even souls consisted of atoms, which the Hindu school denies; hence, it is more compatible with Life as a condition of manifestation. Guenon claims that space is infinitely divisible, something that the Vaisesika obviously denies. Although not the most advanced of the orthodox Hindu schools, it is useful for our purposes since it is more compatible with physics. Instead of atoms, physics regards quanta as ultimate and indivisible. (Of course, in contemporary physics, continuous waves also have validity.)

\paragraph{Physics}
The goal in world creation is not to force Revelation to conform to the latest fad in physics; no, that would be contrary to the real order of things. Rather, the goal is to see what physics would have to be like, given that Revelation is true.

Since time is not infinite, the world had a beginning, let's say at Time 0. Therefore, space likewise had a beginning. Since the Void (empty Space) cannot be manifested, Space at Time 0 would only be a point.

\textbf{St. Clement of Alexandria} described the six directions of Space as East/West, North/South, Up/Down; that is the symbolism of the cross. The origin is the point at which they all meet. There are also six periods of time, but that is another discussion.

Physics cannot describe what was happening at Time 0 and the Point of Origin, since it is a singularity. In other words, Time 0 is not manifested, i.e., it is a part of Non-Being, hence invisible to physics. In \textit{The Symbolism of the Cross}, Guenon explains:

\begin{quotex}
his centre directs all things by its ``actionless activity" (wei wu wei), which although unmanifested, or rather because it is unmanifested, is in reality the plenitude of activity, since it is the activity of the Principle whence all particular activities are derived; this has been expressed by Lao-tze as follows: ``The Principle is always actionless, yet everything is done by it." 

\end{quotex}
In the \textit{Commentary on the Sentences}, Thomas Aquinas writes:

\begin{quotex}
Non-being is prior to being in the thing which is said to be created. … The creature should have non-being prior to being in duration, so that it is said to be out of nothing 

\end{quotex}
To summarize:

\begin{itemize}
\item Non-being is the principle of Being 
\item Time 0 at the Origin Point is like the centre of a wheel: it does not move, yet everything moves around it. 
\item Time (duration) is created with creation. 
\end{itemize}

\paragraph{Day 1}
\begin{quotex}
And God said, ``Let there be light," and there was light. \flright{\textsc{Genesis 1:3}}

\end{quotex}
According to Genesis, Heaven and Earth were created on Day 1. That means Form and Matter, the preconditions for a World. The Earth was nothing in particular, because the world first manifested as Light. So this is our first clue to physics:

\begin{itemize}
\item Matter can exist in any one of three forms: solid, liquid, and gas. This is clear. 
\item Matter and energy (i.e., light) are also convertible into each other. That was proposed and demonstrated last century. 
\item At times close to Time 0, there was only light (i.e., photons) since the other states of matter could not exist at such densities. 
\item Light vibrates at certain frequencies. Given the amount of energy, that frequency would be extremely high. 
\item The higher the frequency, the more information is can carry. Information is the opposite of entropy. Since the Idea of the entire universe was present at Time 0, this Idea was carried as information in the frequency of the energy present at creation. 
\end{itemize}
The objection is usually brought up that the Sun was not created until Day 4, so how could there be light on Day 1? However, we see that there was only light at Day 1 in the form of photons (the ``atoms" of light).

As the universe cooled down, matter existed as pure potentiality as described by Schrodinger's equation; that is, it was nothing in particular, i.e., without form. That is the state of Prime Matter.

TO BE CONTINUED

\flrightit{Posted on 2020-06-14 by Cologero }
