\section{Philosophy and the Noble Life}

\begin{quotex}
Because the same One, who is begotten and born of God the Father, without ceasing in eternity, is born today, within time, in human nature, we make a holiday to celebrate it. Saint Augustine says that this birth is always happening. And yet, if it does not occur in me, how could it help me? Everything depends on that. \flright{\textsc{Meister Eckhart}}

Thus, in the gospel He speaks through the flesh; and this sounded outwardly in the ears of men, that it might be believed and sought inwardly, and that it might be found in the eternal Truth, where the good and only Master teaches all His disciples. \flright{\textsc{Saint Augustine}}

\end{quotex}
\paragraph{Nobility and Nature}
In the \textit{Convivio}, \textbf{Dante} explained that

\begin{quotex}
Nobility is the perfection in each thing of its proper nature.

\end{quotex}
Thus man speaks of a noble stone, plant, horse, falcon, whenever it appears perfect after its own nature. Thus, a man is noble to the extent he is perfect according to his nature. Unlike a stone, a plant, or an animal, whose nature is simply given to them, man, as a free and rational being, chooses to follow, or not, his nature freely. This leads then to the question of how one should live.

In Vedic teachings, men are said to be motivated by pleasure, success, or duty. These motivations should be clear, since all men have experienced them. Pleasure is restricted to sensual pleasure, success may take the form of founding movements, etc. Duty, is higher than the other two, since it is directed outward, yet it is ultimately insufficient. Kierkegaard, in \textit{Either/Or} provides a detailed phenomenological analysis of both a life devoted to sensuality and to a life devoted to the sense of duty, for those interested.

Beyond those three, there is liberation from all desire, the transcendence of one's self and nature. However, the Western path of affirmation seeks instead to affirm one's self and nature, i.e., nobility is the fourth motivation. How that is to be done was Dante's path. That was philosophy which “has wisdom [sophia] for her subject matter and love for her form, and for the composition of the one with the other the practice of speculation.”

\paragraph{True Life}
The Emperor of Suabia said that nobility requires “ancient wealth and gracious manners”, still commonly believed, but a definition that Dante rejects since lineage and money are not one's own nature. Rather, he writes:

\begin{quotex}
Life in man is the exercise of the reason; if his life is his being, then to renounce the exercise of reason is to renounce existence, and this is death.

\end{quotex}
Or better put, “the man [who rejects reason] is dead but the beast survives.” Do not be deceived. Many walking among you may appear to be gracious, to be of a good stock, to have fortune, success, or a glib, sophistical tongue, yet are mere “beasts of the field.”

Some even use the expression “life denying”—not a traditional notion—to refer to the path Dante describes. However, what is truly life denying is a life devoted only to sensuality, at least if you mean a man's life and not a beast's.

\paragraph{Law and Morality}
Yet, the path to Sophia, or Wisdom, involves not just Reason, but also Love. It is not just a matter of thinking, but also of transformation. Dante writes, in a passage that may seem hard to grasp:

\begin{quotex}
The beauty of wisdom is in morality; the moral virtues give pleasure which is sensibly perceived.

\end{quotex}
The dutiful life, such as that of Kierkegaard's Judge Wilhelm, is not pleasurable. So what can Dante mean here? As a poet, he uses images, which can be “sensibly perceived”, to make his point. This Sophia is not an abstraction, but takes on the form of a woman, in his case, Beatrice. Thus the love for Sophia is more than intellectual. In this, he is not far from Boethius.

Evola often wrote that morality cannot be imposed from the outside, but must originate in the “I”. Dante cannot disagree, since morality arises from man's nature. For example, the four pagan virtues of justice, wisdom, courage, and temperance follow from the very nature of the soul in its thinking, willing and feeling functions. Furthermore, the natural law derives from the very nature of man, so to follow it means not more than to live as a man. To reject it is to die.

Guenon complained that Christianity lacked a true law. However, pace Guenon, there is no need for a new law, since the Logos itself is the law. To live in accordance with the Logos, through whom all things were created, is the law. This is unlike all other traditions which have elaborate, and seemingly arbitrary, laws in regards to the regulation of daily life. These include dietary restrictions as well as things that make one unclean, and the rituals necessary to become clean again. What Evola always neglects to mention is that ancient paganism itself had many such laws, from how to keep the hearth going in one's home to whom a man could touch or not. Christ brought no such laws since uncleanness arises in one's heart, not from the outside. Vladimir Solovyov writes:

\begin{quotex}
The law of God's being is no longer manifested a pure arbitrariness (in Himself) and external, compulsory necessity (for humanity); it is manifested now as the internal necessity of true freedom.

\end{quotex}
\paragraph{Nobility and the True Self}
So by living a life of reason, developing the virtues, and acting in harmony with one's true nature, a man becomes noble. But then something interesting happens, specifically the developing awareness of the Self. This is a topic we have previously discussed in relation to \textbf{Otto Weininger}\footnote{\url{https://www.gornahoor.net/?p=57}}. But it also explains why we draw on the likes of \textbf{Julius Evola} or \textbf{Miguel Serrano}; they came to the same understanding. Whether or not they ultimately took a wrong path is for the reader to decide. The point is that this is a path different from that of the East, and accounts for much of the dispute between Evola and Guenon. Evola wrote of the “Absolute Self”, which he claimed was a pagan notion, although I cannot locate a specific source. On the contrary, if we arguably take Plotinus as the highest expression of ancient pagan spirituality, we see that he is far from that notion and is actually much closer to the spirituality of the Vedantists of the East.

Serrano is more honest about the source. In an interview\footnote{\url{http://oregoncoug.wordpress.com/2011/11/23/miguel-serrano-interviewed-by-alessandro-de-felice-2005/}}, he refers to \textbf{Carl Jung}:

\begin{quotex}
[Jung] told me something very curious, very interesting about what happened here in the Occident. Our world, our development went forward quite naturally through Paganism. It was interrupted by another civilization that imposed itself on the Pagan Occident and cut off its advance. Namely, the Roman Empire and Christianity with Charlemagne and all of that. This cut away the possibility of developing the Pagan civilization. Wotan was put away as something apart that could not be touched. This produced in the blood of the unconscious of this Folk a break by which another was imposed onto it and this dichotomy was an internal conflict. Catholicism produced this with the concept of sin.

\end{quotex}
Of course, the concept of sin arises from the failure to follow the natural law; this produced an uncomfortable friction in the pagan mind. In it, individuality was less prominent, so the pagan consciousness simply followed along with the group mind. Even today, among the neo-pagans and alt-rightists, there is the ever constant temptation to want to revert back to the racial mind. Thus they use expressions along the lines of “I want to get back to the psyche (or something similar) of my people.” This difference can be seen in the Swedish legend of the Virgin Spring\footnote{\url{https://www.gornahoor.net/?p=1416}} in which the nobility is Christian and the lower classes still follow the old religion.

However, on the positive side, this friction:

\begin{quotex}
allowed a great surpassing, something incredible that had never occurred before, perhaps unique in the universe, the awareness of oneself, the consciousness of the I, the \emph{Selbst}. 

\end{quotex}
That is, man becomes aware of himself as choosing to act in accordance with his true nature, i.e., against his sin nature. This is unlike a rock, plant, or animal which has no choice in the matter.

As the West has become de-Christianized, it has lost this idea of the Self. Thus it has reverted back to the group mind, as evidenced in political correctness, etc. To the extent the self is recognized, it is only insofar as it is in service to pleasure, success, or duty. None of these are viewed as threats to the system.

\paragraph{Theosis}
At its highest level, then, the Self unites to the highest self, the Logos, or “Christ lives in me.” To avoid further accusations of the ancient heresy of Gnosticism, we can now be quite clear what that means. Guenon correctly writes that liberation represents the actualization of all possibilities of the Being. However, that cannot describe the annihilation of the Self as he intended, but rather the divinization of the Self, or theosis.

For God, essence and existence are One, i.e., there is no potential in God, but only actuality. Hence, for a being to actualize all his possibilities, he is left without potential, specifically, this makes him the same as a god. However, there can be only one such God.

So we can say that a person becomes united to God to the extent he manifests all his possibilities. Of course, we mean here his “real” possibilities, not the sinful or deviant possibilities he may be attracted to. More about this will have to wait.



\flrightit{Posted on 2013-12-29 by Cologero }

\begin{center}* * *\end{center}

\begin{footnotesize}\begin{sffamily}



\texttt{Cologero on 2013-12-30 at 10:09 said: }

Does anyone find it striking that some, or perhaps most, men are described as beasts by Dante? We find a similar idea, including the life of reason as the moral life, in St Anthony\footnote{\url{http://www.gornahoor.net/?p=501}}. He also describes irrational and immoral men as “inhuman”.

Nowadays, however, everyone is described as a “child of God”, no matter what. Hasn't something significant changed in the mentality of Western man over the last several centuries?


\hfill

\texttt{Cologero on 2013-12-30 at 10:17 said: }

Furthermore, the ideal of a moral life has changed. No one today would call that ideal “noble”, perhaps because it seems so prideful. No one, or hardly no one, today would talk of the virtues in the same way, or even acknowledge that man has a certain nature. No, nowadays, the virtuous or moral man is someone who is concerned about the “poor” or the “hungry”.


\hfill

\texttt{Synodius on 2013-12-30 at 10:46 said: }

” So we can say that one becomes united to God to the extent he manifests all his possibilities. Of course, we mean here his “real” possibilities, not the sinful or deviant possibilities he may be attracted to.”

Jung's ideal of becoming whole seems to lack this distinction and he saw the lack of development of these inferior possibilities as a symptom of spiritual sterility.


\hfill

\texttt{Avery Morrow on 2013-12-30 at 13:15 said: }

“To live in accordance with the Logos, through whom all things were created, is the law. This is unlike all other traditions which have elaborate, and seemingly arbitrary, laws in regards to the regulation of daily life.”

The Tao that can be named is not the true Tao…


\hfill

\texttt{Ash on 2013-12-30 at 15:50 said: }

Synodius, I think the idea isn't to ignore the sensual part of man, but rather to develop it in accordance with the higher nature, rather than the reverse. Dr. Ali Shariati, an Iranian Muslim thinker, wrote that “…in Islam Satan is not standing against God but against the divine half of man. And since man is a two-dimensional creature who is kneaded of mud and God, he is in need of both. His ideology, religion, life, and civilization must all be capable of satisfying both of these dimensions.” The nobility produced culture and art in order to uplift man on the sensual and physical plane, whereas much in modern culture tries to bring down the spiritual plane to the basest elements of the physical one. 

The moment that the lowest passions of Man were judged to be as legitimate and equal and beyond judgement as the highest possibilities, the moral lauding of the poorest and most deprived parts of human society was followed naturally. I see a lot of people, even within the Catholic Church, who go beyond lionizing those who help the poor and ascribing a sort of dignity and virtue to poverty in and of itself. St. Francis may have found a nobility of spirit by subjecting himself to poverty, but the fact that it was meant as a struggle for the soul should tell them all it needs to.


\hfill

\texttt{Synodius on 2013-12-30 at 18:54 said: }

Perhaps I didn't express myself clearly, I think that Jung contributed to the confusion of values by considering one-sidedness the greatest sin and not bothering to distinguish higher possibilities from lower.


\hfill

\texttt{JA on 2013-12-30 at 21:56 said: }

For the record, Ali Shariati was a Marxist who opposed the Persian monarchy as well as the legitimate clergy of Islam, he called himself a red shiite in one of his works ! His version of Islam, and also that of his contemporary and ally Khomeini is a Mohammedan version of communist liberation theology.

I am not an expert on Islam as I am a Catholic by baptism, ancestry and Faith, but I respect Moslems such as Seyed Hosein Nasr who was btw an advisor to the late Persian Shah. 

I will comment on the article itself tomorrow when I have had more time to meditate – this is really good and crucial stuff Cologero has given us here.


\hfill

\texttt{Ash on 2013-12-30 at 22:06 said: }

Thanks for the clarification, JA. Full disclosure: I had read a couple of Shariati's works on this subject but wasn't aware of his full spectrum beyond being involved in the revolution. That said, I don't see anything unkosher about the quote I have in and of itself, provided we interpret it correctly. Perhaps others would be willing to critique it if they do.


\hfill

\texttt{JA on 2013-12-31 at 16:48 said: }

I think the key here to discovery of the elitist nature of the Catholic religion is to understand the nature of mystical theology – through our faith in Christ combined with our good works, we all can achieve theosis after our death. However – some (read elite) have been given a gift by God to achieve theosis while in the mortal body. Unlike modern teachings which say any bum can become enlightened, Catholicism teaches that only God can choose whether or not we will see Him in this life, all we can do is devote ourselves in the prayer and hope that God will so bless is with this most wondrous of events. 

the error that all can achieve theosis without the special grace of God is what leads to egalitarianism.


\hfill

\texttt{Cologero on 2014-01-02 at 20:24 said: }

Yes, Avery, that is the point. “Tao” is used to translate “Logos” in the Chinese bible.

If all things are created through the Logos — and thus can be “named” — then the Logos cannot be found among the 10,000 things.

The Logos is “known” through gnosis, not as an object (Christ liveth in me).

When there is not distinction between essence and existence, there is free expression and spontaneous action — non-acting acting


\hfill

\texttt{Andrew on 2014-01-18 at 18:44 said: }

RE: Cologero's early comments regarding the change in the mentality of Western man, the criteria of the moral man being concerned about the “poor” and “hunger”…

Chantal del Sol, a contemporary French critic/philosopher, addresses this in her book, “Icarus Fallen”, as well as her other work. I won't go into it here, but her work may be of interest to some of you.


\end{sffamily}\end{footnotesize}
