\section{A Matter of Choice}

\begin{quotex}
From this deeper principle you must do your works, without a why. I affirm it decisively: even if you do your works for the kingdom of heaven, for God, or your sanctity, although motivated by the other, even then you will not really be in the right. If you ask a true man, a man who acts from his depth: ``Why do you do all your works?" he will answer you rightly only if he says: ``I act only for the action itself." \flright{\textsc{Meister Eckhart}}

To prove freedom metaphysically, without encumbering oneself with all the usual philosophical arguments, it is sufficient to establish that it is a possibility, since the possible and the real are metaphysically identical. \flright{\textsc{Rene Guenon}, \emph{Multiple States of the Being}}

\end{quotex}
Free will is the ability to choose the Good. But first, the intellect needs to know what is the Good, then the will is free to make the choice. This is intuitively obvious, for thought can neither be compelled or caused. Nevertheless, there are two confusions, in particular, about what ``free will" is actually all about.

\begin{enumerate}
\item Will vs Power 
\item Determinism vs Unfree Will 
\end{enumerate}
\paragraph{Will vs Power}
The first moment of the will arises from thought. Yet a person may not have the power to actualize the Good because of obstacles, both interior and exterior. For example, a person may be weak, or fearful of social opprobrium, and thus fails to act on what he knows to be the Good. Or circumstances may limit what is possible. In those cases, the Will must act prudently.

To summarize, the Will is free to choose, but may not always have the power to act on it.

\paragraph{Determinism}
A second misconception is to assume that the Will is unfree because it is the result solely of physical causes. Even assuming that be true, material causes cannot create thought, hence the Will can still choose the good. In the worst case, these causes may prevent the Will from acting on its choice.

When philosophers like a Spinoza or a Schopenhauer assert that the Will is not free, they mean that the Will does not choose the Good known to the Intellect. Rather, it is subject to unknown, unconscious, or unacknowledged motivations. Hence, the Will chooses not what is intellectually good, but may choose based on animal-like motivations: fear, hunger, or the desire for sex. Or there may be strong emotional compulsions: the desire for self-aggrandizement or revenge, for example. It is unnecessary to name them all.

\paragraph{Purification of the Will}
A necessary stage of the spiritual path is the purification of the Will. In practice, this requires inner vigilance to ``see" each and every influence that exercise control of the Will. Only then will it be possible to overcome such influences, which are intruders into our true Being. Ultimately, ``free Will" means that our choice arises from the Self alone, not for those intruders. This can be found in serious thinkers from Eckhart, Spinoza, and even Rene Guenon in \emph{The Multiple States of the Being}.

\flrightit{Posted on 2021-06-30 by Cologero }
