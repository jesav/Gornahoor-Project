\section{Metaphysics of Freedom}

\begin{quotex}
Per virtutem et potentiam idem intelligo.

By virtue and power, I mean the same thing.

\flright{\textsc{Spinoza}, \emph{Ethics}}

\end{quotex}
\paragraph{The Possibility of Freedom}
In \emph{The Multiple States of the Being}, \textbf{Rene Guenon} discusses the meaning and possibility of freedom. He defines freedom as the “absence of constraint”. In nonduality, there is no constraint; this establishes the possibility of freedom. Since it is possible, it can be manifested. When there is a multiplicity of beings, each one is limited by the others, but there is a residue of freedom. He goes on:

\begin{quotex}
A being will be free to the degree to which it participates in this unity [of Being]. In other words, it will be freer as it has more unity in itself, or as it is more “One”.

Absolute freedom belongs to the being who has become absolutely “one” at the degree of pure Being, or “without duality” if his realization surpasses Being. … Then one can speak of a being “who is a law unto himself”, because this being is fully identical with his sufficient reason, which is at the same time his principial origin and his final destiny. 

\end{quotex}
“A law unto himself” is not a Nietzschean immoralism; Guenon relates it to Islamic esotericism and to the Hindu doctrines of \emph{swechhachar}i [autocracy, a concept used by \textbf{Julius Evola} in the \emph{Individual and the Becoming of the World}] or \emph{jivan-mukti} [enlightened in the body]. In Christian Hermetism, we read

\begin{quotex}
To be something, to know something, and to be capable of something endows a person with authority. … a person has authority to the extent that he unites within himself the profundity of mysticism, the wisdom of gnosis, and the productive power of magic [power, will]. … whoever has this to a high degree can “lay down the law”. \flright{\textsc{Valentin Tomberg}, \emph{Meditations on the Tarot} (Emperor)}

\end{quotex}


Evola differs from Guenon to the extent that the former incorporates Hermetism and Tantra into his system, while the latter restricts himself to the orthodox schools of the Hinduism.

\paragraph{The Possibility of Sin}
For Kant, freedom is a postulate of the practical reason, without which there can be no concept of morality. The corollary is that a man is moral only to the extent that he is free. Since a man is free to the extent that he is not subjected to constraint or privation, it follows that not every man is capable of moral action. In contrast to the modernist, universalist moral absolutism of Kant, the Traditional understanding is more nuanced.

The Catechism of Pope Pius X\footnote{\url{https://www.gornahoor.net/library/CatechismSSPX.pdf}}, lists three necessary conditions for a mortal sin:

\begin{enumerate}
\item Grave matter 
\item Full advertence 
\item Perfect consent of the will 
\end{enumerate}
Metaphysically, we can rephrase these conditions.

\subparagraph{Logos}
A grave matter is action that is contrary to the “law of God”, that is, the cosmic order or Logos. By disturbing that order, a fundamental injustice is introduced. Evola refers to Anaximander, Parmenides, and Empedocles in this regard.

\subparagraph{Knowledge or Gnosis}
The second condition requires the moral agent to know that the matter is a grave injustice. To be fully effective, this knowledge of the Logos must be of the highest kind: intuition or episteme. Another word for knowledge as it applies to the idea of Justice is “conscience”.

\subparagraph{Will or Power}
Full consent of the Will means that the agent is capable of performing the act; a man cannot be commanded to do the impossible. The Catechism explains: “Perfect consent of the will is verified in sinning when we deliberately determine to do a thing.” That is, the act is free and not compelled by some necessity or external constraint.

\paragraph{True Will}
Thus we see that the very possibility of moral activity depends on the free will. Hence, the only moral “imperative” is the requirement to be free, without which, no moral activity is even conceivable. The place to start is to overcome weakness and concupiscence which are external constraints to our freedom. As Eckhart writes, a free man acts without desire.

Tomberg relates freedom to conscience.

\begin{quotex}
Conscience is neither a product nor a function of character. It is above it and it is only here where the domain of freedom is found. One is free … when one judges and acts according to Justice or conscience. But justice is only the beginning of a long path in the development of conscience and therefore of the growth of freedom. 

\end{quotex}
In \emph{Revolt against the Modern World}, Evola describes this knowledge:

\begin{quotex}
From the gradual extinction of all images and forms, and eventually of one's own thoughts, will, and knowledge, what arises is a transformed and supernatural knowledge that is carried beyond all forms.

\end{quotex}
This is what Evola means by aristocratic asceticism. He writes:

\begin{quotex}
\textbf{Meister Eckhart} addressed the noble man and the noble soul whose metaphysical dignity is witnessed by the presence of a strength, a light, and a fire within him… The principle of spiritual centrality was affirmed: the true Self is God, God is our real center and we are external only to ourselves … an action dictated by desire … must not be undertaken.

\end{quotex}
Only with God as the center is man “One”. He is then said to have a Holy Will or \textbf{True Will}.

\flrightit{Posted on 2011-03-23 by Cologero}

\begin{center}* * *\end{center}

\begin{footnotesize}\begin{sffamily}

\hfill

\texttt{Vindar on 2011-06-26 at 10:19 said:}

There's a good article On Freedom and Necessity here

\url{http://www.sacredweb.com/online\_articles/sw27\_editorial.pdf}

\hfill

\end{sffamily}\end{footnotesize}
