\section{The Infinite}

This is the dictionary definition of Infinite:

\begin{enumerate}
\item Having no boundaries or limits. 
\item Immeasurably great or large; boundless: infinite patience; a discovery of infinite importance. 
\item Mathematics. 

\begin{enumerate}
\item Existing beyond or being greater than any arbitrarily large value. 
\item Unlimited in spatial extent: a line of infinite length. 
\item Of or relating to a set capable of being put into one-to-one correspondence with a proper subset of itself. 
\end{enumerate}
\end{enumerate}
In metaphysics, we are interested only in the first definition. The second is only a figurative use of the word and is of absolutely no interest. Mathematicians use the word only in a specialized and technical sense; in comparing it to the first definition, it is obvious that the mathematical infinity is not the metaphysical infinity, so care must be taken never to confuse the two. Rene Guenon uses the word “indefinite” to designate mathematical infinity.

From this definition we note that the Infinite is necessarily one, indivisible, eternal, and unchanging.

\paragraph{One}
There cannot be two infinities because Infinity A would limit or set a boundary to Infinity B. Therefore, the Infinite is One.

\paragraph{Indivisible}
If the Infinite were divisible, it could either be divided into two finite parts, which is impossible, or it could be divided into two equally infinite parts, which is also impossible. Thus, the infinite is indivisible. Another word used to express this quality is \textbf{Simplicity}.

\paragraph{Eternal}
It is eternal, otherwise it would be limited or bounded by time.

\paragraph{Unchanging}
Suppose the Infinite could change from state A to state B. That means that the Infinite as B would limit the Infinite as A. By definition this cannot be, therefore the Infinite is unchanging. Thus any system of thought that claim the Infinite is “evolving” is necessarily false.

\paragraph{Hermetic Symbolism}
The circle is the symbol of unity.

The lemniscate is the symbol of infinity. Valentin Tomberg's explanation follows:

\begin{quotex}
The lemniscate is not only the symbol of infinity, but also that of rhythm, of the respiration and circulation — it is the symbol of eternal rhythm or the eternity of rhythm … it represents the state of concentration without effort, i.e., the state of consciousness where the centre directing the will has “descended” (in reality is is elevated) from the brain to the rhythmic system, where the “oscillations of the mental substance” are reduced to silence and to rest, no longer hindering concentration. 

\end{quotex}

\hfill

\flrightit{Posted on 2011-02-22 by Cologero }

\begin{center}* * *\end{center}

\begin{footnotesize}\begin{sffamily}


\hfill

\texttt{GF on 2011-02-23 at 10:02 said: }

I wonder why the figure eight has become symbolic of infinity, when the circle or spiral is more appropriate. In music, unison appears as a spiral while the octave appears as a figure eight.


\hfill

\texttt{Cologero on 2011-02-23 at 19:16 said: }

The circle is the symbol of unity.

The lemniscate is the symbol of infinity. Tomberg's explanation follows:

\begin{quotex}
The lemniscate is not only the symbol of infinity, but also that of rhythm, of the respiration and circulation — it is the symbol of eternal rhythm or the eternity of rhythm. .. it represents the state of concentration without effort, i.e., the state of consciousness where the centre directing the will has “descended” (in reality is is elevated) from the brain to the rhythmic system, where the “oscillations of the mental substance” are reduced to silence and to rest, no longer hindering concentration. 

\end{quotex}

\hfill

\texttt{GF on 2011-02-24 at 09:06 said: }

The circle is not limited to expressing unity only. Recall how often in ancient artifacts Ouroboros appears in a circle.

Tomberg's quotation doesn't explain why the lemniscate is a symbol of infinity; he simply states it is so, then explains why it's also a symbol of eternal rhythm. However, rhythm involves alternation between polarities. Visually the lemiscate is divided in two, and gives an impression of movement. It aptly suggests the eternity of rhythm, but is not so apt as the circle to suggest the infinite. (I withdraw my suggestion of the spiral.)


\hfill

\texttt{Will on 2011-02-25 at 10:25 said: }

Tomberg's comment matches nicely with the article “Opus Magicum: Fire” in Evola's Introduction to Magic. The author (apparently Giulio Parise) discusses “the Self's descent into the heart.”


\end{sffamily}\end{footnotesize}
