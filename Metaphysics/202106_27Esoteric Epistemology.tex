\section{Esoteric Epistemology}

\begin{quotex}
For the whole human being is at one and the same time a mystic, a gnostic, a magician and a philosopher, i.e., he is religious, contemplative, artistic and intelligent. \flright{\textsc{Valentin Tomberg}, \emph{Meditations on the Tarot}}

\end{quotex}
In the secular world, this is the epistemological question:

\begin{quotex}
What do you know and when did you know it? 

\end{quotex}
However, this misses the more important question:

\begin{quotex}
How do you know and what do you know when you know it?

\end{quotex}
In metaphysics, epistemology follow ontology, i.e., knowledge follows the sequence of manifestation, like this:

\begin{enumerate}
\item \textbf{Mystical experience}: The pure act of knowledge of the Self 
\item \textbf{Gnosis}: The intuitive knowledge of the Divine Intellect 
\item \textbf{Magic}: The expression of that knowledge in representations and imagination, whose source is the Manas. 
\item \textbf{Philosophy}: The above three states expressed in language, which is proper to the human state. 
\end{enumerate}
Tomberg describes the stages in more detail:

\begin{quotex}
This transformation of mystical experience into knowledge takes place in stages. The first is the pure reflection or a kind of imaginative repetition of the experience. The second stage is its entrance into memory. The third stage is its assimilation in thought and feeling, in a manner where it becomes a “message” or inner word. The fourth stage, lastly, is reached when it becomes a communicable symbol or “writing”, or “book”—i.e., when it is formulated. \flright{\textsc{Valentin Tomberg}}

\end{quotex}
\paragraph{Short Circuits}
There can be deformations of this transformation, such as:

\begin{enumerate}
\item Remaining stuck in mystical experience, just for the pleasure of it. 
\item Gnosis detached from revelation and tradition lacks any foundation. Gnosis is mysticism that has become conscious of itself. 
\item Imagination without gnosis becomes pure fantasy 
\item Philosophy becomes pure speculation, and lacks truth 
\end{enumerate}
\paragraph{Metaphysical and Profane Knowledge}
IT is a misconception to assume that the method of metaphysical knowledge is identical to our knowledge of the corporeal world. These are the main differences between the two ways of knowing.

\begin{itemize}
\item Metaphysical knowledge is based on the faculty of intuition, which is a direct experience of the real and permits contact between our consciousness and the world of pure mystical experience. Profane knowledge is based on the inferior faculty of the rational mind. 
\item Metaphysical knowledge is deductive, that is, it begins with higher principles and applies them to particular situations. Profane knowledge is inductive, or better said, abductive. 
\item Metaphysical knowledge depends on the level of being of the knower. Hence, spiritual practices such as concentration, meditation, prayer, purification of the soul, and so on, are necessary. Profane knowledge, on the other hand, does not depend on the personal character of the knower. 
\item Metaphysics is known with apodictic certainty and profane knowledge is uncertain and subject to revision. 
\end{itemize}
\paragraph{The Temptation of Philosophy}
There is a certain blindness in the human being. Those with the intellectual talent often believe that the mere accumulation of books will somehow bring them to gnosis and mystical experience. As we see above, the book is the last stage of the process. Not all mystics and gnostics have the ability to write good books, and certainly fewer can write good poetry.

No matter how many times they hear or read that self-knowledge is more important than book knowledge, it fails to make any impact. You know the type; try to avoid it.

Whenever you think you may have some great idea, don't forget that over the centuries people more intelligent, more educated, and more insightful than you have thought about those issues that so interest you. Turn to the classic authors first, whose thought has endured over the centuries; prefer them to more recent authors.

\paragraph{The Temptation of Science}
\begin{quotex}
Science begins with facts (the “characters” of the book of Nature) and ascends from facts to laws and from laws to principles. Gnosis is the reflection of that which is above; science, in contrast, is the interpretation of that which is below. The last stage of gnosis is the world of facts, where it becomes fact itself, i.e., it becomes “book”; the first stage of science is the world of facts which it “reads”, in order to arrive at laws and principles.

\end{quotex}
There are two ways to understand the Fall of Man:

\begin{itemize}
\item From the moral point of view, Adam and Eve disobeyed God's command and were punished for it. 
\item From the gnosis point of view, Adam and Eve replaced the intuitive knowledge of the Good with the empirical knowledge of good and evil. This had enormous consequences. 
\end{itemize}
The age-old question is, “Is something good because God commanded it, or did God command it because it was good?”

The moral view is based on the first premise and the gnosis view on the second premise.

Science is based on knowledge that begins with, and is restricted to, the sense. It can only produce falsifiable predictions, not certain knowledge.



\flrightit{Posted on 2021-06-27 by Cologero }
